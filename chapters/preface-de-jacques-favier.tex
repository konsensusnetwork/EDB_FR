% Options for packages loaded elsewhere
\PassOptionsToPackage{unicode}{hyperref}
\PassOptionsToPackage{hyphens}{url}
%
\documentclass[
  a5paper,
  smalldemyvopaper,10pt,twoside,onecolumn,openright,extrafontsizes,hidelinks]{memoir}

\usepackage{amsmath,amssymb}
\usepackage{iftex}
\ifPDFTeX
  \usepackage[T1]{fontenc}
  \usepackage[utf8]{inputenc}
  \usepackage{textcomp} % provide euro and other symbols
\else % if luatex or xetex
  \usepackage{unicode-math}
  \defaultfontfeatures{Scale=MatchLowercase}
  \defaultfontfeatures[\rmfamily]{Ligatures=TeX,Scale=1}
\fi
\usepackage{lmodern}
\ifPDFTeX\else  
    % xetex/luatex font selection
\fi
% Use upquote if available, for straight quotes in verbatim environments
\IfFileExists{upquote.sty}{\usepackage{upquote}}{}
\IfFileExists{microtype.sty}{% use microtype if available
  \usepackage[]{microtype}
  \UseMicrotypeSet[protrusion]{basicmath} % disable protrusion for tt fonts
}{}
\makeatletter
\@ifundefined{KOMAClassName}{% if non-KOMA class
  \IfFileExists{parskip.sty}{%
    \usepackage{parskip}
  }{% else
    \setlength{\parindent}{0pt}
    \setlength{\parskip}{6pt plus 2pt minus 1pt}}
}{% if KOMA class
  \KOMAoptions{parskip=half}}
\makeatother
\usepackage{xcolor}
\setlength{\emergencystretch}{3em} % prevent overfull lines
\setcounter{secnumdepth}{5}
% Make \paragraph and \subparagraph free-standing
\makeatletter
\ifx\paragraph\undefined\else
  \let\oldparagraph\paragraph
  \renewcommand{\paragraph}{
    \@ifstar
      \xxxParagraphStar
      \xxxParagraphNoStar
  }
  \newcommand{\xxxParagraphStar}[1]{\oldparagraph*{#1}\mbox{}}
  \newcommand{\xxxParagraphNoStar}[1]{\oldparagraph{#1}\mbox{}}
\fi
\ifx\subparagraph\undefined\else
  \let\oldsubparagraph\subparagraph
  \renewcommand{\subparagraph}{
    \@ifstar
      \xxxSubParagraphStar
      \xxxSubParagraphNoStar
  }
  \newcommand{\xxxSubParagraphStar}[1]{\oldsubparagraph*{#1}\mbox{}}
  \newcommand{\xxxSubParagraphNoStar}[1]{\oldsubparagraph{#1}\mbox{}}
\fi
\makeatother


\providecommand{\tightlist}{%
  \setlength{\itemsep}{0pt}\setlength{\parskip}{0pt}}\usepackage{longtable,booktabs,array}
\usepackage{calc} % for calculating minipage widths
% Correct order of tables after \paragraph or \subparagraph
\usepackage{etoolbox}
\makeatletter
\patchcmd\longtable{\par}{\if@noskipsec\mbox{}\fi\par}{}{}
\makeatother
% Allow footnotes in longtable head/foot
\IfFileExists{footnotehyper.sty}{\usepackage{footnotehyper}}{\usepackage{footnote}}
\makesavenoteenv{longtable}
\usepackage{graphicx}
\makeatletter
\def\maxwidth{\ifdim\Gin@nat@width>\linewidth\linewidth\else\Gin@nat@width\fi}
\def\maxheight{\ifdim\Gin@nat@height>\textheight\textheight\else\Gin@nat@height\fi}
\makeatother
% Scale images if necessary, so that they will not overflow the page
% margins by default, and it is still possible to overwrite the defaults
% using explicit options in \includegraphics[width, height, ...]{}
\setkeys{Gin}{width=\maxwidth,height=\maxheight,keepaspectratio}
% Set default figure placement to htbp
\makeatletter
\def\fps@figure{htbp}
\makeatother

% typographical packages
\usepackage{microtype}
\usepackage{setspace}
\tolerance=6000
\hyphenpenalty=1000

% typographical settings for the body text
\setlength{\parskip}{0em}
\setlength{\parindent}{1em}
\linespread{1.3}

% DEFINITIONS TITLE PAGE / COPYRIGHT
\newcommand{\titleoriginal}{Title}
\newcommand{\subtitleoriginal}{Subtitle}
\newcommand{\yearoriginal}{Year}
\newcommand{\subtitletranslation}{Translation Subtitle}
\newcommand{\yeartranslation}{Translation Year}
\newcommand{\stringtranslation}{Translation String}
\newcommand{\stringlicense}{Translation License String.}
\newcommand{\stringpublisher}{Published by String}
\newcommand{\ISBNHC}{978-9916-}
\newcommand{\ISBNSC}{978-9916-}
\newcommand{\ISBNEBOOK}{978-9916-}
%\newcommand{\ISBNAUDIO}{978-9916-}
\newcommand{\press}{Konsensus Network}
\newcommand{\translatorone}{Translator 1}
\newcommand{\translators}{
\large\textit{\stringtranslation:}\\
\translatorone\\
}

% PHYSICAL DOCUMENT SETUP
\setstocksize{210mm}{148mm}
\settrimmedsize{210mm}{148mm}{*}
\setbinding{5mm}
\setlrmarginsandblock{8mm}{15mm}{*}
\setulmarginsandblock{25mm}{26mm}{*}

% FONTS
\usepackage{fontspec}
\setmainfont{stone-serif}[
    Path=./fonts/stone-serif-itc-pro/,
    Scale=0.86,
    Extension=.OTF,
    UprightFont=*-Regular,
    BoldFont=*-SemiBd,
    ItalicFont=*-MediumIt,
    BoldItalicFont=*-SemiBdIt
    ]

\setsansfont{stone-sans}[
    Path=./fonts/stone-sans/,
    Scale=0.82,
    Extension=.otf,
    UprightFont=*-Medium,
    BoldFont=*-Semibold,
    ItalicFont=*-MediumItalic,
    BoldItalicFont=*-SemiBoldItalic
    ]

\usepackage{lettrine}
\setcounter{DefaultLines}{3}
\renewcommand{\DefaultLoversize}{0.1}
\renewcommand{\DefaultLraise}{0}
\renewcommand{\LettrineTextFont}{}
\setlength{\DefaultFindent}{\fontdimen2\font}
\setlength{\DefaultNindent}{0em}

%\usepackage[footskip=8mm]{geometry}

% custom second title page
\makeatletter
\newcommand*\halftitlepage{\begingroup % Misericords, T&H p 153
  \setlength\drop{0.1\textheight}
  \begin{center}
  \vspace*{\drop}
  \rule{\textwidth}{0in}\par
  {\Large\sffamily\thetitle\par}
  \rule{\textwidth}{0in}\par
  \vfill
  \end{center}
\endgroup}
\makeatother

% custom title page
\makeatletter
\newlength\drop
\newcommand*\titleM{\begingroup % Misericords, T&H p 153
  \setlength\drop{0.15\textheight}
  \begin{center}
  \vspace*{\drop}
  {\huge\sffamily\thetitle\par}
  \vspace{2em}
  %{\normalsize\sffamily\textit\subtitletranslation\par}
  %\vspace{2em}
  \rule{5.5cm}{0.3mm}\par
  \vspace{2em}
  {\Large\sffamily\textit\theauthor\par}
  \vspace{3em}
  %{\footnotesize\sffamily\textit\translators\par}
  \vfill
  \includegraphics[width=3.5cm]{figures/knw.png}\par
  \end{center}
\endgroup}
\makeatother

% copyright page
\makeatletter
\newcommand*\copyrightpage{\begingroup
  \setlength\drop{0.1\textheight}
  \vphantom{just for the drop}
  \vfill
  \begin{scriptsize}
  \noindent \copyright\space \yearoriginal: \theauthor
  \par\noindent \textit{\titleoriginal}
  \vspace{0.5\baselineskip}
  %\par\noindent \copyright\space \yeartranslation\space \stringtranslation: \translatorone
  %\par\noindent \textit{\thetitle: \subtitletranslation}
  \vspace{\baselineskip}
  \par\noindent \textit{\stringlicense}
  \vspace{0.5\baselineskip}
  \par\noindent \stringpublisher: \href{https://konsensus.network}{\textit{konsensus.network}}
  \vspace{0.5\baselineskip}
  \par\noindent v1.0.0
  \vspace{0.5\baselineskip}
  \setlength{\parindent}{2em}% default 20pt
  \par\noindent ISBN \ISBNHC \:Hardcover
  \par\hspace{0.28\parindent}\ISBNSC \:Paperback
  \par\hspace{0.28\parindent}\ISBNEBOOK \:E-book\par
  \setlength{\parindent}{0pt}
  \end{scriptsize}
  \vspace{3em}
  \par\noindent \href{https://konsensus.network}{\large\MakeUppercase \press \hspace{3em} \includegraphics[width=1cm]{figures/freestarfish.png}}
  \setcounter{footnote}{0}
  \clearpage
\endgroup}
\makeatother

% HEADER AND FOOTER MANIPULATION
% for normal pages
\nouppercaseheads
\headsep = 10mm
\makepagestyle{mystyle} 
\makeevenhead{mystyle}{\scriptsize\sffamily\mdseries\thepage}{}{}
\makeoddhead{mystyle}{{\scriptsize\sffamily\mdseries\leftmark}}{}{\scriptsize\sffamily\mdseries\thepage}
\makeevenfoot{mystyle}{}{}{}
\makeoddfoot{mystyle}{}{}{}
\makeatletter

% for pages where chapters begin
\makepagestyle{plain}
\makerunningwidth{plain}{\headwidth}
\makeevenfoot{plain}{}{}{}
\makeoddfoot{plain}{}{\scriptsize\sffamily\mdseries\thepage}{}
\pagestyle{mystyle}

\newif\ifmainmatter
\appto\mainmatter{\mainmattertrue}
\appto\backmatter{\mainmatterfalse}
\appto\appendix{\mainmatterfalse}

\renewcommand\chaptermark[1]{%
  \markboth{\MakeUppercase{%
    \ifmainmatter~\oldstylenums\thechapter.~\fi#1}}{}}%

% TOC
\usepackage[]{tocloft}
\renewcommand{\cftsectiondotsep}{\cftnodots}
\renewcommand{\cftpartfont}{\Large\sffamily\MakeUppercase}
\renewcommand{\cftchapterfont}{\small\sffamily}
\renewcommand{\cftsectionfont}{\Small\sffamily}
\renewcommand{\cftpartpagefont}{\Large\sffamily}
\renewcommand{\cftchapterpagefont}{\small}
\renewcommand{\cftchapterpresnum}{HOOFDSTUK\space}
\renewcommand{\cftchapternumwidth}{7em}
\setlength{\cftchapterindent}{0em}
\setlength{\cftsectionindent}{7em}
\setlength{\cftbeforechapterskip}{-0.2em}
\setsecnumdepth{chapter}
\setcounter{tocdepth}{0}


% Redefine footnote presentation
\makeatletter
\renewcommand\@makefntext[1]{%
  \noindent\hb@xt@2em{% <-- Box of fixed size for footnote number and space
    \@thefnmark\quad}% <-- Footnote number followed by a quad space
  \parbox[t]{\dimexpr\linewidth-2em}{#1}% <-- Parbox to control the width of footnote content
}
\makeatother

% layout check and fix
\checkandfixthelayout

% COUNTERS FOOTNOTES
\usepackage{chngcntr}
\counterwithout*{footnote}{chapter}

% TITLE FORMATTING
\usepackage{titlesec}
\titleformat
    {\chapter}[display]
    {\huge\sffamily}
    {\Large\sffamily\chaptertitlename\space\thechapter}
    {0pt}
    {\vspace{28pt}}

\titleformat
  {\section}[block]
  {\sffamily\large\bfseries}
  {}
  {0pt}
  {}
  
\titlespacing*{\section}{0pt}{2em}{0.5em}

\titleformat{\subsection}{\sffamily\bfseries}{}{}{}
\titlespacing*{\subsection}{0pt}{2em}{0em}

% QUOTE FORMATTING
\renewenvironment{quote}%
               {\list{}{\rightmargin=.6cm\leftmargin=.6cm}%
                \itshape \item[]}% <- The effect of \samepage is local!!!
               {\endlist}

% LAYOUT CHECK AND FIX
\checkandfixthelayout

% CUSTOM TITLE PAGE
\makeatletter
\def\@maketitle{%
  % the half title page
  \pagestyle{empty}
  \halftitlepage
  \cleardoublepage

  % the title page
  \titleM
  \clearpage

  % the copyright page
  \copyrightpage
  \cleardoublepage
  \pagestyle{mystyle}
}
\makeatother
% END PREAMBLE
\makeatletter
\@ifpackageloaded{bookmark}{}{\usepackage{bookmark}}
\makeatother
\makeatletter
\@ifpackageloaded{caption}{}{\usepackage{caption}}
\AtBeginDocument{%
\ifdefined\contentsname
  \renewcommand*\contentsname{Inhoudsopgave}
\else
  \newcommand\contentsname{Inhoudsopgave}
\fi
\ifdefined\listfigurename
  \renewcommand*\listfigurename{Lijst van figuren}
\else
  \newcommand\listfigurename{Lijst van figuren}
\fi
\ifdefined\listtablename
  \renewcommand*\listtablename{Lijst van tabellen}
\else
  \newcommand\listtablename{Lijst van tabellen}
\fi
\ifdefined\figurename
  \renewcommand*\figurename{Figuur}
\else
  \newcommand\figurename{Figuur}
\fi
\ifdefined\tablename
  \renewcommand*\tablename{Tabel}
\else
  \newcommand\tablename{Tabel}
\fi
}
\@ifpackageloaded{float}{}{\usepackage{float}}
\floatstyle{ruled}
\@ifundefined{c@chapter}{\newfloat{codelisting}{h}{lop}}{\newfloat{codelisting}{h}{lop}[chapter]}
\floatname{codelisting}{Listing}
\newcommand*\listoflistings{\listof{codelisting}{Lijst van listings}}
\makeatother
\makeatletter
\makeatother
\makeatletter
\@ifpackageloaded{caption}{}{\usepackage{caption}}
\@ifpackageloaded{subcaption}{}{\usepackage{subcaption}}
\makeatother

\ifLuaTeX
\usepackage[bidi=basic]{babel}
\else
\usepackage[bidi=default]{babel}
\fi
\babelprovide[main,import]{dutch}
\babelprovide[import]{greek}
% get rid of language-specific shorthands (see #6817):
\let\LanguageShortHands\languageshorthands
\def\languageshorthands#1{}
\ifLuaTeX
  \usepackage{selnolig}  % disable illegal ligatures
\fi
\usepackage{bookmark}

\IfFileExists{xurl.sty}{\usepackage{xurl}}{} % add URL line breaks if available
\urlstyle{same} % disable monospaced font for URLs
\hypersetup{
  pdftitle={Book Title},
  pdfauthor={Book Author},
  pdflang={nl},
  hidelinks,
  pdfcreator={LaTeX via pandoc}}


\title{Book Title}
\author{Book Author}
\date{+0015-01-01}

\begin{document}
\frontmatter
\maketitle

\renewcommand*\contentsname{Inhoudsopgave}
{
\setcounter{tocdepth}{0}
\tableofcontents
}

\mainmatter
\bookmarksetup{startatroot}

\chapter*{Over dit boek}\label{over-dit-boek}
\addcontentsline{toc}{chapter}{Over dit boek}

\markboth{Over dit boek}{Over dit boek}

\bookmarksetup{startatroot}

\chapter*{Préface de Jacques Favier}\label{pruxe9face-de-jacques-favier}
\addcontentsline{toc}{chapter}{Préface de Jacques Favier}

\markboth{Préface de Jacques Favier}{Préface de Jacques Favier}

Ludovic Lars est assez unanimement considéré comme l'un des grands
«~érudits~» francophones en matière de Bitcoin. Ce mot, qui est revenu
dans plusieurs conversations au sujet de son projet éditorial, m'a
inspiré le fil de trame de cette préface.

En me faisant l'honneur de me demander celle-ci, il m'avait prévenu que
le ton «~libéral~» de son \emph{Élégance du Bitcoin} pourrait trancher
avec ma propre sensibilité. J'y ai perçu une forme d'\emph{élégance}
morale. En vérité, l'auteur a lu les économistes dits «~autrichiens~»
mais aussi Proudhon et il n'est pas davantage que moi maximaliste buté
ou toxique. Étant surtout ennemi des extrémismes idéologiques et des
raisonnements à une seule dimension, je ne saurais m'offusquer de ce
qu'au sein d'une communauté supportant l'essor de solutions
décentralisées règnent des opinions différentes, avec ce que cela
implique comme visions ou comme biais.

Tout en assumant ce que l'on appelait jadis un vrai «~travail de
bénédictin~» l'auteur a d'ailleurs demandé et obtenu la confiance de
multiples spécialistes qui ont assuré à son travail une prise en compte
d'un très large spectre de connaissances et une relecture soigneuse. Il
y a eu, au-delà de ce concours d'experts, un véritable engagement
communautaire, financier et moral pour que soit publié le présent livre.

On trouvera donc ici un travail qui, tant par un ton rarement polémique
que par une profonde érudition et une inscription dans un mouvement
collectif, participe de la tradition de l'\emph{Encyclopédie} française.
On sait que les promoteurs de la \emph{Britannica} accusèrent celle de
Diderot et d'Alembert de «~propager l'anarchie~» et l'on ne peut nier
que, rédigée alors qu'éclosaient les Lumières, cette somme des
connaissances de toute nature -- tant théoriques que pratiques -- n'ait
pris courageusement parti dans les combats politiques et philosophiques
de son temps, avec l'intention explicite d'ouvrir une réflexion critique
et de «~changer la façon commune de penser~». Sans doute pourrait-on en
dire autant ici~: Bitcoin et ce livre ne vous invitent pas tant, ou pas
seulement, à changer de monnaie qu'à changer de pensée.

Ayant senti cela, je suis allé fureter dans l'\emph{Encyclopédie}.
L'article «~érudition~», rédigé par d'Alembert lui-même, expose que
celle-ci «~renferme trois branches principales, la connaissance de
l'Histoire, celle des Langues, \& celle des Livres~». En changeant
peut-être \emph{langues} par \emph{protocoles}, il aurait pu goûter
lui-aussi et préfacer mieux que moi le livre que vous venez d'ouvrir.

Mon propre esprit, formé aux études historiques, s'est délecté des
premiers chapitres, qui constituent de véritables Annales de Bitcoin.
Les historiens d'aujourd'hui et de demain ne pourront qu'apprécier
l'ampleur des informations fiables et des références compilées.
Mathématicien de formation, l'auteur a produit d'abord un très important
travail archivistique, dont atteste près d'un millier de notes savantes.

Comme me l'a écrit le créateur du site Bitcoin.fr «~il déniche et
déchiffre des débats abscons qui ont pourtant eu une importance capitale
dans l'évolution du protocole, et les rend compréhensible à tous~».
Ainsi, si ce qu'il restitue de l'histoire de la monnaie peut être
critiqué ou remis dans la perspective de ses convictions personnelles,
ce qu'il construit de l'histoire de Bitcoin est un apport dont d'autres
feront utilement leur miel.

Au-delà de l'Histoire, il y a donc les Langues et les Livres~: des
références, du code, de la théorie des jeux et des mathématiques. Il y a
beaucoup à glaner~dans ces pages, dont beaucoup de choses austères mais
aussi de petits faits plaisants. Si le livre narre en détail
l'inévitable geste de la fameuse pizza, il rappelle aussi qu'un robinet
à bitcoin a fonctionné 2 ans en envoyant 5 bitcoins à chaque demande, ou
que celui qui a découvert la première faille a gentiment prévenu Satoshi
au lieu de profiter de sa découverte pour tricher. Il souligne ainsi
qu'avant sa phase de «~croissance conflictuelle~» les premières années
de l'aventure ont vu «~une croissance organique et prudente, à l'abri de
l'opportunisme et de la propagande de notre monde~» et que la communauté
a fait montre, depuis l'origine, d'une extraordinaire résilience, chose
qui doit être méditée.

L'abondance des citations rend justice aux cypherpunks, parfois traités
comme de sinistres sires fomentant une révolte fiscale autour d'un
intempestif barbecue. Elle restitue la profondeur historique et
intellectuelle de ce qui fut un mouvement de fond collectif et non une
réaction épidermique sectaire. Accessoirement, les trajets individuels
finement retracés montrent que l'influence autrichienne, non
négligeable, ne fut ni universelle ni complète. Bien des cryptographes,
cypherpunks ou non, n'y ont pas adhéré comme à un dogme révélé ou à une
vérité scientifiquement établie.

Ludovic Lars rappelle en outre ce point crucial~: les cypherpunks ne
furent pas les seuls à essayer de construire des systèmes distribués qui
puissent servir à l'échange monétaire. Parce qu'il y avait un vrai
problème et un vrai besoin. Dans le bouillonnement intellectuel, les
échanges étaient nombreux~: il est amusant de rappeler que Ripple
s'inspira aussi du localisme des SEL~! En fait la différence avec toutes
les autres tentatives c'est que Bitcoin (le premier à ne pas reposer sur
une confiance au sens classique) a réussi comme monnaie parce qu'il a
réussi à construire une communauté élective, philosophique, politique.
Bitcoin est la plus large monnaie communautaire de tous les temps.

S'intéresser à sa (longue) geste avant autant qu'après 2009 n'est donc
pas une marotte d'historien. Outre une compréhension indispensable de
ses racines, des intentions et des ambitions qui animaient précurseurs
et témoins de sa naissance, on trouve de quoi démonter bien des
escroqueries intellectuelles hélas persistantes. Non, les monnaies
numériques de banques centrales ou les \emph{stablecoins} algorithmiques
ne représentent pas des perfectionnements de Bitcoin ni d'ailleurs des
promesses d'amélioration de notre existence à venir.

Les \emph{altcoins} plus ou moins communautaires, souvent
entrepreneuriaux voire bancaires, sont largement cités, essentiellement
pour illustrer le propos, l'enrichir d'exemples, souligner des impasses
ou des objections, jamais, il faut le répéter, et même si l'auteur les
connaît fort bien, pour «~dépasser~» ou «~perfectionner~» Bitcoin, dont
le développement organique et le perfectionnement est l'affaire des
bitcoineurs.

L'auteur est un expert technique mais il sait aussi écrire. Tout ce qui
est technique (et ignoré par beaucoup de gens, même de ceux qui se
présentent comme des «~experts~»), tout ce que d'Alembert nommerait
«~les Langues~» est disséqué dans cet ouvrage avec un scalpel
extrêmement méticuleux et restitué dans la langue où ce qui est bien
conçu «~s'énonce clairement~». Ceci mériterait d'être donné à lire au
prochain politique, financier, économiste ou publiciste qui dira que
«~ça ne repose sur rien~»~!

Le titre du livre est aussi celui d'une conclusion agréablement
équilibrée, entre ceux qui voient en Bitcoin la solution à tout et ceux
qui n'y voient que du mal. Elle pourra surprendre certains adeptes
fervents et naïfs mais elle reste dans l'esprit de d'Alembert~: «~Il y a
dans la critique deux excès à fuir également, trop d'indulgence, \& trop
de sévérité~».

Curieusement, l'auteur s'appesantit peu sur le mot d'\emph{élégance}
lui-même, que sa formation mathématique lui fait sans doute percevoir
comme embrassant les sens de vérité, de beauté et de rigueur. Comme
Aristote, il a pu penser ici à l'ordre, à la précision, à la capacité de
faire jouer ensemble plusieurs concepts, de les ajuster ensemble
efficacement, performance que Satoshi Nakamoto a réalisée au plus haut
point.

Pour m'adresser au lecteur au seuil de ce livre utile, dense et à tous
égards distingué, je donne une dernière fois la parole à d'Alembert qui
opinait que «~les secours que nous avons aujourd'hui pour l'érudition la
facilitent tellement, que notre paresse seroit inexcusable, si nous n'en
profitions pas~».

Jacques Favier, 21 novembre 2023

\bookmarksetup{startatroot}

\chapter*{Remerciements}\label{remerciements}
\addcontentsline{toc}{chapter}{Remerciements}

\markboth{Remerciements}{Remerciements}

Un ouvrage n'est jamais le fruit du seul travail de son auteur attitré.
Ce dernier est toujours aidé, financé, encouragé, inspiré par d'autres
personnes. Le livre que vous tenez entre les mains, ou que vous observez
sur un écran, n'échappe pas à la règle. Je tiens par conséquent à
remercier l'intégralité des gens qui m'ont apporté leur assistance d'une
manière ou d'une autre, et en particulier la communauté francophone de
Bitcoin qui a été là pour soutenir ce projet.

Je remercie d'abord mes lecteurs pour m'avoir lu et avoir partagé mes
articles. Un créateur n'est rien sans son public. Je suis spécialement
reconnaissant envers JohnOnChain pour son soutien de la première heure
vis-à-vis de ma démarche d'écriture. Merci aussi aux gens derrière
Cryptoast et le Journal du Coin avec qui j'ai pu travailler pendant des
années.

Je remercie ceux qui m'ont aidé à mettre en place la campagne de
financement en mars 2022. Merci à Lounès Ksouri pour ses conseils à
propos d'Umbrel. Merci à Benjamin Favre pour son aide à la mise en place
de la campagne sur le serveur. Merci à CryptoSou pour m'avoir apporté de
la liquidité sur Lightning quand j'en ai eu besoin.

Je remercie les contributeurs au financement du projet, par ordre
alphabétique~: Yanis Adoul, Autrement, Valentin Becmeur, Bitcoin.fr,
btcfork, Caulla, Chamigrou, Copinmalin, CryptoSou, Steve Deplus, Marek
Fijalkowski, Édouard Gallego, Alexandre Gonzalez, Gladsponk, Greglem,
Grittoshi, Benoît Huguet, ImTechnicolor, Jacques-Edouard, Jazaronaut,
Lionel Jeannerat, Jeffbeck, JohnOnChain, Clément Junca, François Juno,
Jybe, Kolkoz, Mike Komaransky, Konohime, Maxime Kouamen, Lounès Ksouri,
Leslie, Louferlou, Marco.BTC.fr, Loïc Morel, Ali Mitchell, Yorick de
Mombynes, Nexus 8, Leonardo Noleto, Olivier, Romain Pariset-Wagnon,
PaulADW, Pivi, RaHaN, Anthony Ro฿in, Robin de Cryptoast, Rogzy, Thibaut
Spanier, André Stilmant, François-Xavier Thoorens, Trigger, ainsi que
tous ceux qui ont souhaité préserver leur anonymat.

Je remercie les relecteurs des premières versions de cet ouvrage, qui
m'ont donné de bons conseils pour l'améliorer, tant au niveau de la
forme que du fond. Merci à Jybe, ProfEduStream, Loïc Morel, Steve
Deplus, Alexandre Gonzalez, Romain Daubigny (Recktosaurus), Pierre L.
(alias Scrypto), Jacques-Edouard de BTC Touchpoint, Bastien Desteuque,
Cédric, Meffysto, Beemo, Caroline et Marie-Christine, Gloire
Wanzavalere, Gatien, et Martin Pellemoine.

Je remercie l'équipe de Konsensus, la maison d'édition spécialisée qui
publie ce livre. J'ai une immense gratitude envers Édouard Gallego pour
son soutien indéfectible dans l'édition de cette œuvre. Merci aussi à
David St-Onge pour ses conseils éditoriaux. Merci également à
l'illustrateur de talent ImTechnicolor qui a produit la présente
couverture.

Je remercie profondément Jacques Favier, co-fondateur du Cercle du Coin
et co-auteur de trois ouvrages sur Bitcoin en français, qui m'a fait
l'honneur de lire l'intégralité de l'ouvrage et d'en rédiger une superbe
préface.

Je rends évidemment hommage à Satoshi Nakamoto pour avoir découvert
Bitcoin et l'avoir partagé au monde. Merci aussi à toutes les personnes
qui m'ont permis de mieux comprendre Bitcoin au cours du temps, et en
particulier à Andreas Antonopoulos, Julia Tourianski, Eric Voskuil et
Aaron van Wirdum.

Merci enfin à mes proches -- à ma famille et à mes amis -- qui ont été
des soutiens essentiels au cours de ces longs mois d'écriture et de
relecture.

\bookmarksetup{startatroot}

\chapter*{Avant-propos}\label{avant-propos}
\addcontentsline{toc}{chapter}{Avant-propos}

\markboth{Avant-propos}{Avant-propos}

Depuis sa conception en 2008 par Satoshi Nakamoto, Bitcoin a fait couler
beaucoup d'encre. Au fil des années, il a suscité les plus grandes
passions et il a été l'objet récurrent de débats enflammés. À son sujet,
des milliers d'articles ont été écrits, des centaines de vidéos ont été
tournées, et des dizaines de livres ont été publiés. La hausse de son
prix lui a donné une visibilité extraordinaire dans les médias, à tel
point qu'il s'est fait une place dans l'imaginaire collectif mondial.

Cependant, Bitcoin reste largement incompris. D'un côté, beaucoup de
gens en parlent en n'ayant qu'une connaissance artificielle du sujet et
ne parviennent pas à distinguer son utilité. Certains pensent qu'il ne
sert qu'à spéculer, d'autres imaginent qu'il ne devrait être utilisé que
par les criminels, d'autres encore vont jusqu'à dire qu'il ne s'agit que
d'une pyramide de Ponzi. De l'autre côté, un certain nombre de personnes
nourrissent des attentes démesurées, pensant qu'il pourrait devenir la
monnaie de réserve mondiale, voire remplacer tous les échanges
monétaires dans l'économie en quelques années seulement. Dans cette
délusion, elles s'attachent à l'espoir que son prix atteigne des niveaux
stratosphériques, dans la continuité des hausses spéculatives
précédentes. Toutefois, peu de gens tentent d'adopter un point de vue
réaliste et sobre, qui ferait la part des choses entre la vision des
vendeurs de rêve pour qui Bitcoin serait la solution à tous les
problèmes du monde, et les détracteurs de mauvaise foi pour qui Bitcoin
représenterait un fléau à arrêter à tout prix.

J'ai personnellement entendu parler de Bitcoin pour la première fois en
avril 2013, suite à la crise financière chypriote. Initialement assez
sceptique, je me suis quand même intéressé à ce système, car celui-ci
était mise en avant par les libéraux français et les libertariens
américains que je suivais. Le 9 juillet 2015, j'ai essayé la chose~: je
me suis procuré 50~€ de bitcoins (0,2~BTC) auprès de la plateforme
d'achat-vente suisse Fastcoin (nommée aujourd'hui Bity) que j'ai reçus
sur mon portefeuille Electrum nouvellement créé. J'ai réalisé ma
première transaction sur la chaîne de Bitcoin dans la journée. Ces
quelques fractions de bitcoin m'ont servi à faire des dons~: d'abord au
blogueur H16, puis à l'activiste Adam Kokesh, ensuite au projet
DarkWallet de Amir Taaki et Cody Wilson, et enfin à la plateforme
Sci-Hub gérée par Alexandra Elbakyan.

Mon implication dans Bitcoin n'a débuté réellement qu'au printemps 2017,
lorsque le prix a recommencé à monter après des années de stagnation.
Jusque-là, je m'étais contenté de suivre la cryptomonnaie de loin et
cette hausse m'a intrigué. C'est à ce moment-là que je me suis
pleinement plongé dans cet univers. J'ai lu à ce propos, notamment en me
procurant des ouvrages comme \emph{Bitcoin, la monnaie acéphale} d'Adli
Takkal-Bataille et Jacques Favier, \emph{Mastering Bitcoin} d'Andreas
Antonopoulos ou encore \emph{Digital Gold} de Nathaniel Popper. Je me
suis également mis à spéculer à mon échelle en achetant du bitcoin, puis
toutes sortes de cryptomonnaies alternatives.

En parallèle, j'ai commencé à écrire sur le sujet, si bien que je suis
devenu rédacteur pour des sites spécialisés comme Cryptoast et le
Journal du Coin. Au fil des années, j'ai rédigé plus de 150 articles de
fond, sur divers sujets liés à la cryptomonnaie, que ce soit sous un
angle technique, économique ou politique. Ma vision de Bitcoin a mûri en
conséquence, de telle sorte que je pouvais prétendre «~comprendre
Bitcoin~», même si ma conception restait évidemment parcellaire et
influencée par ma propre perspective.

Cependant, ce n'était pas forcément le cas autour de moi, où les gens en
avaient une idée superficielle, n'ayant probablement pas le temps de
creuser davantage. C'est ce qui m'a poussé à écrire ce livre. En
particulier, puisque le protocole monétaire dépendait des actions
économiques de ses utilisateurs, il me paraissait important de partager
la connaissance réelle qui avait émergé de mes recherches et de mon
expérience. De plus, avec la progression de la censure bancaire, le
développement des monnaies numériques de banque centrale, la guerre
contre l'argent liquide et le retour de l'inflation, je pense qu'il est
plus que jamais essentiel de bien appréhender cet outil afin de pouvoir
l'utiliser correctement à l'avenir.

Cet ouvrage a pour but de présenter Bitcoin de manière claire et
complète, en adoptant de multiples points de vue. Il narre le long
cheminement qui a mené à sa création, ainsi que sa courte mais dense
histoire, des origines à aujourd'hui. Il décrit son fonctionnement
essentiellement économique découlant de sa nature monétaire. Il aborde
les enjeux politiques auxquels Bitcoin répond, et en particulier le
problème de la censure. Enfin, il examine ses rouages techniques de
manière détaillée et précise. En lisant ce livre, vous finirez
peut-être, comme moi, par voir en Bitcoin un ensemble harmonieux, dont
le modèle de base est d'une rare élégance.

J'espère en tout cas que vous saurez apprécier cette modeste
contribution à quelque chose qui nous dépasse~: le projet d'une monnaie
alternative, libre et résiliente, offrant aux simples individus la
possibilité de résister aux puissances de ce monde. \emph{Vires in
numeris}.

Ludovic Lars, 1 décembre 2023

\bookmarksetup{startatroot}

\chapter{Les débuts de Bitcoin}\label{ch:mythe}

\phantomsection\label{enotezch:1}{}

{L}\textsc{e} 31 octobre 2008, un individu se faisant appeler Satoshi
Nakamoto partageait sur Internet un court document qui décrivait le
fonctionnement technique d'un système novateur de monnaie numérique~:
Bitcoin. Ce livre blanc de 9 pages, présenté comme un article
scientifique, s'intitulait en anglais \emph{Bitcoin: A Peer-to-Peer
Electronic Cash System} -- \emph{Bitcoin~: un système d'argent liquide
électronique pair à pair}. Dans celui-ci, Satoshi proposait une solution
au problème des paiements en ligne, par la mise en œuvre d'un serveur
d'horodatage distribué basé sur un algorithme de preuve de travail.

Mais cela allait beaucoup plus loin. Le livre blanc de Bitcoin posait
les bases d'une révolution conceptuelle profonde~: une monnaie
exclusivement numérique qui ne reposait sur aucun tiers de confiance, ni
pour la confirmation des transactions, ni pour l'émission des nouvelles
unités. Ce que Satoshi venait de découvrir, c'était bien plus qu'un
système de paiement~; c'était un nouveau type de monnaie, quelque chose
que nul n'avait su concevoir jusqu'alors, un phénomène économique et
social qui rencontrerait un succès inouï au cours des années qui
suivraient.

En particulier, la création de Satoshi Nakamoto réalisait le vieux rêve
d'une monnaie numérique échappant au contrôle de l'État~: un rêve cher
aux cypherpunks dont le mouvement, remontant au début des années 1990,
prônait l'utilisation proactive de la cryptographie dans le but
d'assurer la confidentialité et la liberté des individus sur Internet.
Ces cryptographes rebelles avaient en effet désiré et tenté de concevoir
un tel argent liquide électronique pendant des années, celui-ci étant un
élément constitutif de leur idéal. Malheureusement, cela n'avait pas
abouti, du moins jusqu'à l'apparition de Bitcoin.

À partir de cette date fatidique, Bitcoin a été mis en œuvre et a connu
un certain nombre d'évènements fondateurs qui l'ont mené où il est
aujourd'hui. Ces évènements ont façonné la compréhension que nous en
avons, et l'histoire des débuts de Bitcoin constitue donc un récit
unique qu'il convient de raconter.

\section*{Une naissance difficile}\label{une-naissance-difficile}
\addcontentsline{toc}{section}{Une naissance difficile}

\markright{Une naissance difficile}

Bitcoin a été conçu par un individu qui utilisait le pseudonyme de
Satoshi Nakamoto et prétendait être un homme japonais de 33
ans\footnote{Certains partent du principe que Satoshi Nakamoto serait un
  pseudonyme utilisé par un groupe d'individus. Néanmoins, nous
  supposerons ici qu'il n'y avait qu'une seule personne derrière les
  messages et le code attribués au créateur de Bitcoin, sans pour autant
  nier que cette personne a pu se faire aider.}. On sait peu de choses
sur lui en dehors de ses messages publics et du code informatique qu'il
a publié. Satoshi a disparu en 2011, et on ignore s'il est toujours
vivant.

D'après son propre témoignage, Satoshi Nakamoto se met à travailler sur
Bitcoin au printemps 2007\footnote{«~Satoshi Nakamoto se met à
  travailler sur Bitcoin au printemps 2007~»~:
  \url{https://bitcointalk.org/index.php?topic=13.msg46\#msg46},
  \url{https://www.metzdowd.com/pipermail/cryptography/2008-November/014863.html}.}.
Pendant plus d'un an, il garde son modèle secret, souhaitant être sûr
qu'il fonctionne correctement avant de le présenter au monde. Il
affirmera ainsi avoir codé le prototype avant d'écrire le
papier\footnote{Satoshi Nakamoto, \emph{Bitcoin P2P e-cash paper},
  /11/2008 01:58:48 UTC~:
  \url{https://www.metzdowd.com/pipermail/cryptography/2008-November/014832.html}.}.

En août 2008, Satoshi a terminé de rédiger le livre blanc et commence à
préparer l'annonce de la sortie de Bitcoin. Le 18 août, il réserve le
nom de domaine Bitcoin.org via le service anonyme
AnonymousSpeech\footnote{Satoshi a également réservé le nom de domaine
  Netcoin.org au même moment, ce qui laisse à penser qu'il n'a pas
  encore finalisé son choix concernant le nom de son modèle. -- Or
  Weinberger sur Twitter, /09/2022 08:54 UTC~:
  \url{https://twitter.com/orweinberger/status/1573234325046558720}.}.
Le nom de domaine sera utilisé pour héberger le site principal de
Bitcoin.

Quelques jours plus tard, il rentre en contact avec Adam
Back\footnote{«~il rentre en contact avec Adam Back~»~: Adam Back,
  \emph{Re: Introduce yourself~:)}, /04/2013 11:27:49 UTC~:
  \url{https://bitcointalk.org/index.php?topic=15672.msg1873483\#msg1873483}.},
le cryptographe et cypherpunk britannique à l'origine de Hashcash, la
technique utilisée dans Bitcoin pour calculer la preuve de travail. Adam
Back le renvoie vers le cryptographe Wei Dai, inventeur en 1998 du
concept de b-money, un concept qui possède des similarités notables avec
Bitcoin. Le 22 août, Satoshi envoie donc un courriel à Wei Dai pour lui
dire qu'il «~se prépare à publier un document qui étend {[}ses{]} idées
à un système complètement fonctionnel~» et pour lui demander «~l'année
de publication de {[}sa{]} page sur la b-money~» afin d'y faire
référence dans le livre blanc\footnote{Gwern Branwen, \emph{Wei
  Dai/Satoshi Nakamoto 2009 Bitcoin emails}, 17 mars 2014~:
  \url{https://gwern.net/doc/bitcoin/2008-nakamoto}.}. Cependant, malgré
ces interactions, Adam Back et Wei Dai ne s'intéressent pas à Bitcoin
immédiatement. Ce ne sera que des années plus tard qu'ils reviendront
vers la découverte révolutionnaire de ce mystérieux personnage.

À l'automne 2008, Satoshi décide de rendre public son système. Le 5
octobre, il s'inscrit sur la plateforme de gestion de projets
SourceForge, là où le code source ouvert de Bitcoin sera hébergé et
maintenu jusqu'en 2011. Le 31 octobre, il publie le livre blanc sur une
liste de diffusion de courrier électronique dédiée à la cryptographie.
Cette liste est la \emph{Metzdowd Cryptography Mailing List} gérée par
Perry Metzger sur son site web Metzdowd.com où participent un certain
nombre d'anciens cypherpunks\footnote{Les archives de la liste de
  diffusion de Metzdowd sont disponibles publiquement à l'adresse
  \url{https://www.metzdowd.com/pipermail/cryptography/}. Les
  cypherpunks présents en 2008 étaient, entre autres~: John Gilmore, Hal
  Finney, James A. Donald, Robert Hettinga, Zooko Wilcox-O'Hearn et Len
  Sassaman.}. Dans son courriel d'introduction, il écrit~:

«~J'ai travaillé sur un nouveau système d'argent liquide électronique
qui est entièrement pair à pair, dépourvu de tiers de
confiance\footnote{Satoshi Nakamoto, \emph{Bitcoin P2P e-cash paper},
  /10/2008 18:10:00 UTC~:
  \url{https://www.metzdowd.com/pipermail/cryptography/2008-October/014810.html}.}.~»

Le livre blanc est centré sur le problème des paiements en ligne et le
but de Bitcoin est clairement énoncé dès le début~:

«~Le commerce sur Internet repose aujourd'hui presque exclusivement sur
des institutions financières qui servent de tiers de confiance pour
traiter les paiements électroniques. Bien que ce système fonctionne
assez bien pour la plupart des transactions, il souffre toujours des
faiblesses inhérentes à son modèle basé sur la confiance. {[}...{]} Ce
dont nous avons besoin, c'est d'un système de paiement électronique basé
sur des preuves cryptographiques plutôt que sur la confiance, qui
permettrait à deux parties volontaires de réaliser directement des
transactions entre elles sans avoir recours à un tiers de
confiance\footnote{Satoshi Nakamoto, \emph{Bitcoin: A Peer-to-Peer
  Electronic Cash System}, 31 octobre 2008.}.~»

D'un point de vue technique, il s'agit de mettre en place un registre de
transactions distribué sur un réseau pair à pair et ouvert
d'ordinateurs. Ce registre est composé de blocs de transactions qui sont
liés les uns à la suite des autres au cours du temps, formant une
«~chaîne de blocs~». Bitcoin constitue ainsi un «~serveur d'horodatage
distribué~», qui répertorie l'ordre des transactions de façon à créer un
historique cohérent, sans «~double dépense~». Cela permet de gérer
l'émission et les échanges d'une unité de compte numérique, qui sera
appelée le bitcoin.

La fiabilité du système repose sur des «~preuves de travail~» qui lient
les blocs entre eux de façon à rendre difficile la modification de la
chaîne. Ces preuves sont produites périodiquement par des membres du
réseau qui fournissent de l'énergie pour cela et qui sont rémunérés par
une «~incitation~» en bitcoins composée des pièces nouvellement créées
et des frais de transaction. Les personnes qui dépensent ainsi leur
énergie électrique sont comparées par Satoshi aux «~mineurs d'or qui
dépensent des ressources pour ajouter de l'or dans la circulation~»,
d'où le nom de mineurs qu'ils prendront plus tard.

Suite à l'annonce de Bitcoin et la publication du livre blanc, Satoshi
reçoit peu de réponses, et beaucoup d'entre elles sont sceptiques.
D'abord, le cypherpunk James A. Donald remet en cause le passage à
l'échelle du système en disant qu'«~il ne semble pas pouvoir s'adapter à
la taille requise\footnote{«~Nous avons vraiment, vraiment besoin d'un
  tel système, mais si je comprends bien votre proposition, il ne semble
  pas pouvoir s'adapter à la taille requise.~» -- James A. Donald,
  \emph{Re: Bitcoin P2P e-cash paper}, /11/2008, 23:46:23 UTC~:
  \url{https://www.metzdowd.com/pipermail/cryptography/2008-November/014814.html}}~».
Ensuite, John Levine critique sa sécurité en évoquant la puissance de
calcul détenue par les «~fermes de machines zombies\footnote{«~Les
  méchants contrôlent couramment des fermes de machines zombies de 100
  000 unités ou plus. Les personnes que je connais qui gèrent une liste
  noire de machines zombies émetteuses de spam me disent qu'elles voient
  souvent un million de nouveaux machines zombies par jour. C'est la
  même raison pour laquelle hashcash ne peut pas fonctionner sur
  l'Internet d'aujourd'hui~: les gentils ont une puissance de calcul
  nettement inférieure à celle des méchants.~» -- John Levine, \emph{Re:
  Bitcoin P2P e-cash paper}, /11/2008 13:32:39 UTC~:
  \url{https://www.metzdowd.com/pipermail/cryptography/2008-November/014817.html}.}~»
composées d'ordinateurs contrôlés par des pirates. Enfin, un troisième
individu du nom de Ray Dillinger s'interroge sur la valeur de l'unité de
compte, déplorant le fait que «~les preuves de travail informatiques
n'ont pas de valeur intrinsèque\footnote{«~Je pense que le vrai problème
  avec ce système est le marché des bitcoins. Les preuves de travail
  informatiques n'ont pas de valeur intrinsèque.~» -- Ray Dillinger,
  \emph{Re: Bitcoin P2P e-cash paper}, /11/2008 05:14:37 UTC~:
  \url{https://www.metzdowd.com/pipermail/cryptography/2008-November/014822.html}.}.~»

Cependant, cet accueil sceptique n'est pas partagé par l'intégralité des
personnes inscrites sur la liste de diffusion. En particulier, Hal
Finney, un informaticien et cryptographe américain d'une cinquantaine
d'années, est résolument enthousiaste et écrit dans son message du 7
novembre que «~Bitcoin semble être une idée très
prometteuse\footnote{Hal Finney, \emph{Re: Bitcoin P2P e-cash paper},
  /11/2008 23:40:12 UTC~:
  \url{https://www.metzdowd.com/pipermail/cryptography/2008-November/014827.html}.}~».
Hal Finney n'est pas une personne comme les autres~: il s'agit d'un
membre historique du mouvement cypherpunk qui a participé au
développement du logiciel de chiffrement PGP dans les années 90 aux
côtés de Philip Zimmermann, qui a expérimenté avec les premiers systèmes
de monnaie électronique et qui a même tenté de créer son propre système
de preuves de travail réutilisables. Malgré son expérience, il reste
optimiste et devient ainsi le tout premier soutien de Satoshi dans son
projet. Quelques années plus tard, il déclarera à ce sujet que «~les
cryptographes grisonnants {[}...{]} ont tendance à devenir cyniques~»
mais que lui «~était plus idéaliste~» ayant «~toujours aimé la
cryptographie, son mystère et son paradoxe\footnote{Hal Finney,
  \emph{Bitcoin and me}, /03/2013 20:40:02 UTC~:
  \url{https://bitcointalk.org/index.php?topic=155054.msg1643833\#msg1643833}.}.~»

Par la suite, Satoshi distribue les principaux fichiers du code aux
personnes intéressées, dont notamment Hal Finney, Ray Dillinger et James
A. Donald\footnote{Satoshi a écrit à James A. Donald~: «~Je t'ai envoyé
  les fichiers principaux. (disponibles sur demande pour le moment,
  publication complète bientôt)~» -- Satoshi Nakamoto, \emph{Re: Bitcoin
  P2P e-cash paper}, /11/2008 17:24:43~:
  \url{https://www.metzdowd.com/pipermail/cryptography/2008-November/014863.html}.}.
Hal et Ray réalisent alors un examen minutieux du code, en se
concentrant chacun sur une partie spécifique du système\footnote{«~Hal
  et Ray réalisent alors un examen minutieux du code~»~: Ray Dillinger,
  \emph{If I'd Known What We Were Starting}, 20 septembre 2017~:
  \url{https://www.linkedin.com/pulse/id-known-what-we-were-starting-ray-dillinger/}.}.
Ce code inclut déjà tous les éléments constitutifs de Bitcoin. Le
prototype est alors prêt à être lancé.

\section*{Une enfance timide}\label{une-enfance-timide}
\addcontentsline{toc}{section}{Une enfance timide}

\markright{Une enfance timide}

Deux mois après la publication du livre blanc, le 8 janvier 2009 à 19
heures 27, Satoshi Nakamoto partage la première version du logiciel sur
la liste de diffusion de Metzdowd. Le code source en C++ est publié de
manière ouverte sous licence libre (MIT), de sorte que n'importe qui
peut copier, modifier et utiliser le logiciel à sa guise. Celui-ci
contient les données du bloc de genèse, le premier bloc de la chaîne à
partir duquel celle-ci doit se prolonger.

Quelques heures plus tard, Satoshi commence à miner. Le deuxième bloc de
la chaîne, le bloc 1, est validé par Satoshi le 9 janvier à 2 heures 54
du matin, ce qui marque le lancement effectif du réseau.

Le 10 janvier, Hal tente de faire fonctionner le logiciel. Après avoir
échangé avec Satoshi pour faire en sorte que le logiciel
fonctionne\footnote{«~Après avoir échangé avec Satoshi~»~:
  \url{https://online.wsj.com/public/resources/documents/finneynakamotoemails.pdf}.},
il se met à miner et trouve son premier bloc (le bloc 78) à 1 heure du
matin (UTC), gagnant de ce fait 50 bitcoins. Deux heures et demie plus
tard, il partage son expérience sur Twitter (média social alors
naissant) en écrivant «~\emph{Running bitcoin}\footnote{Hal Finney sur
  Twitter, /01/2009 3:33 UTC~:
  \url{https://twitter.com/halfin/status/1110302988}.}~». Le lendemain,
dans la nuit du 11 au 12 janvier, Satoshi envoie 10 bitcoins à Hal par
l'intermédiaire de son adresse IP~: il s'agit du premier transfert d'une
personne à une autre sur le réseau\footnote{Cette première transaction
  entre Satoshi et Hal avait pour identifiant `` et a été confirmée dans
  le bloc 170 le 12 janvier à :30.}.

Hal n'est pas la seule personne à expérimenter sur le réseau à ce
moment-là~: c'est également le cas de Dustin Trammell, un chercheur en
sécurité informatique américain ayant découvert Bitcoin par la liste de
diffusion. Celui-ci communique aussi avec Satoshi par courriel, et
reçoit 25 bitcoins de sa part le 15 janvier\footnote{L'identifiant de la
  transaction reçue par Dustin (en P2IP) était ``.}\footnote{«~Dustin
  Trammell communique aussi avec Satoshi par courriel, et reçoit 25
  bitcoins de sa part le 15 janvier~»~:
  \url{http://web.archive.org/web/20131204164149/http://www.dustintrammell.com/files/Satoshi_Nakamoto.zip}.}.

Mais les quelques personnes qui font fonctionner le logiciel ne
suffisent pas. Dès le début, Satoshi sait bien que peu de gens se sont
penchés sérieusement sur son modèle et qu'il va être compliqué d'attirer
de nouveaux utilisateurs et contributeurs. C'est pourquoi il essaie de
susciter l'enthousiasme en vendant son idée du mieux possible.

Le premier élément est le programme d'émission du bitcoin, qui a pour
limite 21~millions d'unités. Dans le courriel d'annonce du prototype,
Satoshi explicite le rythme de création monétaire~:

«~La circulation totale sera de 21~000~000 de pièces. Elles seront
distribuées aux nœuds du réseau lorsqu'ils créeront des blocs, la
quantité émise étant divisée par deux tous les 4 ans. {[}...{]} Lorsque
cela sera épuisé, le système pourra prendre en charge les frais de
transaction si nécessaire\footnote{Satoshi Nakamoto, \emph{Bitcoin v0.1
  released}, /01/2009 19:27:40 UTC~:
  \url{https://www.metzdowd.com/pipermail/cryptography/2009-January/014994.html}.}.~»

Le bitcoin a donc vocation à devenir une monnaie à offre fixe,
déflationniste par nature, et cette particularité crée un enthousiasme.
Le 11 janvier, Hal Finney est le premier à réagir en s'enthousiasmant du
fait que «~le système peut être configuré pour n'autoriser qu'un nombre
maximum certain de pièces à être générées~». Il estime alors que si
«~Bitcoin {[}réussit{]} et {[}devient{]} le système de paiement dominant
utilisé dans le monde entier~», chaque pièce aura alors «~une valeur
d'environ 10 millions~» de dollars\footnote{Hal Finney, \emph{Re:
  Bitcoin v0.1 released}, /01/2009 02:22:01 UTC~:
  \url{https://www.metzdowd.com/pipermail/cryptography/2009-January/015004.html}.}.
L'estimation est contestable mais le raisonnement reste pertinent en
raison du fonctionnement de Bitcoin.

Le 16 janvier, Satoshi reprend ainsi cet élément de communication dans
un courriel qu'il partage à la liste de diffusion, où il déclare qu'il
«~pourrait être judicieux d'en avoir au cas où cela prendrait~» et que
«~si suffisamment de gens pensent la même chose, cela deviendra une
prophétie autoréalisatrice\footnote{Satoshi Nakamoto, \emph{Bitcoin v0.1
  released}, /01/2009 16:03:14 UTC~:
  \url{https://www.metzdowd.com/pipermail/cryptography/2009-January/015014.html}}~».
Cet élément est crucial, comme le montre le témoignage de Dustin
Trammell qui confie à Satoshi que le raisonnement de Hal est «~l'une des
autres raisons pour lesquelles {[}il a{]} démarré un nœud si
rapidement~».

Outre le programme d'émission du bitcoin, Satoshi choisit de communiquer
sur les défaillances du système bancaire, ce qui constitue le deuxième
élément dans sa stratégie pour attirer l'attention.

En réalité, il le fait dès le bloc de genèse en y incluant le titre de
la une du quotidien britannique \emph{The Times} du 3 janvier 2009
annonçant que le ministre des finances britannique est sur le point de
renflouer les banques pour la deuxième fois~:

\emph{The Times 03/Jan/2009 Chancellor on brink of second bailout for
banks}

Cette phrase présente dans le premier bloc de la chaîne possède un rôle
double~: d'une part, elle empêche l'antidatage en prouvant que le
système n'a pas été lancé avant le 3 janvier (Satoshi ne pouvait pas
connaître cette une avant)~; d'autre part, elle indique symboliquement
ce à quoi Bitcoin s'oppose en faisant référence au contexte monétaire et
financier de l'époque.

En janvier 2009, le monde subit en effet de plein fouet les effets de la
crise financière amorcée en 2007 par le dégonflement de la bulle
immobilière aux États-Unis aussi connu sous le nom de la crise des
subprimes. Les États renflouent les banques pour éviter de nouvelles
faillites bancaires après celle de Lehman Brothers survenue le 15
septembre 2008, et les banques centrales procèdent à des
assouplissements quantitatifs en injectant des liquidités sur les
marchés financiers. Cette utilisation d'argent public, qui est
littéralement créé pour l'occasion, choque profondément un certain
nombre de citoyens qui réalisent que le système bancaire est en fait un
système de profits privés et de pertes socialisées.

De par son absence de tiers de confiance, Bitcoin n'est, lui, pas soumis
à l'arbitraire d'une banque centrale. Il contraste ainsi avec les
monnaies étatiques, telles que le dollar ou l'euro, dont la quantité
peut être modifiée arbitrairement par ceux qui contrôlent la création
monétaire, au moyen de ce qu'on appelle une politique monétaire. La
politique monétaire du bitcoin est programmée, inscrite en dur dans le
protocole, pour en théorie ne plus jamais être altérée.

C'est ce que met en avant Satoshi lorsqu'il intervient sur le forum de
la Fondation P2P, une organisation étudiant l'impact des infrastructures
pair à pair sur la société, le 11 février 2009. Dans son message
d'introduction à Bitcoin, il écrit~:

«~Le problème fondamental de la monnaie conventionnelle est toute la
confiance nécessaire pour la faire fonctionner. Il faut faire confiance
à la banque centrale pour qu'elle ne déprécie pas la monnaie, mais
l'histoire des monnaies fiat est pleine de violations de cette
confiance. Il faut faire confiance aux banques pour détenir notre argent
et le transférer par voie électronique, mais elles le prêtent par vagues
de bulles de crédit avec à peine une fraction en réserve\footnote{Satoshi
  Nakamoto, \emph{Bitcoin open source implementation of P2P currency},
  11 février 2009~:
  \url{https://p2pfoundation.ning.com/forum/topics/bitcoin-open-source}.}.~»

Sur son profil où il indique sur son profil être un homme de 33 ans
habitant au Japon, il donne une date de naissance particulière~: le 5
avril 1975. Cette date, probablement fictive et composite, fait
vraisemblablement référence à l'interdiction pour les particuliers de
détenir de l'or aux États-Unis. Le jour du 5 avril se rapporte au jour
de l'instauration de cette interdiction par l'Ordre exécutif 6102 signé
par Franklin Delano Roosevelt le 5 avril 1933, et l'année 1975
correspond à son année d'abrogation lors de l'entrée en vigueur de la
\emph{Public Law} 93-373. Ce détail n'est pas anodin, puisque cette
prohibition a permis en fin de compte d'instaurer un régime monétaire
flottant n'ayant plus aucun lien avec l'or.

Ce n'est pas la seule référence aux métaux précieux. Satoshi écrit dans
les commentaires le 18 février~:

«~Il n'y a personne pour agir en tant que banque centrale ou réserve
fédérale afin d'ajuster l'offre monétaire au fur et à mesure que le
nombre d'utilisateurs augmente. {[}...{]} En ce sens, il se comporte
davantage comme un métal précieux. Plutôt que de faire varier l'offre
pour que la valeur reste la même, on détermine l'offre à l'avance et la
valeur change. Plus le nombre d'utilisateurs augmente, plus la valeur
d'une pièce augmente. Cela peut créer une boucle de rétroaction
positive~; plus le nombre d'utilisateurs grandit, plus la valeur
augmente, ce qui peut attirer davantage d'utilisateurs désireux de
profiter de cet accroissement de la valeur\footnote{Satoshi Nakamoto,
  \emph{Re: Bitcoin open source implementation of P2P currency}, 18
  février 2009~:
  \url{https://p2pfoundation.ning.com/forum/topics/bitcoin-open-source?commentId=2003008:Comment:9562}.}.~»

Cette méthode de communication porte peu à peu ses fruits. Ainsi, même
si certaines personnes finissent de se détourner de Bitcoin à l'instar
de Hal Finney, Satoshi continue de recevoir des messages de la part de
personnes intéressées. Le 11 avril 2009, Mike Hearn, un développeur
britannique travaillant pour Google et s'adonnant au logiciel libre sur
son temps libre, lui envoie un courriel posant une série de questions à
propos de Bitcoin, en précisant qu'«~il est rare de rencontrer des idées
vraiment révolutionnaires\footnote{Mike Hearn, \emph{Questions about
  BitCoin}, /04/2009 22:46 UTC~:
  \url{https://plan99.net/~mike/satoshi-emails/thread1.html}.}~». Hearn
s'intéresse alors aux monnaies numériques, et notamment à Ripple.

Début mai 2009, c'est un jeune étudiant en informatique finlandais qui
contacte Satoshi~: il s'agit de Martti Malmi. Celui-ci a découvert
Bitcoin début avril, s'est mis à miner et a même rédigé une courte
description de Bitcoin sur le forum de Freedomain Radio où il soutenait
l'hypothèse anarchiste que la monnaie pair à pair pourrait faire
disparaître l'État\footnote{Martti Malmi, \emph{P2P Currency could make
  the government extinct?}, /04/2009 17:49:47 UTC~:
  \url{https://web.archive.org/web/20150927195115/https://board.freedomainradio.com/topic/17233-p2p-currency-could-make-the-government-extinct/}.}.
Dans son courriel à Satoshi, il écrit~:

«~J'ai une bonne connaissance des langages Java et C grâce aux cours que
j'ai suivis à l'école (j'étudie l'informatique) mais je n'ai pas encore
beaucoup d'expérience en matière de développement. J'aimerais aider avec
Bitcoin, s'il y a quelque chose que je peux faire\footnote{Nathaniel
  Popper, \emph{Digital Gold: Bitcoin and the Inside Story of the
  Misfits and Millionaires Trying to Reinvent Money}, Harper Paperbacks,
  2016, p.~29.}.~»

Malgré son manque d'expérience, Martti devient dans les mois qui suivent
le principal contributeur à Bitcoin en dehors de Satoshi. Étant
étudiant, il a en effet beaucoup de temps à consacrer au projet.

En particulier, Satoshi lui confie la charge du site web. Dès le mois de
mai, Martti Malmi rédige une première version de la description sur
SourceForge où il présente Bitcoin comme une «~monnaie numérique anonyme
basée sur un réseau pair à pair~» permettant de «~transférer de l'argent
facilement par Internet, sans avoir à faire confiance à des tiers~» et
d'être «~à l'abri de l'instabilité causée par le système de réserves
fractionnaires et par les mauvaises politiques des banques
centrales\footnote{Archive de la page web de Bitcoin~:
  \url{https://web.archive.org/web/20090511173000/http://bitcoin.sourceforge.net/}.}~».
Cette ébauche servira de base pour la présentation de Bitcoin sur le
site web.

À l'époque le bitcoin n'a pas de prix. Les gens qui testent le système
se contentent de lancer le logiciel pour «~générer des pièces~». Les
transactions sont peu nombreuses, et consistent le plus souvent en des
auto-transferts. Les bitcoins sont alors vus comme des collectionnables
réservés aux passionnés d'informatique. Les utilisateurs ont
l'impression de contribuer à quelque chose, à l'instar des projets de
calcul distribué (dits «~@home~») où les gens mettent à disposition
leurs ressources informatiques au service de bonnes causes.

Certains individus minent en continu\footnote{«~Certains individus
  minent en continu~»~: Ludovic Lars, \emph{Les premiers mineurs de
  Bitcoin}, 19 juin 2022~:
  \url{https://journalducoin.com/analyses/premiers-mineurs-bitcoin/}.}.
C'est le cas de Hal Finney qui fait fonctionner le logiciel entre
janvier et mars, de James Howells qui valide des blocs entre février et
avril, de Dustin Trammell qui fait tourner ses serveurs pendant plus
d'un an, ou de Martti Malmi qui met son ordinateur portable à profit à
partir d'avril. Mais le principal mineur de l'année de 2009 reste
Satoshi, qui déploie une puissance de calcul bien plus grande et dont la
production de blocs représente près de la moitié de celle du réseau.

En 2009, la difficulté de minage est de 1, ce qui impose à tous les
nœuds du réseau de réaliser environ 4,3 millions de calculs pour miner
un bloc, et ce n'est pas rien pour un processeur. De ce fait, la
production est plus lente que prévue~: entre le 3 janvier 2009 et le 3
janvier 2010, seulement 32~880 blocs sont trouvés sur les 52560
attendus, ce qui correspond à une durée moyenne entre chaque bloc de 16
minutes au lieu des 10 minutes prévues. En particulier, le mois d'août
2009 constitue le pire mois pour l'activité minière~: seuls 1~564 sur
4~464 blocs attendus sont trouvés, soit un temps moyen de 28 minutes et
30 secondes~!

\section*{Des premiers pas
incertains}\label{des-premiers-pas-incertains}
\addcontentsline{toc}{section}{Des premiers pas incertains}

\markright{Des premiers pas incertains}

Malgré son lancement timide, Bitcoin survit à l'été et franchit une
étape cruciale en octobre~: son unité de compte acquiert un prix. Un
individu utilisant le pseudonyme NewLibertyStandard (NLS), nouvellement
arrivé dans la communauté, met en place sur sa page personnelle un
service de change permettant aux gens de convertir leurs dollars en
bitcoins et inversement. Pour estimer le taux de change, il se base sur
le coût énergétique nécessaire pour obtenir un bitcoin, en prenant en
compte le coût de l'électricité à son emplacement et la fréquence de sa
production personnelle. Les prix sont publiés quotidiennement sur son
site\footnote{«~Les prix sont publiés quotidiennement sur son site~»~:
  \url{https://web.archive.org/web/20091229132610/http://newlibertystandard.wetpaint.com/page/Exchange+Rate}.}.

Le 12 octobre 2009, a ainsi lieu la première vente de bitcoins en
dollars entre Martti Malmi et NewLibertyStandard~: Martti cède 5050
bitcoins à NLS pour 5,02~\$ virés sur son compte PayPal, ce qui
correspond à un prix d'environ 0,001~\$ par bitcoin\footnote{«~J'ai
  trouvé la première transaction connue de bitcoins en USD dans mes
  sauvegardes de courriel. J'ai vendu 5~050 BTC pour 5,02~\$ le
  12-10-2009.~» -- Martti Malmi sur Twitter, 15/01/2014~:
  \url{https://twitter.com/marttimalmi/status/423455561703624704}.
  L'identifiant de la transaction était ``.}. NLS effectuera par la
suite d'autres échanges au cours des mois suivants, constituant la seule
passerelle entre le dollar et le bitcoin.

Le 22 novembre marque l'ouverture du nouveau forum, sobrement appelé le
\emph{Bitcoin Forum}, qui est hébergé sur Bitcoin.org et géré par Martti
Malmi. Ce forum abrite l'essentiel des discussions sur Bitcoin à partir
de cette date. Il sera renommé en Bitcointalk en août 2011 et hébergé à
une nouvelle adresse.

Le 16 décembre 2009, Satoshi annonce la sortie de la version 0.2 du
logiciel, version pour laquelle Martti Malmi est grandement crédité, ce
qui clôt la première période de développement informatique de
Bitcoin\footnote{«~Satoshi annonce la sortie de la version 0.2 du
  logiciel~»~: Satoshi Nakamoto, \emph{Bitcoin 0.2 released!}, /12/2009
  22:45:36 UTC~:
  \url{https://bitcointalk.org/index.php?topic=16.msg73\#msg73}.}.
L'année se termine en beauté lorsque la difficulté augmente enfin, en
passant de 1 à 1,18 le 30 décembre.

Au début de l'année 2010, le bitcoin est désigné comme une
«~cryptomonnaie~» (\emph{cryptocurrency}) sur le site web\footnote{«~Au
  début de l'année 2010, le bitcoin est désigné comme une
  "cryptomonnaie"~»~:
  \url{https://web.archive.org/web/20100106082749/http://www.bitcoin.org/}}.
Le préfixe crypto- (qui vient du grec ancien
\foreignlanguage{greek}{kruptos}, kruptós, indiquant ce qui est caché,
occulté) possède une signification double~: il renvoie à la
cryptographie sur laquelle Bitcoin s'appuie, et à la confidentialité,
Bitcoin étant alors présenté comme une «~monnaie numérique anonyme~».

Ce nouveau terme confirme le but central de Bitcoin~: devenir une
monnaie, c'est-à-dire un intermédiaire dans les échanges. Cela nécessite
des personnes qui génèrent des transactions (par le biais du commerce)
et d'autres qui traitent ces transactions (par le biais du minage).
C'est donc tout naturellement que l'expansion de ces deux aspects
complémentaires se produit à ce moment-là.

Le premier développement est l'essor commercial dont NewLibertyStandard
peut être considéré comme le pionnier. Non seulement il est le premier
commerçant à accepter le bitcoin comme moyen de paiement par
l'intermédiaire de son service d'échange, mais il est aussi l'un des
promoteurs originels de cet effort de construction économique. Dans son
premier message sur le forum le 19 janvier 2010, il écrit ainsi~:

«~Des gens m'ont acheté des bitcoins et m'en ont vendus. L'offre et la
demande, même si elle sont faibles, existent déjà et c'est tout ce qu'il
faut. Proposer d'échanger des bitcoins contre une autre monnaie n'est en
fin de compte pas différent de l'échange de bitcoins contre des biens ou
des services. Les monnaies sont des biens et le change est un service.
{[}...{]} Vous pouvez acheter tous mes dollars ou bitcoins aujourd'hui,
mais il y en aura toujours plus demain et après-demain. Toutes les
personnes qui achètent ou vendent des biens en utilisant des bitcoins, y
compris les changeurs, font progresser l'économie de Bitcoin. Que tout
le monde fasse sa part. Achetez ou vendez quelque chose en échange de
bitcoins~\footnote{NewLibertyStandard, \emph{Re: New Exchange Service:
  "BTC 2 PSC"}, /01/2010 08:06:15 UTC~:
  \url{https://bitcointalk.org/index.php?topic=15.msg111\#msg111}.}!~»

Dans les mois qui suivent, les services de change se développent, comme
BitcoinFX ou Bitcoin Market. C'est pourquoi NLS propose que le bitcoin,
à l'instar des monnaies échangées sur le marché des changes, adopte le
sigle boursier BTC et le symbole du baht thaïlandais\footnote{«~NLS
  propose que le bitcoin {[}...{]} adopte le sigle boursier BTC et le
  symbole du baht thaïlandais~»~: NewLibertyStandard, \emph{Bitcoin
  Currency Symbol ฿}, /02/2010 01:48:53 UTC~:
  \url{https://bitcointalk.org/index.php?topic=41.msg238\#msg238}.}.
L'utilisation du sigle BTC se normalise rapidement. Quant au symbole (le
B majuscule traversé par deux barres verticales rappelant
immanquablement le dollar), c'est Satoshi lui-même qui le conçoit, en
s'inspirant de la proposition de NLS, lors de la création du premier
véritable logo de Bitcoin\footnote{Satoshi Nakamoto, \emph{New
  icon/logo}, /02/2010 21:24:23 UTC~:
  \url{https://bitcointalk.org/index.php?topic=64.msg504\#msg504}.}.

\begin{figure}[H]

{\centering \includegraphics{chapters/img/bitcoin530.png}

}

\caption{Logo de Bitcoin conçu par Satoshi Nakamoto en février 2010.}

\end{figure}%

Les vendeurs de biens et de services apparaissent également. Outre son
service de change, NLS ouvre un magasin en ligne où il propose à la
vente des timbres et des autocollants\footnote{«~NLS ouvre un magasin en
  ligne où il propose à la vente des timbres et des autocollants~»~:
  Liberty Swap Variety Shop,
  \url{https://web.archive.org/web/20100414172623/http://newlibertystandard.wetpaint.com/page/Specialty+Shop}.}.
D'autres services acceptant le bitcoin apparaissent comme le service de
voix sur IP Link2VoIP, l'hébergeur web Vekja.net et le vendeur de noms
de domaines Privacy Shark\footnote{«~D'autres services acceptant le
  bitcoin apparaissent~»~:
  \url{https://web.archive.org/web/20100517040312/http://www.bitcoin.org:80/trade}.}.
En parallèle, la première partie de poker mettant en jeu des bitcoins
est organisée, ce qui inaugure la relation forte qui existera entre le
jeu d'argent et la cryptomonnaie\footnote{«~la première partie de poker
  mettant en jeu des bitcoins est organisée~»~: Kai Sedgwick,
  \emph{Bitcoin History Part 14: The 1,000 BTC Poker Game}, 9 août
  2019~:
  \url{https://news.bitcoin.com/bitcoin-history-part-14-the-1000-btc-poker-game/}.}.

Enfin, en avril 2010, naît MyBitcoin, une application web dépositaire
permettant un usage facile et serein de Bitcoin, notamment sur mobile.
Grâce à celle-ci, les utilisateurs n'ont en effet pas besoin de
télécharger les données complètes pour envoyer et recevoir des
transactions, ni de conserver leurs bitcoins eux-mêmes en sauvegardant
leurs clés privées. À cette époque, les portefeuilles légers n'existent
pas, si bien que Satoshi lui-même juge qu'il est alors acceptable de
passer par ce type d'application, même si cela va à l'encontre du
principe de désintermédiation à la base de Bitcoin~:

«~Le seul inconvénient c'est qu'il faut faire confiance au site, mais
cela ne pose pas de problème pour la petite monnaie, pour les
micropaiements et les dépenses diverses\footnote{Satoshi Nakamoto,
  \emph{Re: Ummmm... where did my bitcoins go?}, /05/2010 20:06:46 UTC~:
  \url{https://bitcointalk.org/index.php?topic=125.msg1149\#msg1149}.}.~»

L'année 2010 est également celle de l'essor du minage, qui se manifeste
en premier lieu par l'émergence du minage par processeur graphique
(GPU). Jusqu'alors, les mineurs sollicitaient leur processeur central
(CPU) pour extraire de nouveaux bitcoins. Néanmoins, ces derniers
processeurs s'avèrent peu performants pour effectuer des opérations
répétées, comparés aux cartes graphiques qui sont largement plus
adaptées à ce type de calcul répétitif. Par conséquent, tout le monde
sait à ce moment-là que cette évolution est inéluctable, y compris
Satoshi qui déclare en décembre 2009 que la communauté doit «~se mettre
d'accord pour reporter la course aux armements des GPU aussi longtemps
que possible pour le bien du réseau\footnote{Satoshi Nakamoto, \emph{Re:
  A few suggestions}, /12/2009 17:52:44 UTC~:
  \url{https://bitcointalk.org/index.php?topic=12.msg54\#msg54}}~».

La boîte de Pandore est ouverte par Laszlo Hanyecz, un développeur
américain d'origine hongroise de 28 ans, qui découvre Bitcoin en avril.
Après avoir acheté des bitcoins à NLS\footnote{«~Après avoir acheté des
  bitcoins à NLS~»~:
  \url{https://blockchair.com/bitcoin/transaction/faf172f5dc06b0ae03268555dddcd65be47e9a8a8bb44a122b12bfaf735f9a81?o=1}}
et essayé le système de transactions, celui-ci programme début mai un
logiciel de minage qui s'adapte aux cartes graphiques\footnote{«~celui-ci
  programme début mai un logiciel de minage qui s'adapte aux cartes
  graphiques~»~: Laszlo Hanyecz, \emph{Generating Bitcoins with your
  video card (OpenCL/CUDA)}, /05/2010, 14:03:57 UTC~:
  \url{https://bitcointalk.org/index.php?topic=133.msg1103\#msg1103}.}.
Cette optimisation lui permet d'occuper rapidement une place importante
dans la production des blocs. Ceci attire l'attention de Satoshi
Nakamoto qui le contacte et lui demande de ralentir ses opérations afin
que le minage reste accessible à tous~:

«~L'un des principaux attraits pour les nouveaux utilisateurs est que
toute personne disposant d'un ordinateur peut générer des pièces
gratuites. Lorsqu'il y aura 5~000 utilisateurs, cette incitation
s'estompera peut-être, mais pour l'instant, c'est toujours vrai. Les GPU
limiteraient prématurément cette incitation à ceux qui disposent d'un
matériel GPU haut de gamme. Il est inévitable que les clusters de calcul
GPU finiront par accaparer toutes les pièces générées, mais je ne veux
pas précipiter l'arrivée de ce jour-là. {[}...{]} Je ne veux pas passer
pour un socialiste, je me moque de la concentration des richesses, mais
pour l'instant, nous obtenons plus de croissance en donnant cet argent à
100~\% des gens qu'en le donnant à 20~\%\footnote{Satoshi Nakamoto, mai
  2010, propos rapportés par Nathaniel Popper~:
  \url{https://www.reddit.com/r/Bitcoin/comments/36vnmr/heres_what_satoshi_wrote_to_the_man_responsible/}.}.~»

Laszlo abaisse sa cadence, mais continue à miner avec sa carte
graphique. Avec sa méthode, il accumule ainsi des dizaines de milliers
de bitcoins.

Toutefois, cela n'est pas entièrement négatif pour le projet car il
finit par réinjecter ses bitcoins dans l'économie de la façon la plus
emblématique possible~: en achetant quelque chose avec, et plus
précisément des pizzas. Le 18 mai 2010, il écrit ainsi l'annonce
suivante sur le forum~:

«~Je paierai 10~000 bitcoins pour deux ou trois pizzas... genre
peut-être 2 grandes pour qu'il m'en reste le lendemain. J'aime avoir des
restes de pizza à grignoter pour plus tard. Vous pouvez faire la pizza
vous-même et l'amener jusqu'à chez moi ou la commander pour moi dans un
service de livraison, mais mon objectif c'est de me faire livrer, en
échange de bitcoins, de la nourriture que je n'ai pas à commander ou à
préparer moi-même. {[}...{]} Si vous êtes intéressé, faites-le moi
savoir et nous pourrons nous arranger\footnote{Laszlo Hanyecz,
  \emph{Pizza for bitcoins?}, /05/2010 00:35:20 UTC~:
  \url{https://bitcointalk.org/index.php?topic=137.msg1141\#msg1141}.}.~»

Cette offre trouve preneur au bout de quatre jours. Le 22 mai, un jeune
Californien du nom de Jeremy Sturdivant accepte l'échange sur la
messagerie instantanée IRC\footnote{«~un jeune Californien du nom de
  Jeremy Sturdivant accepte l'échange sur la messagerie instantanée
  IRC~»~: Bitcoin Who's Who, \emph{A Living Currency}, 22 mai 2015~:
  \url{https://www.bitcoinwhoswho.com/jercosinterview}~; archive~:
  \url{https://web.archive.org/web/20150528074728/http://bitcoinwhoswho.com/jercosinterview/}.}~:
il commande deux pizzas de Papa John's qui sont livrées chez Laszlo à
Jacksonville en Floride, et reçoit en échange 10~000
bitcoins\footnote{L'identifiant de la transaction de la pizza entre
  Laszlo Hanyecz et Jeremy Sturdivant était ``.}, ce qui représente
alors environ 44~\$ sur Bitcoin Market. Cela clôt le premier achat d'un
bien physique en bitcoins~! Cet évènement symbolique sera par la suite
commémoré tous les ans à cette date comme le \emph{Bitcoin Pizza Day}.

Une autre personne vient contribuer au succès du projet. Vers la fin du
mois de mai, un développeur américain de 44 ans, nommé Gavin Andresen,
découvre Bitcoin par le biais d'un article publié sur
InfoWorld\footnote{«~un développeur américain de 44 ans, nommé Gavin
  Andresen, découvre Bitcoin par le biais d'un article publié sur
  InfoWorld~»~: Neil McAllister, \emph{Open source innovation on the
  cutting edge}, 24 mai 2010~:
  \url{https://www.infoworld.com/article/2627013/open-source-innovation-on-the-cutting-edge.html?page=3}.}.
De retour d'Australie, momentanément sans emploi, il se met à travailler
sur son premier projet~: un robinet à bitcoins (\emph{bitcoin faucet})
qui donne des bitcoins à quiconque en fait la requête. Le 11 juin, Gavin
lance son service et le présente sur le forum~:

«~Pour mon premier projet de programmation sur Bitcoin, j'ai décidé de
faire quelque chose qui semble vraiment stupide~: j'ai créé un site web
qui distribue des bitcoins. {[}...{]} Pourquoi~? Parce que je veux que
le projet Bitcoin réussisse, et je pense qu'il a plus de chances de
réussir si les gens peuvent obtenir une poignée de pièces pour
l'essayer\footnote{Gavin Andresen, \emph{Get 5 free bitcoins from
  freebitcoins.appspot.com}, /06/2010, 17:38:45 UTC~:
  \url{https://bitcointalk.org/index.php?topic=183.msg1488\#msg1488}.}.~»

Ce \emph{faucet}, qui offre d'abord 5 bitcoins par requête au tout
début, est approuvé par Satoshi, ce dernier ayant «~prévu de faire
exactement la même chose si quelqu'un d'autre ne l'avait pas
fait\footnote{Satoshi Nakamoto, \emph{Re: Get 5 free bitcoins from
  freebitcoins.appspot.com}, /06/2010, 23:08:34 UTC~:
  \url{https://bitcointalk.org/index.php?topic=183.msg1620\#msg1620}.}~».
Le service, sollicité par beaucoup de personnes, distribuera plus de
19~700 bitcoins jusqu'à sa fermeture deux ans plus tard\footnote{«~jusqu'à
  sa fermeture deux ans plus tard~»~: Gavin Andresen, \emph{Bitcoin
  Faucet Hacked}, 2 mars 2012~:
  \url{https://gavintech.blogspot.com/2012/03/bitcoin-faucet-hacked.html}.}.
De plus, Gavin s'implique dans le développement du logiciel et échange
beaucoup avec Satoshi par courriel. Il en devient rapidement le bras
droit grâce à la confiance qu'il lui inspire.

Malgré cette croissance économique encourageante, l'activité reste
extrêmement réduite sur le réseau. Le 30 juin, sur la liste de diffusion
de Bitcoin, James A. Donald déclare ainsi que «~Bitcoin est en quelque
sorte mort~» et que «~le problème est que le bitcoin a besoin d'une
écologie d'utilisateurs pour être utile\footnote{James A. Donald,
  \emph{Re: {[}bitcoin-list{]} New User}, /06/2010 22:29:16~:
  \url{https://web.archive.org/web/20131016002646/http://sourceforge.net/p/bitcoin/mailman/bitcoin-list/?viewmonth=201006}.
  -- Dans un autre courriel, James A. Donald ajoutait~: «~Je ne voulais
  pas paraître si négatif. Si nous y arrivons, c'est une grande victoire
  pour la liberté -- mais c'est un long périple, et je suis occupé par
  un autre projet.~».}~». Toutefois, quelques jours plus tard, un
évènement vient lui donner tort.

\section*{Le slashdotting}\label{le-slashdotting}
\addcontentsline{toc}{section}{Le slashdotting}

\markright{Le slashdotting}

Le 11 juillet 2010, suite à la sortie de la version 0.3 du logiciel, une
courte présentation de Bitcoin rédigée par un utilisateur est publiée
sur Slashdot, un site d'actualité très populaire traitant de sujets pour
les \emph{nerds} comme l'informatique, les jeux vidéo, la science,
Internet,~etc. L'argumentaire de vente est le suivant~:

«~Que pensez-vous de cette technologie disruptrice~? Bitcoin est une
monnaie numérique basée sur un réseau pair à pair, sans banque centrale,
et sans frais de transaction. À l'aide d'un concept de preuve de
travail, les nœuds brûlent des cycles de processeur pour chercher des
paquets de pièces et diffusent leurs résultats sur le réseau. L'analyse
de la consommation d'énergie révèle que la valeur marchande des bitcoins
est déjà supérieure à la valeur de l'énergie nécessaire pour les
générer, ce qui indique une demande saine. La communauté a bon espoir
que la monnaie restera hors de portée de tout État\footnote{teppy,
  «~\emph{Bitcoin Releases Version 0.3}~», \emph{Slashdot}, 11 juillet
  2010~:
  \url{https://news.slashdot.org/story/10/07/11/1747245/Bitcoin-Releases-Version-03}.}.~»

Ceci provoque un afflux massif de nouveaux visiteurs sur le site et sur
le forum, ainsi qu'une augmentation du nombre d'utilisateurs et de
mineurs sur le réseau. Le réseau tient le coup malgré la montée en
charge\footnote{«~Le réseau tient le coup malgré la montée en charge~»~:
  Gavin Andresen, \emph{Re: Scalability}, /7/2010, 04:22:49 UTC~:
  \url{https://bitcointalk.org/index.php?topic=286.msg2745\#msg2745}.}.
En conséquence, le prix du bitcoin connaît la première hausse majeure de
son histoire, en passant de 0,008~\$ à 0,08~\$ en une semaine, soit une
multiplication par 10~!

Parmi les personnes qui découvrent Bitcoin grâce à Slashdot, il y a Jed
McCaleb\footnote{«~Parmi les personnes qui découvrent Bitcoin grâce à
  Slashdot, il y a Jed McCaleb~»~: The Ripple Blog, \emph{Interview with
  Jed McCaleb, inventor of the Ripple protocol and co-founder of
  OpenCoin}, 17 avril 2013~:
  \url{https://web.archive.org/web/20130428155220/https://ripple.com/blog/interview-with-jed-mccaleb-inventor-of-the-ripple-protocol-and-co-founder-of-opencoin/}.},
un entrepreneur et programmeur américain de 35 ans, connu pour avoir
cofondé et développé le logiciel de partage de fichiers en pair à pair
eDonkey2000 dans les années 2000. Constatant à quel point il est pénible
de se procurer du bitcoin contre des dollars, il décide de créer une
place de marché spécialisée. Pour ce faire, il réutilise un de ses
anciens projets mis au point en 2007~: \emph{Magic The Gathering Online
eXchange} (MTGOX), un site web qui permettait d'acheter et de vendre des
cartes du jeu en ligne \emph{Magic: The Gathering Online}\footnote{Gwern
  Branwen, \emph{2014 Jed McCaleb MtGox interview}, 16 février 2014~:
  \url{https://www.gwern.net/docs/bitcoin/2014-mccaleb}.}. Il reprend le
même nom de domaine au passage~: mtgox.com.

Une semaine plus tard, le 18, la plateforme de change Mt. Gox
(«~\emph{Mount Gox}~») est lancée et annoncée officiellement sur le
forum par Jed\footnote{«~la plateforme de change Mt. Gox {[}...{]} est
  lancée et annoncée officiellement sur le forum par Jed~»~: Jed
  McCaleb, \emph{New Bitcoin Exchange}, /07/2010 01:57:19 UTC~:
  \url{https://bitcointalk.org/index.php?topic=444.msg3866\#msg3866}.}.
Grâce à son expertise, il fait en sorte que la plateforme fonctionne
comme une place de marché automatisée, à l'instar des bourses en ligne
modernes. Elle se distingue de Bitcoin Market par le fait qu'elle est
«~toujours en ligne, automatisée~», que «~le site est plus rapide et a
un hébergement dédié~» et que «~l'interface est plus
agréable\footnote{Jed McCaleb, \emph{Re: New Bitcoin Exchange}, /07/2010
  02:53:07 UTC~:
  \url{https://bitcointalk.org/index.php?topic=444.msg3891\#msg3891}.}~».
Par conséquent, Mt. Gox s'impose rapidement comme le moyen principal de
se procurer du bitcoin, devenant la référence en ce qui concerne le
cotation en dollars.

Le minage connaît également une phase ascendante. L'afflux de nouveaux
mineurs fait passer le taux de hachage du réseau (le nombre de calculs
par seconde) au-dessus du milliard de calculs par seconde (1~GH/s) dès
le 13 juillet. Certains mineurs développent leur propre algorithme de
minage par GPU. C'est le cas de ArtForz, un développeur allemand, qui se
met à miner le 19 juillet et qui construit au cours du temps la première
ferme de minage de Bitcoin, qui sera connue sous le nom
d'«~ArtFarm~»\footnote{Le 13 août 2010, le ferme de minage d'ArtForz
  était constituée de 6 cartes graphiques ATI Radeon HD 5770~; à la fin,
  elle se composait de 24 ATI Radeon HD 5970. -- Tim Swanson, \emph{How
  ArtForz changed the history of Bitcoin mining}, 20 avril 2014~:
  \url{https://www.ofnumbers.com/2014/04/20/how-artforz-changed-the-history-of-bitcoin-mining/}.}.

Mais cette croissance suivant la présentation sur Slashdot provoque
également des problèmes d'ordre technique, mettant le système à
l'épreuve. Deux incidents viennent ainsi perturber le projet.

Le premier incident est la découverte d'une vulnérabilité dans le code
de Bitcoin qui rend possible la dépense de bitcoins à partir de
n'importe quelle adresse (cette vulnérabilité sera appelée le «~1 RETURN
bug~» en référence au script nécessaire pour réaliser cette dépense).
C'est ArtForz qui en découvre l'existence à la fin du mois de juillet
2010. Au lieu d'exploiter cette faille et de s'emparer de la richesse
présente sur le réseau pour la revendre discrètement, il choisit de
prévenir Satoshi et Gavin par courriel. Satoshi s'empresse d'inclure la
correction dans la mise à jour 0.3.6 et recommande à tous les
utilisateurs de mettre à jour leur logiciel\footnote{«~Satoshi
  s'empresse d'inclure la correction dans la mise à jour 0.3.6~»~:
  Satoshi Nakamoto, \emph{*** ALERT *** Upgrade to 0.3.6}, /07/2010
  19:13:06 UTC~:
  \url{https://bitcointalk.org/index.php?topic=626.msg6451\#msg6451}.}.
La vulnérabilité n'est pas exploitée et Bitcoin échappe ainsi au pire.

Le second évènement est le \emph{value overflow incident}. Le 15 août
vers 17 heures, un bloc miné contient une transaction qui crée plus de
184 milliards de bitcoins. Cette création exploite une vulnérabilité de
dépassement de mémoire (\emph{overflow}) dans la représentation des
quantités dans Bitcoin. Une heure plus tard, le problème est repéré par
Jeff Garzik, un ingénieur américain ayant découvert Bitcoin grâce à
Slashdot, qui avertit la communauté sur le forum\footnote{Jeff Garzik,
  \emph{Strange block 74638}, /08/2010, 18:08:49 UTC~:
  \url{https://bitcointalk.org/index.php?topic=822.msg9474\#msg9474}}.

La réaction de Satoshi ne se fait pas attendre. Un peu avant minuit, il
publie un correctif créant une chaîne alternative ne contenant pas la
transaction incriminée\footnote{«~il publie un correctif créant une
  chaîne alternative ne contenant pas la transaction incriminée~»~:
  Satoshi Nakamoto, \emph{Version 0.3.10 - block 74638 overflow PATCH!},
  /08/2010 23:48:22 UTC~:
  \url{https://bitcointalk.org/index.php?topic=827.msg9590\#msg9590}.}.
La situation conflictuelle est résolue lorsque la chaîne correcte
devient plus longue que l'autre le lendemain à 8 heures 10 du
matin\footnote{«~la chaîne correcte devient plus longue que l'autre le
  lendemain à 8 heures 10 du matin~»~: Satoshi Nakamoto, \emph{Re:
  overflow bug SERIOUS}, /08/2010 12:59:38 UTC~:
  \url{https://bitcointalk.org/index.php?topic=823.msg9734\#msg9734}.}.
Cet incident perturbe l'activité du réseau pendant 15 heures environ
mais le problème est vite résolu grâce à une réactivité forte de la
communauté. Suite à cet incident, Satoshi implémente un système d'alerte
dans Bitcoin, lui permettant d'avertir tous les nœuds du réseau en cas
de problème technique\footnote{Satoshi Nakamoto, \emph{Development of
  alert system}, /08/2010 23:55:06 UTC~:
  \url{https://bitcointalk.org/index.php?topic=898.msg10722\#msg10722}.
  -- Après avoir servi entre 2012 et 2015, ce système d'alerte a été
  progressivement désactivé pour finir par être définitivement supprimé
  du logiciel en 2017
  (\url{https://bitcoin.org/en/alert/2016-11-01-alert-retirement}).}.

Au cours de l'automne, la popularisation du minage par processeur
graphique rend le minage par CPU quasi-impossible. C'est ce qui provoque
l'apparition de la première coopérative de minage le 27 novembre,
Bitcoin.cz Mining, une organisation permettant aux petits mineurs de
lisser leurs revenus en regroupant leurs puissances de calcul
respectives\footnote{Marek Palatinus, \emph{Cooperative mining},
  /11/2010, 13:45:41 UTC~:
  \url{https://bitcointalk.org/index.php?topic=1976.msg24844\#msg24844}.}.
Créée par Marek Palatinus (connu sous le pseudonyme de slush), un
architecte informatique tchèque, la coopérative sera par la suite
renommée en Slush Pool en son hommage.

De manière générale, à la fin de l'année 2010, on peut considérer que le
projet Bitcoin a pris son envol~: l'économie s'est fortifiée, notamment
avec les services de change, le minage s'est spécialisé avec
l'apparition du minage par GPU et le protocole a été mis à l'épreuve par
la découverte de failles dans le logiciel. Ces éléments montrent que les
incitations des différents acteurs du système sont alignées. C'est à ce
moment-là que Satoshi décide de disparaître.

\section*{La disparition de Satoshi
Nakamoto}\label{la-disparition-de-satoshi-nakamoto}
\addcontentsline{toc}{section}{La disparition de Satoshi Nakamoto}

\markright{La disparition de Satoshi Nakamoto}

La disparition de Satoshi Nakamoto se fait progressivement à partir de
décembre 2010. Satoshi n'explicite pas les raisons qui le poussent à
s'éclipser, mais nous pouvons les deviner. Tout d'abord, le projet a
pris : il a grossi à tel point qu'il devient difficile de diriger le
mouvement\footnote{«~il a grossi à tel point qu'il devient difficile de
  diriger le mouvement~»~: Pete Rizzo, «~\emph{The Last Days of Satoshi:
  What Happened when Bitcoin's Creator Disappeared}~», \emph{Bitcoin
  Magazine}, 26 avril 2021~:
  \url{https://bitcoinmagazine.com/technical/what-happened-when-bitcoin-creator-satoshi-nakamoto-disappeared}.}.
Mais surtout Satoshi redoute la réaction des agences étatiques, une
préoccupation qu'il exprime dans un message daté du 5 juillet 2010
(commentant le brouillon de la présentation de Bitcoin qui sera proposée
à Slashdot), où il déclare ne pas vouloir mettre trop en avant l'aspect
«~anonyme~» de Bitcoin ou son opposition aux autorités légales qui
constituerait une «~provocation\footnote{«~Nous ne voulons pas mettre
  l'aspect ``anonyme'' au premier plan. (J'avais l'intention de modifier
  la page d'accueil) ``Les développeurs s'attendent à ce que cela se
  traduise par une monnaie stable par rapport à l'énergie et hors de
  portée de tout État.'' -- Je ne fais certainement pas ce genre de
  provocation ou d'affirmation.~» -- Satoshi Nakamoto, \emph{Re:
  Slashdot Submission for 1.0}, /07/2010 21:31:14 UTC~:
  \url{https://bitcointalk.org/index.php?topic=234.msg1976\#msg1976}.}~».

L'élément déclencheur est l'affaire WikiLeaks. WikiLeaks est une
organisation non gouvernementale à but non lucratif fondée par le
cypherpunk Julian Assange en 2006, dont la raison d'être est de donner
une audience aux lanceurs d'alertes et aux fuites d'information, tout en
protégeant leurs sources. À partir de 2010, les documents confidentiels
révélés de l'ONG commencent à être relayés par les grands médias et à
faire du bruit dans l'opinion publique. C'est notamment le cas de
l'\emph{Afghan War Diary}, un ensemble de documents et de rapports
militaires américains secrets sur la guerre en Afghanistan faisant
notamment état de la dissimulation des victimes civiles, qui est publié
le 25 juillet 2010 grâce à la contribution de Bradley Manning, un
analyste militaire de l'armée des États-Unis\footnote{L'histoire de
  Bradley Manning (devenu Chelsea Manning après une transition de genre)
  est narrée par Andy Greenberg dans son ouvrage \emph{This Machine
  Kills Secrets} publié en 2012.}. On peut également citer les
\emph{Iraq War Logs}, documents secrets sur la guerre en Irak entre 2004
et 2009 publiés le 23 octobre et révélant le nombre de victimes civiles
et les actes de torture perpétrés.

Le financement de WikiLeaks repose essentiellement sur les dons du
public. Il s'agit d'une activité sensible pour les firmes réglementées
qui craignent les potentielles représailles des autorités. C'est ainsi
que la société de paiement en ligne Moneybookers gèle le compte de l'ONG
le 14 octobre 2010. À la suite de ces révélations, il est ainsi de plus
en plus probable que WikiLeaks s'expose à davantage de sanctions.

Le 10 novembre, Amir Taaki, un jeune anglais d'origine iranienne ayant
fraîchement découvert Bitcoin, voit dans la situation de WikiLeaks une
opportunité de démontrer l'utilité de la résistance à la censure du
système. Il écrit ainsi sur le forum~:

«~Je voulais envoyer une lettre à Wikileaks à propos de Bitcoin car,
malheureusement, ils ont subi plusieurs incidents où leurs fonds ont été
saisis dans le passé. Quelqu'un sait où leur envoyer un
message\footnote{Amir Taaki, \emph{Wikileaks contact info?}, /11/2010
  12:49:16 UTC~:
  \url{https://bitcointalk.org/index.php?topic=1735.msg21271\#msg21271}.}~?~»

Les réactions sont mitigées. D'après un utilisateur, «~cela peut être
bénéfique pour wikileaks, mais pas nécessairement pour
Bitcoin\footnote{ShadowOfHarbringer, \emph{Re: Wikileaks contact info?},
  /11/2010 13:28:00 UTC~:
  \url{https://bitcointalk.org/index.php?topic=1735.msg21283\#msg21283}.}~».

Un mois plus tard, le 3 décembre, PayPal gèle le compte de
WikiLeaks\footnote{«~PayPal gèle le compte de WikiLeaks~»~: \emph{PayPal
  statement regarding WikiLeaks}, 3 décembre 2010~:
  \url{https://web.archive.org/web/20101206112350/https://www.thepaypalblog.com/2010/12/paypal-statement-regarding-wikileaks/}.}.
Certaines personnes sur le forum suggèrent d'encourager WikiLeaks à
accepter le bitcoin~: cela paraît en effet le «~moment idéal pour
commencer les dons en bitcoins\footnote{Wladimir van der Laan, \emph{Re:
  Wikileaks contact info?}, /12/2010 08:57:41 UTC~:
  \url{https://bitcointalk.org/index.php?topic=1735.msg26737\#msg26737}.}~».
Cela fait réagir Satoshi le lendemain qui s'oppose à cette évolution et
déclare~:

«~Le projet a besoin de grandir progressivement pour que le logiciel
puisse se renforcer en cours de route.

J'appelle WikiLeaks à ne pas commencer à utiliser Bitcoin. Bitcoin est
une petite communauté expérimentale encore naissante. Vous n'obtiendriez
rien de plus que quelques piécettes et l'agitation que vous apporteriez
nous détruirait probablement à ce stade\footnote{Satoshi Nakamoto,
  \emph{Re: Wikileaks contact info?}, /12/2010 09:08:08 UTC,
  \url{https://bitcointalk.org/index.php?topic=1735.msg26999\#msg26999}.}.~»

Dans les jours qui suivent, c'est un véritable blocus financier qui se
met en place contre WikiLeaks, auquel participent Mastercard et Visa,
mais aussi Western Union, Bank of America et d'autres acteurs, ce qui
met en péril la survie financière de l'ONG\footnote{«~met en péril la
  survie financière de l'ONG~»~: Le 24 octobre 2011, un communiqué de
  WikiLeaks (\emph{Banking Blockade}, /10/2011 13:00 UTC,
  \url{https://wikileaks.org/Banking-Blockade.html}) a indiqué que le
  blocus financier a fait disparaître de ses 95~\% des revenus.}. Tout
naturellement certains insistent pour que Bitcoin soit mis à profit.

Le 11 décembre, un article est publié sur PC World pour mettre en avant
la possibilité d'un usage de Bitcoin par WikiLeaks\footnote{«~un article
  est publié sur PC World pour mettre en avant la possibilité d'un usage
  de Bitcoin par WikiLeaks~»~: Keir Thomas, \emph{Could the Wikileaks
  Scandal Lead to New Virtual Currency?}, décembre 2010, 00:30~:
  \url{https://www.pcworld.com/article/499375/could_wikileaks_scandal_lead_to_new_virtual_currency.html}.}.
Cet article est rapidement évoqué sur le forum et la réaction de Satoshi
est sans appel. Il écrit~:

«~Il aurait été bon d'attirer cette attention dans un tout autre
contexte. WikiLeaks a donné un coup de pied dans la fourmilière, et la
colonie se dirige maintenant vers nous\footnote{Satoshi Nakamoto,
  \emph{Re: PC World Article on Bitcoin}, /12/2010 23:39:16 UTC,
  \url{https://bitcointalk.org/index.php?topic=2216.msg29280\#msg29280}.}.~»

C'est son avant-dernier message public. Le lendemain, il poste son
dernier message sur le forum pour annoncer la version 0.3.19 du
logiciel, puis se volatilise. Il transmet les rênes du projet à ses deux
bras droits historiques~: Martti Malmi et Gavin Andresen.

Martti Malmi hérite du site web et du forum. Néanmoins, à l'instar de
Satoshi, il se détourne progressivement de Bitcoin et délègue la gestion
de ces plateformes à d'autres personnes, à qui il cèdera le contrôle
entièrement en 2015\footnote{Le contrôle du site a été cédé à un
  individu utilisant le pseudonyme Cøbra tandis que la charge du forum a
  été donnée à Michael Marquardt (theymos). Les deux personnes co-gèrent
  ces deux plateformes.}. Il vendra ses 55~000 bitcoins pour s'acheter
un appartement près de Helsinki\footnote{«~Il vendra ses 55~000 bitcoins
  pour s'acheter un appartement près de Helsinki~»~: Martti Malmi sur
  Twitter, /12/2020 12:22 UTC~:
  \url{https://twitter.com/marttimalmi/status/1339908783187832834}.}.

De son côté, Gavin Andresen hérite de la clé d'alerte, du dépôt
SourceForge et de la liste de diffusion. Dès le 19 décembre, il annonce
«~commencer à gérer le projet Bitcoin de manière plus
active\footnote{Gavin Andresen, \emph{Development process straw-man},
  /12/2010 16:41:39 UTC~:
  \url{https://bitcointalk.org/index.php?topic=2367.msg31651\#msg31651}.}~»
et crée le dépôt GitHub de Bitcoin, où le projet sera dorénavant
développé. Il ignore alors qu'il est devenu le développeur en chef du
projet et que le créateur de Bitcoin va disparaître.

Satoshi se volatilise définitivement durant le printemps 2011. Le 23
avril, il adresse un dernier courriel à Mike Hearn, l'ingénieur de
Google qui l'avait approché deux ans auparavant et qui était resté en
contact avec lui, dans lequel il écrit~:

«~Je suis passé à autre chose. {[}Bitcoin{]} est entre de bonnes mains
avec Gavin et les autres\footnote{Satoshi Nakamoto, \emph{Re: Holding
  coins in an unspendable state for a rolling time window}, /04/2011
  13:40 UTC~:
  \url{https://plan99.net/~mike/satoshi-emails/thread5.html}.}.~» Il
fait également ses adieux à Gavin et Martti. En particulier, il demande
à Gavin d'éviter de parler de lui comme d'une «~personnalité sombre et
mystérieuse~» à la presse\footnote{Allie Jones, «~\emph{Former Coworker
  Regrets Helping Reveal Identity of Bitcoin's Founder}~», \emph{The
  Wire}, 6 mars 2014, archive~:
  \url{https://web.archive.org/web/20140309041730/http://www.thewire.com/technology/2014/03/bitcoin-founders-coworker-regrets-doxxing-him/358878}.}.
Le 27 avril, Gavin annonce qu'il a été invité par la CIA à faire une
présentation sur Bitcoin\footnote{«~Gavin annonce qu'il a été invité par
  la CIA~»~: Gavin Andresen, \emph{Gavin will visit the CIA}, /04/2011
  19:00:26 UTC~:
  \url{https://bitcointalk.org/index.php?topic=6652.msg97181\#msg97181}.}.
Cette visite se passe le 14 juin\footnote{«~Cette visite se passe le 14
  juin~»~: Gavin Andresen sur Twitter, /06/2011 23:55 UTC~:
  \url{https://twitter.com/gavinandresen/status/80785477342478336}.}. De
manière intéressante, c'est également le jour où WikiLeaks se met
finalement à accepter les dons en bitcoin\footnote{«~WikiLeaks accepte
  désormais les dons anonymes en bitcoin sur ``~» -- WikiLeaks sur
  Twitter, /06/2011 23:12 UTC~:
  \url{https://twitter.com/wikileaks/status/80774521350668288}.}. Ces
deux évènements viennent confirmer ce que Satoshi redoutait.

Satoshi Nakamoto laisse derrière lui une fortune colossale~: 1~122~693
bitcoins selon une estimation de 2020\footnote{Ce montant a été retrouvé
  grâce au Patoshi Pattern, mis en lumière par Sergio Lerner en 2013
  dans un article intitulé \emph{The Well Deserved Fortune of Satoshi
  Nakamoto, Bitcoin creator, Visionary and Genius}
  (\url{https://bitslog.com/2013/04/17/the-well-deserved-fortune-of-satoshi-nakamoto/}).
  L'estimation utilisée ici est celle de Whale Alert publiée en 2020~:
  \url{https://whale-alert.medium.com/the-satoshi-fortune-e49cf73f9a9b}.}.
Cela représente plus de 5 \% de la quantité totale de bitcoins. Ces
fonds ne bougeront jamais.

Quelques messages émaneront de ses différents comptes\footnote{Un
  message provenant du compte de Satoshi sur le forum de la Fondation
  P2P a été publié le 7 mars 2014 pour nier son association à Dorian
  Nakamoto
  (\url{https://p2pfoundation.ning.com/forum/topics/bitcoin-open-source?commentId=2003008:Comment:52186}),
  et un courriel d'opposition à Bitcoin XT a été envoyé le 15 août 2015
  à la liste de diffusion de développement depuis son adresse
  satoshi@vistomail.com
  (\url{https://lists.linuxfoundation.org/pipermail/bitcoin-dev/2015-August/010238.html}).},
mais on supposera qu'ils ont été piratés.

L'identité de Satoshi Nakamoto restera inconnue, celui-ci ayant réussi à
conserver son anonymat grâce à l'usage de Tor et de services respectueux
de la vie privée. Dans les années qui suivront, sa «~personnalité sombre
et mystérieuse~» deviendra un mythe à part entière, suscitant les
spéculations les plus diverses. Tout le monde se demandera «~Qui est
Satoshi Nakamoto~?~» à l'instar des personnages de \emph{La Grève} d'Ayn
Rand s'interrogeant sur l'identité de John Galt. On cherchera à savoir
qui il est, quelques pistes seront privilégiées\footnote{Parmi les
  candidats pour être la figure de Satoshi Nakamoto, ceux qui reviennent
  le plus souvent sont~: Nick Szabo, Hal Finney, Adam Back, Len
  Sassaman.}, mais jamais son identité civile ne sera formellement
identifiée.

En 2013, dans l'un de ses derniers messages sur le forum, Hal Finney
partagera une citation énigmatique du film \emph{Man of Steel} tout
juste sorti, résumant bien la dimension mystérieuse entourant le
créateur de Bitcoin~:

«~Comment retrouver quelqu'un qui a toujours brouillé les pistes~?
{[}...{]} Pour certains, c'était un ange gardien. Pour d'autres, {[}une
énigme,{]} un fantôme, toujours un peu à l'écart. {[}...{]} Que
représente le S~?\footnote{Hal Finney, \emph{Re: Another *Potential*
  Identifying Piece of Evidence on Satoshi}, /06/2013 01:23:42 UTC~:
  \url{https://bitcointalk.org/index.php?topic=234330.msg2479328\#msg2479328}.}~»

En mars 2014, on croira l'avoir trouvé en la personne de Dorian Prentice
Satoshi Nakamoto suite à la publication d'un article de
Newsweek\footnote{Leah McGrath Goodman, «~\emph{The Face Behind
  Bitcoin}~», \emph{Newsweek Magazine}, 6 mars 2014~:
  \url{https://www.newsweek.com/2014/03/14/face-behind-bitcoin-247957.html}.}.
Cet ingénieur des télécommunications, citoyen américain naturalisé
d'origine japonaise, vivant avec sa mère à Temple City dans la banlieue
de Los Angeles, se fera harceler par la presse mais niera en bloc. On
découvrira cependant que la famille de Hal Finney a habité dans la même
municipalité, «~à quelques pâtés de maisons de la maison familiale des
Nakamoto~», durant l'adolescence de Hal, ce qui attirera quelques
soupçons sur ce dernier\footnote{Andy Greenberg, «~\emph{Nakamoto's
  Neighbor: My Hunt For Bitcoin's Creator Led To A Paralyzed Crypto
  Genius}~», \emph{Forbes}, 25 mars 2014~:
  \url{https://www.forbes.com/sites/andygreenberg/2014/03/25/satoshi-nakamotos-neighbor-the-bitcoin-ghostwriter-who-wasnt/}.}.

Hal Finney décèdera en août 2014 des suites de la maladie de Charcot. En
tant que futuriste averti, il se fera cryogéniser par la fondation
Alcor.

\section*{La communauté prend le
relai}\label{la-communautuxe9-prend-le-relai}
\addcontentsline{toc}{section}{La communauté prend le relai}

\markright{La communauté prend le relai}

Alors que Satoshi se met progressivement en retrait, la popularité de
Bitcoin augmente prodigieusement. En particulier, le prix du bitcoin
évolue de manière favorable~: alors qu'il n'était que de 20~centimes en
décembre 2010, il atteint la parité avec le dollar le 9 février 2011 et
s'y maintient pendant quelques temps. Cette hausse du prix attise
l'enthousiasme de la communauté, et notamment celui de Hal Finney qui
déclare avoir «~vraiment de la chance d'être au début d'un nouveau
phénomène potentiellement explosif\footnote{Hal Finney, \emph{Re: Parity
  Party}, /01/2011 21:17:04 UTC~:
  \url{https://bitcointalk.org/index.php?topic=2734.msg37307\#msg37307}.}~».

Cette période coïncide avec l'apparition de Silk Road, une place de
marché du dark web s'appuyant sur Tor et Bitcoin qui permet à ses
utilisateurs d'échanger librement des produits et des services légaux et
illégaux. Celle-ci est lancée à la fin du mois de janvier par un jeune
Texan du nom de Ross Ulbricht, qui en fait mention sur le forum de
Bitcoin en feignant d'avoir découvert le site par hasard\footnote{La
  première mention publique de Silk Road par Ross Ulbricht remonte au 27
  janvier 2011 sur le forum de \emph{The Shroomery}, un site consacré
  aux champignons hallucinogènes, où il prétendait être tombé par hasard
  sur la place de marché. -- Ross Ulbricht, \emph{anonymous market
  online?}, /01/2011 22:28 UTC~:
  \url{https://www.shroomery.org/forums/showflat.php/Number/13860995}.}.

Ross Ulbricht adhère profondément aux principes du libertarianisme, une
philosophie libérale originaire des États-Unis prônant le respect
impératif de la liberté individuelle, des droits de propriété et du
marché. Silk Road est pour lui une incarnation de cet idéal. De ce fait,
la gamme des produits et services qui peuvent être listés sur le site
est restreinte et nécessite qu'aucun mal n'ait été fait à autrui~: on y
retrouve ainsi de la drogue, des médicaments, des pièces de métaux
précieux, mais en aucun cas des cartes bancaires volées, de la
pédopornographie ou des services de tueur à gages\footnote{«~en aucun
  cas des cartes bancaires volées, de la pédopornographie ou des
  services de tueur à gages~»~: Capture du \emph{Seller's Guide} du
  18/9/2012 (GX-120),
  \url{https://antilop.cc/sr/exhibits/GX-120_Redacted.pdf}~: «~Ne pas
  mettre en vente les objets dont le but est de nuire ou de frauder,
  comme les objets ou les informations volés, les cartes de crédit
  volées, la fausse monnaie, les informations personnelles, les
  assassinats et les armes de toutes sortes. Ne pas mettre en vente les
  objets liés à la pédophilie.~»}. Dans l'ensemble, le site sert
principalement à la vente de drogue illicite (dont surtout de petites
quantités de cannabis), chose pour laquelle il deviendra célèbre.

La promotion de Bitcoin s'intensifie également. Le 22 mars, la première
vidéo expliquant Bitcoin de manière qualitative est publiée\footnote{WeUseCoins,
  \emph{What is Bitcoin?} (vidéo), 22 mars 2011~:
  \url{https://www.youtube.com/watch?v=Um63OQz3bjo}.}. Cette vidéo,
intitulée sobrement «~\emph{What is Bitcoin?}~», est produite par Stefan
Thomas grâce à un financement participatif de la communauté. Elle aura
un succès retentissant au fil des années en totalisant plusieurs
millions de vues sur YouTube. Les vidéos de ce type se multiplieront.

Bitcoin est notamment vanté dans les cercles libertariens, où son
caractère libre, anonyme et hors de portée de l'État est mis en avant.
Le 16 mars 2011, une partie de l'épisode de FreeTalkLive du jour est
consacrée à Bitcoin et à Silk Road\footnote{«~une partie de l'épisode de
  FreeTalkLive du jour est consacrée à Bitcoin et à Silk Road~»~:
  \url{https://soundcloud.com/freetalklive/ftl2011-03-16}.}. Cela attire
l'attention de l'entrepreneur et activiste Roger Ver, déjà millionnaire
grâce à sa société de revente de composants informatiques, Memory
Dealers. Il découvre Bitcoin à la fin du mois d'avril en écoutant une
rediffusion de l'épisode et est instantanément conquis~: il se met à
lire tout ce qu'il peut sur le sujet, achète du bitcoin et commence à
l'accepter avec son entreprise\footnote{«~commence à l'accepter avec son
  entreprise~»~: Roger Ver, \emph{Re: Earn 131BTC or 12-13BTC for
  getting shops/organisations to accept Bitcoin!}, /04/2011 08:00:52
  UTC~:
  \url{https://bitcointalk.org/index.php?topic=4667.msg95746\#msg95746}.}.
Il deviendra rapidement l'un des promoteurs les plus zélés de Bitcoin,
ce qui lui vaudra le surnom de \emph{Bitcoin Jesus} pendant un temps.

L'existence de Silk Road est révélée au grand public le 1 juin 2011 avec
un article d'Adrien Chen sur Gawker\footnote{Adrian Chen, «~\emph{The
  Underground Website Where You Can Buy Any Drug Imaginable}~»,
  \emph{Gawker}, 1 juin 2011~:
  \url{https://www.gawker.com/the-underground-website-where-you-can-buy-any-drug-imag-30818160}.},
ce qui a pour effet d'attirer l'attention sur Bitcoin encore un peu
plus, notamment en incitant les consommateurs à se procurer du bitcoin
pour acheter des produits sur la plateforme.

Au cours du printemps 2011, on assiste par conséquent à une forte
poussée du prix, due à l'augmentation de la demande. Après avoir stagné
pendant quelques mois, celui-ci passe ainsi de 1~\$ le 15 avril à plus
de 32~\$ le 8 juin.

Mt. Gox, la principale plateforme de change de l'époque, se retrouve
sous pression. Celle-ci a alors été reprise depuis quelques mois par
Mark Karpelès, un développeur français de 26 ans vivant au Japon, qui
est quelque peu négligent et n'a pas su résoudre les problèmes
d'implémentation de son prédécesseur. C'est ainsi qu'un incident
malencontreux survient le dimanche 19 juin~: un groupe de pirates accède
au compte administrateur de Jed McCaleb et tente d'en extraite un
maximum d'argent.

La limite de retrait journalière étant fixée à 1~000~\$, les pirates
cherchent à faire baisser le prix afin de retirer le plus de bitcoins
possibles. Ils vendent les bitcoins de Jed McCaleb au marché ce qui
provoque un krach éclair sur le cours~: le prix, qui stationne ce
jour-là autour des 17~\$, chute à 0,01~\$ en quelques minutes. C'est la
panique dans la communauté, et beaucoup d'utilisateurs de Mt. Gox
vendent sous le coup de l'émotion afin de conserver ce qui leur reste.
La situation est rétablie dans la journée mais 2~000 bitcoins manquent à
l'appel. Le 23 juin, Mark Karpelès prouve la solvabilité de l'entreprise
en déplaçant 424~242 bitcoins d'une adresse à une autre\footnote{L'identifiant
  de la transaction de preuve de solvabilité de Mt. Gox en 2011 était
  ``.}.

Cet incident entraîne la fin de la folie spéculative sur le bitcoin et
le prix se met à descendre doucement. C'est ce moment-là qu'on assiste à
l'escroquerie de sortie de MyBitcoin~: le 29 juillet, son fondateur
anonyme, Tom Williams, disparaît avec les 154~406 bitcoins présents sur
les comptes de ses clients\footnote{«~son fondateur anonyme, Tom
  Williams, disparaît avec les 154~406 bitcoins présents sur les comptes
  de ses clients~»~: shotgun, \emph{mybitcoin down or just me?},
  /07/2011 22:41:36 UTC~:
  \url{https://bitcointalk.org/index.php?topic=32900.msg411251\#msg411251}.}.
Suite à cet évènement, le prix baissera en flèche jusqu'à atteindre un
creux local de 2~\$ en novembre.

Mais cela ne décourage pas pour autant les membres de la communauté. Du
19 au 21 août 2011 a lieu la première conférence sur Bitcoin à New York,
qui est organisée par Bruce Wagner, l'animateur du \emph{Bitcoin Show},
une émission d'entretiens filmés avec les acteurs de
l'écosystème\footnote{La chaîne Youtube de Bruce Wagner se trouve à
  l'adresse \url{https://www.youtube.com/@vlogwrap}. Les vidéos des
  présentations à la conférence peuvent y être retrouvées.}. La
conférence revêt un caractère amateur (qui caractérise la communauté
d'alors) et seules quatre présentations ont lieu~: celle de Bruce Wagner
ainsi que les interventions de Gavin Andresen, Jeff Garzik et Stefan
Thomas. Cela permet néanmoins aux membres les plus actifs, tels que
Roger Ver, Jesse Powell, Jed McCaleb, Mark Karpelès ou Charlie Lee, de
se réunir en personne pour la première fois.

Le développement logiciel s'organise aussi. Jusqu'ici, il était
centralisé dans les mains de Satoshi, le «~dictateur bienveillant~» du
projet. Mais après le départ du créateur de Bitcoin, il s'ouvre à la
participation de la communauté, sous la supervision de Gavin Andresen.
On voit ainsi des contributeurs talentueux commencer à s'impliquer dans
l'évolution de Bitcoin comme Nils Schneider, Matt Corallo, Pieter
Wuille, Jeff Garzik, Wladimir van der Laan, Luke-Jr ou encore Gregory
Maxwell. Des méthodes de coordination sont rapidement mises en place
comme la liste de diffusion `` permettant de discuter formellement des
changements à apporter\footnote{Jeff Garzik,
  \emph{{[}Bitcoin-development{]} Preparing 0.3.23-rc1 release},
  /06/2011 02:23:58 UTC~:
  \url{https://lists.linuxfoundation.org/pipermail/bitcoin-dev/2011-June/000000.html}.},
et le système des propositions d'amélioration de Bitcoin (\emph{Bitcoin
Improvement Proposals} ou BIP), qui décrivent publiquement ces
changements\footnote{Le système des BIP a été initialement proposé le 19
  septembre 2011 par Amir Taaki sous le nom de \emph{Bitcoin Enhancement
  Proposals}, en référence directe aux \emph{Python Enhancement
  Proposals} (PEP) dont il s'est inspiré. (Amir Taaki,
  \emph{{[}Bitcoin-development{]} Bitcoin Enhancement Proposals (BEPS)},
  /09/2011 00:31:55 UTC,
  \url{https://lists.linuxfoundation.org/pipermail/bitcoin-dev/2011-September/000554.html})}.

L'utilisation de Bitcoin devient plus facile. On assiste à l'apparition
de portefeuilles légers permettant d'utiliser Bitcoin sans avoir à
télécharger et vérifier l'intégralité de la chaîne. Ces derniers
utilisent la vérification de paiement simplifiée décrite par Satoshi
Nakamoto dans la section 8 du livre blanc. Celle-ci est mise en œuvre
par Mike Hearn au sein de sa bibliothèque logicielle \emph{bitcoinj}
programmée en Java, qui permet entre autres une meilleure compatibilité
avec les applications sur les téléphones multifonctions fonctionnant
sous Android. Le premier portefeuille pour mobile, le \emph{Bitcoin
Wallet for Android}, est lancé par Andreas Schildbach en mars
2011\footnote{«~Le premier portefeuille pour mobile {[}...{]} est lancé
  par Andreas Schildbach en mars 2011~»~: Andreas Schildbach,
  \emph{Bitcoin Wallet for Android}, /03/2011 21:25:51 UTC~:
  \url{https://bitcointalk.org/index.php?topic=4384.msg64142\#msg64142}.}.
Celui-ci montre que l'usage direct de Bitcoin dans la vie de tous les
jours est possible. Du côté ordinateur, Thomas Voegtlin crée Electrum en
novembre 2011, présenté comme un portefeuille qui permet à l'utilisateur
de récupérer ses fonds par le biais d'une phrase
mnémotechnique\footnote{«~Thomas Voegtlin crée Electrum en novembre
  2011~»~: Thomas Voegtlin, \emph{{[}Electrum{]} a brainwallet in twelve
  words}, /11/2011 01:06:59 UTC~:
  \url{https://bitcointalk.org/index.php?topic=51397.msg612674\#msg612674}.}.
Cette pratique sera plus tard standardisée et adoptée largement dans
l'écosystème.

Ce développement décentralisé est également source de tensions. Sans son
fondateur, le projet ne dispose plus d'un meneur incontestable~: certes
Gavin Andresen possède le contrôle du dépôt, mais n'a pas l'autorité
technique suffisante pour imposer toutes ses vues aux autres
développeurs. Les décisions sont prises relativement collectivement, ce
qui pose la question de la gouvernance de Bitcoin~: qui décide
d'apporter un changement au protocole~?

À la fin de l'année 2011 et au début de l'année 2012, le premier débat
technique en l'absence de Satoshi a lieu. Le groupe de développeurs est
alors encore très restreint mais cela suffit pour créer un conflit à
propos de l'amélioration de la programmabilité des transactions, qui
permettrait notamment de créer des comptes multisignatures. On s'en
souviendra comme la «~bataille pour P2SH\footnote{Pete Rizzo, Aaron van
  Wirdum, «~\emph{The Battle For P2SH: The Untold Story Of The First
  Bitcoin War}~», \emph{Bitcoin Magazine}, 4 décembre 2020~:
  \url{https://bitcoinmagazine.com/technical/the-battle-for-p2sh-the-untold-story-of-the-first-bitcoin-war}.}~».

De par sa nature informatique, Bitcoin constitue un système de monnaie
programmable qui permet à l'utilisateur d'imposer des conditions au
blocage et au déblocage des fonds. Il dispose pour cela d'un mécanisme
de scripts reposant sur des instructions logiques appelées codes
opérations. Cependant, ces scripts sont compliqués à gérer. Il s'agit
donc de trouver un moyen simple pour l'utilisateur d'envoyer des fonds
vers un script défini préalablement par le récipiendaire. C'est l'idée
derrière la proposition faite par Nicolas van Saberhagen d'ajouter un
nouveau code opération appelé
\texttt{.\ Cette\ proposition\ souffre\ néanmoins\ d\textquotesingle{}un\ problème\ de\ récursivité,\ ce\ qui\ provoque\ rapidement\ l\textquotesingle{}apparition\ de\ deux\ propositions\ concurrentes~:\ *Pay\ to\ Script\ Hash*\ (P2SH)\ proposé\ par\ Gavin\ Andresen\ et}
(CHV) proposé par Luke-Jr.

Une tension émerge entre les deux propositions, ce qui crée le débat.
Amir Taaki, qui ne soutient ni l'une ni l'autre, appelle à la discussion
et déclare le 29 janvier 2012~:

«~Ma crainte c'est qu'un jour Bitcoin soit corrompu. Développeurs :
considérez cet examen supplémentaire comme une opportunité de construire
une culture d'ouverture\footnote{Amir Taaki, \emph{The Truth behind BIP
  16 and 17 (important read)}, /01/2012 03:54:08 UTC~:
  \url{https://bitcointalk.org/index.php?topic=61705.msg719790\#msg719790}.}.~»

Finalement, c'est P2SH qui est choisi pour être intégré à Bitcoin sur
l'ordre de Gavin Andresen. Cette intégration sera réalisée, non sans
difficulté, le 1 avril 2012.

Dans un même temps, la popularisation de Bitcoin se poursuit. Le 28
février, un russo-canadien du nom de Vitalik Buterin, âgé de seulement
18 ans, co-fonde le \emph{Bitcoin Magazine} avec Mihai Alisie, un
développeur roumain. Ce média, d'abord uniquement disponible en version
web, est distribué en édition papier à partir de mai. Le jeune Vitalik y
écrit de nombreux articles documentant l'actualité de l'époque. Par la
suite, de nombreux sites d'information spécialisés verront le jour comme
CoinDesk ou CoinTelegraph.

Le 24 avril 2012, un jeu de hasard en ligne nommé SatoshiDICE est lancé
par l'entrepreneur américain Erik Voorhees\footnote{Erik Voorhees,
  \emph{SatoshiDICE.com - The World's Most Popular Bitcoin Game},
  /04/2012 02:17:31 UTC~:
  \url{https://bitcointalk.org/index.php?topic=77870.msg865877\#msg865877}.}.
Le site repose sur un fonctionnement très simple~: le joueur envoie des
bitcoins à une adresse spécifique et il a une probabilité prédéfinie de
recevoir une récompense qui correspond à un multiple du montant envoyé
(il a par exemple une chance sur deux de recevoir un peu moins de deux
fois sa mise). Le procédé est instantané et aisément vérifiable, ce qui
attire de nombreux parieurs.

En tant que libertarien convaincu vivant dans le New Hampshire, Erik
Voorhees voit en SatoshiDICE une manière d'échapper à la réglementation.
Le 20 août, il réalise même une IPO pour son entreprise sur la
plateforme roumaine MPEx\footnote{«~il réalise même une IPO pour son
  entreprise sur la plateforme roumaine MPEx~»~: Erik Voorhees,
  \emph{S.DICE - SatoshiDICE 100\% Dividend-Paying Asset on MPEx},
  /08/2012 04:14:43 UTC~:
  \url{https://web.archive.org/web/20121024050433/https://bitcointalk.org/index.php?topic=101902.0}.}.
Il revendra la plateforme le 17 juillet 2013 pour 126~315 bitcoins, soit
12,4 millions de dollars au moment de l'acquisition\footnote{«~Il
  revendra la plateforme le 17 juillet 2013 pour 126~315 bitcoins, soit
  12,4 millions de dollars au moment de l'acquisition~»~: Erik Voorhees,
  «~\emph{SatoshiDice Sold for \$12.4 Million}~», \emph{Bitcoin
  Magazine}, 28 juillet 2012~:
  \url{https://bitcoinmagazine.com/markets/satoshidice-sold-12-4-million}.}.

Le succès de SatoshiDICE provoque une augmentation significative du
nombre de transactions sur la chaîne, qui triple en quelques mois. Cette
activité provenant du site est remarquée et dérange certains
développeurs qui la qualifient de «~spam\footnote{Matt Corallo,
  \emph{Huge increase in satoshidice spam over the past day}, /06/2012
  23:21:47 UTC~:
  \url{https://bitcointalk.org/index.php?topic=87444.msg961132\#msg961132}.}~».
À la moitié de l'année 2012, Bitcoin est ainsi complètement lancé et
prêt à être découvert par un public plus large.

\section*{L'amorçage organique de
Bitcoin}\label{lamoruxe7age-organique-de-bitcoin}
\addcontentsline{toc}{section}{L'amorçage organique de Bitcoin}

\markright{L'amorçage organique de Bitcoin}

Les premières années de Bitcoin ont été déterminantes pour son succès.
Il a en effet pu grandir dans la discrétion et connaître une croissance
organique et prudente, à l'abri de l'opportunisme et de la propagande de
notre monde.

Bitcoin a été proposé en 2008 par Satoshi Nakamoto, qui l'a mis en œuvre
en janvier 2009. Les débuts ont été difficiles, à tel point qu'il a
fallu attendre neuf mois avant que le bitcoin n'acquière un prix~!
Satoshi s'est dévoué pleinement à son œuvre sans jamais profiter
personnellement de sa fortune accumulée. En disparaissant en 2011, il a
finalement laissé la communauté s'approprier le projet.

Bitcoin a été façonné dans un creuset mêlant cypherpunks, anarchistes,
libertariens et autres amoureux de la liberté. Il s'est construit en
opposition au système étatico-bancaire traditionnel, où règnent la
censure et les renflouements publics. C'est pourquoi le message derrière
Bitcoin est si radical et que tant de gens se sont pris de passion pour
lui.

Entre 2010 et 2012, les premiers cas d'utilisation de Bitcoin ont
émergé. Financement de projets politiquement sensibles, jeu d'argent en
ligne, achat de drogues à distance, envois de fonds à l'étranger~: il
s'agissait d'usages à la limite de légalité, voire complètement
illégaux, qui démontraient toute l'efficacité du bitcoin en tant que
monnaie incensurable et relativement anonyme. Cependant, cette tendance
a été rapidement tempérée comme on a pu le constater durant les années
qui ont suivi.

\bookmarksetup{startatroot}

\chapter{Une croissance conflictuelle}\label{ch:clivages}

\phantomsection\label{enotezch:2}{}

{A}\textsc{p}rès un début unifié autour de la figure de Satoshi Nakamoto
entre 2009 et 2011, la communauté de Bitcoin s'est rapidement organisée
sans sa médiation, de manière décentralisée. Gavin Andresen avait bien
été nommé responsable du projet, mais il n'avait pas l'autorité morale
suffisante pour imposer une vision claire de Bitcoin aux autres et
préférait la conciliation. De ce fait, la communauté s'est retrouvée en
proie à de multiples conflits internes, qui ont progressivement gagné en
intensité avec l'afflux des nouveaux arrivants lors des différentes
vagues spéculatives. La querelle entre les développeurs à propos de
\emph{Pay to Script Hash} début 2012 n'était ainsi que la préfiguration
de divisions bien plus profondes.

Quatre évolutions majeures ont affecté l'écosystème de Bitcoin au cours
de son histoire et ont mené à la création de clivages majeurs au sein de
la communauté. Ces évolutions ont été~: la financiarisation de
l'économie, caractérisée par le développement des intermédiaires de
confiance~; l'atteinte de la limite de capacité transactionnelle de la
chaîne de blocs, ayant mis en évidence le manque de scalabilité du
système (et donné lieu à la célèbre «~guerre des blocs~»)~; l'essor des
cryptomonnaies alternatives, accueilli de façons très diverses par les
utilisateurs de Bitcoin~; et l'intégration institutionnelle réalisée par
les instances étatiques, posant la question du rapport à entretenir avec
l'autorité.

Bitcoin a ainsi connu une croissance conflictuelle qui a forgé ce qu'il
est devenu et la perception que nous en avons aujourd'hui. C'est
pourquoi nous nous concentrerons sur ces quatre clivages dans ce
chapitre.

\section*{La financiarisation}\label{la-financiarisation}
\addcontentsline{toc}{section}{La financiarisation}

\markright{La financiarisation}

La financiarisation de Bitcoin se caractérise par une
professionnalisation de l'activité d'échange entre le bitcoin et les
monnaies étatiques, ce qu'on appelle formellement le change, et
l'arrivée des acteurs traditionnels dans l'écosystème. Elle s'accompagne
d'une croissance du prix sans précédent, d'une plus grande liquidité du
marché, mais aussi d'un resserrement des contraintes réglementaires et
d'une mutation du discours dominant au sein de la communauté.

Le besoin de disposer de services de change se fait ressentir très
rapidement. En effet, de manière générale les gens possèdent, gagnent et
dépensent de la monnaie fiat comme du dollar ou de l'euro, et non du
bitcoin. Ainsi, même si Bitcoin est un système théoriquement indépendant
du système traditionnel, il est essentiel qu'il existe des passerelles
entre les deux univers, au moins de manière temporaire.

À partir de l'année 2011, on assiste de ce fait à un essor sans
précédent des places de marché, des bourses en ligne traitant de manière
automatisée les ordres d'achat et de vente des clients. C'est en
particulier le cas de la plateforme Mt.~Gox qui, malgré des débuts
houleux, devient rapidement une véritable plaque tournante de la
conversion entre bitcoins et dollars, recueillant un volume journalier
d'au moins 200~000~\$ et dépassant parfois le million de dollars.
D'autres plateformes émergent comme Bitstamp, Bitcoin-Central, TradeHill
ou BTC-e, mais elles ne parviennent pas à concurrencer Mt.~Gox qui
continuera de représenter 90~\% du volume total échangé sur le marché
durant le reste de son existence.

Outre les plateformes où la négociation se fait «~au comptant~» (les
actifs réels sont échangés), on voit aussi apparaître des plateformes de
trading sur marge qui permettent de négocier des contrats et ainsi de
recourir à l'effet de levier (\emph{leverage}) et de faire de la vente à
découvert (\emph{short selling}). La première d'entre elles est
Bitcoinica, qui connaît une existence tumultueuse entre septembre 2011
et mai 2012\footnote{«~Bitcoinica, qui connaît une existence tumultueuse
  entre septembre 2011 et mai 2012~»~: Ludovic Lars, \emph{L'affaire
  Bitcoinica~: le succès et la chute de la plateforme de trading}, 17
  octobre 2020,
  \url{https://journalducoin.com/analyses/affaire-bitcoinica-succes-chute-plateforme-trading-bitcoin/}.},
avant d'être remplacée par la plateforme Bitfinex, qui prend la relève
en octobre 2012.

En parallèle se développe un service nommé BitInstant aux États-Unis,
cofondé en juin 2011 par Gareth Nelson et Charlie Shrem, dont le rôle
est de faciliter les transferts vers et depuis les plateformes de
change. L'entreprise sert d'intermédiaire entre les clients et les
plateformes et permet de rendre les dépôts (et les retraits) instantanés
moyennant une commission. Charlie Shrem, jeune New-Yorkais d'origine
juive syrienne, assure le rôle de PDG et devient rapidement la figure
principale de l'entreprise, bien que d'autres personnes soient
impliquées dans le projet comme Roger Ver et Erik Voorhees. Dès le début
de l'année 2012, BitInstant propose diverses méthodes de transfert
d'argent (Liberty Reserve, Dwolla, Paxum, dépôts d'espèces, ) pour
interagir avec les principales plateformes de l'écosystème, dont
notamment Mt.~Gox qui est basée au Japon. En avril 2013, l'activité de
BitInstant finira par représenter environ 30~\% du volume total échangé
sur les plateformes de change\footnote{Colleen Taylor, «~\emph{With
  \$1.5M Led By Winklevoss Capital, BitInstant Aims To Be The Go-To Site
  To Buy And Sell Bitcoins}~», \emph{TechCrunch}, 17 mai 2013~:
  \url{https://techcrunch.com/2013/05/17/with-1-5m-led-by-winklevoss-capital-bitinstant-aims-to-be-the-go-to-site-to-buy-and-sell-bitcoins/}.}.

Mais les places de marché ne sont pas les seules à fleurir.
Premièrement, on constate un développement des applications
dépositaires, qui permettent d'envoyer et de recevoir facilement des
bitcoins sans devoir en gérer la détention soi-même, dont MyBitcoin
était le précurseur entre 2010 et 2011. C'est le cas de Coinbase, fondé
en mai 2012 par Brian Armstrong et Fred Ehrsam, qui se développe
initialement comme un «~portefeuille Bitcoin hébergé\footnote{Capture du
  site web Coinbase.com, 20 septembre 2012~:
  \url{https://web.archive.org/web/20120920091115/https://coinbase.com/}.}~».
Coinbase intégrera progressivement les fonctionnalités d'une plateforme
de change classique au fil des années.

Deuxièmement, on voit apparaître des processeurs de paiements qui
donnent aux commerçants la possibilité de recevoir des bitcoins et de
les revendre instantanément pour échapper à la volatilité. L'exemple par
excellence de ce type de service est BitPay, un processeur de paiement
fondé en mai 2011 par Tony Gallippi et Stephen Pair qui deviendra
rapidement la solution de facilité pour de nombreux commerçants.

Troisièmement, les services de change de particulier à particulier se
multiplient également. Ceux-ci permettent à deux individus d'échanger du
bitcoin via divers moyens de paiement, dont notamment l'échange en
personne contre des espèces. La plus connue est la plateforme
LocalBitcoins, qui est fondée en juin 2012 par Jeremias Kangas et qui
inspirera les autres plateformes du même type. Dans le même esprit, il
existe également les marchés de gré à gré (\emph{over the counter}) par
lesquels les plus fortunés peuvent procéder à des échanges importants
entre eux, en privé, sans affecter instantanément le cours sur les
places de marché.

Ainsi, l'offre de services financiers se développe considérablement
entre 2012 et 2013. Cela s'explique par une forte demande de la part des
clients de plus en plus désireux de se procurer du bitcoin. Cette
demande s'illustre par l'apparition de la cryptomonnaie dans la culture
populaire, réellement inaugurée par l'épisode de \emph{The Good Wife}
diffusé le 15 janvier 2012 aux États-Unis qui est consacré entièrement à
Bitcoin\footnote{\emph{The Good Wife}, 3x13~: «~Bitcoin for Dummies~»,
  15 janvier 2012.}.

En particulier, l'intérêt des acteurs du monde financier traditionnel
fait la différence. On assiste en effet à la venue d'investisseurs très
fortunés qui s'intéressent au bitcoin, en raison de son offre limitée
(la fameuse limite des 21 millions) et par son potentiel
technologiquement disruptif. Ils placent leur argent non seulement dans
le bitcoin, mais aussi dans les entreprises de l'écosystème.

C'est d'abord le cas de Barry Silbert, un afficionado de Wall Street
ayant fait fortune grâce à SecondMarket, une société facilitant la
négociation d'actifs sur le marché secondaire. Il s'intéresse au bitcoin
en 2012 et en achète pour des centaines de milliers de dollars. Il sera
à l'origine de la création de Grayscale Investments en 2013 et du
Digital Currency Group en 2015.

C'est aussi le cas des frères Tyler et Cameron Winklevoss, qui sont
connus pour leur différend avec Mark Zuckerberg concernant la création
de Facebook et pour avoir été dédommagés de 65~millions de dollars dans
cette affaire. Les jumeaux apprennent l'existence de Bitcoin en août
2012 par le biais de David Azar, un associé de Charlie Shrem. Puis ils
rencontrent ce dernier, qui les convainc d'investir dans le bitcoin. Ils
finissent également par investir dans sa société BitInstant en mai 2013.
Ils seront plus tard à l'origine de la plateforme de change Gemini.

On peut enfin citer l'entrepreneur et philanthrope argentin Wences
Casares, qui achète du bitcoin en février 2013. Il fondera sa propre
société dans le milieu, Xapo, qui est aujourd'hui l'un des plus
importants dépositaires de bitcoin au monde pour les particuliers.

Cette financiarisation apporte ainsi un afflux considérable d'argent,
mais elle s'accompagne aussi d'un changement de discours. Grâce à sa
politique monétaire fixe, le bitcoin est désormais de plus en plus perçu
comme un investissement, comme un actif apportant un profit dû à la
croissance de son économie. De ce fait, on le voit de moins en moins
comme une monnaie permettant d'échanger de la valeur entre particuliers
sans l'intervention des banques ou des États.

Contrairement aux cypherpunks et aux libertariens, les nouveaux
investisseurs ne sont en effet pas vraiment des anarchistes, appartenant
généralement au monde financier traditionnel très à cheval sur la
réglementation. Pour eux, il est nécessaire que les usages les plus
controversés disparaissent afin que Bitcoin se développe et s'étende au
grand public et aux investisseurs institutionnels. Ils voient en
particulier d'un mauvais œil la place de marché Silk Road, qui
représente alors 10 à 20~\% de l'activité économique sur la chaîne de
blocs\footnote{«~la place de marché Silk Road, qui représente alors 10 à
  20~\% de l'activité économique sur la chaîne de blocs~»~:
  \emph{Chainalysis in Action: US Government Agencies Seize More Than
  \$1 Billion in Cryptocurrency Connected to Infamous Darknet Market
  Silk Road}, 5 novembre 2020~:
  \url{https://blog.chainalysis.com/reports/silk-road-doj-seizure-november-2020/}.}
et qui donne à Bitcoin sa réputation de monnaie de la drogue sur
Internet\footnote{Cette animosité à l'égard de Silk Road s'est retrouvée
  dans le commentaire de Tyler Winklevoss quelques semaines après la
  chute de la plateforme~: «~Les prix sont le double de ce qu'ils
  étaient avant la fermeture de Silk Road. La demande d'utilisation de
  bitcoins pour des activités illicites était donc clairement quasi
  nulle.~» -- Matthew J. Belvedere, «~\emph{Bitcoin is nearly halfway to
  the \$400 billion value predicted by the Winklevoss twins four years
  ago}~», \emph{CNBC}, 12 novembre 2013~:
  \url{https://www.cnbc.com/2013/11/12/the-winklevoss-brothers-bitcoin-worth-100-times-more.html}.}.
La tendance est donc à l'amélioration de l'image de la cryptomonnaie,
une stratégie par ailleurs initiée en 2010 -- 2011 par Satoshi Nakamoto
lui-même, comme nous avons pu le constater dans le
chapitre~\hyperref[ch:mythe]{1}.

C'est dans cette optique qu'est créée la Fondation Bitcoin en septembre
2012\footnote{«~est créée la Fondation Bitcoin en septembre 2012~»~:
  Gavin Andresen, \emph{{[}ANN{]} Bitcoin Foundation}, /09/2012 10:18:51
  UTC~:
  \url{https://bitcointalk.org/index.php?topic=113400.msg1224721\#msg1224721}.}.
Conformément au modèle de la Fondation Linux, il s'agit d'un consortium
d'entreprises de l'écosystème dont le rôle est de financer
l'infrastructure logicielle du protocole, de faire du lobbying auprès du
régulateur et d'améliorer l'image publique de Bitcoin\footnote{Bitcoin
  Foundation, \emph{Developing a More Open Economy}~:
  \url{https://web.archive.org/web/20130702232207/https://bitcoinfoundation.org/about/}.}.
Elle est gérée par des acteurs importants dans l'écosystème~: Peter
Vessenes, le PDG de CoinLab, Gavin Andresen, le mainteneur principal du
logiciel de Bitcoin, Mark Karpelès, le PDG de Mt.~Gox, Jon Matonis,
cryptographe et économiste, Patrick Murck, un juriste spécialisé dans
les monnaies virtuelles, et Charlie Shrem, le PDG de BitInstant.

Le 28 novembre 2012, le premier \emph{halving} se produit~: la création
monétaire du protocole est réduite de moitié et passe de 50 bitcoins à
25 bitcoins par bloc, ce qui abaisse le taux annualisé d'émission à
12,5~\%. La transition se passe parfaitement bien. Quelques jours plus
tard, un dénommé Matt Whitlock construit un graphique simple pour
visualiser l'évolution de la création de bitcoins dans le
temps\footnote{Matt Whitlock, \emph{{[}CHART{]} Bitcoin Inflation
  vs.~Time}, /12/2012 15:08:08 UTC~:
  \url{https://bitcointalk.org/index.php?topic=130619.msg1397456\#msg1397456}.}
(voir figure~\hyperref[fig:bitcoin-inflation]{2.1}). Cela confirme le
changement de focalisation qui passe de son caractère incensurable à sa
rareté.

\begin{figure}

{\centering \includegraphics{chapters/img/matt-whitlock-bitcoin-inflation-log.png}

}

\caption{Graphique de Matt Whitlock comparant le taux d'émission et la
base monétaire du bitcoin en fonction du temps (décembre 2012).}

\end{figure}%

L'augmentation de la demande et la diminution du taux de croissance de
l'offre entraînent une hausse sensible du prix du bitcoin. Alors qu'il
avoisinait les 5~\$ durant la première partie de 2012, il passe à 13~\$
en août et se stabilise à ce niveau jusqu'à la fin de l'année. En 2013,
sa progression devient parabolique~: il dépasse les 20~\$ en janvier~;
il surmonte l'ancien plus haut des 30~\$ en février~; et il finit par
atteindre 266~\$ sur Mt.~Gox en avril.

De façon assez symbolique, un évènement coïncide avec cette hausse~: la
faillite des banques chypriotes. À cette époque, la crise financière bat
son plein sur l'île de Chypre, ce qui pousse le système financier à
prendre des mesures drastiques. Le 16 mars 2013, les banques limitent
les retraits des comptes de leurs clients. Le 25, le gouvernement
chypriote et l'UE décident (sans cadre légal préalable) que la Bank of
Cyprus doit être renflouée de manière interne, via une taxation
partielle des dépôts de plus de 100~000~euros. La Laiki Bank, deuxième
banque du pays, est démantelée suite à sa faillite et les avoirs
supérieurs à 100~000~euros sont tout simplement confisqués. Cet
évènement vient démontrer l'utilité du bitcoin, cet «~or numérique~» que
l'on peut posséder directement et qui n'est pas soumis aux contraintes
des banques à réserves fractionnaires\footnote{À cette occasion,
  GoldMoney et Bitcoin Magazine ont co-produit un documentaire, intitulé
  \emph{Cyprus: A Wake Up Call}, qui recueillait les témoignages des
  chypriotes touchés par cette crise. Voir sur Youtube~:
  \url{https://www.youtube.com/watch?v=mGGlYnxSFWM}.}.

Du côté de Silk Road, les choses évoluent également. L'enquête des
services de renseignement étasuniens, qui a débuté en juin 2011 suite à
la publication de l'article d'Adrien Chen sur Gawker et aux injonctions
des sénateurs Chuck Schumer et Joe Manchin appelant à fermer la
plateforme\footnote{«~injonctions des sénateurs Chuck Schumer et Joe
  Manchin appelant à fermer la plateforme~»~: Joe Manchin, \emph{Manchin
  Urges Federal Law Enforcement to Shut Down Online Black Market for
  Illegal Drugs}, 6 juin 2011~:
  \url{https://www.manchin.senate.gov/newsroom/press-releases/manchin-urges-federal-law-enforcement-to-shut-down-online-black-market-for-illegal-drugs}.},
commence à porter ses fruits. Après de longues recherches et quelques
actions litigieuses de la part des enquêteurs\footnote{Deux agents
  corrompus du FBI ont notamment profité de l'enquête pour subtiliser
  plus de 20~000 bitcoins (soit environ 350~000~\$) et pour simuler un
  assassinat. -- Ludovic Lars, «~\emph{Meurtres, escroqueries et vol de
  bitcoins - Le côté obscur de Silk Road}~», \emph{Le Journal du Coin},
  27 juin 2021~:
  \url{https://journalducoin.com/analyses/cote-obscur-silk-road/}.}, le
fondateur de la plateforme, Ross Ulbricht, finit par être suspecté en
juin 2013. C'est Gary Alford, un agent de l'\emph{Internal Revenue
Service} (le fisc étasunien), qui débusque la piste en retrouvant
l'annonce initiale de Ross sur le forum de Bitcoin et en la liant à son
adresse de courriel, où figure son nom civil\footnote{«~C'est Gary
  Alford {[}...{]} qui débusque la piste~»~: Nathaniel Popper, \emph{The
  Unsung Tax Agent Who Put a Face on the Silk Road}, 25 décembre 2015~:
  \url{https://www.nytimes.com/2015/12/27/business/dealbook/the-unsung-tax-agent-who-put-a-face-on-the-silk-road.html}.}.
En juillet 2013, le serveur de Silk Road est saisi par la police
islandaise et une copie est partagée aux agences
étasuniennes\footnote{«~le serveur de Silk Road est saisi par la police
  islandaise~»~:
  \url{https://antilop.cc/sr/files/2014_09_05_Declaration_of_Tarbell.pdf\#page=5}.}.
L'historique contient une connexion depuis une adresse IP située à San
Francisco, non loin d'où loge le jeune Texan. En septembre, grâce à
cette information, Ross est démasqué.

La chute de Silk Road a lieu au début de l'automne 2013. Le 1 octobre,
Ross Ulbricht est maîtrisé par des agents du FBI dans une bibliothèque
de San Francisco, avec sa session ouverte sur la plateforme. Le
lendemain, le site web est fermé, ce qui provoque l'émoi dans la
communauté. Le cours du bitcoin, qui se stabilisait autour des 125~\$
lors des jours précédents, chute brutalement pour toucher un niveau bas
de 85~\$. Entre le 2 et le 25 octobre, près de 174~000 bitcoins
appartenant à Ross sont saisis par les agences fédérales, un trésor qui
représente alors environ 31 millions de dollars\footnote{29~656,52080180
  BTC ont été envoyés à l'adresse
  \texttt{entre\ le\ 2\ et\ le\ 16\ octobre,\ tandis\ que\ 144~336,39429472\ BTC\ ont\ été\ transférés\ vers\ l\textquotesingle{}adresse}
  le 25 octobre. -- U.S. Attorney's Office, Southern District of New
  York, \emph{Manhattan U.S. Attorney Announces Seizure of Additional
  \$28 Million Worth of Bitcoins Belonging to Ross William Ulbricht,
  Alleged Owner and Operator of ``Silk Road'' Website}, 25 octobre
  2013~:
  \url{https://www.justice.gov/usao-sdny/pr/manhattan-us-attorney-announces-seizure-additional-28-million-worth-bitcoins-belonging}.}.

D'autres personnes liées à Silk Road seront arrêtées et condamnées au
cours des années suivantes. Ce sera le cas de Charlie Shrem qui sera
interpellé le 27 janvier 2014 par des agents fédéraux (FBI, IRS, DEA) et
accusé de facilitation de blanchiment d'argent avec sa société
BitInstant. En décembre 2014, il sera condamné à deux ans de prison
ferme pour transferts illicites. Ross Ulbricht sera lui condamné à deux
peines de réclusion à vie et à 40 ans d'enfermement supplémentaires,
pour des charges exclusivement non violentes, dans le but explicite d'en
faire un exemple\footnote{Lors du procès, la juge Katherine Forrest a
  affirmé~: «~Je rends ce jugement en ayant à l'esprit les crimes que
  vous avez commis et la nécessité de vous infliger la peine la plus
  sévère possible. Il ne doit faire aucun doute que le non-respect de la
  loi ne sera pas toléré. Il ne doit faire aucun doute que personne
  n'est au-dessus de la loi, quels que soient son éducation ou ses
  privilèges.~» -- \emph{Ross Ulbricht's sentencing transcript}, 4
  février 2015~:
  \url{https://freeross.org/wp-content/uploads/2018/02/Doc_36_Jan_12_Vol_VI_Appendix_A1314-A1554.pdf\#page=240}.}.

Bien que la chute de Silk Road représente la destruction d'un pan entier
de l'économie de Bitcoin, le cours remonte, conformément aux attentes de
certains investisseurs importants. L'engouement pour Bitcoin réapparaît,
ce qui provoque une nouvelle flambée du prix qui atteint de nouveaux
sommets. Celui-ci, s'étant alors stabilisé au-dessus des 100~\$,
augmente timidement durant la deuxième moitié d'octobre et dépasse,
début novembre, son ancien sommet de 266~\$. À partir de là, il monte en
flèche et atteint, le 4 décembre 2013, un nouveau plus haut niveau
historique sur Mt.~Gox à 1~240~\$. Cet épisode spéculatif pousse alors
beaucoup de médias à parler de Bitcoin pour la première fois.

Cependant, la plateforme Mt.~Gox, sur laquelle repose l'essentiel de
l'activité spéculative, est beaucoup plus fragile qu'elle ne le paraît.
En effet, celle-ci a subi des attaques informatiques tout au long de son
existence, entraînant la disparition progressive des fonds. Au début de
l'année 2014, il s'avère qu'il manque plus de 650~000~bitcoins dans les
caisses de la société, ce qui représente 381 millions de dollars à ce
moment-là\footnote{Kim Nilsson, \emph{The missing MtGox bitcoins}, 19
  avril 2015~:
  \url{https://blog.wizsec.jp/2015/04/the-missing-mtgox-bitcoins.html}}~!

En février 2014, la plateforme Mt.~Gox chute. Après avoir suspendu les
retraits le 7, le site est mis hors ligne le 25, et la faillite est
déclarée le 28. Mark Karpelès présente ses excuses publiques devant les
télévisions japonaises\footnote{«~Mark Karpelès présente ses excuses
  publiques devant les télévisions japonaises~»~:
  \url{https://www.youtube.com/watch?v=NeuCuM9CkBc}}. Cette crise est un
cataclysme pour Bitcoin~: la principale plaque tournante de l'économie
ferme ses portes, les détenteurs fortunés qui conservaient leurs
bitcoins sur la plateforme perdent tout et la confiance du grand public
(qui assimile Bitcoin à Mt.~Gox) s'effondre. Cela met fin à l'engouement
spéculatif de 2013--2014.

Mark Karpelès est suspecté de détournement de fonds et sa réputation est
ternie. Il sera arrêté par la justice japonaise en août 2015, ce qui lui
vaudra d'être surnommé le «~baron du bitcoin~» par les médias
français\footnote{«~ce qui lui vaudra d'être surnommé le "baron du
  bitcoin" par les médias français~»~: Pierre Alonso, \emph{En France,
  le passé trouble de l'ancien « baron du bitcoin »}, 29 juillet 2014~:
  \url{https://www.lemonde.fr/pixels/article/2014/08/01/en-france-le-passe-trouble-de-l-ancien-baron-du-bitcoin_4464044_4408996.html}.}.
Plus tard, il sera montré que les pertes de la plateforme provenaient de
plusieurs piratages ayant eu lieu entre 2011 et 2014, et que Mark
Karpelès était seulement coupable de négligence et n'avait pas
connaissance de la faille qui a permis de retirer la grosse part des
bitcoins entre 2011 et 2014.

Néanmoins, cette chute de Mt.~Gox aura pour effet d'assainir le marché
des plateformes de change. Dorénavant, elles se partageront le marché de
manière plus équitable et l'activité se répartira entre des acteurs
comme Bitfinex, bitFlyer, Bitstamp, Bittrex, BTCChina, BTC-e, Coinbase,
Gemini, OKEx, Kraken ou encore Poloniex.

\section*{Le débat sur la
scalabilité}\label{le-duxe9bat-sur-la-scalabilituxe9}
\addcontentsline{toc}{section}{Le débat sur la scalabilité}

\markright{Le débat sur la scalabilité}

La deuxième péripétie qui marque l'histoire de Bitcoin est le débat sur
la scalabilité, qui porte sur la capacité du système à passer à
l'échelle, c'est-à-dire à continuer de fonctionner de manière
équivalente à mesure que le nombre d'utilisateurs augmente. Cette
discorde s'amorce au cours de l'année 2013, qui est caractérisée par une
forte hausse du prix et de l'activité. Elle dégénère ensuite en une
guerre civile en 2015, pour se conclure en 2017 par le schisme en deux
communautés distinctes suite à la création d'un nouveau réseau nommé
Bitcoin Cash et à l'annulation du projet (artificiel) de compromis
SegWit2X. La période constitue une phase d'apprentissage majeure pour la
communauté. D'une part, cette dernière prend conscience des
imperfections de Bitcoin, qui avait été jusque-là vanté comme un système
de monnaie numérique dépourvu d'autorité centrale permettant d'envoyer
des paiements instantanés à n'importe qui, n'importe où dans le monde et
quasiment sans frais\footnote{Le 18 mai, la page d'accueil de
  Bitcoin.org présentait Bitcoin comme un système de monnaie numérique
  permettant de réaliser des «~transactions instantanées de pair à pair
  {[}...{]} dans le monde entier~» moyennant des «~frais de traitement
  faibles ou nuls~», en précisant que «~la gestion des transactions et
  l'émission des bitcoins sont effectuées collectivement par le
  réseau~». --
  \url{https://web.archive.org/web/20130518024528/http://bitcoin.org/en/}.}.
D'autre part, la communauté se rend compte du mécanisme de gouvernance
qui sous-tend l'évolution du protocole, plus complexe qu'elle n'en a
l'air (voir ch.~\hyperref[ch:changement]{10} et
\hyperref[ch:determination]{11}).

Le débat sur la scalabilité se concentre sur un paramètre présent dans
le protocole qui restreint la capacité transactionnelle du système~: la
limite de la taille des blocs ou \emph{blocksize limit}. Puisque dans
Bitcoin, les transactions sont incluses dans des blocs qui sont ajoutés
à la chaîne toutes les 10 minutes en moyenne, limiter la taille de ces
blocs revient en effet à instaurer un quota sur le nombre de
transactions confirmées.

En 2013, la taille limite est de 1~mégaoctet (1~Mo), ce qui correspond à
un flux théorique maximal de 7,37~transactions par seconde. Ajoutée dans
le protocole par Satoshi Nakamoto le 12 septembre 2010 sans annonce
publique de sa part, cette limite avait initialement pour rôle
d'empêcher les attaques par déni de service et devait être augmentée au
fil du temps\footnote{Le 4 octobre 2010, Satoshi décrivait sur le forum
  comment mettre en œuvre une augmentation de la taille limite des
  blocs. -- Satoshi Nakamoto, \emph{Re: {[}PATCH{]} increase block size
  limit}, /10/2010 19:48:40 UTC,
  \url{https://bitcointalk.org/index.php?topic=1347.msg15366\#msg15366}).}.
Néanmoins, après la disparition précipitée du fondateur, la décision a
été laissée aux membres de la communauté, ce qui a préparé le terrain
pour le conflit.

Dans le débat, deux camps principaux se font face. Le premier est celui
des partisans des gros blocs, ou \emph{big blockers}, qui se réclament
du projet initial de Satoshi, et qui désirent réaliser des mises à
niveau pour accroître la limite, voire la supprimer. Le camp opposé est
celui des partisans des petits blocs, ou \emph{small blockers}, qui
souhaitent restreindre la taille des blocs, afin de minimiser le coût de
gestion d'un nœud.

La première vision, qui est au départ majoritaire, considère que
l'augmentation progressive de la taille limite des blocs peut être
réalisée sans mettre en danger l'intégrité du système. Cette vision
estime que Bitcoin devrait rester un protocole de paiement, qui s'adapte
à la demande sans hausse des frais de transactions. Elle favorise de ce
fait la facilité d'utilisation par rapport à la sécurité. Elle accepte
le recours aux \emph{hard forks}, mises à niveau incompatibles du
protocole qui demandent la coordination de tous les membres du réseau
pour être réalisées proprement. Il s'agit d'une position plutôt
progressiste. De plus, ses partisans jugent généralement que la
détermination du protocole est assurée par les mineurs. Cette approche
est notamment portée par les développeurs Gavin Andresen, Mike Hearn et
Jeff Garzik.

La seconde vision, qui se développe à partir de 2013, se concentre sur
la sécurité aux dépens de la facilité d'utilisation. Son but est de
minimiser le coût de gestion d'un nœud afin de maximiser la
décentralisation du réseau. Cette vision considère que Bitcoin doit être
principalement un protocole de règlement, servant de base à des systèmes
en surcouche, qui peuvent être centralisés ou décentralisés. Elle prône
l'utilisation de \emph{soft forks}, des mises à niveau du protocole
rendues rétrocompatibles grâce à l'action des mineurs, et qui peuvent
donc être adoptées progressivement par les utilisateurs. Il s'agit d'une
position plutôt conservatrice, même si elle admet l'intervention de
certains changements essentiels. En outre, ses partisans considèrent
généralement que l'intégrité du protocole provient des utilisateurs.
Cette approche est notamment soutenue par les développeurs Pieter
Wuille, Gregory Maxwell, Wladimir van der Laan et luke-jr.

Dans ces deux camps, il existe des nuances et des contradictions. Le
sujet de la scalabilité étant complexe et technique, toutes sortes de
positions émergent, notamment sur le niveau auquel devrait être fixé la
limite de la taille des blocs~: 1~Mo, 2~Mo, 8~Mo~?

L'opposition entre les deux tendances se manifeste tout d'abord au sein
du projet logiciel de Bitcoin, dont les contributeurs principaux sont
majoritairement des partisans des petits blocs. Durant le printemps
2014, le logiciel connaît ainsi un tournant majeur. Au niveau de la
forme d'abord, il est renommé «~Bitcoin Core~» le 19 mars dans le but de
«~réduire la confusion entre Bitcoin-le-réseau et
Bitcoin-le-logiciel\footnote{Bitcoin Core, \emph{Bitcoin Core version
  0.9.0 released}, 19 mars 2014~:
  \url{https://bitcoin.org/en/release/v0.9.0\#rebranding-to-bitcoin-core}.}~».
Puis en ce qui concerne le fond, le dépôt GitHub subit une passation de
pouvoir le 7 avril lorsque Gavin Andresen cède son poste de mainteneur
principal à Wladimir van der Laan, pour se consacrer à ses activités de
scientifique en chef de la Fondation Bitcoin.

Ce changement de gestion se matérialise dans l'année, lorsque Mike Hearn
voit sa proposition d'ajout de la requête de réseau `` être rejetée pour
cause de non-unanimité dans l'équipe de Bitcoin Core\footnote{«~Mike
  Hearn voit sa proposition d'ajout de la requête de réseau getutxos
  être rejetée pour cause de non-unanimité dans l'équipe de Bitcoin
  Core~»~:
  \url{https://github.com/bitcoin/bitcoin/commit/70352e11c0194fe4e71efea06220544749f4cd64}.}.
L'ingénieur a besoin de cette fonctionnalité pour le développement de
son application de financement participatif, Lighthouse. De ce fait, il
est contraint de créer Bitcoin XT en décembre 2014, une implémentation
alternative issue de Bitcoin Core qui inclut les changements désirés
mais qui reste compatible avec le réseau\footnote{«~il est contraint de
  créer Bitcoin XT en décembre 2014~»~: Mike Hearn,
  \emph{{[}Bitcoin-development{]} Bitcoin XT}, /12/2014 18:04:08 UTC~:
  \url{https://lists.linuxfoundation.org/pipermail/bitcoin-dev/2014-December/007057.html}.}.

Simultanément, la discorde s'étend à l'ensemble de la communauté. La
logique des petits blocs se diffuse en particulier grâce à une vidéo
produite par le développeur Peter Todd en 2013, qui explique «~pourquoi
la taille limite des blocs permet à Bitcoin de rester libre et
décentralisé\footnote{Keep Bitcoin Free, \emph{Why the blocksize limit
  keeps Bitcoin free and decentralized} (vidéo), 17 mai 2013~:
  \url{https://www.youtube.com/watch?v=cZp7UGgBR0I}.}~». Un autre
argument invoqué est celui du marché des frais, notamment mis en valeur
par l'économiste français Nicolas Houy, qui explique dans un article de
2014 que «~laisser les frais de transaction résulter d'un marché et
rendre la taille des blocs non pertinente ou non contraignante
conduirait à un niveau de sécurité trop faible pour
Bitcoin\footnote{Nicolas Houy, \emph{The economics of Bitcoin
  transaction fees}, GATE, 2014. -- Cette idée, appelée la «~spirale
  fatale des frais~», avait été proposée dès 2011 par un utilisateur sur
  Bitcointalk. Voir Vandroiy, \emph{{[}If tx limit is removed{]}
  Disturbingly low future difficulty equilibrium}, 22 avril 2011~:
  \url{https://bitcointalk.org/index.php?topic=6284.msg92187\#msg92187}.}~».
Enfin, les partisans des petits blocs cherchent à justifier leur point
de vue en proposant des solutions de passage à l'échelle, permettant
d'accroître l'activité économique soutenue par le réseau sans pour
autant augmenter significativement la taille des blocs.

La première est la proposition des chaînes latérales, ou \emph{pegged
sidechains}, qui sont des chaînes secondaires fonctionnant en parallèle
de la chaîne principale, vers et depuis lesquelles des bitcoins peuvent
être transférés grâce à un ancrage bilatéral. Cette proposition est
présentée pour la première fois dans un document technique le 22 octobre
2014 par les développeurs de l'entreprise Blockstream. Co-fondée par
Adam Back, des développeurs de Bitcoin Core et des personnalités de la
finance, cette société dont la création est annoncée le même jour a pour
but de «~trouver une manière d'étendre l'utilisation de Bitcoin, qui
soit architecturalement solide et qui ne nécessite pas d'autorisation,
pour que la cryptomonnaie atteigne son plein potentiel\footnote{Blockstream,
  \emph{Why We Founded Blockstream}, 22 octobre 2014~:
  \url{https://web.archive.org/web/20161022162335/https://www.blockstream.com/2014/10/23/why-we-are-co-founders-of-blockstream.html}.}~».
Elle se focalisera dans un premier temps sur le développement des
sidechains, ce qui donnera naissance au modèle Elements et à sa mise en
œuvre, Liquid.

La seconde proposition est celle du réseau Lightning, ou \emph{Lightning
Network}, un réseau de canaux de paiements bâti en surcouche de Bitcoin
permettant d'effectuer des transferts instantanés et quasi-gratuits de
pair à pair. Le concept est présenté en février 2015 par Joseph Poon et
Thaddeus Dryja lors d'un séminaire de développeurs à San Francisco. Le
28 février, ils publient le livre blanc de leur invention, qui est
intitulé «~\emph{The Bitcoin Lightning Network}~» et qui présente les
éléments de base nécessaires pour construire un tel réseau\footnote{Joseph
  Poon et Thaddeus Dryja, \emph{The Bitcoin Lightning Network DRAFT
  Version 0.5}, 28 février 2015~:
  \url{https://lightning.network/lightning-network-paper-DRAFT-0.5.pdf}.}.
En particulier, une implémentation fluide de Lightning demande de
modifier le protocole Bitcoin par l'ajout de verrous temporels dans le
langage de script et par la correction d'un défaut appelé la
malléabilité des transactions\footnote{La malléabilité des transactions
  est la possibilité de modifier légèrement une transaction après sa
  diffusion sur le réseau, de façon à changer son identifiant. Cette
  capacité disparaît cependant une fois que la transaction a été
  confirmée par un mineur qui l'a incluse dans un bloc de la chaîne.}.

Ces deux propositions font envisager un passage à l'échelle par
surcouche, qui ne nécessiterait pas de s'en remettre totalement à un
tiers de confiance et qui ne compromettrait pas la sécurité de la chaîne
toute entière.

Du côté des partisans des gros blocs, on se concentre surtout sur
l'optimisation du logiciel et du protocole afin d'alléger la charge des
nœuds. Ainsi, le 6 octobre 2014, Gavin Andresen publie une feuille de
route sur le blog de la Fondation Bitcoin dans laquelle il décrit la
façon dont les effets de l'augmentation de l'activité du réseau
pourraient être compensés par des changements du protocole et par
l'évolution exponentielle de la technique décrite par les lois de Moore
et de Nielsen\footnote{Gavin Andresen, \emph{A Scalability Roadmap}, 6
  octobre 2014~:
  \url{https://web.archive.org/web/20150321091124/http://blog.bitcoinfoundation.org:80/a-scalability-roadmap}.}.
L'article mentionne notamment l'élagage des blocs les plus anciens
(\emph{block pruning}) qui permettrait de diminuer la taille de la
chaîne à conserver, l'«~engagement des UTXO~» ayant pour but d'accélérer
le téléchargement initial des blocs, ou encore le relai des blocs par
\emph{Invertible Bloom Lookup Tables} qui serait plus efficace.

Au printemps 2015, la taille moyenne des blocs approchant les 500~ko,
l'idée d'augmenter la capacité transactionnelle du réseau est remise sur
le devant de la scène sous l'impulsion de Gavin Andresen qui publie une
série d'articles à ce sujet sur son blog personnel\footnote{«~Gavin
  Andresen {[}...{]} publie une série d'articles à ce sujet sur son blog
  personnel~»~: Gavin Andresen, \emph{Time to roll out bigger blocks}, 4
  mai 2015~:
  \url{http://gavinandresen.ninja/time-to-roll-out-bigger-blocks}.}.
Dans les mois qui suivent, plusieurs propositions apparaissent,
notamment celle de Gavin d'augmenter la limite à 8~Mo (BIP-101),
conformément aux vœux des coopératives minières chinoises, et celle de
Pieter Wuille d'augmenter cette limite de 17,7~\% par an (BIP-103).
Malheureusement aucune proposition ne satisfait les forces en présence.
Cela a pour effet d'intensifier le conflit au sein de la communauté, qui
dégénère alors en une véritable guerre civile.

\section*{La guerre des blocs}\label{la-guerre-des-blocs}
\addcontentsline{toc}{section}{La guerre des blocs}

\markright{La guerre des blocs}

Le conflit sur la taille des blocs prend un tournant majeur au cours de
l'été 2015, avec la sortie de la version 0.11A de Bitcoin XT le 15 août,
qui inclut une augmentation de la limite de la taille des blocs,
changement incompatible avec les règles du réseau\footnote{Mike Hearn,
  \emph{Why is Bitcoin forking?}, 15 août 2015~:
  \url{https://medium.com/faith-and-future/why-is-bitcoin-forking-d647312d22c1}.}.
La mise à niveau intégrée est le BIP-101 qui programme un passage de la
limite de 1 à 8~Mo, à condition d'atteindre un signalement suffisant des
mineurs, à savoir 75~\% de la puissance de calcul du réseau. De ce fait,
cette version du logiciel a la possibilité de créer une séparation du
réseau et une scission de la chaîne en deux chaînes distinctes.

L'implémentation Bitcoin XT est dirigée par Mike Hearn qui se décrit
comme un «~dictateur bienveillant~» (concept courant dans le monde du
logiciel libre). Gavin Andresen participe au projet et a son mot à dire
sur la direction prise, mais «~Mike prend les décisions finales en cas
de litiges graves\footnote{\emph{FAQ -- BitcoinXT} (archive)~:
  \url{https://web.archive.org/web/20150908031806/https://bitcoinxt.software/faq.html\#who-is-involved}.}~».
L'augmentation de la taille limite des blocs est soutenue par les
grandes coopératives minières chinoises\footnote{«~L'augmentation de la
  taille limite des blocs est soutenue par les grandes coopératives
  minières chinoises~»~: \emph{Why upgrade to 8MB but not 20MB?}, 12
  juin 2015~:
  \url{https://www.reddit.com/r/Bitcoin/comments/3a0n4m/why_upgrade_to_8mb_but_not_20mb/}.}
et par une partie des entreprises de l'industrie\footnote{«~et par une
  partie des entreprises de l'industrie~»~: \emph{Industry Letter
  Regarding Block Size}, 24 août 2015~:
  \url{https://blog.bitmex.com/wp-content/uploads/2017/09/industry-letter.pdf}.}.

Ce mouvement ressemble grandement à un passage en force aux yeux d'un
noyau dur de la communauté, ce qui ne manque pas de provoquer sa
réaction épidermique. Dans la nuit du 16 au 17 août, Michael Macquart
(alias theymos), le modérateur principal du subreddit r/Bitcoin et
co-gestionnaire du forum Bitcointalk, publie ainsi un long article sur
Reddit dans lequel il annonce interdire toutes les discussions à propos
de Bitcoin XT\footnote{Michael Macquart, \emph{It's time for a break:
  About the recent mess \& temporary new rules}, /08/2015 00:50:15~:
  \url{https://www.reddit.com/r/Bitcoin/comments/3h9cq4/its_time_for_a_break_about_the_recent_mess/}.}.
Dans cet article, il explique en particulier pourquoi il considère que
Bitcoin XT ne peut pas être Bitcoin, et que les échanges à son sujet
n'ont par conséquent pas leur place sur le subreddit. Puisque
l'essentiel des conversations se passe sur r/Bitcoin à l'époque, cette
décision a un impact non négligeable.

L'apparition de Bitcoin XT marque donc le début d'une guerre civile au
sein de la communauté, que l'on appellera la guerre des blocs ou la
\emph{blocksize war}\footnote{Le terme anglais est issu de l'ouvrage
  \emph{The Blocksize War} de Jonathan Bier publié en 2021. Le terme
  français a été inventé par Morgan Phuc en 2017~:
  \url{https://bitconseil.fr/bitcoin-guerre-blocs/}.}, au cours de
laquelle la diplomatie fait progressivement place à l'animosité. Dans la
journée du 17 août, le journaliste Alex Hern écrit dans le Guardian~:
«~La guerre des bitcoins a commencé\footnote{Alex Hern,
  «~\emph{Bitcoin's forked: chief scientist launches alternative
  proposal for the currency}~», \emph{The Guardian}, 17 août 2015~:
  \url{https://www.theguardian.com/technology/2015/aug/17/bitcoin-xt-alternative-cryptocurrency-chief-scientist}.}.~»
Celle-ci sera marquée par une forte propagande des deux côtés, par la
censure sur les principaux canaux de communication\footnote{«~la censure
  sur les principaux canaux de communication~»~: John Blocke, \emph{A
  (brief and incomplete) history of censorship in /r/Bitcoin}, 14
  novembre 2016~:
  \url{https://medium.com/@johnblocke/a-brief-and-incomplete-history-of-censorship-in-r-bitcoin-c85a290fe43}.}
et par des attaques par déni de service contre les nœuds utilisant
Bitcoin XT et ses successeurs\footnote{«~des attaques par déni de
  service contre les nœuds utilisant Bitcoin XT et ses successeurs~»~:
  Deryk Makgill, \emph{Cyber Attacks Against Scaling Bitcoin}~:
  \url{https://wakgill.github.io/deryk/bitcoin-cyber-attacks}.}.

Toutefois, au début, beaucoup cherchent à apaiser le conflit et
appellent à la discussion. C'est le but des conférences appelées
\emph{Scaling Bitcoin}, organisées pour présenter les différentes
manières de faire passer Bitcoin à l'échelle. La première édition se
déroule en septembre à Montréal et parvient à réunir des individus des
deux camps dans une bonne foi partagée. Le deuxième édition a lieu en
décembre à Hong Kong, où la tension est déjà plus palpable.

Durant \emph{Scaling Bitcoin II}, une nouvelle proposition pour Bitcoin
est présentée~: Segregated Witness ou, plus simplement, SegWit. Imaginée
par Gregory Maxwell et Pieter Wuille, celle-ci prévoit de faciliter
l'implémentation du réseau Lightning (en corrigeant la malléabilité des
transactions) et d'augmenter la capacité transactionnelle de façon
rétrocompatible pour les nœuds non miniers. Elle fait partie de la
feuille de route de Gregory Maxwell publiée le même jour sur la liste de
diffusion\footnote{Gregory Maxwell, \emph{{[}bitcoin-dev{]} Capacity
  increases for the Bitcoin system}, /12/2015 22:02:17 UTC~:
  \url{https://lists.linuxfoundation.org/pipermail/bitcoin-dev/2015-December/011865.html}.}
et devient rapidement la mise à niveau défendue par les partisans des
petits blocs.

Au début de l'année 2016, Bitcoin XT échoue et Mike Hearn quitte la
communauté dans un abandon rageur retentissant\footnote{Mike Hearn,
  \emph{The resolution of the Bitcoin experiment}, 14 janvier 2016~:
  \url{https://blog.plan99.net/the-resolution-of-the-bitcoin-experiment-dabb30201f7}.}.
Cependant, une nouvelle implémentation émerge du côté des partisans des
gros blocs~: Bitcoin Classic. Celle-ci intègre une version modifiée du
BIP-101 qui porterait une taille limite de 2~Mo en cas d'un signalement
de 75~\% de la puissance de calcul du réseau. Bitcoin Classic gagne
rapidement en popularité au point d'interpeler le camp
opposé\footnote{«~Bitcoin Classic a émergé des cendres du débat entre XT
  et Core. Il s'agit d'une version de Bitcoin qui autoriserait une
  limite de deux mégaoctets, mettant en place des règles pour
  l'augmenter au cours du temps. Elle semble gagner rapidement du
  soutien.~» -- Paul Vigna, «~\emph{Is Bitcoin Breaking Up?}~»,
  \emph{The Wall Street Journal}, 17 janvier 2016, archive~:
  \url{https://web.archive.org/web/20160117220315/https://www.wsj.com/articles/is-bitcoin-breaking-up-1453044493}.}.

Le 20 février 2016, une réunion d'urgence est organisée à Hong Kong.
Cette «~Table Ronde~» réunit les principales coopératives minières,
certaines entreprises de l'écosystème et des contributeurs majeurs de
Bitcoin Core, dont Matt Corallo, Peter Todd ou encore luke-jr. Après de
nombreuses heures de discussions sous pression, les participants
arrivent à une entente, que l'on appellera l'accord de Hong Kong. D'un
côté, les développeurs de Bitcoin Core s'engagent à implémenter SegWit
et un doublement de la limite de la taille de base des blocs. De
l'autre, les mineurs s'engagent à activer SegWit et à n'utiliser que
Bitcoin Core\footnote{Bitcoin Roundtable, \emph{Bitcoin Roundtable
  Consensus}, 20 février 2016~:
  \url{https://medium.com/@bitcoinroundtable/bitcoin-roundtable-consensus-266d475a61ff}.}.

Mais ce sentiment de compromis ne dure pas, car l'année 2016 amène deux
évènements qui changent la donne. Le premier est l'intervention de Craig
S. Wright, un informaticien et entrepreneur australien qui prétend être
Satoshi Nakamoto. Il est propulsé sur le devant de la scène en décembre
2015 suite à la publication de deux enquêtes indépendantes par Wired et
Gizmodo, selon lesquelles il serait probablement le créateur de
Bitcoin\footnote{«~deux enquêtes indépendantes par Wired et Gizmodo,
  selon lesquelles il serait probablement le créateur de Bitcoin~»~:
  Andy Greenberg, Gwern Branwen, \emph{Bitcoin's Creator Satoshi
  Nakamoto Is Probably This Unknown Australian Genius}, 8 décembre 2015
  (\url{https://web.archive.org/web/20151208214655/https://www.wired.com/2015/12/bitcoins-creator-satoshi-nakamoto-is-probably-this-unknown-australian-genius/})~;
  Sam Biddle and Andy Cush, \emph{This Australian Says He and His Dead
  Friend Invented Bitcoin}, 8 décembre 2015
  (\url{https://web.archive.org/web/20151208235451/https://gizmodo.com/this-australian-says-he-and-his-dead-friend-invented-bi-1746958692}).}.
Ces enquêtes se fondent sur des éléments qui laissent en effet penser
qu'il a pu être impliqué dans la conception de la cryptomonnaie aux
côtés de son ami Dave Kleiman, décédé en 2013.

Quelques mois plus tard, le 2 mai 2016, Craig Wright publie un long et
tortueux article de blog\footnote{Craig Wright, \emph{Jean-Paul Sartre,
  Signing and Significance}, 2 mai 2016~:
  \url{https://web.archive.org/web/20160502072011/http://www.drcraigwright.net/jean-paul-sartre-signing-significance/}.},
dans lequel il inclut une signature correspondant à la clé publique
utilisée pour recevoir la récompense du bloc 9 et envoyer le premier
paiement à Hal Finney en janvier 2009. En outre, un entretien de Craig
Wright avec la BBC est mis en ligne ce jour-là, dans lequel l'Australien
affirme avoir été «~l'élément principal~» derrière Satoshi Nakamoto,
mais que «~d'autres personnes l'ont aidé\footnote{BBC News, \emph{Mr
  Bitcoin: "I don't want money, I don't want fame!"} (vidéo), 2 mai
  2016~: \url{https://www.youtube.com/watch?v=5DCAC1j2HTY}.}~». Il
prétend également avoir signé un message en privé devant le journaliste
qui l'interroge.

Cependant, il s'avère que les documents fournis dans les enquêtes et les
éléments avancés par Craig Wright lui-même ne sont pas probants. En
particulier, on découvre très rapidement que la signature présente dans
l'article de blog est la signature d'une transaction existante sur la
chaîne de Bitcoin qui est simplement encodée différemment\footnote{La
  signature fournie par Craig Wright correspond à la clé publique liée à
  l'adresse `` qui a servi à recevoir la récompense du bloc 9 et à
  envoyer le premier paiement à Hal Finney le 12 janvier 2009, et a donc
  été produite par Satoshi Nakamoto. Néanmoins, un utilisateur de Reddit
  (JoukeH) a découvert très rapidement qu'il s'agissait de la signature
  d'une transaction présente sur la chaîne~:
  \url{https://www.reddit.com/r/Bitcoin/comments/4hf4xj/creator_of_bitcoin_reveals_identity/d2pf70v/}.}.
Ce fait incite la communauté à la prudence.

Le même jour, Gavin Andresen publie un article dans lequel il dit croire
que Craig Wright est «~la personne qui a inventé Bitcoin~», ce dernier
lui ayant présenté en personne «~des messages signés avec les clés que
seul Satoshi devrait posséder\footnote{Gavin Andresen, \emph{Satoshi}, 2
  mai 2016~: \url{http://gavinandresen.ninja/satoshi}}~». En
conséquence, le rôle de mainteneur et le \emph{commit access} de Gavin
sur le dépôt de Bitcoin Core sont révoqués dans la foulée, sous prétexte
que les membres de l'équipe de Bitcoin Core craignent qu'il ait été
piraté. De manière générale, cette affirmation douteuse, confirmée en
personne le jour-même lors de la conférence Consensus 2016, a pour effet
de le discréditer et son accès au dépôt GitHub ne sera jamais restauré.
Il reconnaîtra plus tard avoir été dupé\footnote{«~Il reconnaîtra plus
  tard avoir été dupé~»~: \emph{Déposition de Gavin Andresen dans le
  cadre de l'affaire Wright / Kleiman}, 19 juin 2020~:
  \url{https://storage.courtlistener.com/recap/gov.uscourts.flsd.521536/gov.uscourts.flsd.521536.589.3.pdf\#page=88}.}.

Le second évènement qui vient influencer le conflit est un incident qui
ne se passe pas au sein de la communauté de Bitcoin, mais sur Ethereum,
un système alternatif à Bitcoin dédié aux \emph{smart contracts} lancé
en 2015. Il s'agit de la scission entre Ethereum (ETH) et Ethereum
Classic (ETC) qui fait suite au piratage de TheDAO.

Le 17 juin 2016, le contrat de TheDAO, une organisation autonome
décentralisée ayant pour mission d'investir dans l'écosystème, est
piraté et 3,6 millions d'éthers (l'éther est l'unité de compte
d'Ethereum) valant 50 millions de dollars sont dérobés, ce qui
représente 4,4 \% de la quantité totale d'éthers en circulation à
l'époque. La décision est alors prise par une grande majorité de la
communauté d'annuler purement et simplement ce vol par la modification
de l'état du système. Un mois plus tard, le 20 juillet, le changement
est appliqué ce qui conduit à une scission de la chaîne en deux chaînes
distinctes~: celle suivant le protocole modifié (qui est majoritaire et
qui prendra le nom d'Ethereum) et celle suivant le protocole initial
(qui est minoritaire et qui s'appellera Ethereum Classic). Les
détenteurs se retrouvent avec des éthers différents des deux côtés.

Cette séparation n'est pas propre. En particulier, elle n'inclut pas de
protection contre la rediffusion des transactions, ce qui signifie que
des transferts réalisés sur une chaîne peuvent être répliqués sur
l'autre par un tiers, menant à des «~attaques par rediffusion~»
(\emph{replay attacks}). Cela perturbe les plateformes de change qui
doivent composer avec ce problème. Ainsi, même si le prix combiné des
deux éthers finit par être supérieur au prix de l'éther initial, tout le
monde peut observer les effets négatifs d'une scission créée par un hard
fork. Cet exemple conforte de ce fait la position conservatrice des
partisans des petits blocs qui recommandent de faire évoluer le
protocole par des soft forks tels que SegWit.

À cause de ces deux évènements, le camp des partisans des gros blocs
ressort de cette année 2016 particulièrement diminué, tant au niveau
réputationnel qu'argumentatif.

C'est ce moment que choisissent les développeurs de Bitcoin Core pour
lancer le signalement de SegWit par les mineurs, qui commence le 15
novembre 2016, pour une période d'un an\footnote{«~lancer le signalement
  de SegWit par les mineurs, qui commence le 15 novembre 2016~»~:
  Bitcoin Core, \emph{Bitcoin Core version 0.13.1 released}, 27 octobre
  2016~:
  \url{https://bitcoin.org/en/release/v0.13.1\#segregated-witness-soft-fork}.}.
La mise à niveau exige un taux de signalement de 95~\% de la puissance
de calcul pour être activée, dans le but d'assurer largement la
rétrocompatiblité du changement.

Cependant, les principales coopératives minières refusent SegWit (pour
des raisons multiples\footnote{SegWit annulait notamment les effets de
  l'AsicBoost secret, une technique d'optimisation du minage. Voir
  Gregory Maxwell, \emph{{[}bitcoin-dev{]} BIP proposal: Inhibiting a
  covert attack on the Bitcoin POW function}, /04/2017 21:37:45 UTC~:
  \url{https://lists.linuxfoundation.org/pipermail/bitcoin-dev/2017-April/013996.html}.})
et, durant les premiers mois, le taux de blocs en faveur de la mise à
niveau stagne autour des 25~\%, très loin du seuil demandé. SegWit est
bloqué.

Au début de l'année 2017, les blocs commencent à être pleins, ce qui
engendre une augmentation significative des frais de transaction et des
temps de confirmation sur la chaîne. Mi-février, les frais médians
dépassent les 30 centimes de dollar, pour la première fois dans
l'histoire de Bitcoin. C'est dans ce contexte que la demande pour un
changement s'accroît de part et d'autre du conflit.

D'un côté, nous avons Bitcoin Unlimited, une implémentation qui a gagné
en popularité chez les partisans des gros blocs lors de l'été 2016 et
qui a pris la relève de Bitcoin Classic. Celle-ci est notamment soutenue
par Roger Ver, le PDG de l'entreprise Bitcoin.com qui est alors devenu
une personnalité influente de la communauté\footnote{Roger Ver est connu
  pour son prosélytisme de l'adoption du bitcoin dans le commerce et
  pour sa participation éclatante au documentaire \emph{The Bitcoin
  Gospel} diffusé le 1 novembre 2015 sur Youtube. Voir
  \url{https://www.youtube.com/watch?v=8zKuoqZLyKg&t=2831s}.}, ce qui
permet à Bitcoin Unlimited de disposer d'un large financement. En mars
2017, le signalement pour Unlimited dépasse ainsi celui de SegWit.

Cependant, le 17 mars, la possibilité d'un hard fork contentieux pousse
les plateformes de change à considérer la potentielle monnaie créée par
Bitcoin Unlimited comme une cryptomonnaie alternative. Fortes de leur
expérience avec la scission entre ETH et ETC, elles exigent en outre
qu'elle intègre une protection contre la rediffusion, faute de quoi elle
ne sera même pas listée\footnote{«éelles exigent en outre qu'elle
  intègre une protection contre la rediffusion, faute de quoi elle ne
  sera même pas listée~»~: Aaron van Wirdum, \emph{Major Exchanges Will
  Consider Bitcoin Unlimited a "New Asset"}, 17 mars
  2017~:\url{https://bitcoinmagazine.com/technical/major-exchanges-will-consider-bitcoin-unlimited-new-asset}}.
Cette décision est dévastatrice pour les partisans des gros blocs.

De l'autre côté, la pression en faveur de l'activation de SegWit
s'intensifie au sein des partisans des petits blocs, qui commencent à
devenir impatients. Ainsi, le 12 mars 2017, un individu se faisant
appeler Shaolin Fry publie la proposition d'un UASF («~\emph{User
Activated Soft Fork}~») qui permettrait de verrouiller la mise à niveau
sans le signalement des mineurs dès le 1 août\footnote{Shaolin Fry,
  \emph{{[}bitcoin-dev{]} Flag day activation of segwit}, /03/2017
  15:50:27 UTC~:
  \url{https://lists.linuxfoundation.org/pipermail/bitcoin-dev/2017-March/013714.html}.}.
Cette mesure, pour le moins audacieuse, est dangereuse et ne fait pas
l'unanimité parmi les \emph{small blockers}, comme l'illustre
l'opposition de Gregory Maxwell\footnote{«~l'opposition de Gregory
  Maxwell~»~: Gregory Maxwell, \emph{{[}bitcoin-dev{]} I do not support
  the BIP 148 UASF}, /04/2017 07:56:31 UTC~:
  \url{https://lists.linuxfoundation.org/pipermail/bitcoin-dev/2017-April/014152.html}.}.

Cependant, la menace de l'UASF existe et exerce une influence. Ainsi,
devant le désir de la communauté de procéder à SegWit, les gros acteurs
de l'écosystème (entreprises et mineurs) sont amenés à signer un accord
le 23 mai 2017, en marge de la conférence Consensus 2017 à New York. Cet
accord de New York, comme on l'appellera par la suite, représente un
compromis théorique dans le conflit qui fait rage~: il engage les
signataires, d'une part, à activer SegWit avec un seuil de signalement
de 80~\% de la puissance de calcul, d'autre part, à doubler la taille
limite des blocs dans les six mois qui suivent. La mise à niveau
implémentant cet accord prendra le nom de SegWit2X. Ce pseudo-compromis
est néanmoins rapidement critiqué en raison de l'absence des
développeurs de Bitcoin Core à la réunion, qui n'étaient tout simplement
pas conviés.

L'accord mène à l'activation de SegWit durant l'été 2017. En juillet,
les mineurs commencent à signaler massivement la mise à niveau. Le 21,
le processus de verrouillage est enclenché (ce qui rend l'UASF
ineffectif). SegWit finit par être activé le 24 août 2017.

La mise à niveau se passe très bien. Cependant, il se produit au même
moment une autre scission, pleinement désirée. En opposition à l'UASF,
les mineurs décident d'activer un nouveau protocole, incompatible avec
le protocole originel, qui n'intègre pas SegWit et qui implémente une
taille limite des blocs de 8~Mo~: Bitcoin Cash. Le lancement de ce
protocole est placé sous l'autorité d'Amaury Séchet, un développeur
français.

Le 1 août 2017, avec le bloc 478~559 miné à 18 heures 12, naît ainsi
Bitcoin Cash. À la suite de cette scission, les détenteurs de bitcoin se
retrouvent avec un montant similaire en bitcoins (BTC) et en bitcoins
cash (BCH). Ceux qui désapprouvent SegWit rejoignent Bitcoin Cash.

Durant le mois d'août, alors que SegWit est enfin verrouillé, certains
partisans des petits blocs commencent à s'opposer à la deuxième partie
de SegWit2X, le doublement de la taille des blocs, via une campagne de
communication baptisée «~NO2X~». Dans leur argumentaire, ils insistent
en particulier sur l'absence de rediffusion des transactions que
comporte ce hard fork\footnote{«~ils insistent en particulier sur
  l'absence de rediffusion des transactions~»~: Alex Bosworth,
  \emph{{[}Bitcoin-segwit2x{]} Alpha Milestone}, /06/2017 17:35:00 UTC~:
  \url{https://lists.linuxfoundation.org/pipermail/bitcoin-segwit2x/2017-June/000003.html}.}.
En effet, SegWit2X est pensé comme un changement non contentieux et
n'inclut par conséquent pas ce type de procédé qui alourdirait
considérablement la charge de la mise à jour pour les portefeuilles.

L'opposition gronde. Les développeurs de Bitcoin Core refusent
d'approuver ce changement. Les utilisateurs se mobilisent pour
protester, car «~la manière dont l'accord a été conclu va à l'encontre
de l'essence même de Bitcoin\footnote{«~Nous nous opposons au New York
  Agreement et au hard fork Bitcoin SegWit2X de novembre~», Change.org,
  15 octobre 2017~:
  \url{https://www.change.org/p/mineurs-et-entreprises-de-l-éco-système-bitcoin-nous-nous-opposons-au-new-york-agreement-et-au-hard-fork-bitcoin-segwit2x-de-novembre}.}~».
Face à cette opposition, les entreprises signataires de l'accord de New
York se rétractent peu à peu.

Le projet de doublement de la limite de la taille des blocs est
finalement abandonné le 8 novembre 2017, soit une semaine avant son
activation programmée. Les promoteurs du projet -- Mike Belshe, Wences
Casares, Jihan Wu, Jeff Garzik, Peter Smith et Erik Voorhees --
déclarent conjointement~:

«~Notre objectif a toujours été d'améliorer Bitcoin en douceur. Bien que
nous croyons fermement en la nécessité d'augmenter la taille des blocs,
il y a une chose que nous croyons encore plus importante~: garder la
communauté unie. Malheureusement, il est clair que nous n'avons pas
recueilli un consensus suffisant à l'heure actuelle pour une
modification propre de la taille des blocs. Continuer sur la voie
actuelle pourrait diviser la communauté et constituer un revers pour la
croissance de Bitcoin. Cela n'a jamais été l'objectif de
Segwit2x\footnote{Mike Belshe, \emph{{[}Bitcoin-segwit2x{]} Segwit2x
  Final Steps}, /11/2017 16:58:41 UTC~:
  \url{https://lists.linuxfoundation.org/pipermail/bitcoin-segwit2x/2017-November/000685.html}.}.~»

C'est une grande victoire pour la philosophie des petits blocs qui
dominera dorénavant la chaîne. Pour ce qui est des solutions de
scalabilité sur BTC, le réseau Lightning est lancé en version bêta en
mars 2018. Le concept de chaîne latérale est également expérimenté, avec
le lancement de Rootstock en janvier 2018 et celui de la sidechain
Liquid développée par Blockstream en septembre de la même année.

De l'autre bord, suite à l'annulation de SegWit2X, beaucoup de partisans
de l'augmentation de la capacité de la chaîne se dirigent vers d'autres
protocoles comme Bitcoin Cash ou Ethereum.

Malgré le dénigrement constant de ses détracteurs, l'évolution de
Bitcoin Cash suivra son cours. Néanmoins, sa communauté se délitera
progressivement et la cryptomonnaie connaîtra deux scissions majeures
avec la création de Bitcoin SV en novembre 2018 puis celle de «~eCash~»
(XEC) en novembre 2020. La part de marché de l'ensemble dégringolera en
conséquence~: en novembre 2023, la valeur agrégée de ces trois
cryptomonnaies représentait moins de 1~\% de celle du BTC.

\section*{L'essor des cryptomonnaies
alternatives}\label{lessor-des-cryptomonnaies-alternatives}
\addcontentsline{toc}{section}{L'essor des cryptomonnaies alternatives}

\markright{L'essor des cryptomonnaies alternatives}

Bitcoin est un projet libre~: il se base sur un code source ouvert qui
peut être copié et déployé sur un nouveau réseau par n'importe qui.
Cette particularité est excellente pour l'innovation~: un individu, quel
qu'il soit, peut apporter des modifications au code et en faire la base
d'une nouvelle cryptomonnaie s'il le désire. La découverte de Bitcoin
ouvre ainsi la voie à une vraie concurrence des monnaies sur Internet.
Cependant, cette liberté existe aussi pour les personnes mal
intentionnées qui peuvent profiter de cette ouverture pour créer des
projets douteux, allant de la cryptomonnaie inutile à l'escroquerie pure
et simple, en passant par la pyramide de Ponzi ouverte. C'est dans cette
dualité entre l'honnêteté de l'innovateur et la cupidité du malfaiteur
que se produit l'essor de ce qu'on appelle les «~cryptomonnaies
alternatives~» (ou «~\emph{altcoins}~» en anglais).

La première idée d'une cryptomonnaie distincte de Bitcoin apparaît alors
que Satoshi est encore présent dans la communauté. En novembre 2010, une
discussion sur un système distribué de noms de domaine (alors appelé
BitDNS) s'engage sur IRC, puis sur le forum de Bitcoin\footnote{«~une
  discussion sur un système distribué de noms de domaine (alors appelé
  BitDNS) s'engage sur IRC, puis sur le forum de Bitcoin~»~: appamatto,
  \emph{BitDNS and Generalizing Bitcoin}, /11/2010 03:02:31 UTC~:
  \url{https://bitcointalk.org/index.php?topic=1790.msg22019\#msg22019}.}.
Il s'agit d'associer des identifiants de site web (DNS) à des pièces
créées par le protocole, comme les bitcoins dans Bitcoin. Le registre
étant public et difficilement falsifiable, cela améliorerait les choses
par rapport au système existant. Satoshi n'est pas hostile à l'idée et
suggère de miner la chaîne en combinaison (\emph{merge mining}) avec
celle de Bitcoin\footnote{Satoshi Nakamoto, \emph{Re: BitDNS and
  Generalizing Bitcoin}, /12/2010 21:02:42 UTC~:
  \url{https://bitcointalk.org/index.php?topic=1790.msg28696\#msg28696}.}.
Cela donne finalement naissance à Namecoin en avril 2011, créé sous
l'impulsion de Vincent Durham\footnote{«~Cela donne finalement naissance
  à Namecoin~»~: Vincent Durham (vinced), \emph{{[}announce{]} Namecoin
  - a distributed naming system based on Bitcoin}, /04/2011 00:52:59
  UTC~:
  \url{https://bitcointalk.org/index.php?topic=6017.msg88356\#msg88356}.}.

Par la suite, d'autres cryptomonnaies apparaissent, comme Ixcoin ou
Tenebrix. Tenebrix a la particularité d'implémenter l'algorithme de
preuve de travail scrypt, mis au point par le mineur ArtForz et
supposément résistant aux processeurs graphiques. En octobre 2011,
Litecoin est lancé par Charlie Lee, en tant que «~version allégée de
Bitcoin~» dont les blocs sont minés quatre fois plus rapidement, dont
l'unité de compte est quatre fois moins rare et qui intègre l'algorithme
scrypt. Litecoin a pour but d'être «~à l'argent ce que Bitcoin est à
l'or\footnote{Charlie Lee (coblee), \emph{Re: {[}ANN{]} Litecoin - a
  lite version of Bitcoin. Be ready when is launches!}, /10/2011
  06:14:28 UTC~:
  \url{https://bitcointalk.org/index.php?topic=47417.msg564414\#msg564414}.}~».

En août 2012, Sunny King et Scott Nadal lancent PPCoin, un système
introduisant le procédé de preuve d'enjeu, qu'ils présentent comme une
alternative à la preuve de travail «~économe en énergie à long
terme\footnote{Sunny King, \emph{{[}ANN{]} {[}PPC{]} PPCoin Released! -
  First Long-Term Energy-Efficient Crypto-Currency}, /08/2012 19:54:28
  UTC~:
  \url{https://bitcointalk.org/index.php?topic=101820.msg1113938\#msg1113938}~;
  Sunny King, Scott Nadal, \emph{PPCoin: Peer-to-Peer Crypto-Currency
  with Proof-of-Stake}, 19 août 2012, archive~:
  \url{https://web.archive.org/web/20121021014644/http://www.ppcoin.org/static/ppcoin-paper.pdf}.}~».
La preuve d'enjeu est intégrée de manière hybride aux côtés de la preuve
de travail. PPCoin sera progressivement renommé en Peercoin au cours des
années.

En 2013, avec l'engouement financier résultant du succès du bitcoin, la
création de cryptomonnaies originales devient extrêmement rentable. Les
nouveaux protocoles se multiplient à l'instar de Feathercoin en avril,
de Primecoin en juillet ou encore du célèbre Dogecoin en
décembre\footnote{«~du célèbre Dogecoin en décembre~»~: Ludovic Lars,
  \emph{Le dogecoin est-il un concurrent sérieux au bitcoin~?}, 1 mai
  2021~:
  \url{https://www.contrepoints.org/2021/05/01/396380-le-dogecoin-est-il-un-concurrent-serieux-au-bitcoin}.}.
Le site web coinmarketcap.com est créé en mai 2013 pour recenser
l'ensemble des cryptomonnaies et les classer par «~capitalisation
boursière~», c'est-à-dire par leur valeur agrégée (le nombre d'unités
multiplié par le prix unitaire).

Cependant, ce foisonnement n'est pas du goût de tout le monde et un
mouvement de rejet se forme face à ce qui ressemble à une fragmentation
dommageable de l'écosystème. Dès 2011, on voit un certain scepticisme se
développer vis-à-vis des premières cryptomonnaies alternatives,
scepticisme qui transparaît dans les réactions de Hal Finney et de Gavin
Andresen\footnote{«~scepticisme qui transparaît dans les réactions de
  Hal Finney et de Gavin Andresen~»~: Hal Finney, \emph{Re: Early
  speculators' reward}, /05/2011 18:28:34 UTC~:
  \url{https://bitcointalk.org/index.php?topic=10666.msg152988\#msg152988}~;
  Gavin Andresen, \emph{Alternative Block Chains~: be safe!}, /09/2011
  13:21:18 UTC~:
  \url{https://bitcointalk.org/index.php?topic=42465.msg516789\#msg516789}.}.
Puis, le rejet devient plus tranché en août 2013 avec les prises de
positions de Gavin Andresen (encore lui), qui assimile la création de
nouvelles cryptomonnaies à de l'inflation\footnote{Gavin Andresen,
  \emph{The macro-economics of alt-coins}, 19 août 2013~:
  \url{https://gavintech.blogspot.com/2013/08/the-macro-economics-of-alt-coins.html}.},
et de Daniel Krawisz, auteur pour le Satoshi Nakamoto Institute, qui met
en lumière la difficulté extrême de surpasser l'effet de réseau de
Bitcoin\footnote{Daniel Krawisz, \emph{The Problem with Altcoins}, 22
  août 2013~:
  \url{https://nakamotoinstitute.org/mempool/the-problem-with-altcoins/}.}.

En parallèle, on assiste à la formation de la tendance contraire~: un
pluralisme cryptomonétaire qui prône l'ouverture et la tolérance envers
cette diversité des cryptomonnaies. Celui-ci est en particulier défendu
par le jeune Vitalik Buterin en septembre 2013\footnote{Vitalik Buterin,
  «~\emph{In Defense of Alternative Cryptocurrencies}~», \emph{Bitcoin
  Magazine}, 7 septembre 2013~:
  \url{https://bitcoinmagazine.com/business/defense-alternative-cryptocurrencies}.},
qui le mettra en pratique avec l'élaboration de son propre projet,
Ethereum.

À partir de 2014, cette tendance est renforcée par l'apparition de
systèmes cryptoéconomiques fondamentalement plus pertinents que les
simples copies de Bitcoin. Ainsi, pour corriger le manque de
confidentialité de la chaîne de blocs de Bitcoin, plusieurs solutions
voient le jour. C'est le cas de Darkcoin qui démarre en janvier 2014 (et
qui deviendra plus tard Dash). Monero, un protocole intégrant la
confidentialité par défaut dont le nom signifie «~pièce de monnaie~» en
esperanto, est lui lancé en avril. En outre, la publication des
protocoles Zerocoin et Zerocash par Matthew Green en 2013 mèneront à la
création de Zcash en octobre 2016.

Mais la confidentialité n'est pas le seul terrain d'innovation, et
d'autres protocoles séparés émergent pour mettre en œuvre un
perfectionnement de l'aspect programmable de Bitcoin, conformément à
l'idée d'un «~Bitcoin 2.0~» qui se diffuse alors dans la communauté. En
effet, le protocole de Satoshi est peu adapté pour réaliser des
opérations complexes et créer des jetons numériques secondaires, même si
cela peut se faire sur des surcouches comme Omni et Counterparty. C'est
pourquoi on assiste à la naissance de nouveaux systèmes, à l'instar de
Bitshares, une place de marché décentralisée qui a la particularité de
fonctionner par preuve d'enjeu déléguée, et de NXT, une plateforme
incluant un grand nombre de fonctionnalités\footnote{«~NXT, une
  plateforme incluant un grand nombre de fonctionnalités~»~: Bas
  Wisselink, \emph{\textbar{} Nxt \textbar{} Blockchain Platform
  \textbar{} Proof of Stake \textbar{} Official}, /04/2014 19:01:37
  UTC~:
  \url{https://bitcointalk.org/index.php?topic=587007.msg6426512\#msg6426512}.}.
Mais le système qui se distingue le plus est Ethereum.

Ethereum reprend la programmabilité de Bitcoin et la généralise en
constituant une sorte d'ordinateur mondial décentralisé, fonctionnant en
parallèle sur tous les nœuds d'un réseau pair à pair. Ce projet est,
comme on l'a dit, issu de l'esprit de Vitalik Buterin, qui en dresse les
contours à la fin de l'année 2013. Avec ses 7 co-fondateurs, il réalise
une prévente de jetons en juillet-août 2014, qui recueille 31 529
bitcoins\footnote{L'adresse BTC utilisée par EthSuisse était ``. --
  Vitalik Buterin, \emph{Launching the Ether Sale}, 22 juillet 2014~:
  \url{https://blog.ethereum.org/2014/07/22/launching-the-ether-sale}.}
(soit plus de 15 millions de dollars à l'époque) pour en financer le
développement. Ethereum est un système volontairement plus progressiste,
plus innovant et plus flexible que Bitcoin. La chaîne sera
officiellement lancée un an plus tard, le 30 juillet 2015.

L'année 2014 est enfin l'année où apparaît le premier \emph{stablecoin},
le Tether USD (ou USDT), qui est lancé sur la chaîne de Bitcoin le 6
octobre sous le nom de Realcoin\footnote{«~le Tether USD {[}...{]} est
  lancé sur la chaîne de Bitcoin le 6 octobre sous le nom de
  Realcoin~»~:
  \url{https://www.omniexplorer.info/tx/5ed3694e8a4fa8d3ec5c75eb6789492c69e65511522b220e94ab51da2b6dd53f}.}.
Ce jeton numérique est adossé au dollar grâce à la garantie de
l'entreprise Tether Limited, qui s'engage à racheter chaque unité contre
un dollar réel. Cette cryptomonnaie «~stable~» permet aux individus et
aux plateformes de change de bénéficier de la faible volatilité du
dollar, sans avoir à en subir les inconvénients légaux.

Face à ce développement, le mouvement de rejet à l'égard de ces nouveaux
projets continue de croître. Il s'illustre le 22 octobre 2014 par la
publication du document de Blockstream sur les sidechains, qui vient
décrire comment Bitcoin pourrait être à la base de tous les cas
d'utilisation, et d'un article, expliquant la démarche derrière la
fondation de la société. Dans ce dernier, les développeurs travaillant
pour Blockstream écrivent ainsi~:

«~L'approche des altcoins, qui consiste à créer une nouvelle
cryptomonnaie uniquement pour introduire de nouvelles fonctionnalités,
crée une incertitude pour tous ceux qui observent les cryptomonnaies de
l'extérieur. Il ne semble pas y avoir de point d'arrêt naturel, chaque
copie pouvant être copiée à nouveau, à l'infini. Cela crée à la fois une
fragmentation du marché et une fragmentation du développement. Nous
pensons que pour que les cryptomonnaies réussissent dans leur ensemble,
nous devons construire de l'effet de réseau, et non de la
fragmentation\footnote{Blockstream, \emph{Why We Founded Blockstream},
  22 octobre 2014~:
  \url{https://web.archive.org/web/20161022162335/https://www.blockstream.com/2014/10/23/why-we-are-co-founders-of-blockstream.html}.}.~»

Au fil des années, ce rejet prend progressivement le nom de maximalisme
du bitcoin (\emph{bitcoin maximalism}), par réappropriation du terme
utilisé péjorativement par Vitalik Buterin à l'encontre de ceux qui
dénigrent systématiquement les cryptomonnaies alternatives\footnote{L'expression
  utilisée par Vitalik Buterin était «~\emph{bitcoin dominance
  maximalist}~», qu'on peut traduire par «~maximaliste de la dominance
  du bitcoin~»
  (\url{https://www.reddit.com/r/Bitcoin/comments/2is4us/whats_wrong_with_counterparty/cl54c0y/}).
  Dans son article publié le 19 novembre 2014, il définissait le
  maximalisme du bitcoin comme «~l'idée qu'un milieu de multiples
  cryptomonnaies concurrentes est indésirable, qu'il est mal de lancer
  ``encore un autre jeton'', et qu'il est à la fois juste et inévitable
  que la monnaie bitcoin en vienne à atteindre une position de monopole
  sur la scène des cryptomonnaies~». -- Vitalik Buterin, \emph{On
  Bitcoin Maximalism, and Currency and Platform Network Effects}, 19
  novembre 2014~:
  \url{https://blog.ethereum.org/2014/11/20/bitcoin-maximalism-currency-platform-network-effects/}.}.
Le mouvement prône la maximisation de la dominance économique du bitcoin
par rapport à ses concurrents proches et prescrit à ses partisans d'agir
dans ce sens. Il s'agit de mettre en valeur l'effet de réseau, non
seulement parce que c'est techniquement nécessaire mais aussi parce que
c'est moralement désirable\footnote{L'appartenance au maximalisme est
  parfois revendiquée aujourd'hui par des gens qui n'embrassent pas son
  caractère extrémiste (même s'il se trouve dans le terme). C'est
  pourquoi on peut recourir au pléonasme «~maximalisme toxique~» pour
  désigner cette tendance. Jameson Lopp parle aussi de «~puritanisme du
  bitcoin~». Voir Jameson Lopp, \emph{History of Bitcoin Maximalism}, 25
  mars 2023~:
  \url{https://blog.lopp.net/history-of-bitcoin-maximalism/}.}.

Cependant, devant les limites de Bitcoin mises en exergue durant la
guerre des blocs, le phénomène de substitution des autres systèmes
cryptoéconomiques va en s'intensifiant au fur et à mesure du temps.
Ainsi à partir de mars 2017, la dominance du BTC par rapport aux autres
cryptomonnaies décroche du niveau des 85~\% où elle se maintenait
jusqu'alors pour rejoindre les 40~\% en juin. Cela vient en particulier
de l'essor d'Ethereum, qui constitue un moyen simple d'émettre des
jetons programmables sur sa chaîne de blocs. Cette fonctionnalité permet
notamment de lever des fonds en réalisant une prévente de jetons, nommée
\emph{Initial Coin Offering} ou ICO, pour financer des projets dans
lesquels interviendront les jetons en question. Le nombre de levées de
fonds de ce type explose en 2017 -- 2018, à tel point que l'on parle de
«~folie des ICO~». La plus importante d'entre elles, celle d'EOS, lève
4,1 milliards de dollars sur une période d'un an.

En 2019, un autre enthousiasme se dessine~: celui de la finance
décentralisée, appelée \emph{decentralized finance} ou DeFi en anglais.
L'objectif de la DeFi est de reproduire les outils du système financier
traditionnel de manière numérique, décentralisée, ouverte et
transparente. Il s'agit de minimiser l'intermédiation (souvent de
manière imparfaite) intervenant dans l'exécution de diverses opérations
financières~: les échanges, le prêt sur gage (aussi appelé crédit
lombard), la création de produits dérivés, les marchés prédictifs,~etc.
Ce développement a principalement lieu sur Ethereum et est notamment
incarné par l'émergence du protocole Maker qui permet l'existence du
premier stablecoin décentralisé~: le dai. Dans la DeFi, le BTC est
utilisé comme un collatéral de premier choix. À côté de cela, il se crée
une nouvelle mode autour des jetons non fongibles ou NFT (pour
\emph{non-fongible tokens}), un engouement qui finit par toucher le
grand public à partir de 2021.

Cependant, tous ces projets souffrent de failles techniques et humaines
parfois très importantes, de sorte que la critique reste pertinente. À
certains projets, il manque bien souvent la fameuse «~décentralisation~»
qu'ils prétendent posséder. D'autres sont des doublons, qui n'apportent
rien et disparaissent à cause de l'effet de réseau de leurs concurrents.
D'autres enfin sont tout simplement des escroqueries, dans le sens où
leurs promoteurs principaux mentent pour vendre leur jeton. Tout ceci
explique pourquoi le maximalisme du bitcoin subsiste encore à l'écriture
de ces lignes.

\section*{L'intégration
institutionnelle}\label{lintuxe9gration-institutionnelle}
\addcontentsline{toc}{section}{L'intégration institutionnelle}

\markright{L'intégration institutionnelle}

L'ouverture à la finance traditionnelle initiée en 2012 se traduit au
fil du temps par une intégration dans le système légal existant. Cette
tendance est naturelle~: pour que Bitcoin existe, il doit jouir d'une
certaine tolérance de la part de la population et, \emph{in fine}, des
autorités qui «~représentent~» cette dernière.

L'institutionnalisation passe en premier lieu par la réglementation (ou
«~régulation~») du secteur, qui consiste à l'assujettir à un cadre légal
généralement défini et appliqué par l'État. En pratique, il s'agit de
soumettre les acteurs financiers importants à des normes plus ou moins
drastiques. La réglementation est donc synonyme de contrôle, chose à
quoi Bitcoin s'oppose fondamentalement, d'où la tension qui en découle.

Dès les premières années d'existence de Bitcoin, les agences de
renseignement s'y intéressent, aux États-Unis comme en France. Ainsi, en
avril 2011, la CIA invite Gavin Andresen à venir parler de Bitcoin,
chose qu'il fait en juin. Le 9 mai 2012, un rapport interne du FBI sur
Bitcoin fuite sur Internet~: on peut y lire que «~si Bitcoin se
stabilise et gagne en popularité, il deviendra un outil de plus en plus
utile pour diverses activités illégales au-delà du
cyberespace\footnote{Kim Zetter, «~\emph{FBI Fears Bitcoin's Popularity
  with Criminals}~», \emph{Wired}, 9 mai 2012~:
  \url{http://www.wired.com/2012/05/fbi-fears-bitcoin/}.}~». En juillet
2012, un rapport de Tracfin dit que les «~monnaies virtuelles~» posent
un «~risque spécifique en matière de lutte contre le blanchissement des
capitaux et le financement du terrorisme\footnote{Tracfin, \emph{Rapport
  d'activité 2011}, juillet 2012~:
  \url{https://www.economie.gouv.fr/files/files/directions_services/tracfin/Publications/rapports_activite/2011_rapport_FR.pdf}}~».

Ces enquêtes préparent la réglementation financière qui se met en place
à partir de 2013, sous l'effet de la hausse du cours du printemps. C'est
ainsi que, le 18 mars, le FinCEN (\emph{Financial Crimes Enforcement
Network}) publie un document clarifiant sa position sur les monnaies
numériques\footnote{Financial Crimes Enforcement Network,
  \emph{Application of FinCEN's Regulations to Persons Administering,
  Exchanging, or Using Virtual Currencies}, 18 mars 2013,
  \url{https://www.fincen.gov/sites/default/files/shared/FIN-2013-G001.pdf}.}.
Il spécifie dans celui-ci que les plateformes de change sont des
entreprises de services monétaires (\emph{money services business}) et
doivent par conséquent obtenir une licence pour exercer aux États-Unis.

Peu à peu, les normes se durcissent. Les plateformes se mettent à
appliquer une procédure de connaissance du client (appelée en anglais
\emph{Know Your Customer} ou KYC) en imposant une vérification
d'identité pour accéder à leurs services. Elles peuvent également aller
plus loin dans le cadre de la «~lutte contre le blanchiment des capitaux
et le financement du terrorisme~» (LCB-FT).

D'autres réglementations financières s'appliquent aux cryptomonnaies,
comme la taxation des plus-values. En France par exemple, l'utilisateur
est légalement contraint de déclarer ses plus-values réalisées lors des
«~cessions à titre onéreux d'actifs numériques~» et de payer un impôt de
30~\% sur celles-ci, si le total des ventes représente plus de 305~€ sur
l'année\footnote{Code général des impôts, \emph{Article 150 VH bis}, 24
  mai 2019.}.

La réglementation diffère cependant entre les juridictions. Et si
certaines sont clémentes, d'autres le sont beaucoup moins. C'est par
exemple le cas de l'État de New York qui, en 2015, fait passer une
réglementation ultra-restrictive, imposant à une large part des acteurs
de l'écosystème d'obtenir une licence d'exploitation appelée la
«~BitLicence~»\footnote{«~une réglementation ultra-restrictive, imposant
  à une large part des acteurs de l'écosystème d'obtenir une licence
  d'exploitation appelée la "BitLicence"~»~: Davis Polk, \emph{New
  York's Final "BitLicense" Rule: Overview and Changes from July 2014
  Proposal}, 5 juin 2015~:
  \url{https://www.davispolk.com/sites/default/files/2015-06-05_New_Yorks_Final_BitLicense_Rule.pdf}.}.
Dans le même esprit, l'État français fait passer un décret en novembre
2019 pour assujettir les prestataires de services sur actifs numériques
(PSAN) à des conditions strictes\footnote{«~l'État français fait passer
  un décret en novembre 2019~»~: \emph{Décret n° 2019-1213 du 21
  novembre 2019 relatif aux prestataires de services sur actifs
  numériques}~:
  \url{https://www.legifrance.gouv.fr/loda/id/JORFTEXT000039407517/}.}.
Dans les deux cas, cela a pour effet de faire fuir les acteurs locaux
vers des juridictions plus accomodantes.

Outre cette réglementation, le discours des acteurs traditionnels est au
départ ouvertement hostile à la conception originelle de Bitcoin. On
peut le voir en 2014 -- 2015 avec l'apparition du terme
«~\emph{blockchain technology}~», dont le but caché est de nier le côté
rebelle de la cryptomonnaie, en amalgamant l'ensemble des techniques de
consensus distribué au sein d'une même appellation. Cet appel à la
blockchain est popularisé en 2015 par Blythe Masters, une ancienne
opératrice de marché de JPMorgan Chase, connue pour avoir conçu les
contrats de couverture de défaillance (CDS), à l'origine de la crise des
subprimes\footnote{Edward Robinson, Matthew Leising, «~\emph{Blythe
  Masters Tells Banks the Blockchain Changes Everything}~»,
  \emph{Bloomberg}, 1 septembre
  2015~:\url{https://www.bloomberg.com/news/features/2015-09-01/blythe-masters-tells-banks-the-blockchain-changes-everything}.}.
Cependant, le discours s'adoucit au fur et à mesure que les années
passent.

La réglementation amène ainsi de nouvelles contraintes, en opposition
frontale au principe d'absence d'autorisation intrinsèquement lié à
Bitcoin. Mais elle permet peu à peu à de plus gros investisseurs, comme
les fonds d'investissement, d'entrer sur le marché avec de la liquidité,
de légitimer la cryptomonnaie aux yeux du grand public, qui a besoin
d'une approbation officielle pour oser s'y intéresser. Cela pousse ainsi
certains acteurs de la communauté à chercher à coopérer avec le
régulateur.

En 2017, un nouvel engouement spéculatif apparaît, durant lequel le prix
du bitcoin connaît une forte hausse en passant de 1~000~\$ en janvier à
20~000~\$ en décembre. Cette nouvelle bulle attire à nouveau l'attention
des médias. Bitcoin est désormais pris beaucoup plus au sérieux. En
décembre 2017, il entre même à la bourse de Chicago pour faire l'objet
d'un contrat à terme, ce qui correspond à une avancée
notable\footnote{«~En décembre 2017, il entre même à la bourse de
  Chicago~»~: Grégory Raymond, \emph{Le bitcoin débarque à la Bourse de
  Chicago : un moment historique !}, 8 décembre 2017~:
  \url{https://www.capital.fr/crypto/le-bitcoin-debarque-a-la-bourse-de-chicago-un-moment-historique-1259946}.}.

Après la guerre des blocs, le BTC est en effet vu comme une sorte d'or
numérique au sens strict, comme un actif décorrélé des autres marchés
financiers qui constituerait une valeur refuge, à tel point que certains
analystes en viennent à considérer les crypto-actifs comme une nouvelle
classe d'actifs. Pour une part croissante d'utilisateurs, le bitcoin est
même perçu comme une réserve de valeur qui pourrait servir d'étalon au
système monétaire mondial et s'intégrer au système légal des différentes
instances étatiques\footnote{«~servir d'étalon au système monétaire
  mondial~»~: Saifedean Ammous, \emph{The Bitcoin Standard: The
  Decentralized Alternative to Central Banking}, mars 2018.}.

Ce changement de vision cause un conflit croissant entre ceux qui
considèrent le bitcoin comme un intermédiaire d'échange dédié au marché
noir et ceux qui le voient comme une monnaie de réserve. D'un côté, les
partisans de l'aspect libre et sans autorisation de Bitcoin refusent
catégoriquement toute réglementation, considérant cela comme une
compromission des valeurs originelles du projet. De l'autre, les
partisans de la monnaie de réserve perçoivent bien qu'il faut coopérer
avec les autorités en charge pour que les entités les plus fortunées
(fonds d'investissement, grandes entreprises, États) puissent en
acheter. Pour cela, ils parient sur la représentation d'intérêts
(lobbying) et sur le contexte géopolitique (diplomatie).

En 2020, la réaction mondiale à la pandémie de Covid-19 vient accélérer
les choses. Les États occidentaux imposent des confinements durs à leurs
populations et paralysent leurs économies, ce qui provoque le début
d'une crise déflationniste. Les banques centrales réagissent en
conséquence par une injection de liquidités record. Comme en 2008, le
but est de sauver l'économie en créant de l'argent et en l'injectant sur
le marché~: 2~300 milliards de dollars sont émis aux États-Unis et 1~850
milliards d'euros dans l'Union Européenne\footnote{Aux États-Unis, c'est
  le \emph{Coronavirus Aid, Relief, and Economic Security Act} («~CARES
  Act~»), signé en mars 2020, qui a amené cette dépense supplémentaire.
  Il s'agissait d'un programme du Département du Trésor, mais on peut
  supposer qu'il a été financé essentiellement par les «~emprunts~»
  réalisés auprès de la Réserve Fédérale. Du côté de la Banque Centrale
  Européenne, l'injection de liquidités a été mise en place par le
  programme d'achats d'urgence face à la pandémie (PEPP), qui prévoyait
  750 milliards d'euros en mars 2020, auxquels s'ajoutent 600 milliards
  en juin, puis 500 milliards supplémentaires en décembre.}. De plus,
les taux directeurs sont abaissés autour de zéro pour stimuler le
crédit, ce qui contribue à accroître la masse de monnaie scripturale
disponible.

À cause de cette création monétaire démesurée, la menace d'une inflation
élevée reparaît en Occident, alors même qu'elle avait disparu depuis des
décennies. Ce risque pousse les gens à acheter du bitcoin, qui, en tant
qu'actif indépendant de la création monétaire, prend tout son sens. En
particulier, de grandes sociétés cotées en bourse aux États-Unis, ayant
des réserves non négligeables en dollars, rentrent dans le jeu. Le 11
août 2020, l'entreprise Microstrategy, dirigée par Michael Saylor,
annonce ainsi adopter le bitcoin comme principal actif de réserve et
s'être procuré 21~454~BTC, pour un montant d'achat total de 250 millions
de dollars\footnote{«~Microstrategy {[}...{]} annonce {[}...{]} s'être
  procuré 21~454~BTC, pour un montant d'achat total de 250 millions de
  dollars~»~: Business Wire, \emph{MicroStrategy Adopts Bitcoin as
  Primary Treasury Reserve Asset}, 11 août 2020,
  \url{https://www.businesswire.com/news/home/20200811005331/en/MicroStrategy-Adopts-Bitcoin-as-Primary-Treasury-Reserve-Asset}.}.
En octobre, Square, une entreprise américaine spécialisée dans le
paiement mobile cofondée par Jack Dorsey, suit le mouvement et acquiert
4~709 bitcoins. En février 2021, le constructeur automobile de voitures
électriques Tesla, dirigé par Elon Musk, annonce avoir acheté près de
43~000~BTC pour 1,5 milliards de dollars.

De ce fait, un nouvel engouement spéculatif apparaît et le prix du
bitcoin s'envole à nouveau. Alors qu'il oscillait autour des 10~000~\$
depuis 2019, il monte rapidement à l'automne 2020, pour dépasser son
ancien sommet en décembre et atteindre les 64~000~\$ en avril 2021.

\begin{figure}

{\centering \includegraphics{chapters/img/btc-historical-price-20100719-20231130.png}

}

\caption{Évolution du prix du BTC entre le 19 juillet 2010 et le 30
novembre 2023 (source~: buybitcoinworldwide.com).}

\end{figure}%

Mais cette tendance va plus loin que les entreprises et l'engouement
atteint un petit État d'Amérique centrale~: le Salvador. Cette
croissance du prix attire en effet l'attention du jeune président, Nayib
Bukele, qui décide de faire du bitcoin une monnaie ayant cours légal
dans son pays, aux côtés du dollar étasunien. Le président est inspiré
par Jack Mallers, le PDG énergique de Strike, ainsi que par l'expérience
de Bitcoin Beach, un projet de développement d'une économie durable
basée sur le bitcoin autour de la plage d'El Zonte, au sud de la
capitale San Salvador.

La mesure imposant le cours légal est annoncée le 5 juin 2021 par Nayib
Bukele dans une vidéo diffusée lors de la conférence \emph{Bitcoin Miami
2021} et est mise en place le 7 septembre. Elle exige en particulier que
les commerçants acceptent le bitcoin en tant que moyen de paiement et de
règlement de dette, bien que des exceptions existent et que la loi ne
soit pas strictement appliquée.

L'utilisation de Bitcoin apporte théoriquement de nombreux avantages
pour la population~: assurer la réception de fonds transférés depuis
l'étranger, créer une culture de l'épargne, attirer des capitaux, rendre
le pays plus attractif notamment au niveau du tourisme, apporter des
opportunités à l'international, créer un nouvel espoir dans une société
rongée par la pauvreté, la criminalité et la corruption,~etc. De plus,
cette adoption permet au pays, qui repose exclusivement sur le dollar
depuis l'abandon du colón en 2001, d'échapper partiellement au
seigneuriage de la Réserve Fédérale des États-Unis, en accumulant du
bitcoin dans ses réserves de changes. La banque centrale du Salvador se
procure ainsi quelques centaines de bitcoins lors de l'entrée en vigueur
du cours légal.

Cet évènement est applaudi par un certain nombre de bitcoineurs qui y
voient une formidable opportunité et qui se rendent sur place dans les
mois qui suivent. Entre autres, il a pour effet d'asseoir encore un peu
plus la légitimité institutionnelle du bitcoin qui devient une devise
protégée par un État. Cela coïncide avec le sommet de l'épisode
spéculatif, qui amène le prix du bitcoin jusqu'aux 68~000~\$ en novembre
2021.

Cependant, les avis ne sont pas unanimes et de nombreuses critiques sont
émises, que ce soit à l'égard des pratiques autoritaires du
président\footnote{«~critiques {[}...{]} à l'égard des pratiques
  autoritaires du président~»~: Alex Gladstein, «~\emph{The Village and
  the Strongman: The Unlikely Story of Bitcoin and El Salvador}~»,
  \emph{Bitcoin Magazine}, 16 septembre
  2021~:\url{https://bitcoinmagazine.com/culture/the-polarity-of-bitcoin-in-el-salvador}.},
de l'implémentation de Lightning (via l'application Chivo) ou bien de
l'imposition du cours légal elle-même qui va à l'encontre de la
philosophie de Bitcoin\footnote{«~de l'imposition du cours légal
  elle-même qui va à l'encontre de la philosophie de Bitcoin~»~: Ludovic
  Lars, \emph{Cours légal du bitcoin au Salvador, la fausse bonne idée},
  21 septembre 2021~:
  \url{https://www.contrepoints.org/2021/09/21/406280-cours-legal-du-bitcoin-au-salvador-la-fausse-bonne-idee}.}.
Et comme la suite le montrera, l'expérience du Salvador est mitigée~:
l'adoption est loin d'être un succès\footnote{«~l'adoption est loin
  d'être un succès~»~: Nessim Aït-Kacimi, \emph{Salvador : le bitcoin
  peine à s'imposer face au dollar}, 3 mai 2022~:
  \url{https://www.lesechos.fr/finance-marches/marches-financiers/salvador-ladoption-du-bitcoin-comme-monnaie-officielle-ne-seduit-pas-1404569}.},
la population reste méfiante, et la chute du cours (divisé par trois en
l'espace d'un an) décourage toute épargne à moyen terme.

D'une manière générale, Bitcoin acquiert tout de même au fil des années
une légitimité certaine, ce qui pousse ses adversaires à profiter de son
succès pour développer leurs propres projets.

C'est le cas des grandes sociétés avec l'initiative Libra, portée par
Facebook et annoncée le 18 juin 2019\footnote{«~l'initiative Libra,
  portée par Facebook et annoncée le 18 juin 2019~»~: Ludovic Lars,
  \emph{Analyse du projet Libra : quelles répercussions sur Bitcoin~?},
  6 juillet 2019~:
  \url{https://cryptoast.fr/analyse-libra-repercussions-bitcoin/}.}. Il
s'agit d'un projet d'une monnaie numérique adossée à un panier de
devises et d'autres actifs, qui serait géré par un consortium d'une
centaine d'entreprises provenant du monde financier traditionnel (comme
Visa, Mastercard ou PayPal), de la sphère de la cryptomonnaie (comme
Coinbase ou Xapo) ou du secteur technique en général (comme Iliad).

Cette annonce provoque une levée de boucliers, tant du côté du grand
public, qui s'inquiète du potentiel de surveillance que représente ce
système, que des États, qui craignent la perte de leur souveraineté
monétaire. C'est donc tout naturellement que le projet est repoussé par
les régulateurs du monde entier, à commencer par le Congrès américain.
En décembre 2020, Libra prend le nom de Diem, devenant alors un projet
de stablecoin adossé au dollar étasunien. Il finira par être
définitivement abandonné en janvier 2022.

Les instances étatiques s'organisent aussi de leur côté en envisageant
de déployer leurs propres monnaies électroniques dont elles gèreraient
l'émission et les transactions. Durant cette période, on assiste ainsi
au début du développement des monnaies numériques de banque centrale
(MNBC). L'idée, qui s'inspire directement du succès de Bitcoin et des
stablecoins, se veut être une modernisation de la monnaie fiat
traditionnelle, notamment dans le but de favoriser la fluidité des
échanges et l'inclusion financière.

Ce type de monnaie péréniserait les transferts numériques qui forment
déjà l'essentiel des transactions dans les pays occidentaux. Mais elle
offrirait également un levier de contrôle supplémentaire pour le pouvoir
et, ce faisant, introduirait un danger inédit~: celui d'une surveillance
et d'une censure bancaire généralisées. Couplée à une disparition
contrôlée de l'argent liquide, une telle monnaie pourrait de ce fait
constituer la base d'un régime dystopique et totalitaire.

Les projets de MNBC mettent ainsi en exergue l'apport essentiel de
Bitcoin~: un outil de résistance à la censure permettant de rester libre
dans un monde qui ne l'est pas. Le concept découvert par Satoshi
Nakamoto en 2007, déjà très utile, pourrait donc devenir la solution à
un problème qui ne serait que sur le point d'émerger.

\section*{Un déploiement fait de
divisions}\label{un-duxe9ploiement-fait-de-divisions}
\addcontentsline{toc}{section}{Un déploiement fait de divisions}

\markright{Un déploiement fait de divisions}

Bitcoin a donc évolué depuis ses premiers développements sur Mt. Gox et
sur Silk Road. Durant son existence, il a été la source de nombreux
clivages internes concernant la vision qu'il est censé incarner. Ces
clivages ont donné lieu à des conflits qui ont profondément marqué
l'histoire de la cryptomonnaie.

Tout d'abord, l'arrivée d'acteurs financiers dans l'écosystème a mis
l'accent sur la résistance à l'inflation du bitcoin (21~millions) et sur
son caractère déflationniste, aux dépens de sa résistance à la censure,
ce qui a créé une opposition de principe entre les nouveaux
investisseurs et les anciens cypherpunks. Puis, la guerre des blocs a
divisé la communauté à partir de 2015 sur le rôle de la chaîne de blocs,
entre les partisans d'un protocole destiné à effectuer des paiements et
les défenseurs d'un protocole de règlement. Ensuite, l'émergence des
cryptomonnaies alternatives (et notamment d'Ethereum) a fait naître un
enthousiasme pluraliste, mais les pratiques douteuses qui accompagnaient
le lancement des nouveaux projets ont donné naissance à un mouvement de
rejet, le maximalisme du bitcoin. Enfin, une dernière opposition s'est
faite au niveau de l'assimilation institutionnelle, entre d'une part les
personnes désireuses de coopérer avec les autorités en charge et les
régulateurs, et d'autre part celles prônant la confrontation et
dénigrant la soumission et la conformité.

Bitcoin est aujourd'hui pluriel et il est encore en proie à ces tensions
de manière plus ou moins diffuse. Mais ce sont précisément ces tensions
qui lui ont permis de devenir le système organique et antifragile qui a
su se faire une place dans notre société. La découverte de Satoshi
Nakamoto est en effet toujours vivante et continue, bloc après bloc, de
servir ses utilisateurs.

\bookmarksetup{startatroot}

\chapter{Des racines monétaires}\label{ch:monnaie}

\phantomsection\label{enotezch:3}{}

{B}\textsc{i}tcoin constitue un protocole de transfert de valeur qui
gère l'émission et les échanges d'une unité de compte numérique du même
nom, le bitcoin. Comme son nom l'indique (bitcoin est la fusion de
\emph{bit}, chiffre binaire, et de \emph{coin}, pièce de monnaie), le
bitcoin a vocation à être une monnaie. Il a ainsi été présenté comme tel
dès ses débuts, comme l'atteste le titre du livre blanc qui en faisait
«~un système d'argent liquide électronique pair à pair~». C'est pourquoi
il est nécessaire de comprendre l'économie et la monnaie afin de saisir
correctement Bitcoin.

En particulier, le bitcoin est un nouvelle forme de monnaie. Il s'agit
en effet d'une monnaie entièrement numérique qui se base sur un réseau
décentralisé et qui ne nécessite pas d'autorité centrale pour
fonctionner, ce qui constitue un véritable tour de force technique. Ce
modèle original permet au bitcoin d'être résistant à la censure, dans le
sens où il est difficile d'empêcher une transaction d'avoir lieu, et
résistant à l'inflation, dans le sens où il est dur de créer de
nouvelles unités. Grâce à cette proposition de valeur double, il
représente une alternative viable au système monétaire et bancaire
moderne.

Dans ce chapitre, nous nous proposons d'explorer les racines monétaires
de Bitcoin, en expliquant d'abord ce qu'est la monnaie, en décrivant
ensuite la conception qu'en a l'école autrichienne d'économie, avant de
montrer en quoi le modèle du bitcoin est unique et où réside son
utilité.

\section*{Qu'est-ce que la monnaie~?}\label{quest-ce-que-la-monnaie}
\addcontentsline{toc}{section}{Qu'est-ce que la monnaie~?}

\markright{Qu'est-ce que la monnaie~?}

La monnaie est un sujet ardu à appréhender et la conception que s'en
font les gens est souvent floue et inexacte. Pourtant, il s'agit d'un
instrument utilisé massivement dans nos sociétés modernes, caractérisées
par la marchandisation et par la division du travail. Il est donc
crucial d'appréhender cet objet de manière fine et pertinente.

L'importance concrète de la monnaie se retrouve dans la diversité des
termes qui existent pour la désigner en français. D'abord, l'appelation
la plus répandue pour parler de la monnaie est l'argent, si bien qu'on
doit aujourd'hui préciser quand on veut parler du métal précieux.
Ensuite, l'argot regorge de termes variés~: le blé, en référence à la
céréale~; l'oseille, désignant originellement une plante potagère~; le
flouze, venant d'un mot arabe signifiant pièce de cuivre~; le pèze, qui
viendrait peut-être du breton~; le pognon, qu'on échange de main à
main~; la maille et le sou, qui sont les noms d'anciennes pièces de
monnaie. Puis, pour sa forme liquide, on parle de numéraire ou
d'espèces, ou bien de \emph{cash}, un anglicisme venant de l'ancien
français \emph{casse}, qui a donné caisse. Enfin, il y a le mot
«~monnaie~» lui-même, qui provient du latin \emph{moneta}, issu du nom
de temple de Juno Moneta («~Junon la Prévenante~») où se frappait la
monnaie de Rome.

Une monnaie est un intermédiaire d'échange généralement accepté au sein
d'un groupe de personnes donné. Il s'agit d'un outil utilisé dans
l'échange indirect de biens et de services~: une personne \emph{vend}
des biens et des services contre de la monnaie, qui lui sert ensuite à
\emph{acheter} d'autres biens et services.

La monnaie résout en cela le problème de la double coïncidence des
besoins, qui se poserait dans une économie de troc où deux personnes
doivent simultanément désirer le bien de l'autre dans la proportion
souhaitée pour que l'échange puisse se faire d'une manière directe. Par
exemple, si un boulanger souhaitait acquérir une pièce de viande contre
quelques-unes de ses baguettes de pain, il devrait trouver un boucher
désirant obtenir ces baguettes à cet endroit-là, à ce moment-là et pour
ce montant-là. La monnaie est donc un bien intermédiaire que les gens
acquièrent en vue de le céder contre autre chose, et qui fluidifie
grandement leurs échanges.

Ce qui fait qu'un bien est utilisé comme monnaie plutôt qu'un autre,
c'est ce qu'on appelle sa cessibilité\footnote{Le concept de cessibilité
  a été décrit en 1892 par l'économiste autrichien Carl Menger dans son
  essai \emph{On the Origin of Money}. Le terme en allemand est
  \emph{Absatzfähigkeit}, qui désigne, pour une marchandise, la capacité
  à s'écouler facilement, à bien se vendre. Il a été traduit en anglais
  par \emph{saleability} et \emph{marketability}. Il peut aussi être
  traduit par vendabilité ou échangeabilité en français.}, c'est-à-dire
la facilité avec laquelle ce bien peut être échangé sur le marché dès
que son détenteur le désire et en encourant le moins de perte de valeur
possible. Le bien servant de monnaie doit pouvoir être obtenu facilement
sans que cela ne provoque une pénurie de monnaie. Cette propriété est
similaire à la liquidité d'un marché, qui représente la capacité à
acheter ou à vendre rapidement les biens qui y sont cotés sans que
l'opération n'ait d'effet majeur sur les prix. C'est en ce sens qu'on
décrit parfois la monnaie comme \emph{le bien le plus liquide} au sein
d'une économie donnée, et qu'on emploie le terme d'\emph{argent liquide}
en français pour parler de la monnaie physique composée des pièces et
des billets, échangeables facilement et sans contrainte.

La monnaie n'est pas un concept dont les contours sont fixes et rigides.
Un bien peut être plus ou moins une monnaie selon son niveau de
cessibilité dans le groupe humain où il est échangé, de sorte qu'on peut
parler de degré de monétarité ou de liquidité\footnote{L'économiste
  Fritz Machlup parlait de «~degrés de monétarité~» à propos des
  créances en dollar dans le système bancaire européen (Fritz Machlup,
  «~\emph{Euro-dollar creation: a mystery story}~», in \emph{Banca
  Nazionale del Lavoro Quarterly Review}, vol.~23, no.~94, 1970,
  p.~225). De même, Hayek écrivait dans \emph{Pour une vraie concurrence
  des monnaies} en 1976 (p.~93)~: «~Ceci signifie aussi que, bien que
  nous supposions habituellement qu'il existe une distinction claire
  entre ce qui est une monnaie et ce qui n'en est pas -- et la
  législation s'efforce généralement de poser une telle démarcation --,
  cette dichotomie n'existe pas dès lors qu'on considère les propriétés
  qui confèrent à un bien la qualité de monnaie. Ce que nous observons
  est bien davantage un continuum dans lequel des biens dotés de
  différents degrés de liquidité, ou dont les valeurs fluctuent
  indépendamment les unes des autres, se confondent partiellement par le
  degré auquel ils peuvent être utilisés en tant que monnaie.~»}. L'or
et le bitcoin disposent ainsi d'un degré de monétarité moindre que les
devises étatiques en général, mais cela ne les empêche pas d'être
considérés comme des monnaies au sens large. L'or est même mondialement
perçu comme la réserve de valeur par excellence et comme le fondement
historique de la monnaie, ce qui se retrouve dans la culture et en
particulier dans les jeux vidéos.

De plus, la cessibilité d'un bien peut varier selon la situation. Le
dollar n'est pas forcément utile en Europe où l'euro est bien plus
cessible. L'or est un piètre instrument pour les paiements quotidiens,
mais constitue une bonne manière de déplacer de la valeur dans le temps.
Le bitcoin est peu utilisé dans le commerce physique, mais l'est
beaucoup plus sur Internet. Les cigarettes ne servent pas de monnaie
dans la population générale mais ont pu l'être dans certaines prisons.
Le statut de monnaie dépend aussi du contexte.

La cessibilité élevée nécessaire pour que le bien soit sélectionné comme
monnaie se retrouve dans les trois fonctions monétaires classiques,
souvent citées par les économistes et dont l'origine est attribuée au
philosophe Aristote\footnote{«~les trois fonctions monétaires
  classiques, souvent citées par les économistes et dont l'origine est
  attribuée au philosophe Aristote~»~: Dans l'\emph{Éthique à
  Nicomaque}, Aristote énonce ce qui sert de base à ces fameuses trois
  fonctions. Il fait d'abord de la monnaie une unité de compte
  permettant d'évaluer la valeur des choses~:}. Ces fonctions de la
monnaie sont les suivantes~: premièrement, c'est un moyen de paiement,
permettant de régler des échanges de manière directe ou différée
(dette)~; deuxièmement, c'est une réserve de valeur, permettant
d'épargner de la richesse pour l'utiliser plus tard~; troisièmement,
c'est une unité de compte, servant de moyen standard d'exprimer la
valeur des autres biens, sous la forme de prix. Autrement dit, la
monnaie doit posséder une cessibilité qui s'adapte à la fois à l'espace,
au temps et à l'échelle.

De ces trois \emph{fonctions} fondamentales, on dérive usuellement les
\emph{qualités} essentielles de la monnaie, qui sont~:

\begin{itemize}
\item
  La portabilité~: la monnaie doit être facilement transportable pour
  être transmise d'une personne à une autre, ou pour le dire autrement,
  le coût pour la déplacer doit être minimal~;
\item
  La durabilité~: elle doit se conserver dans le temps, ne pas
  s'altérer, ne pas pourrir~;
\item
  La rareté~: sa disponibilité doit être restreinte et ne pas être
  modifiée~;
\item
  La divisibilité~: elle doit pouvoir être scindée en sous-parties plus
  petites~;
\item
  La fongibilité~: chaque unité doit être interchangeable avec une
  autre~;
\item
  La vérifiabilité~: la conformité de la monnaie doit pouvoir être
  vérifiée aisément et rapidement (les espèces doivent être «~sonnantes
  et trébuchantes~»)~;
\item
  La résistance à la censure~: il doit être difficile d'empêcher une
  transaction de se faire (ce qui peut être remis en cause dans les
  solutions numériques)~;
\item
  L'historicité~: la monnaie doit présenter une utilisation ancienne (et
  donc bénéficier de l'effet Lindy\footnote{L'effet Lindy est le fait
    que l'espérance de vie future d'une chose non périssable, telle
    qu'une technique ou une idée, est proportionnelle à son âge actuel.}\footnote{«~l'effet
    Lindy~»~: Le nom de l'effet Lindy a été créé par l'auteur américain
    Albert Goldman, en référence aux restaurants Lindy's à New York où
    il se disait que «~l'espérance de vie d'un comédien de télévision
    est {[}inversement{]} proportionnelle au montant total de son
    exposition sur les ondes~» (Albert Goldman, «~\emph{Lindy's Law}~»,
    \emph{The New Republic}, pp.~34--35, 13 juin 1964~:
    \url{https://gwern.net/doc/statistics/probability/1964-goldman.pdf}).
    Son sens actuel lui a été donné par Benoît Mandelbrot dans son livre
    \emph{The Fractal Geometry of Nature} publié en 1982.}).
\end{itemize}

Ces qualités se sont retrouvées, de manière partielle ou totale, dans
les différentes monnaies qui ont émergé et se sont imposées au cours de
l'histoire.

\section*{Les différentes monnaies}\label{les-diffuxe9rentes-monnaies}
\addcontentsline{toc}{section}{Les différentes monnaies}

\markright{Les différentes monnaies}

On peut regrouper les monnaies qui ont existé en différentes catégories.
Cinq formes plus ou moins distinctes se dégagent ainsi~: la
monnaie-marchandise, la monnaie représentative, le papier-monnaie, la
monnaie-crédit et la monnaie numérique. Ces formes ont toutes des
qualités singulières, qui sont le résultat de l'évolution monétaire
mondiale.

Une monnaie-marchandise est, comme son nom l'indique, une marchandise
qui est amenée à servir d'intermédiaire d'échange au sein d'un groupe
donné. Une marchandise, dans le sens où nous l'entendons ici\footnote{C'est
  le sens qu'on porte au mot \emph{commodity} en anglais.}, est un
produit standardisé, essentiel et courant, dont les qualités sont
parfaitement définies et connues des acheteurs, comme une matière
minérale, un produit agricole ou un produit manufacturé. De ce fait, le
bien utilisé possède originellement une utilité intrinsèque autre que
monétaire~: industrielle, alimentaire ou esthétique.

Les marchandises utilisées comme intermédiaire d'échange ont été
nombreuses au cours de l'histoire de l'humanité. On a pu utiliser des
restes d'animaux comme les coquillages et les ossements, des produits
artisanaux comme les pagnes ou les couteaux, des denrées alimentaires
comme le blé, les épices, les graines de cacao ou le sel\footnote{Le mot
  salaire vient du latin \emph{salarium}, qui désignait la «~ration de
  sel~», puis la «~solde pour acheter du sel~» versés aux soldats
  romains dans l'Antiquité~:
  \url{https://www.lexilogos.com/latin/gaffiot.php?q=salarium}.}, des
produits de l'élevage dont notamment le gros bétail, ou des matières
naturelles comme les pierres ou les métaux.

Toutes ces marchandises possédaient de plus ou moins bonnes qualités
monétaires, mais certaines souffraient de gros défauts, ce qui les
rendaient moins adéquates à l'utilisation comme intermédiaire d'échange
que d'autres. Le bétail avait une très mauvaise portabilité et n'était
pas divisible. Les céréales comme le blé ou le riz étaient peu durables.
La rareté des coquillages pouvait être élevée dans les terres, mais
l'était peu près des côtes. Les produits artisanaux et les bijoux
différaient légèrement les uns des autres ce qui nuisait à leur
fongilibilité.

De manière générale, ce sont les métaux précieux, et tout
particulièrement l'or, l'argent et le cuivre (sous forme de bronze), qui
ont été sélectionnés au fil du temps pour finir par devenir la base
monétaire mondiale. Cette convergence peut s'expliquer par le fait que
ces trois métaux (de symboles chimiques respectifs Au, Ag et Cu)
appartiennent tous les trois au groupe 11 de la classification
périodique des éléments et qu'ils possèdent par conséquent des
propriétés chimiques similaires, dont notamment une grande résistance à
la corrosion et à l'oxydation, et une malléabilité élevée. L'utilisation
de métaux multiples s'explique par leur portabilité imparfaite~: l'or
permet de déplacer beaucoup de valeur, mais n'est pas adapté aux petits
paiements quotidiens, contrairement à l'argent et au cuivre.

Les métaux précieux ont pu être utilisés à l'état brut, sous la forme de
lingots plus ou moins gros. Cependant, ils ont surtout été frappés sous
la forme de pièces de monnaie, sur lesquelles une institution de
confiance (généralement un État) inscrivait sa marque et certifiait le
poids et la teneur en métal. Cette inscription constituait entre autres
choses un certificat, intégré à la monnaie, qui avait pour but de
faciliter l'échange par la non-nécessité de procéder à une vérification
à chaque paiement.

Cette certification peut également être déconnectée de la monnaie,
auquel cas on parle de monnaie représentative. Une monnaie
représentative est une monnaie constituée de certificats, imprimés ou
numériques, qui sont convertibles à vue contre une marchandise de base,
comme de l'or ou de l'argent, auprès d'un tiers de confiance. L'aspect
central d'une telle monnaie est qu'elle est théoriquement adossée à une
réserve intégrale de monnaie de base, détenue par une ou plusieurs
institutions. Les certificats sont par essence des substituts
monétaires, c'est-à-dire des créances juridiquement exécutoires sur un
débiteur pour un montant de monnaie de base déterminé.

L'archétype de la monnaie représentative est le système de l'étalon-or
classique, qui était en vigueur durant la Belle Époque dans le monde
occidental, où la monnaie était constituée de pièces d'or et de billets
de banque convertibles en or. Cependant, avec le temps, la
convertibilité a été progressivement abandonnée et les billets ont été
transformés en une simple monnaie fiduciaire papier.

Une monnaie fiduciaire est une monnaie dont la valeur d'usage est
négligeable par rapport à sa valeur nominale. La valeur initiale de la
monnaie fiduciaire provient de la confiance (\emph{fiducia} en latin)
accordée à d'autres acteurs plutôt que de ses propriétés intrinsèques,
comme c'est le cas des monnaies-marchandises. Cette confiance peut être
placée dans un État, dans une firme ou bien dans une communauté. Elle se
fonde non seulement sur la conviction que le dépositaire n'en dégradera
pas les propriétés (dont notamment la rareté), mais aussi, dans le cas
de l'État, sur l'assurance qu'il fera usage de la violence pour en
contraindre l'utilisation, auquel cas on parle de monnaie fiat (du latin
\emph{fiat}, «~qu'il soit fait~», qui véhicule l'idée de décret).
Contrairement à la monnaie représentative, la monnaie fiduciaire ne
représente pas une marchandise ou une autre monnaie de base~:
\emph{c'est} la monnaie de base.

L'exemple type de monnaie fiduciaire est le papier-monnaie, qui est une
monnaie basée sur un support physique, dont la valeur d'usage est
largement inférieure à la valeur nominale ou faciale indiquée sur le
support. Le support peut être fabriqué à partir de papier, de tissu ou
de plastique (billets) ou bien d'alliages de métaux composés par exemple
de cuivre, de zinc et de nickel (pièces). Le maintien de sa valeur est
garanti par la limitation de la production et la répression du
faux-monnayage~: sans cela, la monnaie deviendrait une
monnaie-marchandise et la valeur d'échange des supports tendrait
rapidement vers leur coût de production, généralement inférieur à leur
valeur nominale.

Cette forme de monnaie a été expérimentée par les États à de multiples
reprises au cours de l'histoire conduisant la plupart du temps à des
inflations dramatiques, comme l'illustrent la tentative d'instauration
d'une monnaie fiat par la dynastie Song entre le \textsc{xi} et le
\textsc{xii}~siècle en Chine\footnote{«~la tentative d'instauration
  d'une monnaie fiat par la dynastie Song entre le XIe et le XIIe en
  Chine~»~: Peter St-Onge, \emph{How Paper Money Led to the Mongol
  Conquest: Money and the Collapse of Song China}, 2017.} ou l'épisode
des assignats durant la Révolution française. Ce n'est que depuis le
\textsc{xx}~siècle et l'abandon de l'étalon-or que ce modèle s'est
généralisé.

Une monnaie-crédit, aussi appelée monnaie scripturale, est une monnaie
qui se rapporte à l'écriture (\emph{scriptura} en latin) d'une dette
dans un registre bancaire. Elle se distingue de la monnaie
représentative par le fait qu'elle n'oblige pas le dépositaire à
conserver le bien représenté en réserve. Les banques sont en effet des
organismes de crédit, et pas des entrepôts de monnaie~: lorsqu'une
personne «~dépose~» des fonds sur un compte en banque, elle prêtent en
réalité son argent à la banque qui «~crédite~» son compte en conséquence
(d'où l'adage «~les dépôts font les crédits~»).

Tout comme la monnaie représentative, la monnaie-crédit est un substitut
monétaire. La monnaie-crédit doit se fonder sur une unité de compte de
base, issue d'une monnaie-marchandise (comme l'or) ou d'une monnaie
fiduciaire (comme le dollar), qui sert à régler la dette lorsqu'elle
arrive à échéance.

Aujourd'hui, le crédit est largement monétisé dans les sociétés
occidentales, comme l'illustrent les moyens de paiement modernes que
sont les chèques, les virements et les cartes bancaires. La monnaie
scripturale compose plus de 90~\% de la quantité en circulation de la
monnaie au sens large.

Une monnaie numérique est une forme particulière de monnaie fiduciaire
dont l'existence repose sur un registre géré informatiquement. Elle se
distingue de la monnaie scripturale par le fait que l'entrée dans le
registre n'est pas une créance en monnaie sur un tiers, mais \emph{est
la monnaie}. La monnaie est ainsi stockée sur une mémoire électronique,
d'où le fait qu'on parle parfois de monnaie électronique\footnote{D'un
  point de vue légal, la monnaie électronique désigne un type spécifique
  de monnaie scripturale. L'article L315-1 du \emph{Code monétaire et
  financier} la définit comme «~une valeur monétaire qui est stockée
  sous une forme électronique, y compris magnétique, représentant une
  créance sur l'émetteur, qui est émise contre la remise de fonds aux
  fins d'opérations de paiement définies à l'article L. 133-3 et qui est
  acceptée par une personne physique ou morale autre que l'émetteur~».
  C'est pouquoi nous préférons ici employer le terme de monnaie
  numérique.}. Une conséquence de la nature particulière de cette forme
de monnaie est qu'elle est généralement programmable, dans le sens où on
peut inscrire les conditions de dépense dans le système informatique qui
la soutient.

Le premier exemple de monnaie numérique est celle qui est gérée de
manière centralisée par une banque centrale. Elle constitue, avec les
pièces et les billets, la base monétaire, qui est aussi appelée
«~monnaie de banque centrale~» ou «~monnaie centrale~». Plus
précisément, il s'agit des avoirs monétaires détenus par les titulaires
de comptes auprès de la banque centrale (c'est-à-dire des banques
commerciales). Ce type de monnaie a permis de ne plus reposer uniquement
sur des supports physiques, qui rendaient le règlement difficile et
risqué. À l'avenir, la monnaie numérique étatique devrait être étendue à
une monnaie numérique de banque centrale (MNBC) disponible pour les
organismes financiers et, probablement, pour les particuliers.

Le second exemple de monnaie numérique est la cryptomonnaie, gérée de
manière décentralisée par un réseau pair à pair, dont l'archétype est le
bitcoin. Il s'agit d'une monnaie numérique de marché dans le sens où son
existence n'est pas dépendante de l'intervention (ou de l'absence
d'intervention) de l'État. C'est la monnaie sur laquelle se focalise cet
ouvrage.

\section*{L'école autrichienne et la valeur de la
monnaie}\label{luxe9cole-autrichienne-et-la-valeur-de-la-monnaie}
\addcontentsline{toc}{section}{L'école autrichienne et la valeur de la
monnaie}

\markright{L'école autrichienne et la valeur de la monnaie}

Puisque Bitcoin est un système monétaire, la compréhension de son
fonctionnement et de ses enjeux passe par la connaissance de l'économie.
Il existe de multiples manières d'aborder le sujet mais nous adoptons
ici la perspective du courant économique dit «~autrichien~», qui est
probablement la plus pertinente pour décrire Bitcoin, car elle a
inspiré, au moins indirectement, sa création et son développement.

L'école autrichienne d'économie, parfois aussi appelée école de Vienne,
est une école de pensée économique créée en Autriche au
\textsc{xix}~siècle autour de la figure de Carl Menger. Elle s'est
initialement développée dans ce pays d'Europe centrale avec des penseurs
comme Eugen von Böhm-Bawerk et Friedrich von Wieser. Après la Première
Guerre mondiale et le démantèlement de l'Autriche-Hongrie en 1918, elle
s'est exportée à l'étranger, dont notamment aux États-Unis, avec des
économistes d'origine autrichienne comme Ludwig von Mises et Friedrich
Hayek (ce dernier ayant reçu le prix «~Nobel~» d'économie en 1974). Par
la suite, elle s'est étendue à des penseurs de toutes origines, dont les
principales figures sont Murray Rothbard, Jesús Huerta de Soto et
Hans-Hermann Hoppe.

L'école autrichienne se caractérise par son approche méthodologique --
l'individualisme méthodologique -- qui se fonde sur la praxéologie,
c'est-à-dire l'étude rationnelle de l'action humaine. Cette méthode est
aprioriste (ou axiomatique) dans le sens où elle repose sur un certain
nombre d'axiomes qui ont trait au comportement humain. Elle part donc de
la partie (l'individu) pour en déduire des conséquences logiques sur le
tout (l'économie). L'école autrichienne s'oppose de ce fait aux écoles
de pensée économiques qui se basent essentiellement sur l'observation et
qui cherchent à modéliser «~mathématiquement~» l'économie, comme le
néokeynésianisme, aujourd'hui majoritaire.

L'école autrichienne a en particulier une analyse fine de la valeur,
c'est-à-dire l'intérêt ou l'importance qu'une personne porte à une
chose.

Plusieurs conceptions de l'origine de la valeur existent. Certains
considèrent que la valeur provient de la terre et de l'activité s'y
rapportant, une thèse défendue par les économistes physiocrates du
\textsc{xviii}~siècle. D'autres postulent que la valeur tire son origine
du travail, à l'instar d'Adam Smith, de David Ricardo, et surtout de
Karl Marx, dont les partisans soutiennent cette théorie depuis le
\textsc{xix}~siècle.

L'école autrichienne d'économie diffère de ces conceptions en prônant
une conception subjective de la valeur. Pour les autrichiens en effet,
la valeur n'est pas un phénomène objectif et dépend du point de vue
individuel. Selon Carl Menger~:

«~La valeur n'est pas inhérente aux biens, elle n'en est pas une
propriété, elle n'est pas une chose qui existe en soi. C'est un jugement
que les sujets économiques portent sur l'importance des biens dont ils
peuvent disposer pour maintenir leur vie et leur bien-être. Il en
résulte que la valeur n'existe pas en dehors de la conscience des
hommes\footnote{Carl Menger, \emph{Principles of Economics}, Ludwig von
  Mises Institute, 2007, pp.~120--121~:
  \url{https://cdn.mises.org/principles_of_economics.pdf}.}.~»

Ainsi, un individu peut accorder une immense grande valeur à un bien (un
tableau par exemple) tandis qu'un autre ne lui en accordera aucune. De
même, la valeur prodiguée peut varier selon le contexte~: une personne
vivant dans le désert ne portera pas le même intérêt à un litre d'eau
que quelqu'un résidant dans une région humide.

La valeur d'un même bien peut être différente aux yeux d'un individu
selon sa consommation antérieure. S'il est affamé, il accordera une
grande valeur à une pomme~; mais à mesure qu'il se sustentera, la valeur
qu'il donnera aux pommes suivantes décroîtra. C'est ce qu'on appelle
l'utilité marginale.

La valeur que l'individu tire d'un bien est appelée la «~valeur
d'usage~» par les économistes. Le sens porté à ce terme est différent
selon les personnes qui l'utilisent. Ainsi, il renvoie souvent à la
valeur d'usage objective, qui est la relation entre une chose et l'effet
qu'elle a la capacité d'entraîner, comme par exemple le pouvoir de
chauffage du bois. Mais le terme peut également, dans le contexte
autrichien, faire référence à la valeur d'usage subjective, qui n'est
pas toujours fondée sur un critère objectif d'évaluation.

L'estimation de la valeur des biens et des services permet à l'individu
de savoir comment orienter sa production et sa consommation. Mais cette
appréciation intervient également dans le commerce~: un échange a lieu
seulement si les deux parties de cet échange donnent \emph{plus de
valeur} au bien économique possédé par autrui. De ce fait, si un bien
appartenant à autrui vaut pour moi plus que quatre pièces d'argent et
que cette autre personne accorde plus de valeur à deux pièces d'argent
qu'à ce bien, alors un échange à un prix de trois pièces d'argent sera
bénéfique pour nous deux. C'est pour cette raison qu'à long terme le
marché libre \emph{crée} de la richesse. Le prix ainsi obtenu dans le
commerce est parfois appelé «~valeur d'échange~».

Même si la valeur est subjective, cela n'empêche pas les êtres humains
d'accorder de la valeur aux mêmes choses. D'abord, en tant qu'ils sont
semblables, ils valorisent naturellement les biens qui leur permettent
de satisfaire leurs besoins physiologiques primaires (eau potable,
nourriture, vêtements, abris, etc.). Ensuite, ils ont tendance à copier
le désir d'autrui pour des choses non nécessaires, conformément à la
nature mimétique du désir\footnote{«~la nature mimétique du désir~»~:
  René Girard, \emph{Mensonge romantique et vérité romanesque}, 1961.},
ce qui a pour effet de créer les engouements et les effets de mode
autour d'objets communs. Enfin, ils accordent de la valeur aux biens
d'ordre supérieur, que ce soient des outils (capital) ou des matières
premières, qui permettent de fabriquer les biens de consommation
désirés.

La monnaie est un cas particulier dans l'analyse de la valeur. Elle
repose sur un phénomène intersubjectif~: une construction psychologique
qui se fait au sein de chaque personne et qui se renforce à mesure
qu'elle s'enracine dans l'esprit des autres. Chacun acquiert de la
monnaie parce qu'il pense qu'il pourra l'échanger contre un autre bien
plus tard, ce qui affermit la conviction des autres que la monnaie peut
effectivement être utilisée. C'est un cercle vertueux conforme à l'effet
de réseau.

De ce fait, même si la valeur est évaluée subjectivement, la valeur de
la monnaie converge nécessairement vers une valeur d'échange objective
commune à tous, qu'on appelle le pouvoir d'achat. Ce pouvoir d'achat
peut varier selon la période et selon la localité, au gré des variations
naturelles du marché et des distorsions causées par l'État. Lorsqu'il
baisse durablement (ce qui se manifeste par une augmentation généralisée
des prix), on parle d'inflation. Lorsqu'il augmente durablement (ce qui
se traduit par une baisse généralisée des prix), on parle de déflation.

Au sein de la valorisation du bien utilisé comme monnaie, il est donc
possible de distinguer deux valeurs mutuellement exclusives~: sa valeur
non monétaire, c'est-à-dire l'utilité alimentaire, industrielle,
esthétique,~etc. que la personne en retire~; et sa valeur strictement
monétaire, qui découle de l'avantage provenant de l'utilisation du bien
comme intermédiaire d'échange. Pour les monnaies-marchandises par
exemple, on peut distinguer la demande dite «~intrinsèque~» de la
demande monétaire~: l'or ne tire pas sa valeur uniquement de sa demande
esthétique (bijoux) et industrielle (microprocesseurs), mais aussi, et
surtout, de sa demande en tant qu'intermédiaire d'échange, qui provient
notamment des banques centrales.

Les économistes autrichiens minimisent le rôle de l'État dans la
création de la monnaie, postulant qu'elle a largement émergé de
l'échange économique, du moins en ce qui concerne sa forme la plus
primitive. Ils s'opposent en cela aux chartalistes et aux partisans de
la théorie monétaire moderne, qui affirment que la monnaie est née de
l'intervention étatique et que sa valeur provient de son emploi pour le
paiement l'impôt\footnote{Le chartalisme (du latin \emph{charta},
  «~papier~», «~lettre~») est une théorie de la monnaie qui a été
  développée par l'économiste allemand Georg Friedrich Knapp en 1905
  dans son ouvrage \emph{Staatliche Theorie des Geldes}. La théorie
  monétaire moderne (\emph{Modern Monetary Theory}) forme un
  néochartalisme.}. Tel que l'écrivait Carl Menger~:

«~L'origine de la monnaie (qu'il faut distinguer des pièces de monnaie,
qui n'en sont qu'une variété) est {[}...{]} tout à fait naturelle, et
elle n'est donc qu'en de très rares circonstances le résultat d'une
influence de la législation. La monnaie n'est ni une invention de
l'État, ni le produit d'un acte législatif, et la sanction d'un tel acte
par l'autorité de l'État est donc étrangère à la notion même de
monnaie\footnote{Carl Menger, \emph{Principles of Economics}, Ludwig von
  Mises Institute, 2007, pp.~261--262.}.~»

Dans cette perspective, la monnaie tire son origine de l'échange entre
les groupes d'individus qui ne se faisaient pas confiance, mais qui
étaient désireux de coopérer. Ainsi les protomonnaies (ou paléomonnaies)
ont émergé, non pas au sein des tribus humaines, dont le fonctionnement
interne reposait largement sur le don et le crédit, mais \emph{entre}
ces tribus. Cela pouvait concerner l'échange simple de marchandises, la
résolution de conflits, le règlement de mariages et le paiement de
tributs\footnote{Nick Szabo, \emph{Shelling Out: The Origins of Money},
  2002~; George Selgin, \emph{The Myth of the Myth of Barter}, 2016~:
  \url{https://www.alt-m.org/2016/03/15/myth-myth-barter/}.}.

Avec la mondialisation progressive de la planète, les protomonnaies ont
subi une sélection~: beaucoup d'entre elles ont disparu au profit de
celles qui satisfaisaient les propriétés d'une bonne monnaie. En
particulier, le bien sélectionné devait être facile à cacher (résistance
à la censure), difficile à produire (rareté) et sa valeur devait pouvoir
être aisément approximée (vérifiabilité). La monnaie a convergé vers les
pièces de métal précieux, le plus souvent d'or et d'argent, de
préférence frappées par une autorité reconnue. Les premières pièces
frappées sont vraisemblablement apparues au \textsc{vii}~siècle avant
Jésus-Christ en Asie Mineure sous l'impulsion des Lydiens\footnote{«~Les
  premières pièces frappées sont vraisemblablement apparues au VIIe
  siècle avant Jésus-Christ en Asie Mineure sous l'impulsion des
  Lydiens~»~: John H. Kroll, \emph{The Coins of Sardis}, 2010~:
  \url{https://sardisexpedition.org/en/essays/latw-kroll-coins-of-sardis}.},
et étaient constituées d'électrum, un alliage naturel d'or et d'argent.
Par la suite, de nombreuses pièces différentes se sont succédées~: la
darique perse, la drachme grecque, le denarius romain, le solidus
byzantin (besant), etc.

L'utilisation de pièces s'est faite pendant des siècles et s'est
généralisée à la planète entière. Néanmoins, cet usage a progressivement
reculé à partir de la Renaissance avec l'émergence des billets de
banque, qui se sont généralisés au cours du \textsc{xix}~siècle grâce à
l'action des États. Le passage au papier-monnaie fiduciaire a eu lieu
durant le \textsc{xx}~siècle avec l'abandon total de toute référence aux
métaux précieux dans le système monétaire en 1971. Nous avons assisté à
une véritable corruption de la monnaie, qui a apporté quelques bénéfices
mais qui a surtout permis aux autorités de davantage profiter de la
création monétaire par le biais de la fameuse «~planche à billets~».

Pour les partisans de la liberté, il est crucial de procéder à une
rédemption de la monnaie\footnote{Bitcoin and Bible Group, \emph{Thank
  God for Bitcoin: The Creation, Corruption and Redemption of Money},
  Whispering Candle, 2020.}, en revenant à ce que les économistes
autrichiens appellent une monnaie saine. Une monnaie saine est une
monnaie librement choisie par le marché qui reste à l'abri des
ingérences coercitives. Tel que l'écrivait Ludwig von Mises dans sa
\emph{Théorie de la monnaie et du crédit} en 1912~:

«~Le principe de la monnaie saine comporte deux aspects. Il est positif
en ce qu'il approuve le choix par le marché d'un intermédiaire d'échange
couramment utilisé. Il est négatif en ce qu'il fait obstacle à la
propension du gouvernement à s'immiscer dans le système
monétaire\footnote{Ludwig von Mises, \emph{The Theory of Money and
  Credit}, Yale University Press, 1953, p.~414~:
  \url{https://cdn.mises.org/Theory\%20of\%20Money\%20and\%20Credit.pdf}.}.~»

Plusieurs projets politiques ont émergé dans le but de rétablir un
système monétaire mondial basé sur une monnaie saine. Le premier était
celui de Mises (et de Rothbard) visant à restaurer l'étalon-or. En
effet, pour Mises, «~une monnaie saine signifie un étalon métallique~»
et l'étalon-or «~rend la détermination du pouvoir d'achat de l'unité
monétaire indépendante des États et des partis politiques~».

Le second projet était celui de Friedrich Hayek, développé plus tard,
qui prônait une concurrence de monnaies (représentatives ou fiduciaires)
qui seraient émises par des banques privées\footnote{Friedrich Hayek,
  \emph{Pour une vraie concurrence des monnaies}, Presses Universitaires
  de France, 2015.}. Cela a inspiré le modèle de la banque libre, dans
lequel les organismes financiers pourraient agir librement sans
intervention d'une banque centrale ou d'une autre instance, un modèle
notamment soutenu par les économistes Lawrence White, George Selgin et
Kevin Dowd.

Aucun de ces deux projets politiques n'a jamais abouti, malgré des
décennies d'évènements démontrant la validité des thèses autrichiennes.
Comme l'a montré l'histoire, le contrôle étatique sur la monnaie s'est
progressivement étendu jusqu'à devenir ce qu'il est aujourd'hui, un
contrôle tendant vers le totalitarisme. Cependant, il existe une
alternative, une solution non pas politique, mais économique~: Bitcoin.

\section*{Une nouvelle forme de
monnaie}\label{une-nouvelle-forme-de-monnaie}
\addcontentsline{toc}{section}{Une nouvelle forme de monnaie}

\markright{Une nouvelle forme de monnaie}

Si l'on parle autant de Bitcoin, c'est qu'il apporte quelque chose de
nouveau, non seulement d'un point de vue technique, mais aussi et
surtout dans une perspective économique. La découverte de ce système par
Satoshi Nakamoto en 2008 représente en effet un véritable bouleversement
dans le domaine monétaire. Le bitcoin constitue une forme de monnaie
inédite~: une monnaie \emph{sui generis} (pour reprendre l'expression de
Jacques Favier), de son propre genre, qu'il est difficile de placer dans
les cases existantes.

Premièrement, il s'agit comme on l'a évoqué d'une monnaie entièrement
numérique. Le bitcoin se base sur un registre de propriété public (la
chaîne de blocs) qui définit la monnaie~: les entrées de ce registre ne
correspondent pas à des créances, comme c'est le cas pour la
monnaie-crédit, mais à la monnaie elle-même.

Deuxièmement, cette monnaie numérique innove par le fait qu'elle ne
nécessite pas de tiers de confiance pour fonctionner. Le contenu du
registre ne dépend pas d'une institution financière comme une banque
centrale, mais d'un ensemble d'acteurs agissant par le biais d'un réseau
distribué d'ordinateurs.

Troisièmement, sa sécurité est assurée de manière économique~: elle ne
repose pas sur un bénévolat altruiste (même s'il joue évidemment un
rôle), mais sur les incitations économiques des différents acteurs
impliqués. Cela donne au système une stabilité à long terme dont n'ont
jamais disposé les différentes monnaies privées qui l'ont précédé.

Ces propriétés permettent au bitcoin de constituer une monnaie
fiduciaire distribuée, dans le sens où il ne possède pas d'utilisation
non monétaire significative et où sa valeur provient de la confiance
accordée à une économie de commerçants plutôt qu'à un tiers. On peut
également le qualifier de monnaie réticulaire (du latin
\emph{reticulum}, «~filet à petites mailles~», «~réseau~») dans la
mesure où la confiance est répartie sur le réseau des nœuds des
commerçants plutôt qu'être concentrée sur un serveur central.

Bien que le bitcoin semble se rapprocher par ses caractéristiques des
monnaies-marchandises\footnote{«~Il n'y a personne pour agir en tant que
  banque centrale ou réserve fédérale afin d'ajuster l'offre monétaire
  au fur et à mesure que le nombre d'utilisateurs augmente. {[}...{]} En
  ce sens, il a plus les caractéristiques d'un métal précieux.~» --
  Satoshi Nakamoto, \emph{Re: Bitcoin open source implementation of P2P
  currency}, 18 février 2009,
  \url{https://p2pfoundation.ning.com/forum/topics/bitcoin-open-source?commentId=2003008:Comment:9562}.},
il ne s'agit aucunement d'une marchandise. Les propriétés de Bitcoin
émergent d'un accord atteint par l'ensemble de ses utilisateurs, pas de
caractéristiques intrinsèques du monde physique comme c'est le cas pour
l'or ou l'argent. Il est ainsi possible de modifier les règles de
consensus du système, même si un tel changement est très difficile.

En réalité, une monnaie est toujours un accord concernant un
intermédiaire mutuellement acceptable dans le commerce. Dans le cas de
la monnaie-marchandise, cet accord converge naturellement vers une
denrée qui est déjà échangée au sein de la société. Dans le cas de la
monnaie fiat, l'agrément est maintenu par un décret étatique disposant
du respect de la population. Dans le cas de Bitcoin, la coordination est
réalisée de manière volontaire autour de règles de consensus
spécifiques.

C'est l'étendue de cet accord qui donne sa force à la monnaie, par effet
de réseau~: son utilité augmente en effet de manière superlinéaire par
rapport à la taille de l'économie l'utilisant. C'est ce qui fait qu'une
monnaie peut difficilement être remplacée par une autre, et c'est aussi
ce qui rend difficile l'altération des règles de consensus du système,
comme nous le verrons dans le chapitre~\hyperref[ch:determination]{11}
sur la détermination du protocole.

L'avantage premier de la monnaie-marchandise n'est pas de disposer d'une
valeur intrinsèque, mais d'exiger un coût infalsifiable pour sa
production\footnote{«~exiger un coût infalsifiable pour sa
  production~»~: Nick Szabo, \emph{Antiques, time, gold, and bit gold},
  28 août 2008~:
  \url{https://unenumerated.blogspot.com/2005/10/antiques-time-gold-and-bit-gold.html}.},
de façon à éviter qu'une création monétaire excessive ne détruise son
pouvoir d'achat. En effet, dans le cas d'une monnaie fiduciaire étatique
ou privée, la détermination de la monnaie se trouve entièrement entre
les mains de l'émetteur, qui peut bénéficier de la situation en créant
plus d'unités à son avantage, surtout s'il jouit d'un privilège légal.

Bitcoin est différent et n'est pas soumis à un tel risque~: son
fonctionnement distribué répartit sa détermination dans l'économie et
l'empêche d'être soumis à l'arbitraire d'un tiers. Cette particularité
lui permet de disposer d'une caractéristique inédite~: une rareté
absolue, découlant d'une quantité fixe d'unités émises selon un
programme prédéfini. C'est d'ailleurs l'un des facteurs qui ont
construit sa renommée~: le fait que l'offre monétaire soit limitée à 21
millions de bitcoins.

La «~valeur intrinsèque~» n'est donc pas un élément essentiel à la
qualité de la monnaie. Le bitcoin, qui est une forme pure de monnaie,
valorisée quasi exclusivement pour son rôle de monnaie, en est la
preuve. Avec les monnaies-marchandises, les propriétés physiques de la
monnaie constituaient un garde-fou contre les interventions privées et
étatiques~; dans Bitcoin, c'est le réseau qui possède cette fonction.

\section*{Bitcoin et le théorème de
régression}\label{bitcoin-et-le-thuxe9oruxe8me-de-ruxe9gression}
\addcontentsline{toc}{section}{Bitcoin et le théorème de régression}

\markright{Bitcoin et le théorème de régression}

Certains économistes autrichiens refusent d'admettre que le bitcoin ait
pu émerger sans avoir de valeur d'usage objective. Ils font pour cela
référence au théorème de régression de Ludwig von Mises, qui stipule que
la valeur d'échange de la monnaie est calculée par rapport à sa valeur
précédente et doit, par régression, être ramenée à sa valeur en tant que
marchandise.

L'essentiel de ce théorème se trouve dans la \emph{Théorie de la monnaie
et du crédit}, un ouvrage publié en 1912, où Mises écrit~:

«~La théorie de la valeur de la monnaie en tant que telle peut faire
remonter la valeur d'échange objective seulement jusqu'au point où elle
cesse d'être la valeur de la monnaie et devient uniquement la valeur
d'une marchandise. À ce point, la théorie doit laisser cours pour toute
investigation ultérieure à la théorie générale de la valeur, qui n'a
alors plus aucune difficulté à résoudre le problème. Il est vrai que
l'évaluation subjective de la monnaie présuppose une valeur d'échange
objective existante, mais la valeur qui a besoin d'être présupposée
n'est pas la même que la valeur qu'il faut expliquer. Ce qui est
présupposé est la valeur d'échange d'\emph{hier} et il est parfaitement
légitime de l'utiliser pour expliquer celle d'aujourd'hui. La valeur
d'échange objective de la monnaie qui s'établit sur le marché
d'aujourd'hui découle de celle d'hier sous l'influence des évaluations
subjectives des individus fréquentant le marché, tout comme celle d'hier
découlait à son tour, sous l'influence des évaluations subjectives, de
la valeur d'échange objective de la monnaie d'avant-hier. Si de cette
façon nous retournons de façon continuelle en arrière, nous devons
arriver à un point où nous ne trouvons plus aucune composante dans la
valeur d'échange objective qui provienne des évaluations basées sur la
fonction de la monnaie comme moyen d'échange commun~; un point où la
valeur de la monnaie n'est rien d'autre que la valeur de l'objet qui est
utile d'une autre façon que comme monnaie\footnote{Ludwig von Mises,
  \emph{The Theory of Money and Credit}, Yale University Press, 1953,
  pp.~120--121.}.~»

Le théorème comporte deux éléments~: la régression et la première
valorisation.

Concernant la régression, le raisonnement se tient~: la valeur attribuée
à la monnaie se base sur sa valeur précédente, de sorte qu'on peut
remonter à une valeur entièrement non monétaire. Il n'y a pas besoin que
cette première valeur se maintienne~: une fois que la monnaie a été
établie, sa valorisation peut reposer uniquement sur la mémoire des prix
précédents.

Cette régression se vérifie historiquement. La valeur de nos billets
fiduciaires actuels en Occident peut être retracée de proche en proche
jusqu'à la valeur des billets en tant certificats or, qui est issue de
la valeur des pièces de monnaies, qui provient elle-même de la valeur de
l'or sous forme brute. Cet or a été valorisé premièrement pour des
motifs ornementaux et religieux avant de commencer à servir
d'intermédiaire d'échange.

Concernant la première valorisation, ce qu'affirme Ludwig von Mises est
plus inexact. Il parle d'une «~marchandise~» (\emph{commodity} en
anglais, \emph{Ware} en allemand) qui est initialement valorisée pour
son utilité «~industrielle\footnote{Ludwig von Mises, \emph{The Theory
  of Money and Credit}, Yale University Press, 1953, p.~110. Voir aussi
  Ludwig von Mises, \emph{Human Action}, Ludwig von Mises Institute,
  1998, p.~405~: \url{https://cdn.mises.org/Human\%20Action_3.pdf}.}~»
(\emph{industrial} en anglais, \emph{industriell} en allemand), donc
d'un produit standardisé et courant dont les qualités sont définies et
connues. À un autre endroit, il s'oppose explicitement à la théorie
lockéenne de l'origine de la monnaie\footnote{«~la théorie lockéenne de
  l'origine de la monnaie~»~: John Locke, \emph{Some Considerations of
  the Consequences of the Lowering of Interest and the Raising the Value
  of Money}, 1691~: «~Car l'humanité, ayant consenti à mettre une valeur
  imaginaire à l'or et à l'argent à cause de leur durabilité, de leur
  rareté et du fait qu'ils ne sont pas très susceptibles d'être
  contrefaits, en a fait par consentement général les gages communs, par
  lesquels les hommes sont assurés, en échange d'eux, de recevoir des
  choses de même valeur que celles qu'ils ont données pour toute
  quantité de ces métaux. C'est ainsi que la valeur intrinsèque de ces
  métaux, qui font l'objet d'un troc commun, n'est rien d'autre que la
  quantité que les hommes en donnent ou en reçoivent.~»} qui, selon ses
termes, fait «~découler l'origine de la monnaie d'un accord général qui
aurait attribué des valeurs fictives à des choses intrinsèquement sans
valeur\footnote{Ludwig von Mises, \emph{The Theory of Money and Credit},
  Yale University Press, 1953, p.~110.}~». Il semble ainsi que Mises
exclut qu'un bien intangible sans valeur d'usage objective puisse
devenir un intermédiaire d'échange sans être adossé à une monnaie
précédente.

Pourtant c'est exactement ce qui s'est passé avec le bitcoin, dont le
succès constitue un contre-exemple limpide au théorème de régression
dans son acception la plus stricte. L'erreur de Mises semble venir de
son biais en faveur des métaux précieux lié à la période durant laquelle
il écrivait, c'est-à-dire le début du \textsc{xx}~siècle. Il était en
effet difficile d'imaginer un système monétaire aussi fantaisiste que
Bitcoin, des décennies avant la révolution technique de l'ordinateur
personnel et d'Internet. Comme l'expliquait Satoshi Nakamoto en août
2010~:

«~Je pense que les critères traditionnels de la monnaie ont été décrits
en partant du principe qu'il y avait tellement d'objets rares en
concurrence dans le monde, qu'un objet bénéficiant de l'amorce
automatique d'une valeur intrinsèque l'emporterait sûrement sur ceux
sans valeur intrinsèque. Mais s'il n'y avait dans le monde rien qui ait
de valeur intrinsèque et qui puisse être utilisé comme monnaie, s'il y
avait seulement des objets rares mais sans valeur intrinsèque, je pense
que les gens opteraient quand même pour quelque chose\footnote{Satoshi
  Nakamoto, \emph{Re: Bitcoin does NOT violate Mises' Regression
  Theorem}, /08/2010 17:32:07 UTC~:
  \url{https://bitcointalk.org/index.php?topic=583.msg11405\#msg11405}.}.~»

Toutefois, malgré cette conception erronée, le théorème de régression
reste valide dans une acception plus large. Pour pouvoir servir
d'intermédiaire d'échange, toute monnaie a dû \emph{nécessairement}
posséder en premier lieu une valeur d'usage non monétaire. Il a par
conséquent fallu que quelqu'un donne une valeur au bitcoin «~pour une
raison ou pour une autre\footnote{Satoshi Nakamoto, \emph{Re: Bitcoin
  does NOT violate Mises' Regression Theorem}, /08/2010 17:32:07 UTC~:
  \url{https://bitcointalk.org/index.php?topic=583.msg11405\#msg11405}.}~»
avant qu'une utilisation monétaire, comme par exemple «~transférer de la
richesse sur une longue distance~», devienne possible.

Il existait donc un problème d'amorçage. Le cryptographe Hal Finney, qui
avait expérimenté les systèmes d'argent liquide numérique au début des
années 1990, en était notamment conscient. Dès les débuts de Bitcoin en
janvier 2009, il écrivait~:

«~Un des problèmes immédiats avec n'importe quelle nouvelle monnaie est
de savoir comment la valoriser. Même en ignorant le problème pratique
lié au fait que quasiment personne ne l'acceptera au début, il est
toujours difficile de trouver un argument raisonnable en faveur d'une
valeur particulière non nulle pour les pièces\footnote{Hal Finney,
  \emph{Re: Bitcoin v0.1 released}, /01/2009 01:22:01 UTC~:
  \url{https://www.metzdowd.com/pipermail/cryptography/2009-January/015004.html}.}.~»

Mais cet amorçage a fini par avoir lieu.

\section*{L'émergence de la valeur du
bitcoin}\label{luxe9mergence-de-la-valeur-du-bitcoin}
\addcontentsline{toc}{section}{L'émergence de la valeur du bitcoin}

\markright{L'émergence de la valeur du bitcoin}

Selon le théorème de régression, le bitcoin a dû posséder une valeur
d'usage non monétaire (objective ou subjective) avant d'être valorisé en
tant qu'intermédiaire d'échange. Au cours des années, différentes
hypothèses de première valorisation ont été proposées pour expliquer
l'émergence de la valeur du bitcoin sur le marché. Examinons-en les
principales, en commençant par les moins plausibles pour finir par les
plus vraisemblables.

Tout d'abord, une hypothèse malheureusement trop souvent citée est la
valeur qui découlerait de l'énergie utilisée pour sa production. Cette
hypothèse provient notamment sur l'estimation de NewLibertyStandard,
qui, à partir d'octobre 2009, vendait et achetait des bitcoins à un taux
basé sur le coût énergétique de sa production personnelle. Il s'agit
essentiellement d'une version revisitée de la théorie de la
valeur-travail des marxistes. Cette explication était déjà critiquée par
Satoshi Nakamoto en février 2010, qui écrivait que le coût de production
était une conséquence du prix, et non pas une cause~:

«~En l'absence d'un marché pour établir le prix, l'estimation de
NewLibertyStandard basée sur le coût de production est une bonne
estimation et un service utile (merci). Le prix de toute marchandise
tend à graviter vers le coût de production. Si le prix est inférieur au
coût, alors la production ralentit. Si le prix est supérieur au coût, il
est possible de réaliser des bénéfices en produisant et en vendant
davantage. Dans le même temps, l'augmentation de la production
accroîtrait la difficulté, poussant le coût de production vers le
prix\footnote{Satoshi Nakamoto, \emph{Re: Current Bitcoin economic model
  is unsustainable}, /02/2010 05:44:24 UTC~:
  \url{https://bitcointalk.org/index.php?topic=57.msg415\#msg415}. -- Il
  a réitéré cette objection en juillet 2010~: «~{[}La monnaie{]} n'est
  pas stable par rapport à l'énergie. Ce sujet a fait l'objet d'une
  discussion. Elle n'est pas liée au coût de l'énergie. L'estimation de
  NLS basée sur l'énergie était un bon point de départ, mais les forces
  du marché domineront de plus en plus.~» (Satoshi Nakamoto, \emph{Re:
  Slashdot Submission for 1.0}, /07/2010 21:31:14 UTC~:
  \url{https://bitcointalk.org/index.php?topic=234.msg1976\#msg1976})}.~»

Certaines personnes ont également suggéré que la valeur proviendrait du
fait que le bitcoin a été échangé contre du dollar, avançant l'idée que
la régression se transmettrait avec cette conversion\footnote{«~la
  valeur proviendrait du fait que le bitcoin a été échangé contre du
  dollar, avançant l'idée que la régression se transmettrait avec cette
  conversion~»~: xc, \emph{Bitcoin does NOT violate Mises' Regression
  Theorem}, /07/2010, 02:09:27 AM~:
  \url{https://bitcointalk.org/index.php?topic=583.msg5984\#msg5984}~;
  AristippusofCyrene, \emph{Bitcoin and the Regression Theorem of
  Money}, 7 décembre 2012~:
  \url{https://voluntaryistreader.wordpress.com/2012/12/07/bitcoin-and-the-regression-theorem-of-money/}.}.
Cependant, le change avec le dollar a toujours été réalisé à taux
variable, selon l'offre et la demande, sans aucune entité pour garantir
un taux fixe. De ce fait, cette hypothèse ne peut pas être valide.

Une autre hypothèse de première valorisation évoquée est celle qui
ferait résider la valeur initiale du bitcoin dans sa capacité à être un
système de paiement\footnote{«~celle qui ferait résider la valeur
  initiale du bitcoin dans sa capacité à être un système de paiement~»~:
  Brice Rothschild, \emph{Théorème de régression et Bitcoin}, 6 novembre
  2013~:
  \url{https://www.contrepoints.org/2013/11/06/145305-theoreme-de-regression-et-bitcoin}~;
  Jeffrey Tucker, \emph{What Gave Bitcoin Its Value?}, 27 août 2014~:
  \url{https://fee.org/articles/what-gave-bitcoin-its-value/}.}. Mais
cet argument doit être écarté car il est circulaire~: nul ne peut payer
en bitcoins si ce dernier n'a de valeur pour personne. De plus, même si
le réseau avait permis de traiter des transferts en dollars ou en euros,
les paiements réalisés n'auraient pas été sécurisés du tout, en raison
du caractère économique de la sécurité minière de Bitcoin, que nous
décrirons dans le chapitre~\hyperref[ch:confirmation]{8}.

Une hypothèse apparentée est que des individus auraient attribué de la
valeur au bitcoin pour sa capacité à servir pour l'horodatage,
c'est-à-dire l'association d'une date et d'une heure à une information
spécifique\footnote{«~des individus auraient attribué de la valeur au
  bitcoin pour sa capacité à servir pour l'horodatage~»~: Simon Gaines,
  \emph{Bitcoin: Intrinsically Worthless?}, 24 avril 2019~:
  \url{https://medium.com/@ahuroad/bitcoin-intrinsically-worthless-5d626645e1c6}.}.
Bitcoin permet en effet d'écrire des données arbitraires sur sa chaîne
de blocs, ce qui garantit leur authenticité notariale, et on peut par
exemple publier l'empreinte d'un document dans une transaction pour
montrer que ce document existait antérieurement à la date de
confirmation de la transaction. Néanmoins, pour que cet ancrage sur la
chaîne possède une quelconque utilité, il faut qu'il soit difficile de
modifier le registre. Puisque la sécurité de Bitcoin est essentiellement
économique, cet usage ne peut donc pas avoir permis de première
valorisation. De plus, cela ne s'est pas passé d'une telle manière~: si
on met de côté le message contenu dans le premier bloc (ayant pour but
d'empêcher l'antidatage du lancement), aucune donnée arbitraire n'a été
inscrite sur la chaîne avant 2011.

S'il faut chercher des raisons à la première valorisation du bitcoin, on
doit les trouver dans les préférences strictement subjectives de
l'individu, non pas dans une hypothétique valeur d'usage objective. Au
vu de l'histoire de Bitcoin durant ses premières années d'existence, on
peut observer qu'il a existé deux raisons principales derrière une telle
valorisation~: la dimension culturelle et l'aspect spéculatif.

La première raison derrière la valorisation initiale est la motivation
culturelle. Selon cette hypothèse, le bitcoin a été un collectionnable,
représentant les principes en lesquels croyaient les individus qui lui
ont porté de l'intérêt. C'est ce qui a poussé les gens à s'en procurer
alors même qu'ils n'en avaient aucun avantage matériel à en retirer.
Cela rejoint en un sens l'idée de valorisation en tant que système de
paiement, à une nuance près~: l'individu ne donne de la valeur au
bitcoin parce que le système est un bon système de paiement à un instant
précis, mais parce qu'il souhaite que ce projet réussisse.

Dans cette logique, l'économiste autrichien Konrad S. Graf parlait en
2013 de «~composantes de la valeur de consommation directe~» qui
seraient «~psychologiques ou sociologiques dans le sens où elles se
rapportent à des facteurs tels que l'attrait inhérent pour les geeks, le
défi professionnel lancé aux spécialistes, la curiosité et le signal
d'appartenance\footnote{Konrad S. Graf, \emph{Bitcoins, the regression
  theorem, and that curious but unthreatening empirical world}, 27
  février 2013~:
  \url{https://www.konradsgraf.com/blog1/2013/2/27/in-depth-bitcoins-the-regression-theorem-and-that-curious-bu.html}.}~».
Dans le même ordre d'idées, Ross Ulbricht, le créateur de la place de
marché Silk Road, expliquait dans un essai rédigé en 2019~:

«~C'est comme par magie que le bitcoin a pu en quelque sorte provenir de
rien et, sans valeur préalable ni décret autoritaire, devenir une
monnaie. Mais Bitcoin n'a pas émergé du vide. C'était la solution d'un
problème sur lequel les cryptographes buttaient depuis de nombreuses
années~: Comment créer une monnaie numérique sans autorité centrale qui
ne puisse pas être contrefaite et qui soit digne de confiance.

Ce problème a persisté si longtemps que certains ont laissé sa
résolution aux autres et ont rêvé à la place de ce que serait notre
avenir si la monnaie numérique décentralisée devenait réalité d'une
manière ou d'une autre. Ils rêvaient d'un avenir où le pouvoir
économique du monde serait accessible à tous, où la valeur pourrait être
transférée n'importe où en appuyant sur un bouton. Ils rêvaient de
prospérité et de liberté, qui ne dépendraient uniquement que des
mathématiques du chiffrement fort\footnote{Ross Ulbricht, \emph{Bitcoin
  Equals Freedom}, 25 septembre 2019~:
  \url{https://rossulbricht.medium.com/bitcoin-equals-freedom-6c33986b4852}.}.~»

C'est donc le rêve d'une monnaie numérique libre qui a en partie motivé
la valorisation initiale du bitcoin. L'objectif était, dès le début, de
créer une monnaie, et le bitcoin a été valorisé pour cette propension.

Bitcoin était notamment conforme à l'idéal libertarien étasunien,
représenté à l'époque par l'homme politique Ron Paul, qui proposait
notamment d'«~abolir la Fed\footnote{«~Ron Paul {[}...{]} proposait
  notamment d'"abolir la Fed"~»~: Ron Paul, \emph{End The Fed}, 2009.}~»
et qui a brigué l'investiture républicaine pour l'élection
présidentielle de 2008 puis celle de 2012. Les cypherpunks,
majoritairement originaires des États-Unis, se rapprochaient largement
de cette idéologie. Satoshi lui-même était conscient de cette proximité,
déclarant dès novembre 2008 que le concept de Bitcoin était «~très
attrayant du point de vue libertarien\footnote{Satoshi Nakamoto,
  \emph{Re: Bitcoin P2P e-cash paper}, /11/2008 18:55:35 UTC~:
  \url{https://www.metzdowd.com/pipermail/cryptography/2008-November/014853.html}.}~».
C'est donc tout naturellement que les premières personnes à apporter de
la valeur au bitcoin ont été ces libertariens, à l'instar de Martti
Malmi, de NewLibertyStandard ou, plus tard, de Ross Ulbricht.

La deuxième raison derrière la première valorisation du bitcoin est la
valeur spéculative qui provenait de sa potentielle utilisation en tant
que monnaie. La promesse de Bitcoin faisait qu'il pouvait être
intelligent de parier là-dessus. En particulier, le bitcoin devait
devenir au fil du temps une monnaie à quantité fixe (21 millions), dont
la rareté était absolue.

Cette caractéristique unique a bouleversé l'imagination des gens. S'il y
avait un nombre limité de bitcoins et que l'utilité monétaire du réseau
augmentait, alors leur prix unitaire subirait théoriquement une forte
hausse. C'est sur quoi se sont basées les engouements spéculatifs
successifs qui ont jalonné l'histoire de la cryptomonnaie.

Cette idée est apparue en janvier 2009, lorsque Hal Finney a estimé dans
un courriel que le prix unitaire du bitcoin pourrait atteindre la
coquette somme de 10~millions de dollars~:

«~Comme expérience de pensée amusante, imaginez que Bitcoin réussisse et
devienne le système de paiement dominant utilisé dans le monde entier.
Alors, la valeur totale de la devise devrait être égale à la valeur
totale de toutes les richesses du monde. Les estimations actuelles que
j'ai trouvées de la richesse totale des ménages dans le monde varient de
100 à 300 milliards de dollars. Avec 20 millions de pièces, cela donne à
chaque pièce une valeur d'environ 10 millions\footnote{Hal Finney,
  \emph{Re: Bitcoin v0.1 released}, /01/2009 01:22:01 UTC~:
  \url{https://www.metzdowd.com/pipermail/cryptography/2009-January/015004.html}.}.~»

Cette estimation était plus que contestable (la monnaie n'est pas censée
représenter toute la richesse du monde), mais elle portait en elle la
notion que chacun pouvait profiter de la hausse du cours.

Par la suite, Satoshi a lui-même utilisé cette logique pour attirer les
utilisateurs potentiels. Il déclarait ainsi le 16 janvier qu'«~il
pourrait être judicieux d'en avoir au cas où cela prendrait~» et que
«~si suffisamment de gens {[}pensaient{]} la même chose, cela
{[}deviendrait{]} une prophétie autoréalisatrice\footnote{Satoshi
  Nakamoto, \emph{Bitcoin v0.1 released}, /01/2009 16:03:14 UTC~:
  \url{https://www.metzdowd.com/pipermail/cryptography/2009-January/015014.html}.}~».
Le 18 février, il écrivait qu'«~à mesure que le nombre d'utilisateurs
{[}croissait{]}, la valeur par pièce {[}augmenterait{]}~» ce qui
constituerait une «~boucle de rétroaction positive\footnote{Satoshi
  Nakamoto, \emph{Re: Bitcoin open source implementation of P2P
  currency}, 18 février 2009~:
  \url{https://p2pfoundation.ning.com/forum/topics/bitcoin-open-source?commentId=2003008:Comment:9562}}~»
pour le système.

Ainsi, ce sont ces deux raisons -- culturelle et spéculative -- qui ont
principalement contribué à la première valorisation du bitcoin. Si les
premiers mineurs ont daigné utiliser leur ordinateur et dépenser de
l'énergie, c'est parce que l'idée de Bitcoin correspondait à leurs
valeurs morales et qu'ils avaient «~le sentiment d'apporter une
contribution bénéfique au monde\footnote{Hal Finney, \emph{Re: Bitcoin
  P2P e-cash paper}, /11/2008 15:24:18 UTC,
  \url{https://www.metzdowd.com/pipermail/cryptography/2008-November/014848.html}.}~»
ou bien parce qu'ils entrevoyaient le potentiel profit\footnote{«~J'ai
  vu le message {[}de Hal{]} et c'est l'une des raisons pour lesquelles
  j'ai démarré un nœud si rapidement. Mes systèmes ne font pas grand
  chose d'autre lorsqu'ils sont inactifs, alors pourquoi ne pas créer
  des BitCoins~? Et s'ils valent quelque chose un jour...~? Bonus~!~» --
  Dustin Trammell, \emph{Re: Bitcoin v0.1 released}, /01/2009 01:14:27
  UTC.}. Les personnes prêtes à accepter du bitcoin contre quelque chose
d'autre l'ont fait pour les mêmes raisons. NewLibertyStandard, qui été
le premier individu à accepter d'échanger des dollars contre des
bitcoins en octobre 2009, était notamment convaincu que Bitcoin était
«~une révolution économique~» et «~la référence de la monnaie
numérique\footnote{Capture du site web de NewLibertyStandard, décembre
  2009~:
  \url{https://web.archive.org/web/20091229132559/http://newlibertystandard.wetpaint.com/}.}~».

Comme l'a écrit un internaute anonyme en 2012, «~les premiers adeptes de
Bitcoin étaient le genre de personnes (du fait de leur intérêt pour les
crypto-monnaies) à considérer Bitcoin comme quelque chose de
beau\footnote{qbg, \emph{Comment: Bitcoin and the Regression Theorem of
  Money}, 8 décembre 2012~:
  \url{https://voluntaryistreader.wordpress.com/2012/12/07/bitcoin-and-the-regression-theorem-of-money/\#comment-135}.}~».
Et c'est cette beauté qui a été à l'origine de la réalité monétaire
qu'on connaît aujourd'hui.

\section*{La monnaie de la
désobéissance}\label{la-monnaie-de-la-duxe9sobuxe9issance}
\addcontentsline{toc}{section}{La monnaie de la désobéissance}

\markright{La monnaie de la désobéissance}

Quand on présente Bitcoin, la question de sa proposition de valeur se
pose immédiatement. Pourquoi Bitcoin~? Qu'est-ce qui le démarque des
monnaies étatiques qui sont ses principales concurrentes~? Quel est
l'intérêt d'utiliser le bitcoin comme monnaie, et pas le dollar ou
l'euro~?

Car le bitcoin est, de par son aspect décentralisé et libre, une moins
bonne monnaie que le dollar ou l'euro dans de nombreux cas~: il est
largement moins accepté, son utilisation est plus difficile, il implique
de payer des frais de transaction, son pouvoir d'achat fluctue davantage
et il pose plus de risques légaux. Toutes ces raisons font que les
monnaies fiat et les solutions centralisées sont (et resteront) plus
efficaces que Bitcoin dans la grande majorité des situations.

Cela ne veut pas dire que Bitcoin est inutile, mais seulement qu'il doit
être appréhendé d'un point de vue particulier. Bitcoin est un système de
monnaie «~sans permission~», \emph{permissionless}, qui peut être
utilisé sans avoir à demander l'autorisation de qui que ce soit. C'est
un argent liquide électronique permettant l'échange direct et
confidentiel entre particuliers, sans avoir recours à un intermédiaire.
Il offre la possibilité d'exercer un contrôle total sur ses unités et de
réaliser des transferts sans crainte d'être observé ou censuré, vers
n'importe quel destinataire, n'importe où dans le monde et à n'importe
quel moment.

Sa proposition de valeur découle de ces simples caractéristiques. En
étant pour ainsi dire incontrôlable, Bitcoin constitue un instrument de
\emph{désobéissance} aux normes sociales et, surtout, au pouvoir
politique. En particulier, il retire aux banques et aux États le pouvoir
de contrôle et de sélection des transactions, qui leur permet de
superviser l'activité économique, ainsi que celui sur l'émission
monétaire, qui leur permet de tirer un revenu de seigneuriage. Bitcoin
constitue donc un concept de monnaie résistante à la censure, dans le
sens où il est difficile d'empêcher une transaction, et résistante à
l'inflation, dans le sens où il est difficile de créer plus d'unités
qu'initialement prévues.

Bitcoin se construit en opposition aux autorités en charge, et s'inscrit
dans la lutte ancienne contre l'asservissement des hommes. Par son
existence, il affirme la primauté du droit naturel sur la loi positive,
la supériorité de la propriété individuelle par rapport à la
collectivité. Il s'intègre par là dans la tradition libérale du droit de
résistance (\emph{jus resistendi}), justifiant la sécession d'un
individu ou d'un groupe d'individus face aux lois injustes, droit
reconnu par deux grandes révolutions du \textsc{xviii}~siècle, qu'ont
été la révolution américaine\footnote{«~droit de résistance {[}...{]}
  reconnu par {[}...{]} la révolution américaine~»~: Thomas Jefferson,
  \emph{Déclaration unanime des treize États unis d'Amérique}, 4 juillet
  1776~: «~Lorsqu'une longue suite d'abus et d'usurpations, tendant
  invariablement au même but, marque le dessein de soumettre {[}les
  hommes{]} au despotisme absolu, il est de leur droit, il est de leur
  devoir de rejeter un tel gouvernement et de pourvoir, par de nouvelles
  sauvegardes, à leur sécurité future.~»} et la révolution
française\footnote{«~et la révolution française~»~: \emph{Déclaration
  des Droits de l'Homme et du Citoyen}, 26 août 1789~: «~Le but de toute
  association politique est la conservation des droits naturels et
  imprescriptibles de l'Homme. Ces droits sont la liberté, la propriété,
  la sûreté et la résistance à l'oppression.~»}.

Il est un outil de désobéissance civile, conception ébauchée par Étienne
de La Boétie au \textsc{xvi}~siècle\footnote{«~désobéissance civile,
  conception ébauchée par Étienne de La Boétie au XVIe siècle~»~:
  Étienne de La Boétie, \emph{Discours de la servitude volontaire},
  1574.}, théorisée par Henry David Thoreau en 1849\footnote{«~théorisée
  par Henry David Thoreau en 1849~»~: Henry David Thoreau, \emph{La
  Désobéissance civile}, 1849.} et mise en pratique par Gandhi dans sa
démarche du \emph{satyāgraha} en Inde et par Martin Luther King dans le
cadre du mouvement des droits civiques contre la ségrégation raciale aux
États-Unis. Il est un message envoyé au souverain terrestre, refusant
ses décrets et affirmant~: «~Je n'utiliserai plus votre monnaie.~»

Bitcoin a été créé en vue de gagner en indépendance individuelle.
Bitcoin est issu du mouvement cypherpunk, un mouvement de désobéissance
technique prônant l'utilisation proactive de la cryptographie sur
Internet afin de protéger la confidentialité et la liberté. Lorsque
Satoshi Nakamoto a lancé le réseau, il l'a fait sans demander
l'autorisation aux autorités en charge, dans le but explicite
d'accroître la liberté. Le 6 novembre 2008, en réponse à une personne
qui lui disait qu'il ne «~{[}trouverait{]} pas de solution aux problèmes
politiques dans la cryptographie~», il déclarait ainsi~:

«~Oui, mais nous pouvons remporter une bataille majeure dans la course
aux armements et conquérir un nouveau territoire de liberté pour
plusieurs années\footnote{Satoshi Nakamoto, \emph{Re: Bitcoin P2P e-cash
  paper}, /11/2008 20:15:40 UTC,
  \url{https://www.metzdowd.com/pipermail/cryptography/2008-November/014823.html}.}.~»

De ce fait, il est naturel que le noyau dur de l'activité construite sur
Bitcoin se situe à la marge de ce qui est généralement approuvé par le
grand public. Bitcoin implique de reprendre sa souveraineté individuelle
contre l'autorité, une démarche qui peut être impopulaire quand les lois
émanent d'une acceptation majoritaire. Il sert donc à combler une niche
de marché plus ou moins grande, dont la taille évolue selon la
proportion de la population qui est prête à désobéir.

Parmi les utilisations centrales de Bitcoin, il y a notamment
l'opposition politique. En effet, ce dernier peut constituer un moyen
alternatif de financement (réception) et de paiement (envoi) pour les
organisations politiques, souvent classifiées à l'extrême-droite ou à
l'extrême-gauche, dont l'intégrité financière est mise à mal par le
pouvoir en place.

L'exemple de Julian Assange et de WikiLeaks est particulièrement
illustratif. En effet, suite aux révélations publiées en 2010 sur les
pratiques de l'armée étasunienne en Afghanistan et en Irak,
l'organisation a subi un blocus financier de la part de Mastercard,
Visa, Western Union, Bank of America et d'autres, qui a fait disparaître
95~\% de ses revenus\footnote{WikiLeaks, \emph{Banking Blockade},
  /10/2011 13:00 UTC, \url{https://wikileaks.org/Banking-Blockade.html}.}.
Cet épisode l'a poussé à accepter les donations en bitcoins en juin 2011
qui, à défaut d'être subtantielles sur le moment, le sont devenues avec
la hausse du cours quelques années plus tard\footnote{«~donations en
  bitcoins {[}...{]} qui, à défaut d'être subtantielles sur le moment,
  le sont devenues avec la hausse du cours quelques années plus tard~»~:
  \url{https://bitinfocharts.com/bitcoin/address/1HB5XMLmzFVj8ALj6mfBsbifRoD4miY36v}}.

On peut également citer le cas du lanceur d'alerte Edward Snowden,
l'ancien employé de la CIA et de la NSA, qui a révélé en 2013
l'existence d'une surveillance de masse par la NSA d'Internet et du
réseau téléphonique aux États-Unis. Celui-ci est depuis poursuivi par
les États-Unis pour «~espionnage, vol et utilisation illégale de biens
gouvernementaux~» et s'est exilé en Russie, dont il a acquis la
nationalité en 2022. Snowden est un soutien solide de Bitcoin, en ayant
fait usage en 2013 pour payer les serveurs qui ont servi à partager les
informations de manière anonyme\footnote{Jamie Crawley, «~\emph{Edward
  Snowden says use crypto, don't invest in it: `Bitcoin is what I used
  to pay for the servers pseudonymously'}~», \emph{Fortune}, 11 juin
  2022~:\url{https://fortune.com/2022/06/11/edward-snowden-says-use-crypto-dont-invest-in-it-bitcoin-is-what-i-used-to-pay-for-the-servers-pseudonymously/}.}.
Il a également fait la promotion de la cryptomonnaie ZCash pour son
modèle de confidentialité, dont il a participé à la cérémonie
d'initialisation en 2016\footnote{«~il a participé à la cérémonie
  d'initialisation {[}de ZCash{]} en 2016~»~: Zcash Media, \emph{Edward
  Snowden: I participated in the Zcash ceremony under the pseudonym of
  John Dobbertin} (vidéo), 28 avril
  2022~:\url{https://www.youtube.com/watch?v=8qSA29vWWds}.}.

Un troisième exemple est celui d'Alexeï Navalny, le principal opposant à
Vladimir Poutine en Russie, fondateur de la Fondation anti-corruption
(FBK) dont les comptes en banques se sont faits geler en 2019 avant
qu'elle soit liquidée en 2021. L'activiste a notamment utilisé Bitcoin
pour se financer depuis 2017 et l'équivalent de plusieurs millions de
dollars ont transité par son adresse\footnote{L'adresse principale
  d'Alexeï Navalny était ``.}\footnote{«~l'équivalent de plusieurs
  millions de dollars ont transité par son adresse~»~:
  \url{https://bitinfocharts.com/bitcoin/address/3QzYvaRFY6bakFBW4YBRrzmwzTnfZcaA6E}.}.
Selon son bras droit, Leonid Volkov, cet apport en bitcoin aurait
représenté 10~\% de leur financement total\footnote{«~cet apport en
  bitcoin aurait représenté 10~\% de leur financement total~»~: Anton
  Zverev et Catherine Belton, \emph{Bitcoin donations surge to jailed
  Kremlin critic Navalny's cause: data}, 11 février 2021~:
  \url{https://www.reuters.com/article/us-russia-politics-navalny-crypto-curren-idUSKBN2AB2GR}.}.
Navalny a été incarcéré en Russie en janvier 2021 et l'était toujours en
novembre 2023.

Une autre utilisation de Bitcoin si situant aussi dans la démarche de
désobéissance est le financement de la plateforme de partage d'articles
scientifiques Sci-Hub, fondée en 2011 par Alexandra Elbakyan, une jeune
femme kazakhe inspirée par les idéaux communistes. Le but du site
(toujours en ligne) est de fournir un libre accès au savoir, par la
partage gratuit d'articles et d'œuvres en tous genre, au mépris des lois
sur le droit d'auteur. En raison de son caractère illégal, la plateforme
a accepté les donations en bitcoins dès ses débuts et a reçu des
centaines de milliers de dollars par ce moyen\footnote{La page de
  donation se situe à l'adresse \url{https://sci-hub.se/donate}.
  L'ancienne adresse
  \texttt{(d\textquotesingle{}après\ une\ capture\ antérieure\ du\ site~:\ \textless{}https://web.archive.org/web/20160202212649/http://sci-hub.la/\textgreater{})\ a\ reçu\ 94,42594975~BTC\ entre\ le\ 03/07/2015\ et\ le\ 14/11/2020.\ Les\ autres\ adresses\ liées\ à\ Sci-Hub\ sont}
  et ``.}. Elle a également eu des soucis récurrents avec PayPal, sa
seule autre source de revenu, qui a clôturé définitivement son compte en
2020.

De manière plus large, Bitcoin est utile dans le contexte géopolitique.
Le système n'est pas lié à une juridiction particulière et n'est pas
concerné par la notion de frontière. Il permet par conséquent d'envoyer
des fonds à l'étranger en échappant aux diverses contraintes et
réglementations en vigueur.

C'est pourquoi il peut servir aux personnes émigrées pour envoyer de
l'argent à leurs proches restés dans leur pays d'origine. Ces transferts
monétaires, appelés \emph{remittances} en anglais, reposent en effet
généralement sur des solutions centralisées comme Western Union et
MoneyGram, qui facturent souvent des frais élevés pour leurs services.

Dans le même ordre d'idées, Bitcoin peut aussi être utilisé pour
contourner les sanctions économiques qu'imposent les différents États à
leurs populations respectives dans le contexte de leurs rapports de
force. Toute l'utilité de la cryptomonnaie a par exemple pu être
constatée suite au début du conflit russo-ukrainien en 2022, lorsque le
bloc occidental a décidé d'instaurer des sanctions financières lourdes à
l'encontre de la Russie. Bitcoin a ainsi pu servir tant du côté des
citoyens russes, qui ne pouvaient plus recevoir de fonds de l'étranger,
que des ressortissants ukrainiens habitant dans les régions occupées.

Ainsi, toute personne vivant sous un régime autoritaire et désireuse de
contourner ses lois ou de se révolter contre l'ordre établi trouvera un
intérêt à Bitcoin. C'est là où se situe le cœur de l'utilisation de
Bitcoin~: dans ce qui est interdit et dans ce qui peut facilement le
devenir.

\section*{La monnaie du marché noir}\label{la-monnaie-du-marchuxe9-noir}
\addcontentsline{toc}{section}{La monnaie du marché noir}

\markright{La monnaie du marché noir}

Bitcoin est un système d'argent liquide électronique qui peut être
utilisé de manière confidentielle, sans requérir d'autorisation, avec
peu de risques de censure. Il est par conséquent particulièrement adapté
pour l'activité économique qui échappe à la supervision de l'État et à
ses prélèvements, c'est-à-dire ce que nous appelons communément le
marché noir.

Le terme «~marché noir~» est apparu dans la langue française au cours de
la Seconde Guerre mondiale sous l'occupation allemande. Il s'agit d'une
traduction littérale du mot allemand du même sens, \emph{Schwarzmarkt},
qui daterait de la Première Guerre. Dès l'origine, le terme désignait un
marché clandestin où la réglementation du commerce était contournée.

L'expression n'était pas employée auparavant car la réglementation
n'était pas assez présente pour donner un nom à ce concept~: l'échange
de marchandises se faisait sur le marché tout court. Seul existait le
terme «~au noir~» qui servait à qualifier l'activité réalisée de façon
non déclarée, cachée au souverain. Mais avec le développement d'une
société de plus en plus policée et réglementée reposant sur des lois
explicites plutôt que des normes sociales implicites, la nécessité de
distinguer le marché réglementé du marché libre s'est fait ressentir,
d'où l'émergence de l'appellation.

La notion de marché noir est floue~: elle peut recouvrir à la fois
l'activité commerciale exercée par des particuliers et celle des
trafiquants de grande envergure, la vente de biens et services légitimes
comme celle de produits issus de l'exploitation criminelle. C'est
pourquoi il est nécessaire de la clarifier. Ici, nous entendons par
marché noir l'économie libre où s'échangent, de manière non réglementée
et non taxée, des biens et des services, légaux et illégaux, qui ne sont
pas des produits directs de l'agression. Cette définition du marché noir
inclut le «~marché gris~» où s'échangent des biens et services légaux
par ailleurs (comme le travail au noir) et exclut le «~marché rouge~» où
se monnaie le crime (comme le meurtre, l'extorsion ou l'esclavage).

Pour désigner l'ensemble des activités marchandes qui échappent au
contrôle de l'État et à l'impôt, il arrive aussi que les gens parlent
d'économie souterraine, d'économie clandestine ou d'économie parallèle.
Cette économie rentre dans le cadre plus large de l'économie informelle,
qui n'est pas nécessairement marchande et qui comprend des activités
comme le travail domestique\footnote{Le travail domestique (cuisine,
  ménage, linge, éducation des enfants, etc.) était auparavant
  principalement assuré par les femmes, avant qu'elles n'abandonnent le
  foyer et que ce travail ne devienne un travail taxé comme un autre. La
  société promue par l'État moderne est avant tout une société
  mercantile, où tout se vend et où tout peut être taxé de la naissance
  à la mort de l'individu.}. Cette dernière représente une part énorme
de l'économie dans les pays en voie de développement.

Le marché noir profite toujours des circonstances restrictives qui
pèsent sur l'économie. Durant la Seconde Guerre\footnote{Joël Drogland,
  «~La France du marché noir~», \emph{La Cliothèque}, 2 mai 2008~:
  \url{https://clio-cr.clionautes.org/la-france-du-marche-noir-1940-1949.html}.},
son succès provenait des grandes pénuries créées par la guerre, par le
contrôle des prix et par les conditions sévères imposées par l'occupant
allemand. La population française était en effet soumise à des
rationnements drastiques (la ration alimentaire officielle d'un adulte
représentait 1100 calories par jour en 1942) et à des prélèvements
outranciers, sous forme de pillages, de taxes ou de réquisitions. Le
recours au marché noir est alors devenu une question de nécessité.
L'économie souterraine a également prospéré au sein des régimes les plus
répressifs~: c'était notamment le cas du marché noir en Union Soviétique
qui représentait une «~seconde économie\footnote{Le nom est tiré de
  l'article «~\emph{The Second Economy of the USSR}~» écrit par Gregory
  Grossman en 1977.}~» sur laquelle reposait la survie du pays.

Mais le marché noir n'est pas qu'un phénomène controversé~; c'est aussi
la pierre angulaire d'une véritable doctrine, appelée l'agorisme.
L'agorisme (terme dérivé du grec ancien \foreignlanguage{greek}{ágorá},
\emph{agora} signifiant «~place de marché~») est une philosophie
politique dérivée du libertarianisme, qui préconise la pratique de
l'économie souterraine comme moyen pacifique de réduire l'influence de
l'État. Cette doctrine a été théorisée dans les années 1970 par Samuel
Edward Konkin \textsc{iii}\footnote{Le terme «~\emph{agorism}~» a été
  forgé par Konkin pour sa présentation au \emph{Free Enterprise Forum}
  de février 1974.}, un canadien vivant aux États-Unis, grand lecteur de
Mises et Rothbard, qui cherchait à radicaliser la vision développée par
l'école autrichienne d'économie. Après avoir pratiqué lui-même sa
philosophie, il l'a mise sur papier en 1980 dans un long essai, le
\emph{Manifeste néo-libertarien}.

L'idée derrière l'agorisme était de joindre le geste à la parole, en
unifiant la théorie libertarienne, fondée sur le principe de
non-agression\footnote{Le libertarianisme se base sur l'axiome de
  non-agression formulé par l'économiste autrichien Murray Rothbard dans
  \emph{For a New Liberty: The Libertarian Manifesto} en 1973~: «~Aucun
  individu ni groupe d'individus n'a le droit d'agresser quelqu'un en
  portant atteinte à sa personne ou à sa propriété {[}...{]},
  ``agression'' étant défini comme le fait de prendre l'initiative
  d'utiliser la violence physique (ou de menacer de l'utiliser) à
  l'encontre d'une autre personne ou de sa propriété.~»}, et la pratique
du marché noir (que Konkin appelait la «~contre-économie\footnote{Le
  terme contre-économie est calqué sur le mot contre-culture, faisant
  référence à la culture alternative des années 60, à laquelle Konkin
  avait participé.}~»), fondée sur la recherche du profit. Il s'agissait
d'une stratégie visant à éradiquer l'agression (dont celle de l'État) de
manière progressive et à créer une société libre («~l'agora~») par le
biais d'actions individuelles intéressées.

L'idée de Konkin était d'appliquer les analyses de Mises et Rothbard à
l'économie souterraine. Il suivait une démarche raisonnée qui consistait
notamment à prendre en compte le risque lié à l'activité illégale (se
manifestant par les amendes, l'emprisonnement et les dommages physiques)
comme un risque entrepreneurial. Il écrivait~:

«~Pourquoi les gens s'engagent-ils dans la contre-économie sans
protection~? parce que le gain par rapport au risque qu'ils prennent est
plus grand que la perte attendue. Cette affirmation est vraie au sujet
de toute activité économique, bien sûr, mais concernant la
contre-économie, elle mérite une attention particulière~:

Le principe fondamental de la contre-économie est d'échanger du risque
contre du profit\footnote{Samuel Edward Konkin \textsc{iii}, \emph{New
  Libertarian Manifesto}, KoPubCo, 2006.}.~»

Ainsi, l'agorisme consistait bien plus à passer entre les mailles du
filet étatique pour améliorer sa vie qu'à jouer les barons de la drogue.
Si les activités les plus controversées et moralement sensibles du
marché noir sont utiles pour illustrer le mécanisme, elles n'en sont pas
moins inenvisageable pour la plupart des gens, qui ont une aversion au
risque élevée. Il ne s'agit pas non plus d'un appel à violer la loi sans
aucune restriction~: le jeu n'en vaut souvent pas la chandelle.

Contrairement aux autres théories politiques qui ne font qu'énoncer des
principes, l'agorisme fournissaient à la fois une fin à viser --
l'\emph{agora}, la société sans État -- et un moyen permettant d'y
parvenir. La cohérence du processus reposait sur les incitations
économiques des individus~: les actions clandestines amélioraient
directement leur vie à court terme, et contribuaient à réduire la place
de l'État à long terme en le privant de son revenu fiscal.

Mais quel rapport avec la monnaie et avec Bitcoin~? La monnaie a
toujours été une préoccupation de Samuel Konkin. L'idée agoriste a été
développée dans les années 1970 aux États-Unis soit précisément au
moment de l'abandon définitif de l'étalon-or et de l'inflation qui s'est
ensuivie. Konkin imaginait alors résoudre le problème par l'utilisation
d'une banque illégale permettant d'échanger de l'or de manière
pratique\footnote{«~Konkin imaginait alors résoudre le problème par
  l'utilisation d'une banque illégale~»~: Samuel Edward Konkin
  \textsc{iii}, \emph{Counter-Economics: From the Back Alleys... To the
  Stars}, KoPubCo, 2018.}, mais cette idée comportait des risques trop
élevés, comme nous le verrons dans le
chapitre~\hyperref[ch:adversaire]{4}.

Le marché noir a ainsi manqué d'une monnaie endogène vraiment efficace.
Depuis les années 70, l'essentiel de l'économie souterraine a fonctionné
grâce aux monnaies fiat disponibles sous forme liquide, et en
particulier aux billets verts américains répandus aux quatre coins du
monde. L'utilisation des espèces est très pratique, car elles sont
acceptées quasi partout, mais elle a pour effet de permettre aux
autorités de bénéficier indirectement des activités illégales grâce au
prélèvement caché du seigneuriage, issu du privilège de création
monétaire. De plus, au vu des récentes évolutions, le contrôle sur la
monnaie a tendance à s'accroître et l'argent liquide étatique est voué à
devenir de plus en plus anecdotique, ce qui représente une menace
existentielle pour le marché noir et la liberté en général.

Les métaux précieux comme l'or et l'argent semblent être des candidats
acceptables. Cependant, ils possèdent deux défauts majeurs~: leur coût
de vérification élevé, ce qui explique le succès de la certification
étatique au moyen des pièces frappées~; et leur portabilité réduite, ce
qui explique le développement du crédit et l'apparition subséquente des
monnaies fiat.

Bitcoin corrige ces insuffisances. Le coût de vérification est celui de
l'entretien d'un nœud, qui peut être réparti entre plusieurs personnes,
et sa portabilité est supérieure, notamment en ce qui concerne les
paiements à distance.

Bitcoin constitue un concept de monnaie numérique particulièrement bien
adapté pour le marché noir grâce à sa résistance à la censure et à
l'absence de seigneuriage. Bitcoin est une manière pour les agoristes de
littéralement «~rendre à César ce qui est à César, et à Dieu ce qui est
à Dieu\footnote{«~rendre à César ce qui est à César, et à Dieu ce qui
  est à Dieu~»~: Mt :15-22.}~» par l'abandon de la monnaie fiat,
étroitement liée au prélèvement de richesse de l'État, et par
l'utilisation exclusive d'une monnaie libre, neutre et décentralisée par
essence.

C'est tout naturellement que les premiers contributeurs à Bitcoin
s'inscrivaient dans cette démarche, voire y souscrivaient complètement.
Martti Malmi, le jeune développeur finlandais qui a aidé Satoshi
Nakamoto au début, avait typiquement ce genre de motivation. En avril
2009, dans une courte introduction présentant Bitcoin sur le forum
anarcho-capitaliste de Freedomain Radio, il écrivait~:

«~Le système est anonyme, et aucun État ne pourrait possiblement taxer
ou empêcher les transactions. Il n'y a pas de banque centrale qui puisse
déprécier la devise avec la création illimitée de nouvelle monnaie.
L'adoption généralisée d'un tel système ressemblerait à quelque chose
qui pourrait avoir un effet dévastateur sur la capacité de l'État à se
nourrir à partir de son bétail\footnote{Martti Malmi, \emph{P2P Currency
  could make the government extinct?}, 9 avril 2009~:
  \url{https://web.archive.org/web/20150927195115/https://board.freedomainradio.com/topic/17233-p2p-currency-could-make-the-government-extinct/}.}.~»

Cette capacité a été illustrée par l'émergence de la place de marché
Silk Road en 2011, qui utilisait le bitcoin comme unique intermédiaire
d'échange. Les vendeurs y postaient des annonces, les acheteurs payaient
et les produits étaient envoyés par voie postale. La plateforme
garantissait la confidentialité des deux parties par l'utilisation du
réseau Tor, et protégeait les acheteurs en intégrant un système de
réputation pour la sélection des vendeurs et un procédé de dépôt
fiduciaire pour arbitrer les échanges. Les produits disponibles sur la
plateforme étaient divers mais il s'agissait essentiellement de drogue
illicite, et notamment de cannabis. Cet «~Amazon de la drogue~» générait
plus d'un million de dollars de volume mensuel en bitcoins à partir de
2012.

Silk Road a été créée par Ross Ulbricht qui a ouvertement admis avoir
été influencé par l'école autrichienne d'économie et par la philosophie
agoriste\footnote{Dread Pirate Roberts, \emph{chat}, 20 mars 2012~:
  \url{https://antilop.cc/sr/users/dpr/threads/20120320-1103-chat.html}.}\footnote{«~Ross
  Ulbricht qui a ouvertement admis avoir été influencé par l'école
  autrichienne d'économie et par la philosophie agoriste~»~: En mars
  2012, Ross Ulbricht a témoigné de son état d'esprit sur le forum de
  Silk Road sous le pseudonyme de Dread Pirate Roberts. Il écrivait~:}.
Sa place de marché en ligne était dans son esprit un moyen de détruire
les structures agressives, et en particulier les cartels de la drogue
dont l'influence nuisait aux trafiquants individuels. De manière
générale, les initiatives comme Silk Road devaient mettre à bas l'État
tel que nous le connaissons. Comme Ross le déclarait dans son entrevue
avec Adrien Chen publiée le 1 juin 2011~:

«~L'État est la principale source de violence, d'oppression, de vol et
de toute forme de coercition. Arrêtez de financer l'État avec l'argent
de vos impôts et dirigez votre énergie productive vers le marché
noir\footnote{Adrian Chen, «~\emph{The Underground Website Where You Can
  Buy Any Drug Imaginable}~», \emph{Gawker}, 1 juin 2011~:
  \url{https://www.gawker.com/the-underground-website-where-you-can-buy-any-drug-imag-30818160}.}.~»

Néanmoins, son orgueil et les risques inconsidérés qu'il a pris l'ont
mené là où on finit généralement lorsqu'on défie frontalement l'État~:
en prison. Il a été arrêté en 2013 et condamné à l'emprisonnement à
perpétuité sans possibilité de libération conditionnelle en 2015.

Silk Road a été un élément essentiel du développement de Bitcoin, le
premier cas d'utilisation majeur de la cryptomonnaie, et son héritage
est toujours présent. De nombreux utilisateurs ont ainsi découvert
Bitcoin soit en recherchant une mise en application des idées
libertariennes (à l'instar de Roger Ver ou de Vitalik
Buterin\footnote{«~Ce sont les médias en ligne agoristes qui m'ont fait
  découvrir Bitcoin. L'agorisme ne nécessite peut-être pas
  d'ordinateurs, mais la technique est l'arme la plus puissante que la
  liberté ait à sa disposition.~» -- Vitalik Buterin, \emph{Re: Bitcoin
  on AgoristRadio.com}, /05/2011 18:36:45 UTC~:
  \url{https://bitcointalk.org/index.php?topic=9177.msg133853\#msg133853}.}),
soit en cherchant à se procurer de la drogue sur Silk Road (comme Peter
McCormack\footnote{Nugget's News, \emph{Peter McCormack -- Bitcoin,
  Addiction \& Podcasts} (vidéo), 19 juillet 2019~:
  \url{https://www.youtube.com/watch?v=3aDMnE6dnHk}.}). Cette
particularité est une incarnation de la vision de Konkin, qui voulait
réconcilier les «~libertariens de bibliothèque~» et les
«~contre-économistes~» de l'économie souterraine.

\section*{La proposition de valeur de
Bitcoin}\label{la-proposition-de-valeur-de-bitcoin}
\addcontentsline{toc}{section}{La proposition de valeur de Bitcoin}

\markright{La proposition de valeur de Bitcoin}

Bitcoin est un concept de monnaie numérique fonctionnant sur Internet,
résistante à la censure et résistante à l'inflation. Il diffère de ses
alternatives que sont le dollar (et les monnaies fiat en général) et
l'or (et les métaux précieux en général), par ses propriétés nouvelles,
liées à son absence de tiers de confiance.

Bitcoin est par essence un outil qui donne à l'individu le pouvoir de
préserver sa liberté et sa richesse. Grâce à la résistance à la censure,
il vise à permettre à quiconque de décider comment il veut dépenser son
argent et, par conséquent, de choisir s'il veut le céder à autrui ou
non. Grâce à la résistance à l'inflation, il permet à ses utilisateurs
de ne pas subir le seigneuriage de l'État, en disposant d'une monnaie
dont l'émission est prédéterminée et dont la quantité maximale est
limitée.

De ce fait, le cœur de l'utilisation de Bitcoin se situe aux confins de
ce qui est autorisé par les puissances de ce monde. Il est
principalement, et restera, une monnaie de désobéissance, utilisée dans
l'économie parallèle, pour acheter des biens et des services et pour
conserver de la valeur à long terme. Avec l'inéluctable numérisation de
la monnaie, il pourrait devenir un «~territoire de liberté~» équivalant
à ce qui est aujourd'hui offert par l'argent liquide physique. Il a,
après tout, été présenté au monde comme un «~argent liquide
électronique~».

\bookmarksetup{startatroot}

\chapter{La nécessité de décentralisation}\label{ch:adversaire}

\phantomsection\label{enotezch:4}{}

{B}\textsc{i}tcoin donne aux individus une propriété entière et
souveraine sur leur argent. D'une part, il leur permet de l'envoyer à
n'importe quelle personne, n'importe où dans le monde, à n'importe quel
moment et quel que soit le motif, en empêchant le gel des transactions.
D'autre part, il leur permet de préserver pleinement leur pouvoir
d'achat en interdisant la création arbitraire d'unités supplémentaires.
Par cette double proposition de valeur, Bitcoin s'inscrit dans un
rapport antagoniste avec l'État, qui revendique une prérogative
exclusive sur la monnaie et un contrôle inquisitorial sur son
utilisation.

Souvent présenté comme l'institution qui possède le «~monopole de la
violence légitime~», l'État se caractérise plutôt par le transfert de
richesse non consenti qu'il assure. Ce transfert se manifeste de deux
manières principales que sont l'impôt, c'est-à-dire le prélèvement
direct du contribuable, et le seigneuriage, à savoir la spoliation
indirecte de l'épargnant par l'émission de monnaie. Et ces moyens de
prélèvement reposent tous deux sur le contrôle monétaire~: le premier
est facilité par la surveillance et le blocage des transactions~; le
second est issu de la maîtrise sur la définition de l'unité de compte.

De ce fait, l'État ne tolère aucune concurrence sérieuse en matière
monétaire. En tant qu'outil de liberté, Bitcoin remet ce contrôle en
question et constitue en ceci une menace du point de vue étatique. C'est
la raison d'être de son architecture distribuée, fondamentale dans sa
conception.

Dans ce chapitre, nous étudierons d'abord le transfert de richesse
organisé par l'État et ses conséquences. Puis, nous verrons comment le
contrôle monétaire a pu se renforcer au cours de l'histoire, et comment
il menace de s'accroître à nouveau par l'intermédiaire de la monnaie
numérique de banque centrale. Enfin, nous expliquerons en quoi les
systèmes alternatifs centralisés ne sont pas viables et pourquoi la
décentralisation constitue une nécessité.

\section*{L'État et l'impôt}\label{luxe9tat-et-limpuxf4t}
\addcontentsline{toc}{section}{L'État et l'impôt}

\markright{L'État et l'impôt}

Du point de vue sociologique, l'État se définit classiquement comme une
autorité souveraine qui s'exerce sur un territoire déterminé et sur un
peuple qu'elle représente officiellement. Il en ressort trois éléments
qui le caractérisent~: un pouvoir sur une population, un territoire et
une certaine acceptation.

Premièrement, la nature de l'État est d'utiliser la force physique~: son
existence repose sur la contrainte, imposée par la violence ou la menace
de violence, par l'intermédiaire d'une police et d'une armée. Cette
violence s'exerce sur un groupe de personnes sous sa domination,
appelées des sujets ou des citoyens, dont il restreint la liberté
naturelle, le plus souvent au moyen de lois et de décrets délimitant les
interdictions. En particulier, il lève un impôt (terme venant du latin
\emph{impōnere}, «~charger~», «~faire peser sur~») qui est, dans les
faits, un prélèvement de richesse ne disposant pas du consentement
individuel\footnote{Même si l'on considère que l'impôt constitue un
  «~mal nécessaire~», ou qu'il se justifie par l'«~intérêt général~» ou
  par la «~démocratie~», il n'en demeure pas moins, par nature, un
  transfert de richesse non consenti, c'est-à-dire un vol pour le dire
  crûment.}.

Deuxièmement, l'autorité de l'État s'exerce au moyen de la domination
sur un territoire donné. Cette caractéristique lui permet de consolider
son prélèvement au sein de frontières déterminées~: puisque les êtres
humains ont besoin de la terre (ou de la mer) pour exercer leurs
facultés, le contrôle du territoire facilite énormément leur soumission.
C'est la domination sur la terre qui explique l'organisation féodale (du
latin médiéval \emph{feodum}, «~fief~») de l'État dans les sociétés
agraires.

L'impôt est aujourd'hui levé grâce à une multitude de contrôles réalisés
par l'État. Ces contrôles passent en premier lieu par la surveillance
financière, qui s'applique notamment dans le domaine bancaire~: les
banques et autres institutions financières sont responsables devant
l'administration fiscale, à qui elles doivent transmettre les
informations douteuses concernant leurs clients. Cette surveillance est
facilitée par un certain nombre de lois, comme par exemple les
restrictions sur l'utilisation de l'argent liquide. À l'intérieur du
territoire, la collecte de l'impôt se base sur le contrôle fiscal,
c'est-à-dire l'ensemble des méthodes d'intervention permettant
d'examiner les déclarations, de les confronter à la réalité des faits et
de réhausser, le cas échéant, les bases d'imposition. La préservation du
revenu fiscal de l'État repose également sur l'entrave des flux de
richesse sortant du territoire, par le biais des contrôles douaniers et
des contrôles de capitaux. Tous ces contrôles sont étroitement liés à la
question de la censure financière, traitée dans le
chapitre~\hyperref[ch:censure]{9} du présent ouvrage.

La collecte de l'impôt sur le territoire se fait principalement par
l'intermédiaire des acteurs économiques établis, même lorsqu'ils ne sont
pas directement taxés. En France, la taxe sur la valeur ajoutée,
prélevée sur la vente des biens et services, est ainsi payée par le
client, mais versée par le commerçant qui doit l'ajouter à son prix de
vente. De même, de nombreuses charges fiscales sont retenues à la source
par les entreprises mais payées par leurs employés, comme la
contribution sociale généralisée et (aujourd'hui) l'impôt sur le revenu.
Ce recours à l'impôt «~indirect~» permet de réduire le nombre de
personnes à surveiller et de rendre le prélèvement «~indolore~» pour
ceux qui le paient réellement.

Troisièmement, l'État bénéficie d'une \emph{large acceptation} de la
part de la population, qui peut aller de l'approbation active à la
résignation passive. C'est cet élément qui le différencie des groupes
criminels organisés qui ne bénéficient pas en général d'une telle aura.
Bien que temporaire et partielle, cette acceptation est à l'origine de
l'idée de «~contrat social~», qui n'a rien d'un réel contrat juridique,
mais qui forme une constatation de la situation existante. L'État tire
ainsi son nom du fait qu'il incarne l'état actuel du rapport de force au
sein de la société.

L'acceptation de l'État assure la pérennité du prélèvement fiscal, en
faisant en sorte que son pouvoir n'ait pas pas besoin d'être maintenu
par la force pure. L'État affirme sa légitimité en prétendant
représenter les intérêts du peuple qui vit sur son territoire, au moyen
d'idéologies diverses, de façon à rendre les contributeurs dociles et à
limiter les révoltes.

En particulier, l'État revendique un monopole sur la violence
défensive\footnote{Comme l'écrivait Max Weber en 1919~: «~L'État est
  l'institution qui possède, dans une collectivité donnée, le monopole
  de la violence légitime.~» -- Max Weber, \emph{Le savant et le
  politique}, 10/18, 2002.}, et garantit le maintien de l'ordre
intérieur (par l'intermédiaire de la police) et la défense contre les
ennemis extérieurs (par le biais de l'armée). Ce service réel n'est pas
réalisé de manière purement altruiste~: l'intérêt de l'État est de
défendre les forces productives contre les perturbations internes et
externes, tout en les empêchant d'organiser elles-mêmes leur propre
protection, dans le but de stabiliser son revenu fiscal. Ce monopole
s'apparente ainsi à un chantage à la protection accepté par la
population comme un moindre mal.

Puisque l'impôt est la pierre angulaire de la construction étatique, son
paiement possède un caractère sacré. C'est ce qui explique pourquoi son
évitement est systématiquement dénigré, y compris lorsqu'il est légal.
C'est aussi la raison derrière la répression sévère de la résistance
fiscale, qui passe notamment par la limitation de la liberté
d'expression dans le domaine. En France, il est par exemple interdit
d'appeler à arrêter de payer l'impôt, sous peine d'une amende de 3~750~€
et d'un emprisonnement de six mois\footnote{L'article 1747 du \emph{Code
  général des impôts} dispose~: «~Quiconque, par voies de fait, menaces
  ou manoeuvres concertées, aura organisé ou tenté d'organiser le refus
  collectif de l'impôt, sera puni des peines prévues à l'article 1er de
  la loi du 18 août 1936 réprimant les atteintes au crédit de la nation
  {[}c'est-à-dire de deux ans de prison et d'une amende de 9~000
  euros{]}.}.

La capacité de prélèvement de l'État n'est cependant pas infinie.
D'abord, il existe une pondération subtile entre le niveau effectif du
prélèvement de richesse et la destruction économique induite, qui varie
selon la préférence temporelle des bénéficiaires\footnote{Voir
  Hans-Hermann Hoppe, «~La préférence temporelle, l'État et le processus
  de décivilisation~», in \emph{Démocratie, le dieu qui a échoué}, 2020,
  Éditions Résurgence.}. Ensuite, le niveau de prélèvement dépend du
niveau d'acceptation de la population et il existe nécessairement un
point au-delà duquel l'accroissement du taux de prélèvement se traduit
par un amoindrissement du prélèvement total, phénomène illustré par la
courbe de Laffer\footnote{«~illustré par la courbe de Laffer~»~: Arthur
  B. Laffer, \emph{The Laffer Curve: Past, Present, and Future}, 1 juin
  2004~:
  \url{https://www.heritage.org/taxes/report/the-laffer-curve-past-present-and-future}.}.
Enfin, la capacité de prélèvement dépend des moyens techniques possédés
par l'appareil étatique, notamment en ce qui concerne la surveillance,
et des outils à la disposition de la population pour résister
fiscalement, ce qui inclut bien entendu Bitcoin.

L'État est donc l'incarnation de la violence institutionnalisée, qui a
pour fonction primaire d'assurer un transfert de richesse non consenti
individuellement. Plus qu'un groupe de personnes identifié, il doit être
bien plus compris comme un ensemble d'actions réalisées par des
individus dans un contexte spécifique. Ainsi, nous nous référerons ici à
l'État au singulier en tant que concept pouvant se manifester dans des
instances particulières plus ou moins indépendantes, mais toujours selon
les mêmes principes.

\section*{La monnaie et le
seigneuriage}\label{la-monnaie-et-le-seigneuriage}
\addcontentsline{toc}{section}{La monnaie et le seigneuriage}

\markright{La monnaie et le seigneuriage}

Si nous parlons de l'État, c'est parce qu'il entretient une relation
étroite avec la monnaie. Comme nous l'avons vu dans le
chapitre~\hyperref[ch:monnaie]{3}, celui-ci s'est arrogé une prérogative
de plus en plus grande sur la définition de l'intermédiaire d'échange,
en garantissant d'abord sa certification, puis en contrôlant aujourd'hui
son émission. Cet état de fait a conduit certains théoriciens, dont
notamment les chartalistes, à adopter une approche fiscale de la monnaie
qui fait du paiement de l'impôt la source originelle de la valeur de
cette dernière\footnote{«~approche fiscale de la monnaie~»~: Tcherneva
  Pavlina, «~\emph{Chartalism and the Tax-Driven Approach to Money}~»,
  \emph{A Handbook of Alternative Monetary Economics}, ch.~5, 2007.}.

Ce lien étroit s'explique par le fait que l'État en tire un bénéfice
considérable, le seigneuriage\footnote{Le nom «~seigneuriage~» est issu
  de l'ancien français \emph{seignorage}, qui désignait le privilège de
  battre monnaie au Moyen Âge, généralement réservé aux seigneurs
  féodaux.}, qui est l'avantage financier direct qui découle de
l'émission de monnaie pour l'émetteur. Le seigneuriage constitue en
effet la seconde source principale de revenu de l'État, aux côtés de
l'impôt, permettant de financer la dépense publique. Il est par nature
une spoliation indirecte du détenteur de monnaie, beaucoup plus efficace
à court terme que le prélèvement direct du contribuable. L'endettement,
souvent cité comme un troisième moyen, n'est en réalité qu'un impôt
différé ou un seigneuriage déguisé.

Le seigneuriage est ainsi le fait de tirer profit d'une industrie
particulière~: la production de monnaie. Il est le résultat de quatre
mesures légales fondamentales que sont la contrefaçon légalisée, le
monopole sur la production, l'imposition du cours légal et la suspension
des paiements\footnote{Jörg Guido Hülsmann, \emph{The Ethics of Money
  Production}, Ludwig von Mises Institute, 2008.}. Comme dans le cas de
l'impôt, ces actions sont largement acceptées dans la mesure où elles
émanent de la puissance publique.

La contrefaçon légalisée consiste à faire circuler de la monnaie dont le
certificat ne correspond pas à ce qui est attendu par la population
générale. Typiquement, il s'agit de faire circuler des pièces possédant
une teneur moindre en métal que les pièces similaires existantes ou bien
de billets représentatifs dont la monnaie de base de garantie est en
réalité conservée de manière fractionnaire.

Le monopole sur la production de monnaie est le privilège exclusif
d'émission monétaire accordé à une entité, la dédouanant de toute
concurrence et lui permettant de vendre sa monnaie à un prix supérieur à
ce qu'il aurait été sur le marché libre. Ce privilège est habituellement
délégué à une entité contrôlée par l'État comme un hôtel de la Monnaie
ou une banque centrale.

Le cours légal est l'obligation imposée aux acteurs économiques
d'accepter une monnaie à la valeur nominale dictée par l'État.
L'imposition du cours légal peut être restreinte et ne concerner que les
paiements différés (c'est le sens de la \emph{legal tender}
anglo-saxonne) ou bien être plus large et se rapporter à tous les
paiements (comme c'est souvent le cas en Europe
continentale\footnote{Par exemple, le cours légal de l'argent liquide
  est imposé en France par l'article R642-3 du code pénal~: «~Le fait de
  refuser de recevoir des pièces de monnaie ou des billets de banque
  ayant cours légal en France selon la valeur pour laquelle ils ont
  cours est puni de l'amende prévue pour les contraventions de la 2e
  classe.~»}). Il s'agit d'avantager la monnaie dont l'État dispose d'un
monopole d'émission en la surestimant par rapport aux monnaies
concurrentes.

Ce cours légal a pris plusieurs formes au cours de l'histoire. Il se
retrouvait dans le bimétallisme (double étalon) où le ratio entre l'or
et l'argent était fixé à une valeur arbitraire, avantageant l'un ou
l'autre des deux métaux. Il se manifestait pendant la période de
l'étalon-or classique lorsque les certificats représentatifs devaient
être échangés au même cours que le numéraire. Il était également
institué par l'étalon de change-or qui imposait que les monnaies
nationales des pays secondaires aient cours à un taux déterminé par
rapport à la livre ou au dollar, le taux du marché étant maintenu
artificiellement haut par le contrôle des changes. Aujourd'hui, le cours
légal se définit par rapport au cours sur le marché des changes et il se
manifeste par l'interdiction de proposer systématiquement un prix
différent selon l'intermédiaire d'échange utilisé.

La suspension des paiements consiste, pour une banque centrale, à
interrompre momentanément le remboursement de ses clients, auquel cas on
parle de cours forcé. Dans le cas des billets représentatifs, la
possibilité de recourir à cette mesure légale permettait de ne pas
conserver l'intégralité de l'or en réserve, en empêchant les retraits
dans le cas d'une chute de confiance.

Depuis l'Antiquité jusqu'au \textsc{xix}~siècle, la monnaie était
constituée de pièces de métaux précieux, essentiellement de l'or, de
l'argent et parfois du cuivre. Il était donc impossible pour le
souverain de créer de nouvelles unités à partir de rien. Cependant, il
pouvait dévaluer les pièces existantes en réduisant leur teneur en
métal.

À l'époque romaine, le \emph{denarius} d'argent (qui a donné son nom au
denier) a été dévalué à de nombreuses reprises, lentement d'abord, avant
de voir sa teneur en métal être réduite à l'excès au cours de la crise
du \textsc{iii}~siècle. Le seigneuriage retiré a permis à l'Empire
romain de continuer à financer sa domination, sans pour autant continuer
son expansion territoriale. De même, tous les souverains européens ont
procédé à ce type de manipulation au cours du Moyen Âge. Ces pratiques
ont notamment été observées par le philosophe chrétien Nicolas Oresme au
\textsc{xiv}~siècle\footnote{«~Ces pratiques ont notamment été observées
  par le philosophe chrétien Nicolas Oresme~»~: Benoît Malbranque,
  \emph{Oresme et les dangers de la dévaluation monétaire}, 14 juillet
  2017~:
  \url{https://www.institutcoppet.org/oresme-et-les-dangers-de-la-devaluation-monetaire/}.}.

De plus, cette manipulation des pièces de monnaie a un effet autrement
malencontreux~: celle de chasser de la circulation la monnaie
sous-estimée, qui se retrouve thésaurisée ou exportée à l'étranger. Ce
phénomène porte le nom de loi de Gresham, loi économique faisant
référence à Sir Thomas Gresham, un grand marchand et financier anglais
du \textsc{xvi}~siècle, qui avait établi le lien causal entre la
disparition des meilleures pièces d'argent de la circulation et les
mesures légales du pouvoir de l'époque\footnote{La loi de Gresham a été
  formalisée par l'économiste écossais Henry Dunning Macleod dans ses
  \emph{Elements of Political Economy} publiés en 1858.}. Cette loi,
couramment résumée par l'expression proverbiale «~la mauvaise monnaie
chasse la bonne~», stipule qu'en l'existence d'un taux de change légal
fixe entre deux monnaies, la mauvaise monnaie (c'est-à-dire celle qui
est surestimée) a tendance à remplacer la bonne monnaie (c'est-à-dire
celle qui est sous-estimée) en tant que moyen de paiement dans le
commerce. Cette loi s'applique également, dans une moindre mesure, pour
la monnaie représentative et pour la monnaie fiat.

Le développement des banques modernes à partir de la Renaissance a
provoqué l'apparition des billets de banque convertibles à vue, bien
plus pratiques pour déplacer de la valeur dans l'espace. Le pouvoir a
repris cette invention à son compte en monopolisant l'émission des
billets et en faisant des instruments supposés représentatifs.

Dans ce cas, le seigneuriage consiste à créer plus de billets qu'il n'y
a de métal précieux en réserve, c'est-à-dire réaliser une fraude
financière. Mais une contrainte subsiste~: une grande partie du métal
doit être conservée, sous peine de voir les créanciers vider les
coffres. L'État peut choisir de suspendre les paiements (ce qu'il a fait
dans l'histoire), mais une telle mesure s'accompagne alors d'une baisse
drastique de confiance dans les billets par rapport au métal qu'ils sont
censés représenter. C'est pourquoi le régime de l'étalon-or est resté
relativement stable au niveau monétaire. Il a cependant pavé la voie à
un régime autrement plus inflationniste~: celui du papier-monnaie.

Le seigneuriage a acquis un rôle majeur avec l'apparition du
papier-monnaie, qui est une monnaie fiduciaire basée sur un support
physique. Le seigneuriage consiste alors juste à créer plus de billets
dont l'usage est imposé sur le territoire, ce qui est considérablement
plus efficace que la dévaluation des pièces en métal précieux et la
fraude sur les billets représentatifs.

Dès l'origine, le papier-monnaie a permis de financer les projets
pharaoniques des États. Il est notamment indissociable de la guerre
moderne. L'émission des \emph{greenbacks} américains, désignés comme
tels à cause de l'encre verte utilisée pour imprimer le verso, entre
1861 et 1865 a permis de soutenir la guerre de Sécession aux États-Unis.
De même, la Première guerre mondiale a été majoritairement financée par
la création monétaire et par la réduction de la dette liée à
l'inflation\footnote{«~la Première guerre mondiale a été majoritairement
  financée par la création monétaire et par la réduction de la dette
  liée à l'inflation~»~: Vincent Duchaussoy et Éric Monnet, \emph{La
  Banque de France et le financement direct et indirect du ministère des
  Finances pendant la Première Guerre mondiale~: un modèle français~?},
  \url{https://books.openedition.org/igpde/4132}.}.

Pour éviter une fuite trop importante vers des monnaies concurrentes
jugées plus fiables, les États ont également mis en place des mesures de
contrôle des changes qui réglementaient l'achat et la vente de devises
étrangères. Ces mesures servaient à maintenir la valeur de la monnaie à
un niveau artificiellement haut, alors même que la confiance dans
celle-ci s'effondrait. Le prétexte invoqué était souvent la «~lutte
contre la spéculation~».

Toutefois, même si l'émergence du papier-monnaie constituait une manne
inédite, la capacité de profiter de la monnaie n'est pas devenue
illimitée~: la production de pièces et de billets fiduciaires et la
lutte contre la contrefaçon privée ont un coût incompressible réduisant
le seigneuriage. C'est en partie pourquoi il existe aujourd'hui une
volonté de remplacer cet argent liquide par une monnaie intégralement
numérique.

Enfin, tout comme l'impôt, le seigneuriage repose sur l'acceptation de
la population, qui est soutenue en particulier par une limitation de
l'expression. En France, il est ainsi interdit de faire douter le public
de la solidité de la monnaie, celui qui désobéit à cette loi s'exposant
à une amende de 9~000~€ et à deux ans de prison\footnote{L'article 1 de
  la loi du 18 août 1836 réprimant les atteintes au crédit de la nation
  dispose~: «~Sera puni de deux ans de prison et d'une amende de
  9~000~euros quiconque, par des voies ou des moyens quelconques, aura
  sciemment répandu dans le public des faits faux ou des allégations
  mensongères de nature à ébranler directement ou indirectement sa
  confiance dans la solidité de la monnaie, la valeur des fonds d'État
  de toute nature, des fonds des départements et des communes, des
  établissements publics et, d'une manière générale, de tous les
  organismes où les collectivités précédentes ont une participation
  directe ou indirecte.~»}.

\section*{L'inflation des prix}\label{linflation-des-prix}
\addcontentsline{toc}{section}{L'inflation des prix}

\markright{L'inflation des prix}

La principale conséquence du seigneuriage est l'inflation des prix. Ce
terme, qui vient du latin \emph{inflatio} signifiant «~gonflement~»,
«~enflure~», «~dilatation~», désigne la perte du pouvoir d'achat de la
monnaie qui se traduit par une augmentation générale et durable des
prix. Il s'agit ainsi d'un phénomène qui touche l'économie dans son
ensemble à long terme.

Contrairement à ce qui est parfois supposé, toute hausse des prix n'est
pas une manifestation de l'inflation. À cause de son caractère durable,
l'inflation est par nature structurelle et non conjoncturelle. Les
mesures temporaires imposées par un État peuvent faire augmenter les
prix, mais cet effet ne constitue pas en soi de l'inflation.

L'inflation des prix est un phénomène qui a pu être observé dans de
nombreuses économies. Elle était déjà présente à l'époque de l'Empire
romain, dont elle a accompagné l'effondrement à partir du
\textsc{iii}~siècle, en culminant sous le règne de l'empereur
Dioclétien. Elle a également pu être observée dans nos économies
modernes suite aux deux guerres mondiales, dans les années 1970 et plus
récemment dans les années 2020.

L'inflation peut provenir d'une augmentation générale de la demande ou
d'une diminution de l'offre de biens et de services. Elle peut
théoriquement être le fait de plusieurs facteurs comme l'inflation
monétaire, la raréfaction de l'énergie, la destruction de richesse par
la guerre ou la fuite des capitaux. En pratique, c'est-à-dire dans le
cas d'une économie croissante, pacifiée et indépendante, l'inflation des
prix à long terme est, en règle générale, une conséquence de l'inflation
monétaire.

L'inflation monétaire est l'excédent de production de monnaie par
rapport à la production naturelle sur le marché libre\footnote{Cette
  définition nous vient de Guido Hülsmann pour qui l'inflation est
  «~l'augmentation de la quantité nominale d'un moyen d'échange au-delà
  de la quantité qui aurait été produite sur le marché libre~». -- Jörg
  Guido Hülsmann, \emph{The Ethics of Money Production}, p.~85.}. Elle
résulte de la manipulation de la monnaie par le pouvoir en place, qui
cherche à en tirer profit par le biais du seigneuriage. Il arrive ainsi
régulièrement que l'État sacrifie le pouvoir d'achat de sa monnaie à
long terme pour obtenir un revenu à court terme, par exemple dans le
contexte d'une crise militaire, politique ou sanitaire.

Le phénomène de l'inflation est souvent mal appréhendé car il n'est pas
le phénomène uniforme et instantané que l'on a tendance à se
représenter. Une injection de monnaie dans l'économie exerce un effet
progressif et différencié sur les prix au fur et à mesure que la monnaie
se diffuse par les échanges. C'est ce qu'on nomme l'effet Cantillon, qui
a été observé en 1730 par l'économiste physiocrate Richard Cantillon
dans son \emph{Essai sur la Nature du Commerce en Général} où il
déclarait qu'«~une augmentation d'argent effectif {[}causait{]} dans un
État une augmentation proportionnée de consommation, qui
{[}produisait{]} par degrés l'augmentation des prix\footnote{Richard
  Cantillon, \emph{Essai sur la Nature du Commerce en Général}, McMaster
  University Archive for the History of Economic Thought, 1755.}~».

Cet effet Cantillon s'applique à l'espace et au temps. La monnaie
produite peut se retrouver dans des espaces spécifiques (les aires
urbaines par exemple), se concentrer dans certaines régions du monde
(hors du territoire national notamment) ou se concentrer dans certains
secteurs économiques particuliers (comme la finance). La propagation
peut être ralentie par certaines pratiques, comme le paiement de
salaires mensuels. Cependant, l'effet de la hausse de la quantité finit
par se répercuter progressivement sur l'ensemble de l'économie.

Entretemps, les personnes proches de l'émission monétaire
s'enrichissent. Le producteur de monnaie la dépense en apportant une
demande supplémentaire, quitte à proposer un prix supérieur pour obtenir
le bien désiré. Le commerçant qui la reçoit, devenu momentanément plus
riche, réitère cette dépense plus généreuse auprès d'un autre
commerçant. Et ce phénomène se poursuit jusqu'à atteindre les confins de
la société économique, de telle sorte que les personnes les plus
éloignées de l'émission monétaire s'en retrouvent les plus lésées.

L'inflation des prix est donc une manifestation différée de l'inflation
monétaire résultant du seigneuriage excessif de l'État. Si le phénomène
s'emballe, celui-ci peut conduire \emph{in fine} à la destruction de
l'unité de compte\footnote{«~Si le phénomène s'emballe, celui-ci peut
  conduire \emph{in fine} à la destruction de l'unité de compte~»~: Les
  cas d'hyperinflation dans l'histoire des siècles précédents sont
  nombreux. Ils coïncident la plupart du temps avec les premières
  expériences de papier-monnaie durant une période troublée par la
  guerre ou par la révolution. Nous pouvons notamment citer les exemples
  de l'assignat révolutionnaire français qui a connu l'hyperinflation
  entre 1793 et 1795, du \emph{papiermark} de la république de Weimar
  dont la valeur s'est effondrée entre 1922 et 1924, du rouble russe
  dont le pouvoir d'achat a été annihilé entre 1917 et 1922 et du yuan
  nationaliste chinois qui s'est écroulé entre 1946 et 1949. Dans
  l'histoire récente, on peut faire mention de l'hyperinflation du
  rouble soviétique entre 1991 et 1993, de celle du dollar zimbabwéen de
  2000 à 2009 et de l'inflation galopante du bolivar vénézuélien qui
  sévit depuis 2016.}, ce qu'on appelle une hyperinflation\footnote{«~ce
  qu'on appelle une hyperinflation~»~: La Commission européenne définit
  une économie hyperinflationniste par les caractéristiques suivantes~:
  1) la population en général préfère conserver sa richesse en actifs
  non monétaires ou en une monnaie étrangère relativement stable~; 2) la
  population en général apprécie les montants monétaires, non pas dans
  la monnaie locale, mais dans une monnaie étrangère relativement
  stable, les prix pouvant être exprimés dans cette monnaie~; 3) les
  ventes et les achats à crédit sont conclus à des prix qui tiennent
  compte de la perte de pouvoir d'achat attendue durant la durée du
  crédit, même si cette période est courte~; 4) les taux d'intérêt, les
  salaires et les prix sont liés à un indice de prix~; et 5) le taux
  cumulé de l'inflation sur trois ans approche ou dépasse 100~\%. Elle
  est ainsi liée à la perte des fonctions de réserve de valeur et
  d'unité de compte de la monnaie. -- Voir IAS 29~: «~Information
  financière dans les économies hyperinflationnistes~», 29 octobre
  2018~:
  \url{http://www.focusifrs.com/menu_gauche/normes_et_interpretations/textes_des_normes_et_interpretations/ias_29_information_financiere_dans_les_economies_hyperinflationnistes}.}.
Dans ce cas, l'inflation n'est plus nourrie par la production de monnaie
(qui peine à suivre le rythme), mais par la fuite de la valeur vers
d'autres monnaies jugées plus fortes ou vers des biens liquides.

\section*{Les banques centrales}\label{les-banques-centrales}
\addcontentsline{toc}{section}{Les banques centrales}

\markright{Les banques centrales}

De nos jours, le système monétaire mondial repose sur le modèle de la
banque centrale. Une banque centrale est une institution qui possède un
monopole d'émission de la monnaie ayant cours légal sur un territoire
donné. Tous les États du monde ont recours à une telle institution pour
gérer leur monnaie fiat.

La banque centrale est le résultat de la prise de contrôle sur
l'activité bancaire par le pouvoir central. La banque moderne,
consistant à faire commerce de la monnaie et du crédit, s'est développée
au cours de la Renaissance. Elle se basait sur deux innovations
majeures~: le dépôt à vue et la lettre de change, qui sont devenus le
compte courant et le billet de banque, lorsque le crédit s'est
popularisé en tant que substitut monétaire.

Le pouvoir a peu à peu centralisé cette activité en créant des banques
publiques qui bénéficiaient d'avantages par rapport à leurs concurrentes
privées. Ces banques publiques étaient initialement cantonnées à une
ville. C'était par exemple le cas de la Banque du Rialto créée à Venise
en 1587, de la banque d'Amsterdam fondée en 1609 avec la bénédiction des
Provinces-Unies ou bien de la Banque de Stockholm créée en 1656 par
Johan Palmstruch. Puis, des banques nationales ont été formées, comme la
banque des États du royaume de Suède (plus tard renommée en
\emph{Sveriges Riksbank}) qui a été fondée en 1668, la Banque
d'Angleterre qui a vu le jour en 1694, ou encore l'éphémère Banque
générale, qui s'est développée en France de 1716 à 1720 sous la
supervision de l'écossais John Law\footnote{«~l'éphémère Banque
  générale, qui s'est développée en France de 1716 à 1720~»~: L'exemple
  de la Banque générale, devenue Banque royale en 1719, est fascinant
  car cette dernière a contribué à créer l'une des premières bulles
  financières mondiales de l'histoire. Le système de Law était en effet
  étroitement lié à la Compagnie du Mississippi, ayant pour but de de
  prendre en charge la dette à court terme de l'État accumulée par le
  défunt Louis \textsc{xiv} et de développer le potentiel commercial de
  la Louisiane française en émettant des actions de la Compagnie. --
  Antoin E. Murphy, «~John Law et la bulle de la Compagnie du
  Mississippi~», \emph{L'Économie politique}, 2010/4 (n° 48), p.~7-22~:
  \url{https://www.cairn.info/revue-l-economie-politique-2010-4-page-7.htm}.}.

Les banques centrales ont émergé de ces banques publiques en acquérant
le monopole d'émission des billets. La Banque d'Angleterre a acquis ce
privilège grâce au \emph{Bank Charter Act} de 1844. En Prusse, le décret
du 11 avril 1846 a permis à la Banque royale de Prusse de bénéficier
d'un monopole d'émission sur le même modèle que le Royaume-Uni. La
Banque de France, fondée en 1800 par Napoléon Bonaparte, a vu son
privilège d'émission (à l'origine limité à Paris) être étendu à
l'ensemble du territoire en 1848. Aux États-Unis, la banque centrale n'a
été créée que tardivement avec la fondation de la Réserve Fédérale en
1913.

Contrairement à ce qui est communément affirmé par les responsables
politiques, la banque centrale n'est pas indépendante de l'État. Elle
repose sur la force de l'État pour assurer son monopole et l'application
du cours légal~; et ce dernier dépend de la banque centrale pour
prélever un seigneuriage. La banque centrale n'est ainsi qu'une
institution qui joue un rôle dans l'appareil étatique.

Cette implémentation des banques centrales a mené à une installation
durable de la monnaie fiat papier. D'abord, les billets étaient adossés
à une quantité de métal précieux, et notamment à l'or durant la période
de l'étalon-or classique de 1873 à 1914. Ensuite, leur convertibilité
directe a été interrompue, au début de manière temporaire au cours
d'épisodes de cours forcé plus ou moins longs, puis de manière
définitive à partir de la Première Guerre mondiale en 1914 pour l'Europe
et de la Nouvelle donne de Franklin Roosevelt en 1933 pour les
États-Unis. Enfin, toute référence à l'or dans le système monétaire
mondial a été abandonnée en 1971, avec l'abrogation de l'étalon de
change-or de Bretton Woods par Richard Nixon.

La banque centrale possède aujourd'hui un rôle prépondérant. Elle
intervient largement dans l'économie par sa politique monétaire. Ses
missions principales sont la limitation de l'inflation des prix, qui se
traduit souvent par un objectif de hausse de l'IPC à 2~\% par an, et le
prêt en dernier ressort\footnote{Le rôle de prêteur en dernier ressort
  de la banque centrale a été théorisé au cours du \textsc{xix}~siècle.
  -- Henry Thornton, \emph{An Enquiry into the Nature and Effects of the
  Paper Credit of Great Britain}, 1802~; Walter Bagehot, \emph{Lombard
  Street: A Description of the Money Market}, 1873.}, consistant à
fournir de la liquidité aux banques en difficulté lors d'un resserrement
du crédit. Elle peut avoir d'autres missions secondaires comme le
soutien à la baisse du chômage.

Pour réaliser ces missions, trois leviers d'action lui sont généralement
octroyés~: la production de la monnaie physique, le rachat de titres sur
les marchés financiers et l'influence sur l'émission du crédit par le
biais de taux directeurs. Tout d'abord, la banque centrale peut avoir
pour tâche de fabriquer le papier-monnaie. Mais cette tâche peut
également être déléguée. La Fed délègue cette tâche au Bureau de la
gravure et de l'impression, la BCE aux banques nationales des
États-membres de l'Union Européenne.

Ensuite, la banque centrale peut se rendre sur les marchés financiers
afin d'y intervenir. Elle réalise traditionnellement des opérations
d'open market, c'est-à-dire des achats et des ventes de titres, en
particulier d'obligations publiques (bons du Trésor), sur le marché
interbancaire. Les politiques monétaires non conventionnelles lui
permettent également de mener des opérations d'assouplissement
quantitatif (QE), plus longues et plus agressives, ce qui permet
d'apporter de la liquidité pour soutenir l'économie en cas de crise.
Mais ces achats permettent surtout de financer la dette de l'État~:
puisque la taille du bilan est strictement croissante, on peut
considérer qu'une partie de ces achats représente un pur seigneuriage.

Enfin, la banque centrale influence l'émission du crédit bancaire, à
l'aide de ses taux directeurs. Ces taux, appelés différemment selon les
pays, sont généralement au nombre de trois~: le taux de refinancement,
qui est taux usuel pour lequel les banques commerciales peuvent obtenir
de la monnaie centrale, le taux du prêt marginal, qui est le taux de
prêt à courte échéance servant à obtenir des fonds en cas d'urgence, et
le taux de rémunération des dépôts, le taux d'intérêt payé par la banque
centrale pour la conservation de monnaie centrale en réserve. Le
premier, le plus important, sert à limiter la création de crédit
bancaire~; le deuxième, qui est nécessairement le plus élevé, permet de
maintenir le système bancaire en place en cas de crise grave~; le
troisième, qui doit être le moins élevé, a pour rôle de décourager ou
d'encourager le prêt commercial à court terme. La banque centrale fixe
également un niveau de réserves obligatoires. Ces taux ne sont pas des
taux d'intérêt issus du marché et peuvent donc être négatifs.

Le fonctionnement des taux directeurs permet à la banque centrale, et
donc à l'État, de prélever un seigneuriage en prêtant à intérêt la
monnaie centrale aux banques commerciales. Ces dernières peuvent ensuite
prêter ces fonds à leurs emprunteurs qui, par leurs actions économiques
comme l'investissement et la consommation, les diffuse dans l'économie
toute entière.

Dans ce fonctionnement pyramidal, les banques commerciales tirent aussi
profit de leur position. Le système bancaire, formé comme un cartel,
bénéficie d'un privilège d'émission de crédit et est protégé des
conséquences économiques par la banque centrale, qui constitue un
prêteur en dernier ressort, et par le Trésor, qui peut procéder à un
renflouement externe\footnote{Les clients des banques commerciales sont
  encouragés à garder leurs fonds en banque en étant partiellement
  couverts contre le risque de faillite par un système de garantie des
  dépôts, géré par exemple par le Fonds de Garantie des Dépôts et de
  Résolution (FGDR) en France et par la \emph{Federal Deposit Insurance
  Corporation} (FDIC) aux États-Unis.}. Ce privilège lui permet, dans la
mesure où la banque centrale l'autorise, de prélever elles-aussi un
seigneuriage sur le crédit qu'elles émettent. De plus, cette situation
encourage l'expansion du crédit et stimule les cycles
économico-financiers haussiers et baissiers, qui ont des effets
terriblement néfastes sur l'économie comme le malinvestissement et les
crises récessionnistes.

Ce système banco-monétaire a été largement critiqué au cours des
décennies qui ont suivi l'abandon définitif des accords de Bretton Woods
en 1971. Satoshi Nakamoto s'est lui-même joint à la critique en février
2009 lorsque, soucieux d'amener les gens à s'intéresser à Bitcoin, il a
mis en avant les conséquences de ce fonctionnement bancaire~:

«~Le problème fondamental de la monnaie conventionnelle est toute la
confiance nécessaire pour la faire fonctionner. Il faut faire confiance
à la banque centrale pour qu'elle ne déprécie pas la monnaie, mais
l'histoire des monnaies fiat est pleine de violations de cette
confiance. Il faut faire confiance aux banques pour détenir notre argent
et le transférer par voie électronique, mais elles le prêtent par vagues
de bulles de crédit avec à peine une fraction en réserve\footnote{Satoshi
  Nakamoto, \emph{Bitcoin open source implementation of P2P currency},
  11 février 2009~:
  \url{https://p2pfoundation.ning.com/forum/topics/bitcoin-open-source}.}.~»

Les banques centrales sont ainsi issues de la centralisation de
l'activité bancaire. Cependant, ce ne sont plus aujourd'hui des banques,
dans le sens où elles n'émettent plus des substituts monétaires, mais la
monnaie elle-même. Elles sont en effet devenues des institutions de
création monétaire, prenant la place des hôtels de la Monnaie qui
frappaient les pièces.

Il est intéressant de constater que c'est l'adoption du billet de banque
et la prise de contrôle totale sur celui-ci qui ont mené à
l'installation durable de la monnaie fiat. En garantissant sa
convertibilité, l'État lui a d'abord octroyé un avantage par rapport aux
espèces sonnantes et trébuchantes, le billet offrant une portabilité
largement supérieure aux pièces de métaux précieux. Une fois le billet
devenu monnaie courante, la puissance publique n'a ensuite eu qu'à
suspendre progressivement la convertibilité pour conclure la
transformation.

De même, une telle prise de contrôle sur les comptes bancaires est
aujourd'hui en cours. Avec la monétisation générale du crédit bancaire
numérique soutenue par la garantie étatique des dépôts, tous les
ingrédients sont présents pour l'accomplissement d'une nouvelle
mutation. C'est l'objet du développement de la monnaie numérique de
banque centrale.

\section*{La monnaie numérique de banque
centrale}\label{la-monnaie-numuxe9rique-de-banque-centrale}
\addcontentsline{toc}{section}{La monnaie numérique de banque centrale}

\markright{La monnaie numérique de banque centrale}

Une monnaie numérique de banque centrale (MNBC), de l'anglais
\emph{central bank digital currencies} (CBDC), est une monnaie
fiduciaire numérique émise par une banque centrale. Il s'agit d'une
sorte de monnaie entièrement numérique qui ne représente pas une
créance. Les systèmes informatiques de MNBC sont actuellement en phase
de conception tout autour du monde. Leur déploiement pourrait constituer
une évolution majeure dans l'histoire de la monnaie \emph{via}
l'appropriation indirecte des dépôts bancaires par l'État.

Disposer d'une monnaie qui serait gérée intégralement par une banque
centrale et qui concurrencerait la monnaie scripturale des banques
commerciales n'est pas une idée nouvelle. Cette idée remonte en effet à
une période antérieure à la démocratisation d'Internet. On la retrouve
sous la plume de l'économiste keynésien James Tobin, lauréat du prix
Nobel, qui faisait une suggestion approchante en 1987 en écrivant~:

«~Je pense que l'État devrait mettre à la disposition du public un
intermédiaire de paiement offrant la commodité des dépôts et la sécurité
des espèces, qui serait essentiellement de la monnaie sous forme de
dépôt, transférable pour tout montant par chèque ou autre
ordre\footnote{James Tobin, «~\emph{The Case for Preserving Regulatory
  Distinctions}~», in \emph{Proceedings - Economic Policy Symposium -
  Jackson Hole}, 1987, pp.~167--205~:
  \url{https://www.kansascityfed.org/documents/3828/1987-S87TOBIN.pdf}.}.~»

Avec l'émergence de Bitcoin dans les années 2010, l'idée d'une monnaie
numérique gérée par une banque centrale et mise à disposition des
particuliers a été remise au goût du jour. Elle est tout d'abord venue
de l'intérieur de la communauté de Bitcoin~: elle a été évoquée par un
utilisateur le 26 mars 2013 sous la forme de «~Fedcoin~», un concept
satirique d'une «~une alternative centralisée aux monnaies pair-à-pair~»
qui serait contrôlée par la Réserve fédérale des États-Unis\footnote{peculium,
  \emph{Fedcoin: A centrally-issued alternative to peer-to-peer
  currencies}, 26 mars 2013, archive~:
  \url{https://web.archive.org/web/20130404231341/http://peculium.net/2013/03/26/fedcoin-a-centrally-issued-alternative-to-peer-to-peer-currencies/}.}.
Du côté de l'Europe, l'idée d'un Eurocoin a été évoquée par Bitcoin.fr
le 1 avril 2014\footnote{Jean-Luc (Bitcoin.fr), \emph{Naissance de
  l'Eurocoin}, 1 avril 2014~:
  \url{https://bitcoin.fr/naissance-de-l-eurocoin/}.}. Bien
qu'initialement ironique, cette idée a mené à diverses réflexions sur la
pertinence d'un tel système et sur les conséquences de sa potentielle
implémentation\footnote{John Paul Koning, \emph{Fedcoin}, 19 octobre
  2014~: \url{https://jpkoning.blogspot.com/2014/10/fedcoin.html}.}.

Le sujet est devenu plus sérieux au début de l'année 2015 lorsque David
Andolfatto, alors vice-président de la Federal Reserve Bank de
Saint-Louis, en a fait la promotion dans une présentation donnée durant
la \emph{P2P Financial Systems Conference} à Francfort, puis dans un
article publié sur son blog\footnote{David Andolfatto, \emph{Fedcoin: On
  the Desirability of a Government Cryptocurrency}, 3 février 2015~:
  \url{https://andolfatto.blogspot.com/2015/02/fedcoin-on-desirability-of-government.html}.}.
Sa proposition était de faire en sorte que, contrairement à Bitcoin, le
système d'émission monétaire soit contrôlé par la Réserve fédérale, qui
se chargerait d'assurer la convertibilité de l'unité numérique en
dollars. Son modèle restait néanmoins mesuré~: pour Andolfatto, Fedcoin
devrait être un système ouvert et anonyme.

Le concept de monnaie numérique de banque centrale a pleinement émergé
avec le discours du 2 mars 2016 de Ben Broadbent, gouverneur adjoint
pour la politique monétaire à la Banque d'Angleterre, prononcé à la
London School of Economics, qui donnait naissance au terme de «~central
bank digital currency~»\footnote{Ben Broadbent, \emph{Central banks and
  digital currencies}, 2 mars 2016~:
  \url{https://www.bankofengland.co.uk/speech/2016/central-banks-and-digital-currencies}.}.
Dans ce discours, le banquier expliquait comment un registre distribué
pouvait permettre de remplacer l'actuel modèle de compensation et de
règlement interbancaire, d'en élargir l'accès aux acteurs financiers et
aux particuliers en leur permettant de posséder un compte auprès de la
banque centrale, et de faire ainsi concurrence à l'argent liquide et aux
dépôts dans les banques commerciales.

Depuis, le concept a été mis en œuvre de manière expérimentale. La
Banque populaire de Chine, qui a monté un programme de recherche (appelé
\emph{Digital Currency Electronic Payment} ou DCEP\footnote{«~\emph{Digital
  Currency Electronic Payment} ou DCEP~»~: Xinyu Liu, Fan Lu, Wanlu
  Shan, Jiayuan Zhang, \emph{The Progress of Digital Currency Electronic
  Payment}, 2021~:
  \url{https://www.atlantis-press.com/article/125965904.pdf}.}) dès
2014, a commencé à déployer progressivement son yuan numérique
(\emph{digital renminbi}) en 2020. La Riksbank suédoise a envisagé de
mettre en place une couronne électronique (ou e-Krona) en novembre
2016\footnote{«~couronne électronique (ou e-Krona)~»~: Cecilia
  Skingsley, \emph{Skingsley: Borde Riksbanken ge ut e-kronor?}, 16
  novembre 2016~:
  \url{https://www.riksbank.se/sv/press-och-publicerat/Tal/2016/Skingsley-Borde-Riksbanken-ge-ut-e-kronor/}~;
  archive~:
  \url{https://web.archive.org/web/20161117155655/https://www.riksbank.se/sv/press-och-publicerat/Tal/2016/Skingsley-Borde-Riksbanken-ge-ut-e-kronor/}.},
qui est toujours en phase de tests.

Aux États-Unis, l'effort est pris en charge par la \emph{Digital
Currency Initiative} du MIT Media Lab, une initiative créée en 2015 dans
le but «~de réunir les esprits les plus brillants {[}...{]} pour mener
les recherches nécessaires au développement des monnaies numériques et
de la technologie blockchain\footnote{Digital Currency Initiative,
  \emph{About the MIT Digital Currency Initiative}~:
  \url{https://dci.mit.edu/about}.}~» et qui a notamment financé
certains développeurs de Bitcoin Core comme Gavin Andresen, Wladimir van
der Laan et Cory Fields. Cette initiative a abouti au projet Hamilton en
février 2022, un prototype de monnaie numérique développé conjointement
avec la \emph{Federal Reserve Bank} de Boston\footnote{James Lovejoy,
  Cory Fields, Madars Virza, Tyler Frederick, David Urness, Kevin
  Karwaski, Anders Brownworth, Neha Narula, \emph{A High Performance
  Payment Processing System Designed for Central Bank Digital
  Currencies}, 3 février 2022~:
  \url{https://www.bostonfed.org/publications/one-time-pubs/project-hamilton-phase-1-executive-summary.aspx}.}.

Du côté de la Grande-Bretagne, la Banque d'Angleterre a annoncé former
un groupe de travail en avril 2021 en collaboration avec le trésor de Sa
Majesté\footnote{«~la Banque d'Angleterre a annoncé former un groupe de
  travail en avril 2021~»~: Bank of England, \emph{Bank of England
  statement on Central Bank Digital Currency}, 19 avril 2021~:
  \url{https://www.bankofengland.co.uk/news/2021/april/bank-of-england-statement-on-central-bank-digital-currency}.}.
En Europe continentale, la BCE a annoncé en juillet 2021 vouloir
développer un euro numérique\footnote{«~la BCE a annoncé en juillet 2021
  vouloir développer un euro numérique~»~:
  \url{https://www.ecb.europa.eu/press/pr/date/2021/html/ecb.pr210714~d99198ea23.en.html}}.

Le concept de monnaie numérique de banque centrale se fonde sur un
modèle déjà existant~: celui de la monnaie numérique interbancaire
composée des avoirs monétaires détenus par les banques commerciales
auprès de la banque centrale. Cette monnaie est destinée à fluidifier
les règlements entre les banques, plutôt que de passer par des espèces.
Elle constitue, avec les pièces et les billets en circulation, ce qu'on
appelle la monnaie centrale ou monnaie de base. Celle-ci est fiduciaire
par nature, dans le sens où elle tire essentiellement sa valeur de la
confiance que ses utilisateurs accordent à l'entité qui l'émet et non
pas à une propriété physique intrinsèque.

L'idée derrière la monnaie numérique de banque centrale est d'étendre
l'accès de cette monnaie interbancaire aux autres entreprises et aux
particuliers. Les banques centrales parlent parfois de «~MNBC de
détail~» (\emph{retail CBDC}) pour différencier ce projet de celui d'une
modernisation de la monnaie interbancaire existante, qui constituerait
une «~MNBC de gros\footnote{Fabio Panetta, \emph{Demystifying wholesale
  central bank digital currency}, 26 septembre 2022~:
  \url{https://www.ecb.europa.eu/press/key/date/2022/html/ecb.sp220926~5f9b85685a.en.html}.}~»
(\emph{wholesale CBDC}). Nous parlerons ici uniquement de la MNBC de
détail.

D'un point de vue technique, une monnaie numérique de banque centrale
serait basée sur un registre de compte, distribué entre quelques
serveurs grâce à un mécanisme de consensus de type classique, très bien
adapté pour traiter un volume transactionnel élevé. La réplication des
données financières à différents endroits permettrait d'éviter toute
perte liée à une panne ou une cyberattaque.

Le système serait accessible via une identification de l'utilisateur,
probablement grâce un système d'identité numérique, dans le but de
satisfaire les exigences de lutte contre le blanchiment et le
financement du terrorisme. Les transactions des utilisateurs seraient
cachées au public, mais pourraient être observées par une autorité
homologuée.

Comme tout système informatique, un tel dispositif serait programmable,
et des conditions de dépenses pourraient être ajoutées aux fonds. De
plus, ce modèle pourrait être modifié au cours du temps pour inclure de
nouvelles fonctionnalités.

Les apports directs de la monnaie numérique pour l'utilisateur seraient
multiples. D'abord, elle éliminerait le risque de contrepartie lié au
crédit~: l'utilisateur pourrait jouir théoriquement de tous les
avantages apportés par un compte bancaire sans subir le risque de
faillite de la banque. Ensuite, elle fournirait une plus grande
accessibilité et favoriserait l'inclusion financière en permettant de
«~bancariser les non-bancarisés~» à moindre frais. Enfin, elle
automatiserait les opérations financières de façon à améliorer
considérablement la qualité des services en ligne.

Grâce à ces avantages, la monnaie numérique de banque centrale paraît
représenter un progrès, une modernisation de la monnaie physique
dépassée par la numérisation de la société. Toutefois, c'est ignorer son
potentiel majeur pour le pouvoir et les inconvénients majeurs pour
l'utilisateur individuel.

Pour l'État, le potentiel des monnaies numériques de banque centrale est
double. Premièrement, la monnaie numérique de banque centrale a le
potentiel d'apporter un contrôle financier total.

D'une part, la généralisation de la monnaie de banque centrale formerait
une base légale à partir de laquelle supprimer l'argent liquide. En
effet, contrairement au crédit bancaire, la MNBC constituerait une
monnaie de base dont il serait aisé de définir le cours légal sur le
territoire. On pourrait donc assister à une disparition progressive des
supports physiques de la monnaie.

D'autre part, elle permettrait d'améliorer la surveillance financière et
offrirait une possibilité d'intervention supérieure, notamment grâce au
traitement automatisé par intelligence artificielle. En particulier, une
MNBC faciliterait la collecte de l'impôt, en généralisant le prélèvement
direct sur le compte du contribuable. Cet aspect est traité dans la
section apparentée du chapitre~\hyperref[ch:censure]{9} sur la
résistance à la censure.

Deuxièmement, la monnaie numérique de banque centrale possède un
potentiel inflationniste non négligeable. D'une part, le remplacement de
l'argent liquide permettrait d'éliminer les coûts de production, de
distribution et de destruction des supports monétaires (pièces et
billets). Cela améliorerait le seigneuriage sur la monnaie de base, en
diminuant largement le coût de production. C'est déjà le cas avec la
monnaie centrale interbancaire.

D'autre part, le remplacement progressif du crédit bancaire permettrait
de récupérer le seigneuriage réalisé sur le crédit par les banques
commerciales, comme cela se fait déjà partiellement grâce au taux de
refinancement. Cette capture se ferait aux dépend des banques, qui
verrait leur capacité à prêter être réduite voire annihilée. C'est
pourquoi elles devraient gagner quelque chose au change, par exemple en
obtenant à la place un rôle d'intermédiaire dans le système\footnote{«~{[}les
  banques commerciales{]} devraient gagner quelque chose au change, par
  exemple en obtenant à la place un rôle d'intermédiaire dans le
  système~»~: C'est le sens de l'idée de MNBC «~synthétique~» évoquée
  par Tobias Adrian (économiste du FMI) en 2019. -- Tobias Adrian,
  \emph{Stablecoins, Central Bank Digital Currencies, and Cross-Border
  Payments: A New Look at the International Monetary System}, 13 mai
  2019~:
  \url{https://www.imf.org/en/News/Articles/2019/05/13/sp051419-stablecoins-central-bank-digital-currencies-and-cross-border-payments}).
  Cette idée a été intégrée dans le prototype Aurum de la Banque des
  règlements internationaux (BRI) présenté en octobre 2022
  (\emph{Project Aurum: a prototype for two-tier central bank digital
  currency (CBDC)}, 21 octobre 2021~:
  \url{https://www.bis.org/publ/othp57.htm}).}.

Les banques commerciales pourraient être pleinement absorbées par la
banque centrale, dont elles deviendraient les succursales. Ainsi, le
vieux rêve marxiste de centraliser le crédit entre les mains d'une seule
banque serait réalisé\footnote{Le point 5 du programme dressé dans le
  Manifeste du parti communiste prône la «~centralisation du crédit
  entre les mains de l'État, au moyen d'une banque nationale, dont le
  capital appartiendra à l'État et qui jouira d'un monopole exclusif.~»
  -- Karl Marx, \emph{Manifeste du parti communiste}, Ère Nouvelle,
  1895.}. À l'instar de la Gosbank, la banque centrale de l'Union
soviétique et seule banque autorisée 1932 et 1987, cette banque unique
suivrait les directives du pouvoir central en accordant des prêts
financés par création monétaire, non aux emprunteurs solvables, mais aux
entités favorisées par la planification économique.

Tout ceci constitue une prospective qui semble peu probable au premier
abord. Quand on voit les dangers que créent la généralisation d'un tel
système, on peut penser que la population ne pourrait pas accepter cette
mutation. Ces systèmes ne sont en effet pas naturellement adoptés par
les citoyens, comme en témoignait l'échec de l'eNaira au Nigéria en
2023. En Occident, une réaction de rejet existe, notamment à droite, et
des personnalités publiques attachées aux libertés ont déjà affirmé leur
opposition, comme le lanceur d'alerte Edward Snowden qui a qualifié
cette potentielle monnaie numérique de «~monnaie
cryptofasciste\footnote{Edward Snowden, \emph{Your Money AND Your Life},
  9 octobre 2021~: \url{https://edwardsnowden.substack.com/p/cbdcs}.}~»
en octobre 2021.

L'acceptation promet donc d'être complexe, mais elle est loin d'être
impossible. Elle pourrait reposer sur des incitations légales
encourageant l'utilisation de la MNBC et pénalisant son refus, par la
récompense et la punition. La récompense serait constituée de diverses
subventions pour encourager l'usage, versées aux commerçants et aux
consommateurs, comme cela est déjà fait en Chine dans le cadre du yuan
numérique\footnote{«~diverses subventions pour encourager l'usage
  {[}...{]} comme cela est déjà fait en Chine dans le cadre du yuan
  numérique~»~: China Daily, \emph{E-CNY boosts holiday consumption}, 1
  février 2023~:
  \url{https://www.chinadaily.com.cn/a/202302/01/WS63d9bb3fa31057c47ebac36e.html}.}.
La punition, qui arriverait dans un second temps, pourrait se composer
de l'imposition d'un cours légal qui contraindrait les commerçants à
accepter la monnaie numérique centrale, du refus d'accès aux services
publics aux personnes ne disposant pas d'un compte à la banque centrale,
et de la censure des opinions anti-MNBC, jugées complotistes.

Quoi qu'il en soit, la monnaie numérique de banque centrale repose,
comme pour toute mesure étatique, sur l'acceptation de la population
générale. L'opinion publique est donc le champ de bataille ici, mais on
est en droit d'imaginer que l'État l'emportera au bout d'une période
plus ou moins longue, comme il l'a fait avec le papier-monnaie. Dans ce
cas, Bitcoin deviendrait la seule porte de sortie monétaire viable pour
la résistance.

\section*{L'arbitrage juridictionnel}\label{larbitrage-juridictionnel}
\addcontentsline{toc}{section}{L'arbitrage juridictionnel}

\markright{L'arbitrage juridictionnel}

Un concept régulièrement invoqué comme un moyen de protéger se liberté
individuelle face au contrôle de l'État est celui d'«~arbitrage
juridictionnel~», terme calqué sur l'anglais \emph{jurisdictional
arbitrage}. Il s'agit, pour une personne, de tirer parti des divergences
qui existent entre des juridictions concurrentes pour optimiser ses
conditions de vie. La forme la plus simple de cet arbitrage est
l'expatriation fiscale qui consiste à émigrer pour bénéficier d'un taux
de prélèvement moins élevé. Cette méthode aurait aussi pour conséquence
d'inciter les États, par l'effet de la concurrence, à respecter la
liberté de leurs citoyens, et serait de ce fait une forme de «~vote avec
ses pieds~».

L'arbitrage juridictionnel est un phénomène qui a émergé avec la baisse
drastique du coût de changement de juridiction\footnote{«~coût de
  changement de juridiction~»~: Patri Friedman, \emph{Dynamic Geography:
  A Blueprint for Efficient Government}, 2002~:
  \url{https://patrifriedman.com/old_writing/dynamic_geography.html}.}
ayant eu lieu au cours des siècles passés, par l'assouplissement des
restrictions migratoires, la baisse des frais de voyage et
l'accroissement de la liquidité des actifs. De plus, la facilitation de
la communication liée à l'arrivée d'Internet a amplifié cet effet en
fournissant aux individus un moyen de se soustraire partiellement à
l'influence de leurs autorités locales. C'est pourquoi cette stratégie
est aujourd'hui beaucoup mise en avant.

La notion d'arbitrage juridictionnel a été notamment décrite par les
auteurs à succès Rees-Mogg et Davidson dans leur ouvrage \emph{The
Sovereign Individual} publié en 1997, dont la thèse principale était de
prédire le recul des États-Nations face à l'innovation technique. Ils y
formulaient un théorème d'inéquivalence qui, en opposition à
l'équivalence ricardienne, postulait que les acteurs économiques ne
réduiraient pas leur consommation par anticipation d'une hausse d'impôt
due à une relance budgétaire, mais changeraient de juridiction~:

«~À l'Ère de l'Information, {[}...{]} la personne rationnelle ne réagira
pas à la perspective d'une augmentation des impôts pour financer les
déficits en augmentant son taux d'épargne~; elle déplacera son domicile
ou effectuera ses transactions ailleurs. {[}...{]} Il faut donc
s'attendre à ce que les Individus Souverains et les autres personnes
rationnelles fuient les juridictions ayant d'importants engagements non
financés\footnote{William Rees-Mogg, James Dale Davidson, \emph{The
  Sovereign Individual: Mastering the Transition to the Information
  Age}, Touchstone, 1999, p.~247.}.~»

Cette vision était également partagée par les cypherpunks, dont beaucoup
voyaient le cyberespace émergent comme une juridiction indépendante à
part entière, hors d'atteinte de l'État\footnote{John Perry Barlow,
  \emph{A Declaration of the Independence of Cyberspace}, 8 février
  1996~: \url{https://www.eff.org/fr/cyberspace-independence}.}. Ils
envisageaient en particulier l'émission d'une cybermonnaie échappant au
contrôle des États. C'était le cas d'Eric Hughes, qui confiait au
journaliste Kevin Kelly en 1994~:

«~La question la plus importante est la suivante~: quelle est l'ampleur
des flux monétaires sur les réseaux avant que l'État n'exige la
déclaration de chaque petite transaction~? Car si les flux peuvent
devenir suffisamment importants, au-delà d'un certain seuil, il pourrait
y avoir suffisamment de fonds agrégés pour inciter économiquement un
service transnational à émettre une monnaie, et les actions d'un État
n'auraient pas d'importance\footnote{Kevin Kelly, «~\emph{E-Money}~», in
  \emph{Out of Control: The New Biology of Machines, Social Systems, and
  the Economic World}, Addison-Wesley, 1994~:
  \url{https://kk.org/mt-files/outofcontrol/ch12-f.html}.}.~»

Une des conséquences de l'arbitrage juridictionnel généralisé est
l'émergence naturelle d'une monnaie saine. Puisque, dans l'acception
naïve du concept, les États sont en concurrence et que les individus
peuvent se déplacer librement, ces derniers finiront par favoriser la
monnaie la moins taxée, c'est-à-dire celle empêchant le plus le
prélèvement involontaire. On pourrait ainsi voir des États émettre une
monnaie basée sur l'or, ou sur le bitcoin, pour faire concurrence aux
autres devises et bénéficier d'un attrait supplémentaire pour
prospérer\footnote{Cette dynamique a été formalisée par Parker Lewis
  sous la forme d'un dilemme du prisonnier appliqué à la question de
  l'interdiction de Bitcoin. Voir Parker Lewis, \emph{Bitcoin Cannot be
  Banned}, 11 août 2019~:
  \url{https://unchained.com/blog/bitcoin-cannot-be-banned/}.}.

Cependant, cette théorie séduisante résiste difficilement à l'épreuve de
la réalité, car elle néglige les rapports de domination qui existent
entre les États dans le cadre de leur interaction géopolitique. Les
États ne sont en effet pas des entités indépendantes~: ils sont sans
cesse en lutte pour prélever un revenu sur des populations,
principalement par leur contrôle du territoire, un conflit qui peut se
manifester, au niveau le plus extrême, par la guerre. Ces rapports de
domination s'exercent aujourd'hui au niveau mondial, car la baisse du
coût du transport et des télécommunications a non seulement amplifié
l'arbitrage juridictionnel, mais a aussi étendu l'interaction des États
entre eux.

Cette théorie fait en particulier abstraction d'un phénomène appelé
l'impérialisme, qui est la volonté d'un État d'étendre son pouvoir
au-delà de ses frontières naturelles, et qui se manifeste actuellement
par les actions des États-Unis, de la Russie et de la Chine dans leurs
sphères d'influence respectives. En effet, un État qui s'affaiblit en
renonçant à une partie de son revenu fiscal (même si la relation n'est
pas exactement proportionnelle) devient plus sensible à une ingérence
étrangère impérialiste. C'est pour cette raison que la concurrence entre
les États est beaucoup moins économique que ce qu'on imagine, celle-ci
étant soumise à des interventions politiques comme l'application de
sanctions qui restreignent les flux commerciaux, financiers et
migratoires vers et depuis l'État concerné.

L'une des facettes de l'impéralisme est l'impérialisme monétaire, qui
consiste à favoriser, par la violence ou la menace de violence, l'usage
d'une monnaie sur un territoire étranger pour en retirer un
avantage\footnote{Voir à ce sujet Hans-Hermann Hoppe, «~\emph{Banking,
  Nation States, and International Politics: A Sociological
  Reconstruction of the Present Economic Order}~», in \emph{The Review
  of Austrian Economics}, vol.~4, no. 3, 1990, pp.~55--87~:
  \url{https://mises.org/library/banking-nation-states-and-international-politics-sociological-reconstruction-present}.}.
L'avantage visé ordinairement est le revenu de seigneuriage
supplémentaire rendu possible grâce à une plus grande utilisation de la
monnaie, quelque chose qui est parfois schématisé par l'idée que l'État
dominant «~exporte son inflation~». C'est précisément ce qu'ont pratiqué
les États-Unis avec le dollar depuis le début du \textsc{xx}~siècle,
notamment au moyen du système d'étalon de change-or de Bretton Woods.

Il est ainsi illusoire de croire qu'un État dominant puisse tolérer
qu'un État sous son influence émette, ou autorise ses citoyens à
émettre, une meilleure monnaie utilisable à grande échelle. L'arbitrage
juridictionnel ne s'applique ici qu'à la marge, c'est-à-dire dans la
mesure où il n'affaiblit pas le pouvoir central de manière
significative. La réelle façon de changer les choses dans le domaine
monétaire à moyen terme repose sur la désobéissance individuelle.

\section*{Les monnaies alternatives
centralisées}\label{les-monnaies-alternatives-centralisuxe9es}
\addcontentsline{toc}{section}{Les monnaies alternatives centralisées}

\markright{Les monnaies alternatives centralisées}

Face à cet ordre monétaire imposé par la force de façon plus ou moins
directe, certaines personnes ont tenté de construire des systèmes
alternatifs. Nous ne parlons pas de monnaies locales complémentaires
sans ambition~; nous parlons de monnaies dont le but était de
représenter un véritable contrepoids à la monnaie étatique. Et les
exemples les plus représentatifs de ces réelles alternatives nous
viennent des États-Unis.

Les États-Unis possèdent en effet une grande culture des monnaies
privées, conformément à l'esprit de liberté individuelle qui les a
caractérisés. Pendant la période coloniale et durant la première moitié
du \textsc{xix}~siècle, la frappe privée de pièces était tout à fait
autorisée et pratiquée\footnote{«~la frappe privée de pièces était tout
  à fait autorisée et pratiquée~»~: Brian Summers, \emph{Private Coinage
  in America}, 1 juillet 1976~:
  \url{https://fee.org/articles/private-coinage-in-america/}.}. De même,
l'activité bancaire a été relativement libre à partir de 1837, année de
fin du mandat de la \emph{Second Bank of the United States}, la banque
nationale de l'époque.

Cette liberté monétaire et bancaire a été cependant interrompue par les
mesures prises à la suite de la guerre de Sécession. D'une part, une loi
du Congrès du 8 juin 1864 a interdit la frappe privée des
pièces\footnote{Cette loi du 8 juin 1864 est devenue la section 486 du
  titre 18 du Code des États-Unis (intitulée \emph{18 U.S. Code § 486 -
  Uttering coins of gold, silver or other metal}) qui dispose~:
  «~Quiconque, sauf dans le cas où cela est autorisé par la loi,
  fabrique, met en circulation ou fait passer, ou tente de mettre en
  circulation ou de faire passer, des pièces d'or ou d'argent ou
  d'autres métaux, ou des alliages de métaux, destinées à être utilisées
  comme monnaie courante, qu'elles ressemblent à des pièces des
  États-Unis ou de pays étrangers, ou qu'elles soient de conception
  originale, sera condamné à une amende en vertu du présent titre ou à
  une peine d'emprisonnement de cinq ans au maximum, ou aux deux.~»}.
D'autre part, les \emph{National Banking Acts} de 1863 et 1864 ont
définitivement mis fin à l'horizontalité et l'indépendance des banques.

C'est à cette occasion qu'a été fondé le \emph{Secret Service}, une
agence étatique ayant pour mission de lutter contre le faux-monnayage et
la fraude financière en général. Créé le 14 avril 1865, le jour de
l'assassinat d'Abraham Lincoln, il servait, d'une façon détournée, à
affermir le monopole sur la production de monnaie.

Cette transition a été finalisée avec la création de la Réserve Fédérale
en 1913 et la prohibition de la détention d'or promulguée par l'ordre
exécutif 6102 signé par F.D. Roosevelt le 5 avril 1933.

Après l'abandon de toute référence à l'or dans le système monétaire
mondial (et l'abrogation consécutive de l'ordre exécutif en 1975) et le
développement d'Internet, l'idée de déployer une monnaie privée est
réapparue. Puisque l'État fédéral pouvait gérer arbitrairement sa
monnaie, pourquoi ne pouvait-il pas en être autant des individus~? C'est
ainsi que des individus ont entrepris, dans une démarche purement
hayekienne, de déployer leur propre monnaie sur le marché. Parmi ces
projets de monnaie privée, nous pouvons en citer quatre~: ALH\&Co, le
Liberty Dollar, e-gold et Liberty Reserve.

Le premier était ALH\&Co, une banque libre offrant la possibilité à ses
clients d'avoir des comptes bancaires libellés en or ou en
dollars\footnote{Wendy McElroy, «~\emph{Anthony L. Hargis And The
  Trusted Third Party Trap}~», \emph{Agorist Nexus}, 14 mai 2020~:
  \url{https://www.agoristnexus.com/anthony-l-hargis-and-the-trusted-third-party-trap/}.}.
Cette banque a été créée par Anthony L. Hargis, un libertarien proche de
Samuel Edward Konkin et de son idée agoriste. Bien que la banque
elle-même faisait toutes les démarches pour rester légale, elle
n'empêchait pas l'évasion fiscale. Konkin lui-même a décrit le
fonctionnement de ALH\&Co dans son ouvrage \emph{Counter-Economics}
publié à titre posthume\footnote{Samuel Edward Konkin \textsc{iii},
  \emph{Counter-Economics: From the Back Alleys... To the Stars},
  KoPubCo, 2018.}.

Les clients pouvaient rédiger des «~ordres de transfert~» qui
fonctionnaient comme des chèques bancaires entre les différentes
entreprises qui les reconnaissaient, ou bien les soumettre à AHL\&Co et
recevoir en retour un chèque bancaire classique ou demander à ALH\&Co de
payer leurs factures régulières. ALH\&Co a existé pendant près de 30
ans, entre 1976 et 2004, du fait de son caractère confidentiel. À un
moment donné, la banque avait 253 clients et utilisait 9 comptes
bancaires classiques sur lesquels étaient déposés 7,2 millions de
dollars.

En mai 1993, les locaux d'ALH\&Co ont subi une descente des agents
fédéraux, suite à un signalement de suspicion de blanchiment d'argent
lié au trafic de drogue. Les agents se sont emparés des dossiers des
clients. Cependant, les opérations d'ALH\&Co on pu continuer pendant une
décennie.

Hargis a finalement été inculpé en mars 2004, et ALH\&Co a
définitivement été fermée. L'IRS a estimé que l'évasion fiscale des
clients s'élevait à 24 millions de dollars\footnote{«~L'IRS a estimé que
  l'évasion fiscale des clients s'élevait à 24 millions de dollars~»~:
  \url{https://www.latimes.com/archives/la-xpm-2004-mar-10-fi-taxscam10-story.html}}.

Le deuxième exemple contemporain de monnaie privée aux États-Unis est le
Liberty Dollar, une monnaie basée sur l'or et l'argent qu'on pouvait
retrouver sous forme de pièces d'argent, de billets représentatifs et,
un peu plus tard, d'unités électroniques. Le Liberty Dollar a été créé
en 1998 par Bernard von NotHaus \emph{via} son organisation à but non
lucratif NORFED\footnote{«~NORFED~»~: NORFED est l'acronyme de
  \emph{National Organization for the Repeal of the Federal Reserve and
  Internal Revenue Code}, en français~: l'Organisation nationale pour
  l'abrogation de la Réserve fédérale et de l'\emph{Internal Revenue
  Code}.}.

Ce système a connu un certain succès, notamment après l'introduction du
système de monnaie numérique en 2003. Outre les pièces de monnaies en
circulation, les coffres de NORFED contenaient environ 8 millions de
dollars en métaux précieux pour assurer la convertibilité de la devise,
dont 6 pour garantir l'unité numérique\footnote{P. Carl Mullan, \emph{A
  History of Digital Currency in the United States}, Palgrave Macmillan,
  2016.}.

Toutefois, en septembre 2006, la Monnaie des États-Unis, l'institution
en charge de frapper et mettre en circulation les pièces de monnaie
américaines, a émis un communiqué de presse, écrit conjointement avec le
département de la Justice, dans lequel elle concluait que les
«~médaillons~» de NORFED violaient la section 486 du titre 18 du Code
des États-Unis et constituaient «~un crime\footnote{United States Mint,
  \emph{Liberty Dollars Not Legal Tender, United States Mint Warns
  Consumers}, 14 septembre 2006~:
  \url{https://www.usmint.gov/news/press-releases/20060914-liberty-dollars-not-legal-tender-united-states-mint-warns-consumers}.}~».
Le communiqué rappelait également que les pièces frappées ressemblaient
au dollar ce qui s'apparentait à de la contrefaçon\footnote{«~les pièces
  frappées ressemblaient au dollar ce qui s'apparentait à de la
  contrefaçon~»~: \emph{18 U.S. Code § 485 - Coins or bars}~:
  \url{https://www.law.cornell.edu/uscode/text/18/485}.}.

Après une descente du FBI dans les locaux de NORFED en 2007\footnote{«~Après
  une descente du FBI dans les locaux de NORFED en 2007~»~: Bernard von
  NotHaus, \emph{FBI Raids Liberty Dollar}, 14 novembre 2007~:
  \url{http://www.libertydollar.org/ld/legal/raidday1.htm}.}, les
violations ont été retenues contre NotHaus et ses associés, qui ont été
arrêtés en 2009 et jugés en mars 2011. En conséquence de ce jugement,
les pièces pouvaient être considérées comme de la contrebande et être
saisies comme telles\footnote{«~les pièces pouvaient être considérées
  comme de la contrebande et être saisies comme telles~»~:
  \url{https://www.coinworld.com/news/precious-metals/liberty-dollars-may-be-subject-to-seizure.html}}.
Les ventes de ces pièces ont également été interdites sur eBay en
décembre 2012, sous la pression du Secret Service\footnote{«~Les ventes
  de ces pièces ont également été interdites sur eBay en décembre
  2012~»~: Jon Matonis, \emph{U.S. Secret Service Bans Certain Gold and
  Silver Coins On eBay}, 15 décembre 2012~:
  \url{https://www.forbes.com/sites/jonmatonis/2012/12/15/u-s-secret-service-bans-certain-gold-and-silver-coins-on-ebay/}.}.
En 2014, Bernard von NotHaus a été condamné à six mois d'assignation à
résidence et à trois ans de liberté conditionnelle.

Le Liberty Dollar n'était pas inconnu des premiers utilisateurs de
Bitcoin. Ainsi, Dustin Trammell, l'un des premiers opérateurs de nœud
sur le réseau, s'intéressait à ce système avant de découvrir la monnaie
de Nakamoto comme en témoigne son article sur le sujet en décembre
2008\footnote{Dustin Trammell, \emph{The Problem With the Liberty
  Dollar}, 7 décembre 2008~:
  \url{https://blog.dustintrammell.com/the-problem-with-the-liberty-dollar/}.}.

Le troisième cas de monnaie privée est l'e-gold\footnote{«~Le troisième
  cas de monnaie privée est l'e-gold~»~: Ludovic Lars, \emph{L'e-gold de
  Douglas Jackson~: la cryptomonnaie "or"}, 8 mars 2020~:
  \url{https://journalducoin.com/analyses/gold-douglas-jackson-cryptomonnaie-or-1996/}.},
une «~devise en or numérique~» (\emph{digital gold currency}) transférée
électroniquement et garantie à 100~\% par une quantité équivalente en or
conservée en lieu sûr. Le système e-gold a été co-fondé par Douglas
Jackson et Barry Downey en 1996, deux ans avant PayPal. Douglas Jackson
était un oncologue américain vivant en Floride. Adepte de Hayek, il
souhaitait créer une meilleure monnaie avec e-gold.

L'e-gold était par essence une monnaie représentative, chaque montant
d'e-gold pouvant être converti en or réel. La détention et la conversion
d'or était administrée par une société créée pour l'occasion et basée
aux États-Unis, \emph{Gold \& Silver Reserve Inc.} (G\&SR). La société
garantissait également de l'e-silver, de l'e-platinum et de
l'e-palladium sur le même modèle.

Le système informatique était géré par une deuxième entreprise,
\emph{e-gold Ltd.}, enregistrée à Saint-Christophe-et-Niévès dans les
Caraïbes. Pour l'époque, il était très performant, mettant à profit un
système à règlement brut en temps réel inspiré du virement
interbancaire. Le système tirait profit des navigateurs web et en
particulier de Netscape, de sorte que chaque client pouvait avoir accès
à son compte depuis le site web.

Le système e-gold a rencontré ainsi un grand succès, à tel point qu'il
représentait à un moment donné le deuxième système de paiement en ligne
mondial derrière PayPal. À son apogée en 2006, il garantissait 3,6
tonnes d'or, soit plus de 80 millions de dollars, traitait 75~000
transactions par jour, pour un volume annualisé de 3 milliards de
dollars, et gérait plus de 2,7 millions de comptes.

Toutefois, ce succès fulgurant a été de courte durée. Au terme d'une
enquête menée par le Secret Service\footnote{«~Au terme d'une enquête
  menée par le Secret Service~»~:
  \url{https://www.secretservice.gov/press/releases/2008/07/us-secret-service-led-investigation-digital-currency-business-e-gold-pleads}},
Douglas Jackson, ses deux sociétés et ses associés ont été inculpés le
27 avril 2007 par le département de la Justice pour facilitation de
blanchiment d'argent et activité de transfert d'argent sans
licence\footnote{«~activité de transfert d'argent sans licence~»~:
  \emph{18 U.S. Code § 1960 - Prohibition of unlicensed money
  transmitting businesses}~:
  \url{https://www.law.cornell.edu/uscode/text/18/1960}.}.

Jackson a été condamné à 3 ans de liberté surveillée, incluant 6 mois
d'assignation à résidence sous surveillance électronique, et à 300
heures de travail communautaire. Ses deux entreprises ont dû payer une
amende de 300~000~\$. Après une tentative infructueuse d'obtenir une
licence, e-gold a dû fermer ses portes définitivement en novembre
2009\footnote{«~e-gold a dû fermer ses portes définitivement en novembre
  2009~»~:
  \url{https://web.archive.org/web/20100103135107/http://blog.e-gold.com/2009/11/egold-update-value-access.html}.}.

Un indicateur du succès d'e-gold est l'émergence de systèmes similaires
de devise en or numérique~: nous pouvons citer GoldMoney, fondé par
James Turk et son fils en février 2001, qui s'est aujourd'hui adapté aux
réglementations financières~; e-Bullion, fondé par James Fayed en
juillet 2001 et fermé en 2008~; et Pecunix fondé par Simon Davis en
2002, entreprise enregistrée au Panama, qui a fermé ses portes en 2015,
dans le cadre d'une escroquerie de sortie. Le Liberty Dollar
électronique (eLD) lancé en 2003 ne faisait ainsi que suivre la vague.

Ces devises en or numérique étaient encore utilisées du temps de
Bitcoin, de sorte que ses premiers utilisateurs en avaient connaissance.
Satoshi Nakamoto lui-même savait bien comment ces systèmes
fonctionnaient, comme le montre l'un de ses courriels adressé à la
\emph{Cryptography Mailing List}\footnote{«~Il est intéressant de noter
  que l'un des systèmes d'e-gold a déjà une forme de spam appelé
  ``dusting''. Les spammeurs envoient une minuscule quantité de
  poussière d'or afin de placer un message de spam dans le champ de
  commentaire de la transaction.~» -- Satoshi Nakamoto, \emph{Re:
  Bitcoin v0.1 released}, /01/2009 15:47:10 UTC~:
  \url{https://www.metzdowd.com/pipermail/cryptography/2009-January/015041.html}.}.
De même, Ross Ulbricht avait envisagé d'utiliser Pecunix pour Silk Road
avant de trouver Bitcoin\footnote{Correspondance par courriel entre Ross
  Ulbricht et Arto Bendiken (GX-270), septembre 2009~:
  \url{https://antilop.cc/sr/exhibits/253456462-Silk-Road-exhibits-GX-270.pdf}}.

Le quatrième et dernier exemple de projet de monnaie privée était le
système Liberty Reserve, qui permettait de détenir et de transférer des
devises indexées sur le dollar étasunien, sur l'euro ou sur
l'or\footnote{«~le système Liberty Reserve~»~: Ludovic Lars, \emph{La
  Liberty Reserve d'Arthur Budovsky~: plongée dans l'obscure préhistoire
  de Bitcoin}, 21 mars 2020~:
  \url{https://journalducoin.com/analyses/liberty-reserve-bitcoin/}.}.
Le système était la création d'Arthur Budovsky, un Américain d'origine
ukrainienne, aux côtés de Vladmir Kats. En 2006, Budovsky s'est expatrié
au Costa Rica, qui était alors considéré comme un paradis fiscal
facilitant le blanchiment d'argent, où il a enregistré sa société,
Liberty Reserve S.A.

L'inculpation d'e-gold en avril 2007 a profité grandement à Liberty
Reserve qui a pu prendre la relève. Le système a ainsi rencontré un
grand succès. En mai 2013, l'acte d'accusation du département de la
Justice étasunienne estimait que Liberty Reserve possédait plus d'un
million d'utilisateurs dans le monde, dont plus de 200~000 aux
États-Unis, et traitait 12 millions de transactions financières
annuellement, pour un volume combiné de plus de 1,4 milliards de
dollars\footnote{«~l'acte d'accusation du département de la Justice
  étasunienne estimait que Liberty Reserve possédait plus d'un million
  d'utilisateurs dans le monde, dont plus de 200~000 aux États-Unis, et
  traitait 12 millions de transactions financières annuellement, pour un
  volume combiné de plus de 1,4 milliards de dollars~»~: United States
  District Court for the Southern District of New York, \emph{Liberty
  Reserve, et al.~Indictment}, 28 mai 2013~:
  \url{https://www.justice.gov/sites/default/files/usao-sdny/legacy/2015/03/25/Liberty\%20Reserve\%2C\%20et\%20al.\%20Indictment\%20-\%20Redacted_0.pdf}.}.

Toutefois, ce succès s'est accompagné de complications sérieuses. En
2009, la \emph{Superintendencia General de Entidades Financieras}
(SUGEF) costaricaine s'est intéressée au cas de Liberty Reserve, lui
demandant d'obtenir une licence (chose qu'elle n'est pas parvenue à
faire). En mars 2011, une enquête a été ouverte. En novembre 2011, c'est
au tour du FinCEN étasunien qui délivre un avis selon lequel LR était
«~utilisée par les criminels pour effectuer des transactions
anonymes\footnote{United States District Court for the Southern District
  of New York, \emph{Liberty Reserve, et al.~Indictment}, 28 mai 2013~:
  \url{https://www.justice.gov/sites/default/files/usao-sdny/legacy/2015/03/25/Liberty\%20Reserve\%2C\%20et\%20al.\%20Indictment\%20-\%20Redacted_0.pdf}.}~».

La fin de Liberty Reserve a été retentissante, au terme d'une opération
coordonnée de manière internationale. Le 24 mai 2013, Arthur Budovsky et
les principaux gestionnaires de Liberty Reserve ont été inculpés et
arrêtés, dans des juridictions différentes~: en Espagne, aux États-Unis
et au Costa Rica. Après environ un an et demi de détention, en octobre
2014, Arthur Budovsky a été extradé de l'Espagne vers New York aux
États-Unis, où s'est déroulé son procès. En 2016, Arthur Budovsky a
finalement plaidé coupable pour blanchiment d'argent, et, a été condamné
à 20 ans de prison ferme.

Liberty Reserve a probablement été la dernière monnaie libre centralisée
de grande envergure sur Internet. Le système était encore massivement
utilisé lorsque Bitcoin faisait ses premiers pas. Liberty Reserve a
ainsi été utilisé pour acheter du bitcoin sur les toutes premières
plateformes de change, y compris sur la fameuse plateforme Mt. Gox~!

Si l'on tente de récapituler, il est \emph{de facto} interdit de fournir
des services bancaires sans licence (ALH\&Co), de frapper ses propres
pièces de monnaie et d'imprimer ses propres billets (Liberty Dollar), ou
de gérer des comptes électroniques en or (e-gold) ou dans la devise
nationale (Liberty Reserve), dans la mesure où cela fait concurrence à
l'État. Bien qu'il y ait des raisons multiples aux fermetures de ces
systèmes, on ne peut que constater que toutes les alternatives sérieuses
au système monétaire étatique ont été éliminées\footnote{Lawrence H.
  White, «~\emph{The Troubling Suppression of Competition from
  Alternative Monies: The Cases of the Liberty Dollar and E-Gold}~», in
  \emph{Cato Journal}, vol.~34, no. 2, 2014, pp.~281--301~:
  \url{https://ciaotest.cc.columbia.edu/journals/cato/v34i2/f_0031473_25521.pdf}.}.

Le monopole monétaire est souvent imposé subtilement excluant les
concurrents potentiels du marché par des lois liées à la contrefaçon ou
au blanchiment d'argent. C'est pour cette raison qu'il n'y a aujourd'hui
aucune alternative légale. C'est pourquoi les innovations dans le
domaine financier, comme PayPal ou GoldMoney, se sont conformées aux
réglementations existantes~: pour survivre\footnote{En particulier, la
  vision originelle de PayPal, produit développé par Confinity Inc.~au
  tout début, était révolutionnaire. Voici quel était le discours de son
  PDG Peter Thiel à l'automne 1999, rapporté par Eric Jackson en 2012~:
  «~Ce que nous qualifions de ``pratique'' pour les utilisateurs
  américains sera révolutionnaire pour les pays en développement. Les
  États de nombre de ces pays jouent avec leur monnaie. Ils ont recours
  à l'inflation et parfois à des dévaluations monétaires massives, comme
  nous l'avons vu en Russie et dans plusieurs pays d'Asie du Sud-Est
  l'année dernière, pour priver leurs citoyens de leurs richesses. La
  plupart des gens ordinaires n'ont jamais l'occasion d'ouvrir un compte
  à l'étranger ou de mettre la main sur plus de quelques billets d'une
  monnaie stable comme le dollar américain. Un jour, PayPal sera en
  mesure de changer cette situation. À l'avenir, lorsque notre service
  sera disponible en dehors des États-Unis et que la pénétration
  d'Internet continuera à s'étendre à tous les niveaux économiques,
  PayPal permettra aux citoyens du monde entier d'exercer un contrôle
  plus direct sur leurs monnaies qu'ils ne l'ont jamais fait auparavant.
  Il sera pratiquement impossible pour les États corrompus de voler les
  richesses de leurs citoyens par leurs anciens moyens, car, dans le cas
  où ils essaient, les citoyens se tourneront vers le dollar, la livre
  ou le yen, abandonnant ainsi leur monnaie locale sans valeur pour
  quelque chose de plus sûr.~» -- Voir Eric M. Jackson, \emph{The PayPal
  Wars: Battles With Ebay, the Media, the Mafia, and the Rest of Planet
  Earth}, World Ahead Pub., 2012.}.\footnote{Luke Nosek, ancien
  vice-président de Confinity chargé du marketing, a confirmé la vision
  originelle de PayPal durant le Forum économique mondial de Davos le 31
  janvier 2019~:}

L'intervention étatique est là pour s'immiscer dans le système monétaire
et le contrôler, en détruisant au besoin les alternatives. C'est pour
résister à cette force inouïe que Bitcoin a été conçu tel qu'il existe
aujourd'hui.

\section*{Bitcoin contre l'État}\label{bitcoin-contre-luxe9tat}
\addcontentsline{toc}{section}{Bitcoin contre l'État}

\markright{Bitcoin contre l'État}

L'État est ainsi l'incarnation organisée, territorialisée et
institutionnalisée du transfert de richesse non consenti. En tant que
tel, le contrôle sur la monnaie constitue logiquement un élément qu'il
revendique comme sa prérogative, d'autant plus qu'il en tire un revenu,
appelé le seigneuriage. Au fil du temps, ce contrôle sur la monnaie est
devenu de plus en plus pernicieux, et il s'est accéléré avec l'émergence
de la banque durant la Renaissance. L'usage des billets de banque a
progressivement été récupéré par l'État au moyen d'une banque centrale,
qui s'est arrogé le monopole exclusif sur leur production, jusqu'à les
transformer en papier-monnaie. De même, l'usage des dépôts bancaires,
qui est aujourd'hui surveillé et contrôlé minutieusement, pourrait être
repris dans un futur proche par l'État par le biais de la monnaie
numérique de banque centrale.

À moyen terme, il est illusoire de croire que l'État renoncera à son
prélèvement, ou même le rendra plus transparent~: il faudrait pour cela
que ses bénéficiaires demandent eux-mêmes cette transition. On pourrait
croire qu'un petit État aurait la possibilité et l'intérêt
géostratégique de le faire, mais ce serait ignorer les vélléités
impérialistes des puissances dominantes. Il ne suffit donc pas de faire
tourner un serveur dans une juridiction accomodante pour gérer une
monnaie numérique comme l'a montré le cas de Liberty Reserve.

C'est pourquoi Bitcoin est comme il est. Il est spécifiquement conçu
pour résister à l'intervention de l'État et constitue une tentative de
construire une alternative robuste au système monétaire actuel. Bitcoin
résout cette problématique en distribuant le fonctionnement du système
au sein d'un réseau pair-à-pair de nœuds. Cette distribution à égalité
permet de partager les risques entre les personnes qui s'en occupent, et
de faire en sorte que la sécurité du système repose sur leurs actions
économiques combinées plutôt que sur celle d'un seul individu ou d'une
seule entreprise.

\bookmarksetup{startatroot}

\chapter{Un mouvement technologique}\label{ch:cypherpunks}

\phantomsection\label{enotezch:5}{}

{B}\textsc{i}tcoin est un objet technique et doit être pensé en tant que
tel. La technique (du grec ancien \foreignlanguage{greek}{téknh},
«~habileté~», «~art~», «~métier~») est l'ensemble des procédés pratiques
issus du savoir humain employés en vue d'atteindre des objectifs
concrets, le plus souvent de manière reproductible. Ces procédés peuvent
tout aussi bien intervenir dans la fabrication de produits manufacturés
que dans la réalisation de services. Ils permettent aussi de construire
des biens intermédiaires, appelés outils, servant à produire d'autres
biens et services.

L'évolution technique fait partie intégrante de l'histoire du monde,
ayant modifié les rapports humains en profondeur à de multiples
reprises. Les innovations techniques majeures coïncident en effet avec
les grandes mutations historiques~: la maîtrise de la métallurgie avec
le développement des premières civilisations, l'émergence de
l'imprimerie avec la réforme protestante, la révolution industrielle
avec l'urbanisation et la planification. À notre époque, le
développement des ordinateurs et leur mise en réseau est en train de
transformer notre culture comme jamais auparavant.

L'enjeu politique est donc plus que jamais \emph{technologique}, dans le
sens où l'étude de la technique devient nécessaire pour appréhender
correctement les rapports de domination. Il s'agit essentiellement de
choisir quels procédés employer et comment en faire usage. En
particulier, cet enjeu est aujourd'hui au centre d'une opposition entre
l'individu et l'État, entre la liberté et l'autorité, entre
l'émancipation et l'oppression.

En tant qu'assemblage de procédés, Bitcoin s'inscrit pleinement dans
cette opposition technique. La guerre pour le contrôle sur la monnaie se
transpose de plus en plus dans le monde numérique, avec le développement
de la MNBC et l'essor de la cryptomonnaie. Nous nous ne vivons plus à
l'heure des pièces de métaux précieux ou des billets de banque (qui
conservent malgré tout une certaine utilité), mais à l'époque de la
monnaie numérique, bien plus adaptée à nos moyens de communication et
d'échange économique modernes. Satoshi en était conscient quand il
déclarait dès novembre 2008 que Bitcoin pourrait permettre de
«~remporter une bataille majeure dans la course aux armements~» et de
«~conquérir un nouveau territoire de liberté pour plusieurs années~».

Dans ce chapitre, nous nous proposons de revenir sur les évolutions
techniques et les idées politiques apparentées qui ont amené Bitcoin à
exister. Dans un premier temps, nous nous efforcerons de décrire comment
la cryptographie, l'ordinateur et Internet ont modifié nos moyens de
communiquer. Puis, dans un second temps, nous nous concentrerons sur les
mouvements des libristes, des extropiens et des cypherpunks, qui ont
formé le terreau techno-idéologique au sein duquel Bitcoin a germé.

\section*{La cryptographie symétrique et
l'ordinateur}\label{la-cryptographie-symuxe9trique-et-lordinateur}
\addcontentsline{toc}{section}{La cryptographie symétrique et
l'ordinateur}

\markright{La cryptographie symétrique et l'ordinateur}

Bitcoin est avant tout basé sur la communication, c'est-à-dire le fait
de transmettre des informations à autrui. Cette communication a
longtemps été restreinte géographiquement, du fait des limitations
techniques qui caractérisaient les sociétés pré-industrielles. L'échange
avec le lointain était très rare, ce qui expliquait l'existence de
langues et de cultures distinctes.

Toutefois, l'évolution technique a modifié cet état des choses. À partir
de la moitié du \textsc{xix} siècle, la télécommunication, ou la
transmission d'information à distance\footnote{«~la télécommunication,
  ou la transmission d'information à distance~»~: Le préfixe télé- vient
  du grec ancien \foreignlanguage{greek}{th̃le}, tẽle, «~loin~».}, a
connu un bond prodigieux. Ceci s'est fait d'abord grâce à l'apparition
du télégraphe électrique, qui permettait d'envoyer et de recevoir des
messages écrits (ou télégrammes) d'une manière rapide et fiable. Puis
elle s'est renforcée avec l'arrivée du téléphone, qui donnait la
possibilité de transférer des paroles à distance. En outre, la
radiocommunication, basée sur l'usage des ondes radioélectriques pour
partager de l'information, a rendu ces techniques beaucoup plus
pratiques. Il est ainsi devenu possible de communiquer rapidement d'un
continent à l'autre, chose qui a notamment profité aux États, qui
pouvaient désormais gérer leurs territoires éloignés d'une manière plus
fluide et centralisée.

Cette évolution de la télécommunication a considérablement accru le
besoin de sécuriser l'information transmise, afin d'éviter qu'elle soit
interceptée par l'ennemi lors d'une guerre par exemple. C'est pourquoi
la cryptographie, qui existait depuis l'Antiquité, a connu un essor sans
précédent au cours du \textsc{xx} siècle.

La cryptographie est la discipline mathématique qui a pour but la
sécurisation de la communication en présence de tiers malveillants. À
l'origine, il s'agit de dissimuler de l'information par une méthode de
chiffrement, ce qui explique le mot, qui vient du grec ancien
\foreignlanguage{greek}{kruptós}, kruptós («~caché~») et de
\foreignlanguage{greek}{gráfein}, gráphein («~écrire~»). Par la suite,
la cryptographie s'est étendue à l'authentification de messages avec la
signature numérique et à la vérification de données par l'intermédiaire
des fonctions de hachage. En résumé, cette discipline permet d'assurer
la confidentialité (chiffrement), l'authenticité (signature) et
l'intégrité (hachage) de l'information transmise.

La cryptographie porte en elle la notion d'adversaire ou d'antagoniste
(de l'anglais \emph{adversary}) qui est une entité malveillante dont le
but est d'empêcher les utilisateurs d'un cryptosystème de réaliser leurs
objectifs. Il n'y a en effet pas besoin de cacher, d'authentifier ou de
vérifier quoi que ce soit en l'absence d'une menace externe. C'est pour
cette raison que le chiffrement et son pendant, la cryptanalyse, ont été
développés en premier lieu par et pour les États.

Le chiffrement est un procédé permettant de rendre impossible la
compréhension d'un message pour les personnes qui ne disposent pas d'une
information spécifique, appelée la clé de déchiffrement. La cryptanalyse
est la technique visant à déduire un texte en clair à partir d'un
message chiffré sans disposer de la clé.

À l'origine, le chiffrement se fait uniquement de manière symétrique,
c'est-à-dire que la clé de chiffrement et de déchiffrement sont les
mêmes et que les deux parties doivent avoir connaissance de cette clé
secrète pour communiquer. L'exemple typique de chiffrement symétrique
est le code de César, ou chiffrement par décalage, qui est l'une des
méthodes les plus simples et les plus connues pour chiffrer un texte. Le
texte chiffré s'obtient en remplaçant chaque lettre du texte clair
original par une lettre à distance fixe, toujours du même côté, dans
l'ordre de l'alphabet. La clé est alors le nombre correspondant au
décalage. Par exemple, un décalage de 21 lettres vers la droite
transforme le mot «~bitcoin~» en «~wdoxjdi~». Cette méthode tient son
nom du fait que Jules César l'utilisait dans ses correspondances
secrètes.

Le chiffrement symétrique pose néanmoins un problème logistique. La clé
doit en effet être transmise entre les deux parties qui communiquaient
et peut donc être interceptée. De plus, le nombre de clés à partager
augmente de manière exponentielle en fonction du nombre de personnes
impliquées (3 clés pour 2 personnes, 6 pour 3, 10 pour 4, etc.), ce qui
multiplie considérablement les risques. C'est ce qui explique pourquoi
l'apparition du chiffrement asymétrique, qui permettait de s'affranchir
de cette contrainte, a constitué une innovation.

Le développement du chiffrement a motivé la conception de machines de
plus en plus perfectionnées. Le chiffrement correct d'un message à la
main pouvait prendre des heures, de sorte que l'utilisation d'un
automate devenait pertinent. Après la Première Guerre mondiale, durant
laquelle la cryptologie avait joué un rôle clé notamment avec l'affaire
du télégramme Zimmermann, les premières machines de chiffrement sont
ainsi apparues à l'instar de la machine Enigma et des machines de
Lorentz utilisées par l'Allemagne.

Au cours de la Seconde Guerre, le besoin de cryptanalyse a poussé les
bélligérants à construire des machines à calculer programmables
spécialisées, pouvant évaluer un grand nombre de possibilités dans un
contexte précis. La Bombe de Turing et le Colossus ont ainsi été
fabriqués par les services de cryptanalyse britanniques afin de casser
les codes allemands. En parallèle, d'autres calculateurs (appelés
\emph{computers} en anglais) ont été développés dans le but de calculer
les trajectoires balistiques. C'était le cas de la machine Zuse Z3 en
Allemagne ou de l'\emph{Atanasoff--Berry Computer} aux États-Unis. Le
premier ordinateur au sens moderne du terme (Turing-complet\footnote{«~Turing-complet~»~:
  Alan Turing, \emph{On Computable Numbers, with an Application to the
  Entscheidungsproblem}, 28 mai 1936~:
  \url{https://www.cs.virginia.edu/~robins/Turing_Paper_1936.pdf}.},
entièrement électronique, à mémoire enregistrée) a été l'ENIAC, qui a
été conçu en 1945 par des ingénieurs de la \emph{Moore School of
Electrical Engineering} et dont l'architecture a été reprise en 1948 par
le mathématicien américain John von Neumann.

Après la Seconde Guerre mondiale, les ordinateurs sont devenus
progressivement plus efficaces grâce à l'invention du transistor (1947),
du circuit intégré (1958) et du microprocesseur (1971). Ceci a débouché,
au cours des années 1970, sur l'apparition de l'ordinateur personnel
(\emph{personal computer}), un ordinateur destiné à l'usage d'une
personne et dont les dimensions sont assez réduites pour tenir sur un
bureau. L'exemple le plus célèbre est sans doute l'Apple II, conçu par
Steve Wozniak et sorti en 1977 qui est le premier ordinateur personnel
fabriqué à grande échelle.

La développement des ordinateurs a naturellement coïncidé avec
l'élaboration des premiers langages de programmation, compilés et
interprétés~: le FORTRAN est apparu en 1957, le LISP en 1958, le COBOL
en 1959, le BASIC en 1964, et le C en 1972. Le langage C++, dans lequel
Satoshi Nakamoto a écrit la première version du logiciel de Bitcoin, a
fait son apparition plus tard, en 1985. Cette évolution a entraîné la
démocratisation de la programmation informatique. Elle a aussi marqué le
début de la sous-culture des \emph{hackers}, axée sur la compréhension
approfondie des systèmes informatiques et sur le détournement de leur
rôle prédéfini.

Les systèmes d'exploitation standards ont été conçus à partir des années
1970. Unix a été présenté par AT\&T au public pour la première fois en
1973. DOS, l'ancêtre de Windows, a été créé en 1981. Le Système 1
d'Apple, adapté à ses ordinateurs Macintosh, a été lancé en 1984. Le
système libre GNU/Linux a quant à lui été créé en 1991\footnote{«~ Le
  système libre GNU/Linux a quant à lui été créé en 1991~»~: Linus
  Benedict Torvalds, \emph{What would you like to see most in minix?},
  /08/1991 20:57:08 UTC~:
  \url{https://groups.google.com/g/comp.os.minix/c/dlNtH7RRrGA/m/SwRavCzVE7gJ}.}.

\section*{L'apparition de la cryptographie
moderne}\label{lapparition-de-la-cryptographie-moderne}
\addcontentsline{toc}{section}{L'apparition de la cryptographie moderne}

\markright{L'apparition de la cryptographie moderne}

La deuxième avancée majeure dans l'histoire technique qui a mené à
Bitcoin est l'apparition de la cryptographie moderne regroupant le
chiffrement asymétrique, la signature numérique et le hachage de
données. L'utilisation de plus en plus répandue des ordinateurs,
notamment au sein des universités américaines, a poussé les
cryptographes à imaginer des méthodes plus gourmandes en puissance de
calcul, mais bien plus efficaces pour le chiffrement. La percée a été
réalisée en 1976 lorsque les chercheurs Whitfield Diffie et Martin
Hellman ont publié un article scientifique, intitulé \emph{New
Directions in Cryptography}, dans lequel ils décrivaient un algorithme
d'échange de clés (destiné à la transmission des clés secrètes pour le
chiffrement symétrique) ainsi qu'un procédé de signature électronique.
L'introduction commençait comme suit~:

«~Nous sommes aujourd'hui à la veille d'une révolution dans le domaine
de la cryptographie. Le développement de matériel numérique bon marché a
permis de s'affranchir des limites de conception de l'informatique
mécanique et de ramener le coût des dispositifs cryptographiques de
haute qualité à un niveau tel qu'ils peuvent être utilisés dans des
applications commerciales telles que les distributeurs de billets
distants et les terminaux d'ordinateurs. À leur tour, ces applications
créent un besoin pour de nouveaux types de systèmes cryptographiques qui
minimisent la nécessité de canaux de distribution de clés sécurisés et
fournissent l'équivalent d'une signature écrite. Dans le même temps, les
développements théoriques de la théorie de l'information et de
l'informatique promettent de fournir des cryptosystèmes dont la sécurité
est prouvée, transformant ainsi cet art ancien en science\footnote{Whitfield
  Diffie et Martin Hellman, «~\emph{New Directions in Cryptography}~»,
  \emph{IEEE Transactions on Information Theory}, vol.~22, no. 6,
  novembre 1976, pp.~644--654~:
  \url{https://ee.stanford.edu/~hellman/publications/24.pdf}.}.~»

S'ils ont été les premiers à publier ces méthodes, ils n'ont pas été les
seuls à faire ces découvertes au cours de la période. Clifford Cocks,
James Ellis et Malcolm Williamson avaient déjà mis au point un tel
cryptosystème quelques années plus tôt (qu'ils appelaient le
«~chiffrement non secret~») pour le compte du GCHQ britannique, mais
leurs recherches sont restées classifiées\footnote{«~Clifford Cocks,
  James Ellis et Malcolm Williamson avaient déjà mis au point un tel
  cryptosystème {[}...{]} mais leurs recherches sont restées
  classifiées~»~: James H. Ellis, «~\emph{The Possibility of Secure
  Non-Secret Digital Encryption}~», \emph{CESG Report}, janvier 1970~:
  \url{https://cryptocellar.org/cesg/possnse.pdf}~; Clifford C. Cocks,
  «~\emph{A Note on Non-Secret Encryption}~», \emph{CESG Report}, 20
  November 1973~: \url{https://cryptocellar.org/cesg/notense.pdf}~;
  Malcolm J. Williamson, «~\emph{Non-Secret Encryption Using a Finite
  Field}~», \emph{CESG Report}, 21 janvier 1974~:
  \url{https://cryptocellar.org/cesg/secenc.pdf}.}. Le cryptographe
Ralph Merkle avait également décrit l'échange de clés de Diffie et
Hellman dans un article écrit en 1974 et publié en 1978 par
l'intermédiaire de ce qu'on appelle les puzzles de Merkle\footnote{«~Ralph
  Merkle avait également décrit l'échange de clés de Diffie et Hellman
  dans un article écrit en 1974 et publié en 1978~»~: Ralph C. Merkle,
  \emph{Publishing a new idea}, 2005~:
  \url{https://www.ralphmerkle.com/1974/}~; Ralph C. Merkle,
  «~\emph{Secure Communications over Insecure Channel}~»,
  \emph{Communications of the ACM}, avril 1978.}.

L'avancée de Diffie et Hellman a marqué le début de la cryptographie
asymétrique, ou cryptographie à clé publique, une discipline qui
regroupe le chiffrement asymétrique et la signature numérique. Dans un
système asymétrique, deux clés se distinguent~: une clé privée, censée
rester secrète, et une clé publique, dérivée de la clé privée. La clé
privée ne peut pas être retrouvée facilement à partir de la clé
publique, ce qui fait que cette dernière peut être partagée à tous en
toute quiétude.

Le chiffrement asymétrique consiste à utiliser la clé publique comme une
clé de chiffrement et la clé privée comme une clé de déchiffrement. Le
destinataire génère une paire de clés, garde la clé privée pour lui et
partage la clé publique à son interlocuteur pour qu'il lui envoie des
messages. Le fonctionnement de ce chiffrement est ainsi analogue à celui
d'une boîte aux lettres que le destinataire utiliserait pour recevoir
des lettres et dont lui seul posséderait la clé.

La signature numérique quant à elle, repose sur le fait d'utiliser la
clé privée comme une clé de signature et la clé publique comme clé de
vérification. L'expéditeur signe un message à l'aide de la clé privée et
l'envoie à son interlocuteur, qui peut vérifier son authenticité en
utilisant la clé publique.

Le système cryptographique asymétrique le plus connu a été conçu juste
après la publication du papier de Diffie et Hellman~: il s'agit de
l'algorithme de chiffrement RSA, créé en 1977 par Ronald Rivest, Adi
Shamir et Leonard Adleman et breveté par le MIT en 1983\footnote{«~l'algorithme
  de chiffrement RSA, créé en 1977 par Ronald Rivest, Adi Shamir et
  Leonard Adleman et breveté par le MIT en 1983~»~: Ron Rivest, Adi
  Shamir, Leonard Adleman, \emph{A Method for Obtaining Digital
  Signatures and Public-Key Cryptosystems}, février 1978~:
  \url{https://people.csail.mit.edu/rivest/Rsapaper.pdf}~; archive~:
  \url{https://web.archive.org/web/20070615132925/https://people.csail.mit.edu/rivest/Rsapaper.pdf}.}.
Celui-ci se base sur des opérations algébriques et sa sécurité provient
de la difficulté à décomposer de très grands nombres en facteurs
premiers. Il permet de chiffrer un message pour l'envoyer à quelqu'un,
mais aussi (grâce à l'interversion des rôles des clés) de signer
électroniquement ce message pour le publier. Cet algorithme est encore
aujourd'hui utilisé très largement sur Internet, et en particulier dans
le commerce électronique.

Du côté du chiffrement, d'autres algorithmes ont été conçus par le
suite. C'était le cas de l'algorithme de chiffrement d'ElGamal qui a été
présenté par Taher ElGamal en 1984\footnote{«~l'algorithme de
  chiffrement d'ElGamal qui a été présenté par Taher ElGamal en 1984~»~:
  Taher ElGamal, «~\emph{A Public Key Cryptosystem and a Signature
  Scheme Based on Discrete Logarithms}~», 1984~:
  \url{https://people.csail.mit.edu/alinush/6.857-spring-2015/papers/elgamal.pdf}.},
et dont la fiabilité reposait sur le problème du logarithme discret,
c'est-à-dire la difficulté mathématique à retrouver l'exposant d'un
élément dans un groupe cyclique fini\footnote{Soit
  \(G = (\mathbb{Z}_n^*, \cdot)\) un groupe cyclique fini et soit \(g\)
  un point générateur. Le problème du logarithme discret consiste, pour
  \(x \in G\), à retrouver \(k < n\) tel que \(x = g^k\). On écrit alors
  \(\mathrm{log}_g (x) = k\).}.

Du côté de la signature, des algorithmes destinés à servir uniquement à
cet usage ont été également développés. C'était le cas du modèle
d'ElGamal, qu'il a présenté en même temps que son système de chiffrement
en 1984, du schéma de signature de Schnorr, conceptualisé par
Claus-Peter Schnorr en 1991, et du \emph{Digital Signature Algorithm}
(DSA), conçu la même année par le NIST. Tous les trois se basaient aussi
sur le problème du logarithme discret.

La cryptographie sur courbes elliptiques est apparue en 1985 grâce aux
contributions indépendantes de Neal Koblitz et de Victor
Miller\footnote{«~contributions indépendantes de Neal Koblitz et de
  Victor Miller~»~: Neal Koblitz, «~\emph{Elliptic Curve
  Cryptosystems}~», 1987~:
  \url{https://community.ams.org/journals/mcom/1987-48-177/S0025-5718-1987-0866109-5/S0025-5718-1987-0866109-5.pdf}~;
  Victor S. Miller, «~\emph{Use of elliptic curves in cryptography}~»,
  1985.}. Elle a amené un bon nombre d'innovations, dont le procédé
d'échange de clés ECDH et l'algorithme de chiffrement hybride ECIES. Le
schéma de signature ECDSA, qui est l'algorithme principal utilisé dans
Bitcoin pour autoriser les transferts, a été créé en 1992.

La cryptographie asymétrique ouvrait également la voie aux fonctions à
sens unique, des fonctions dont le calcul d'une image est facile mais
dont l'obtention d'un antécédent est difficile. En effet, les systèmes
de chiffrement à clé publique pouvaient former eux-mêmes des fonctions
de ce type. De ce fait, la recherche dans la découverte de telles
fonctions s'est développée à partir de cette base.

On a assisté en particulier au développement des premières fonctions de
hachage cryptographiques, dont les premiers modèles datent de la fin des
années 1970\footnote{«~premières fonctions de hachage cryptographiques,
  dont les premiers modèles datent de la fin des années 1970~»~: Voir en
  particulier~: Michael O. Rabin, «~Digitalized Signatures~»,
  \emph{Foundations of Secure Computation}, 1978.}. Ces fonctions
avaient pour particularité de transformer un message de taille variable
en une empreinte de taille fixe. Entre 1989 et 1991, plusieurs
algorithmes de hachage (MD2, MD4, MD5) ont été conçus par Ronald Rivest
pour le MIT. Puis, l'algorithme SHA-0 a été créé en 1993 et SHA-1 en
1995. La suite d'algorithmes SHA-2, qui incluait le fameux SHA-256
largement utilisé dans Bitcoin, a été publiée en 2001.

En parallèle, les idées pour l'utilisation de ces fonctions de hachage
ont fleuri. Ces dernières permettaient de garantir l'intégrité de
l'information de façon à ce que tout changement soit détecté en sortie.
En 1979, Ralph Merkle a mis au point les arbres de hachage qui
permettaient d'authentifier un ensemble volumineux de données, auxquels
il a donné son nom\footnote{«~Ralph Merkle a mis au point les arbres de
  hachage qui permettaient d'authentifier un ensemble volumineux de
  données~»~: Ralph C. Merkle, «~\emph{Protocols for public key
  cryptosystems}~», in \emph{Proceedings of the 1980 IEEE Symposium on
  Security and Privacy}, IEEE Computer Society, pp.~122--133, avril
  1980~: \url{https://www.ralphmerkle.com/papers/Protocols.pdf}.}. Ces
arbres ont également été inclus dans la conception originelle de
Bitcoin\footnote{Les détails concernant l'utilisation de ces éléments
  cryptographiques (ECDSA, SHA-256, arbres de Merkle) dans Bitcoin
  seront discutés dans les chapitres \hyperref[ch:propriete]{7} et
  \hyperref[ch:confirmation]{8}.}.

\section*{L'usage civil de la
cryptographie}\label{lusage-civil-de-la-cryptographie}
\addcontentsline{toc}{section}{L'usage civil de la cryptographie}

\markright{L'usage civil de la cryptographie}

Ces découvertes de la cryptographie moderne ont inspiré les esprits
libres, qui ont tout de suite imaginé les applications qui pouvaient en
découler. En quelques années, un vaste domaine d'études venait d'être
ouvert dans le monde civil, et beaucoup d'individus allaient s'y
engouffrer.

C'était le cas de David Chaum, informaticien et cryptographe américain,
né en 1955 dans une famille juive à Los Angeles et étudiant à
l'université de Californie à Berkeley, qui s'est vite pris de passion
pour la protection de la vie privée. À partir de 1979, ce dernier a
contribué de manière primordiale au monde de la cryptographie par la
publication d'articles fondateurs. En 1981, il publiait l'article
\emph{Untraceable Electronic Mail, Return Addresses, and Digital
Pseudonyms}\footnote{David L. Chaum, «~\emph{Untraceable Electronic
  Mail, Return Addresses, and Digital Pseudonyms}~», in
  \emph{Communications of the ACM}, vol.~24, no. 2, 1981, pp.~84---90.},
où il décrivait les bases de la communication anonyme au travers de
réseaux de mélange (\emph{mix networks}), qui serait notamment utilisée
par les services de relai de courriel (Mixmaster) et par les réseaux
anonymes Tor, I2P et Freenet. En 1982, il décrivait le procédé de
signature aveugle, qui permettait notamment de mettre en place une
monnaie électronique anonyme\footnote{David L. Chaum, «~\emph{Blind
  signatures for untraceable payments}~», in \emph{Advances in
  Cryptology: Proceedings of CRYPTO '82}, 1982, pp.~199--203~:
  \url{https://sceweb.sce.uhcl.edu/yang/teaching/csci5234WebSecurityFall2011/Chaum-blind-signatures.PDF}.},
que Chaum mettrait en œuvre quelques années plus tard \emph{via} sa
société DigiCash, ainsi que l'émission de certificats automatiques,
utilisée par exemple dans ZeroLink aujourd'hui. Durant la même année, il
a également publié sa thèse de doctorat écrite en 1979, qui présentait
un système de coffres cryptographiques ayant pour but d'arriver à un
consensus au sein d'un ensemble d'acteurs ne se faisant pas
confiance\footnote{«~sa thèse de doctorat~»~: David L. Chaum,
  \emph{Computer Systems Established, Maintained and Trusted by Mutually
  Suspicious Groups}, 1982~:
  \url{https://chaum.com/publications/research_chaum_2.pdf}~; archive~:
  \url{https://web.archive.org/web/20151112100526/https://chaum.com/publications/research_chaum_2.pdf}.}.
En 1985, il a publié un protocole permettant de résoudre le problème du
dîner des cryptographes en garantissant l'anonymat de l'auteur d'un
message partagé au sein d'un groupe\footnote{David L. Chaum,
  «~\emph{Security without identification: transaction systems to make
  big brother obsolete}~», in \emph{Communications of the ACM}, vol.~28,
  no. 10, octobre 1985, pp.~1030---1044~:
  \url{https://www.cs.ru.nl/~jhh/pub/secsem/chaum1985bigbrother.pdf}.}.

David Chaum était obsédé par la protection de la vie privée, qu'il
estimait être en danger. Même si cette obsession n'atteignait pas la
radicalité des cypherpunks (dont il en était un précurseur), il n'en
restait pas moins qu'il était très inquiet pour l'avenir de la liberté
et de la confidentialité dans la société informatisée. En juillet 1995,
il déclarait ainsi devant la Chambre des représentants des États-Unis~:

«~Les ``techniques de confidentialité'' permettent aux personnes de
protéger leurs propres informations et leurs autres intérêts, tout en
maintenant une sécurité très élevée pour les organisations. Il s'agit
essentiellement de faire la différence entre, d'une part, un système
centralisé dans lequel les participants sont privés de leurs droits
(comme des animaux marqués électroniquement dans des fermes
d'engraissement) et, d'autre part, un système dans lequel chaque
participant est en mesure de protéger ses propres intérêts (comme les
acheteurs et les vendeurs sur une place de marché)\footnote{David Chaum,
  \emph{Testimony for US House of Representatives}, 25 juillet 1995,
  archive~:
  \url{https://web.archive.org/web/19970111170802/http://digicash.com/publish/testimony.html}.}.~»

Un autre exemple était Philip Zimmermann, informaticien et cryptographe
américain originaire de Philadelphie et ayant fait ses études en
Floride. Activiste politique opposé aux armes nucléaires, il avait
travaillé pour la \emph{Nuclear Weapons Freeze Campaign} à Boulder dans
le Colorado. Passionné par les énigmes et les secrets, il a découvert
l'existence de la cryptographie asymétrique par le biais d'un article de
Martin Gardner\footnote{Martin Gardner, «~\emph{A new kind of cipher
  that would take millions of years to break}~», in \emph{Scientific
  American}, août 1977~:
  \url{https://www2.math.upenn.edu/~kazdan/210S19/Notes/crypto/Gardner-RSA-1977.pdf}.}.
Il a publié un article dans la revue \emph{IEEE Computer} en 1986 sur
RSA\footnote{«~publié un article dans la revue \emph{IEEE Computer} en
  1986 sur RSA~»~: Philip R. Zimmermann, «~\emph{A Proposed Standard
  Format for RSA Cryptosystems}~», in \emph{IEEE Computer}, 1986.} avant
de concevoir PGP.

PGP (de l'anglais \emph{Pretty Good Privacy}) était un logiciel de
chiffrement hybride, qui se basait sur RSA pour l'échange de clés et sur
un algorithme de chiffrement symétrique pour la communication. Il
permettait aussi de générer des signatures. Il était spécialisé dans
l'échange de courriels\footnote{L'algorithme de chiffrement symétrique
  dans la version 1 était BassOmatic, conçu par Zimmermann lui-même. Il
  a été remplacé par IDEA dans la version 2 et les versions supérieures.
  La version 3 ajoutait les algorithmes ElGamal et DSA pour la partie
  asymétrique, et l'algorithme CAST-128 pour le côté symétrique.}.

Le 5 juin 1991, Phil Zimmermann en a publié la version 1.0 sous licence
libre. Dans le manuel d'utilisation il expliquait sa démarche~:

«~Si la confidentialité est interdite, seuls les hors-la-loi en
bénéficieront. Les agences de renseignement ont accès à une bonne
technique cryptographique. Il en va de même pour les grands trafiquants
d'armes et de drogue. Il en va de même pour les entreprises de défense,
les compagnies pétrolières et les autres géants de l'industrie. En
revanche, les citoyens ordinaires et les organisations politiques
populaires n'ont généralement pas accès à une technique cryptographique
à clé publique ``de qualité militaire'' à un prix abordable. PGP permet
aux gens de prendre leur confidentialité en main. La société en a de
plus en plus besoin. C'est pourquoi je l'ai écrit\footnote{Philip R.
  Zimmermann, «~\emph{Why do you need PGP?}~», in \emph{PGP User's
  Guide}, 5 juin 1991~:
  \url{https://www.tech-insider.org/free-software/research/acrobat/910605.pdf}.}.~»

Il a diffusé la première version de PGP depuis les États-Unis par
l'intermédiaire d'Internet ce qui fait que, en raison de la nature
internationale du réseau, le logiciel de chiffrement est rapidement
devenu disponible dans le monde entier. En faisant cela, Zimmermann
était conscient qu'il risquait d'attiser une réponse du pouvoir~: la
Réglementation américaine sur le trafic d'armes au niveau international
(\emph{International Traffic in Arms Regulations} ou ITAR) considérait
en effet les produits cryptographiques comme des «~munitions~» et en
interdisait l'exportation sans licence. En février 1993, alors que PGP
commençait à se populariser, une enquête contre Zimmermann a par
conséquent été ouverte par l'État fédéral. Heureusement cette enquête a
été abandonnée quelques années plus tard, notamment suite à la réaction
des cypherpunks, dont Zimmermann est toutefois resté à l'écart (à
l'instar de Chaum).

Enfin, les scientifiques Stuart Haber et Scott Scornetta ont aussi été
inspirés par ces découvertes. Stuart Haber était cryptographe et
informaticien, Scott Scornetta était physicien et chercheur. Les deux
hommes se sont rencontrés dans les locaux de Bell Communications
Research («~Bellcore~»), un consortium de recherche et développement
dans la télécommunication pour lequel ils travaillaient.

Ils ont conceptualisé le premier système d'horodatage de documents dans
l'article \emph{How to time-stamp a digital document} publié en 1991, et
qui a plus tard été cité au sein du livre blanc de Bitcoin\footnote{Stuart
  Haber, Wakefield Scott Stornetta, «~\emph{How to time-stamp a digital
  document}~», in \emph{Journal of Cryptology}, vol.~3, 1991,
  pp.~99--111~:
  \url{http://www.staroceans.org/e-book/Haber_Stornetta.pdf}.}. Il
s'agissait d'appliquer une fonction de hachage (par exemple MD4) à un
document numérique et de publier l'empreinte résultante dans un registre
public, de sorte à prouver l'existence du document à une date donnée.
Ils ont mis leur idée en application par la publication d'empreintes
dans les petites annonces du New York Times à partir de 1992\footnote{«~publication
  d'empreintes dans les petites annonces du New York Times à partir de
  1992~»~:
  \url{https://cypherpunks.venona.com/date/1992/11/msg00019.html}.}. Ils
ont ensuite créé leur propre société en 1994, Surety Technologies, dans
le but de se consacrer pleinement à cette activité. Ils sont ainsi
connus pour avoir créé la première chaîne temporelle d'horodatages,
préfigurant la chaîne de blocs de Bitcoin, en incluant l'empreinte
précédente dans le calcul de la nouvelle empreinte à publier dans le
journal\footnote{Daniel Oberhaus, \emph{The World's Oldest Blockchain
  Has Been Hiding in the New York Times Since 1995}, 27 août 2018~:
  \url{https://www.vice.com/en/article/j5nzx4/what-was-the-first-blockchain}.}.

De manière générale, tous ces individus ont, par leur compréhension de
la cryptographie asymétrique, largement préfiguré les cypherpunks. Ces
derniers ont été cependant bien plus loin en radicalisant les idées
politiques qu'esquissaient ces techniques mathématiques.

\section*{L'émergence d'Internet et le partage de
données}\label{luxe9mergence-dinternet-et-le-partage-de-donnuxe9es}
\addcontentsline{toc}{section}{L'émergence d'Internet et le partage de
données}

\markright{L'émergence d'Internet et le partage de données}

Avec l'émergence des ordinateurs est venue la volonté de les connecter
en réseau. C'est ainsi que les premiers réseaux informatiques se sont
formés dans les années 50. Mais ces réseaux n'étaient pas
interconnectés. Pour cela, il a fallu attendre un effort public et
ouvert, qui a été fait à partir des années 70, par l'intermédiaire du
développement du réseau des réseaux international~: Internet\footnote{«~réseau
  des réseaux international~»~: Ronda Hauben, \emph{The Internet: On its
  International Origins and Collaborative Vision (A Work In Progress)},
  2004~: \url{https://www.ais.org/~jrh/acn/ACn12-2.a03.txt}.}.

L'idée derrière Internet était de transmettre des paquets de données (et
plus spécifiquement des datagrammes) par le biais d'une technique nommée
la commutation de paquets, initialement décrite en 1964 par
l'informaticien polono-américain Paul Baran\footnote{Paul Baran,
  «~\emph{On Distributed Communications Networks}~», in \emph{IEEE
  Transactions on Communications Systems}, vol.~12, no.1, mars 1964,
  pp.~1--9~: \url{https://web.cs.ucla.edu/classes/cs217/Baran64.pdf}.}.
Cette technique consistait à indiquer la destination dans l'en-tête des
paquets de sorte à ce qu'il puissent être relayés sur le réseau,
notamment au moyen de routeurs. Elle s'opposait à la commutation de
circuits, qui reposait sur une liaison déterminée entre l'expéditeur et
le destinataire pour transmettre les données. À l'époque, les
communications transitaient au travers des lignes téléphoniques au moyen
d'un modem.

\begin{figure}

{\centering \includegraphics{chapters/img/baran1964-networks.png}

}

\caption{Réseaux~: (a) centralisé~; (b) décentralisé~; (c) distribué.
(Paul Baran, «~\emph{On Distributed Communications Networks}~», 1964)}

\end{figure}%

Le premier réseau d'Internet tire son origine dans la recherche
militaire. Il s'agissait du réseau ARPANET, conçu par l'ARPA, une agence
de recherche technique rattachée au département de la
Défense\footnote{L'ARPA (\emph{Advanced Research Projects Agency},
  «~Agence pour les projets de recherche avancée~») a été créée en 1958.
  Elle a été renommée en DARPA (\emph{Defense Advanced Research Projects
  Agency}, «~Agence pour les projets de recherche avancée de défense~»)
  en 1972. Elle est brièvement redevenue l'ARPA en 1993 avant d'adopter
  définitivement le nom de DARPA en 1996.}. Le but était de développer
un réseau de communication qui puisse résister aux attaques nucléaires
dans le cadre de la Guerre froide. Par la suite, d'autres réseaux se
sont développés de manière similaire dans le monde
militaro-universitaire comme le réseau du NPL au Royaume-Uni, le Merit
Network aux États-Unis ou le réseau Cyclades en France.

Le concept proprement dit d'Internet est apparu en 1974, avec
l'émergence d'une suite de protocoles facilitant l'interconnexion des
réseaux~: la suite TCP/IP\footnote{Vinton G. Cerf, Robert E. Kahn,
  «~\emph{A Protocol for Packet Network Intercommunication}~», in
  \emph{IEEE Transactions on Communications}, vol.~22, no. 5, mai 1974,
  pp.~637--648~:
  \url{https://www.cs.princeton.edu/courses/archive/fall06/cos561/papers/cerf74.pdf}.}.
Ces protocoles permettaient de standardiser la communication des
paquets. La standardisation a été finalisée avec la publication de la
version 4 de IP et la version de 4 de TCP en 1981 et leur intégration
dans ARPANET (le réseau fédérateur d'Internet) en 1983. En 1985, a été
créé le NSFNET, qui a rapidement pris de l'ampleur, à tel point qu'il a
remplacé ARPANET en tant que réseau fédérateur. Le projet ARPANET a été
officiellement mis hors service en 1990. Mais on pouvait considérer
qu'Internet était alors lancé.

Internet a provoqué un choc sans précédent sur la possibilité de
diffusion des informations. Toutefois, son développement et son adoption
ont été progressives, à mesure que les gens estimaient son potentiel et
son utilité. Cette croissance est passée par l'apparition de cas
d'utilisation diverses qui ont amené de plus en plus de gens à utiliser
le réseau des réseaux.

Le courrier électronique a été la première application d'Internet. Au
début, il s'agissait d'envoyer des textes par l'intermédiaire du
protocole FTP, puis des protocoles spécifiques ont été développés dans
les années 80. Le premier courriel a été envoyé en 1971\footnote{«~Le
  premier courriel a été envoyé en 1971~»~:
  \url{https://web.archive.org/web/20060506003539/https://openmap.bbn.com/~tomlinso/ray/firstemailframe.html}.}.
Les listes de diffusion sont également apparues rapidement avec le
développement de logiciels permettant d'envoyer le même message à un
ensemble de personnes. Le logiciel LISTSERV est ainsi sorti en
1986\footnote{«~Le logiciel LISTSERV est ainsi sorti en 1986~»~: L-Soft,
  \emph{History of LISTSERV}~:
  \url{https://www.lsoft.se/corporate/history-listserv.asp}.}, Majordomo
en 1992, GNU Mailman en 1999.

Un autre cas d'utilisation est l'émergence de forums de discussions, qui
permettaient aux gens de discuter publiquement de sujets spécifiques.
Usenet, un réseau de forums de discussion, a ainsi été lancé en 1980 et
est devenu entièrement compatible avec Internet en 1986. L'utilisateur y
accédait par un logiciel appelé un lecteur de nouvelles. Usenet a été
très populaire à la fin des années 80 et au cours des années 90,
notamment grâce aux universités. C'est de Usenet que provient le concept
de «~septembre éternel~», qui fait référence au mois de septembre 1993,
durant lequel de nombreux nouveaux utilisateurs étaient arrivés, faisant
drastiquement baisser la qualité du discours, tant au niveau du fond que
de la forme\footnote{«~Le mois de septembre 1993 entrera dans l'histoire
  du net comme le mois de septembre qui n'a jamais pris fin.~» -- Dave
  Fischer, \emph{Re: longest USENET thread ever}, /01/1994 01:58:52
  UTC~:
  \url{https://groups.google.com/g/alt.folklore.computers/c/wF4CpYbWuuA/m/jS6ZOyJd10sJ}.}.
Usenet a été la cause du développement des premiers fournisseurs d'accès
à Internet (FAI), qui permettaient à leurs clients d'y accéder sans
restrictions, sans matériel nécessaire, contre le paiement d'un
abonnement. Notons enfin qu'Usenet a été cité par Satoshi Nakamoto dans
le livre blanc de Bitcoin et dans plusieurs de ses messages, ce qui
témoigne de son influence dans la cyberculture.

C'est également à cette époque qu'est apparu le protocole de
communication textuelle IRC (pour \emph{Internet Relay Chat}), qui
permettait à des individus d'échanger des messages en temps réel.

Mais l'évènement vraiment déterminant dans le développement d'Internet a
été l'arrivée du Web, qui a réellement encouragé l'afflux du grand
public. Celui-ci a été conçu en 1989 par le chercheur Tim Berners-Lee
pour le compte du CERN, qui a été aidé par l'ingénieur Robert Cailliau
pour en définir les spécificités\footnote{«~{[}le Web{]} conçu en 1989
  par le chercheur Tim Berners-Lee {[}...{]} aidé par l'ingénieur Robert
  Cailliau pour en définir les spécificités~»~: Tim Berners-Lee, Robert
  Cailliau, \emph{WorldWideWeb: Proposal for a HyperText Project}, 12
  novembre 1990~:
  \url{https://cds.cern.ch/record/2639699/files/Proposal_Nov-1990.pdf}.}.
Le modèle a été finalement rendu public en août 1991\footnote{«~rendu
  public en août 1991~»~: Tim Berners-Lee, \emph{Re: Qualifiers on
  Hypertext links...}, /08/1991 14:56:20 UTC~:
  \url{https://www.w3.org/People/Berners-Lee/1991/08/art-6484.txt}.}.

Le World Wide Web, abrégé communément en Web, et parfois appelé «~la
Toile\footnote{«~la Toile~»~: L'image de la toile d'araignée qui a donné
  son nom au Web vient des hyperliens qui lient les pages web entre
  elles.}~» en français, est un système hypertexte public fonctionnant
sur Internet, c'est-à-dire un système permettant de passer d'une page à
l'autre (via des hyperliens) sans devoir revenir à la racine. Même si
l'idée n'était pas nouvelle (le concept d'hypertexte avait été inventé
par Ted Nelson en 1965, dans le cadre de son projet Xanadu\footnote{«~le
  concept d'hypertexte avait été inventé par Ted Nelson en 1965, dans le
  cadre de son projet Xanadu~»~: Theodor H. Nelson, «~\emph{A File
  Structure for The Complex, The Changing and the Indeterminate}~», in
  \emph{ACM Proceedings of the 20th National Conference}, 1965,
  pp.~84--100~:
  \url{https://blogs.baruch.cuny.edu/art3057fall2010/files/2010/08/Nelson-AFileStructureForThe-ComplexTheChangingAndTheIndeterminate.pdf}.}),
le Web innovait par trois caractéristiques~: les adresses sous forme
d'URL, le protocole de communication HTTP, et le langage informatique
HTML.

L'accès à la Toile se faisait par le biais d'un navigateur Web développé
par Berners-Lee, baptisé WorldWideWeb, qui ne constituait guère plus
qu'une preuve de concept. Ainsi, le Web n'a vraiment décollé que grâce
aux navigateurs Mosaic, créé en 1993, et surtout Netscape, conçu en
1994. Le Web a engendré un engouement sans précédent, notamment grâce à
l'idée du commerce électronique. Cela a finalement abouti à une bulle
financière appelée la bulle Internet (que les anglophones nomment la
\emph{dot-com bubble}), qui a éclaté en mars 2000.

Les années 2000 ont aussi été marquées par le développement du partage
de fichiers en pair-à-pair. En 1999, Napster permettait de partager de
la musique avec d'autres utilisateurs. Néanmoins, il reposait sur un
serveur central pour référencer les fichiers, ce qui l'a contraint à
fermer en 2001 sous la pression de la RIAA, l'association représentant
l'industrie du disque aux États-Unis.

Afin de résoudre ce problème, des protocoles purement pair à pair sont
apparus. Il s'agissait de créer un réseau où tous les ordinateurs
(appelés nœuds) possédaient le même niveau de privilège, par opposition
au modèle client-serveur, de sorte qu'il n'y ait plus de point de
défaillance unique à attaquer pour faire cesser le partage. C'était le
cas de Gnutella et de eDonkey, tous deux créés en 2000. Mais surtout
c'était le cas de BitTorrent, dont la première version a été publiée en
2001\footnote{Bram Cohen, \emph{BitTorrent - a new P2P app}, 2 juillet
  2001, archive~:
  \url{https://web.archive.org/web/20080129085545/http://finance.groups.yahoo.com/group/decentralization/message/3160}.}.
Ces protocoles formaient une alternative beaucoup plus fiable puisqu'il
fallait poursuivre chaque utilisateur individuellement, ce qui
représentait une charge considérable pour l'État\footnote{«~ce qui
  représentait une charge considérable pour l'État~»~: Marc Rees,
  \emph{Hadopi~: 82 millions d'euros de subventions publiques, 87 000
  euros d'amendes}, 3 août 2020~:
  \url{https://www.nextinpact.com/article/30433/109205-hadopi-82-millions-deuros-subventions-publiques-87000-euros-damendes}.}.

Une dernière innovation a été le routage en ognon qui venait ajouter de
la confidentialité dans la transmission de données. Le routage en ognon
a été inventé en 1996 par Paul Syverson, aux côtés de David Goldschlag
et Michael Reed, pour le compte du \emph{Naval Research Laboratory}, un
laboratoire de recherche rattaché à la Navy\footnote{David M.
  Goldschlag, Michael G. Reed, Paul F. Syverson, «~\emph{Hiding Routing
  Information}~», in \emph{Proceedings of the First International
  Workshop on Information Hiding}, mai 1996, pp.~137---150~:
  \url{https://www.onion-router.net/Publications/IH-1996.pdf}.}. Les
trois hommes avaient pour mission de construire un réseau de mélange
pour protéger les communications des agences étasuniennes. La mise en
œuvre de cette technique a été réalisée quelques années plus tard, par
le biais du réseau Tor, dont le nom est l'acronyme de \emph{The Onion
Router} et qui a été lancé en 2002 grâce à une subvention de la DARPA.
Il a été rendu public en 2003, afin d'agrandir l'ensemble d'anonymat
dans lequel pouvait se fondre les communications fédérales. Cela avait
l'avantage de créer un réseau anonyme dans lequel pouvaient œuvrer les
hors-la-loi.

Internet, et plus particulièrement le partage de pair à pair et le
routage en ognon, semblaient donner la possibilité aux gens de continuer
leurs activités malgré le réticence des autorités en charge, de sorte
qu'elles ont inspiré la conception originelle de Bitcoin. Par son
architecture distribuée, le réseau permettait de répartir les risques
pour ne pas subir une attaque qui puisse mettre le système à genoux.
Satoshi écrivait ainsi dans son courriel du 6 novembre 2008~:

«~Les États sont bons pour couper les têtes des réseaux contrôlés de
manière centralisée comme Napster, mais les réseaux purement pair à pair
comme Gnutella et Tor semblent tenir le coup\footnote{Satoshi Nakamoto,
  \emph{Re: Bitcoin P2P e-cash paper}, /11/2008 20:15:40 UTC~:
  \url{https://www.metzdowd.com/pipermail/cryptography/2008-November/014823.html}.}.~»

\section*{La philosophie du logiciel
libre}\label{la-philosophie-du-logiciel-libre}
\addcontentsline{toc}{section}{La philosophie du logiciel libre}

\markright{La philosophie du logiciel libre}

La possibilité de diffusion des informations apportée par l'émergence
d'Internet a remis au goût du jour la critique à l'encontre de la
«~propriété intellectuelle~», c'est-à-dire du monopole intellectuel
exercé par certaines personnes sur certaines idées. En effet, il
devenait facile d'accéder à l'information et de la propager ce qui
rendait l'application de cette propriété beaucoup plus complexe. De ce
fait, pour un certain nombre de personnes, les restrictions liées à ce
monopole paraissaient totalement absurdes\footnote{«~les restrictions
  liées à ce monopole paraissaient totalement absurdes~»~: Richard M.
  Stallman, \emph{The GNU Manifesto}, mars 1985~:
  \url{https://www.gnu.org/gnu/manifesto.en.html}.}.

Le «~propriété intellectuelle~» est un privilège accordé à un acteur
économique sur une production de l'esprit, qui peut être une invention
industrielle (auquel cas on parle de brevet) ou une création littéraire
ou artistique (auquel cas on parle de droit d'auteur). Il ne s'agit pas
simplement d'autoriser l'auteur d'une invention ou d'une œuvre à
l'utiliser ou à la diffuser~; il s'agit d'interdire à tous les autres de
l'utiliser ou de la diffuser sans son autorisation.

Le monopole intellectuel est par essence contraire au droit naturel en
raison de l'absence de rareté liée à l'information\footnote{N. Stephan
  Kinsella, «~\emph{Against Intellectual Property}~», in \emph{Journal
  of Libertarian Studies}, vol.~15, no. 2, 2001, pp.~1--53~:
  \url{https://cdn.mises.org/Against\%20Intellectual\%20Property_2.pdf}.}.
Pour le dire autrement, copier n'est pas voler. Thomas Jefferson, qui a
pourtant participé à l'établissement du bureau américain des brevets,
écrivait ainsi en 1813~:

«~Celui qui reçoit une idée de moi reçoit un savoir sans diminuer le
mien~; tout comme celui qui allume sa bougie à la mienne reçoit la
lumière sans me plonger dans la pénombre\footnote{Thomas Jefferson,
  \emph{Letter to Isaac McPherson}, 13 août 1813~:
  \url{https://press-pubs.uchicago.edu/founders/documents/a1_8_8s12.html}.}.~»

Le monopole intellectuel permet à des personnes de toucher des
redevances sans avoir signé un quelconque contrat avec celui qui les
paie. Il encourage la consolidation de l'activité économique en grandes
entreprises. Dans le domaine informatique, il a permis à des sociétés de
devenir de grands empires reposant sur le paiement de licences de leurs
logiciels «~propriétaires~». L'exemple le plus parlant est celui de Bill
Gates et de son entreprise Microsoft\footnote{«~'exemple le plus parlant
  est celui de Bill Gates et de son entreprise Microsoft~»~: William
  Henry Gates \textsc{iii}, \emph{An Open Letter to Hobbyists}, 3
  février 1976~:
  \url{https://en.wikisource.org/wiki/Open_Letter_to_Hobbyists}.}. Il
permet de contrôler l'utilisation d'une œuvre, et par conséquent
d'influencer la culture d'une société.

La manière légale de s'opposer à cet ensemble de privilèges dans
l'informatique a été l'émergence des licences libres. Celles-ci
permettaient de prendre l'adversaire à son propre jeu en publiant un
contenu sous une licence interdisant à quiconque de se l'approprier ou
de l'inclure dans un contenu non libre. Ces licences ont émergé dans le
cadre du développement logiciel, qui était soumis au droit d'auteur aux
États-Unis.

Le mouvement a été initié dans les années 80 par Richard Stallman, un
physicien ayant grandi à New York et ayant étudié à Harvard. Ce dernier
avait travaillé pour le département de recherche en intelligence
artificielle au MIT où il avait été introduit à la culture des
\emph{hackers} et fait l'expérience des problématiques posées par les
licences dans le cadre du développement du langage LISP.

Il a fondé le projet GNU en 1983 dans le but de concevoir une
alternative entièrement libre au système d'exploitation UNIX\footnote{«~projet
  GNU~»~: GNU est un acronyme récursif signifiant «~\emph{GNU's Not
  Unix}~».}. Le projet a été lancé par un courriel diffusé sur le forum
Usenet net.unix-wizards\footnote{«~courriel diffusé sur le forum Usenet
  net.unix-wizards~»~: Richard M. Stallman, \emph{new UNIX
  implementation}, 27 septembre 1983~:
  \url{https://groups.google.com/g/net.unix-wizards/c/8twfRPM79u0/m/1xlglzrWrU0J}.}.
En 1985, il écrivait le manifeste GNU\footnote{Richard M. Stallman,
  \emph{The GNU Manifesto}, mars 1985~:
  \url{https://www.gnu.org/gnu/manifesto.en.html}.} et fondait la
\emph{Free Software Foundation}, ce qui marquait la naissance du
mouvement du logiciel libre, et de la mouvance libriste en général.

Richard Stallman a formellement décrit la notion de logiciel libre pour
la première fois en 1986, au sein du premier bulletin d'informations de
GNU, qu'il réduisait à deux libertés de base~:

«~Premièrement, la liberté de copier un programme et de le redistribuer
à vos voisins, qu'ils puissent ainsi l'utiliser aussi bien que vous.
Deuxièmement, la liberté de modifier un programme, que vous puissiez le
contrôler plutôt qu'il vous contrôle~; pour cela, le code doit vous être
accessible\footnote{Richard M. Stallman, \emph{What is the Free Software
  Foundation?}, février 1986~:
  \url{https://www.gnu.org/bulletins/bull1.txt}.}.~»

Il a par la suite raffiné cette définition pour qu'elle inclue quatre
libertés fondamentales~: la liberté d'utiliser le code dans n'importe
quel but, la liberté de l'étudier et de le modifier, la liberté de le
distribuer sans restriction et la liberté d'en distribuer des versions
modifiées\footnote{«~raffiné cette définition pour qu'elle inclue quatre
  libertés fondamentales~»~: Richard M. Stallman, \emph{What is Free
  Software?}, v1.11, 21 décembre 2001~:
  \url{https://www.gnu.org/philosophy/free-sw.en.html}.}.

Deux types de licences libres se sont distingués~: le type permissif, et
le type contaminant dit \emph{copyleft}. Le premier type de licence
exigeait que le code soit librement utilisable, copiable, distribuable
et modifiable tout en permettant la réutilisation dans un programme non
libre. Le second type était encore plus restrictif et imposait à tout
programme utilisant le code d'être publié sous la même licence.

La première licence libre a été la licence MIT\footnote{«~licence
  MIT~»~: Gordon Haff, \emph{The mysterious history of the MIT License},
  26 avril 2019~:
  \url{https://opensource.com/article/19/4/history-mit-license}.}, qui a
été développée par le \emph{Massachusetts Institute of Technology} à
partir de 1985. Licence permissive, elle était initialement présente au
sein du protocole de fenêtrage \emph{X Window System} développé
conjointement avec la DEC et IBM. Une version standarde a été publiée en
1987 (X11), puis sa version finale a été publiée en 1998 pour être
utilisée pour la bibliothèque Expat.

Une autre licence permissive à apparaître rapidement a été la licence
BSD, dont la première version a été publiée en 1988 pour distribuer,
comme son nom l'indique, le code du système d'exploitation BSD.
Plusieurs variantes de cette licence ont été publiées au cours des
années~: la licence BSD proprement dite, à 4 clauses, en 1990~; la
licence BSD modifiée, à 3 clauses, en 1999~; la licence FreeBSD, à 2
clauses, en 1999 également~; et la licence BSD à zéro clause en 2013.

La première licence contaminante (\emph{copyleft}) a été la \emph{GNU
General Public License}, plus connue sous l'abréviation de GPL, qui a
été créée par Richard Stallman en février 1989\footnote{«~GPL, qui a été
  créée par Richard Stallman en février 1989~»~: Leonard H. Tower Jr.,
  \emph{New General Public License}, 25 février 1989~:
  \url{https://groups.google.com/g/gnu.announce/c/m0Jjj_64PeQ/m/8xL1xkVKJb8J?pli=1}.}.
Une version 2 a été partagée en 1991 et une version 3 en 2007.

La notion d'\emph{open source} ou de code source ouvert n'est venue
qu'après, avec l'invention du terme par Christine Peterson en 1998 et
l'implication d'Eric Steven Raymond. Le terme se rapportait à l'origine
seulement au logiciel libre, dans le but de lever l'ambiguïté de
l'appellation «~\emph{free software}~» en anglais (\emph{free} signifie
à la fois libre et gratuit). Mais il a fini par désigner tous les
logiciels dont le code source était disponible publiquement, qu'ils
soient publiés sous licence libre ou non.

Le code du prototype de Bitcoin (v0.1) a été publié en 2009 sous licence
MIT. Pour un tel système ouvert, il était en effet nécessaire que le
code soit ouvert. De plus, dans le but de réduire au maximum le contrôle
sur le protocole, il fallait que le logiciel soit libre, comme nous
l'expliquerons dans les chapitres~\hyperref[ch:changement]{10} et
\hyperref[ch:determination]{11}.

\section*{La tendance extropienne}\label{la-tendance-extropienne}
\addcontentsline{toc}{section}{La tendance extropienne}

\markright{La tendance extropienne}

L'évolution technique prodigieuse qui s'est produite durant le
\textsc{xx} siècle, et qui ne s'est pas cantonnée à l'informatique et à
la cryptographie, a fait évoluer la vision du monde des gens et la façon
dont ils envisageaient l'avenir. Le développement de procédés de plus en
plus avancés faisait entrevoir des possibilités inédites pour l'homme,
comme l'amélioration de ses capacités, la conception de machines
perfectionnées, la production de substances psychotropes, le voyage
spatial et la création de mondes virtuels. C'est ce qui a mené à la
fondation du mouvement des extropiens au sein de la Silicon Valley à la
fin des années 80.

L'extropianisme était une philosophie transhumaniste libérale optimiste,
qui préconisait l'utilisation proactive de la technique en vue
d'accroître les capacités humaines individuelles et civilisationnelles.
Cette tendance se fondait sur l'extropie, un terme créé pour l'occasion
pour désigner le principe d'organisation qui s'oppose à l'entropie et
qui forme la base de la vie matérielle\footnote{En thermodynamique,
  l'entropie est une grandeur physique qui caractérise le degré de
  désorganisation d'un système physique. Le deuxième principe de la
  thermodynamique énonce que l'entropie d'un système isolé croît avec le
  temps, ce qui implique que l'entropie de l'univers croît à mesure de
  son vieillissement, et qu'il finira par mourir. L'extropie se
  rapproche ainsi de la néguentropie, la baisse locale d'entropie à
  certains endroits, sans être définie de façon aussi formelle.}. Le
cœur de l'extropianisme était ainsi la survie et la prospérité dans un
univers matériel souvent hostile, résolument entropique et finalement
mortel.

Les extropiens ont été précédés par des individus qui ont par la suite
été présentés comme des «~high-tech hayekians\footnote{Don Lavoie,
  Howard Baetjer, William Tulloh, «~\emph{High-Tech Hayekians: Some
  Possible Research Topics in the Economics of Computation}~», in
  \emph{Market Process}, vol.~8, 1990~:
  \url{http://www.philsalin.com/hth/hth.html}.}~», dont l'économiste et
futuriste Phil Salin, le pionnier des nanotechnologies Eric Drexler et
l'informaticien et programmeur Mark S. Miller. Ceux-ci adhéraient au
principe de l'ordre spontané -- selon lequel le laissez-faire aboutit à
un ordre supérieur à celui décrété par une autorité constructiviste --
qui avait été développé par les économistes de l'école autrichienne et
qui avait été spécialement mis en valeur par le prix Nobel d'économie
Friedrich Hayek. Ayant assisté à l'accélération de la propagation de
l'information apportée par le développement d'Internet, ils anticipaient
l'ordre nouveau qui allait en résulter. Ils ont cherché à construire des
systèmes qui s'inscrivaient dans cette évolution, comme l'\emph{American
Information Exchange} (AMIX), une place de marché automatisée dédiée à
l'information\footnote{«~AMIX {[}...{]} une place de marché automatisée
  dédiée à l'information~»~: John Walker, \emph{Understanding AMIX}, 7
  septembre 1989~:
  \url{https://www.fourmilab.ch/autofile/e5/chapter2_76.html}.}, et le
projet Agorics, un modèle d'échange de calcul informatique\footnote{«~le
  projet Agorics, un modèle d'échange de calcul informatique~»~: K. Eric
  Drexler, Mark S. Miller, «~\emph{Markets and Computation: Agoric Open
  Systems}~», in \emph{The Ecology of Computation}, 1988~:
  \url{https://papers.agoric.com/assets/pdf/papers/markets-and-computation-agoric-open-systems.pdf}.}.

Le mouvement extropien a, lui, été fondé en janvier 1988 par Max T.
O'Connor (futur Max More) et Tom W. Bell (aussi connu sous le nom de Tom
Morrow), deux étudiants en philosophie de l'Université de Californie du
Sud qui partageaient la même passion pour l'anticipation futuriste de
l'évolution du monde. Au cours de l'automne 88, ils ont lancé un
magazine appelé \emph{Extropy}, dans lequel ils présentaient leur
doctrine de manière détaillée et autour duquel le mouvement s'est
ensuite construit. Une liste de diffusion a été mise en place durant
l'été 1991 par Perry Metzger, par l'intermédiaire de laquelle les
extropiens pouvaient échanger par courriel sur des sujets divers. Les
extropiens présents dans la région de la baie de San Francisco ne
manquaient pas non plus de se rencontrer dans la vraie vie, au moyen de
ce qu'ils appelaient des Extropaganzas. Un institut, appelé
l'\emph{Extropy Institute}, a aussi été fondé en mai 1992 dans le but de
faire la promotion des principes extropiens. La première conférence
organisée par l'institut, nommée «~Extro 1~», a eu lieu en avril 1994 à
Sunnyvale dans la Silicon Valley. Elle faisait notamment intervenir,
outre Max More et Tom Bell, le spécialiste en robotique Hans Moravec et
le cryptographe Ralph Merkle\footnote{«~Elle faisait notamment
  intervenir {[}...{]} Ralph Merkle~»~: Ralph C. Merkle,
  «~\emph{Cryonics, Cryptography, and Maximum Likelihood Estimation}~»,
  in \emph{Proceedings of the First Extropy Institute Conference on
  TransHumanist Thought}, 1994~:
  \url{https://www.ralphmerkle.com/merkleDir/cryptoCryo.html}.}. Cette
conférence a été relatée au cours de l'automne par le magazine Wired,
jetant un peu de lumière sur le mouvement\footnote{Ed Regis, \emph{Meet
  the Extropians}, 1 octobre 1994~:
  \url{https://www.wired.com/1994/10/extropians/}.}.

Rationaliste, cette doctrine reposait sur quatre principes, définis en
1990~: l'expansion illimitée, l'auto-transformation, l'optimisme
dynamique et la technologie intelligente\footnote{Max More, «~\emph{The
  Extropian Principles}~», in \emph{Extropy}, vol.~6, 1 juillet 1990~:
  \url{https://github.com/Extropians/Extropy/blob/master/ext6.pdf}.}.
L'extropianisme représentait ainsi un transhumanisme\footnote{«~L'extropianisme
  représentait ainsi un transhumanisme~»~: Max More,
  \emph{Transhumanism: A Futurist Philosophy}~», \emph{Extropy}, vol.~6,
  1 juillet 1990~:
  \url{https://github.com/Extropians/Extropy/blob/master/ext6.pdf}.},
une volonté de transcender la nature humaine, déjà envisagée auparavant
par des personnes comme Julian Huxley, Robert Ettinger et
FM-2030\footnote{«~une volonté de transcender la nature humaine, déjà
  envisagée auparavant par des personnes comme Julian Huxley, Robert
  Ettinger et FM-2030~»~: Julian Huxley, \emph{Transhumanism}~», in
  \emph{New Bottles for New Wine}, 1957~; Robert C. W. Ettinger,
  \emph{Man Into Superman}, 1972~; FM-2030, \emph{Upwingers Manifesto},
  1973~; FM-2030, \emph{Are You a Transhuman?}, 1989.}.

La philosophie extropienne n'était pas seulement descriptive, mais
prescriptive\footnote{«~La philosophie extropienne n'était pas seulement
  descriptive, mais prescriptive~»~: En anglais, les initiales des
  quatre premiers principes extropiens (\emph{Boundless Expansion},
  \emph{Self-Transformation}, \emph{Dynamic Optimism}, \emph{Intelligent
  Technology}) formaient l'acronyme «~BEST DO IT~», c'est-à-dire «~le
  mieux est de le faire~», ce qui montrait la dimension proactive de
  cette philosophie. -- Max More, «~\emph{The Extropian Principles}~»,
  \emph{Extropy}, vol.~6, 1 juillet 1990~:
  \url{https://github.com/Extropians/Extropy/blob/master/ext6.pdf}.}.
Conformément au principe de l'optimisme dynamique (ou pragmatique), les
extropiens souhaitaient intervenir pour accélérer l'avènement de
l'avenir qu'ils anticipaient. Ils promouvaient ainsi la recherche et
l'expérimentation dans les domaines scientifiques qui avaient pour but
d'améliorer la condition matérielle de l'homme.

D'abord, dans leur lutte contre la mort, les extropiens étaient en
particulier enthousiastes à propos de la cryogénisation, c'est-à-dire de
la conservation à très basse température de corps de défunts dans
l'espoir de les ressusciter grâce à un futur progrès technique.
L'\emph{Alcor Life Extension Foundation}, fondée en 1972 par Fred
Chamberlain \textsc{iii} et sa femme, basée en Arizona, était la
principale organisation qui prenait en charge ce type de service.

Les extropiens étaient également ouvertement hostiles à l'autorité. Ils
promouvaient le principe de l'ordre spontané, décentralisé par nature,
par opposition au technocratisme centralisé, qu'ils considéraient comme
ralentissant le progrès technique\footnote{«~Le progrès durable et la
  prise de décision intelligente et rationnelle requièrent des sources
  d'information et des points de vue diversifiés, rendus possibles par
  des ordres spontanés. La direction centrale limite l'exploration, la
  diversité, la liberté et les opinions divergentes. Respecter l'ordre
  spontané, c'est soutenir les institutions volontaristes qui maximisent
  l'autonomie, par opposition aux groupements hiérarchiques rigides et
  autoritaires avec leur structure bureaucratique, la suppression de
  l'innovation et de la diversité, et l'étouffement des incitations
  individuelles. Comprendre les ordres spontanés nous rend très méfiants
  à l'égard des ``autorités'' qui nous sont imposées, et sceptiques à
  l'égard des dirigeants coercitifs, de l'obéissance inconditionnelle et
  des traditions non remises en question.~» -- Max More, «~\emph{The
  Extropian Principles v. 2.0}~», \emph{Extropy}, vol.~9, 1992~:
  \url{https://github.com/Extropians/Extropy/blob/master/ext9.pdf}.}.
Certains extropiens s'inspiraient notamment de l'ouvrage de David
Friedman \emph{The Machinery of Freedom}, qui décrivait comment pouvait
s'organiser une société sans État et dont la seconde édition a été
publiée en 1989.

Ensuite, c'est tout naturellement que la cryptographie forte constituait
un des centres d'intérêt des extropiens. Nécessaire pour préserver leur
liberté, elle constituait une des briques de base pour parvenir à leurs
fins. C'est pourquoi le mouvement extropien était en réalité étroitement
lié au mouvement cypherpunk, lui aussi inspiré par le développement
technique, de nombreuses personnes s'investissant dans les deux, comme
Tim May, Hal Finney (qui a été cryogénisé par la Fondation Alcor en
2014) ou Nick Szabo.

Les extropiens s'intéressaient enfin à la monnaie. Une monnaie solide
était en effet nécessaire pour imaginer pouvoir conserver de la valeur à
très long terme, par exemple dans le cas d'une cryogénisation. Le sujet
était ainsi abordé dans le magazine Extropy. En 1993, Hal Finney a
présenté le fonctionnement du système d'argent liquide électronique
eCash\footnote{Hal Finney, \emph{Protecting privacy with electronic
  cash}, \emph{Extropy}, vol.~10, 1993~:
  \url{https://github.com/Extropians/Extropy/blob/master/Extropy-10.pdf}.}.
En 1995, le numéro 15 de la revue a été ouvertement dédié à la monnaie
électronique et à la concurrence des monnaies, comme l'attestait sa
couverture illustrée par un billet de banque privée à l'effigie de
Hayek\footnote{\emph{Extropy}, vol.~15, 1995~:
  \url{https://github.com/Extropians/Extropy/blob/master/ext15.pdf}.}.

\begin{figure}[H]

{\centering \includegraphics{chapters/img/fifteen-hayeks-note-extropy-15.png}

}

\caption{Le billet fictif de 15 hayeks en couverture du magazine
Extropy.}

\end{figure}%

La monnaie numérique constituait donc l'un des enjeux mis en avant par
les extropiens. Mais ces derniers ne le faisaient pas autant que les
cypherpunks qui, des années plus tard, tenteraient de mettre en pratique
leur connaissance de la cryptographie pour en créer une.

\section*{Le mouvement des
cypherpunks}\label{le-mouvement-des-cypherpunks}
\addcontentsline{toc}{section}{Le mouvement des cypherpunks}

\markright{Le mouvement des cypherpunks}

Le mouvement cypherpunk est apparu en 1992 dans la Silicon Valley. Les
cypherpunks étaient des gens qui prônaient l'utilisation proactive de la
cryptographie en vue d'assurer la confidentialité et la liberté des
individus sur Internet. Ils s'opposaient à la surveillance, à la censure
et à l'exploitation des données personnelles, et préconisaient la
programmation et la publication ouverte de logiciels, préférablement
sous licence libre, dans le but de combattre ces menaces. Leur nom,
calqué sur cyberpunk, était un mot-valise composé des mots anglais
\emph{cypher}, signifiant «~chiffre~» (dans le sens de code secret), et
\emph{punk}, désignant originellement un voyou. Les cypherpunks étaient
donc formellement des rebelles amateurs de cryptographie.

Les cypherpunks s'inspiraient partiellement du cyberpunk, un mouvement
culturel construit autour de la littérature de science-fiction, qui
prenait sa source à la fois dans la sous-culture des punks et dans la
mouvance des hackers. Même si l'esthétique de ce dernier datait de la
fin des années 70, le genre littéraire a largement été inauguré par
l'écrivain William Gibson via la publication de ses premières nouvelles
à partir de 1981 et surtout de son roman \emph{Neuromancien}. Le mot,
qui faisait référence à la cybernétique, c'est-à-dire la science des
systèmes complexes et des réseaux, a quant à lui a été inventé en 1983
par Bruce Bethke\footnote{«~Le mot {[}...{]} a quant à lui a été inventé
  en 1983 par Bruce Bethke~»~: Bruce Bethke, \emph{Cyberpunk: a short
  story by Bruce Bethke}, 1997~:
  \url{http://www.infinityplus.co.uk/stories/cpunk.htm}.}, et a été
popularisé par Gardner Dozois en décembre 1984 dans un éditorial pour le
Washington Post\footnote{«~a été popularisé par Gardner Dozois~»~:
  Gardner Dozois, \emph{Science Fiction in the Eighties}, 30 décembre
  1984~:
  \url{https://www.washingtonpost.com/archive/entertainment/books/1984/12/30/science-fiction-in-the-eighties/526c3a06-f123-4668-9127-33e33f57e313/}.}.

La caractéristique principale du genre cyberpunk était de décrire un
futur dystopique où la technique de pointe était omniprésente (implants
informatiques, réalité augmentée, réalité virtuelle, intelligence
artificielle, robots) et où la société était sujette à la consommation à
outrance (drogue, sexe, etc.), au crime généralisé et à l'avarice des
corporations. Le cyberpunk décrivait ainsi un monde combinant haute
technologie et bassesse humaine, pour reprendre l'expression de Bruce
Sterling\footnote{«~combinant haute technologie et bassesse humaine,
  pour reprendre l'expression de Bruce Sterling~»~: Bruce Sterling,
  «~\emph{Preface}~», in William Gibson, \emph{Burning Chrome}, Arbor
  House, 1986.}, dont le héros tentait de s'extraire tant bien que mal.

De ce genre cyberpunk est né tout un mouvement d'individus qui
partageaient la même vision du monde, formant notamment une
contreculture cyberdélique, née de la fusion de la cyberculture et du
psychédélisme. Cette sous-culture en vogue dans la Silicon Valley était
incarnée par la revue \emph{High Frontiers}, fondée en 1984 par R. U.
Sirius, qui est plus tard devenue \emph{Reality Hackers} puis
\emph{Mondo 2000}.

Les cypherpunks tiraient leur inspiration de ce mouvement. Toutefois,
ils n'étaient pas pour autant des cyberpunks~: s'ils avaient bien
conscience des scénarios dystopiques qui pouvaient dériver de
l'évolution technique (notamment en ce qui concerne la surveillance),
ils ne partageaient pas la vision pessimiste relayée par le cyberpunk.
De ce fait, le mouvement cypherpunk constituait en quelque sorte une
réaction au cyberpunk, dans le sens où il postulait, à l'instar des
extropiens, que l'évolution technique pouvait amener les êtres humains à
s'émanciper plutôt qu'à tomber dans l'esclavage mutuel.

Les cypherpunks basaient en particulier leurs réflexions sur une longue
nouvelle publiée en 1980 par l'auteur de science-fiction Vernor Vinge,
intitulée \emph{True Names}. Cette nouvelle, qui abordait des thèmes
propres au genre cyberpunk sans strictement en faire partie\footnote{«~Cette
  nouvelle, qui abordait des thèmes propres au genre cyberpunk sans
  strictement en faire partie~»~: Vernor Vinge (interrogé par Michael
  Synergy), «~\emph{Hurtling Towards the Singularity}~», in \emph{Mondo
  2000}, issue 1, 1989~:
  \url{https://archive.org/details/Mondo.2000.Issue.01.1989/page/n115/mode/2up}.},
contait l'histoire de Roger Pollack, un individu agissant au sein d'un
groupe de pirates dans un monde virtuel appelé «~\emph{The Other
Plane}~», utilisant le pseudonyme de Mr.~Slippery et faisant attention à
ne surtout pas révéler son «~Vrai Nom~» (à savoir son nom civil) au
risque de subir une «~Vraie Mort~» (par exécution étatique). Cet enjeu
correspondait par conséquent à l'enjeu principal de la cryptographie~:
la préservation de l'anonymat dans le but de conserver sa liberté et,
\emph{in fine}, sa vie.

Les cypherpunks avaient ainsi le regard tourné vers l'avenir. Mais leur
préoccupation concernaient surtout l'avenir proche~: c'était la
confidentialité dans le cyberespace naissant\footnote{Le terme
  «~cyberespace~» (\emph{cyberspace}) a été forgé par William Gibson
  dans sa nouvelle \emph{Gravé sur Chrome} publiée en juillet 1982, pour
  désigner la représentation virtuelle des flux de données sur Internet.
  Le terme «~matrice~» (\emph{matrix}) était utilisé en tant que
  synonyme.}. C'est pourquoi leur mouvement pouvait rassembler des
optimistes et des pessimistes, des extropiens et des cyberpunks, qui
trouvaient du sens dans cette lutte contre la surveillance de masse.

À l'origine, le mouvement cypherpunk a été le fruit de la pensée et de
l'action de Timothy C. May, dit Tim May. Ce dernier était un
scientifique, ingénieur et informaticien né en 1951 en périphérie de
Washington D.C. Passionné de science-fiction et de physique, il avait
travaillé pour Inter de 1974 où il avait contribué à résoudre le
problème des particules alpha dans les circuits intégrés\footnote{«~il
  avait contribué à résoudre le problème des particules alpha dans les
  circuits intégrés~»~: Timothy C. May, Murray H. Woods,
  \emph{Alpha-Particle-Induced Soft Errors in Dynamic Memories}, janvier
  1979~: \url{https://gwern.net/doc/cs/hardware/1979-may.pdf}.}. Il
avait accumulé une certaine fortune au cours de ces années, si bien
qu'en 1986, à l'âge de 35 ans, il a décidé de prendre sa retraite pour
se consacrer à ses passions politiques.

Tim May a rencontré Phil Salin en 1987, avec qui il a pu discuter des
implications de la cryptographie. Ses discussions avec Salin, ainsi
qu'avec d'autres personnes comme Marc Stiegler, l'ont poussé à écrire le
\emph{Manifeste crypto anarchiste} en août 1988\footnote{«~Ses
  discussions avec Salin, ainsi qu'avec d'autres personnes comme Marc
  Stiegler, l'ont poussé à écrire le \emph{Manifeste crypto anarchiste}
  en août 1988~»~: Timothy C. May, \emph{Cyphernomicon}, 16.3.4.}. Dans
ce manifeste, il posait les bases de ce qui allait devenir la doctrine
des cypherpunks et décrivait le potentiel d'émancipation individuelle
apporté par la cryptographie et par l'anonymat. Le manifeste, pastiche
ironique du \emph{Manifeste du parti communiste}, décrivait comment
l'avènement des méthodes cryptographiques modernes allait, d'après lui,
déstabiliser l'État en permettant aux individus d'échanger librement de
l'information et de la richesse. En particulier, il écrivait~:

«~Tout comme la technique de l'imprimerie a altéré et réduit le pouvoir
des corporations médiévales et la structure sociale de pouvoir, les
méthodes cryptologiques altèrent fondamentalement la nature de
l'interférence de l'État et des grandes entreprises dans les
transactions économiques\footnote{Timothy C. May, \emph{The Crypto
  Anarchist Manifesto}, /11/1992 20:11:24 UTC~:
  \url{https://cypherpunks.venona.com/date/1992/11/msg00204.html}.}.~»

Tim May n'était pas seul à penser de cette manière et communiquait avec
d'autres personnes qui partageaient ses idées. C'était le cas de son ami
Eric Hughes, un jeune mathématicien et programmeur ayant grandi dans une
famille mormone en Virginie près de Washington et à Salt Lake City. Ce
dernier avait travaillé brièvement pour DigiCash à Amsterdam avant de
revenir sur la côte Ouest\footnote{«~Ce dernier avait travaillé
  brièvement pour DigiCash à Amsterdam avant de revenir sur la côte
  Ouest~»~: Timothy C. May, \emph{Hackers Conference Report}, /11/1992
  08:55:26 UTC~:
  \url{https://cypherpunks.venona.com/date/1992/11/msg00019.html}.}. En
mai 1992, alors qu'il cherchait à emménager dans la Silicon Valley, lui
et Tim May ont longuement discuté de cryptographie, à tel point qu'ils
ont décidé de reproduire ce type d'échange avec un plus grand nombre de
personnes en organisant des réunions physiques.

La première réunion du mouvement cypherpunk a ainsi eu lieu au cours de
la journée du 19 septembre 1992, dans la maison d'Eric Hughes à Oakland.
L'accès à cette réunion se faisait uniquement sur invitation afin de
préserver la discrétion du groupe. Libertariens pour la plupart,
extropiens pour certains, les invités étaient des connaissances de May
et Hughes issues de la communauté des hackers et des entreprises
informatiques de la région. Durant la réunion, Tim May y a lu le
\emph{Manifeste crypto anarchiste}. En guise d'animation, les personnes
conviées ont également participé à un «~jeu de la crypto anarchie~», qui
consistait à simuler un réseau de mélange par l'échange et l'ouverture
d'enveloppes de papier\footnote{Certains détails de la formation des
  cypherpunks sont issus de l'ouvrage \emph{Crypto: How the Code Rebels
  Beat the Government--Saving Privacy in the Digital Age} (pp.~257 --
  266) de Steven Levy publié en 2001.}.

Parmi les invités se trouvait John Gilmore, un informaticien américain
connu pour avoir été l'un des premiers employés de Sun Microsystems. Il
avait aussi co-créé la hiérarchie ouverte alt.* sur Usenet et était un
contributeur majeur du projet GNU. Alors en retraite anticipé depuis
1986, tout comme May, il s'était engagé dans l'activisme dans le but de
protéger les libertés civiles sur Internet. En 1989, il avait co-fondé
Cygnus Support, une entreprise spécialisée dans le support professionnel
de composants fondés sur GNU. Il avait également participé à la création
de l'\emph{Electronic Frontier Foundation} (EFF), une ONG internationale
de protection des libertés sur Internet, aux côtés de Mitch Kapor et de
John Perry Barlow en 1990. Lui aussi voyait la cryptographie comme une
moyen de libération individuelle\footnote{«~Et si nous pouvions
  construire une société dans laquelle les informations ne seraient
  jamais collectées~? {[}...{]} C'est le genre de société que je veux
  construire. Je veux que soit garantie - par la physique et la
  mathématique, pas par des lois - la possibilité de bénéficier de
  choses telles qu'une véritable confidentialité des communications
  personnelles, {[}...{]} une véritable confidentialité des
  enregistrements personnels, {[}...{]} une véritable liberté de
  commerce, {[}...{]} une véritable confidentialité financière {[}et{]}
  un véritable contrôle de l'identification.~» -- John Gilmore,
  \emph{Privacy, Technology, and the Open Society}, 28 mars 1991~:
  \url{http://www.toad.com/gnu/cfp.talk.txt}~; archive~:
  \url{https://web.archive.org/web/19991003163945/http://www.toad.com/gnu/cfp.talk.txt}.}.

Une autre personne présente durant cette réunion fondatrice était
l'activiste Judith Milhon, une femme née en 1939 qui avait participé au
mouvement des droits civiques pour l'abolition des discriminations
raciales dans les années 60 et avait été emprisonnée pour désobéissance
civile\footnote{«~Judith Milhon, une femme née en 1939 qui avait
  participé au mouvement des droits civiques~»~: Sean Dodson,
  \emph{Judith Milhon: making the Internet a feminist issue}, 8 août
  2003~:
  \url{https://www.theguardian.com/technology/2003/aug/08/guardianobituaries.obituaries}.}.
Programmeuse, hackeuse, elle était alors la co-éditrice de la revue
cyberpunk \emph{Mondo 2000}, à laquelle elle participait sous le nom de
plume de St.~Jude. Elle était également la compagne d'Eric Hughes,
malgré leur grande différence d'âge.

C'est elle qui a donné leur nom aux cypherpunks lors de cette réunion,
sur le ton de la plaisanterie. «~Je pense que vous êtes des
cryptoanarchistes -- ce que j'appellerais des cypherpunks~!~», a-t-elle
écrit par la suite\footnote{Judith Milhon, \emph{secretions}, /09/1992
  10:01:26 UTC~:
  \url{https://cypherpunks.venona.com/date/1992/09/msg00013.html}.}. Le
terme capturait bien l'esprit de la cryptoanarchie, tout en donnant au
mouvement un côté moins formel et dogmatique. En effet, les gens
préoccupés par ces enjeux n'étaient pas tous anarchistes~: ils pouvaient
s'opposer fermement à l'autoritarisme et à la surveillance, sans pour
autant vouloir remettre en cause les fondements même de
l'État\footnote{«~De plus, cela donne à tort l'impression que
  ``cypherpunk'' est synonyme d'\,``anarchiste''. Il se trouve que je
  suis anarchiste, mais ce n'est pas ce en quoi croient la plupart des
  personnes associées au terme ``cypherpunk'', et il n'est pas juste de
  les dépeindre ainsi -- bon sang, de nombreuses personnes sur cette
  liste de diffusion sont ouvertement hostiles à l'anarchisme. Je ne
  veux pas que les gens pensent qu'il faut détester l'idée même d'État
  pour aimer la cryptographie.~» -- Perry E. Metzger, \emph{Re: PC Expo
  summary!!}, /07/1994 12:13:09 UTC~:
  \url{https://cypherpunks.venona.com/date/1994/07/msg00014.html}.}.
C'est ce côté informel qui a fait que le terme a été adopté
immédiatement.

Après la réunion, Eric Hughes, avec l'aide de Hugh Daniel, a créé une
liste de diffusion de courrier électronique nommée «~Cypherpunks~». Le
courriel de bienvenue\footnote{«~courriel de bienvenue~»~: Eric Hughes,
  \emph{No Subject}, Sep 92 05:43:46 UTC~:
  \url{https://cypherpunks.venona.com/date/1992/09/msg00001.html}.} a
été envoyé dans la soirée du 21 septembre (PDT). La liste était relayée
par le serveur associé au nom de domaine `` appartenant à John Gilmore.
Ce dernier a aussi offert la disponibilité des locaux de Cygnus pour les
réunions ultérieures.

La liste a accueilli de nombreuses discussions relatives à la
cryptographie et à son utilisation concrète, dont notamment l'argent
liquide électronique. Beaucoup de gens sont intervenus dès les premiers
mois, comme par exemple l'ancien pirate téléphonique John Draper. En un
an à peine, la liste recensait ainsi plus de 500 participants.

L'un de ces participants était Harold T. Finney \textsc{ii}, dit Hal
Finney, informaticien et cryptographe américain, diplômé de Caltech et
programmeur de jeux vidéos pour les consoles Intellivision et Atari VCS.
Extropien et enthousiasmé par la popularisation d'Internet, il était
obsédé par la cryptographie, à tel point qu'il était rentré en contact
avec Phil Zimmermann pour travailler avec lui sur la version 2.0 de PGP,
sortie le 2 septembre 1992. Hal Finney était aussi fasciné par les idées
de David Chaum. En novembre 1992, il écrivait à la liste de diffusion~:

«~Nous voici confrontés aux problèmes de la perte de confidentialité, de
l'informatique trompeuse, des bases de données massives, de
l'augmentation de la centralisation -- et Chaum propose une direction à
suivre complètement différente, une direction qui met le pouvoir entre
les mains des individus plutôt que celles des États et des grandes
entreprises. L'ordinateur peut être utilisé comme un outil pour libérer
et protéger les personnes, plutôt que pour les contrôler\footnote{Hal
  Finney, \emph{Why remailers...}, /11/1992 01:30:02 UTC~:
  \url{https://cypherpunks.venona.com/date/1992/11/msg00108.html}.}.~»

La vision des cypherpunks était claire~: mettre en pratique ce qui avait
été jusque-là de vagues spéculations. Il était en effet stérile de
théoriser des choses si cela ne se traduisait pas par des actions
concrètes. Cet esprit pratique a été parfaitement résumé par Eric Hughes
dans son \emph{Manifeste d'un Cypherpunk} envoyé à la liste de diffusion
en mars 1993, où il écrivait alors~:

«~Nous devons défendre notre propre vie privée si nous voulons en avoir
une. Nous devons nous rassembler et créer des systèmes qui rendent
possibles les transactions anonymes. Depuis des siècles, les gens
défendent leur vie privée par des chuchotements, par l'obscurité, par
des enveloppes, des portes fermées, des poignées de main secrètes et des
messagers. Les techniques du passé ne permettaient pas une forte
confidentialité, mais les techniques électroniques le permettent.

Nous, les Cypherpunks, nous consacrons à construire des systèmes
anonymes. Nous défendons notre confidentialité avec la cryptographie,
avec les systèmes anonymes de transfert de courriels, avec les
signatures numériques, et avec la monnaie électronique.

Les Cypherpunks écrivent du code. Nous savons que quelqu'un doit écrire
un logiciel pour défendre la vie privée, et puisque nous ne pouvons pas
avoir de vie privée si nous ne le faisons pas tous, nous allons
l'écrire. Nous publions notre code pour que nos collègues Cypherpunks
puissent le mettre en pratique et expérimenter avec. Notre code est
libre d'utilisation pour tous, dans le monde entier. Nous ne nous
soucions guère que vous n'approuviez pas les logiciels que nous
écrivons. Nous savons que les logiciels ne peuvent pas être détruits et
qu'un système largement dispersé ne peut pas être arrêté\footnote{Eric
  Hughes, \emph{RANTS: A Cypherpunk's Manifesto}, /03/1993 19:51:06
  UTC~: \url{https://cypherpunks.venona.com/date/1993/03/msg00392.html}.}.~»

Deux mois plus tard, en mai 93, le mouvement était définitivement
lancé~: les cypherpunks faisaient la une du magazine Wired, récemment
fondé dans le but de parler de l'incidence culturelle, économique et
politique des techniques émergentes. Tim May, Eric Hughes et John
Gilmore apparaissaient masqués sur la couverture, et un long article
détaillait leurs idées et leurs revendications\footnote{Steven Levy,
  «~\emph{Crypto Rebels}~», \emph{Wired}, 1 février 1993~:
  \url{https://www.wired.com/1993/02/crypto-rebels/}. -- Par la suite,
  le mouvement a également été présenté dans les revues \emph{Whole
  Earth Review} et \emph{The Village Voice}.}. C'était la préfiguration
du rôle qu'ont joué par la suite les cypherpunks dans la sauvegarde de
la liberté sur Internet.

\section*{L'action des cypherpunks pour la
liberté}\label{laction-des-cypherpunks-pour-la-libertuxe9}
\addcontentsline{toc}{section}{L'action des cypherpunks pour la liberté}

\markright{L'action des cypherpunks pour la liberté}

Le mouvement cypherpunk est né juste après le triomphe des États-Unis
dans la guerre froide les opposant à l'URSS et au début de l'adoption
d'Internet par le grand public, amorcée notamment par la popularisation
de Usenet et par l'apparition du World Wide Web. Il est apparu en
quelque sorte «~au bon moment~» pour accompagner cette mutation majeure
qui a marqué le monde entier.

Le premier accomplissement des cypherpunks a été leur intervention dans
la guerre contre la cryptographie orchestrée par l'État fédéral
étasunien. Cette guerre a été inaugurée en février 1993 par la bataille
contre PGP, lorsque Phil Zimmermann a été poursuivi en justice pour en
avoir publié les deux premières versions en ligne, l'exportation de
produits cryptographiques sans licence étant prohibée par la
réglementation américaine (ITAR).

Cette décision a naturellement suscité une forte réaction de la part des
cypherpunks qui, en réponse à la tentative d'application de cette
réglementation absurde, se sont mis à partager le code de chiffrement
dans une démarche de désobéissance civile. Le jeune britannique Adam
Back l'a ainsi fait imprimer sur des t-shirts qu'il distribuait aux
autres et certains ont été jusqu'à se le tatouer sur leur
corps\footnote{«~Adam Back l'a ainsi fait imprimer sur des t-shirts
  qu'il distribuait aux autres et certains ont été jusqu'à se le tatouer
  sur leur corps~»~: \url{http://www.cypherspace.org/adam/rsa/}.}. En
1995, Phil Zimmermann a publié la version 2.6.2 de PGP dans un
livre\footnote{«~Phil Zimmermann a publié la version 2.6.2 de PGP dans
  un livre~»~: Philip R. Zimmermann, \emph{PGP: Source Code and
  Internals}, 1995~:
  \url{https://philzimmermann.com/EN/essays/BookPreface.html}.}, dans le
but de réduire au maximum la distinction entre le code et l'expression,
cette dernière étant protégée par le premier amendement de la
Constitution des États-Unis.

Les charges contre Zimmermann ont finalement été abandonnées en 1996,
notamment grâce au soutien de membres du MIT\footnote{«~grâce au soutien
  de membres du MIT~»~: Steven Levy, \emph{Cypher Wars}, 1 novembre
  1994~: \url{https://www.wired.com/1994/11/cypher-wars/}.}. Cela lui a
permis de créer son entreprise pour travailler sur PGP et engager des
employés, comme Hal Finney. En novembre de la même année, Bill Clinton
signait l'Ordre exécutif 13026 qui assouplissait considérablement les
restrictions sur l'exportation des produits cryptographiques.

Cependant, la guerre contre la cryptographie ne s'arrêtait pas là. En
effet, elle ne concernait pas que l'interdiction d'utiliser la
cryptographie forte, mais également l'obligation pour les constructeurs
de matériel informatique d'intégrer des portes dérobées dans leurs
produits. Le projet de loi sénatoriale 266, proposé par Joe Biden en
1991, devait ainsi faire en sorte que tous les appareils de
communication puissent être surveillés par l'État fédéral~:

«~Le Congrès estime que les fournisseurs de services de communications
électroniques et les fabricants d'équipements de services de
communications électroniques doivent veiller à ce que les systèmes de
communications permettent au gouvernement d'obtenir le contenu en texte
clair des communications vocales, de données et autres lorsque la loi
l'autorise de manière appropriée\footnote{\emph{Comprehensive
  Counter-Terrorism Act of 1991}, 24 janvier 1991~:
  \url{https://www.congress.gov/bill/102nd-congress/senate-bill/266/text}.}.~»

Ce projet s'est matérialisé le 16 avril 1993 par l'annonce de la puce
Clipper par la Maison-Blanche, un cryptoprocesseur servant à chiffrer
les messages vocaux et les données, qui implémentait (au moyen de son
algorithme Skipjack) un dispositif d'autorité de séquestre permettant
aux agences étasuniennes de déchiffrer les communications au besoin.
Cette puce était développée et produite par la NSA et était destinée à
équiper les appareils électroniques vendus au grand public. La
Maison-Blanche se justifiait en prétendant que la puce pourrait «~à la
fois fournir aux citoyens respectueux de la loi un accès au chiffrement
dont ils ont besoin et empêcher les criminels de l'utiliser pour cacher
leurs activités illégales\footnote{The White House, \emph{White House
  Annoucement of the Clipper Initiative}, 16 avril 1993~:
  \url{https://groups.csail.mit.edu/mac/classes/6.805/articles/crypto/clipper-announcement.html}.}~».

Cette annonce a provoqué une levée de boucliers chez les cypherpunks qui
y voyaient un projet orwellien et s'y sont opposés en bloc. Cependant,
la lutte n'a pas été longue~: en juin 1994, le cypherpunk Matt Blaze a
découvert une vulnérabilité au sein du dispositif d'autorité de
séquestre, qui rendait le dispositif inefficace et permettait à la puce
d'être utilisée pour chiffrer les données normalement\footnote{«~Matt
  Blaze a découvert une vulnérabilité au sein du dispositif d'autorité
  de séquestre~»~: John Markoff, \emph{At AT\&T, No Joy on Clipper
  Flaw}, 3 juin 1994~:
  \url{https://www.nytimes.com/1994/06/03/business/at-at-t-no-joy-on-clipper-flaw.html}~;
  Matt Blaze, \emph{Paper available via ftp}, /06/1994 00:01:57 UTC~:
  \url{https://cypherpunks.venona.com/date/1994/06/msg00319.html}.}. À
partir de là, le projet a perdu progressivement en ampleur pour être
définitivement abandonné en 1996. La liberté avait gagné, au moins
temporairement\footnote{Cette victoire contre la puce Clipper n'a pas
  empêché les agences étasuniennes d'espionner leur propre population de
  manière massive, comme l'ont montré les révélations d'Edward Snowden
  en 2013. -- Voir Glenn Greenwald, \emph{NSA collecting phone records
  of millions of Verizon customers daily}, 6 juin 2013~:
  \url{https://www.theguardian.com/world/2013/jun/06/nsa-phone-records-verizon-court-order}.}.

Comme on l'a observé, l'optique des cypherpunks était d'être dans
l'action, d'écrire du code et de partager des programmes qui puissent
être utilisés. Ils se sont donc focalisés sur la construction de
systèmes axés sur trois aspects majeurs~: la protection de la vie
privée, la diffusion de l'information et le commerce en ligne.

Le premier domaine d'innovation a été celui des serveurs de courriel
anonyme, qui permettaient de retransmettre les courriers électroniques
de façon à masquer l'identité de leur expéditeur. Le premier serveur de
ce type a mis en place par Eric Hughes et Hal Finney pour la liste des
cypherpunks dès octobre 1992, et il utilisait PGP pour le
chiffrement\footnote{«~Le premier serveur de ce type a mis en place par
  Eric Hughes et Hal Finney pour la liste des cypherpunks dès octobre
  1992~»~: Hal Finney, \emph{New remailer...}, /10/1992 20:31:48 UTC~:
  \url{https://cypherpunks.venona.com/date/1992/10/msg00082.html}.}. En
1994, Lance Cottrell a amélioré la chose en proposant le modèle
Mixmaster, qui permettait d'envoyer des courriels par paquets de taille
fixe et de les réordonner, pour empêcher le traçage des courriels par la
surveillance de l'activité du serveur\footnote{«~Lance Cottrell a
  amélioré la chose en proposant le modèle Mixmaster~»~: Lance Cottrell,
  \emph{1st Draft Mixmaster chaining instructions}, /11/1994 01:07:02
  UTC~: \url{https://cypherpunks.venona.com/date/1994/11/msg00158.html}.}.

Outre le courriel, l'objectif des cypherpunks était de rendre la
navigation sur Internet plus anonyme, le fonctionnement du Web étant
trop transparent. C'était l'idée des frères Austin et Hamnett Hill qui
ont lancé le réseau Freedom en 1999 par l'intermédiaire de leur
entreprise Zero-Knowledge Systems, qui employait notamment les
cypherpunks Ian Goldberg et Adam Back\footnote{«~Austin et Hamnett Hill
  qui ont lancé le réseau Freedom en 1999~»~: Chris Oakes,
  \emph{Zero-Knowledge: Nothing Personal}, 9 février 1999~:
  \url{https://www.wired.com/1999/02/zero-knowledge-nothing-personal/}.}.
Mais cette expérience s'est arrêtée en 2001, faute d'utilisation
suffisante.

Un projet du même type dans lequel les cypherpunks se sont impliqués
était le réseau Tor, lancé publiquement en 2003, qui se basait, comme on
l'a déjà expliqué, sur le routage en ognon. En effet, si Tor était le
résultat d'une recherche militaire provenant de la Navy, les individus
ayant travaillé sur son implémentation n'en avaient pas moins des
convictions allant dans le sens des cypherpunks. Roger Dingledine et
Nick Mathewson, les deux informaticiens qui ont aidé Paul Syverson dans
cette conception, en faisaient partie~: le premier était derrière le
projet Free Haven, qui avait pour but de développer un système
décentralisé de stockage de données\footnote{«~le projet Free Haven~»~:
  \url{https://www.freehaven.net/}.}~; le second est crédité pour avoir
créé le programme de serveur de courriel anonyme Mixminion\footnote{«~programme
  de serveur de courriel anonyme Mixminion~»~: George Danezis, Roger
  Dingledine, Nick Mathewson, \emph{Mixminion: Design of a Type III
  Anonymous Remailer Protocol}, 2003~:
  \url{https://git.gnunet.org/bibliography.git/plain/docs/minion-design.pdf}.}.
On peut également citer le jeune Jacob Appelbaum, qui s'est fortement
impliqué dans le projet Tor entre 2004 et 2016.

Un troisième secteur auquel les cypherpunks ont contribué a été la
fluidification des flux informationnels, notamment face à la censure. En
1993, Tim May a repris le modèle de l'AMIX de Phil Salin pour introduire
un concept de place de marché de l'information appelé BlackNet. Cette
plateforme devait servir à échanger des secrets commerciaux, des
recettes de fabrication, des techniques relatives aux nanotechnologies,
des informations sur les décisions d'entreprises, au moyen de
«~CryptoCredits~», la monnaie interne du système. Il s'agissait donc de
libérer l'information des contraintes étatiques~: «~BlackNet est
officiellement non idéologique, mais considère les États-nations, les
lois d'exportation, les lois sur les brevets, les considérations de
sécurité nationale, etc. comme des reliques de l'ère
pré-cyberspatiale~», écrivait Tim May\footnote{Timothy C. May, \emph{no
  subject (file transmission)}, 17 août 1993,
  \url{https://cypherpunks.venona.com/date/1993/08/msg00538.html}.}.

Le concept de BlackNet était une simple expérience de pensée et n'a
jamais été mis en œuvre. Toutefois, il a préfiguré d'autres modèles qui
ont ouvert la voie au partage d'informations sensibles sur Internet.
C'était par exemple le cas de Cryptome, un site web lancé en 1996 par le
cypherpunk John Young pour héberger des documents sensibles et censurés
par les États\footnote{«~Cryptome, un site web lancé en 1996 par le
  cypherpunk John Young~»~: \emph{Cryptome JYA Archive}~:
  \url{https://cryptome.org/jya/}.}.

Mais c'était surtout le cas de WikiLeaks, une plateforme facilitant la
publication de documents classifiés fondée en 2006 par l'informaticien
australien Julian Assange. Julian Assange était un cypherpunk assumé~:
il envoyait des courriels sur la liste depuis au moins 1995 et a par la
suite co-écrit un livre à ce sujet intitulé \emph{Cypherpunks: Freedom
and the Future of the Internet}\footnote{«~un livre à ce sujet~»~:
  Julian Assange, Jacob Appelbaum, Andy Müller-Maguhn, Jérémie
  Zimmermann, \emph{Cypherpunks: Freedom and the Future of the
  Internet}, OR Books, 2012.}. WikiLeaks a permis le développement de
l'activité des lanceurs d'alertes (\emph{whistleblowers}) révélant les
agissements illégaux ou injustes de leurs employeurs, et en particulier
des États, largement inaugurée par la publication des Pentagon Papers en
1971 par Daniel Ellsberg. Grâce à l'utilisation du chiffrement et de
Tor, WikiLeaks permettait aux personnes à l'origine des fuites de
conserver leur anonymat.

Enfin, les cypherpunks se sont aussi investis dans le développement du
pair-à-pair. En 2000, le développeur Jim McCoy, cypherpunk de la
première heure, a ainsi lancé Mojo Nation, un projet de plateforme
d'échange de fichiers en pair-à-pair intégrant une devise
interne\footnote{Damien Cave, \emph{The Mojo solution}, 9 octobre 2000~:
  \url{https://www.salon.com/2000/10/09/mojo_nation/}.}. En 2001, Bram
Cohen, qui travaillait avec lui a quitté le projet et lancé BitTorrent,
qui est devenu la référence pour le partage de fichiers au cours des
années. Mojo Nation, alors renommé en Mnet, a été repris par Zooko
Wilcox. Ce dernier a lancé son propre système, Tahoe-LAFS, en 2006.

Mais ce qu'il manquait à tous ces systèmes, c'était une monnaie
numérique robuste qui soit adaptée au cyberespace, chose à laquelle les
cypherpunks aspiraient depuis le début. Leurs modèles possédaient
parfois des unités de compte internes, mais elles étaient très
instables. Malheureusement, une telle monnaie ne serait conçue que des
années plus tard, en 2008, sous la forme de Bitcoin.

Satoshi Nakamoto était-il un cypherpunk~? À notre connaissance, il n'a
pas participé au mouvement originel des années 90, ni ne s'est jamais
réclamé explicitement de celui-ci. Toutefois, il a très clairement été
influencé par l'héritage des cypherpunks comme le suggèrent plusieurs
éléments. D'abord, il semblait bien connaître ce qui s'était passé et de
ce qui avait été fait précédemment dans le domaine de la monnaie
numérique, malgré quelques lacunes (voir
chapitre~\hyperref[ch:cybermonnaie]{6}). Puis, il a publié le livre
blanc sur la liste de diffusion dédiée à la cryptographie gérée par
Perry Metzger, qui était la digne héritière de la liste des cypherpunks,
dont l'usage avait malheureusement périclité vers 1997. Ensuite, il a
utilisé un pseudonyme et, par de bonnes pratiques comme l'utilisation de
PGP\footnote{«~l'utilisation de PGP~»~: Satoshi Nakamoto, \emph{Re:
  md5?}, /07/2010 22:06:57 UTC~:
  \url{https://bitcointalk.org/index.php?topic=458.msg5772\#msg5772}.},
Tor et Namecheap, il est parvenu à préserver son anonymat malgré une
activité en ligne s'étalant sur presque trois années. Enfin, il a
«~écrit du code~» en programmant un outil émancipateur, conformément à
l'appel à la pratique d'Eric Hughes. Il est donc tout à fait raisonnable
d'associer Satoshi aux cypherpunks, tout en l'en dissociant
partiellement, ne serait-ce parce qu'il est toujours resté très mesuré
dans ses quelques jugements politiques.

\section*{Une guerre perpétuelle}\label{une-guerre-perpuxe9tuelle}
\addcontentsline{toc}{section}{Une guerre perpétuelle}

\markright{Une guerre perpétuelle}

Bitcoin s'inscrit pleinement dans la guerre technologique opposant
l'autorité à la liberté. Son code n'est pas neutre~: il n'est pas une
vague technique qu'on puisse utiliser dans un sens ou dans l'autre, mais
il a pour objectif clair d'amener plus d'autonomie individuelle. Bitcoin
n'est pas un assemblage aléatoire de procédés, mais un objet ancré dans
son époque, qui prend racine dans les mouvements techno-idéologiques qui
l'ont précédé.

Bitcoin est ainsi issu de mouvements qui appellent à la pratique. Les
libristes promouvaient la publication sous licence libre dans le but de
mettre en commun l'ensemble des connaissances de l'humanité. Les
extropiens préconisaient la recherche et l'expérimentation pour
améliorer drastiquement les conditions de vie matérielles de l'être
humain. Les cypherpunks prônaient le fait d'écrire du code afin de
préserver la confidentialité des individus dans le cyberespace. Il est
donc naturel que la communauté de Bitcoin s'inscrive dans la même
démarche en encourageant la pratique monétaire en vue de résister au
contrôle de plus en plus grand de l'État et des banques sur le transfert
d'argent.

Cependant, pour arriver à ce résultat, il a fallu concevoir un système
qui permette de répartir les risques entre les participants sans
nécessiter l'intervention d'un tiers de confiance. Une quête que nous
raconterons dans le prochain chapitre.

\bookmarksetup{startatroot}

\chapter{La cybermonnaie avant Nakamoto}\label{ch:cybermonnaie}

\phantomsection\label{enotezch:6}{}

{L}\textsc{a} cybermonnaie est une monnaie dont le fonctionnement repose
entièrement sur un réseau informatique appartenant à Internet. Elle est
définie au sein de ce réseau et est transférée par son intermédiaire. Il
s'agit d'une monnaie native du cyberespace, le nouvel espace créé par
l'émergence d'Internet, conçu comme une juridiction séparée du monde
physique.

Une forme plus spécifique de cybermonnaie est l'argent liquide
numérique, ou \emph{digital cash} en anglais, qui transcrit les
propriétés des espèces sonnantes et trébuchantes dans le cyberespace.
Toutefois, bien que cette conception remonte à l'émergence même
d'Internet, elle n'a pas tout de suite pu voir le jour en raison de
limitations techniques et conceptuelles. L'argent liquide numérique a
fait l'objet d'une véritable quête, à laquelle ont participé de nombreux
individus désireux d'utiliser Internet pour créer un nouveau paradigme
économique, dont les cypherpunks.

Bitcoin est le résultat de cette quête. Il n'est pas sorti de nulle
part~: il est le fruit de réflexions, de recherches et
d'expérimentations en tous genres. La découverte de Satoshi Nakamoto
constitue ainsi une percée dans un domaine qui existait antérieurement.

\section*{L'échange monétaire sur
Internet}\label{luxe9change-monuxe9taire-sur-internet}
\addcontentsline{toc}{section}{L'échange monétaire sur Internet}

\markright{L'échange monétaire sur Internet}

Internet a généralisé le partage informationnel et, ce faisant, a créé
un nouvel espace d'interactions humaines~: le cyberespace. L'apparition
de cet espace a naturellement mené à l'émergence d'une demande pour
l'échange monétaire, demande qui s'est manifestée par le développement
du commerce électronique dans les années 90. Comme le résumait très bien
Robert Hettinga en 1998, la problématique était la suivante~:

«~Depuis l'invention du télégraphe, le règlement des transactions
financières se heurte à un problème~: comment faire des affaires à
distance alors que le moyen le plus simple d'exécuter, de compenser et
de régler une transaction est l'échange de certificats au
porteur\footnote{Robert A. Hettinga, \emph{Digital Bearer Settlement},
  avril 1998~:
  \url{http://www.systemics.com/legal/digigold/discovery/postings/Geoecon.pdf}.}~?~»

La première solution était d'utiliser du crédit bancaire. L'usage de
celui-ci comme intermédiaire d'échange s'était progressivement
généralisé en Occident avec la bancarisation de la société. Au cours du
temps, une solution technique avait prévalu~: la carte de paiement,
aussi appelée carte de débit ou carte de crédit selon son
fonctionnement. Cette solution n'avait pas constitué quelque chose de
novateur\footnote{Le terme «~\emph{credit card}~» a été utilisé en 1888
  par Edward Bellamy, écrivain et journaliste socialiste américain et
  précurseur du mouvement technocratique, dans son roman de fiction
  spéculative \emph{Looking Backward}, pour désigner la carte de
  paiement des citoyens de sa supposée utopie. Ce type de carte s'est
  ensuite développé dans les années 1920--1930 aux États-Unis sous la
  forme de cartes délivrées indépendamment par Western Union, par les
  grands magasins, par les compagnies pétrolières et compagnies
  aériennes.}, mais s'était considérablement popularisée à partir des
années 60, par le biais de l'adoption bancaire et de la formation de
sociétés spécialisées dans le transfert électronique de fonds comme
NBI~/~Visa et Interbank~/~MasterCard\footnote{Sur les origines du réseau
  Visa, voir David L. Stearns, \emph{Electronic Value Exchange: Origins
  of the VISA Electronic Payment System}, Springer, 2011. Le titre du
  livre est une référence au projet ambitieux de Dee Hock (le fondateur
  de Visa) de créer un protocole d'échange de valeur électronique (EVE)
  permettant d'effectuer l'intégralité des transactions sous forme
  électronique, ce qui donnerait lieu à «~la genèse d'une nouvelle forme
  de monnaie mondiale~».}.

Mais le paiement par carte bancaire n'était pas forcément adapté au
cyberespace, car difficile à mettre en place, coûteux et peu sécurisé à
l'époque. C'est pourquoi on a vu émerger différentes solutions
techniques permettant de faire des paiements sur Internet au milieu des
années 90, comme CyberCash\footnote{«~CyberCash~»~: Peter Wayner,
  \emph{Cybercash's Lesson in Web Survival}, 10 août 1998~:
  \url{https://www.nytimes.com/1998/08/10/business/cybercash-s-lesson-in-web-survival.html}~;
  archive~:
  \url{https://web.archive.org/web/20150527080844/https://www.nytimes.com/1998/08/10/business/cybercash-s-lesson-in-web-survival.html}.},
First Virtual\footnote{«~First Virtual~»~:
  \url{https://www.nytimes.com/1994/10/15/business/company-news-a-credit-card-for-on-line-sprees.html}}
ou Open Market. Des systèmes de micropaiements ont également fait leur
apparition à l'instar de CyberCoin (géré par CyberCash),
NetBill\footnote{«~NetBill~»~: Benjamin Cox, J. D. Tygar, Marvin Sirbu,
  \emph{NetBill Security and Transaction Protocol}, 1995~:
  \url{https://people.eecs.berkeley.edu/~tygar/papers/Netbill_securiy_and_transaction_protocol.pdf}.}
et MilliCent\footnote{«~MilliCent~»~: Martín Abadi, Paul Gauthier, Steve
  Glassman, Mark S. Manasse, Patrick Sobalvarro, \emph{The Millicent
  Protocol for Electronic Commerce}, 1995~:
  \url{https://www.w3.org/Conferences/WWW4/Papers/246/}.}.

Ces systèmes ont fini par échouer, mais c'est dans cette niche que s'est
développé le service PayPal à partir de 1999. Celui-ci était conçu pour
être simple d'accès (PayPal signifie littéralement «~payer copain~»)~:
il permettait des paiements faciles et sans frais, entre adresses de
courrier électronique. Son modèle économique se fondait sur la
perception des intérêts liés à la conservation des fonds des clients en
banque, afin de payer les coûts de fonctionnement et de rémunérer les
actionnaires. C'était donc un service de troisième couche, bâti
au-dessus du système bancaire, lui-même basé sur le système de monnaie
centrale\footnote{Sur l'histoire des débuts du service PayPal, voir Eric
  M. Jackson, \emph{The PayPal Wars: Battles With Ebay, the Media, the
  Mafia, and the Rest of Planet Earth}, World Ahead Pub., 2012.}.

En dépit des bonnes intentions de leurs créateurs, ces systèmes étaient
complètement à la merci du régulateur. Ceux qui ont survécu se sont par
conséquent engagés dans la surveillance et la censure, à un niveau
jamais vu auparavant.

La deuxième solution pour échanger de la valeur sur Internet était
d'émettre une nouvelle monnaie numérique, de manière centralisée, si
besoin en l'adossant à une monnaie existante. Cette solution consistait
à ne pas demander l'autorisation, en jouant sur le flou juridique qui
pouvait exister dans un domaine relativement nouveau.

Les jeux vidéos en ligne massivement multijoueurs, dont les fameux
MMORPG, ont contribué à installer l'idée de monnaie numérique
indépendante dans les esprits. On peut penser au Token, la monnaie
native de Habitat, l'un des premiers MMORPG graphiques, développé en
1985 par Lucasfilm Games et jouable sur Commodore 64\footnote{«~Habitat,
  l'un des premiers MMORPG graphiques, développé en 1985~»~: Chip
  Morningstar, F. Randall Farmer, \emph{The Lessons of Lucasfilm's
  Habitat}, mai 1990~: \url{http://www.fudco.com/chip/lessons.html}.}.
On peut aussi citer les cas des pièces de métaux précieux dans Everquest
en 1999, du dollar Linden de Second Life en 2003 ou encore de l'or de
World of Warcraft en 2004. Tous ces exemples prouvaient qu'une économie
réelle pouvait émerger d'une monnaie virtuelle\footnote{«~une économie
  réelle pouvait émerger d'une monnaie virtuelle~»~: Julian Dibbell,
  \emph{The Life of the Chinese Gold Farmer}, 17 juin 2007~:
  \url{https://www.nytimes.com/2007/06/17/magazine/17lootfarmers-t.html}.}.

Un exemple anecdotique de ce type de système de monnaie numérique était
le Hawthorne Exchange, lancé le 24 mars 1993 sur la liste de diffusion
extropienne par un individu du nom de Brian Holt Hawthorne\footnote{«~le
  Hawthorne Exchange, lancé le 24 mars 1993 sur la liste de diffusion
  extropienne~»~: Brian Holt Hawthorne, \emph{HEX: Introducing the
  Hawthorne Exchange}, /03/1993 06:20:21 UTC~:
  \url{https://diyhpl.us/~bryan/irc/extropians/raided-mailing-list-archives/unzipped/disk-07/DIG30152}.}.
Il s'agissait d'un marché de réputation pour les membres de la liste,
dont l'unité de compte servant à l'échange était le Thorne. Le système
était peu accessible et peu robuste, mais les extropiens l'ont utilisé
et ont donné de la valeur au Thorne, par anticipation du futur. Quelques
échanges monétaires contre du dollar et des services ont été réalisés
entre les membres de la liste. Toutefois, le Hawthorne Exchange était
simplement une expérimentation, le Thorne n'ayant aucune prétention à
être une réelle monnaie, que son auteur a décidé d'arrêter en 94.

Un système bien plus sérieux est apparu en 1996~: e-gold. Comme décrit
dans le chapitre~\hyperref[ch:adversaire]{4}, il s'agissait d'un modèle
de monnaie en or numérique dont l'unité de compte était théoriquement
adossée à de l'or. Le système reposait sur l'entreprise \emph{Gold \&
Silver Reserve Inc.} fondée par Douglas Jackson, qui conservait l'or
physique dans ses coffres. Il a connu un grand succès dans les années
2000 avant d'être fermé en 2007 par le Secret Service.

Toutefois, le problème avec ce type de monnaie était qu'il dépendait
toujours d'une entité qui constituait un point de défaillance unique.
Ainsi, même si les personnes qui le géraient n'étaient pas
malintentionnées, ce type de système n'était pas robuste et ne pouvait
pas perdurer à long terme.

La troisième solution de cybermonnaie était la conception d'un argent
liquide électronique, confidentiel, non contrôlé et décentralisé. L'idée
était de diminuer le rôle du tiers de confiance le plus possible pour
que la monnaie en question se rapproche au mieux de l'argent liquide
physique, de minimiser la confiance impliquée. Idéalement l'objectif
était d'obtenir un «~or numérique~» qui soit à la fois «~infalsifiable,
sans inflation, et intraçable\footnote{Hadon Nash, \emph{Digital gold},
  /08/1993 20:23:30 UTC~:
  \url{https://cypherpunks.venona.com/date/1993/08/msg00698.html}.}~».

Les cypherpunks considéraient que ce type de monnaie numérique était
quelque chose d'essentiel dans leur combat pour la liberté et la
confidentialité\footnote{«~ce type de monnaie numérique était quelque
  chose d'essentiel dans leur combat~»~: Eric Hughes, \emph{RANTS: A
  Cypherpunk's Manifesto}, /03/1993 19:51:06 UTC~:
  \url{https://cypherpunks.venona.com/date/1993/03/msg00392.html}.}. Ils
prévoyaient en effet d'utiliser ce type d'unité de compte dans leurs
projets, comme en témoignent les Cryptocredits de BlackNet ou le mojo de
Mojo Nation. C'est donc tout naturellement qu'ils ont cherché à
développer une telle monnaie.

Cependant, la conception d'un argent liquide numérique, d'une
authentique cybermonnaie, n'était pas un tâche facile. La quête pour sa
réalisation a mis de longues années avant d'aboutir. Et la première
étape dans cette quête a été l'apparition de eCash, qui a eu le mérite
de poser sur la table une proposition cohérente, répondant aux exigences
des cypherpunks.

\section*{eCash~: l'argent liquide
chaumien}\label{ecash-largent-liquide-chaumien}
\addcontentsline{toc}{section}{eCash~: l'argent liquide chaumien}

\markright{eCash~: l'argent liquide chaumien}

eCash est un concept de monnaie numérique confidentielle qui a été conçu
par le crytographe David Chaum dans les années 80 et qui a été mis en
application au cours des années 90. Il a été décrit initialement par
Chaum en 1982\footnote{«~décrit initialement par Chaum en 1982~»~: David
  L. Chaum, «~\emph{Blind signatures for untraceable payments}~», in
  \emph{Advances in Cryptology: Proceedings of CRYPTO '82}, 1982,
  pp.~199--203~:
  \url{https://sceweb.sce.uhcl.edu/yang/teaching/csci5234WebSecurityFall2011/Chaum-blind-signatures.PDF}.}
avant d'être mis en avant en 1985 dans son article intitulé
\emph{Security without Identification} qui promettait de «~rendre Big
Brother obsolète\footnote{David L. Chaum, «~\emph{Security without
  identification: transaction systems to make big brother obsolete}~»,
  in \emph{Communications of the ACM}, vol.~28, no. 10, octobre 1985,
  pp.~1030---1044~:
  \url{https://www.cs.ru.nl/~jhh/pub/secsem/chaum1985bigbrother.pdf}.}~».
Le modèle repose sur le mécanisme de signature aveugle, qui garantit la
propriété de la monnaie et l'anonymat des échanges.

Le modèle eCash gère des billets numériques de différentes coupures qui
peuvent être conservés par les utilisateurs. Les billets sont émis et
remplacés par de serveurs appelés des banques (\emph{banks}) ou des
monnaieries (\emph{mints}). Lorsqu'un billet est transféré, le
destinataire l'envoie à sa banque qui se charge de le vérifier et de lui
en redonner un autre. Les banques du système entretiennent chacune un
registre des billets dépensés pour empêcher la double dépense. Le
système est chapeauté par une autorité centrale qui délivre les
habilitations.

L'émission d'un billet numérique utilise comme on l'a dit le mécanisme
de signature aveugle. Pour l'utilisateur, il s'agit essentiellement de
choisir un grand nombre et de le faire signer par sa banque, de manière
à ce que ce nombre reste uniquement connu de lui. Le fonctionnement de
ce procédé mathématique est analogue à la signature d'un billet physique
en papier carbone représentant une quantité précise d'unités monétaires
(coupure). Voici comment se passe la création d'un billet par Alice~:

\begin{enumerate}
\def\labelenumi{\arabic{enumi}.}
\item
  Alice crée un billet en papier carbone (en générant aléatoirement un
  très grand nombre \(x\))~;
\item
  Alice place le billet dans une enveloppe fermée (en utilisant une
  fonction de commutation \(c\) qu'elle seule connaît)~;
\item
  Alice envoie l'enveloppe contenant son billet à la banque et lui
  communique la coupure souhaitée~;
\item
  La banque signe l'enveloppe en indiquant la quantité d'unités que le
  billet représente (la banque dispose d'une clé privée pour chaque
  coupure), ce qui a pour effet de signer le billet en papier carbone à
  l'intérieur~;
\item
  La banque renvoie l'enveloppe à Alice~;
\item
  Alice ouvre l'enveloppe pour récupérer son billet signé (en utilisant
  la fonction d'inversion \(c'\))~;
\item
  Alice vérifie que la signature de la banque est authentique (en
  vérifiant qu'elle correspond à la clé publique de la banque liée à la
  coupure demandée).
\end{enumerate}

Le transfert du billet signé se fait en le donnant à quelqu'un d'autre.
Ainsi, le paiement de Bob par Alice pour un service rendu se compose des
étapes suivantes~: d'abord, Alice transmet le billet à Bob~; puis, Bob
vérifie qu'il a bien été signé par la banque d'Alice~; ensuite, il
envoie immédiatement le billet réceptionné à sa banque~; enfin, la
banque de Bob vérifie que le billet n'a pas déjà été utilisé et, le cas
échéant, signe un nouveau billet de la même coupure pour le donner à
Bob.

\begin{figure}[H]

{\centering \includegraphics{chapters/img/chaumian-ecash.png}

}

\caption{Création et remplacement d'un billet chaumien.}

\end{figure}%

Les billets numériques peuvent être émis pour eux-mêmes, auquel cas ils
forment une monnaie de base. Mais ils peuvent également être adossés à
une autre monnaie comme le dollar. Dans ce dernier cas, l'utilisateur
peut à tout moment rendre ses billets à sa banque pour récupérer la
somme correspondante.

La principale conséquence du procédé est qu'aucune banque du système ne
peut relier le paiement à l'identité d'Alice. La banque d'Alice sait
qu'un billet signé par elle a été dépensé, mais elle ne peut pas savoir
de manière absolue qu'il s'agissait d'un billet appartenant à Alice. La
banque de Bob sait que Bob a reçu le paiement et qu'il provenait de la
banque d'Alice, mais rien de plus. C'est pour cette raison qu'eCash peut
être considéré comme un modèle respectueux de la vie privée.

Toutefois, cette confidentialité du système repose sur une supposition
forte~: la bienveillance des banques du système. En effet, si une banque
voulait obtenir des informations liées à un billet particulier (par
exemple sous la pression étatique), elle pourrait les demander à son
propriétaire en échange de l'autorisation du transfert. On peut ainsi
imaginer un système eCash qui respecte pleinement les normes de
surveillance, comme le suggère l'implémentation de Chaum pour une MNBC
conceptualisée en 2021\footnote{David Chaum, Christian Grothoff, Thomas
  Moser, \emph{How to Issue a Central Bank Digital Currency}, mars
  2021~:
  \url{https://www.snb.ch/n/mmr/reference/working_paper_2021_03/source/working_paper_2021_03.n.pdf}.}.

\section*{Magic Money, les CyberBucks et les
banques}\label{magic-money-les-cyberbucks-et-les-banques}
\addcontentsline{toc}{section}{Magic Money, les CyberBucks et les
banques}

\markright{Magic Money, les CyberBucks et les banques}

Le concept d'eCash a été mis en application au cours des années 90. À
l'époque, le Web venait tout juste d'apparaître, le commerce
électronique était inexistant et cette idée constituait une formidable
opportunité. Cette mise en œuvre a été réalisée d'abord par les
cypherpunks par l'intermédiaire du protocole Magic Money, puis par la
société de David Chaum, DigiCash, au moyen de jetons d'essai appelés les
CyberBucks et d'un déploiement dans le système bancaire classique.

Le protocole Magic Money a été présenté sur la liste de diffusion des
cypherpunks le 4 février 1994 par un développeur anonyme se faisant
appeler Pr0duct Cypher et utilisant PGP pour s'identifier. Magic Money
permettait de créer sa monnaie en faisant fonctionner un serveur de
courrier électronique qui servait de monnaierie eCash\footnote{«~Magic
  Money est un système d'argent liquide numérique conçu pour être
  utilisé par courrier électronique. Le système est en ligne et
  intraçable. En ligne signifie que chaque transaction implique un
  échange avec un serveur, pour éviter les doubles dépenses. Intraçable
  signifie qu'il est impossible pour quiconque de retracer les
  transactions, de faire correspondre un retrait avec un dépôt, ou de
  faire correspondre deux pièces de quelque manière que ce soit.~» --
  Pr0duct Cypher, \emph{Magic Money Digicash System}, /02/1994 20:44:27
  UTC~: \url{https://cypherpunks.venona.com/date/1994/02/msg00247.html}.}.
Magic Money utilisait l'algorithme RSA et la signature aveugle, deux
techniques qui étaient brevetées à l'époque, de sorte que son
déploiement était \emph{de facto} illégal et devait se confiner à
l'expérimentation. Cette annonce a toutefois été accueillie
favorablement sur la liste, notamment par Hal Finney\footnote{«accueillie
  favorablement sur la liste, notamment par Hal Finney~»~: Hal Finney,
  \emph{Re: Magic Money Digicash System}, /02/1994 21:58:18 UTC~:
  \url{https://cypherpunks.venona.com/date/1994/02/msg00251.html}.}.

Le premier système basé sur Magic Money a été mis en ligne par Mike
Duvos quelques semaines plus tard avec les Tacky Tokens, dont les pièces
étaient émises en valeurs de 1, 2, 5, 10, 20, 50 et 100
unités\footnote{«~Tacky Tokens, dont les pièces étaient émises en
  valeurs de 1, 2, 5, 10, 20, 50 et 100 unités~»~: Mike Duvos, \emph{Fun
  With Magic Money}, /02/1994 00:51:40 UTC~:
  \url{https://cypherpunks.venona.com/date/1994/02/msg01391.html}.}.
Malgré des propositions, aucune transaction réelle n'a été réalisée, ce
qui a poussé Tim May à se demander «~pourquoi l'argent liquide numérique
{[}n'était{]} pas utilisé\footnote{Timothy C. May, \emph{Why Digital
  Cash is Not Being Used}, /05/1994 19:48:18 UTC~:
  \url{https://cypherpunks.venona.com/date/1994/05/msg00155.html}.}~».
D'autres implémentations fantaisistes de Magic Money ont vu le jour par
la suite, comme les GhostMarks, les DigiFrancs ou les NexusBucks, mais
n'ont pas connu un plus grand succès. L'activité a très rapidement
reculé au cours des semaines\footnote{«~L'activité a très rapidement
  reculé au cours des semaines~»~: Mike Duvos, \emph{In Search of
  Genuine DigiCash}, /08/1994 06:06:49 UTC~:
  \url{https://cypherpunks.venona.com/date/1994/08/msg00695.html}.}.

Le concept d'eCash a ensuite été mis en pratique par la société DigiCash
B.V., fondée par David Chaum en 1990 et basée à Amsterdam aux Pays-Bas,
qui avait pour mission de mettre en application les idées du
cryptographe\footnote{«~Le concept d'eCash a ensuite été mis en pratique
  par la société DigiCash B.V.~»~: La chronologie de DigiCash se
  retrouve sur le site personnel de David Chaum. -- David Chaum,
  \emph{eCash}~: \url{https://chaum.com/ecash/}.}. Plusieurs cypherpunks
ont travaillé pour cette entreprise comme Eric Hughes, Bryce Wilcox (le
futur Zooko Wilcox-O'Hearn) et Nick Szabo. Après quelques années de
développement, un prototype a été présenté en mai 1994 lors de la
première conférence internationale sur le World Wide Web au CERN à
Genève\footnote{«~présenté en mai 1994 lors de la première conférence
  internationale sur le World Wide Web au CERN~»~: DigiCash,
  \emph{World's first electronic cash payment over computer networks},
  27 mai 1994~:
  \url{https://chaum.com/wp-content/uploads/2022/01/05-27-94-World_s-first-electronic-cash-payment-over-computer-networks.pdf}.}.

DigiCash a ensuite réalisé un essai qui a débuté le 19 octobre de cette
année, avec l'émission de CyberBucks. Bien que leur nom fasse référence
à la monnaie étasunienne («~\emph{a buck}~»), ceux-ci n'étaient pas
adossés au dollar et possédaient donc un prix flottant par rapport à ce
dernier. Une distribution initiale de 100 CyberBucks par nouvel
utilisateur a été effectuée afin d'aider l'amorçage du système. Les
cypherpunks se sont appropriés la chose en effectuant des échanges
réels~: la récompense pour la résolution d'un problème, la vente de
t-shirts, la vente de logiciels, et bien sûr le change avec le
dollar\footnote{Jim Crawley, «~Electronic Cash~», \emph{The Computists'
  Weekly}, vol.~5, no. 25, 11 juillet 1995~:
  \url{https://www.nzdl.org/cgi-bin/library?e=d-00000-00---off-0tcc--00-0----0-10-0---0---0direct-10---4-------0-1l--11-ro-50---20-preferences---10-0-1-00-0--4----0-0-11-10-0utfZz-8-00&a=d&cl=CL2.5&d=HASH0199d48acda6ba6861de2d9e.2}.}.
Divers commerçants acceptaient les CyberBucks dans le cadre de cette
expérience.

Mais les CyberBucks ne constituaient qu'une monnaie d'essai, qui a
périclité en octobre 1995 lorsque la Mark Twain Bank, une petite banque
du Missouri, a lancé sa propre version du protocole en partenariat avec
DigiCash. Contrairement à l'essai précédant, l'unité échangée était
adossée au dollar étasunien. Bien que l'expérience des CyberBucks ne se
soit pas techniquement arrêtée là, leur valeur s'est effondrée à cause
de cette nouvelle\footnote{«~Mark Twain est arrivée sur le marché avec
  de l'argent liquide numérique \emph{réel}, et les gens ont
  complètement cessé d'échanger les certificats bêta. Je ne me souviens
  même pas du dernier prix de règlement, mais il s'agissait de quelques
  centimes de dollars.~» -- Robert Hettinga, \emph{e\$: Interbank
  Digital Cash Clearing, Better Living through Walletware,
  Microintermediation, Net.Currencies and ECM}, 3 juin 1996, archive~:
  \url{https://web.archive.org/web/19980204144728/http://www.shipwright.com/rants/rant_14.html}.}.

Par la suite, DigiCash a conclu des partenariats avec différentes
banques pour s'inscrire dans le milieu financier traditionnel. Entre
1996 et 1998, six banques situées aux quatres coins du monde ont suivi
la Mark Twain Bank~: la Merita Bank en Finlande, la Deutsche Bank en
Allemagne, l'Advance Bank en Australie, la Bank Austria en Autriche, la
Den norske Bank en Norvège et le Crédit Suisse en Suisse. On lui
promettait alors un avenir radieux\footnote{Antoine Champagne,
  «~\emph{L'argent liquide numérique (crypto-curency) est né en 1995~:
  souvenirs}~», \emph{Reflets.info}, 11 janvier 2014~:
  \url{https://reflets.info/articles/l-argent-liquide-numerique-crypto-curency-est-ne-en-1995-souvenirs}.}.

C'était toutefois sans compter sur le caractère de David Chaum, qui
était têtu, suspicieux et souhaitait garder le contrôle sur son
entreprise\footnote{«~\emph{Hoe DigiCash alles verknalde}~», \emph{Next!
  Magazine}, 1 janvier 1999, archive~:
  \url{https://web.archive.org/web/19990427142412/https://www.nextmagazine.nl/ecash.htm}.
  Une traduction (en anglais) est disponible à l'adresse
  \url{https://cryptome.org/jya/digicrash.htm}.}. Ainsi, ce dernier a
refusé des partenariats avec de grands acteurs comme ING et ABN AMRO
(deux des trois plus grandes banques néerlandaises à l'époque), Visa,
Netscape et Microsoft. Finalement, il a dû, sous pression des
actionnaires et des employés, quitter son poste de PDG pour devenir
directeur technique et céder sa place à Michael Nash, ancien employé de
Visa, en 1997. Le siège social de DigiCash a été déménagé en Californie
et l'entreprise est par conséquent devenue une société étasunienne.

Le 17 septembre 1998, la Mark Twain Bank (rachetée par la Mercantile
Bancorporation en 1996) a annoncé abandonner eCash, ce qui a provoqué la
fin de DigiCash. Le 3 novembre, l'entreprise est entrée en faillite et a
été placée sous la protection du chapitre 11 aux États-Unis, de sorte
que ses possessions ont été progressivement revendues au fil des années,
dont ses brevets en 2002. Avec DigiCash, c'était le concept même d'eCash
qui disparaissait de la circulation.

En 1999, Chaum a expliqué les raisons de l'échec de sa société, à savoir
le manque d'adoption dû à la difficulté d'utilisation\footnote{«~le
  manque d'adoption dû à la difficulté d'utilisation~»~: Julie Pitta,
  \emph{Requiem for a Bright Idea}, 1 novembre 1999~:
  \url{https://www.forbes.com/forbes/1999/1101/6411390a.html}.}. Cette
disparition a progressivement laissé les cartes de paiement et PayPal
triompher.

La fin de DigiCash a ainsi laissé un vide sur le marché de l'argent
liquide numérique. Mais la demande n'a jamais disparu, de sorte qu'on
pouvait pressentir qu'il réapparaitrait d'une façon ou d'une autre. Tel
que le prédisait Milton Friedman, prix Nobel d'économie et fondateur de
l'École de Chicago, en 1999, au micro de la National Taxpayers Union
Foundation~:

«~Je pense qu'Internet va devenir l'une des forces majeures qui va
réduire le rôle de l'État. La seule chose qui manque, mais qui sera
bientôt développée, c'est un argent liquide électronique fiable, une
méthode qui permette de transférer des fonds de A à B sur Internet sans
que A connaisse B ou que B connaisse A\footnote{Milton Friedman,
  \emph{Milton Friedman Full Interview on Anti-Trust and Tech} (vidéo),
  1999~: \url{https://www.youtube.com/watch?v=mlwxdyLnMXM}, :32.}.~»

\section*{libtech-l~: révolutionner la
monnaie}\label{libtech-l-ruxe9volutionner-la-monnaie}
\addcontentsline{toc}{section}{libtech-l~: révolutionner la monnaie}

\markright{libtech-l~: révolutionner la monnaie}

Après l'échec d'eCash en octobre 1998, l'idée d'un argent liquide
numérique réel a progressivement été délaissée par les cypherpunks, qui
se sont contentés pour la plupart des expériences de monnaie privée et
des systèmes de paiement existants. Mais ce n'était pas le cas de tous
les membres du mouvement. Un petit nombre d'entre eux s'est ainsi
regroupé sur une liste de diffusion privée appelée libtech-l où ils ont
pu parler de comment la monnaie évoluerait.

La liste libtech-l, créée en 1994 par Nick Szabo\footnote{libtech-l@netcom.com
  -- Timothy C. May, \emph{Re: Regional Lists}, /06/1994 05:48:50 UTC~:
  \url{https://cypherpunks.venona.com/date/1994/06/msg01156.html}~;
  Timothy C. May, \emph{Cyphernomicon}, 2.4.27.}, avait pour but, comme
son nom l'indique, d'héberger des discussions sur les techniques
libératoires, à même de protéger la liberté individuelle face à
l'autorité, dans l'esprit des mouvements extropien et cypherpunk dont
les membres étaient par ailleurs des participants. On pouvait y
retrouver notamment les interventions des cypherpunks Wei Dai et Hal
Finney, ainsi que celles des économistes Larry White et George Selgin.
Ces cinq individus formaient le noyau dur de cette liste privée, dont
émergeraient plusieurs idées de monnaie numérique.

Nicholas J. Szabo, dit Nick Szabo, était un informaticien américain
d'origine hongroise. Extropien, puis cypherpunk, il s'était notamment
illustré par son implication dans le combat contre la puce Clipper. En
1994, il avait formalisé la notion de \emph{smart contract}, qu'il avait
définie comme «~un protocole de transaction informatisé qui exécute les
termes d'un contrat\footnote{Nick Szabo, \emph{Smart Contracts}, 1994,
  archive~:
  \url{https://web.archive.org/web/20011102030833/http://szabo.best.vwh.net:80/smart.contracts.html}.}~»,
et l'avait approfondie dans les années qui ont suivi\footnote{«~l'avait
  approfondie dans les années qui ont suivi~»~: Nick Szabo,
  «~\emph{Smart Contracts: Building Blocks for Digital Markets}~»,
  \emph{Extropy}, vol.~16, 1 janvier 1996~:
  \url{https://github.com/Extropians/Extropy/blob/master/Extropy-16.pdf}~;
  Nick Szabo, «~\emph{Smart Contracts: Formalizing and Securing
  Relationships on Public Networks}~», \emph{First Monday}, vol.~2, no.
  9, 1 septembre 1997~:
  \url{https://firstmonday.org/ojs/index.php/fm/article/view/548/469}.}.

Nick Szabo avait une personnalité curieuse et éclectique, de sorte qu'il
s'intéressait à une multitude de domaines, tels que l'informatique,
l'économie, la politique et la biologie, et écrivait de manière
prolifique à leur sujet\footnote{On peut retrouver les écrits de Nick
  Szabo sur son ancienne page personnelle szabo.best.vwh.net et sur son
  blog Unenumerated débuté en 2005. -- Archive de la page personnelle~:
  \url{https://web.archive.org/web/20160709091851/http://szabo.best.vwh.net/}~;
  Unenumerated~: \url{https://unenumerated.blogspot.com/}.}. Il avait un
intérêt particulier pour le droit, dont il possédait une conception
libérale et jusnaturaliste, un intérêt qui le pousserait par la suite à
retourner étudier et à obtenir un diplôme dans la discipline en 2006.

Il avait travaillé pendant six mois comme consultant pour DigiCash à
Amsterdam vers 1995. Il y avait appris le rôle néfaste (et, finalement,
fatal) des tiers de confiance. C'est de cette expérience que provenait
son obsession pour la minimisation de la confiance, qu'il s'efforçait à
mettre en valeur au sein de ses travaux\footnote{Nick Szabo,
  \emph{Trusted Third Parties are Security Holes}, 2001, archive~:
  \url{https://web.archive.org/web/20020423191203/http://szabo.best.vwh.net/ttps.html}.}.

Hal Finney, comme déjà indiqué dans le chapitre précédent, était un
informaticien et cryptographe qui vivait dans la région de Los Angeles.
Extropien et cypherpunk de la première heure, il travaillait alors pour
Phil Zimmermann sur le développement de PGP, officieusement depuis 1992,
puis officiellement à partir de 1996. Hal Finney s'était aussi passionné
pour les idées de David Chaum dont son fameux eCash\footnote{«~Lorsque
  j'ai découvert les travaux de Chaum, j'ai été époustouflé. Le premier
  article que j'ai trouvé, je crois, était son article dans CACM, qui
  donnait un aperçu de beaucoup des choses qui étaient possibles. J'ai
  commencé à essayer de retrouver d'autres articles de Chaum. On y
  trouvait toutes les techniques nécessaires pour faire fonctionner le
  monde de Vinge, des techniques que Vinge connaissait apparemment
  longtemps avant moi.~» -- Hal Finney, \emph{Why remailers...},
  /11/1992 01:30:02 UTC~:
  \url{https://cypherpunks.venona.com/date/1992/11/msg00108.html}.}.

Wei Dai était un jeune cryptographe sino-américain vivant à Seattle.
Ayant fui la Chine communiste et émigré aux États-Unis avec ses parents
à l'âge de 10 ans\footnote{«~Ayant fui la Chine communiste et émigré aux
  États-Unis avec sa famille à l'âge de 10 ans~»~: Wei Dai, \emph{A tale
  from Communist China}, /10/2020 17:37 UTC~:
  \url{https://www.lesswrong.com/posts/osYFcQtxnRKB4F4HA/a-tale-from-communist-china}~;
  Wei Dai, \emph{Re: Tell Your Rationalist Origin Story}, /09/2009 02:53
  UTC~:
  \url{https://www.lesswrong.com/posts/BHMBBFupzb4s8utts/tell-your-rationalist-origin-story?commentId=ByhZPzBDyYdYs9SRN}.},
il avait réussi à se faire une place dans le monde du travail et avait
été rapidement engagé par Microsoft, où il avait participé à
l'élaboration de plusieurs brevets\footnote{«~participé à l'élaboration
  de plusieurs brevets~»~: Voir les brevets US5724279A et US5724279
  assignés à Microsoft. Wei Dai est donc \emph{a priori} son nom civil.}.
Il avait découvert le mouvement cypherpunk en 1994 et s'y était joint.
Le jeune prodige avait contribué au domaine de la cryptographie
notamment avec Crypto++, une bibliothèque de fonctions cryptographiques
en C++, et Pipenet, un protocole de communication anonyme\footnote{«~
  Pipenet, un protocole de communication anonyme~»~: Wei Dai,
  \emph{PipeNet description}, /01/1998 07:53:25 UTC~:
  \url{https://cypherpunks.venona.com/date/1998/01/msg00878.html}~; Wei
  Dai, \emph{PipeNet 1.1}, /11/1998 23:33:49 UTC~:
  \url{http://www.weidai.com/pipenet.txt}.}. Il s'était intéressé aux
monnaies numériques et aux contrats autonomes à partir de 1995, et avait
conceptualisé un modèle de crédit anonyme en 1997\footnote{«~un modèle
  de crédit anonyme en 1997~»~: Wei Dai, \emph{anonymous credit},
  /04/1997 09:08:04 UTC~:
  \url{https://cypherpunks.venona.com/date/1997/04/msg00398.html}.}. En
1998, Wei Dai disait être «~fasciné par la crypto-anarchie de Tim May~»,
où «~l'État {[}n'était{]} pas temporairement anéanti mais définitivement
oublié et inutile~» et où «~la violence {[}était{]} impossible parce que
ses membres ne {[}pouvaient{]} pas être reliés à leur vrai nom ou à leur
localisation géographique\footnote{Wei Dai, \emph{b-money}, /11/1998
  23:33:49 UTC, archive~:
  \url{https://web.archive.org/web/19990219124653/http://www.eskimo.com/~weidai/bmoney.txt}.}~».

Lawrence H. White, dit Larry White, et George A. Selgin étaient deux
économistes ayant été formés dans des universités prestigieuses. Ils
étaient tous les deux inspirés par les idées de l'école autrichienne
d'économie, sans pour autant y adhérer pleinement. Ils avaient été
marqués par les travaux de Friedrich Hayek, et notamment par son ouvrage
\emph{The Denationalization of Money} publié en 1976 qui faisait
l'apologie de la concurrence absolue dans le domaine monétaire et
bancaire. C'est pourquoi, à partir des années 80, ils s'étaient évertués
à promouvoir le système de la banque libre dans lequel des monnaies
privées pourraient être librement émises par des sociétés financières,
menant à un équilibre de marché.

Ces individus, présents sur la liste libtech-l, souhaitaient améliorer
la monnaie. Ils avaient vu la chute de DigiCash et l'échec d'eCash, et
étaient conscients des problèmes liés aux tiers de confiance. C'est
ainsi que Wei Dai, Nick Szabo et Hal Finney ont tous les trois développé
leur propre concept de monnaie numérique~: Wei Dai a imaginé un concept
appelé b-money, Nick Szabo a mis au point un modèle nommé bit gold et
Hal Finney a construit le système RPOW.

Leurs projets se fondaient sur la notion de preuve de travail, une
notion qui avait été mise en pratique en 1997 par Adam Back avec son
algorithme Hashcash, dont le but initial était de fournir un moyen de
lutter contre le courrier électronique indésirable\footnote{Pour une
  explication technique de la preuve de travail, se référer à la section
  dédiée dans le chapitre~\hyperref[ch:confirmation]{8}.}. Le cypherpunk
britannique avait pensé à en faire la base d'une monnaie numérique, mais
il avait conscience que les preuves de travail ainsi obtenues ne
pouvaient pas être transférées d'une manière pleinement distribuée (à
cause du problème de la double dépense) et qu'il fallait par conséquent
passer par un système de monnaieries à la eCash\footnote{Adam Back,
  \emph{Re: Bypassing the Digicash Patents}, /04/1997 09:09:37 UTC~:
  \url{https://cypherpunks.venona.com/date/1997/04/msg00822.html}.}.

L'idée d'utiliser ce type de preuve de travail comme base de la devise
était répandue. Par exemple, en 1996, Ronald Rivest et Adi Shamir
avaient décrit MicroMint, un système de micropaiement centralisé dont
les pièces devaient être impossibles à contrefaire grâce à la production
de preuves de travail\footnote{Ronald L. Rivest, Adi Shamir,
  «~\emph{PayWord and MicroMint: Two Simple Micropayment Schemes}~», in
  \emph{Security Protocols Workshop}, 1996: pp.~69--87~:
  \url{https://people.csail.mit.edu/rivest/pubs/RS96a.pdf}. Voir aussi
  \url{https://people.csail.mit.edu/rivest/pubs/RS96a.slides.pdf}.}.
Mais elle manquait d'un bon agencement qui puisse lui donner vie de
manière robuste et durable.

\section*{Le concept b-money}\label{le-concept-b-money}
\addcontentsline{toc}{section}{Le concept b-money}

\markright{Le concept b-money}

Le premier concept de monnaie numérique à être sorti de la liste
libtech-l était b-money, proposé par Wei Dai. Il s'agissait d'un concept
de protocole décentralisé gérant une unité de compte du même nom, la
b-money, dont la valeur était censée suivre le cours d'un panier de
marchandises.

Wei Dai a travaillé sur son idée à partir de 1995\footnote{«~Wei Dai a
  travaillé sur son idée à partir de 1995~»~: Wei Dai, \emph{Re: AALWA:
  Ask any LessWronger anything}, /03/2014 06:14 UTC~:
  \url{https://www.lesswrong.com/posts/YdfpDyRpNyypivgdu/aalwa-ask-any-lesswronger-anything?commentId=ZvJDryrskf2Gy6nhG}.}.
Comme il l'a expliqué par la suite, sa motivation était de «~rendre
possible l'établissement d'une économie en ligne qui soit purement
volontaire, une économie qui ne puisse pas être taxée et réglementée par
la menace de violence\footnote{Morgen E. Peck, «~\emph{Bitcoin: The
  Cryptoanarchists' Answer to Cash}~», \emph{IEEE Spectrum}, 30 mai
  2012~:
  \url{https://spectrum.ieee.org/bitcoin-the-cryptoanarchists-answer-to-cash}.}~».

Le texte descriptif de b-money a été publié le 26 novembre 1998 par Wei
Dai sur sa page personnelle\footnote{«~Le texte descriptif de b-money a
  été publié le 26 novembre 1998 par Wei Dai sur sa page personnelle~»~:
  Wei Dai, \emph{b-money}, /11/1998 23:33:49 UTC, archive~:
  \url{https://web.archive.org/web/19990219124653/http://www.eskimo.com/~weidai/bmoney.txt}.}.
Il en a fortuitement partagé le lien à la liste de diffusion cypherpunk
dans un courriel où il décrivait b-money comme «~un nouveau protocole
d'échange monétaire et d'exécution des contrats pour les
pseudonymes\footnote{Wei Dai, \emph{PipeNet 1.1 and b-money}, /11/1998
  23:33:49 UTC~:
  \url{https://cypherpunks.venona.com/date/1998/11/msg00941.html}.}~».

Le texte était court (un peu plus de 1000 mots) mais riche
conceptuellement. Deux versions du protocole étaient décrites par Wei
Dai. L'une était irréalisable mais simple, l'autre était plus complexe
mais partait d'hypothèses plus réalistes.

Dans la première version du protocole, chaque participant faisait partie
d'un réseau pair à pair intraçable. Chacun était identifié par un
«~pseudonyme numérique~» (c'est-à-dire une clé publique) et chaque
message transactionnel était signé par l'expéditeur et chiffré pour le
destinataire. Chacun maintenait une base de données séparée qui
recensait combien d'unités de b-money possédait chaque pseudonyme.

La création monétaire était ouverte à tous les participants et se
faisait par preuve de travail en diffusant la solution d'un problème
informatique connu et précédemment non résolu. Le nombre d'unités créées
dépendait alors du coût de cet effort exprimé par rapport à un panier
standard de marchandises, pouvant inclure des métaux précieux par
exemple~: lorsque son cours par rapport au panier de marchandises
augmentait, les acteurs économiques déployaient plus de puissance de
calcul pour abreuver le marché~; à l'inverse lorsque son cours baissait,
les acteurs économiques étaient incités à utiliser moins de puissance de
calcul, ce qui ralentissait la production de b-money. Il s'agissait donc
d'un «~stablecoin~» décentralisé avant l'heure\footnote{Ce
  fonctionnement pour garantir la stabilité de la b-money ne manque pas
  de rappeler le stablecoin géré par Maker DAO sur Ethereum, appelé
  précisément le dai~! Plus tard, Wei Dai a reproché à Satoshi Nakamoto
  la politique monétaire fixe de Bitcoin, qui devait entraîner selon lui
  «~une forte volatilité du prix qui impose un coût élevé pour ses
  utilisateurs~». -- Wei Dai, \emph{Re: Bitcoins are not digital
  greenbacks}, /04/2013 07:56 UTC~:
  \url{https://www.lesswrong.com/posts/P9jggxRZTMJcjnaPw/bitcoins-are-not-digital-greenbacks?commentId=3XvTroRzb23NpHQDc}.}.

Le système offrait également la possibilité de créer et d'exécuter des
contrats directement sur le réseau, grâce à un procédé rudimentaire de
dépôt fiduciaire. Dans un contrat, les parties impliquées étaient
contraintes de mettre en jeu une caution et de désigner un arbitre qui
avait pour rôle d'intervenir en cas de litige. Sans résolution à
l'amiable, c'était le réseau qui devait trancher selon les éléments
diffusés, la position de l'arbitre étant en théorie privilégiée.

Dans la seconde version du protocole, le registre de propriété n'était
plus conservé par tout le monde, mais uniquement par un sous-ensemble de
participants appelés des serveurs. Les participants à une transaction
devaient alors vérifier que leur transaction avait bien été traitée en
envoyant des requêtes à un échantillon aléatoire de serveurs. Puisqu'il
était nécessaire de faire confiance à ceux-ci dans une certaine mesure,
un mécanisme économique de preuve d'enjeu était mis en place pour faire
en sorte qu'ils restent honnêtes. Chaque serveur devait déposer un
montant de b-money sur un compte spécial afin d'être pénalisé en cas de
mauvaise conduite, et était en outre contraint de publier régulièrement
sa création de monnaie et son registre.

Le concept de b-money présenté par Wei Dai était donc assez ingénieux
pour l'époque. Toutefois, il n'était pas fonctionnel et présentait
quelques défauts majeurs. D'abord, la première version du protocole
était impossible à mettre en place à grande échelle, notamment parce
qu'elle ne résistait pas à la multiplication excessive des identités
(attaque Sybil), chacun pouvant ajouter facilement de nouveaux
ordinateurs sur le réseau. Ensuite, la seconde version semblait plus
réaliste, mais avait pour effet de centraliser le système en un petit
nombre de serveurs, le rendant ainsi plus vulnérable aux attaques et à
la corruption. Enfin, la stabilité par rapport à un panier de devises
aurait requis ce qu'on appelle aujourd'hui un système décentralisé
d'oracles, ce qui est un problème complexe à résoudre\footnote{Satoshi
  avait pensé lui-même au problème de l'oracle. Il écrivait ainsi en
  février 2009~: «~Il n'y a personne pour agir en tant que banque
  centrale ou réserve fédérale afin d'ajuster l'offre monétaire au fur
  et à mesure que le nombre d'utilisateurs augmente. Il aurait fallu
  qu'un tiers de confiance détermine la valeur, car je ne connais aucun
  moyen pour un logiciel de connaître la valeur des choses dans le monde
  réel. S'il existait un moyen astucieux, ou si nous voulions faire
  confiance à quelqu'un pour gérer activement l'offre monétaire afin de
  l'ancrer à quelque chose, les règles auraient pu être programmées à
  cet effet.~» -- Satoshi Nakamoto, \emph{Re: Bitcoin open source
  implementation of P2P currency}, 18 février 2009~:
  \url{https://p2pfoundation.ning.com/forum/topics/bitcoin-open-source?commentId=2003008:Comment:9562}.}.

b-money a attiré l'attention des cypherpunks, et en particulier celle
d'Adam Back. Néanmoins, Wei Dai n'a jamais implémenté son modèle, non
seulement parce qu'il était dysfonctionnel, mais aussi à cause de sa
désillusion à l'égard de la cryptoanarchie. Comme il l'a affirmé en
2014~:

«~Je n'ai pris aucune mesure pour coder b-money. Cela a en partie été dû
au fait que b-money n'était pas encore un concept complètement pratique,
mais je n'ai pas continué à travailler sur ce concept parce que j'étais
un peu désenchanté par la cryptoanarchie au moment où j'ai fini d'écrire
b-money, et que je ne prévoyais pas qu'un système comme celui-ci, une
fois mis en œuvre, pourrait attirer autant d'attention et d'utilisation
en dehors d'un petit groupe de cypherpunks inconditionnels\footnote{Wei
  Dai, \emph{Re: AALWA: Ask any LessWronger anything}, /03/2014 20:34
  UTC~:
  \url{https://www.lesswrong.com/posts/YdfpDyRpNyypivgdu/aalwa-ask-any-lesswronger-anything?commentId=XKwphuwm366RegQ3d}.}.~»

\section*{Le modèle bit gold}\label{le-moduxe8le-bit-gold}
\addcontentsline{toc}{section}{Le modèle bit gold}

\markright{Le modèle bit gold}

Le deuxième modèle qui a émergé de la liste libtech-l était le système
bit gold. Celui-ci était censé gérer la création et les échanges d'une
ressource virtuelle appelée le bit gold. Contrairement à l'e-gold qui
était garanti par de l'or physique, ou la b-money indexée en théorie sur
un panier de marchandises, le bit gold ne devait être adossé à aucun
autre bien, mais posséder une rareté infalsifiable intrinsèque, et
constituer ainsi un or intégralement numérique.

C'est au cours de l'année 1998 que Nick Szabo a développé son idée de
bit gold, qu'il a décrite initialement sur libtech-l, avant d'héberger
une ébauche de livre blanc en 1999 sur son site personnel\footnote{Nick
  Szabo, \emph{Bit Gold: Towards Trust-Independent Digital Money}, 2005,
  archive~:
  \url{https://web.archive.org/web/20140406003811/http://szabo.best.vwh.net/bitgold.html}.}.
Il a présenté bit gold au grand public en décembre 2005 dans un article
publié sur son blog Unenumerated\footnote{Nick Szabo, \emph{Bit gold},
  29 décembre 2005~:
  \url{https://unenumerated.blogspot.com/2005/12/bit-gold.html}.}. La
logique derrière bit gold était de minimiser la confiance dans le but de
reproduire autant que possible la cherté de production des métaux
précieux dans le cyberespace.

L'élément central du protocole était que la création monétaire se
faisait par preuve de travail~: les morceaux de bit gold étaient créés
grâce à la puissance de calcul des ordinateurs\footnote{Le calcul de la
  preuve de travail dans bit gold ne passait pas par l'inversion
  partielle d'une fonction de hachage (Hashcash) mais par une
  \emph{secure benchmark function} qui mesurait la difficulté du
  problème à résoudre sur une machine précise. Cela devait permettre
  d'approximer le niveau d'énergie utilisé. -- Voir Nick Szabo,
  \emph{Intrapolynomial Cryptography}, 1999, archive~:
  \url{https://web.archive.org/web/20011217091748/http://szabo.best.vwh.net/intrapoly.html}.}.
Chaque solution était calculée à partir d'une autre, ce qui conduisait à
former une chaîne de preuves de travail. Les acteurs responsables de
cette production étaient appelés des «~mineurs~» par Szabo.

La date et l'heure de production de ces preuves de travail étaient
certifiées au moyen de serveurs d'horodatage multiples. Cette diversité,
bien que peu satisfaisante, permettait de limiter le risque lié à un
service particulier.

La garantie de la possession et des échanges se faisait par le biais
d'un registre public de titres de propriété\footnote{«~un registre
  public de titres de propriété~»~: Nick Szabo, \emph{Secure Property
  Titles with Owner Authority}, 1998, archive~:
  \url{https://web.archive.org/web/20020202165211/http://szabo.best.vwh.net/securetitle.html}.}.
Les utilisateurs étaient identifiés par leur clé publique et signaient
les transactions grâce à leur clé privée. Le registre était vérifié et
maintenu par un réseau de serveurs appelé «~club de propriété~»,
serveurs qui se mettaient d'accord par un algorithme de consensus~: le
\emph{Byzantine Quorum System} de Malkhi et Reiter\footnote{«~le
  \emph{Byzantine Quorum System} de Malkhi et Reiter~»~: Dahlia Malkhi,
  Michael Reiter, «~\emph{Byzantine quorum systems}~», in
  \emph{Proceedings of the 29th Annual ACM Symposium on Theory of
  Computing}, 1997.}. Ainsi, n'importe qui pouvait se référer à ce
registre pour connaître le propriétaire d'un morceau de bit gold.

Il est intéressant de noter que les trois éléments constitutifs de bit
gold -- les preuves de travail, leurs horodatages et le registre de
propriété -- étaient séparés. En particulier, ils étaient pris en charge
par des acteurs différents~: les mineurs, les serveurs d'horodatage et
les membres du club de propriété. Plus tard, dans Bitcoin, ces trois
éléments seraient combinés en un seul et même concept~: la chaîne de
blocs.

Malgré sa volonté de minimiser la confiance, le système imaginé par
Szabo possédait quelques problèmes conceptuels. Tout d'abord, la façon
dont étaient produits les morceaux de bit gold faisaient qu'ils
n'étaient pas fongibles, c'est-à-dire qu'ils ne pouvaient pas être
mélangés entre eux, et devaient donc être évalués sur un
marché\footnote{«~les morceaux de bit gold {[}...{]} devaient donc être
  évalués sur un marché~»~: Nick Szabo, \emph{Bit gold markets}, 10
  avril 2008~:
  \url{https://unenumerated.blogspot.com/2008/04/bit-gold-markets.html}.},
de sorte à pouvoir être utilisés pour servir de base à une réelle unité
de compte homogène. Ensuite, bit gold faisait usage de serveurs
d'horodatage centralisés, qui formaient des points de défaillance
uniques non négligeables. Enfin, le système reposait sur un algorithme
de consensus dit «~classique~» qui requérait que les membres du club de
propriété soient choisis à l'avance, qu'ils soient connus de tous, et
que 66~\% d'entre eux se comportent correctement.

À l'époque, bit gold était pensé comme un système de règlement
permettant de gérer une monnaie de réserve rare, et au-dessus duquel
serait construit une économie bancaire libre, si possible utilisant le
modèle chaumien. Nick Szabo a longtemps réfléchi à comment mettre en
application son idée, redemandant même de l'aide en avril 2008 dans un
commentaire sur son blog\footnote{«~{[}Bit gold{]} bénéficierait
  grandement d'une démonstration, d'un marché expérimental (avec par ex.
  un tiers de confiance pour se substituer à la sécurité complexe
  nécessaire au système réel). Quelqu'un veut m'aider à en programmer
  une~?~» -- Nick Szabo, \emph{Re: Bit gold markets}, 10 avril 2008,
  archive~:
  \url{https://web.archive.org/web/20171227190431/http://unenumerated.blogspot.com/2008/04/bit-gold-markets.html?showComment=1207799580000\#c3741843833998921269}.}.
Szabo n'a jamais pu mettre en œuvre son concept, contrairement à Hal
Finney qui a essayé au moyen de son système RPOW.

\section*{Le système RPOW}\label{le-systuxe8me-rpow}
\addcontentsline{toc}{section}{Le système RPOW}

\markright{Le système RPOW}

Hal Finney a repris le concept de Nick Szabo et l'a simplifié pour
l'implémenter dans un système inédit~: celui des preuves de travail
réutilisables ou \emph{Reusable Proofs of Work}, abrégées en RPOW, qu'il
a décrit le 15 août 2004. Ce système se basait sur un serveur
transparent qui permettait de rendre transférables les preuves de
travail produites par Hashcash. Ainsi, le modèle de sécurité ne venait
pas d'un réseau distribué comme dans les cas de b-money et de bit
gold\footnote{Voir Hal Finney, \emph{RPOW Theory}, 15 août 2004,
  archive~:
  \url{https://web.archive.org/web/20040815154951/http://rpow.net/theory.html}.}.

Dans ce système, les jetons de preuve de travail réutilisable étaient
gérés par un serveur qui se chargeait de les signer à l'aide du
chiffrement RSA. Ils étaient fabriqués par la production d'une preuve de
travail via Hashcash, ou bien à partir d'un jeton de RPOW précédent.
Chaque jeton de RPOW se composait d'une valeur (définie comme une
puissance de 2) et des données relatives à la signature du serveur. Un
utilisateur pouvait par conséquent vérifier lui-même l'intégrité du
jeton.

Le système RPOW reposait sur l'utilisation d'un serveur central qui
s'occupait de détruire et de recréer les preuves de travail impliquées
dans chaque opération, notamment en vérifiant qu'elles ne faisaient pas
l'objet d'une double dépense. Afin d'assurer la divisibilité de l'unité
de compte, le système permettait de séparer une RPOW en plusieurs RPOW
de valeur moindre et de combiner plusieurs RPOW pour en obtenir une
seule.

Lors d'un paiement, l'expéditeur donnait ses jetons de RPOW au
destinataire qui s'empressait de communiquer avec le serveur pour
recevoir un ou plusieurs nouveaux jetons, dont la valeur globale est
égale à la valeur en entrée. Le fonctionnement des RPOW était ainsi
similaire à celui des billets numériques dans eCash~: les jetons de RPOW
reposaient sur les informations qu'ils contenaient et pouvaient être
transférés de manière confidentielle, mais chaque transaction demandait
d'interagir avec le serveur pour garantir l'absence de double dépense.

Le modèle de sécurité reposait sur le type de serveur mis en place~: il
s'agissait d'un «~serveur transparent\footnote{Hal Finney, \emph{RPOW
  Security}, 15 août 2004, archive~:
  \url{https://web.archive.org/web/20040815154806/http://rpow.net/security.html}.}~»
utilisant le cryptoprocesseur IBM 4758 Secure Cryptographic Coprocessor,
un cryptoprocesseur de haute sécurité résistant aux falsifications, qui
permettait, par un procédé d'authentification conçu par IBM, de vérifier
quels programmes étaient exécutés sur la machine. De cette manière, un
utilisateur externe pouvait s'assurer à tout instant que le serveur RPOW
faisait fonctionner le bon programme, dont le code était par ailleurs
disponible publiquement.

Avec son système RPOW, Hal Finney tentait ainsi de réduire la confiance
requise à un minimum. Le système était confidentiel dans le sens où
l'utilisateur n'avait jamais à s'identifier auprès du serveur et pouvait
communiquer avec lui de manière chiffrée. La transparence du serveur
permettait, autant que faire se peut, de s'assurer que le système
n'était pas corrompu. En particulier, on pouvait rigoureusement supposer
que la quantité de jetons de RPOW dépendait d'une production réelle de
preuves de travail, propriété qui permettait d'assimiler les jetons de
RPOW à de l'or. Il s'agissait, en somme, d'une mise en œuvre partielle
du bit gold de Nick Szabo\footnote{«~Il s'agissait, en somme, d'une mise
  en œuvre partielle du bit gold de Nick Szabo~»~: À propos du bit gold
  de Nick Szabo, Hal Finney écrivait~:}.

Le système a été lancé en production le même jour que sa description, le
15 août 2004. Hal Finney l'a annoncé sur la liste des
cypherpunks\footnote{«~Hal Finney l'a annoncé sur la liste des
  cypherpunks~»~: Hal Finney, \emph{RPOW - Reusable Proofs of Work},
  /08/2004 17:43:09 UTC~:
  \url{https://lists.cpunks.org/pipermail/cypherpunks-legacy/2004-August/134945.html}.}
et l'annonce a été retransmise sur la liste de Metzdowd.com par Robert
Hettinga\footnote{«~l'annonce a été retransmise sur la liste de
  Metzdowd.com par Robert Hettinga~»~: Robert Hettinga, \emph{FW: RPOW -
  Reusable Proofs of Work}, /08/2004 18:36:51 UTC~:
  \url{https://www.metzdowd.com/pipermail/cryptography/2004-August/007362.html}.}.
Le système a été mis à jour plusieurs fois pour améliorer son
fonctionnement et est resté opérationnel pendant des mois.

Hal Finney a présenté son système à la CodeCon 2005 organisée à San
Francisco. Il y a fait part des usages qu'il envisageait pour RPOW à
savoir le transfert de la valeur, la régulation du courrier indésirable
(dans la droite lignée de Hashcash), le commerce dans les jeux vidéos,
le jeu d'argent en ligne comme le poker, et l'anti-parasitisme sur les
protocoles de partage de fichiers comme BitTorrent\footnote{Hal Finney,
  \emph{Reusable Proofs of Work}, 1 février 2005, archive~:
  \url{https://web.archive.org/web/20050204193327/http://rpow.net/slides/slide001.html}.}.
Fidèle à son optimisme, Hal Finney envisageait un avenir prometteur pour
RPOW et comptait notamment multiplier le nombre de serveurs autour du
monde, une fois que son déploiement initial serait réalisé\footnote{«~comptait
  notamment multiplier le nombre de serveurs autour du monde~»~: Hal
  Finney, \emph{World of RPOW}, 15 août 2004~:
  \url{http://rpow.net/world.html}~; archive~:
  \url{https://web.archive.org/web/20040816004128/http://rpow.net/world.html}.}.

Toutefois, RPOW présentait des défauts intrinsèques qui peuvent
expliquer pourquoi il n'a pas rencontré le succès escompté.
L'inconvénient majeur était son modèle de sécurité qui démontrait une
certaine faiblesse~: le ou les serveurs devaient être connus et
pouvaient être arrêtés facilement, de sorte qu'ils constituaient des
points de défaillance unique. De plus, un autre inconvénient venait de
sa politique monétaire qui, bien que théoriquement viable, n'était pas
spécialement attractive en raison de la hausse exponentielle des
performances informatiques.

De ce fait, l'utilisation réelle de RPOW a été anecdotique. Le système
était loin d'être parfait et ne pouvait pas, de toute évidence, devenir
un système monétaire solide. Toutefois, il a eu le mérite de constituer
une preuve de concept expérimentale, quatre ans avant Bitcoin.

\section*{Le projet Ripple}\label{le-projet-ripple}
\addcontentsline{toc}{section}{Le projet Ripple}

\markright{Le projet Ripple}

Les cypherpunks n'étaient pas les seuls à essayer de construire des
systèmes distribués qui puissent servir à l'échange monétaire. C'était
le cas du développeur canadien Ryan Fugger, qui en 2004 a conçu un
protocole de crédit distribué du nom de Ripple. Ce protocole s'inspirait
du concept du système d'échange local (SEL) imaginé par Michael Linton
dans les années 1980. Fugger lui-même avait participé à un tel système à
Vancouver avant de concevoir Ripple\footnote{«~Fugger lui-même avait
  participé à un tel système à Vancouver~»~: Bailey Reutzel,
  \emph{Disruptor Chris Larsen Returns with a Bitcoin-Like Payment
  System}, 7 décembre 2012~:
  \url{https://www.americanbanker.com/news/disruptor-chris-larsen-returns-with-a-bitcoin-like-payment-system}~;
  archive~:
  \url{https://web.archive.org/web/20140323151243/http://www.paymentssource.com/news/disruptor-chris-larsen-returns-with-bitcoin-like-payments-system-3012580-1.html?zkPrintable=1&nopagination=1}.}.
Son invention était ainsi un pur produit du localisme monétaire.

Ryan Fugger a publié le livre blanc de Ripple en 2004\footnote{Ryan
  Fugger, \emph{Money as IOUs in Social Trust Networks \& A Proposal for
  a Decentralized Currency Network Protocol}, version 2, 18 avril 2004,
  archive~:
  \url{https://web.archive.org/web/20060221162102/http://ripple.sourceforge.net/decentralizedcurrency.pdf}.}.
Le concept se fondait sur l'idée que la monnaie était essentiellement
constituée de reconnaissances de dette (\emph{IOUs}), c'est-à-dire de
crédit.

Le système Ripple s'établissait sur un réseau pair à pair dont les liens
étaient des relations de crédit entre des personnes. Chaque relation
était composée de deux paramètres~: la dette existante, qui indiquait
combien une partie devait à l'autre, et la dette potentielle, qui
prenait en compte la propension à prêter et à emprunter des deux
parties. Ripple constituait ainsi un système où tous les participants
étaient des banquiers. Concernant la monnaie de base, le protocole
pouvait gérer de multiples unités de compte (du dollar, de l'euro ou
même des heures de travail), mais celles-ci devaient être converties
pour être transférées dans une autre devise.

Dans Ripple, les paiements se faisaient par le routage d'une série
d'emprunts. En supposant des relations de confiance entre Alice et Bob,
Bob et Carole, et Carole et David, Alice réalisait un paiement de 10~\$
vers David en prêtant 10~\$ à Bob, et en demandant à Bob de faire de
même auprès de Carole, puis à Carole de faire de même auprès de David.
Le compte de David était alors crédité de 10~\$ issus de la création
monétaire d'Alice. Ce fonctionnement par propagation du crédit au sein
d'un réseau de confiance explique le nom du projet, le mot \emph{ripple}
signifiant ondulation en anglais.

La compensation se faisait par la découverte de cycles de crédit dans le
réseau. Si Bob devait 5~\$ à Alice, Carole 5~\$ à Bob et Alice 5~\$ à
Carole, alors leurs dettes mutuelles disparaissaient, ce qui permettait
à tous les acteurs d'avoir une capacité d'emprunt plus grande pour
recevoir de futurs paiements. Une dette pouvait également être éteinte
par son règlement entre les deux parties dans l'unité de compte
indiquée.

En 2006, dans l'idée de faire progresser son projet, Ryan Fugger a mis
en ligne une preuve de concept appelée Ripplepay\footnote{«~une preuve
  de concept appelée Ripplepay~»~:
  \url{https://web.archive.org/web/20070702210719/http://ripplepay.com/}.}.
Celle-ci était basée sur un serveur central (``) et permettait aux
utilisateurs de se connecter avec une simple adresse de courrier
électronique. Ryan Fugger a également créé un Google Group en janvier
2007\footnote{«~Ryan Fugger a également créé un Google Group en janvier
  2007~»~: Ryan Fugger, \emph{Welcome...}, /01/2007 23:47:26 UTC~:
  \url{https://groups.google.com/g/rippleusers/c/oRDaKz-qPjQ/m/zHV3hMPwMg0J}.}.

Malgré l'enthousiasme de sa communauté et quelques milliers
d'utilisateurs, Ripple possédait des défauts inhérents à son
fonctionnement distribué. En particulier, il souffrait du «~problème de
l'engagement décentralisé\footnote{fiatjaf, \emph{Ripple and the problem
  of the decentralized commit}, /10/2020 13:56 UTC~:
  \url{https://fiatjaf.com/3cb7c325.html}.}~»~: durant un paiement, les
participants ne pouvaient pas s'engager d'une façon sûre pour assurer la
chaîne de prêts\footnote{Ce problème a été résolu d'une certaine manière
  par le réseau Lightning qui possède la même structure que Ripple, à
  l'exception que l'unité échangée n'est pas du crédit à proprement
  parler. -- Voir fiatjaf, \emph{The Lightning Network solves the
  problem of the decentralized commit}, /10/2020 19:09 UTC~:
  \url{https://fiatjaf.com/e3624832.html}.}.

Voyant que son implémentation n'allait nulle part, Ryan Fugger a laissé
les rênes de son projet entre les mains des dirigeants de l'entreprise
OpenCoin Inc., Chris Larsen et Jed McCaleb, en novembre 2012, qui
l'avaient approché quelques mois plus tôt\footnote{«~Ryan Fugger a
  laissé les rênes de son projet entre les mains des dirigeants de
  l'entreprise OpenCoin Inc., Chris Larsen et Jed McCaleb, en novembre
  2012~»~: Ryan Fugger, \emph{Ripple.com and taking the project to the
  next level}, /11/2012 21:21:12 UTC~:
  \url{https://groups.google.com/g/rippleusers/c/IVin3Qwrp7k/m/urzaH_VrQcQJ}.}.
Ces derniers voulaient combiner son idée avec un nouvel algorithme de
consensus, développé par Jed, David Schwartz et Arthur Britto. Le
résultat a été sensiblement différent du concept initial, faisant
intervenir une unité de compte native, le XRP, et étant bien plus
centralisé et contrôlé que ce qu'on attendrait d'un protocole de crédit
universel. La société OpenCoin a été renommée en Ripple Labs en 2013.
Ryan Fugger a finalement modifié le nom de sa preuve de concept en
Rumplepay en 2020 pour éviter la confusion\footnote{«~Ryan Fugger a
  finalement modifié le nom de sa preuve de concept en Rumplepay en 2020
  pour éviter la confusion~»~: Ryan Fugger, \emph{New Name}, 26 août
  2020~: \url{https://rumplepay.com/}.}.

\section*{Vers Bitcoin}\label{vers-bitcoin}
\addcontentsline{toc}{section}{Vers Bitcoin}

\markright{Vers Bitcoin}

Tous ces concepts de monnaie numérique ont, directement ou
indirectement, mené à Bitcoin, soit parce que Satoshi Nakamoto avait
connaissance de ces projets, soit parce qu'il partageait les mêmes
références que leurs inventeurs. Bitcoin constituait en effet
l'aboutissement de ces tentatives de construire une forme de monnaie
numérique native du cyberespace.

Tout d'abord, Satoshi Nakamoto était pleinement familier de l'eCash de
David Chaum et avait de toute évidence lu les échanges des cypherpunks à
son sujet. Dans le livre blanc de Bitcoin, il y faisait clairement
référence au moment d'aborder le problème de la double dépense tout en
utilisant le terme \emph{mint} (traduit ici par monnaierie), qui était
un vocable courant parmi les cypherpunks pour désigner les banques
chaumiennes~:

«~Une solution courante consiste à introduire une autorité centrale de
confiance, ou monnaierie, qui vérifie chaque transaction pour s'assurer
qu'il n'y a pas de double dépense\footnote{Satoshi Nakamoto,
  \emph{Bitcoin: A Peer-to-Peer Electronic Cash System}, 31 octobre
  2008.}.~»

De plus, Satoshi Nakamoto a explicitement reconnu la faiblesse de eCash
dans ses interventions publiques et privées. Ainsi, dans un courriel
adressé à la liste de diffusion p2p-research en février 2009, il
réagissait à la comparaison entre Bitcoin et eCash par Martien van
Steenbergen en disant~:

«~Bien sûr, la plus grande différence est l'absence de serveur central.
C'était le talon d'Achille des systèmes chaumiens~; lorsque l'entreprise
centrale fermait ses portes, la monnaie disparaissait\footnote{Satoshi
  Nakamoto, \emph{{[}p2p-research{]} Re: Bitcoin open source
  implementation of P2P currency}, /02/2009 19:01:24 UTC~:
  \url{https://diyhpl.us/~bryan/irc/bitcoin-satoshi/p2presearch-again/p2pfoundation.net/backups/p2p_research-archives/2009-February.txt.gz}.}.~»

En privé, il écrivait aussi à Dustin Trammell en 2009~:

«~Vous savez, je pense qu'il y avait beaucoup plus de gens qui étaient
intéressés {[}par la monnaie électronique{]} dans les années 90, mais
après plus d'une décennie d'échecs de systèmes basés sur des tiers de
confiance (Digicash,~etc.), ils voient cela comme une cause perdue.
J'espère qu'ils sauront distinguer que c'est la première fois, à ma
connaissance, que nous essayons un système qui n'est pas fondé sur la
confiance\footnote{Satoshi Nakamoto, \emph{Re: Bitcoin v0.1 released},
  /01/2009 07:55:20 UTC~:
  \url{http://web.archive.org/web/20131204164149/http://www.dustintrammell.com/files/Satoshi_Nakamoto.zip}.
  -- Il a fait une remarque similaire sur le forum de Fondation P2P.
  Voir Satoshi Nakamoto, \emph{Re: Bitcoin open source implementation of
  P2P currency}, 15 février 2009~:
  \url{https://p2pfoundation.ning.com/forum/topics/bitcoin-open-source?commentId=2003008:Comment:9493}.}.~»

La dernière référence au modèle de Chaum était la première version du
livre blanc datant du mois d'août 2008 qui s'intitulait de façon limpide
«~\emph{Electronic Cash Without a Trusted Third Party}~» («~Un argent
liquide électronique sans tiers de confiance~» en français) et dont le
nom du fichier était ``\footnote{Gwern Branwen, \emph{Wei Dai/Satoshi
  Nakamoto 2009 Bitcoin emails}, 17 mars 2014~:
  \url{https://gwern.net/doc/bitcoin/2008-nakamoto}.}.

Satoshi Nakamoto s'était donc clairement intéressé à eCash avant de
concevoir Bitcoin. Toutefois, ce n'était pas le cas des concepts de
b-money, bit gold et RPOW, dont il n'avait vraisemblablement pas
connaissance en 2007. Ces systèmes ont cependant joué un rôle indirect
dans l'histoire de Bitcoin.

Comme raconté dans le chapitre~\hyperref[ch:mythe]{1}, dans sa
préparation à la publication de son concept en août 2008, Satoshi est
rentré en contact avec Adam Back, qui l'a renvoyé vers Wei Dai, car le
britannique avait remarqué les similitudes de Bitcoin avec b-money.
C'est à ce moment-là que Satoshi a ajouté la référence à b-money au
livre blanc.

Satoshi a appris l'existence du modèle bit gold imaginé par Szabo plus
tard, probablement grâce à la première intervention de Hal Finney sur la
liste de diffusion le 7 novembre 2008. Ce dernier a instantanément noté
la similarité entre Bitcoin et le système de Szabo~:

«~Je crois aussi qu'il y a une valeur potentielle dans une forme de
jeton infalsifiable dont le taux de production est prédictible et qui ne
peut pas être influencé par des personnes corrompues. Ceci serait plus
comparable à l'or qu'aux monnaies fiat. Nick Szabo a décrit il y a
plusieurs années ce qu'il appelait ``bit gold'' et ceci serait une
implémentation de ce concept\footnote{Hal Finney, \emph{Bitcoin P2P
  e-cash paper}, /11/2008 23:40:12 UTC~:
  \url{https://www.metzdowd.com/pipermail/cryptography/2008-November/014827.html}.}.~»

La référence à bit gold a fini par être ajoutée sur la page web de
Bitcoin.org au début de l'année 2009, aux côtés du lien vers le texte
descriptif de b-money\footnote{«~La référence à bit gold a fini par être
  ajoutée sur la page web de Bitcoin.org au début de l'année 2009~»~:
  \url{https://web.archive.org/web/20090303195936/http://bitcoin.org/}.}.

Satoshi Nakamoto a reconnu \emph{a posteriori} la ressemblance de ces
deux concepts avec son propre modèle. Le 20 juillet 2010, au sein d'une
discussion sur le forum parlant de la possible suppression de l'article
concernant Bitcoin par Wikipédia, il écrivait pour montrer le sérieux du
projet~:

«~Bitcoin est une implémentation de la b-money proposée par Wei Dai sur
la liste de diffusion des Cypherpunks en 1998 et du Bitgold proposé par
Nick Szabo\footnote{Satoshi Nakamoto, \emph{Re: They want to delete the
  Wikipedia article}, /07/2010 18:38:28 UTC~:
  \url{https://bitcointalk.org/index.php?topic=342.msg4508\#msg4508}.}.~»

Cette phrase, censée démontrer l'inscription de Bitcoin dans l'histoire
de la monnaie numérique, est restée gravée dans les esprits, à tel point
que la b-money et le bit gold sont régulièrement cités comme des
précurseurs de la cryptomonnaie.

En revanche, Satoshi Nakamoto n'a jamais indiqué à une seule occasion
qu'il connaissait le système RPOW de Hal Finney. Celui-ci n'était après
tout qu'un modèle eCash basé sur la preuve de travail, dont le serveur
central avait la particularité d'être transparent aux yeux des
utilisateurs. Néanmoins, Hal Finney a joué un rôle majeur dans les
débuts de Bitcoin et a évoqué son système en 2013 sur
Bitcointalk\footnote{«~Je m'intéressais depuis longtemps aux systèmes de
  paiement cryptographiques. De plus, j'avais eu la chance de rencontrer
  Wei Dai et Nick Szabo et de correspondre avec eux, généralement
  reconnus pour avoir créé des idées qui allaient se concrétiser avec
  Bitcoin. J'avais tenté de créer ma propre monnaie basée sur la preuve
  de travail, appelée RPOW. J'ai donc trouvé Bitcoin fascinant.~» -- Hal
  Finney, \emph{Bitcoin and me}, /03/2013 20:40:02 UTC~:
  \url{https://bitcointalk.org/index.php?topic=155054.msg1643833\#msg1643833}.},
de sorte que RPOW est aujourd'hui lui aussi considéré comme un
prédécesseur de la découverte de Satoshi.

La proximité des idées de ces trois hommes avec Bitcoin est étonnante de
prime abord, si bien que beaucoup ont spéculé que Satoshi Nakamoto était
l'un ou plusieurs d'entre eux. Ces hommes, qui ont publiquement
découvert l'existence de Bitcoin assez rapidement (Wei Dai lorsque
Satoshi l'a contacté en août 2008, Hal Finney lors de la publication du
livre blanc, Nick Szabo courant 2009\footnote{«~Nick Szabo courant
  2009~»~: Nick Szabo, \emph{Liar-resistant government}, /05/2009 23:13
  UTC~:
  \url{https://unenumerated.blogspot.com/2009/05/liar-resistant-government.html}.}),
avaient le profil pour avoir imaginé le concept, malgré quelques
éléments contradictoires. Cependant, ils ont tous les trois démenti la
chose\footnote{«~ils ont tous les trois démenti la chose~»~: Wei Dai~:
  «~Ce que je comprends c'est que le créateur de Bitcoin, qui se fait
  appeler Satoshi Nakamoto, n'a même pas lu mon article avant de
  réinventer l'idée lui-même. Il l'a appris par la suite et m'a crédité
  dans son papier. Donc ma connexion avec le projet est assez limitée.~»
  -- Wei Dai, \emph{Re: Making money with Bitcoin?}, /02/2011 15:27
  UTC~:
  \url{https://www.lesswrong.com/posts/ijr8rsyvJci2edxot/making-money-with-bitcoin?commentId=hbEu9ue9eymNzaF2J}.}.

Le dernier projet de monnaie numérique qui a marqué l'histoire de
Bitcoin est le projet Ripple de Ryan Fugger. Même si celui-ci ne
ressemblait pas vraiment à Bitcoin, il a néanmoins eu son influence sur
le développement de ce dernier. Satoshi Nakamoto connaissait en effet
Ripple. En février 2009, sur la liste de diffusion de la Fondation P2P,
il répondait à Martien van Steenbergen qui y faisait
référence\footnote{«~il répondait à Martien van Steenbergen qui y
  faisait référence~»~: Martien van Steenbergen faisait aussi référence
  à d'autres projets issus de la communauté P2P~: Pekunio et Wizard
  Rabbit Treasurer. -- Martien van Steenbergen, \emph{{[}p2p-research{]}
  Re: Bitcoin open source implementation of P2P currency}, /02/2009
  20:01:03 UTC~:
  \url{https://diyhpl.us/~bryan/irc/bitcoin-satoshi/p2presearch-again/p2pfoundation.net/backups/p2p_research-archives/2009-February.txt.gz}.}~:

«~En ce qui concerne les systèmes de confiance, Ripple est unique en ce
qu'il répartit la confiance plutôt que de la concentrer\footnote{Satoshi
  Nakamoto, \emph{{[}p2p-research{]} Re: Bitcoin open source
  implementation of P2P currency}, /02/2009 02:31:20~:
  \url{https://diyhpl.us/~bryan/irc/bitcoin-satoshi/p2presearch-again/p2pfoundation.net/backups/p2p_research-archives/2009-February.txt.gz}.}.~»

Le deuxième lien entre Ripple et Bitcoin est l'implication du
développeur Mike Hearn. Ce dernier s'était intéressé au Ripple de Ryan
Fugger dès ses débuts et, en 2007, il avait été l'une des premières
personnes à intervenir sur le Google Group nouvellement créé\footnote{«~il
  avait été l'une des premières personnes à intervenir sur le Google
  Group nouvellement créé~»~: Mike Hearn, \emph{Hello from a Ripple
  fan}, /05/2007 12:20:53 UTC~:
  \url{https://groups.google.com/g/rippleusers/c/i8OyR5yLC0Q/m/SyztfhBGYJEJ}.}.
En découvrant Bitcoin en avril 2009, Hearn n'a ainsi pas pu s'empêcher
de demander à Satoshi Nakamoto ce qu'il pensait de Ripple, et ce dernier
lui avait alors répondu~:

«~Ripple est intéressant dans la mesure où c'est le seul autre système
qui fait quelque chose de la confiance en dehors de la concentrer au
sein d'un serveur central\footnote{Satoshi Nakamoto, \emph{Re: Questions
  about BitCoin}, /04/2009 20:44 UTC~:
  \url{https://plan99.net/~mike/satoshi-emails/thread1.html}.}.~»

Mais Ripple différait sensiblement de Bitcoin, en particulier par le
fait qu'il constituait, à proprement parler, un système de crédit
distribué et non pas une monnaie de base décentralisée. C'est ce qui a
éloigné Ryan Fugger, qui ne voyait pas «~pourquoi un bitcoin aurait une
quelconque valeur, puisqu'il n'y avait apparemment rien pour le
garantir~», mais qui s'est finalement rendu à l'évidence que le modèle
de Satoshi Nakamoto était «~une excellente idée\footnote{Ryan Fugger,
  \emph{Re: Is the cryptocurrency Bitcoin a good idea?}, /05/2011
  07:44:33 UTC~:
  \url{https://www.quora.com/Is-the-cryptocurrency-Bitcoin-a-good-idea/answer/Ryan-Fugger}.}~».

Bitcoin ajoutait donc la dernière pierre à l'édifice de l'argent liquide
électronique. Il apportait enfin une technique permettant de construire
une monnaie numérique réellement solide et durable. Le 26 janvier 2009,
Zooko Wilcox-O'Hearn témoignait de cette volonté dans un article de
blog, qui serait relayé quelques semaines plus tard sur Bitcoin.org. En
voici le texte intégral~:

«~Depuis quelque temps, je réfléchis à la manière dont des services de
jeux comme World of Warcraft et Second Life (qui prétend ne pas être un
jeu) ont réussi là où nous, à DigiCash, avons échoué~: l'argent liquide
numérique programmable, pratique et largement utilisé. Le problème est
que chacune de ces nouvelles monnaies est contrôlée de manière
centralisée par une seule entité. Cela limite le champ d'action des
personnes qui s'appuient sur cette monnaie et la valeur qu'elles sont
prêtes à risquer sur cette monnaie. Des idées circulent sur la manière
de faciliter les transactions entre les monnaies, mais cela ne
résoudrait pas le problème. Une pléthore de services centralisés
concurrents n'est pas la même chose qu'un service décentralisé. Même
s'il était bon marché et pratique d'échanger des LindenBucks contre de
l'or de WoW, cela ne ferait que nous ramener à l'équivalent des monnaies
des États-nations modernes : essentiellement centralisées (en raison de
l'effet de réseau), lourdement taxées/réglementées/manipulées, et
sujettes à des échecs désastreux. Ce que je veux, c'est une monnaie que
tout le monde peut utiliser de manière pratique et peu coûteuse, mais
que personne n'a le pouvoir de manipuler. Une monnaie pour laquelle
personne n'a le pouvoir d'en gonfler ou d'en dégonfler la quantité en
circulation, pour laquelle personne n'a le pouvoir d'en surveiller, d'en
taxer ou d'en empêcher les transactions. Un véritable équivalent
numérique de l'or, dans les périodes et les lieux où l'or était la
monnaie universelle. Voyez l'idée de BitGold de Nick Szabo et l'idée de
b-money de Wei Dai, ainsi que le récent effort pour mettre en œuvre
quelque chose de ce genre~: le BitCoin de Satoshi Nakamoto\footnote{Zooko
  Wilcox-O'Hearn, \emph{Decentralized Money}, 26 janvier 2009, archive~:
  \url{https://web.archive.org/web/20090303195936/http://testgrid.allmydata.org:3567/uri/URI:DIR2-RO:j74uhg25nwdpjpacl6rkat2yhm:kav7ijeft5h7r7rxdp5bgtlt3viv32yabqajkrdykozia5544jqa/wiki.html\#\%5B\%5BDecentralized\%20Money\%5D\%5D}.}.~»

\section*{L'aboutissement d'une
quête}\label{laboutissement-dune-quuxeate}
\addcontentsline{toc}{section}{L'aboutissement d'une quête}

\markright{L'aboutissement d'une quête}

La conception de Bitcoin a ainsi constitué la conclusion logique de la
quête de l'argent liquide numérique. D'une part, il exploitait des
techniques envisagées précédemment, comme la signature numérique,
l'horodatage et la preuve de travail. D'autre part, il s'inscrivait dans
une lignée de systèmes ingénieux qui n'avaient pas rencontré le succès
escompté à cause de leurs défauts intrinsèques, à l'instar de eCash, de
b-money, de bit gold, du système RPOW et du projet Ripple.

La particularité de Bitcoin était qu'il résolvait le problème de la
double dépense sans reposer sur un tiers de confiance, d'une manière
jamais vue auparavant. Sa robustesse et sa simplicité permettaient
d'enfin disposer d'une cybermonnaie solide et durable, qui puisse
résister aux aléas de la réalité. Bitcoin représentait le Saint Graal de
la monnaie numérique, trouvé par Satoshi Nakamoto en 2007 et offert au
monde le 31 octobre 2008.

\bookmarksetup{startatroot}

\chapter{La valeur de l'information}\label{ch:propriete}

\phantomsection\label{enotezch:7}{}

{B}\textsc{i}tcoin est un concept de monnaie numérique libre. En tant
que tel, il doit garantir la propriété des unités de compte sans
nécessiter d'identification auprès d'un tiers de confiance. Il repose
pour cela sur un algorithme de signature numérique qui permet
d'autoriser une transaction grâce à la connaissance d'une information,
la clé privée.

Pour la première fois dans l'histoire de l'humanité, Bitcoin rend ainsi
possible la possession souveraine d'un bien numérique rival,
c'est-à-dire de quelque chose qui ne puisse pas simplement être copiée.
Puisque cette possession est exercée par la connaissance exclusive des
clés privées, l'information possède plus que jamais de la valeur. Il en
découle un certain nombre de conséquences qui diffère du modèle
traditionnel de la propriété.

Dans ce chapitre technique, nous verrons comment les données sont
représentées au sein de Bitcoin, comment la cryptographie et la
signature numérique interviennent, ce qu'est le hachage. Puis, nous
décrirons comment est réalisée la génération des clés et des adresses,
ce que sont les portefeuilles et comment ils se structurent. Nous
examinerons enfin les conséquences de ce modèle, à commencer par la
responsabilité conférée au gardien des clés.

\section*{La représentation des
données}\label{la-repruxe9sentation-des-donnuxe9es}
\addcontentsline{toc}{section}{La représentation des données}

\markright{La représentation des données}

En informatique, une information est un ensemble de données stockées sur
un support matériel. Elle est communément représentée sous forme de
chiffres binaires (appelés bits par contraction de l'anglais
\emph{binary digits}), pour refléter le fonctionnement de l'électronique
numérique utilisée dans les ordinateurs. Les deux valeurs possibles (0
et 1) correspondent en effet à deux états électriques distincts, comme
par exemple la présence ou l'absence de courant.

Dans ce contexte, l'information est essentiellement un nombre. Même si
elle prend l'allure d'un contenu multimédia, une information doit être
encodée pour être traitée et interprétée par les ordinateurs.
Typiquement, l'encodage\footnote{Le mot «~codage~» est également
  largement utilisé en français.} d'un texte pourra se faire en ASCII ou
en UTF-8, celui d'une image en JPEG ou en PNG, celui d'une musique en
MP3 ou en FLAC et celui d'une vidéo en MPEG ou en H.264. De cette façon,
tout se ramène aux nombres.

Dans notre monde moderne occidental, nous avons pour habitude de
représenter les nombres au moyen d'un système de numération à 10
chiffres, fondé sur la base 10. Il s'agit d'une convention, liée au fait
que nous avons longtemps compté avec nos 10 doigts. Mais ce système
décimal n'est pas le seul qui existe, et l'informatique fait usage de
plusieurs autres bases.

Tout d'abord, comme on l'a dit, les ordinateurs sont basés sur un
système binaire, composé de deux chiffres (le 0 et le 1). Ces deux
chiffres sont donc utilisés pour écrire les nombres~: 0, 1, 10, 11,
100,~etc. Dans ce système, le nombre 21 (base 10) s'exprime comme suit~:

\[21 = 16 + 4 + 1 = \mathbf{1} \times 2^4 + \mathbf{0} \times 2^3 + \mathbf{1} \times 2^2 + \mathbf{0} \times 2^1 + \mathbf{1} \times 2^0 = \mathtt{0b10101}\]

Le préfixe \texttt{0b} est usuellement placé avant le nombre pour
indiquer que ce dernier est exprimé en binaire.

Un autre système de numération communément utilisé en informatique est
le système hexadécimal, qui est composé de 16 chiffres, symbolisés par
les 10 chiffres arabes et les 6 premières lettres de l'alphabet latin~:

\begin{verbatim}
0123456789abcdef
\end{verbatim}

Dans cette base, le «~a~» représente le nombre 10, le « b » 11,~etc.
jusqu'au «~f~» qui représente le 15.

Le système hexadécimal permet de condenser la représentation des
données. En particulier, il est très adapté pour écrire les octets
(appelés \emph{bytes} en anglais), qui sont des ensembles de 8 bits, et
qui peuvent être symbolisés par 2 caractères hexadécimaux. De cette
manière, le nombre 2008 (base 10) s'écrit~:

\[2008 = 1792 + 208 + 8 = 7 \times 16^2 + 13 \times 16^1 + 8 \times 16^0 = \mathtt{0x7d8}\]

On place usuellement le préfixe \texttt{0x} avant le nombre pour
indiquer qu'on utilise le système hexadécimal.

Dans Bitcoin, deux bases de numération supplémentaires interviennent,
notamment pour représenter certaines informations capitales, comme les
clés privées et les adresses. La première est la base 58. Dans ce
système à 58 chiffres, les nombres sont écrits en utilisant tous les
caractères alphanumériques (chiffres arabes, lettres latines minuscules,
lettres latines majuscules) à l'exception des caractères \texttt{0}
(zéro), \texttt{O} (o majuscule), \texttt{l} (L minuscule) et \texttt{I}
(i majuscule), qui peuvent être confondus entre eux et constituer une
source d'erreur. Les chiffres de cette base sont donc, dans l'ordre~:

\begin{verbatim}
123456789ABCDEFGHJKLMNPQRSTUVWXYZabcdefghijkmnopqrstuvwxyz
\end{verbatim}

Le seconde est la base 32, moins compacte mais plus adaptée pour les
codes QR. Les symboles utilisés dans cette base sont les chiffres arabes
et les lettres latines minuscules, auxquels on retranche le \texttt{1},
le \texttt{b}, le \texttt{i} et le \texttt{o} pour éviter les
confusions, à savoir les caractères suivants~:

\begin{verbatim}
qpzry9x8gf2tvdw0s3jn54khce6mua7l
\end{verbatim}

Ces systèmes d'encodage permettent de représenter l'information de
manière brute. Cependant, elle peut également être encodée dans un
format particulier incluant une somme de contrôle. Une somme de contrôle
(\emph{checksum}) est une courte séquence de données numériques calculée
à partir d'un ensemble de données plus important permettant de vérifier,
avec une très haute probabilité, que l'intégrité de cet ensemble a été
préservée lors d'une opération de copie, de stockage ou de transmission.
Celle-ci est généralement placée après l'information pour que le tout
soit ensuite représenté dans la base adéquate.

Dans le cas de Bitcoin, la somme de contrôle est essentielle pour
transmettre les informations sensibles, comme les clés privées et les
adresses, afin qu'une faute de frappe soit détectée immédiatement. Les
trois encodages qui mettent en œuvre ce type de somme de contrôle dans
BTC sont les formats Base58Check, Bech32 et Bech32m. Le premier a été
mis en place par Satoshi dès les débuts de la cryptomonnaie et consiste
à calculer une somme de contrôle grâce à l'empreinte cryptographique
tronquée de l'information. Il concerne les clés privées et les adresses
dites «~traditionnelles~», comme par exemple l'adresse ``.

Les deux autres ont vu le jour en 2017 et en 2021 (respectivement). Ils
font intervenir des sommes de contrôle par code BCH
(Bose--Chaudhuri--Hocquenghem), qui permettent non seulement de détecter
la présence d'erreurs de frappe mais aussi de les localiser\footnote{«~de
  détecter la présence d'erreurs de frappe mais aussi de les
  localiser~»~: Samuel Dobson, \emph{(Some of) the math behind Bech32
  addresses}, 2 septembre 2019~:
  \url{https://medium.com/@meshcollider/some-of-the-math-behind-bech32-addresses-cf03c7496285}.}.
Ces formats servent à encoder (respectivement) les adresses natives de
SegWit, comme par exemple
\texttt{,\ et\ les\ clés\ publiques\ utilisées\ dans\ Taproot,\ telle\ que}.
Le format Bech32 est également utilisé pour encoder les demandes de
paiement sur le réseau Lightning.

\section*{La cryptographie et
Bitcoin}\label{la-cryptographie-et-bitcoin}
\addcontentsline{toc}{section}{La cryptographie et Bitcoin}

\markright{La cryptographie et Bitcoin}

La cryptographie est la discipline mathématique qui a pour but la
sécurisation de la communication en présence de tiers malveillants. Elle
avait pour rôle initial de dissimuler de l'information grâce au
chiffrement, mais s'est par la suite étendue à l'authentification de
l'auteur d'un message (grâce à la cryptographie asymétrique) et à la
vérification de données (grâce aux fonctions à sens unique).
Aujourd'hui, la cryptographie permet donc d'assurer la confidentialité
(chiffrement), l'authenticité (signature) et l'intégrité (hachage) de
l'information transmise.

Bitcoin est un produit cryptographique. D'un point de vue technique, il
repose sur des méthodes développées dans les dernières décennies du
\textsc{xx}~siècle, comme les arbres de Merkle ou la preuve de travail.
D'un point de vue idéologique, il est issu du mouvement cypherpunk, qui
préconisait une utilisation proactive de la cryptographie pour
sauvegarder la confidentialité et la liberté des individus sur Internet.
C'est dans ce double sens qu'on le désigne comme une cryptomonnaie.

Dans le contexte de Bitcoin, le chiffrement peut être utile pour
protéger les clés privées ou pour envoyer des messages à d'autres
utilisateurs. Dans de nombreux portefeuilles, il est courant que les
clés privées soient chiffrées à l'aide d'un mot de passe (clé secrète)
pour éviter qu'une personne malveillante ayant accès à l'appareil puisse
dépenser les fonds. Dans Electrum par exemple, les clés privées sont
chiffrées par le biais de l'algorithme symétrique
AES-256-CBC\footnote{«~Dans Electrum par exemple, les clés privées sont
  chiffrées par le biais de l'algorithme symétrique AES-256-CBC~»~:
  Electrum Documentation, \emph{Frequently Asked Questions}, 3 octobre
  2021~:
  \url{https://electrum.readthedocs.io/en/latest/faq.html\#how-is-the-wallet-encrypted}.}.

Néanmoins, contrairement à ce qu'on imagine parfois, aucun chiffrement
n'est impliqué directement dans le protocole de Bitcoin~: toutes les
données sont publiques en raison du fonctionnement ouvert et sans
autorisation du système. Bitcoin n'est pas un produit cryptographique
parce que les transactions seraient chiffrées (elles ne le sont pas),
mais parce qu'il repose sur les deux autres fonctions de la
cryptographie~: l'authentification grâce à la signature numérique et la
vérification des données avec le hachage. La signature numérique permet
d'authentifier la personne à l'origine d'une transaction pour assurer au
réseau qu'il s'agit du propriétaire des bitcoins dépensés. Le hachage
intervient lui dans la dérivation des clés et des adresses, dans la
construction des blocs et dans le fonctionnement du minage.

\section*{La signature numérique}\label{la-signature-numuxe9rique}
\addcontentsline{toc}{section}{La signature numérique}

\markright{La signature numérique}

Bitcoin étant conçu pour l'échange de valeur, il repose de manière
centrale sur les transactions. Celles-ci sont dans la plupart des cas
des transferts d'unités entre deux propriétaires, même si elles peuvent
prendre des formes beaucoup plus complexes comme nous le verrons dans le
chapitre~\hyperref[ch:rouages]{12}. L'unité transférée est usuellement
le satoshi, qui forme la plus petite unité (indivisible) gérée par le
protocole et qui correspond à un cent-millionième de bitcoin~: 1 satoshi
= 0,00000001 bitcoin. Elle a été nommée comme telle en hommage au
créateur de Bitcoin, Satoshi Nakamoto\footnote{Le terme satoshi a été
  initialement proposé par ribuck sur le forum de Bitcoin, d'abord en
  novembre 2010 pour désigner 0,01 bitcoin, puis en février 2011 pour
  nommer la plus petite unité. L'appellation a ensuite été adoptée par
  la communauté. -- ribuck, \emph{Re: How did ``satoshi'' become the
  name of the base unit?}, /01/2014 20:49:00 UTC~:
  \url{https://bitcointalk.org/index.php?topic=407442.msg4415850\#msg4415850}.}.

Dans le protocole, la signature numérique est utilisée pour autoriser
ces mouvements de fonds. Comme nous l'avons décrit dans le
chapitre~\hyperref[ch:cypherpunks]{5}, ce procédé se base sur une paire
de clés~: une clé privée, secrète, qui \emph{signe} le message, et une
clé publique, connue de tous, qui permet de \emph{vérifier} la signature
produite. Dans le cas d'un transfert simple, le message à signer est la
transaction et le signataire du message est le propriétaire des satoshis
à envoyer.

L'algorithme de signature produit une signature différente pour chaque
transaction. Il ne s'agit pas de se contenter de révéler un secret pour
effectuer une dépense, auquel cas tout le monde sur le réseau pourrait
tenter de dépenser les fonds, mais bien de produire une donnée qui
puisse ensuite être vérifiée par le réseau conformément à ce qui est
attendu.

Ce fonctionnement confère un rôle fondamental à la clé privée. Tout
individu la connaissant peut accéder aux fonds qu'elle protège et s'en
emparer. C'est pourquoi elle doit rester absolument secrète~: car celui
qui la connaît devient le propriétaire \emph{de facto} des bitcoins
concernés.

L'algorithme principal utilisé dans Bitcoin est ECDSA (\emph{Elliptic
Curve Digital Signature Algorithm}), une variante de DSA utilisant la
cryptographie sur courbes elliptiques. L'algorithme fait appel à des
notions d'algèbre complexes, mais on peut tenter d'en expliquer
brièvement le fonctionnement.

La variante d'ECDSA utilisée dans Bitcoin se base sur la courbe
elliptique secp256k1\footnote{Le nom secp256k1 est un peu barbare, mais
  chaque lettre à une importance. Le sigle SEC désigne \emph{Standards
  for Efficient Cryptography}, l'ouvrage dont elle est issue
  (\url{https://www.secg.org/SEC2-Ver-1.0.pdf}). Le P-256 indique que le
  nombre premier \(p\) utilisé est encodé sur 256 bits. Le k indique
  qu'il s'agit d'une courbe de Koblitz~: les paramètres sont choisis
  pour rendre les opérations plus efficaces, et n'ont donc pas été
  sélectionnés aléatoirement (r). Le 1 désigne l'index de la courbe par
  rapport aux autres courbes similaires.}, qui sert à dériver la clé
publique de la clé privée, à signer les transactions grâce à la clé
privée et à vérifier les signatures à l'aide de la clé publique.
L'équation mathématique de cette courbe est \(y^2 = x^3 + 7\) dont les
coordonnées \(x\) et \(y\) évoluent dans le corps fini des nombres
entiers modulo \(p\), où \(p\) est un nombre premier
spécifique\footnote{Le nombre premier choisi pour secp256k1 est~:
  \(p = 2^{256} - 2^{32} - 2^9 - 2^8 - 2^7 - 2^6 - 2^4 - 1\).} inférieur
à \(2^{256}\).

\begin{figure}

{\centering \includegraphics{chapters/img/secp256k1-curve.png}

}

\caption{Représentation graphique de la courbe secp256k1 sur les nombres
réels (source~: Loïc Morel, \emph{Bitcoin démocratisé}, 2022).}

\end{figure}%

Une addition est définie sur cette courbe pour faire en sorte que la
somme de deux points soit également un point de la courbe\footnote{L'addition
  est définie par \(P + Q = R\) où
  \(x_R = \lambda_{P,Q}^2 - x_P - x_Q \pmod p\) et
  \(y_R = \lambda_{P,Q}~( x_P - x_R ) - y_P \pmod p\) avec
  \(\lambda_{P,Q} = \{ \frac{3 x_P^2}{2 y_P} \pmod p~\mathrm{si}~P = Q~;  \frac{y_Q - y_P}{x_Q - x_P} \pmod p~\mathrm{sinon} \}\).}.
La multiplication par un scalaire est définie comme étant le fait
d'additionner le même point à plusieurs reprises~:
\(m~P = P + \ldots + P~(m~\mathrm{fois})\). En fixant un point sur la
courbe, appelé point de base et noté \(G\)\footnote{Le point de base de
  secp256k1 est~: \$\$\textbackslash begin\{aligned\}}, on peut définir
une opération transformant un entier \(d\) en un point de la courbe~:
\(Q = d~G\).

Ces opérations peuvent être représentées géométriquement sur la courbe.
Par exemple, l'équivalent géométrique du doublement du point \(G\)
(addition avec lui-même) consiste à tracer la tangente du point, à
considérer l'intersection de cette tangente avec la courbe et à en
prendre l'opposé, comme représenté sur la
figure~\hyperref[fig:secp256k1-multiplication]{7.2}. Toutes ces
opérations sont non réversibles.

\begin{figure}

{\centering \includegraphics{chapters/img/secp256k1-multiplication.png}

}

\caption{Représentation géométrique du doublement du point G sur
secp256k1 (source~: Loïc Morel, \emph{Bitcoin démocratisé}, 2022).}

\end{figure}%

En choisissant une clé privée \(k\), on peut ainsi calculer la clé
publique \(K\) qui est \(K = k~G\). Puisque la multiplication par un
scalaire est non réversible, le passage de la clé privée à la clé
publique constitue une fonction à sens unique. En d'autres termes, il
est en pratique impossible de retrouver la clé publique à partir de la
clé privée sans essayer chaque possibilité une à une.

Regardons ce que cela donne en pratique. La clé privée est un nombre
choisi aléatoirement. Elle doit être comprise entre \(1\) et \(n - 1\)
où \(n\) est l'ordre du point \(G\) (qui approche \(2^{256}\))~:

\[n = \mathtt{0xfffffffffffffffffffffffffffffffebaaedce6af48a03bbfd25e8cd0364141}\]

Par exemple, le nombre héxadécimal suivant est tout à fait valide pour
servir de clé privée~:

\[k = \mathtt{0x999bb87eea489b2fc6219226e7b95d9083a3b627246ea852e85567ac4d72444f}\]

La clé publique est un point de la courbe défini par \(K = k~G\). Si
l'on calcule ce point à partir de la clé privée précédente, on obtient~:

\[\begin{aligned}
K = &~(~\mathtt{0xf6a6c7c39c88b767bfac4ac687c3ff32372e76c9fb633e2278e54472e300b3bd}, \\
    &~\mathtt{0x5822f24e0fdb4e568f97a7fff246c07ba486c1756f82971765cc9cf8e45ff5e6}~)\end{aligned}\]

Dans Bitcoin, cette clé publique est représentée de manière sérialisée.
Elle peut l'être sous forme non compressée, auquel cas elle est précédée
par le préfixe \texttt{0x04}. Dans notre cas, son expression sérialisée
est~:

\begin{verbatim}
04 f6a6c7c39c88b767bfac4ac687c3ff32372e76c9fb633e2278e54472e300b3bd
5822f24e0fdb4e568f97a7fff246c07ba486c1756f82971765cc9cf8e45ff5e6
\end{verbatim}

Il existe également une représentation compressée de la clé publique.
Celle-ci est rendu possible par la symétrie de la courbe par rapport à
l'axe des abscisses~: car en effet, le fait que le point \((x, y)\)
appartienne à la courbe implique que le point \((x, - y)\) y soit aussi.
Pour compresser l'information, il suffit ainsi de donner l'abcisse \(x\)
et un préfixe qui vaut \texttt{0x02} si \(y\) est pair ou \texttt{0x03}
si \(y\) est impair\footnote{Dans le corps fini \(\mathbb{F}_p\),
  prendre l'opposé d'un élément non nul \(y\) inverse sa polarité. En
  effet, si \(y \in [\![ 1, p - 1 ]\!]\), alors
  \(-y + p \in [\![ 1, p - 1 ]\!]\).}. On peut ensuite retrouver grâce à
l'équation de la courbe. Dans notre cas, la clé publique compressée
est~:

\begin{verbatim}
02 f6a6c7c39c88b767bfac4ac687c3ff32372e76c9fb633e2278e54472e300b3bd
\end{verbatim}

Ce format permet de réduire la taille des transactions (et donc les
frais)~: c'est pour cela qu'il est utilisé dans la plupart des
portefeuilles récents, et qu'il est imposé dans le cas de SegWit. Le
format non compressé tend ainsi à disparaître, même s'il reste toujours
valide dans les transactions classiques.

Dans Bitcoin, la clé publique servait initialement à recevoir les fonds
directement («~\emph{Pay to Public Key}~»), de sorte qu'on la confond
encore aujourd'hui avec la notion d'adresse. Toutefois, c'est son
empreinte obtenue par hachage («~\emph{Pay to Public Key Hash}~»), qui
sert généralement d'adresse de réception, comme nous le décrirons plus
bas.

L'algorithme de signature ECDSA s'applique à un message \(m\) qui est
précédemment haché et produit une signature \((r, s)\). Il est ensuite
possible de faire correspondre la signature avec la clé publique \(K\)
grâce à un algorithme de vérification qui ne nécessite pas de connaître
la clé \(k\)\footnote{L'algorithme de signature ECDSA est le suivant. En
  notant \(H(m)\) l'empreinte cryptographique du message à signer, la
  signature est obtenue en appliquant les étapes suivantes~:}.

Dans Bitcoin, le message est la transaction. L'agorithme de vérification
montre ainsi que la personne qui a produit la signature connaît \(k\)
tel que \(K = k~G\), c'est-à-dire qu'elle est propriétaire des bitcoins.
C'est ce qui permet aux nœuds du réseau de s'assurer de la validité des
signatures, et par conséquent de celle de la transaction. Un exemple de
signature correspondant à notre clé publique et à une transaction
réalisée sur le réseau principal est\footnote{Il s'agit de la signature
  de la transaction d'identifiant
  \texttt{.\ Sous\ forme\ sérialisée\ (DER),\ cette\ signature\ est}.}~:

\[\begin{aligned}
(r, s) = &~(~\mathtt{0x19b83a5e354ef62e98413e6ef3f37ad0c69f75cea7daa6a352cf66f4668a9a0b}, \\
    &~\mathtt{0x4c13f9b6f2c8ea7af224b3f6a3d9cfdfe5085bbafa150fb1aa72a20ce7cac6b0}~)\end{aligned}\]

Notez que l'algorithme ECDSA présenté ici n'est pas le seul qui existe.
En novembre 2021, BTC a intégré un autre algorithme, le schéma de
signature numérique de Schnorr, qui est basé sur la même courbe
elliptique mais qui apporte des bénéfices majeurs. Certaines autres
variantes de Bitcoin comme Monero utilisent EdDSA, un algorithme de
signature basée sur une courbe d'Edwards tordue.

\section*{Le hachage}\label{le-hachage}
\addcontentsline{toc}{section}{Le hachage}

\markright{Le hachage}

Bitcoin fait également usage du hachage. Le hachage est un procédé
cryptographique permettant de garantir l'intégrité d'une information
numérique. Le nom de ce procédé est issu d'une analogie avec la cuisine,
où des aliments peuvent être coupés en petits morceaux et regroupés dans
un hachis. Il est mis en œuvre par une fonction de hachage qui
transforme un \emph{message} de taille variable en une \emph{empreinte}
de taille fixe. Cette empreinte est aussi appelé condensat ou
\emph{hash}.

Les fonctions de hachage sont des fonctions déterministes, facilement
exécutables, qui possèdent en théorie trois
caractéristiques\footnote{Il s'agit de suppositions et certaines
  fonctions satisfont ces caractéristiques plus que d'autres. Ainsi, des
  collisions ont été trouvées au sein des fonctions MD5 et SHA-1 alors
  qu'on les croyait sûres.}~:

\begin{itemize}
\item
  Elles sont irréversibles~: ce sont des fonctions à sens unique
  construites de telle sorte qu'il est difficile de retrouver le message
  à partir d'une empreinte donnée (résistance à la préimage)~;
\item
  Elles sont imprédictibles~: toute modification du message initial
  résulte en une empreinte profondément différente, si bien qu'il est
  difficile de trouver une empreinte similaire~;
\item
  Elles sont résistantes aux collisions~: il est difficile de trouver
  deux messages dont l'empreinte résultante soit la même.
\end{itemize}

L'une des fonctions les plus connues est SHA-256, dont le nom vient de
l'abréviation de \emph{Secure Hash Algorithm} et de la taille des
empreintes qu'elle produit (256 bits, soit 32 octets). Par exemple, si
on considère le message «~Bitcoin~», le fait de l'orthographier en
minuscules ou d'ajouter un point change complètement son empreinte,
comme montré dans le tableau~\hyperref[table:sha256-hashes]{7.1}. Cette
particularité permet notamment de détecter si le message comporte une
erreur.

\phantomsection\label{table:sha256-hashes}
\begin{longtable}[]{@{}cc@{}}
\caption{Empreintes par SHA-256 de messages légèrement
différents.}\tabularnewline
\toprule\noalign{}
\textbf{Message} & \textbf{Empreinte (SHA-256)} \\
\midrule\noalign{}
\endfirsthead
\toprule\noalign{}
\textbf{Message} & \textbf{Empreinte (SHA-256)} \\
\midrule\noalign{}
\endhead
\bottomrule\noalign{}
\endlastfoot
Bitcoin & `` \\
bitcoin & `` \\
Bitcoin. & `` \\
\end{longtable}

Le hachage intervient à de multiples endroits dans Bitcoin~: dans
l'algorithme de signature (hachage du message), dans le calcul des
adresses, dans la dérivation des clés, pour le calcul des sommes de
contrôle, pour le calcul des identifiants des transactions et des blocs,
dans la construction des arbres de Merkle dans les blocs, et enfin au
cœur du minage.

Trois fonctions de hachage sont utilisées~: SHA-256, qui produit des
empreintes de 256 bits (32 octets)~; RIPEMD-160, dont le nom est le
sigle de l'anglais \emph{RACE Integrity Primitives Evaluation Message
Digest}) et qui résulte en des condensats de 160 bits~; et SHA-512, qui
hache les données en des empreintes de 512 bits.

La fonction la plus présente est le double SHA-256 (noté SHA-256d ou
HASH-256), qui intervient presque partout. Il est supposé que ce
doublement mis en place par Satoshi avait pour rôle la protection contre
les attaques par extension de longueur\footnote{«~Il est supposé que ce
  doublement mis en place par Satoshi avait pour rôle la protection
  contre les attaques par extension de longueur~»~:
  \url{https://bitcoin.stackexchange.com/questions/6037/why-are-hashes-in-the-bitcoin-protocol-typically-computed-twice-double-computed/6042\#6042}.}.
La composée de SHA-256 par RIPEMD-160 est utilisée pour le calcul des
adresses. C'est le seul endroit où RIPEMD-160 intervient de manière
substantielle\footnote{«~Les adresses Bitcoin sont le seul endroit où le
  hachage de 160 bits est utilisé.~» -- Satoshi Nakamoto, \emph{Re:
  Stealing Coins}, /07/2010 20:48:01 UTC~:
  \url{https://bitcointalk.org/index.php?topic=571.msg5754\#msg5754}.}.
Enfin, SHA-512 intervient dans l'algorithme de dérivation des clés mis
en place dans les portefeuilles.

\section*{Les clés privées}\label{les-cluxe9s-privuxe9es}
\addcontentsline{toc}{section}{Les clés privées}

\markright{Les clés privées}

Par essence, la clé privée est une information numérique, c'est-à-dire
un nombre. Plus précisément, il s'agit d'un très grand nombre compris
entre \(1\) et \(n-1\), où \(n\) est l'ordre du point \(G\) et approche
\(2^{256}\) soit \(1,1579 \times 10^{77}\). L'intervalle est
considérablement grand, si bien qu'il est statistiquement impossible de
tomber sur une même clé privée en la choisissant au hasard. À titre de
comparaison, le nombre d'atomes dans l'univers observable est proche de
\(10^{80}\).

La clé privée est créée au hasard, la plupart du temps grâce à des
algorithmes générateurs de nombres pseudo-aléatoires permettant de
reproduire le hasard de la manière la plus fidèle possible en
informatique. Cette génération repose sur l'entropie informatique issue
de l'appareil, c'est-à-dire la quantité d'aléatoire qu'il collecte par
le biais de sources matérielles (variance du bruit du ventilateur ou du
disque dur) ou de sources extérieures (mouvement de la souris, signaux
du clavier, etc.). Les outils utilisés pour générer des clés privées
sont le plus souvent considérés comme cryptographiquement fiables
(CSPRNG).

La caractère aléatoire du procédé est fondamental, constituant la base
de la sécurité du modèle. Par exemple, une personne qui choisirait le
nombre 1 comme clé privée ne pourrait jamais utiliser l'adresse
correspondante, car la sécurité liée à cette clé est nulle. Tous les
bitcoins qui seraient déposés sur cette adresse seraient instantanément
débités par un programme spécialisé\footnote{On peut observer l'adresse
  `` (correspondant à la clé 1) pour se convaincre qu'il ne s'agit pas
  d'un bon choix.}.

Il en est de même des \emph{brain wallets}, portefeuilles «~cérébraux~»
reposant sur la mémorisation d'une information, qui sont souvent créés
de manière non sécurisée. Les gens partent le plus souvent d'une phrase
cohérente (comme une citation tirée d'un livre ou d'une chanson) de
sorte à pouvoir la retenir facilement, puis la hachent et utilisent
l'empreinte résultante en tant que clé privée. Cette manière de faire
est hautement risquée en raison de la forte prévisibilité du langage
humain, et les adresses créées comme cela ont de bonnes chances d'être
vidées, comme l'a montré une enquête de BitMEX Research\footnote{BitMEX
  Research, \emph{Call me Ishmael}, 13 octobre 2020~:
  \url{https://blog.bitmex.com/call-me-ishmael/}.}.

Cette importance du hasard se retrouve également dans l'algorithme ECDSA
qui repose sur la génération d'une clé éphémère pour produire la
signature. Dans le cas où cette valeur ne serait pas correctement
générée, un attaquant pourrait déduire les clés privées à partir des
signatures. C'est notamment ce qui s'est passé en août 2013, lorsqu'une
vulnérabilité (CVE-2013-7372) a été découverte au sein de la fonction
SecureRandom de Java et a affecté la sécurité de plusieurs portefeuilles
logiciels sur Android\footnote{Bitcoin.org, \emph{Android Security
  Vulnerability}, 11 août 2013~:
  \url{https://bitcoin.org/en/alert/2013-08-11-android}.}.
L'exploitation de cette faille a mené à la perte d'au moins 55,82
bitcoins, soit 5~200~\$ à l'époque\footnote{«~la perte d'au moins 55,82
  bitcoins~»~: Burt Wagner, \emph{Bad signatures leading to 55.82152538
  BTC theft (so far)}, /08/2013, 22:53:13
  UTC~:\url{https://bitcointalk.org/index.php?topic=271486.msg2907468\#msg2907468}.}.

Après avoir été générées, les clés privées doivent ensuite être encodées
dans le but de faciliter leur transmission, pour l'import dans un
portefeuille ou pour l'export. Dans Bitcoin, elles sont ainsi
représentées grâce à l'encodage Base58Check. C'est pourquoi on parle
parfois de \emph{Wallet Import Format} (WIF).

L'encodage d'une clé suit une série d'étapes simples. Tout d'abord, la
clé est préfixée par l'octet de version \texttt{0x80} qui indique qu'il
s'agit d'une clé privée. Puis, un suffixe \texttt{0x01} est ajouté (ou
non) pour indiquer si l'on souhaite en dériver une clé publique
compressée (ou non compressée). Dans le cas de notre clé-exemple, on
obtient les octets suivants~:

\begin{verbatim}
80 999bb87eea489b2fc6219226e7b95d9083a3b627246ea852e85567ac4d72444f 01
\end{verbatim}

Ensuite, la somme de contrôle est calculée en prenant les 4 premiers
octets de l'empreinte par le double SHA-256 et ajoutée après
l'ensemble~:

\begin{verbatim}
80 999bb87eea489b2fc6219226e7b95d9083a3b627246ea852e85567ac4d72444f 01
1dd28791
\end{verbatim}

Enfin, le tout est encodé en base 58. Dans le cas «~compressé~», la clé
commence toujours par un K ou un L. Ici, notre clé privée s'écrit~:

\begin{verbatim}
L2NJfKog9SEdoAkAkm8ZNYDcpWQop95orPepbhsTE2t5Bf1yFmYk
\end{verbatim}

Dans le cas «~non compressé~» (de moins en moins utilisé), la clé
commence toujours par un 5. Ici, notre clé privée devient~:

\begin{verbatim}
5JywJHwyuD4YSsErniGJkrDNi87kggSZNADCEkhRyRScqfMMTEt
\end{verbatim}

\section*{Les adresses}\label{les-adresses}
\addcontentsline{toc}{section}{Les adresses}

\markright{Les adresses}

Dans Bitcoin, une adresse constitue en quelque sorte un numéro de compte
servant à recevoir des fonds. Cette donnée est disponible publiquement
sur la chaîne de bloc et n'importe qui peut en vérifier le solde.
Néanmoins, un utilisateur peut générer autant d'adresses qu'il le désire
afin de ne pas dévoiler l'entièreté de son activité.

De manière générale, une adresse est l'empreinte d'une clé publique
(PKH), la clé publique elle-même (PK), ou bien l'empreinte d'un script
(SH). Ici nous parlerons des adresses simples, dérivées de clés publique
par hachage, qui sont les plus utilisées sur le réseau BTC.

Une adresse simple est obtenue par les hachages successifs de la clé
publique sérialisée par les fonctions SHA-256 et RIPEMD-160. La composée
de ces deux fonctions est communément appelée HASH-160. La fonction
RIPEMD-160 a été choisie par Satoshi dans le but de diminuer la longueur
des adresses, car elle produisait des empreintes de 20 octets au lieu
des 64 octets d'une clé publique ou des 32 octets produits par SHA-256.
En notant \(A\) l'adresse, on a ainsi~:

\[A = \mathrm{HASH160}(~K~) = \mathrm{RIPEMD160}(~\mathrm{SHA256}( K )~)\]

Puisque cette composée est elle-même une fonction de hachage, elle a
pour particularité d'être de même une fonction à sens unique. Il est de
ce fait virtuellement impossible de retrouver la clé publique à partir
de l'adresse.

Le risque de collision est lui aussi statistiquement nul, même s'il y a
moins d'adresses que de clés privées. La fonction de hachage RIPEMD-160
produit en effet des empreintes de 160 bits, et il existe par conséquent
\(2^{160}\) (environ \(1,4615 \times 10^{48}\)) adresses possibles, soit
approximativement \(8 \times 10^{28}\) fois moins d'adresses que de clés
privées. Néanmoins ce nombre est suffisamment élevé pour que le risque
de tomber par hasard sur la même adresse soit complètement
négligeable\footnote{Supposons qu'une population mondiale de
  10~milliards d'êtres humains utilise Bitcoin activement de sorte que
  chaque individu génère 1~million d'adresses en moyenne. La
  probabilitié d'une collision serait alors de~:
  \[10^{16}~/~2^{160} \simeq 0.000000000000000000000000000000684~\%~.\]
  Même si un individu tentait de construire une machine spécialisée
  générant et vérifiant un trillion (\(10^{18}\)) d'adresses par seconde
  et fonctionnant en continu, la probabilité d'accéder à une adresse
  déjà utilisée serait toujours négligeable (de l'ordre de
  \(10^{-21}\)). Nos cerveaux ne sont pas faits pour nous représenter de
  tels nombres.}.

Comme une clé publique admet deux représentations sérialisées
(compressée et non compressée), il est possible de calculer deux
empreintes. Nous nous focalisons ici sur la représentation compressée.
L'empreinte de notre clé publique compressée est~:

\begin{verbatim}
a18bd7f41b42c7cc6ebfa4de43e6b63248536ebc
\end{verbatim}

On peut en dériver trois adresses de type différent~: une adresse
traditionnelle, une adresse SegWit native et une adresse SegWit
imbriquée. Dans les trois cas, le principe est le même, bien que l'usage
spécifique de l'empreinte dans le protocole diffère.

L'adresse traditionnelle est obtenue grâce à un encodage de l'empreinte
en Base58Check avec l'octet de version \texttt{0x00}. À cause de cet
octet de version, les adresses traditionnelles simples commencent
toujours par un 1 (purement symbolique car il vaut 0 en base 58). Notre
adresse est~:

\begin{verbatim}
1FjBKPQ7MTiPSDkJ2ZwPgAXUKQ8yoGbVJX
\end{verbatim}

Ce type d'adresse est appelé P2PKH (\emph{Pay to Public Key Hash}) et a
été le premier type d'adresse dans Bitcoin.

L'adresse SegWit native est encodée grâce au format Bech32. Celui-ci
inclut un préfixe indiquant le réseau (\texttt{bc} pour BTC) et un
séparateur (\texttt{1}). De manière similaire à l'encodage des adresses
traditonnelles, il s'agit de prendre l'information brute (la «~charge
utile~»), de la préfixer avec l'octet de version (\texttt{0x00} pour la
première version de SegWit), de calculer une somme de contrôle et
d'exprimer le tout dans la base appropriée, à savoir la base 32. Ce
procédé fait que l'adresse résultante commencera toujours par
\texttt{bc1q}. Dans le cas de notre empreinte de clé publique, on
obtient~:

\begin{verbatim}
bc1q5x9a0aqmgtrucm4l5n0y8e4kxfy9xm4udhygr2
\end{verbatim}

Ce type d'adresse est appelé P2WPKH (\emph{Pay to Witness Public Key
Hash}).

Enfin, on peut également inclure cette donnée sous la forme d'un script
dans une adresse P2SH, créant une adresse SegWit dite «~imbriquée~». Le
script, composé de l'octet de version de SegWit (\texttt{0x00}) et de
l'empreinte, est haché pour constituer la nouvelle adresse. Comme dans
le cas de toutes les adresses P2SH, l'empreinte résultante est encodée
en Base58Check avec l'octet de version \texttt{0x05}. Cet octet de
version a pour conséquence de faire commencer l'adresse par un 3. Notre
empreinte devient ici~:

\begin{verbatim}
3JqPHkGuvW7nsUJDgm5CPSNUb47WczCC5e
\end{verbatim}

Ce type d'adresse est appelé P2SH-P2WPKH (\emph{P2SH-nested Pay to
Witness Public Key Hash}). Nous aborderons plus en détail les différents
schémas de script qui sous-tendent ces types d'adresse dans le
chapitre~\hyperref[ch:rouages]{12}.

Une fois qu'elles ont été encodées, les adresses peuvent être partagées
facilement d'une personne à une autre. Grâce à la somme de contrôle,
faire une faute de frappe ne crée théoriquement pas de risque, car le
logiciel la détectera et refusera de procéder au paiement. Les adresses
sont aussi souvent représentées par des codes QR (voir
figure~\hyperref[fig:address-qr-codes]{7.3}), plus adaptés pour
l'interaction avec un téléphone multifonction.

\begin{figure}

{\centering \includegraphics{chapters/img/address-qr-codes.png}

}

\caption{Codes QR des adresses.}

\end{figure}%

En résumé~: lorsqu'un utilisateur veut recevoir un paiement, il génère
une clé privée, en dérive une clé publique et crée à partir de celle-ci
une adresse~; il communique son adresse à un autre utilisateur qui lui
envoie des fonds~; il peut ensuite dépenser les fonds reçus en signant
une transaction à l'aide de sa clé privée. Le réseau pair à pair de
Bitcoin vérifie alors que la signature est conforme à la clé publique.

La clé publique n'est révélée au réseau que lors de la transaction. Cela
implique que les fonds sont protégés face à l'éventualité d'une mauvaise
implémentation de l'algorithme de signature (comme dans le cas de
l'exploitation de la faille au sein de SecureRandom en 2013) ou de la
compromission généralisée d'ECDSA (par un ordinateur quantique par
exemple). Il s'agit d'un bénéfice secondaire issu de l'utilisation de
nouvelles adresses à chaque paiement.

Au-delà de BTC, les autres cryptomonnaies ont leur encodage propre pour
les adresses, qui n'est souvent qu'une variante du standard modifiant la
version ou le préfixe. Ainsi, dans Litecoin, les adresses
traditionnelles commencent par un \texttt{L} (comme par exemple
\texttt{)\ et\ les\ adresses\ SegWit\ par\ un\ \textasciigrave{}ltc1q\textasciigrave{}\ (comme\ par\ exemple}).

Bitcoin Cash possède également son propre format d'adresse, appelée
CashAddr, qui s'inspire fortement du format Bech32. Ce format a été
introduit pour différencier les adresses BTC des adresses BCH. Une
adresse BCH est simplement une représentation alternative du type
P2PKH~: dans ce format, l'adresse \texttt{devient}.

\section*{Les portefeuilles}\label{les-portefeuilles}
\addcontentsline{toc}{section}{Les portefeuilles}

\markright{Les portefeuilles}

Un portefeuille, de l'anglais \emph{wallet}, parfois aussi qualifié de
portemonnaie, est un procédé de stockage des clés privées donnant accès
aux pièces de cryptomonnaie de l'utilisateur. Ce procédé est souvent
combiné avec la gestion de la cryptomonnaie~: sa réception avec la
lecture de la chaîne de blocs et son envoi avec la production des
signatures. Le moyen utilisé peut être une simple feuille de papier ou
un fichier informatique, mais il s'agit généralement d'un logiciel sur
mobile ou ordinateur, ou bien d'un appareil spécialisé.

Un portefeuille est donc par essence un \emph{porte-clés}. Son rôle
principal est de conserver les clés privées dans le temps pour garantir
la propriété des bitcoins. La plupart du temps, les clés sont générées
par ces portefeuilles de manière déterministe à partir d'une phrase de
récupération de 12 à 24 mots. L'utilisateur doit donc conserver
précieusement cette phrase sur un autre support, dans l'éventualité de
retrouver ses fonds si son appareil est perdu, cassé ou volé.

En revanche, un compte auprès d'un dépositaire comme une plateforme de
change centralisée n'est pas un portefeuille à proprement parler, car
ces services conservent les clés privées de leurs utilisateurs à des
fins de sécurité et de facilité d'usage. Ainsi, des applications qui
ressemblent à s'y méprendre à des portefeuilles, comme le \emph{Wallet
of Satoshi} ou l'application Coinbase, n'en sont pas.

On peut classifier les portefeuilles existants en deux grandes
catégories~: les portefeuilles «~à chaud~» (\emph{hot wallets}) qui sont
connectés à Internet lors de leur utilisation, et les portefeuilles «~à
froid~» (\emph{cold wallets}) qui ne le sont jamais de manière directe.
De plus, on retrouve au sein de ces deux catégories différents types de
portefeuilles, qui possèdent chacun leurs qualités et leurs défauts.

Le stockage à chaud des clés privées, qui utilise des appareils
directement connectés à Internet, concerne notamment les portefeuilles
logiciels (\emph{software wallet}) que l'on peut installer sur un
mobile, une tablette ou un ordinateur généraliste. Ces logiciels mettent
généralement leur code source à disposition du public pour des raisons
évidentes de sécurité. Les clés sont conservées sur l'ordinateur et sont
généralement chiffrées. Cette catégorie inclut les logiciels de nœud
complet, les portefeuilles légers, les extensions de navigateur et les
portefeuilles web.

L'implémentation de nœud complet (\emph{full node implementation}),
aussi appelée client complet, est le premier type de portefeuille qui
est apparu et le seul qui existait du temps de Satoshi. Comme son nom
l'indique, un tel logiciel réalise toutes les opérations nécessaires au
maintien d'un nœud sur le réseau pair à pair~: il télécharge
l'intégralité de la chaîne de blocs et il vérifie et relaie les
transactions non confirmées et les blocs. Bitcoin Core est le logiciel
de nœud complet le plus connu. Cependant, en raison de la difficulté
d'utilisation, ce type de portefeuille n'est généralement plus utilisé
directement, les néophytes préférant utiliser des applications plus
légères et les utilisateurs confirmés privilégiant des solutions plus
sécurisées, qu'ils peuvent ensuite connecter à leur nœud personnel s'ils
le souhaitent.

Le portefeuille léger (\emph{lightweight wallet}), aussi appelé
portefeuille SPV (pour \emph{Simplified Payment Verification}), est un
logiciel qui ne télécharge pas la chaîne de blocs mais qui procède à une
vérification simplifiée des transactions à partir de la chaîne des
entêtes qui ne nécessite que peu de ressources informatiques. Ce type de
portefeuille est particulièrement adapté aux petits appareils comme les
téléphones. Le logiciel peut interagir avec l'ensemble des nœuds
complets du réseau pair-à-pair, comme le fait BRD (anciennement appelé
\emph{breadwallet}), mais il passe de manière générale par
l'intermédiaire d'une infrastructure de serveurs dédiés qui rendent
l'utilisation plus agréable, comme c'est le cas d'Electrum ou de
Samourai. Ce type de portefeuille garantit la sûreté des fonds, mais
peut avoir un effet dommageable à d'autres niveaux, notamment en ce qui
concerne la confidentialité. L'utilisateur peut également choisir de
connecter son portefeuille à son propre nœud complet.

Un portefeuille peut aussi prendre la forme d'une extension de
navigateur web, que ce soit sur Chrome, Firefox ou Brave. Contrairement
aux clients légers, ces portefeuilles ne procèdent pas toujours à la
vérification des transactions et font confiance au serveur auquel elles
sont connectées.

Enfin, le dernier type de stockage à chaud est le portefeuille web. Ces
derniers sont des interfaces en ligne permettant de gérer des fonds.
Contrairement aux plateformes de change, l'utilisateur garde le contrôle
de ses clés privées lorsqu'il passe par ce genre de service~: celles-ci
sont gérées par le navigateur et ne sont jamais révélées à autrui. Le
portefeuille de ce type le plus connu est celui de Blockchain.com.

Mais ces solutions à chaud ne sont pas les seules, et il existe des
méthodes de conservation à froid des clés privées, qui sont coupées de
tout accès direct à Internet. Cette conservation a le mérite de réduire
la surface d'attaque et donc le risque de vol par piratage informatique.
Il s'agit de la solution recommandée pour mettre en sécurité des grosses
sommes de cryptomonnaie.

Dans l'absolu, il faut disposer d'un appareil qui reste constamment
hors-ligne pour générer les clés et les adresses. Cet appareil peut être
un vieil ordinateur non connecté à Internet ou bien un appareil
spécialisé. Les deux méthodes principales pour réaliser du stockage à
froid sont le portefeuille papier et le portefeuille matériel.

Le portefeuille papier (\emph{paper wallet}) est le type de portefeuille
le plus simple qu'on puisse imaginer~: les clés privées générées
hors-ligne (et les adresses qui leur correspondent) sont écrites sur une
feuille de papier. L'information écrite peut également être une phrase
mnémotechnique. Le portefeuille papier présente néanmoins un
inconvénient majeur~: l'impossibilité de signer des transactions sans
l'importer dans une interface connectée à Internet. Cette méthode n'est
pas du tout pratique, car l'utilisateur ne peut pas signer de
transaction sans compromettre la sécurité de son portefeuille et doit se
contenter de recevoir des paiements. Pour résoudre ce problème, il
existe ce qu'on appelle les portefeuilles matériels.

Le portefeuille matériel (\emph{hardware wallet}) est un appareil dont
la spécificité est de générer et de conserver les clés privées de
manière isolée et de permettre de signer des transactions hors-ligne. Il
s'agit aujourd'hui de la solution la plus sûre de détenir du bitcoin.
Ces portefeuilles sont construits de telle manière que quelqu'un qui
s'en emparerait ne pourrait pas dépenser les fonds sans le mot de passe
de l'utilisateur.

Il existe une diversité de portefeuilles matériels. Les plus connus sont
les portefeuilles de Satoshi Labs (le Trezor One et le Trezor model T)
et ceux de Ledger (le Nano S et le Nano X), qui sont les modèles les
plus anciens et les plus reconnus. Ceux-ci peuvent être connectés à
l'ordinateur de manière sûre et les transactions sont toujours signées
sur l'appareil. Certains autres perfectionnent la sécurité en étant
physiquement isolés de tout ordinateur tiers (grâce à un air gap) comme
la Cold Card Mk4. D'autres portefeuilles mettent l'accent sur la
facilité d'utilisation comme les cartes Satochip qui se basent sur des
smartcards.

Tous les portefeuilles impliquent une certaine confiance~: vous devez
vous fier au logiciel que vous utilisez pour conserver vos bitcoins, au
programme dont vous vous servez pour générer un portefeuille papier, au
matériel spécialisé dans le stockage à froid. Bien entendu, les
solutions ouvertes peuvent être considérées comme plus sûres dans le
sens où d'autres personnes que les concepteurs ont pu vérifier le
produit final~: c'est notamment le cas de nombreux portefeuilles
logiciels et de l'infrastructure matérielle des portefeuilles
Trezor\footnote{«~l'infrastructure matérielle des portefeuilles
  Trezor~»~: \emph{Hardware design of
  Trezor}~:\url{https://github.com/trezor/trezor-hardware}.}. Dans tous
les cas, une composante basée sur la réputation subsiste.

De manière générale, chaque type de portefeuille possède une utilité~:
c'est donc à l'utilisateur de déterminer quel portefeuille conviendra
mieux à ses besoins.

\section*{La dérivation des clés}\label{la-duxe9rivation-des-cluxe9s}
\addcontentsline{toc}{section}{La dérivation des clés}

\markright{La dérivation des clés}

Durant les débuts de Bitcoin, les clés privées étaient générées
aléatoirement par le logiciel à chaque utilisation. Il s'ensuivait que
les clés étaient conservées dans un fichier, appelé \texttt{wallet.dat},
stocké sur le disque dur de l'ordinateur. Cela rendait la perte des clés
plus probable.

Néanmoins, les portefeuilles modernes ne fonctionnent plus comme cela.
Les clés et les adresses sont dérivées de manière déterministe à partir
d'une seule information générée aléatoirement, qui se présente sous la
forme d'une phrase mnémotechnique allant de 12 à 24 mots. Ces mots
peuvent être des mots en anglais, en français ou dans une autre langue.

elder process crowd gentle proof taxi bean patient around warm source
boil

De ce fait, c'est la conservation de cette phrase, appelée phrase de
récupération, qui garantit la sécurité des bitcoins. Cette phrase vous
permet de retrouver vos fonds si votre appareil est volé ou cassé. C'est
pourquoi elle doit rester secrète.

Ce type de portefeuille est parfois appelé HD wallet pour
\emph{Hierarchical Deterministic Wallet}~: portefeuille déterministe
hiérarchique. Le concept a été développé pour Bitcoin à partir de
2011\footnote{«~Le concept a été développé pour Bitcoin à partir de
  2011~»~: Gregory Maxwell, \emph{Deterministic wallets}, /06/2011
  21:27:29~:
  \url{https://bitcointalk.org/index.php?topic=19137.msg239768\#msg239768}.}.
Il a été standardisé en 2012 au sein du BIP-32 écrit par Pieter Wuille,
et des propositions BIP-39 et BIP-44 écrites par Marek Palatinus et
Pavol Rusnak. Il a été élargi aux autres cryptomonnaies en
2014\footnote{«~Il a été élargi aux autres cryptomonnaies en 2014~»~:
  \emph{SLIP-0044 : Registered coin types for BIP-0044}~:
  \url{https://github.com/satoshilabs/slips/blob/master/slip-0044.md}.}.

En règle générale, la phrase secrète, ou phrase de récupération, est
générée par l'appareil de l'utilisateur, qu'il s'agisse d'un téléphone
mobile, d'un ordinateur ou d'un portefeuille matériel. Pour ce faire,
une entropie est d'abord créée par l'appareil de manière
pseudo-aléatoire. L'information, qui possède un nombre de bits précis,
est ensuite enrichie d'une somme de contrôle de quelques bits,
permettant de détecter les erreurs de saisie, et l'ensemble est divisé
en segments de 11 bits. Enfin, chacun de ces segments est associé à un
mot dans la liste standarde de 2048 mots, ce qui permet de former la
phrase. Cette dérivation est représentée dans la
figure~\hyperref[fig:from-entropy-to-mnemonic]{7.4}.

Le nombre de mots de la phrase dépend de la taille de l'entropie
désirée. Ainsi une entropie de 128 bits est dotée d'une somme de
contrôle de 4 bits, ce qui donne une phrase de 12 mots de 11 bits. Pour
256 bits, on a une somme de contrôle de 8 bits et donc une phrase de 24
mots.

\begin{figure}

{\centering \includegraphics{chapters/img/from-entropy-to-mnemonic.png}

}

\caption{De l'entropie à la phrase secrète.}

\end{figure}%

Divers procédés cryptographiques sont utilisés pour dériver les clés et
les adresses à partir de cette phrase. Ces procédés de dérivation ont
sensiblement les mêmes propriétés que les fonctions de hachage en
produisant des résultats irréversibles, imprédictibles et résistants aux
collisions.

Le premier est le code d'authentification de message HMAC-SHA512 (HMAC
pour \emph{Hash-Based Message Authentication Code}) qui calcule une
empreinte en utilisant la fonction de hachage SHA-512 en combinaison
avec une clé secrète. Le second est la fonction de dérivation de clé
PBKDF2 (\emph{Password-Based Key Derivation Function 2}) qui applique de
manière répétée une fonction choisie par l'utilisateur à un message de
taille arbitraire avec un sel cryptographique. L'intérêt est de
nécessiter une quantité de calcul importante pour éviter un cassage par
force brute de l'information supérieure.

Dans Bitcoin, PBKDF2 est utilisée pour dériver une graine à partir de la
phrase mnénmotechnique, en appliquant la fonction HMAC-SHA512 à 2048
reprises. Le sel cryptographique est le terme \texttt{mnemonic} auquel
on peut ajouter une phrase de passe (\emph{passphrase}) pour renforcer
la sécurité du procédé. La graine résultante est une information de 512
bits (64 octets), à partir de laquelle la clé maîtresse et les clés
suivantes sont dérivées.

La dérivation des clés se fait grâce à la fonction HMAC-SHA512. Tout
d'abord, on procède à une première dérivation à partir de la graîne. On
applique le HMAC à la graine et au sel cryptographique
\texttt{Bitcoin\ seed}, ce qui nous donne une clé maîtresse (premiers
256 bits du résultat) et un code de chaîne maître (derniers 256 bits du
résultat). Le passage de la phrase secrète à la clé maîtresse et au code
de chaîne maître est résumé dans la
figure~\hyperref[fig:from-mnemonic-to-root]{7.5}.

\begin{figure}

{\centering \includegraphics{chapters/img/from-mnemonic-to-root.png}

}

\caption{De la phrase secrète à la clé maîtresse.}

\end{figure}%

Ces deux informations permettent de réaliser toutes les dérivations
suivantes. Le code de chaîne intervient dans la chaîne de dérivation des
clés, de sorte qu'il est impossible de procéder à la dérivation sans
lui.

Plutôt que de gérer ces deux informations indépendamment, on préfère
faire appel aux clés privées étendues (\emph{extended private keys}),
qui incluent la clé privée et le code de chaîne, ainsi que d'autres
informations comme la profondeur et l'indice de la clé enfant. La clé
privée étendue est encodée en Base58Check avec un préfixe spécial qui
dépend du type d'adresse dérivé, faisant que le résultat commence par
\texttt{xprv} (adresses traditionnelles et clés Taproot), par
\texttt{yprv} (adresses SegWit imbriquées) ou par \texttt{zprv}
(adresses SegWit natives). Dans notre cas, la clé privée étendue issue
de la clé privée maîtresse et du code de chaîne maître est~:

\begin{verbatim}
xprv9s21ZrQH143K3KSN1mSK8myNuDcXNvNoCDcU4KBxMTuj1Wo83zNn
jaj8dKFT81GttcgPftdB4XhAzzQLXJEGDtFp35yssYnxDV3yVDEqv1b
\end{verbatim}

De même, la clé publique étendue (\emph{extended private key}) regroupe
la clé publique et le code de chaîne correspondant à la clé privée dont
elle dérive. En Base58Check, cette clé commence toujours par
\texttt{xpub}, \texttt{ypub} ou \texttt{zpub}. La clé publique étendue
correspondant à la clé privée maîtresse est~:

\begin{verbatim}
xpub661MyMwAqRbcFoWq7nyKVuv7TFT1nP6eZSY4rhbZuoShtK8GbXh3
HP3cUapsPsqEd52TRk1vhkgkhtAReezgSBi4ELh3YoxjmZgKBk7U98h
\end{verbatim}

La dérivation des clés (\emph{child key derivation}) consiste à utiliser
l'algorithme HMAC-SHA512 pour dériver des clés étendues «~enfant~» à
partir d'une clé étendue «~parent~». Les codes de chaîne sont utilisés
comme sel cryptographique. Deux types de dérivation existent~: la
dérivation normale et la dérivation endurcie.

La dérivation normale fait intervenir la clé publique étendue dans le
processus, ce qui rend possibles deux opérations~: l'obtention de la clé
publique (étendue) enfant à partir de la clé publique (étendue) parent,
et l'obtention de la clé privée (étendue) enfant à partir de la clé
privée (étendue) parent. Le fonctionnement de ce type de dérivation est
schématisé par la
figure~\hyperref[fig:normal-child-key-derivation]{7.6}.

\begin{figure}

{\centering \includegraphics{chapters/img/normal-child-key-derivation.png}

}

\caption{Dérivation normale des clés par HMAC-SHA512.}

\end{figure}%

Cette particularité de la dérivation se révèle extrêmement utile pour
générer de nouvelles adresses sans compromettre la clé privée racine. Un
utilisateur peut ainsi importer la clé publique étendue dans un
processeur de paiement afin de vérifier son solde et générer de
nouvelles adresses sans avoir à fournir la clé privée. Cela permet aussi
aux commerçants d'avoir des employés qui reçoivent des paiements à
différentes adresses sans se soucier de la sécurité des fonds.

Cependant, cette particularité comporte un risque potentiel~: si une clé
privée enfant est divulguée, alors la connaissance de la clé publique
étendue parente (et donc du code de chaîne correspondant) permet
d'obtenir toutes les clés privées enfant ainsi que la clé privée parent.

C'est pour cela qu'il existe un deuxième type de dérivation, la
dérivation endurcie (\emph{hardened derivation}), qui, contrairement à
la première, est restreinte au calcul de clés privées (étendues) enfant,
ce qui assure une meilleure sécurité. Celle-ci est représentée dans la
figure~\hyperref[fig:hardened-child-key-derivation]{7.7}.

\begin{figure}

{\centering \includegraphics{chapters/img/hardened-child-key-derivation.png}

}

\caption{Dérivation endurcie des clés par HMAC-SHA512.}

\end{figure}%

Chaque dérivation fait intervenir un indice, encodé sur 32 bits comme un
entier signé, dont le bit de signe indique si elle doit être endurcie ou
non et dont la valeur indique le numéro de la clé enfant. Ainsi, on peut
produire 2~147~483~648 (de \(0\) à \(2^{31} - 1\)) clés enfants normales
et 2~147~483~648 clés enfants endurcies (de \(-0\) à \(- 2^{31} + 1\)) à
partir d'une même clé parent.

L'usage veut qu'on utilise une apostrophe pour désigner ce
signe\footnote{On utilise aussi parfois la lettre \texttt{h} (pour
  \emph{hardened}).}. L'indice \texttt{2} indique qu'il s'agit de la
troisième clé enfant normale. L'indice \texttt{44’} indique qu'il s'agit
de la 45 clé enfant endurcie.

Les dérivations successives permettent de créer des arbres de
dérivation, dont la position de chaque clé peut être retrouvée grâce à
un chemin, le chemin de dérivation. Ce dernier est composé des indices
successifs des clés, qui sont séparés par des barres obliques
(\texttt{/}). On le fait généralement commencer par la lettre \texttt{m}
pour indiquer qu'on part de la clé privée maîtresse. Un exemple de
chemin de dérivation est ``.

Chaque portefeuille peut utiliser son propre chemin de dérivation.
Néanmoins, un standard a émergé, le BIP-44. Celui-ci simplifie la
construction de portefeuilles à usages multiples, supportant plusieurs
cryptomonnaies et donnant la possibilité de créer plusieurs comptes pour
chacune d'entre elles\footnote{Tous les portefeuilles ne respectent
  néanmoins pas ce standard. Le BRD wallet (ex Bread Wallet) utilise
  ainsi le chemin \texttt{m/0’} pour dériver le compte principal,
  conformément aux recommandations initiales du BIP-32.}.

Dans ce standard, on procède à trois dérivations endurcies puis à deux
dérivations normales pour arriver à une clé privée et à l'adresse
correspondante. Chaque dérivation apporte une information~:

\begin{itemize}
\item
  La première dérivation (endurcie) sert à définir le but du
  portefeuille~: le 44 (qui fait référence au BIP-44) permet de dériver
  un compte utilisant des adresses traditionnelles, le 49 (BIP-49) pour
  les adresses SegWit imbriquées, le 84 (BIP-84) pour les adresses
  SegWit natives, le 48 ou le 45 (BIP-45) pour les adresses
  multisignatures, le 86 (BIP-86) pour dériver les clés publiques liées
  à Taproot,~etc.~;
\item
  La deuxième dérivation (endurcie) indique le protocole
  cryptoéconomique et a fortiori les jetons liés~: le chiffre 0 est
  utilisé pour BTC, le 1 pour le testnet, le 2 pour LTC, le 60 pour ETH,
  le 128 pour XMR, le 145 pour BCH,~etc.
\item
  La troisième dérivation (endurcie) donne l'indice du compte~: 0, 1,
  2,~etc.~;
\item
  La quatrième dérivation (normale) indique le rôle des adresses~: le 0
  signale qu'il s'agit d'une adresse externe, dont le rôle est de
  réceptionner des bitcoins, le 1 d'une adresse interne, utilisée pour
  accueillir la sortie complémentaire lors d'un envoi de bitcoins
  (phénomène que nous décrirons dans le
  chapitre~\hyperref[ch:rouages]{12})~;
\item
  La cinquième dérivation (normale) donne l'indice de la clé et de
  l'adresse considérée~: 0, 1, 2,~etc.
\end{itemize}

De ce fait, le chemin de dérivation ressemble à ceci~:

\begin{verbatim}
m / but' / protocole' / compte' / rôle_adresse / indice_adresse
\end{verbatim}

Par exemple, la clé
\texttt{correspond\ à\ la\ première\ adresse\ de\ réception\ d\textquotesingle{}un\ compte\ Bitcoin\ utilisant\ les\ adresses\ traditionnelles.\ De\ même,\ la\ clé}
correspond à la 18 adresse de reste du premier compte Bitcoin utilisant
les adresses SegWit natives.

Toutes les adresses d'un portefeuille restent valides même si elles ont
été utilisées. Même si l'on peut générer des adresses à l'infini, le
portefeuille balaie usuellement 20 adresses à partir de la dernière
adresse active.

\section*{La propriété dans
Bitcoin}\label{la-propriuxe9tuxe9-dans-bitcoin}
\addcontentsline{toc}{section}{La propriété dans Bitcoin}

\markright{La propriété dans Bitcoin}

La propriété est le contrôle absolu exercé sur un bien par une personne
à l'exclusion de toutes les autres. Bien souvent, la propriété s'exerce
par l'intermédiaire d'un droit de propriété qui établit \emph{de jure}
le rapport de force. Le bien possédé peut être un livre, une voiture ou
un terrain.

La propriété est à la base de la monnaie~: sans maîtrise réelle sur les
unités monétaires, l'échange est impossible. En effet, la cession de
pièces de métal précieux ou de billets fiduciaires requiert que le
porteur les contrôle entièrement et puisse les abandonner au moment de
la transaction. C'est pourquoi on parle aussi d'argent \emph{liquide}.

Sans cette propriété, les caractéristiques de la monnaie s'effritent.
Aujourd'hui, l'essentiel des transactions a lieu par l'échange du crédit
bancaire, que ce soit par le biais d'un paiement par carte bancaire,
d'un virement ou d'un autre moyen numérique. Cette situation fait que
les gens s'exposent de plus en plus aux formes de censure issues des
contraintes réglementaires et de l'arbitraire bancaire, comme
l'interdiction d'envoyer un virement ou le gel de compte sans préavis,
outre le risque de solvabilité de la banque.

Bitcoin permet de redevenir pleinement propriétaire de son argent tout
en conservant le côté numérique et immatériel de son usage. Cette
propriété est de nature différente de celle exercée sur les objets~:
elle est en effet indissociable de la connaissance exclusive d'une
information (les clés privées) et de la protection de cette information.

Ainsi, l'information possède, plus que jamais, de la valeur. On a
toujours associé une valeur au savoir en raison du pouvoir que ce
dernier apporte (\emph{scientia potentia est}), mais cette valeur était
indirecte. Aujourd'hui, une information peut procurer un accès direct à
un certain montant de cryptomonnaie~: si quelqu'un connaît la clé privée
qui correspond à une adresse contenant des bitcoins, il possède \emph{de
facto} ces bitcoins.

Un utilisateur peut conserver du bitcoin extrêmement facilement en
gardant en mémoire la clé privée ou la phrase de récupération. Il peut
par exemple franchir une frontière étatique en ayant en sa possession un
papier sur lequel se trouve l'information en question, ou bien tout
simplement en la gardant en tête. C'est par exemple le cas d'un criminel
allemand qui, après avoir miné frauduleusement 1~700 bitcoins en
installant un logiciel sur des ordinateurs à l'insu de leurs
propriétaires, a pu conserver sa fortune malgré son emprisonnement de
deux ans\footnote{Clément Wardzala, «~\emph{Bitcoin~: la police
  allemande à la recherche d'un mot de passe à 65 millions de
  dollars}~», \emph{Cryptoast}, 5 février 2021~:
  \url{https://cryptoast.fr/bitcoin-police-allemande-recherche-mot-de-passe-65m/}.}.

Un utilisateur peut recevoir des bitcoins en générant une nouvelle clé
privée sur un appareil. Il ne nécessite aucune autorisation du réseau,
même s'il doit bien entendu avoir accès à Internet pour vérifier les
paiements entrants. En raison de la résistance du système à la censure,
il peut faire ce qu'il veut de ses bitcoins~: financer des causes
sensibles, acheter de la drogue sur le dark web, jouer au casino en
ligne, envoyer des fonds à l'étranger,~etc. Il n'y a pas de limite de
montant, ce qui confère à un individu fortuné un moyen d'avoir un impact
autrement plus grand sur le monde.

\section*{Le risque de garde}\label{le-risque-de-garde}
\addcontentsline{toc}{section}{Le risque de garde}

\markright{Le risque de garde}

Même si Bitcoin permet l'échange libre au travers du cyberespace, il n'a
pas fait disparaître les tiers de confiance pour autant. En effet,
beaucoup de gens sont peu confiants dans leur capacité à conserver
eux-mêmes leurs bitcoins, et préfèrent déléguer cette responsabilité à
des services dépositaires, comme les services de garde spécialisés, les
places de marché en ligne ou les applications de paiement. Il est aussi
plus pratique de passer par une banque pour prêter son argent et le
faire fructifier, ce qui profite aux plateformes de prêt en ligne.

Bien que ce comportement se comprenne, il faut insister sur le fait que
ceux qui épargnent des bitcoins par l'intermédiaire d'un dépositaire ne
possèdent pas réellement ces bitcoins. La créance qu'ils possèdent sur
le tiers de confiance n'est pas la propriété des bitcoins, puisque c'est
le tiers en question qui les garde en son pouvoir. La loi étatique peut
intervenir, mais cela n'empêche pas ce contrôle réel de s'exprimer dans
une multitude de cas. C'est le sens de l'adage «~pas vos clés, pas vos
bitcoins~» («~\emph{not your keys, not your coins}~»), popularisé par
Andreas Antonopoulos\footnote{Andreas Antonopoulos, \emph{Bitcoin Q\&A:
  How Do I Secure My Bitcoin?} (vidéo), 7 juillet 2017~:
  \url{https://www.youtube.com/watch?v=vt-zXEsJ61U}.}, qui rappelle que
celui qui ne gère pas lui-même ses clés privées, ne possède pas
réellement les bitcoins qu'il estime détenir.

Si la délégation de la propriété apporte certains avantages, elle a
aussi ses inconvénients et fait courir des risques à ceux qui y ont
recours. Tout d'abord, les dépositaires peuvent faire faillite dans le
cas où leurs réserves deviennent trop basses pour les demandes de
retrait. En cas de faillite, le client ne retrouve pas l'intégralité de
ses fonds, à moins qu'une autre entité rachète les pertes de la
plateforme.

Premièrement, cette faillite peut se matérialiser suite à une perte de
fonds, comme ce qui est arrivé en juillet 2011 à la plateforme de change
polonaise Bitomat qui avait perdu les clés privées liées à 17 000 BTC
suite à un incident technique.

Deuxièmement, elle peut provenir d'un vol externe à la plateforme, issu
par exemple d'un piratage, dont l'exemple le plus connu est le cas de la
plateforme Mt. Gox qui a connu de multiples piratages entre 2011 et 2013
ayant mené à la volatilisation de 650~000 bitcoins, et qui a fait
faillite en 2014\footnote{«~la plateforme Mt. Gox qui a connu de
  multiples piratages entre 2011 et 2013 ayant mené à la volatilisation
  de 650~000 bitcoins, et qui a fait faillite en 2014~»~: Ludovic Lars,
  \emph{Mt. Gox et ses 842 109 bitcoins disparus, la lente descente aux
  enfers d'un géant du bitcoin}, 24 décembre
  2020~:\url{https://journalducoin.com/analyses/mt-gox-lente-descente-enfers/}.}.
La dette (en dollars) des créanciers de la plateforme devrait être
remboursée en 2024, dix ans après les faits.

Troisièmement, cette faillite peut résulter d'une escroquerie de sortie
ou d'un vol interne, où le gestionnaire de la plateforme «~s'enfuit avec
la caisse~». Ce type d'incident a été illustré en juillet 2011 par la
fermeture du service MyBitcoin après le vol de 78~740 BTC par son
fondateur anonyme Tom Williams. Un autre cas est celui de la plateforme
canadienne QuadrigaCX, qui a fait faillite en 2019 suite à la mort de
son fondateur et PDG, Gerald Cotten, qui s'avérait avoir dépensé les
fonds pour financer son train de vie et son addiction à la spéculation.
La faillite de la plateforme d'échange populaire FTX qui est survenue en
novembre 2022 suite à l'utilisation frauduleuse des fonds de ses clients
constitue un autre exemple explosif de ce type d'évènement.

Quatrièmement, même si aucune perte ou aucun vol de fonds ne survient,
un fonctionnement par réserves fractionnaires du dépositaire peut le
pousser à faire faillite à cause d'un resserrement du crédit. C'est
notamment arrivé aux plateformes de prêt Celsius, Three Arrows Capital,
Voyager Digital, BlockFi et Genesis Files en 2022--2023.

Ensuite, outre le risque de faillite, l'utilisation d'un dépositaire
comporte le risque d'intervention étatique. La plateforme, pourvu
qu'elle agisse sur le marché légal, se soumet aux diverses
réglementations de LCB-FT et peut donc être amenée à geler un compte,
voire à saisir les fonds qui s'y trouvent. C'est ce qu'a fait la place
de marché Coinbase le 7 mars 2022 en bloquant 25~000 adresses dans le
contexte des sanctions occidentales contre la Russie\footnote{Paul
  Grewal, \emph{Using Crypto Tech to Promote Sanctions Compliance}, 7
  mars 2022~:
  \url{https://blog.coinbase.com/using-crypto-tech-to-promote-sanctions-compliance-8a17b1dabd68}.}.
La plateforme peut également être fermée par les pouvoirs publics, comme
cela a été le cas de BTC-e en juillet 2017 qui a été saisie par le
département de la Justice des États-Unis\footnote{Department of Justice,
  \emph{Russian National And Bitcoin Exchange Charged In 21-Count
  Indictment For Operating Alleged International Money Laundering Scheme
  And Allegedly Laundering Funds From Hack Of Mt. Gox}, 26 juillet
  2017~:\url{https://www.justice.gov/usao-ndca/pr/russian-national-and-bitcoin-exchange-charged-21-count-indictment-operating-alleged}.}.

Enfin, un autre inconvénient lié à l'utilisation d'un dépositaire est le
cas des scissions, qui sont des duplications permanentes de la chaîne de
blocs créant deux monnaies distinctes, et des \emph{airdrops}
(«~largages~»), qui sont des distributions gratuites de jetons à des
fins publicitaires. Dans les deux cas, l'adresse de l'utilisateur est
créditée d'un actif supplémentaire qui devient sa propriété. Cependant,
si la personne passe par l'intermédiaire d'un dépositaire, ce dernier
peut choisir de ne pas le lui céder, généralement d'une manière non
frauduleuse, selon les critères déterminés par les conditions
d'utilisation lors de l'inscription. En ce qui concerne les scissions,
on peut citer l'exemple de la plateforme Bitstamp qui a refusé de céder
les bitcoins SV de ses utilisateurs après la séparation entre BCH et BSV
en novembre 2018 et qui continue de les conserver\footnote{Patrick
  Thompson, «~\emph{Crypto exchanges delisting, denying access and
  stealing BSV}~», \emph{CoinGeek} 17 janvier 2020~:
  \url{https://coingeek.com/crypto-exchanges-delisting-denying-access-and-stealing-bsv/}.}.
Pour les airdrops, on peut évoquer le cas de HEX, pyramide de Ponzi
ouverte, dont la genèse en 2020 a été déterminée en partie par la
possession de bitcoins~: chaque détenteur de bitcoin pouvait prétendre à
un montant de jetons HEX proportionnel en publiant une signature
numérique sur la chaîne d'Ethereum, mais il semble qu'aucune plateforme
n'a pris le risque de tirer profit de ce largage.

La non-distribution des fruits des scissions et des \emph{airdrops}
représente ainsi un manque à gagner, voire une perte sèche pour le
client, surtout s'il s'agit d'une scission entre deux économies de
taille équivalente. Toutefois, rien ne peut forcer en soi un dépositaire
à proposer le retrait de ces gains, car la mise en œuvre technique a un
coût non négligeable. Dans le cas contraire, les plateformes seraient
contraintes de soutenir toutes les créations de ce type, y compris les
plus fantaisistes, commes les scissions opportunistes de BTC qui ont eu
lieu en 2017--2018 (Bitcoin Gold, Bitcoin Diamond, Bitcoin Private,
etc.)

D'une manière générale, le recours à un dépositaire comporte des
inconvénients majeurs qui font qu'un utilisateur ne bénéficie pas de la
résistance à la censure et de la résistance à l'inflation de Bitcoin.
Conserver du bitcoin sur des plateformes réglementées permet à
l'utilisateur de profiter vaguement de l'indulgence temporaire de l'État
vis-à-vis des transferts et des plus-values réalisés. En outre, la
généralisation de la garde de fonds présente un risque systémique comme
nous le verrons. C'est pourquoi le recours aux dépositaires doit être
considérée comme une exception, et non la règle, en ce qui concerne la
conservation des bitcoins.

\section*{Propriété et
responsabilité}\label{propriuxe9tuxe9-et-responsabilituxe9}
\addcontentsline{toc}{section}{Propriété et responsabilité}

\markright{Propriété et responsabilité}

Si Bitcoin permet à l'utilisateur de posséder son argent de manière
souveraine, cette propriété s'accompagne d'une responsabilité. Cet
utilisateur doit comprendre comment le système fonctionne, au moins de
manière rudimentaire. Il doit choisir quels logiciels et quel matériel
utiliser. Il doit manipuler les fonds, vérifier les adresses, rester
vigilant à tout instant. Dans le cas d'une scission sans protection
contre la rediffusion des transactions, il doit procéder lui-même à la
séparation des pièces d'un côté et de l'autre. Il est seul face à
l'incertitude, et surtout, face à lui-même. Cette responsabilité
constitue le prix à payer pour la liberté monétaire.

Il est donc compréhensible que certaines personnes manquant de
connaissances techniques finissent par déléguer cette gestion, notamment
dans le but de spéculer. Cependant, l'intérêt primordial de Bitcoin
n'est pas de revenir à un système bancaire~: c'est de posséder
pleinement ses fonds, sans que ceux-ci puissent être gelés par un tiers
de confiance ou dilués par l'inflation monétaire.

Puisque la sécurisation des bitcoins repose sur la connaissance d'une
information, la conservation des bitcoins est inextricablement liée au
dilemme qui existe entre la perte de données et la fuite de données.
Pour conserver ses bitcoins, il faut à la fois garder l'accès à ses clés
privées (éviter la perte de données) et en exclure les autres personnes
(éviter la fuite de données), ce qui ne peut jamais être réalisé
totalement.

Ce dilemme n'est résoluble que par un compromis entre la sécurité contre
la perte et la sécurité contre le vol, qui est propre à chaque personne.
Ainsi, quelqu'un peut simplement mémoriser sa phrase de 12 ou 24 mots
pour conserver ses bitcoins, au risque de l'oublier et de les perdre
définitivement. À l'inverse, une autre personne peut conserver des
sauvegardes multiples à différents endroits au risque de voir un tiers
accéder à l'une d'entre elles et s'emparer de ses fonds.

D'un côté, nous avons le vol de bitcoins. Celui-ci peut se faire par un
cambriolage~: une personne s'introduit chez autrui et s'empare du
support physique sur lequel se trouve la sauvegarde ou le mot de passe.
Mais il peut également être réalisé par intimidation~: les propriétaires
sont attaqués physiquement pour être extorqués\footnote{«~les
  propriétaires sont attaqués physiquement pour être extorqués~»~:
  Jameson Lopp maintient un registre (non exhaustif) des attaques
  physiques connues contre les propriétaires de bitcoins, où ceux-ci
  subissent des menaces de violences voire de torture afin de transférer
  des fonds~: \emph{Known Physical Bitcoin Attacks},
  \url{https://github.com/jlopp/physical-bitcoin-attacks/blob/master/README.md}.}.
La famille de Hal Finney a ainsi été ciblée par un maître-chanteur, qui
lui a fait subir un swatting en réussissant à convaincre les unités
spéciales de police d'intervenir en urgence au domicile
familial\footnote{«~La famille de Hal Finney a ainsi été ciblée par un
  maître-chanteur~»~: Robert McMillan, \emph{An Extortionist Has Been
  Making Life Hell for Bitcoin's Earliest Adopters}, 29 décembre 2014~:
  \url{https://www.wired.com/2014/12/finney-swat/}.}.

Il existe des bonnes pratiques pour ne pas s'exposer à ce type de
vol.~Tout d'abord, il est primordial de préserver sa confidentialité en
évitant de déclarer qu'on possède des cryptomonnaies, combien on en
possède, depuis combien de temps,~etc. Ce conseil s'applique également
vis-à-vis des plateformes de change, qui connaissent l'identité de leurs
clients et leurs adresses de retrait, et qui peuvent dévoiler ces
informations suite à une requête étatique ou à une fuite.

Puis, l'utilisateur peut améliorer sa conservation. Il peut éviter de
conserver ses sauvegardes dans les lieux les plus sensibles (comme son
domicile). Il peut également répartir les fonds dans des portefeuilles
gérés différemment afin d'atténuer l'impact d'un vol, bien que cela
augmente également le risque de survenue de ce vol.

Il est ensuite possible de mettre en place un compte secondaire caché au
sein d'un portefeuille matériel en exploitant l'utilisation de la phrase
de passe. C'est une fonctionnalité que Ledger intègre dans ses
produits\footnote{«~C'est une fonctionnalité que Ledger intègre dans ses
  produits~»~: Ledger Documentation, \emph{Comment configurer une
  passphrase~?}~:
  \url{https://support.ledger.com/hc/fr-fr/articles/115005214529-Comment-configurer-une-passphrase-?}.}.
Cette technique a le mérite de créer un «~déni plausible~» à présenter à
l'assaillant qui menace ou qui torture le détenteur.

On peut enfin rendre la propriété des bitcoins collective, soit de
manière explicite par la mise en place d'un compte multisignatures où
chaque participant dipose de ses propres clés privées, soit de manière
implicite par l'algorithme de partage de clés secrètes de Shamir
(\emph{Shamir's Secret Sharing}). Cela permet d'impliquer d'autres
personnes pour rendre l'extorsion plus difficile.

De l'autre côté, nous avons la perte de bitcoins, qui représente le
risque opposé de la conservation. La perte n'est pas en soi un problème
pour le système. En effet, elle ne fait que renforcer le côté
déflationniste du bitcoin~: comme le disait Satoshi Nakamoto, la perte
ne fait qu'«~augmenter légèrement la valeur des pièces des autres~» et
peut être considérée «~comme un don à tous\footnote{Satoshi Nakamoto,
  \emph{Re: Dying bitcoins}, /06/2010, 17:48:26 UTC~:
  \url{https://bitcointalk.org/index.php?topic=198.msg1647\#msg1647}.}~».
Toutefois, il s'agit assurément d'un problème au niveau individuel, et
la perte des clés a été pendant longtemps le principal risque pour
l'utilisateur.

Certains des premiers mineurs ont ainsi perdu les bitcoins qu'ils
avaient extrait. C'est le cas de James Howells, un ingénieur britannique
qui a miné 8~000 bitcoins pendant un peu plus de 2 mois en 2009 et qui a
perdu la clé permettant d'y accéder\footnote{James Howells a miné entre
  le 15 février (bloc \(4~334\)) et le 24 avril 2009 (bloc \(12~098\)).
  Il a accumulé son revenu de minage à l'adresse ``. En date du 26 avril
  2009, cette adresse contenait exactement 8~000 bitcoins.}. Au cours de
l'été 2013, il a en effet jeté son ordinateur contenant le fichier du
portefeuille, en le déposant à la décharge publique près de chez lui. Il
a réalisé son erreur quelques mois plus tard avec la hausse du cours et
la médiatisation associée, mais il était trop tard. Son cas a été rendu
public en novembre 2013 dans un article du \emph{Guardian}\footnote{Alex
  Hern, «~\emph{Missing: hard drive containing Bitcoins worth £4m in
  Newport landfill site}~», \emph{The Guardian}, 27 novembre 2013~:
  \url{https://www.theguardian.com/technology/2013/nov/27/hard-drive-bitcoin-landfill-site}.}.

Un autre exemple (médiatisé en 2021\footnote{Nathaniel Popper,
  «~\emph{Lost Passwords Lock Millionaires Out of Their Bitcoin
  Fortunes}~», \emph{The New York Times}, 12 janvier 2021~:
  \url{https://www.nytimes.com/2021/01/12/technology/bitcoin-passwords-wallets-fortunes.html}.})
est celui de Stefan Thomas, le programmeur allemand qui a été payé en
bitcoins pour produire la première vidéo qualitative sur Bitcoin. Après
avoir payé les frais pour cette vidéo, il a conservé le reste sur son
portefeuille\footnote{Les adresses de Stefan Thomas sont \texttt{et}. En
  date du 8 juin 2011, leur solde combiné était de 7~003,21 bitcoins.}.
Il a procédé à une sauvegarde sur une clé USB chiffrée (IronKey) mais a
fini par oublier son mot de passe de chiffrement.

Les pertes sont donc courantes et il est nécessaire de se prémunir
contre ce risque. L'adoption des portefeuilles déterministes
hiérarchiques (\emph{HD wallets}), où les clés sont dérivées d'une seule
phrase secrète, a grandement aidé à raffermir la sécurité contre la
perte. Avant, on devait conserver un fichier contenant ses clés privées
sur un appareil~; aujourd'hui la simple conservation de cette phrase
suffit, ce qui facilite la copie sur un support physique.

La première mesure à mettre en place pour éviter la perte est la mise en
place de sauvegardes multiples. L'utilisateur peut placer la phrase à
différents endroits géographiques, si bien qu'il conserve la propriété
de ses bitcoins en cas de sinistre de l'un de ces endroits (incendie,
inondation, cyclone, etc.) Il peut utiliser une feuille en papier simple
ou cartonnée, ou bien il peut également faire le choix de graver ses
mots sur une plaque d'acier forgée à cet effet\footnote{«~graver ses
  mots sur une plaque d'acier forgée à cet effet~»~: Jameson Lopp,
  \emph{Metal Bitcoin Seed Storage Reviews}~:
  \url{https://jlopp.github.io/metal-bitcoin-storage-reviews/}.}.

L'utilisateur peut même, pour ses portefeuilles les moins fournis,
conserver une sauvegarde numérique sur son ordinateur (si possible en la
chiffrant) ou sur le \emph{cloud}, ce qui augmente sensiblement le
risque de vol mais permet d'être sûr de pouvoir accéder aux bitcoins.
Cet usage est généralement déconseillé, mais c'est à l'individu
d'arbitrer la situation.

L'aspect programmable de Bitcoin peut également être mis à profit contre
la perte. On peut ainsi mettre en place des systèmes de récupération de
fonds, comme ce qui est fait par exemple dans le portefeuille
Liana\footnote{Jean-Luc (Bitcoin.fr), \emph{Sortie de la version 1.0 de
  Liana}, 12 mai 2023~:
  \url{https://bitcoin.fr/sortie-de-la-version-1-0-de-liana/}.}. Aucun
standard de contrat de ce type ne s'est pour l'instant imposé, si bien
que cette pratique reste déconseillée pour le novice.

Il peut être profitable pour l'utilisateur de tenir un ou plusieurs
registres listant ses différents portefeuilles, même les plus anciens,
afin de ne pas oublier où sont ses fonds. Cependant, encore une fois, il
ne faut pas que ce registre soit trouvé, auquel cas les fonds pourraient
être retrouvés plus facilement.

De même, on ne doit jamais supprimer la sauvegarde d'un portefeuille,
même si ce dernier paraît vide. Celui-ci pourrait en effet contenir des
cryptomonnaies issues de scissions ou pourrait recevoir des paiements à
l'avenir (par exemple s'il inclut une adresse de donation publique). Il
est en ce sens recommandé «~de le mettre de côté et de conserver
l'ancienne copie au cas où\footnote{Satoshi Nakamoto, \emph{Re: Version
  0.3.13, please upgrade}, /10/2010 20:54:07 UTC~:
  \url{https://bitcointalk.org/index.php?topic=1327.msg15136\#msg15136}.}~».

Enfin, l'utilisateur doit se souvenir qu'il va mourir. À moins qu'il ne
veuille emporter ses possessions numériques dans sa tombe, il lui faut
mettre en place un plan de succession pour ses bitcoins à destination de
ses héritiers. Il existe de multiples manières de faire, mais le modèle
le plus réputé est celui présenté par Pamela Morgan dans son
\emph{Cryptoasset Inheritance Planning}\footnote{Pamela Morgan,
  \emph{Cryptoasset Inheritance Planning}, Merkle Bloom LLC, 2018.}.
Celui-ci consiste à écrire une lettre dans laquelle l'utilisateur inclut
les coordonnées de gens de confiance à contacter pour aider ses
héritiers (nos proches ne sont \emph{a priori} pas autant à l'aise que
nous avec la manipulation de bitcoins) ainsi que l'inventaire de ses
avoirs (dans le but de récupérer les sauvegardes et de restaurer les
portefeuilles). La lettre est scellée et placée dans un lieu sûr, comme
un coffre-fort personnel, un coffre en banque ou chez un notaire.

\section*{Bitcoin et l'information}\label{bitcoin-et-linformation}
\addcontentsline{toc}{section}{Bitcoin et l'information}

\markright{Bitcoin et l'information}

Bitcoin permet donc pour la première fois dans l'histoire d'être
propriétaire d'un bien numérique rival. Cette propriété s'exerce par la
connaissance exclusive d'informations, les clé privées, qui sont
générées et gérées par des outils appelés les portefeuilles. Grâce au
procédé de signature numérique, ce sont en effet ces clés privées qui
permettent de signer les transactions dépensant les bitcoins.

Couplée à la résistance à la censure, cette assurance de la propriété
permet de réaliser des transactions librement sur Internet, sans
craindre le gel de compte. Mais elle s'accompagne également d'une
responsabilité qui impose à l'utilisateur de prendre un certain de
nombres de mesures pour ne pas voir ses fonds disparaître.

Ainsi, le système de signatures numériques «~fournit un contrôle fort de
la propriété~». Cependant, il «~reste incomplet sans moyen d'empêcher la
double dépense\footnote{Satoshi Nakamoto, \emph{Bitcoin: A Peer-to-Peer
  Electronic Cash System}, 31 octobre 2008.}~». C'est la résolution de
ce problème qui constitue l'objet du prochain chapitre.

\bookmarksetup{startatroot}

\chapter{Le consensus par le minage}\label{ch:confirmation}

\phantomsection\label{enotezch:8}{}

{B}\textsc{i}tcoin est un modèle décentralisé de monnaie numérique issu
de l'informatique distribuée, une discipline développée au moment de
l'émergence d'Internet. Il se fonde plus précisément sur un réseau pair
à pair d'ordinateurs, dans lequel les participants possèdent tous les
mêmes responsabilités. En tant que tel, il constitue un \emph{système
d'argent liquide électronique pair à pair}.

Tout l'enjeu de Bitcoin est ainsi de se mettre d'accord sur le contenu
d'un registre déterminant qui possède quoi, c'est-à-dire d'arriver à un
consensus sur la propriété des unités. En particulier, l'établissement
d'un tel accord permet de résoudre le problème de la double dépense, qui
se pose dans le monde numérique en raison de la facilité de reproduction
des données.

Le consensus -- accord unanime au sein d'un groupe de personnes -- n'est
pas une chose facile à atteindre entre les êtres humains. La
conciliation sociale peut fonctionner concernant des règles générales,
mais n'est pas adaptée quant aux détails particuliers. C'est pourquoi
les organisations humaines sont bien souvent obligées de s'en remettre à
une autorité centrale chargée de prendre les décisions.

Bitcoin a précisément pour contrainte d'éviter le recours à un tiers de
confiance. Il utilise à cette fin un mécanisme de consensus distribué et
ouvert, qui repose sur une activité appelée communément le minage, où la
confirmation des transactions, c'est-à-dire leur inclusion dans le
registre, est assurée par un procédé nommé la preuve de travail. Dans ce
chapitre, nous détaillerons le fonctionnement de cet algorithme de
consensus novateur.

\section*{Le problème des généraux
byzantins}\label{le-probluxe8me-des-guxe9nuxe9raux-byzantins}
\addcontentsline{toc}{section}{Le problème des généraux byzantins}

\markright{Le problème des généraux byzantins}

L'enjeu du consensus est illustré par le problème des généraux
byzantins, qui est un problème d'informatique distribuée formalisé en
1982 par Leslie Lamport, Robert Shostak et Marshall Pease\footnote{Leslie
  Lamport, Robert Shostak, Marshall Pease, «~\emph{The Byzantine
  Generals Problem}~», in \emph{ACM Trans. Program. Lang. Syst.},
  vol.~4, no.~3, 1982, pp.~382---401~:
  \url{https://lamport.azurewebsites.net/pubs/byz.pdf}.}. Ce problème
traite de la remise en cause de la fiabilité des transmissions et de
l'intégrité des participants dans les systèmes distribués, et il
s'applique dans les cas où les composants d'un système informatique ont
besoin d'être en accord.

Le problème est énoncé sous la forme d'une métaphore faisant intervenir
des généraux de l'armée de l'Empire byzantin, l'Empire romain d'Orient
qui a subsisté jusqu'en 1453 suite à la chute de la partie occidentale
en 476\footnote{Selon Leslie Lamport, l'appellation byzantine a été
  choisie pour ne pas offenser le sentiment patriotique du lecteur
  (l'armée dans la métaphore comporte des traîtres), car cette
  appellation a été faite \emph{a posteriori} par les historiens et les
  Byzantins eux-mêmes se considéraient comme romains. -- Voir Leslie
  Lamport, \emph{My Writings}~:
  \url{http://lamport.azurewebsites.net/pubs/pubs.html\#byz}.}. Ces
généraux assiègent une ville ennemie avec leurs troupes dans le but de
l'attaquer. Ils ne peuvent communiquer qu'à l'aide de messages relayés
oralement et ils doivent trouver un moyen d'établir un plan de bataille
commun par ce moyen. Par exemple, les généraux peuvent chercher à
coordonner une attaque à l'aube, et partagent leurs intentions entre eux
en envoyant le message «~attaque~» par le biais d'un message pour
confirmer l'assaut, et «~retraite~» pour l'annuler.

Cependant, un petit nombre de ces généraux s'avèrent être des traîtres
au service de l'ennemi qui essaient de semer la confusion au sein de
l'armée. Ces traîtres envoient ainsi des messages contradictoires à
leurs interlocuteurs, pour faire en sorte que certains généraux loyaux
attaquent, et que d'autres battent en retraite au moment de l'assaut,
causant par là une défaite certaine, comme illustré sur la
figure~\hyperref[fig:byzantine-generals-attack]{8.1}.

\begin{figure}

{\centering \includegraphics{chapters/img/byzantine-generals-attack.png}

}

\caption{Attaque des généraux byzantins contre la ville~: succès et
échec.}

\end{figure}%

Le problème est de trouver une stratégie (c'est-à-dire un algorithme)
permettant de s'assurer que tous les généraux loyaux se mettront
d'accord sur le plan de bataille. Les traîtres battront alors en
retraite, mais puisque leur nombre est supposément restreint, l'attaque
sera quand même un succès.

La situation fait qu'il est difficile de parvenir à un consensus. On ne
peut pas désigner un commandant auxquels les généraux subordonnés
obéiront, car le commandant peut être lui-même un traître. Lamport,
Shostak et Pease ont montré que le problème peut être résolu de manière
absolue si (et seulement si) les généraux loyaux représentent
strictement plus des deux tiers de l'ensemble des généraux\footnote{Cette
  propriété est démontrée dans l'article original de Lamport et al.~La
  condition plus précise est \(n \ge 3 m + 1\) où \(n\) est le nombre
  total de généraux et \(m\) le nombre de traîtres.}~; autrement dit,
qu'il ne peut pas y avoir plus d'un tiers de traîtres au sein de
l'armée.

La métaphore des généraux byzantins s'applique directement aux systèmes
distribués, c'est-à-dire aux systèmes dont les composants sont séparés
et doivent communiquer les uns avec les autres pour se synchroniser. Les
généraux représentent les composants du système, les traîtres les
composants défaillants, et les messages les données transmises entre les
composants. Le but est d'obtenir un algorithme permettant de détecter
les défaillances, appelées pannes byzantines, et de permettre aux autres
composants de les écarter. La résilience obtenue est appelée la
tolérance aux pannes byzantines~; le système est dit BFT, pour
\emph{Byzantine Fault Tolerant}.

Le problème a été initialement décrit pour les systèmes informatiques
reposant sur des composants présents à différents endroits et dans
lesquels la bonne transmission des données est critique, comme les
systèmes aéronautiques ou aérospatiaux\footnote{L'infrastructure du
  Boeing 777 repose notamment sur le bus informatique ARINC 629 qui
  réplique en quadruple les messages envoyés afin de garantir un
  résultat avec une latence très faible. -- Elaine Ou, \emph{Byzantine
  Fault Tolerant Airplanes}, 12 février 2017~:
  \url{https://elaineou.com/2017/02/12/byzantine-fault-tolerant-airplanes/}.}.
Mais il concerne aussi (ce qui nous intéresse ici) les systèmes pair à
pair reposant sur un réseau horizontal de participants, et en
particulier les systèmes cryptoéconomiques comme Bitcoin, dans lesquels
les nœuds du réseau ont besoin de se mettre d'accord sur le contenu d'un
registre. L'objectif est alors de trouver un algorithme permettant à
tous les nœuds honnêtes de parvenir à un consensus en présence de nœuds
traîtres (dits «~byzantins~»).

Avant Bitcoin, le problème était résolu par des algorithmes dits
«~classiques~» basés sur les idées de Lamport, Shostak et Pease. Le plus
connu est probablement l'algorithme de consensus PBFT (pour
\emph{Practical Byzantine Fault Tolerance}), mis au point par Miguel
Castro et Barbara Liskov en 1999\footnote{«~l'algorithme de consensus
  PBFT {[}...{]} mis au point par Miguel Castro et Barbara Liskov en
  1999~»~: Miguel Castro, Barbara Liskov, \emph{Practical Byzantine
  Fault Tolerance}, février 1999.}, qui permettait à un nombre donné de
participants de se mettre d'accord en gérant des milliers de requêtes
par seconde avec une latence de moins d'une milliseconde.

Bien avant Bitcoin, Wei Dai et Nick Szabo envisageaient d'utiliser ce
type d'algorithme pour leurs systèmes de monnaie électronique, b-money
et bit gold. De même, de nombreux systèmes cryptoéconomiques en font
encore aujourd'hui usage pour des raisons de performance, à l'instar
d'Ethereum dont le consensus est basé sur l'algorithme Casper FFG.

Cependant, ces algorithmes impliquent des contraintes fortes~: les nœuds
doivent connaître l'ensemble des autres nœuds et ils doivent communiquer
avec tous les autres. De ce fait, il faut sélectionner les nœuds ayant
le droit de participer au consensus avant de lancer l'algorithme, ce qui
se fait généralement par preuve d'autorité (\emph{proof of authority}),
via une liste blanche de nœuds, ou par preuve d'enjeu (\emph{proof of
stake}), via un montant de jetons possédés ou délégués. Cela implique
une moins bonne robustesse du système, car les validateurs sont alors
connus de tous et donc davantage exposés aux attaques.

Bitcoin résout ce problème d'une manière différente, grâce à un nouveau
type d'algorithme~: l'algorithme de consensus de Nakamoto par preuve de
travail. Celui-ci est plus robuste dans le sens où les nœuds du réseau
n'ont pas besoin de connaître l'ensemble des autres nœuds et où aucune
identification n'est requise.

Puisque le rôle principal de Bitcoin est le transfert de valeur,
l'objectif est de se mettre d'accord sur qui possède quoi, c'est-à-dire
sur l'\emph{état} du système. La solution proposée par Satoshi Nakamoto
consiste à employer un registre recensant l'intégralité des transactions
réalisées depuis le lancement du système, «~la seule façon de confirmer
l'absence d'une transaction {[}étant{]} d'être au courant de toutes les
transactions\footnote{Satoshi Nakamoto, \emph{Bitcoin: A Peer-to-Peer
  Electronic Cash System}, 31 octobre 2008.}~». Ce registre formant
l'\emph{historique} du système est organisé comme une succession de
blocs de transactions, de sorte qu'il est communément appelé la
\emph{chaîne de blocs}. Les nœuds du réseau entretiennent chacun une
copie complète de la chaîne dont ils se transmettent des éléments sur
demande.

Les nouveaux blocs sont ajoutés à la chaîne de manière régulière grâce à
la production d'une preuve de travail. Les acteurs réalisant cette
opération sont appelés des mineurs. Les nœuds du réseau arrivent à un
consensus en considérant que la chaîne la plus longue est la chaîne
correcte. Ainsi, comme l'a écrit Satoshi Nakamoto~:

«~La chaîne de preuves de travail est une solution au problème des
généraux byzantins\footnote{Satoshi Nakamoto, \emph{Re: Bitcoin P2P
  e-cash paper}, /11/2008, 22:56:55 UTC~:
  \url{https://www.metzdowd.com/pipermail/cryptography/2008-November/014849.html}.}.~»

La spécificité novatrice de cet algorithme est qu'il résout le problème
de manière probabiliste plutôt que de manière absolue\footnote{Plus
  précisément, il s'agit de sacrifier un peu de la propriété de sécurité
  au sens de Lamport pour améliorer la tolérance aux pannes byzantines.}.
Par conséquent, les transactions incluses dans le registre ne sont
jamais strictement finales, mais sont (probabilistiquement parlant)
considérées comme telles au bout d'un temps. Ce fonctionnement permet de
n'avoir besoin que de 51~\% de validateurs honnêtes, au lieu des 67~\%
requis par les algorithmes classiques.

\section*{La preuve de travail}\label{la-preuve-de-travail}
\addcontentsline{toc}{section}{La preuve de travail}

\markright{La preuve de travail}

La preuve de travail, de l'anglais \emph{proof of work}, est un procédé
permettant à un appareil informatique de démontrer de manière objective
et quantifiable qu'il a dépensé de l'énergie. Ce moyen est utilisé pour
sélectionner les ordinateurs dans le cadre de l'accès à un service ou à
un privilège.

La preuve de travail est un mécanisme de résistance aux attaques Sybil,
qui rend difficile la multiplication des identités à l'excès par un
acteur qui chercherait à prendre le contrôle du réseau. Une attaque
Sybil\footnote{«~attaque Sybil~»~: Voir John R. Douceur, «~\emph{The
  Sybil Attack}~», in \emph{Peer-to-Peer Systems}, 2002, pp.~251--260.
  La pratique a été décrite en 1993 par le cypherpunk L. Detweiler sous
  le nom de \emph{pseudospoofing}~:
  \url{https://cypherpunks.venona.com/date/1993/10/msg00760.html}.} est
une attaque intervenant au sein d'un réseau ouvert basé sur un système
de réputation qui consiste à dupliquer les profils à moindre coût pour
en altérer le fonctionnement. C'est par exemple un problème
particulièrement présent sur les médias sociaux, où les comptes de
robots sont utilisés en masse pour augmenter la visibilité d'un contenu
donné.

Le concept de preuve de travail a été décrit pour la première fois par
Cynthia Dwork et Moni Naor en 1992, dans un article visant à présenter
une méthode permettant de combattre le courrier indésirable
(\emph{spam}) dans les boîtes de réception\footnote{Cynthia Dwork, Moni
  Naor, \emph{Pricing via Processing or Combatting Junk Mail}, 1992.}.
Le terme «~\emph{proof of work}~» est quant à lui apparu en 1999 sous la
plume de Markus Jakobsson et Ari Juels\footnote{Markus Jakobsson, Ari
  Juels, \emph{Proofs of Work and Bread Pudding Protocols (Extended
  Abstract)}, 1999.}.

L'idée de Dwork et Naor a été implémentée par le cypherpunk britannique
Adam Back en 1997 au moyen de Hashcash, un algorithme produisant de
manière simple des preuves de travail avec une fonction de hachage, qui
devait principalement servir pour le courrier électronique\footnote{Adam
  Back, \emph{{[}ANNOUNCE{]} hash cash postage implementation}, /03/1997
  16:52:26 UTC~:
  \url{https://cypherpunks.venona.com/date/1997/03/msg00774.html}~; Adam
  Back, \emph{Hashcash -- A Denial of Service Counter-Measure}, 1 août
  2002~: \url{http://www.hashcash.org/hashcash.pdf}.}. Cette
implémentation a été reprise dans le système de preuves de travail
réutilisables (RPOW) de Hal Finney mis en application en 2004.

L'algorithme de preuve de travail de Hashcash consiste à trouver une
collision partielle de la fonction de hachage considérée, c'est-à-dire à
obtenir deux messages ayant une empreinte commençant par les mêmes bits
de données. À partir de la version 1.0 sortie en 2002, il s'agit plus
précisément de découvrir une collision partielle pour l'empreinte zéro,
à savoir trouver un antécédant dont l'empreinte commence par un nombre
de zéros binaires déterminés. Puisque la fonction de hachage est à sens
unique (résistance à la préimage), une telle obtention ne peut être
réalisée qu'en testant une à une les différentes possibilités, ce qui
demande de l'énergie. L'antécédant obtenu est appelé une preuve de
travail.

La preuve de travail est réalisée par le calcul successif d'empreintes
d'une chaîne de caractères, composée d'une information de base, et d'un
nombre qu'on fait varier, appelé le compteur ou le nonce. L'information
de base comporte généralement des indications sur le contexte dans
lequel la preuve de travail a été produite (identifiant, date, heure,
protocole,~etc.) pour démontrer que cette preuve de travail n'a pas déjà
été utilisée.

Prenons un exemple pour illustrer le propos. D'abord, on choisit une
information de base propre au contexte~: pour produire une preuve de
travail liée à cet ouvrage et à sa date d'écriture, on peut opter pour
l'information de base ``. Puis on détermine le degré de la preuve de
travail, c'est-à-dire le nombre de zéros binaires par lequel doit
commencer l'empreinte, ici 16. On procède ensuite à la recherche du
résultat voulu en incrémentant le nonce~: à chaque itération, on le met
bout à bout avec l'information de base et on vérifie si l'empreinte de
l'ensemble est satisfaisante. Le travail s'arrête enfin lorsque
l'empreinte commence avec un nombre suffisant de zéros~: ici 95~690
tentatives. Notre preuve de travail est donc~:

\begin{verbatim}
20231031181000:BitcoinElegance:95690
\end{verbatim}

Et l'empreinte correspondante, commençant par 4 zéros hexadécimaux (soit
16 zéros binaires), est~:

\begin{verbatim}
0000387b99b1412e3cb6e49548cc0d11bdc797138e1a0f5ff095279a710b895a
\end{verbatim}

Les étapes de cette procédure sont décrites dans le
tableau~\hyperref[table:hashcash-hashes]{8.1}.

\phantomsection\label{table:hashcash-hashes}
\begin{longtable}[]{@{}cc@{}}
\caption{Recherche de la preuve de travail à partir de l'information de
base \texttt{20231031181000:BitcoinElegance:}.}\tabularnewline
\toprule\noalign{}
\textbf{Nonce} & \textbf{Empreinte (SHA-256)} \\
\midrule\noalign{}
\endfirsthead
\toprule\noalign{}
\textbf{Nonce} & \textbf{Empreinte (SHA-256)} \\
\midrule\noalign{}
\endhead
\bottomrule\noalign{}
\endlastfoot
0 & `` \\
1 & `` \\
2 & `` \\
3 & `` \\
95~690 & `` \\
\end{longtable}

Statistiquement, ce type de recherche implique d'essayer 65~536
possibilités (\(2^{16}\)) pour tomber sur une solution. En moyenne, la
production d'une telle preuve de travail démontre donc qu'un effort
approchant a été effectué. De plus, il existe une asymétrie entre la
production et la vérification, cette dernière ne nécessitant qu'une
seule application de la fonction de hachage et étant par conséquent peu
coûteuse.

Le coût de production moyen confère une certaine rareté aux preuves de
travail~: plus leur degré est élevé, plus elles sont difficiles à
produire. D'où le fait qu'on puisse les utiliser en tant que marques de
qualité pour le courrier électronique comme dans Hashcash, ou bien en
tant que pièces monétaires de base comme dans bit gold et RPOW.

Le minage de Bitcoin intègre le procédé de preuve de travail de Hashcash
sous la forme d'une variante~: l'objectif est de trouver une empreinte
inférieure à une valeur cible précise, et non pas une empreinte
commençant par un nombre de zéros déterminés. Ce procédé est appliqué
entre les blocs de transactions, de sorte que ces blocs, ou plutôt leurs
entêtes comme nous l'expliquerons plus bas, constituent eux-mêmes les
preuves de travail.

Dans Bitcoin, le rôle de la preuve de travail est double~: exiger un
coût pour la fabrication des nouveaux bitcoins et faire en sorte que le
réseau puisse arriver à un consensus. D'une part, elle a pour but
d'imposer la cherté de l'unité de compte. Cela rappelle les modèles qui
ont précédé Bitcoin, et c'est pourquoi Hal Finney a été jusqu'à
qualifier les bitcoins de «~jetons de preuve de travail\footnote{Hal
  Finney, \emph{Bitcoin v0.1 released}, /01/2009 16:48:03 UTC~:
  \url{https://www.metzdowd.com/pipermail/cryptography/2009-January/015036.html}.}~»
(\emph{POW tokens}) en 2009. Toutefois, les bitcoins ne sont pas
exactement des preuves de travail dans le sens où la difficulté de
production est variable, évoluant selon la puissance de calcul totale
déployée sur le réseau. Ainsi, mis à part dans le cas limite de la
difficulté minimale du système, le but est de s'assurer que la
production des unités demande de l'énergie, pas d'exiger un coût en
travail fixe. D'autre part, la preuve de travail a pour objectif de
garantir le consensus sur le réseau, en faisant en sorte que les nœuds
honnêtes se mettent d'accord sur qui possède quoi. Elle limite l'accès à
la production des blocs~: la sélection du validateur (mineur) se fait
selon le montant d'énergie dépensé. La preuve de travail joue ici son
rôle de défense contre les attaques Sybil en empêchant les attaquants de
mettre en place un grand nombre de nœuds pour contrôler le
système\footnote{«~Si la majorité était basée sur le principe de vote
  par adresse IP (une adresse IP, une voix), elle pourrait être
  détournée par toute personne capable de s'octroyer de nombreuses
  adresses IP. La preuve de travail est essentiellement basée sur la
  puissance de calcul~: un processeur, une voix.~» -- Satoshi Nakamoto,
  \emph{Bitcoin: A Peer-to-Peer Electronic Cash System}, 31 octobre
  2008.}.

Ce fonctionnement fait que la chaîne de blocs forme une chaîne de
preuves de travail, qui récapitule l'ensemble du travail effectué depuis
le début. De ce fait, la chaîne constitue un historique linéaire
difficilement malléable comme nous le verrons.

\section*{La chaîne de blocs}\label{la-chauxeene-de-blocs}
\addcontentsline{toc}{section}{La chaîne de blocs}

\markright{La chaîne de blocs}

La chaîne de blocs, ou \emph{blockchain} en anglais, est la structure de
données regroupant l'ensemble des transactions réalisées depuis le
lancement du système. Cette structure est une suite de blocs de
transactions, liés les uns aux autres par un procédé appelé
l'horodatage.

L'horodatage est une technique permettant d'associer une date et une
heure à une information, qui a été décrite en 1991 par Stuart Haber et
Scott Stornetta dans le cas particulier de l'horodatage de
documents\footnote{«~décrite en 1991 par Stuart Haber et Scott Stornetta
  dans le cas particulier de l'horodatage de documents~»~: Stuart Haber,
  Wakefield Scott Stornetta, «~\emph{How to time-stamp a digital
  document}~», \emph{Journal of Cryptology}, 1991~:
  \url{http://www.staroceans.org/e-book/Haber_Stornetta.pdf}.}. Le
principe est simple~: il consiste à hacher une information (ou un
document) et de partager l'empreinte obtenue pour prouver que
l'information (ou le document) existait à la date de partage. Cette
méthode est notamment mise en œuvre par l'intermédiaire de serveurs
d'horodatage centralisés qui se chargent d'enregistrer les empreintes,
auquel cas on parle d'horodatage certifié ou de \emph{trusted
timestamping}.

Le principe derrière la chaîne de blocs est de lier les blocs les uns
aux autres par ce procédé d'horodatage en inscrivant l'empreinte du bloc
horodaté dans le bloc suivant. Cela crée des références récursives~: le
dernier bloc contient l'empreinte de l'avant-dernier bloc,
l'avant-dernier de l'antépénultième, etc. pour remonter jusqu'au bloc de
genèse (\emph{genesis block}), c'est-à-dire le premier bloc de la
chaîne, considéré comme valide par défaut. Pour la version principale de
Bitcoin, ce bloc contient le titre de la une du \emph{Times} du 3
janvier 2009, ce qui prouve que la chaîne n'a pas été lancée avant et
empêche par conséquent l'antidatage.

La particularité de cette structure est qu'elle fait reposer la sécurité
des maillons précédents sur les nouveaux maillons. Comme l'écrivait
Satoshi Nakamoto dans le livre blanc~:

«~Chaque horodatage inclut l'horodatage précédent dans son empreinte,
formant ainsi une chaîne, au sein de laquelle chaque horodatage
supplémentaire renforce le précédent\footnote{Satoshi Nakamoto,
  \emph{Bitcoin: A Peer-to-Peer Electronic Cash System}, 31 octobre
  2008.}.~»

En 2008, l'idée n'était pas nouvelle car elle avait déjà été appliquée
en 1995 par les mêmes Haber et Stornetta, qui avaient publié chaque
semaine une empreinte cryptographique dans les petites annonces du
\emph{New York Times} afin d'authentifier les documents des clients de
leur société\footnote{«~Haber et Stornetta, qui avaient publié chaque
  semaine une empreinte cryptographique dans les petites annonces du
  \emph{New York Times}~»~: Daniel Oberhaus, \emph{The World's Oldest
  Blockchain Has Been Hiding in the New York Times Since 1995}, 27 août
  2018~:
  \url{https://www.vice.com/en/article/j5nzx4/what-was-the-first-blockchain}.}.
C'était alors la manière la plus sûre de garantir l'intégrité des
empreintes, le journal étant distribué quotidiennement à plus d'un
million de personnes.

Satoshi Nakamoto a reproduit cette idée de diffusion publique des
données en faisant de son système un «~serveur d'horodatage
distribué\footnote{Satoshi Nakamoto, \emph{Bitcoin: A Peer-to-Peer
  Electronic Cash System}, 31 octobre 2008.}~» reposant sur un réseau
pair à pair librement accessible sur Internet. Dans Bitcoin, chaque bloc
comporte en effet une date et une heure inscrites par le mineur, si bien
que le résultat obtenu constitue une chaîne temporelle
(\emph{timechain}\footnote{La dénomination \emph{timechain} a été
  utilisée par Satoshi dans les commentaires du code source de novembre
  2008.}) témoignant de l'avancée du temps dans le monde réel.

Cette chaîne a rapidement été qualifiée de chaîne de blocs par les
premières personnes impliquées dans Bitcoin. Si le livre blanc parlait
déjà d'une «~\emph{chain of blocks}~», le terme «~\emph{block chain}~»
(en deux mots) a lui été créé par Hal Finney dans son premier courriel
de réponse à Satoshi le 7 novembre 2008\footnote{Hal Finney, \emph{Re:
  Bitcoin P2P e-cash paper}, /11/2008 23:40:12 UTC~:
  \url{https://www.metzdowd.com/pipermail/cryptography/2008-November/014827.html}}.
L'appellation a ensuite été reprise par le fondateur dans le code source
de la version 0.1 de Bitcoin et dans ses messages publics\footnote{«~La
  chaîne de blocs est une structure en forme d'arbre qui a pour racine
  le bloc de genèse, chaque bloc pouvant avoir plusieurs candidats à sa
  suite. pprev et pnext établissent un chemin à travers la chaîne
  principale~/~la chaîne plus longue. Un blockindex peut avoir plusieurs
  pprev qui pointent vers lui, mais pnext ne pointera que vers la
  branche la plus longue, ou sera nul si le bloc ne fait pas partie de
  la chaîne la plus longue.~» -- Satoshi Nakamoto, code source de la
  version 0.1 du logiciel Bitcoin~:
  \url{https://github.com/trottier/original-bitcoin/blob/4184ab26345d19e87045ce7d9291e60e7d36e096/src/main.h\#L1001-L1008}.}.
Le mot \emph{blockchain} s'est progressivement popularisé au sein de la
communauté pour parler de la chaîne de blocs de Bitcoin, puis, par
métonymie, de son mécanisme de consensus. Il a enfin (non sans
controverse) été élargi à la communication publique pour désigner (sous
le nom de «~technologie blockchain~» ou de «~blockchain~» tout court)
l'ensemble des techniques de consensus au sein de systèmes distribués,
que celles-ci fassent intervenir une chaîne de blocs ou non.

La particularité de Bitcoin est d'avoir combiné l'horodatage
d'informations et la preuve de travail produite par Hashcash. Puisque
ces deux procédés se fondent tous les deux sur une fonction de hachage,
il est en effet possible de les fusionner en un seul. La chaîne de blocs
est donc à la fois une chaîne temporelle d'horodatages et une chaîne de
preuves de travail.

\section*{L'agencement d'un bloc}\label{lagencement-dun-bloc}
\addcontentsline{toc}{section}{L'agencement d'un bloc}

\markright{L'agencement d'un bloc}

Comme son nom l'indique, la chaîne de blocs est une structure constituée
de blocs, qui sont des ensembles horodatés et travaillés de
transactions. Celle-ci débute par un bloc de genèse, valide par défaut,
à partir duquel sont comptés les blocs~: cet indice est appelé la
\emph{hauteur} et indique la position du bloc dans la chaîne dans
l'ordre de minage. Les blocs peuvent également être comptés dans l'autre
sens à partir du tout dernier bloc miné, auquel cas on parle de
\emph{profondeur}.

Chaque bloc possède un identifiant unique qui le démarque des autres.
Celui-ci est obtenu par hachage de l'entête du bloc (les données placées
avant les transactions) par le double SHA-256. Chaque bloc contient
l'identifiant du bloc précédent de sorte que l'ensemble forme une
chaîne. Puisque seul l'entête est impliqué dans le calcul de
l'identifiant, la chaîne de blocs peut en réalité être réduite à une
chaîne d'entêtes, auxquels les transactions sont liées
cryptographiquement. L'identifiant commence par un certain nombre de
zéros témoignant du fait qu'un travail a été demandé. Ainsi, le bloc
lui-même constitue la preuve de travail.

Les blocs sont tous organisés de la même façon, si bien qu'il suffit
d'en examiner un seul en détail pour comprendre comment la chaîne se
structure. Étudions donc un bloc de la version principale de Bitcoin
(BTC) en prenant pour exemple le bloc de hauteur 751~005 miné le 25 août
2022, qui contient 6 transactions.

Chaque bloc se décompose en un entête de 80 octets, qui contient ses
informations essentielles, et une succession brute de transactions. Par
convention, la première transaction du bloc est la transaction de
récompense (\emph{coinbase transaction}) servant à rémunérer le mineur
de ce bloc comme nous le verrons plus bas.

L'entête est, lui, divisé en six éléments~: la version du bloc,
l'identifiant du bloc précédent qui l'associe au bloc présent, une
racine de Merkle qui engage cryptographiquement l'ensemble des
transactions à l'entête, l'horodatage du bloc, la valeur cible du réseau
et le nonce relatif au minage. Les différentes informations contenues
dans l'entête sont transmises avec un ordre des octets inverse (dit
«~\emph{little-endian}~» ou «~petit-boutiste~») par rapport à l'ordre
ordinaire de lecture (qu'on appelle «~\emph{big-endian}~» ou
«~gros-boutiste~»). Nous les donnons ici dans l'ordre ordinaire.

\subsection{La version du bloc}\label{la-version-du-bloc}

La version du bloc indique l'ensemble des règles respectées par le bloc.
Historiquement, la version 1 marquait un respect des règles du protocole
originellement défini par Satoshi. Les versions 2 à 4 ont servi à
imposer l'application de certains changements du protocole entre 2013 et
2015. Depuis 2016, ce champ de version est utilisé pour le signalement
des mineurs dans le cadre de l'application d'un soft fork par
l'exécution du BIP-9 ou par un mécanisme équivalent. Le champ de version
de notre bloc est~:

\begin{verbatim}
0b00100000000000000000000000000100
\end{verbatim}

\subsection{L'identifiant du bloc
précédent}\label{lidentifiant-du-bloc-pruxe9cuxe9dent}

L'identifiant du bloc précédent sert à lier l'entête du bloc présent à
l'entête du bloc précédent. Dans le cas du bloc de genèse, ce champ est
fixé à zéro par convention. Dans notre bloc, il constitue l'identifiant
du bloc 751~004, qui est~:

\begin{verbatim}
000000000000000000073ad6c18c81f2f67b2ca5b5ace8d23cce95812af8c7b6
\end{verbatim}

\subsection{La racine de Merkle}\label{la-racine-de-merkle}

Le troisième élément de l'entête est la racine de Merkle, qui correspond
à l'empreinte finale de l'agencement des transactions en arbre de
Merkle.

Un arbre de Merkle, aussi appelé arbre de hachage, est une structure de
données conceptualisée en 1979 par le cryptographe Ralph Merkle
permettant de vérifier le contenu d'un volume de données sans avoir
besoin de toutes les inspecter. Dans une telle structure, les données
(constituant alors les feuilles de l'arbre) sont rangées dans un certain
ordre et hachées respectivement. Puis leurs empreintes sont combinées
deux à deux pour être hachées à leur tour, et ceci jusqu'à ce qu'il ne
reste plus qu'une seule empreinte, qu'on appelle la racine. Les chaînes
de hachages qui relient les feuilles à la racine sont appelées les
branches.

Dans les blocs de Bitcoin, ce sont les transactions qui sont les données
hachées. Elles sont d'abord hachées une première fois (ce qui correspond
à leur identifiant)~:

\[H_A = \mathrm{SHA256d}(~\mathrm{tx}_A~)\]

Puis les empreintes résultantes sont concaténées deux à deux (la
deuxième empreinte est placée à la suite de la première) et l'ensemble
est passé par la même fonction de hachage~:

\[H_{A\!B} = \mathrm{SHA256d}(~H_A \parallel H_B~)\]

Le procédé est ensuite réitéré. Dans le cas où le nombre d'empreintes à
combiner est impair, la dernière est concaténée avec elle-même~:

\[H_{E\!F\!E\!F} = \mathrm{SHA256d}(~H_{E\!F} \parallel H_{E\!F}~)\]

Une fois qu'il ne reste qu'une seule empreinte, l'arbre est complet~:
l'empreinte finale obtenue est la racine de Merkle.

La racine de Merkle du bloc 751~005 est ainsi~:

\begin{verbatim}
268a15b56fe847a067624bd0be186c375baccae9ac6db304438e9da657fe51d9
\end{verbatim}

Le fait de placer la racine dans l'entête interdit à quiconque de
modifier, d'ajouter ou de supprimer une transaction, sans modifier
l'entête lui-même et devoir reproduire la preuve de travail. L'ensemble
des transactions est ainsi attaché à l'entête, ce qui assure l'intégrité
du bloc.

\begin{figure}

{\centering \includegraphics{chapters/img/merkle-tree.png}

}

\caption{Représentation d'un arbre de Merkle à six feuilles.}

\end{figure}%

Cette organisation se révèle particulièrement utile pour les
portefeuilles légers (dits à vérification de paiement simplifiée ou SPV)
qui ne conservent pas la chaîne de blocs entière mais uniquement la
chaîne des entêtes, qui est bien moins volumineuse (un peu plus de 62
Mio en novembre 2023). En effet, pour s'assurer de la présence d'une
transaction dans un bloc, ils peuvent se contenter de demander les
informations liées à la branche (chemin de Merkle) et procéder aux
hachages eux-mêmes\footnote{Démontrer qu'une feuille fait partie d'un
  arbre de Merkle requiert de calculer un nombre d'empreintes
  proportionnel au logarithme binaire du nombre de feuilles
  (\(\log_{2}(n)\)), et non pas proportionnel au nombre de feuilles
  \(n\). Pour un bloc de 3~000 transactions (moyenne haute sur BTC),
  cela représente 12 empreintes de 32 octets à obtenir et 12 hachages à
  effectuer.}. Par exemple, un utilisateur voulant vérifier la
confirmation de la transaction \(\mathrm{tx}_D\) doit simplement
demander les informations \(H_C\), \(H_{A\!B}\) et \(H_{E\!F\!E\!F}\)
aux nœuds du réseau et procéder aux différents hachages pour comparer la
racine obtenue avec celle contenue dans l'entête. Cela a pour effet de
réduire considérablement la charge des portefeuilles légers.

Depuis l'activation de SegWit le 24 août 2017, chaque bloc contient un
arbre de Merkle supplémentaire, subordonné à l'arbre classique des
transactions décrit plus haut. Il s'agit de l'arbre témoin qui est
l'arbre des transactions intégrant les signatures des transactions
SegWit (séparées des transactions classiques). La racine de l'arbre
témoin est placée dans la transaction de récompense, de sorte qu'elle
est prise en compte dans la racine de Merkle principale, ce qui garantit
l'intégrité de l'ensemble.

\subsection{L'horodatage}\label{lhorodatage}

L'horodatage indique la date et l'heure de construction du bloc qui sont
déclarées par le mineur. D'un point de vue technique, il est donné par
l'heure Unix, c'est-à-dire le nombre de secondes écoulées depuis le 1
janvier 1970 :00:00 UTC. Pour notre bloc, l'horodatage est de
\(1661407005\) ce qui correspond à la date du 25 août 2022 à 5 heures 56
minutes et 45 secondes (UTC).

Le mineur ne peut pas choisir cet horodatage au hasard. L'heure déclarée
doit se situer dans le futur par rapport au temps médian passé (MTP) --
la médiane des horodatages des 11 derniers blocs, qui retarde
généralement d'une heure sur le temps réel -- et ne doit pas dépasser
l'horloge des nœuds récepteurs de deux heures\footnote{«~ne doit pas
  dépasser l'horloge des nœuds récepteurs de deux heures~»~:
  \url{https://github.com/bitcoin/bitcoin/blob/24.x/src/validation.cpp\#L3483-L3490}.}.
Cette contrainte relativement permissive permet au temps réseau de
rester relativement cohérent avec la réalité.

\subsection{La valeur cible}\label{la-valeur-cible}

La valeur cible est la valeur minimale que l'identifiant du bloc peut
prendre pour que ce bloc constitue une preuve de travail. Plus cette
valeur cible est petite, plus il est facile de trouver une solution et
de miner un bloc. Elle est déterminée par le réseau selon les règles de
l'algorithme d'ajustement de la difficulté.

La valeur cible est encodée comme un nombre flottant où le premier octet
représente un exposant particulier et où la mantisse est déterminée par
les 3 octets suivants. Ici, elle est égale à
\(\mathtt{0x09ed88} \times 256^{(\mathtt{0x17} - 3)}\) c'est-à-dire~:

\begin{verbatim}
00000000000000000009ed880000000000000000000000000000000000000000
\end{verbatim}

Cette information donne aussi la difficulté de minage du bloc, qui est
inversement proportionnelle à la valeur cible. Il s'agit du quotient de
la valeur cible maximale du système par la valeur cible du
réseau\footnote{En notant \(c\) la valeur cible, la difficulté est
  définie par~: \[d = \frac{C_{\mathrm{max}}}{c}\] où
  \(C_{\mathrm{max}} = \mathtt{0x00ffff} \times 256^{26}\) est la valeur
  cible maximale du réseau.}. La difficulté minimale du protocole est
donc de 1 et celle de notre bloc (arrondie à l'unité près) est quant à
elle de \(28~351~606~743~494\), ce qui représente un différentiel
énorme~! Elle donne également la quantité de travail du bloc, qui est le
nombre moyen de hachages nécessaires pour tomber sur une
solution\footnote{En termes mathématiques, le travail d'un bloc est le
  quotient du nombre d'empreintes possibles par le nombre d'empreintes
  satisfaisant le problème. En notant \(c\) la valeur cible, le travail
  est~: \[T = \frac{2^{256}}{c + 1} ~.\]}.

\subsection{Le nonce}\label{le-nonce}

Le nonce désigne le nombre que le mineur fait varier pour produire la
preuve de travail. Ce mot provient de l'expression anglaise «~\emph{for
the nonce}~» signifiant «~pour la circonstance, pour l'occasion~», ce
qui indique la spécificité de son rôle\footnote{Une étymologie populaire
  prétend qu'il serait une contraction de l'expression «~\emph{number
  used once}~», mais celle-ci est incorrecte.}. Le mineur fait également
varier un nonce supplémentaire au sein de la transaction de récompense,
le champ du nonce étant trop petit (8 octets) pour la difficulté de
minage actuelle. Le nonce de notre bloc est \(4~224~551~499\).

Ces deux derniers paramètres (valeur cible et nonce) sont relatifs à la
preuve de travail et interviennent dans la formulation du problème
mathématique résolu par le mineur. Ce problème se présente sous la forme
d'une inégalité mathématique. En notant \(c\) la valeur cible du réseau
et \(\mathrm{EB}\) l'entête du bloc, il s'agit de trouver un nonce \(n\)
tel que~:

\[\mathrm{SHA256d} ( \ \mathrm{EB} ( \ n \ ) \ ) ~ \le ~ c\]

Comme on l'a dit, le résultat est utilisé comme identifiant du bloc. La
preuve de travail est facilement vérifiable~: chaque membre du réseau
peut, à partir des données du bloc, s'assurer que le mineur a bien
trouvé une solution valide. Dans notre cas, si on compare l'identifiant
et la valeur cible, on obtient bien un résultat qui satisfait
l'inégalité exigée~:

\[\begin{aligned}
\mathtt{0x000000000000000000065aebf106c8824f4b565d54d6d6df32498b2b041cfd07} & \le \\ \mathtt{0x00000000000000000009ed880000000000000000000000000000000000000000} & ~\end{aligned}\]

\begin{figure}

{\centering \includegraphics{chapters/img/bitcoin-segwit-block.png}

}

\caption{Schéma d'un bloc de Bitcoin (avec SegWit).}

\end{figure}%

\section*{Le revenu du minage}\label{le-revenu-du-minage}
\addcontentsline{toc}{section}{Le revenu du minage}

\markright{Le revenu du minage}

L'une des innovations de Bitcoin est de récompenser la confirmation des
transactions à l'aide de son unité de compte interne. Cette propriété
crée une incitation économique poussant les mineurs à bien se comporter,
ce qui contribue à la solidité du système.

La récompense liée à l'ajout d'un bloc à la chaîne provient en partie de
la création monétaire du protocole, d'où le nom de minage employé pour
désigner cette activité. Le procédé est en effet analogue à l'extraction
minière de l'or dans le monde réel~: les mineurs déploient du capital et
dépensent de l'énergie pour obtenir les nouveaux bitcoins. Comme
expliqué par Satoshi dans le livre blanc~:

«~L'ajout régulier d'une quantité constante de nouvelles pièces est
analogue aux mineurs d'or qui dépensent des ressources pour ajouter de
l'or à la circulation\footnote{Satoshi Nakamoto, \emph{Bitcoin: A
  Peer-to-Peer Electronic Cash System}, 31 octobre 2008.}.~»

La deuxième partie de la récompense provient des frais de transaction
payés par les utilisateurs, qui sont collectés sur les transactions
incluses dans le bloc. Le tout est reversé au mineur lorsque le bloc est
vérifié et accepté par le réseau.

Le minage est ainsi l'activité économique consistant à rassembler les
transactions au sein d'un bloc, à produire la preuve de travail et à
diffuser le résultat sur le réseau. Ici, nous le distinguons ainsi du
simple hachage, qui consiste juste à réaliser les calculs pour créer la
preuve de travail et qui peut être réalisé indépendamment de la
sélection de transactions, notamment au sein des coopératives de minage
(\emph{mining pools}). Dans ce cadre, les mineurs sont les personnes ou
les groupes de personnes réalisant l'activité complète, et les entités
se contentant de mettre en place des machines et de déléguer leur
pouvoir sur la sélection des transactions ne sont que des «~hacheurs~».

Le minage se déroule de manière cyclique. Tout d'abord, le mineur
sélectionne des transactions à partir de la réserve des transactions
(appelée \emph{mempool}) de son nœud. Puis, il construit un bloc
candidat en imposant un entête, en assemblant les transactions et en
prenant soin de construire une transaction de récompense qui le
rémunère. Il fait ensuite varier le nonce et d'autres éléments du bloc
candidat afin de produire la preuve de travail. Enfin, dans le cas où il
trouve une solution, il diffuse le bloc sur le réseau le plus rapidement
possible pour que les autres nœuds le vérifient et l'acceptent comme le
nouveau bloc de la chaîne. Dans le cas contraire, si un nouveau bloc est
trouvé entretemps, le mineur l'accepte et abandonne son bloc candidat.
Dans les deux cas, la procédure reprend du début avec des transactions
différentes.

La récompense de minage est ainsi récupérée par le mineur \emph{via}
l'inclusion d'une transaction de récompense au sein du bloc. Celle-ci
doit être, par convention, la première transaction du bloc. Elle possède
une entrée unique spécifique ne faisant référence à aucune transaction
existante. La transaction de récompense est aussi appelée la base de
pièce ou \emph{coinbase}, car c'est à partir d'elle que sont formés les
nouveaux bitcoins. Le mineur dirige cette transaction vers une adresse
qu'il contrôle, de sorte qu'il est récompensé si et seulement si son
bloc est valide aux yeux du réseau. La récompense que le mineur peut se
verser doit être inférieure à la somme de la création monétaire et des
frais de transaction. Le mineur peut ainsi se rémunérer moins que ce qui
est prévu par le protocole, même si cela n'a aucun sens économique
direct\footnote{En décembre 2017, le mineur du bloc 501~726 s'est ainsi
  rémunéré de la coquette somme de 0~BTC~!}.

La création monétaire se fait intégralement par le biais de la
transaction de récompense. Tous les bitcoins dans le système sont ainsi
le résultat d'une série de transferts commençant par une telle
transaction.

La particularité de cette création monétaire est qu'elle est fixée dans
le temps et qu'elle n'est pas proportionnelle à la puissance de calcul
déployée. Cela est rendu possible par l'algorithme d'ajustement de la
difficulté, qui dérive du fait que le système constitue un serveur
d'horodatage distribué. En effet, les blocs étant horodatés, il est
possible de mesurer leur rythme de production passé et d'ajuster la
difficulté de minage en conséquence. Ainsi, comme l'écrivait Satoshi~:

«~Afin de compenser l'augmentation de la vitesse du matériel et la
variation de l'intérêt des nœuds actifs au fil du temps, la difficulté
de la preuve de travail est déterminée par une moyenne mobile visant un
nombre moyen de blocs par heure. Si ces blocs sont générés trop
rapidement, la difficulté augmente\footnote{Satoshi Nakamoto,
  \emph{Bitcoin: A Peer-to-Peer Electronic Cash System}, 31 octobre
  2008.}.~»

Dans la version principale de Bitcoin, l'intervalle de temps entre
chaque bloc (temps de bloc) visé est de 10 minutes ou 600 secondes.
L'ajustement a lieu tous les 2016 blocs, ce qui correspond environ à
deux semaines, selon la moyenne simple du temps de bloc sur cette
période. La nouvelle valeur cible est calculée\footnote{Dans Bitcoin
  Core, l'algorithme d'ajustement est décrit par la fonction
  `\texttt{dans\ le\ fichier}pow.cpp`. La variation est limitée à un
  facteur 4 (multiplication comme division) pour éviter les
  instabilités. L'algorithme \emph{surestime} la puissance de calcul
  déployée car le temps écoulé est mesuré sur 2~015 intervalles, et non
  pas 2~016 comme cela devrait se faire.}\footnote{«~la fonction
  CalculateNextWorkRequired dans le fichier pow.cpp~»~:
  \url{https://github.com/bitcoin/bitcoin/blob/24.x/src/pow.cpp\#L49-L72}.}
à partir de la valeur cible précédente (\(c_{k-1}\)) et du temps écoulé
depuis le dernier ajustement (\(t_{k-1}\))~:

\[c_{k} = \frac{c_{k-1} \cdot t_{k-1}}{2016 \cdot 600}\]

Grâce à cet ajustement, le bitcoin possède une politique monétaire
déterminée, qui n'est pas soumise à l'arbitraire direct d'un tiers de
confiance ou à la quantité de capital déployé. Cette caractéristique le
différencie de la monnaie fiat (comme le dollar) qui est émise de
manière discrétionnaire par une banque centrale, ou du métal précieux
(comme l'or) dont la quantité extraite connaît ses propres variations et
suit la demande du marché à long terme. Cette politique monétaire a été
décrite précisément pour la première fois par Satoshi Nakamoto dans son
courriel de lancement du 8 janvier 2009 où il écrivait~:

«~La quantité en circulation totale sera de 21~000~000 pièces. Elle sera
distribuée aux nœuds du réseau lorsqu'ils créeront des blocs, le montant
étant divisé par deux tous les 4 ans.

les quatre premières années : 10~500~000 pièces\\
les quatre années suivantes : 5~250~000 pièces\\
les quatre années suivantes : 2~625~000 pièces\\
les quatre années suivantes : 1~312~500 pièces\\
etc...

Lorsque cela est épuisé, le système peut prendre en charge les frais de
transaction si nécessaire. Il est basé sur la concurrence du marché
ouvert, et il y aura probablement toujours des nœuds prêts à traiter les
transactions gratuitement\footnote{Satoshi Nakamoto, \emph{Bitcoin v0.1
  released}, /01/2009 19:27:40 UTC~:
  \url{https://www.metzdowd.com/pipermail/cryptography/2009-January/014994.html}.}.~»

Elle est bien évidemment inscrite dans le code\footnote{«~Elle est bien
  évidemment inscrite dans le code~»~:
  \url{https://github.com/bitcoin/bitcoin/blob/24.x/src/validation.cpp\#L1473-L1484}},
où elle est appelée subvention ou \emph{subsidy} en anglais.

L'originalité principale de cette politique monétaire est que la
création monétaire est réduite de moitié de manière brusque tous les
210~000 blocs (soit environ 4 ans) lors de ce qu'on appelle couramment
un \emph{halving}. En 2023, trois réductions de moitié avaient déjà eu
lieu sur le réseau Bitcoin principal~: la première s'est produite le 28
novembre 2012, lorsque la subvention du protocole est passée de 50
bitcoins par bloc à 25~; la deuxième le 9 juillet 2016, avec une baisse
à 12,5 bitcoins par bloc~; la troisième le 11 mai 2020, où la subvention
a été réduite à 6,25 bitcoins par bloc. La prochaine réduction de moitié
devrait se passer en avril 2024, après laquelle les nouveaux bitcoin
émis seront de 3,125 par bloc. Sauf modification des règles de
consensus, la dernière réduction de moitié sera la 33 et aura lieu aux
alentours de 2140. En effet, le montant de création monétaire par bloc
passera alors en dessous du satoshi, soit zéro par troncature à l'unité.

À long terme, cette politique monétaire atypique fait du bitcoin une
monnaie à quantité fixe. En effet, le montant maximal de bitcoins en
circulation doit tendre, au fil du temps, vers une limite~: la fameuse
limite des 21 millions. Celle-ci n'est qu'une déduction des conditions
d'émission susmentionnées, ce qui s'exprime en termes mathématiques par
la convergence de la série des montants minés entre les
halvings\footnote{Cette convergence est illustrée par le paradoxe
  d'Achille et de la tortue formulé par le philosophe grec Zénon. La
  suite \(\left( \sum_{i=1}^{n} (1/2)^i \right)\) converge vers \(1\)
  lorsque \(n\to+\infty\).}~:

\[N_{\mathrm{max}} = \sum_{i=0}^{+\infty} \left( {210~000 \cdot \frac{50}{2^i}} \right) = 21~000~000 \cdot \sum_{i=1}^{+\infty} \left(\frac{1}{2}\right)^i = 21~000~000\]

La limite des 21 millions est une borne supérieure~: en l'absence d'un
changement des règles de consensus, elle ne sera jamais formellement
atteinte, en raison de la nature optionnelle de la récompense de minage,
du caractère discret des unités et de la perte irrémédiable de bitcoins.
De plus, les bitcoins dont les propriétaires ont perdu leurs clés
privées réduisent considérablement la quantité réelle de bitcoins en
circulation sans pour autant que cela ne soit pris en compte dans le
calcul.

La création monétaire a ainsi vocation à s'amenuiser et à devenir
négligeable, et ce plus rapidement que l'on imagine. En effet, en 2023,
le nombre de bitcoins dépensables avait déjà dépassé les 19,5 millions.
C'est pourquoi cette subvention doit en toute logique être remplacée par
l'autre source de revenu pour les mineurs, à savoir les frais de
transaction\footnote{«~Une fois qu'un nombre prédéterminé de pièces a
  été mis en circulation, l'incitation peut être entièrement financée
  par les frais de transaction et ne plus requérir aucune inflation.~»
  -- Satoshi Nakamoto, \emph{Bitcoin: A Peer-to-Peer Electronic Cash
  System}, 24 mars 2009.}.

Les frais de transaction sont les commissions payées par les
utilisateurs pour la confirmation de leurs transactions. Les frais d'une
transaction peuvent être versés directement par l'expéditeur (client) ou
indirectement par le destinataire (commerçant) par l'intermédiaire d'une
remise sur le produit vendu. Ils sont récupérés par le mineur sur chaque
transaction du bloc selon une règle implicite~: il s'agit de la
différence entre le montant en entrée de la transaction et son montant
en sortie. Cette différence peut être de zéro (transaction gratuite),
mais elle est toujours comptabilisée. Les frais sont ajoutés à la
transaction de récompense indistinctement des bitcoins issus de la
création monétaire. Bitcoin intègre ainsi un système interne et optimisé
de frais de transaction, qui évite l'alourdissement inutile des
transactions et des blocs.

L'existence des frais de transaction a vocation à perdurer par
conception, même si ceux-ci devenaient très bas. Contrairement à
l'opinion exprimée par Satoshi, la confirmation d'une transaction a en
général un coût, même marginal\footnote{Le seul cas envisageable de
  transaction gratuite est celui d'une grande transaction de
  consolidation qui amoindrirait la charge des mineurs en réduisant
  considérablement l'ensemble des sorties transactionnelles non
  dépensées.}, et une transaction qui paie trop peu de frais par rapport
à la charge apportée n'a aucune raison économique d'être confirmée. De
ce fait, il n'y a pas lieu de s'imaginer que la chaîne de blocs
s'arrête.

En outre, les règles du protocole restreignent usuellement l'espace de
bloc par le biais d'une limite explicite sur la taille (ou le poids) des
blocs. Cette restriction crée un plafond de production qui, lorsqu'il
est atteint, fait que le mineur rationnel sélectionne les transactions
qui paient le taux le plus élevé de frais, toutes choses étant égales
par ailleurs. Il existe donc, dans le cas d'une congestion du réseau, un
effet d'enchères pouvant faire augmenter le niveau moyen des frais de
manière drastique.

Bien que les frais constituent la façon principalement envisagée de
rémunérer les mineurs à terme, des méthodes alternatives de financement
ont été proposées.

La première est l'émission de queue (\emph{tail emission}), qui consiste
à maintenir une création monétaire constante au cours du temps, dans le
but que le revenu de minage ne tombe pas trop bas\footnote{Peter Todd,
  \emph{Surprisingly, Tail Emission Is Not Inflationary}, 9 juillet
  2022~:
  \url{https://petertodd.org/2022/surprisingly-tail-emission-is-not-inflationary}.}.
L'instauration de cette caractéristique aurait pour effet de modifier la
politique monétaire du bitcoin et de faire disparaître la limite des 21
millions, d'où son caractère hautement controversé.

Pour donner un exemple, l'émission de queue est mise en place dans la
variante Monero depuis 2015. Elle est devenue effective le 9 juin 2022,
date depuis laquelle il se crée 0,3~monero par minute, soit un taux de
création monétaire annualisé de 0,87~\% à ce moment-là. Une telle
émission de queue existe également dans Dogecoin depuis 2015, à raison
de 10~000 dogecoins par minute, pour un taux annualisé de 3,7~\% en
novembre 2023.

La deuxième méthode de financement proposée est le demeurage, ou coût de
détention, qui consiste à prélever la monnaie demeurée immobile depuis
un temps donné\footnote{Jorge Timón, \emph{Freicoin: bitcoin with
  demurrage}, /02/2011 11:56:03 UTC~:
  \url{https://bitcointalk.org/index.php?topic=3816.msg54170\#msg54170}.}.
Les bitcoins de Satoshi, qui représentent une manne financière
importante, sont notamment concernés. Toutefois, il s'agirait d'une
atteinte au système de propriété de Bitcoin et il y a donc peu de
chances que cette méthode rencontre le succès.

\section*{La chaîne la plus longue}\label{la-chauxeene-la-plus-longue}
\addcontentsline{toc}{section}{La chaîne la plus longue}

\markright{La chaîne la plus longue}

Venons-en maintenant au sujet central de ce chapitre~: l'atteinte du
consensus par le minage. Comme nous l'avons expliqué ci-dessus, le
minage est le procédé permettant aux mineurs d'ajouter des blocs à la
chaîne, chose pour laquelle ils sont rémunérés. Mais nous n'avons pas
exposé comment il permettait d'arriver à un accord dans un contexte
antagoniste, en présence d'acteurs malveillants «~byzantins~».

Les nœuds suivent un protocole composé des règles de réseau, qui leur
permettent de rentrer en communication, et des règles de consensus, qui
concernent la forme des transactions et des blocs, que nous détaillerons
dans le chapitre~\hyperref[ch:changement]{10}. Les nœuds qui enfreignent
ces règles voient leurs connexions être fermées par leurs pairs et sont
mis sur liste noire si nécessaire. Il est donc impossible de faire
accepter une transaction ou un bloc au réseau qui ne soit valide selon
les règles de consensus.

Néanmoins, les nœuds byzantins peuvent semer la discorde dans le respect
des règles de consensus, en produisant des blocs concurrents. En effet,
rien n'empêche \emph{a priori} un attaquant de produire des blocs de
transactions qui soient valides mais qui ne soient pas rattachés à la
branche principale et de les soumettre au réseau.

Ce problème est résolu par le biais d'un principe simple mais efficace~:
le principe de la chaîne la plus longue. Celui-ci a été décrit par
Satoshi dans le livre blanc~:

«~La décision majoritaire est représentée par la chaîne la plus longue,
sur laquelle le plus grand effort de preuve de travail a été
investi\footnote{Satoshi Nakamoto, \emph{Bitcoin: A Peer-to-Peer
  Electronic Cash System}, 31 octobre 2008.}.~»

Les nœuds du réseau se mettent d'accord en sélectionnant la chaîne
possédant le plus de travail accumulé\footnote{«~la chaîne possédant le
  plus de travail accumulé~»~: Ce principe a été redéfini le 25 juillet
  2010 au sein de la version 0.3.3 du logiciel~:
  \url{https://github.com/bitcoin/bitcoin/commit/3b7cd5d89a226426df9c723d1f9ddfe08b7d1def}.},
ce qui se matérialise généralement par une chaîne plus longue en nombre
de blocs\footnote{En réalité, au début c'était bel et bien la chaîne
  possédant le plus de blocs qui était sélectionnée. Mais ce principe a
  été redéfini le 25 juillet 2010 au sein de la version 0.3.3 du
  logiciel pour prendre en compte la notion de travail.}. Lorsqu'une
chaîne possédant une quantité strictement plus grande de travail est
publiée, les nœuds suivent cette chaîne, que celle-ci soit dans la
continuité de la dernière ou qu'elle fasse référence à une branche plus
ancienne. Cette règle fait en sorte que les nœuds suivent toujours la
chaîne sur laquelle un montant supérieur d'énergie a été investi.
L'algorithme de consensus résultant de l'application de ce principe est
appelé l'algorithme de consensus de Nakamoto par preuve de travail, en
hommage à son concepteur.

La meilleure manière d'appréhender le fonctionnement de cet algorithme
est de prendre le cas d'un embranchement (appelé \emph{fork} en anglais)
de la chaîne. Celui-ci peut être créé par un acteur malveillant, mais
dans la réalité il est généralement engendré de manière accidentelle,
lorsque deux mineurs éloignés trouvent chacun un bloc différent dans un
intervalle de temps réduit et que les nœuds du réseau ne reçoivent pas
le même bloc en premier. Il n'y a alors aucun moyen de départager les
deux branches, celles-ci étant également correctes en vertu du principe
de la chaîne la plus longue. Ce type d'embranchement accidentel est
commun et se produit de temps en temps sur le réseau pour des raisons de
latence.

Cette situation et sa résolution ont été décrites par Satoshi dans le
livre blanc~:

«~Si deux nœuds transmettent simultanément des versions différentes du
bloc suivant, certains nœuds peuvent recevoir l'une ou l'autre version
en premier. Dans ce cas, ils travaillent sur la première version qu'ils
ont reçue, mais conservent l'autre branche au cas où elle deviendrait
plus longue. L'égalité est rompue lorsque la preuve de travail suivante
est trouvée et qu'une branche devient plus longue~; les nœuds qui
travaillaient sur l'autre branche passent alors sur la chaîne la plus
longue\footnote{Satoshi Nakamoto, \emph{Bitcoin: A Peer-to-Peer
  Electronic Cash System}, 31 octobre 2008.}.~»

Le réseau passe par trois étapes. Tout d'abord, il se comporte de
manière attendue~: les mineurs prolongent la chaîne la plus longue, sur
laquelle le reste des nœuds se coordonnent. Puis, le conflit a lieu~:
deux branches correctes coexistent et les mineurs travaillent pour
prolonger la chaîne à partir du bloc reçu en premier. Enfin,
l'embranchement est résolu~: un mineur trouve un nouveau bloc et sa
chaîne, qui devient plus longue, est acceptée par le réseau.

Il se produit alors ce qu'on appelle une recoordination
(\emph{reorganization}) qui réconcilie les nœuds du réseau entre eux. Le
bloc de la branche faible est considéré comme incorrect et mis de côté.
On dit que ce bloc est rendu orphelin (\emph{orphaned}) car il perd son
attachement à la chaîne mère\footnote{L'appellation (quelque peu
  ambigüe) de «~bloc orphelin~» a été introduite par Satoshi Nakamoto au
  sein de la première version du logiciel. On parle aussi de «~bloc
  oncle~» (en référence au fait qu'il ne donne pas de descendance
  fertile) ou bien de «~bloc périmé~» (\emph{stale block}).}. La branche
forte (possédant le plus de travail accumulé) est considérée comme la
version correcte de la chaîne.

\begin{figure}

{\centering \includegraphics{chapters/img/blockchain-common-fork.png}

}

\caption{Schéma d'un embranchement commun de la chaîne.}

\end{figure}%

Tout conflit sur le réseau est résolu de la sorte, ce qui a pour
conséquence de conférer une nature particulière à l'algorithme de
Nakamoto, et par extension à Bitcoin.

Ce fonctionnement impose tout d'abord deux contraintes majeures sur la
sécurité. La première est que la sécurité minière du réseau repose sur
la supposition qu'une majorité de la puissance de calcul («~51~\%~») se
comporte de manière honnête. Comme l'expliquait Satoshi~:

«~Le système est sécurisé tant que les nœuds honnêtes contrôlent
collectivement plus de puissance de calcul qu'un groupe de nœuds qui
coopéreraient pour réaliser une attaque\footnote{Satoshi Nakamoto,
  \emph{Bitcoin: A Peer-to-Peer Electronic Cash System}, 31 octobre
  2008.}.~»

La seconde est que la sécurité d'une transaction donnée est probabiliste
et dépend de la profondeur à laquelle elle se trouve dans la chaîne. La
transaction est d'abord vérifiée par le réseau (zéro confirmation), puis
confirmée au sein d'un bloc (une confirmation) et finit par être
considérée comme irréversible, généralement à partir de six
confirmations pour les montants ordinaires sur la version principale de
Bitcoin. Cela contraint l'utilisateur à estimer le nombre de
confirmations qu'il doit attendre en fonction de la sécurité désirée.

Cette particularité se transcrit dans le fonctionnement du minage par la
maturité de la base de pièce (\emph{coinbase maturity}), qui est le
nombre de confirmations nécessaire pour que la sortie de la transaction
de récompense devienne dépensable. Cette contrainte est mise en place
pour éviter la mauvaise utilisation des fonds due à une recoordination
peu profonde. Le délai sur le réseau BTC est aujourd'hui de 101
confirmations\footnote{«~Le délai sur le réseau BTC est aujourd'hui de
  101 confirmations~»~:
  \url{https://github.com/bitcoin/bitcoin/blob/23.x/src/consensus/consensus.h\#L18-L19}}.

L'algorithme de Nakamoto possède également trois avantages principaux.
D'abord, il a pour intérêt d'avoir un critère objectif sur lequel se
reposer~: tout le monde peut reconstituer la chaîne à partir du bloc de
genèse et constater qu'il s'agit de la chaîne correcte. Même dans le cas
extrême d'un cloisonnement mondial et prolongé du réseau dû à une guerre
ou une catastrophe naturelle, le système peut finir par se
recoordonner\footnote{Satoshi Nakamoto, \emph{Re: Anonymity}, /07/2010,
  19:12:00 UTC~:
  \url{https://bitcointalk.org/index.php?topic=241.msg2071\#msg2071}.}.

Ensuite, il permet la participation ouverte au consensus~: tout ce qui
est requis du mineur est une preuve de travail valide, de sorte que le
minage est anonyme par essence.

Enfin, cet algorithme par preuve de travail assure la robustesse du
réseau~: un mineur n'a pas à connaître tous les autres participants, ce
qui permet au réseau d'être composé de dizaines (voire de centaines) de
milliers de nœuds.

\section*{La résistance à la double
dépense}\label{la-ruxe9sistance-uxe0-la-double-duxe9pense}
\addcontentsline{toc}{section}{La résistance à la double dépense}

\markright{La résistance à la double dépense}

La double dépense est le fait pour un acteur de faire accepter
successivement deux transactions au réseau dans le but de déstabiliser
l'état du système et d'en bénéficier d'une manière ou d'une autre. La
deuxième transaction peut constituer une annulation de la première, dans
laquelle l'acteur malveillant réalise un transfert vers lui-même.

La double dépense constitue un problème dans le cas des transactions non
confirmées, c'est-à-dire des transactions qui ont été diffusées sur le
réseau, vérifiées par les nœuds et placées dans leurs \emph{mempools},
mais qui n'ont pas encore été incluses dans un bloc de la chaîne. Aucun
consensus n'a été réalisé à propos de ces transactions, mais le
commerçant peut décider de les accepter dans le cas où les montants
engagés sont faibles\footnote{Dans un message sur le forum en juillet
  2010, Satoshi écrivait à propos de l'acceptation des transactions non
  confirmées~: «~Je pense qu'il sera possible pour une entreprise de
  traitement des paiements de fournir comme service la distribution
  rapide de transactions avec une vérification suffisante en 10 secondes
  ou moins. Les nœuds du réseau n'acceptent que la première version
  d'une transaction qu'ils reçoivent pour l'incorporer dans le bloc
  qu'ils essaient de générer. Lorsqu'on diffuse une transaction et que
  quelqu'un d'autre diffuse une double dépense au même moment, c'est une
  course à la propagation vers le plus grand nombre de nœuds qui a lieu.
  Si l'un d'elles a une légère avance, elle se propagera géométriquement
  dans le réseau plus rapidement et atteindra la plupart des nœuds.
  {[}...{]} Le processeur de paiement a des connexions avec de nombreux
  nœuds. Lorsqu'il reçoit une transaction, il l'envoie et, en même
  temps, surveille le réseau pour détecter les doubles dépenses. S'il
  reçoit une double dépense sur l'un de ses nombreux nœuds d'écoute, il
  signale que la transaction est mauvaise.~» -- Satoshi Nakamoto,
  \emph{Re: Bitcoin snack machine (fast transaction problem)}, /07/2010
  22:29:13 UTC~:
  \url{https://bitcointalk.org/index.php?topic=423.msg3819\#msg3819}.}.
Le risque est qu'un fraudeur reparte avec la marchandise et réussisse à
faire accepter une version alternative de la transaction vue par le
commerçant, soit en la diffusant au même moment et en espérant qu'elle
arrive en premier au mineur, soit en payant plus de frais (ce qui peut
être fait systématiquement avec Replace-by-Fee) pour soudoyer le mineur,
soit encore en minant préalablement un bloc contenant la transaction
(attaque Finney\footnote{Hal Finney, \emph{Re: Best practice for fast
  transaction acceptance - how high is the risk?}, /02/2011, 21:48:44
  UTC~:
  \url{https://bitcointalk.org/index.php?topic=3441.msg48384\#msg48384}.}).

La solution à ce problème est de se mettre d'accord sur la transaction
correcte pour faire disparaître la double dépense, \emph{ce qui est
précisément le but du minage}. Cependant, le minage n'empêche pas la
double dépense de manière absolue, étant plutôt un mécanisme de
résistance. Voyons ce qui garantit cette caractéristique.

Un certain nombre de perturbations opportunistes peuvent avoir lieu au
niveau de l'activité minière comme l'attaque vector76\footnote{vector76,
  \emph{Re: Fake Bitcoins?}, /08/2011 17:37:56 UTC~:
  \url{https://bitcointalk.org/index.php?topic=36788.msg463391\#msg463391}.}
ou le minage égoïste\footnote{Ittay Eyal, Emin Gün Sirer, \emph{Majority
  is not Enough: Bitcoin Mining is Vulnerable}, 2013.}, mais la plus
importante d'entre elles est l'attaque de double dépense par
recoordination de chaîne. Celle-ci consiste à utiliser une part
importante de la puissance de calcul du réseau (généralement une
majorité) afin de réécrire le passé de la chaîne et modifier une ou
plusieurs transactions. Cette attaque a été décrite précisément par
Satoshi Nakamoto dans le livre blanc\footnote{«~Nous considérons le
  scénario d'un attaquant qui tente de générer une chaîne alternative
  plus rapidement que la chaîne honnête. Même en cas de réussite, cela
  n'expose pas le système à des modifications arbitraires {[}...{]}. Un
  attaquant peut seulement essayer de modifier l'une de ses propres
  transactions afin de récupérer l'argent qu'il a récemment dépensé.~»
  -- Satoshi Nakamoto, \emph{Bitcoin: A Peer-to-Peer Electronic Cash
  System}, 31 octobre 2008.} et dans son courriel de réponse à John
Levine du 3 novembre 2008\footnote{«~Même si un individu malintentionné
  parvenait à maîtriser le réseau, ce n'est pas comme s'il devenait
  instantanément riche. Tout ce qu'il pourrait faire, c'est récupérer
  l'argent qu'il a lui-même dépensé, comme un chèque sans provision.
  Pour l'exploiter, il faudrait qu'il achète une chose à un commerçant,
  qu'il attende qu'elle soit expédiée, puis qu'il prenne le contrôle du
  réseau et essaie de récupérer son argent. Je ne pense pas qu'il puisse
  se faire autant d'argent en essayant de monter un tel stratagème qu'en
  générant des bitcoins. Avec une ferme de machines zombies aussi
  grande, il pourrait générer plus de bitcoins que tous les autres
  réunis.~» -- Satoshi Nakamoto, \emph{Re: Bitcoin P2P e-cash paper},
  /11/2008 16:23:49~:
  \url{https://www.metzdowd.com/pipermail/cryptography/2008-November/014818.html}.}.

Cette attaque est réalisée en trois étapes. Elle peut être faite à
l'aide d'une minorité de la puissance de calcul, auquel cas elle ne
possède qu'une certaine probabilité de réussir. Cependant, par souci de
simplicité, nous supposerons qu'un mineur a réuni la majorité de la
puissance de calcul du réseau. L'attaque constitue donc une attaque des
51~\%, aussi appelée attaque de la majorité.

La première étape est l'achat d'un bien ou d'un service auprès d'un
commerçant. L'attaquant procède à une transaction en bitcoins (dite
«~légitime~») en l'échange de quoi le commerçant lui fournit une chose
de même valeur. Typiquement, il s'agira d'une autre cryptomonnaie ou du
dollar auprès d'une plateforme de change.

La deuxième étape est le minage d'une chaîne parallèle. Une fois que la
transaction légitime a été confirmée au sein d'un bloc, l'attaquant
construit une chaîne parallèle en secret à partir du bloc précédent,
qu'il prend soin de ne pas dévoiler au reste du réseau. Dans le même
temps, il crée et signe une autre transaction (dite «~frauduleuse~») qui
dépense les mêmes bitcoins que la première et qui les renvoie vers une
adresse en son contrôle. Il inclut cette transaction frauduleuse dans sa
chaîne parallèle. Puisque l'attaquant dispose de la majorité de la
puissance de calcul du réseau, il est sûr qu'à un moment ou à un autre,
cette chaîne sera plus longue que l'autre.

La troisième étape est la recoordination de chaîne, représentée sur la
figure~\hyperref[fig:doublespending-attack]{8.5}. L'attaquant a continué
de miner sa chaîne parallèle jusqu'à la livraison du bien économique
acheté. À ce moment-là, il dévoile sa chaîne au reste du réseau, qui
doit accepter celle-ci en vertu du principe de la chaîne la plus longue.
Les nœuds procèdent alors à une recoordination~: les blocs de l'ancienne
chaîne sont écartés (rendus orphelins), leurs transactions sont remises
dans la mempool et les nouveaux blocs sont vérifiés et ajoutés à la
chaîne. Comme la transaction légitime dépense les mêmes fonds que la
transaction frauduleuse, qui est incluse dans la nouvelle chaîne, cette
transaction légitime est invalidée en tant que double dépense. Le
commerçant ne possède plus les bitcoins, qui reviennent à l'attaquant.

\begin{figure}

{\centering \includegraphics{chapters/img/mining-attack-doublespending.png}

}

\caption{Attaque de double dépense par recoordination de chaîne.}

\end{figure}%

Il s'agit d'une attaque opportuniste~: elle est motivée par un gain,
c'est-à-dire le bien économique obtenu, qui doit être supérieur au coût
(matériel, logistique, électrique et logiciel) nécessaire pour y
procéder. Sur le réseau Bitcoin principal, ce coût se chiffre
aujourd'hui en milliards de dollars\footnote{Braiins, \emph{How Much
  Would it Cost to 51\% Attack Bitcoin?}, 11 janvier 2021~:
  \url{https://braiins.com/blog/how-much-would-it-cost-to-51-attack-bitcoin}.}.

Cette attaque doit être distinguée de la censure, que nous décrirons
dans le chapitre~\hyperref[ch:censure]{9}, et qui consiste à refuser de
confirmer des transactions selon un critère arbitraire. Cette dernière
repose en effet sur des incitations \emph{extérieures} à l'économie de
Bitcoin, le mineur rationnel n'ayant, au sein du système, aucun intérêt
économique à ne pas inclure les transactions payant un taux de frais
suffisant dans ses blocs.

Comme souligné par Satoshi, le système est sécurisé tant que la majorité
de la puissance de calcul est associée à des nœuds honnêtes,
c'est-à-dire des nœuds qui ne cherchent pas à réaliser des doubles
dépenses, ni censurer. La sécurité minière repose donc sur une barrière
de sécurité, qui représente la charge financière de l'attaquant pour
réaliser une double dépense.

Cette barrière n'est pas construite de manière bénévole mais repose sur
la récompense du protocole, faisant du minage un procédé essentiellement
\emph{économique}. En particulier, la résistance à la double dépense --
à savoir la difficulté à effectuer une attaque de double dépense --
dérive directement du revenu minier total, qui incite les nœuds à rester
honnêtes. Tel que l'écrivait Satoshi dans le livre blanc~:

«~L'incitation peut contribuer à encourager les nœuds à rester honnêtes.
Si un attaquant cupide est capable de réunir plus de puissance de calcul
que l'ensemble des nœuds honnêtes, il aura à choisir entre l'utiliser
pour escroquer des gens en leur récupérant ses paiements, ou l'utiliser
pour générer de nouvelles pièces. Il devrait trouver plus rentable de
respecter les règles du jeu, celles-ci lui permettant d'obtenir plus de
nouvelles pièces que tous les autres réunis, plutôt que de saper le
système et la validité de sa propre richesse\footnote{Satoshi Nakamoto,
  \emph{Bitcoin: A Peer-to-Peer Electronic Cash System}, 31 octobre
  2008.}.~»

Non seulement la récompense peut être supérieure au gain d'une attaque
de double dépense, mais la valeur des bitcoins servant à réaliser la
transaction peut aussi être réduite par ladite attaque. En effet, si
l'attaque était amenée à être couronnée de succès, on peut imaginer que
les différents acteurs diminueraient leur confiance dans le système,
arrêteraient de l'utiliser pour le commerce et cèderaient une partie de
leur épargne, faisant baisser le revenu de minage et la valeur d'échange
du bitcoin. De plus, la spécialisation du matériel de minage (quand elle
existe) alourdit le coût de l'attaque, car ce matériel perd dans ce cas
en utilité. D'un point de vue purement opportuniste, il est donc la
plupart du temps bien plus rentable d'utiliser son capital de manière
honnête.

Il est ainsi arrivé que des agrégats de mineurs rassemblent plus de
51~\% de la puissance de calcul, comme la coopérative GHash.io en
juillet 2014, sans qu'aucune attaque ne se produise. Et même si une
telle attaque avait lieu, celle-ci ne serait pas forcément fatale pour
le système à long terme. Comme l'écrivait Satoshi~:

«~Même en cas de réussite, cela n'expose pas le système à des
modifications arbitraires, comme la création de valeur \emph{ex nihilo}
ou l'appropriation d'argent n'ayant jamais appartenu à l'attaquant. Les
nœuds ne vont pas accepter une transaction invalide comme paiement, et
les nœuds honnêtes n'accepteront jamais un bloc les
contenant\footnote{Satoshi Nakamoto, \emph{Bitcoin: A Peer-to-Peer
  Electronic Cash System}, 31 octobre 2008.}.~»

Ainsi de nombreuses attaques de ce type ont déjà eu lieu sur certaines
variantes de Bitcoin au fil des années, réduisant leur réputation au
passage, mais sans qu'elles ne soient pour autant anéanties. On peut
notamment citer Ethereum Classic qui a subi plusieurs recoordinations
agressives entre 2019 et 2020\footnote{«~Ethereum Classic qui a subi
  plusieurs recoordinations agressives entre 2019 et 2020~»~: Plus
  précisément~: le 7 janvier 2019, le 31 juillet 2020, le 6 août 2020 et
  le 29 août 2020.}.

\section*{L'industrie minière}\label{lindustrie-miniuxe8re}
\addcontentsline{toc}{section}{L'industrie minière}

\markright{L'industrie minière}

Le minage est une activité économique à part entière, la récompense de
minage servant à rémunérer le service apporté par le mineur. Cette
récompense paie pour le coût de l'électricité, de l'infrastructure
matérielle et logistique, et de la maintenance logicielle. Elle compense
le risque de production de blocs orphelins. Elle rémunère la
confirmation des transactions censurées. Et enfin elle récompense la
renonciation temporaire à la liquidité (intérêt originaire du prêteur)
et le risque économique général (profit de l'entrepreneur).

Du côté de l'infrastructure matérielle, les mineurs ont besoin de
déployer un certain nombre d'éléments~: les machines de hachage
(systèmes de refroidissement compris) pour procéder aux calculs liés à
la preuve de travail, le processeur pour traiter les blocs et vérifier
les signatures, la mémoire pour conserver la chaîne (l'historique),
l'ensemble des sorties transactionnelles non dépensées (l'état) et la
réserve des transactions en attente, la bande passante pour envoyer et
recevoir les transactions et les blocs,~etc. Et force est de constater
que tout cela s'est industrialisé au fur et à mesure des années.

L'amélioration de la machine pour procéder au hachage illustre bien
cette industrialisation. Initialement les mineurs minaient avec le
processeur central (CPU) de leur ordinateur. Puis, en 2010, sous
l'impulsion de Laszlo Hanyecz puis d'ArtForz, le minage par processeur
graphique (GPU) s'est développé. En 2011, est apparu le premier circuit
logique programmable FPGA consacré au minage, qui donnait un meilleur
rendement que les cartes graphiques\footnote{«~En 2011, est apparu le
  premier circuit logique programmable FPGA consacré au minage~»~:
  fpgaminer, \emph{Official Open Source FPGA Bitcoin Miner (Spartan-6
  Now Tops Performance per \$!)}, /05/2011 02:33:56 UTC~:
  \url{https://bitcointalk.org/index.php?topic=9047.msg130885\#msg130885}.}.
Enfin, en 2013, les premiers circuits intégrés spécialisés (ASIC) ont
été mis sur le marché, avec la sortie de l'Avalon ASIC\footnote{«~la
  sortie de l'Avalon ASIC~»~: ngzhang, \emph{"Avalon" ASIC, announcement
  \& pre-order}, /09/2012 07:48:26 UTC~:
  \url{https://bitcointalk.org/index.php?topic=110090.msg1197494\#msg1197494}.}.
À partir de là, les ASIC sont devenus de plus en plus performants,
notamment par le travail de l'entreprise chinoise Bitmain sur ses
Antminers.

Certains acteurs se sont mis à miner de manière industrielle en
entassant cette puissance de hachage dans des grands entrepôts
spécialisés contenant des centaines de machines, appelés des fermes de
minage. Ces fermes ont été installées dans des endroits suivant des
facteurs spécifiques dont notamment le coût de l'électricité, la
température (coût du refroidissement), la bande passante et
l'instabilité politique. Cette émergence de fermes de minage composées
d'appareils spécialisés avait été prévue par Satoshi qui écrivait dès
novembre 2008~:

«~Au début, la plupart des utilisateurs feront fonctionner des nœuds de
réseau, mais à mesure que le réseau grandira, au-delà d'un certain
point, cette tâche sera de plus en plus déléguée à des spécialistes
possédant des fermes de serveurs composées de matériel spécialisé. Une
ferme de serveurs n'aura besoin que d'un seul nœud sur le réseau et le
reste du réseau local sera connecté à ce nœud\footnote{Satoshi Nakamoto,
  \emph{Re: Bitcoin P2P e-cash paper}, /11/2008, 01:37:43 UTC~:
  \url{https://www.metzdowd.com/pipermail/cryptography/2008-November/014815.html}.}.~»

La puissance de calcul du réseau a par conséquent explosé. Le taux de
hachage\footnote{La puissance de hachage apparente \(P\) du réseau sur
  une période donnée peut être retrouvée grâce aux informations de la
  chaîne que sont la difficulté \(d\) et le temps de bloc moyen
  \(\Delta t\). La formule est~:
  \[P = \frac{T}{\Delta t} = \frac{1}{\Delta t} \left( \frac{2^{256}}{\frac{C_{\mathrm{max}}}{d} + 1} \right)~.\]
  où \(T\) est le travail d'un bloc et
  \(C_{\mathrm{max}} = \mathtt{0x00ffff} \times 256^{26}\) est la valeur
  cible maximale du réseau.}, mesuré en hachages par secondes (H/s), a
ainsi connu une spectaculaire croissance au cours des années. En 2009,
il oscillait entre 1 et 7 millions de hachage par seconde (1~MH/s).
Durant la première partie de 2010, il a progressé pour atteindre les
200~MH/s début juillet. Puis, il a connu deux hausses majeures
coïncidant avec les engouements spéculatifs mais aussi avec
l'utilisation de méthodes optimisées. La première a été celle de
2010--2011 où le prix est passé de moins d'un centime à 30~\$ et où les
premières fermes de cartes graphiques ont été utilisées~: entre juillet
2010 et août 2011, le taux de hachage est passé de 200~MH/s à 15~TH/s
(soit une multiplication par 75~000). La seconde a été celle de
2013--2014, période durant laquelle le prix a été quasiment multiplié
par 100 et où les premiers ASIC ont été déployés~: le taux de hachage
est passé de 25~TH/s en janvier 2013 à 300~PH/s en décembre 2014 (soit
une multiplication par 12~000). Le taux de hachage a enfin lentement
progressé pour atteindre environ 450~EH/s en novembre 2023 (ce qui
correspond à une multiplication par 1~500 depuis décembre 2014).

Avec cette croissance énorme de la puissance de calcul, la difficulté du
minage a suivi. Dès 2010, il devenait difficile d'espérer miner un bloc
avec le processeur de son ordinateur. Cela a eu pour effet de
désavantager les petits mineurs. L'augmentation de la difficulté a mis
en évidence un défaut inhérent du minage~: le défaut de variance.
Puisque le minage est soumis aux probabilités, le mineur individuel doté
d'un ASIC performant peut ne pas trouver de bloc du tout, tout comme il
peut trouver plus de blocs que prévu, faisant reposer son revenu sur le
hasard.

C'est pour corriger ce défaut de variance que sont nées les coopératives
de minage (appelées \emph{mining pools} en anglais). Ces dernières sont
des regroupements de hacheurs qui délèguent leur pouvoir sur la
sélection des transactions à un opérateur, afin de participer de manière
commune à l'effort de calcul et de lisser leurs revenus. Le
fonctionnement par coopératives se base sur la production de preuves de
travail partielles (PPoW) mis en place par le protocole Stratum. Il
s'agit pour le hacheur de produire une preuve de travail de degré
moindre pour un bloc candidat donné, afin de prouver qu'il a dépensé de
l'énergie et d'être rémunéré en conséquence par la coopérative. La
coopérative reçoit la récompense de minage à chaque fois qu'une preuve
de travail partielle produite par le hacheur s'avère être également une
preuve de travail complète (FPoW).

La première coopérative de minage a été lancée le 27 novembre 2010 par
Marek Palatinus (aussi connu sous le pseudonyme de \emph{slush}). Elle
portait initialement le nom de Bitcoin.cz Mining avant d'être plus tard
rebaptisée Slush Pool en hommage à son fondateur, puis de devenir
Braiins Pool en septembre 2022. Aujourd'hui, les coopératives de minage
sont nombreuses et concentrent l'essentiel de la puissance de calcul du
réseau. Elles sont généralement basées dans les juridictions où le
minage est très présent, comme la Chine (jusqu'en 2021) ou plus
récemment les États-Unis.

Les coopératives ont pour habitude de signaler les blocs qu'elles minent
dans un souci de transparence. Par exemple, la transaction de récompense
du bloc 751~005 contient la chaîne de caractères ``, ce qui indique que
ce bloc a très probablement été validé par la coopérative chinoise
Poolin. Ce signalement n'est pas obligatoire (le minage est anonyme par
essence), mais permet d'avoir une idée de la répartition des différentes
coopératives (comme on peut le voir sur la
figure~\hyperref[fig:hashrate-distribution]{8.6}) et d'estimer par
conséquent la centralisation de l'activité minière.

\begin{figure}

{\centering \includegraphics{chapters/img/hashrate-distribution-coin-dance-week-20231012.png}

}

\caption{Répartition du taux de hachage apparent entre les coopératives
de minage de BTC, semaine du 5 au 12 octobre 2023. (source~:
coin.dance)}

\end{figure}%

Un autre défaut inhérent du minage est la latence liée à l'annonce des
blocs. Comme expliqué dans la section sur la chaîne la plus longue,
cette latence produit des blocs orphelins, qui sont valides mais ne sont
pas rattachés à la chaîne principale. Cela fait que des mineurs mal
connectés ont une puissance de hachage apparente inférieure à leur
puissance de hachage réelle.

Pour tenter d'atténuer les effets de ce défaut, les mineurs ont mis en
place des relais de communication permettant de s'envoyer des blocs
mutuellement de manière plus efficace en supprimant les protections
contre le déni de service nécessaires sur le réseau pair à pair ouvert.

Le premier relai a été créé par Matt Corallo sous le nom de
\emph{Bitcoin Relay Network}. Il a été lancé en 2013\footnote{«~Le
  premier relai {[}...{]} a été lancé en 2013~»~:
  \url{https://lists.linuxfoundation.org/pipermail/bitcoin-dev/2013-November/003596.html}.}
et est devenu pleinement fonctionnel en 2015\footnote{«~{[}il{]} est
  devenu pleinement fonctionnel en 2015~»~:
  \url{https://web.archive.org/web/20150628233706/https://bitcoinrelaynetwork.org/}.}.
Le réseau était composé de plusieurs nœuds spécialisés hébergés sur
l'infrastructure Amazon Web Services. Un concurrent était le réseau
Falcon, géré par une équipe de l'université Cornell dirigée par Emin Gün
Sirer\footnote{«~Un concurrent était le réseau Falcon~»~:
  \url{https://web.archive.org/web/20160609081540/https://www.falcon-net.org/}}.
Le Bitcoin Relay Network a été remplacé en 2016 par le réseau
FIBRE\footnote{Matt Corallo, \emph{The Future of The Bitcoin Relay
  Network(s)}, 7 juillet 2016~:
  \url{https://bluematt.bitcoin.ninja/2016/07/07/relay-networks/}.}
(pour \emph{Fast Internet Bitcoin Relay Engine}), un réseau basé sur UDP
(protocole alternatif à TCP) qui implémente l'optimisation
\texttt{cmpctblock}, toujours géré par Matt Corallo. C'est ce réseau qui
est utilisé par la plupart des mineurs aujourd'hui.

Cette industrialisation du minage a mené à la centralisation de
l'activité minière, à la fois au niveau de la puissance de hachage
(fermes de minage) que de la sélection des transactions (coopératives et
relais). Si cette agrégation n'est pas fatale (les hacheurs sont libres
de quitter leur coopérative pour une autre et les mineurs sont libres de
ne pas utiliser le relai), elle n'en diminue pas moins la sécurité
minière de la chaîne.

Certaines améliorations ont été proposées dans le minage pour corriger
ce problème. La première est le protocole P2Pool, qui est un protocole
de minage coopératif basé sur un réseau pair à pair de
mineurs\footnote{«~P2Pool qui est un protocole de minage coopératif basé
  sur un réseau pair à pair de mineurs~»~: \url{http://p2pool.in/}.}.
Celui-ci met en communication les hacheurs en se basant sur une chaîne
latérale -- la «~chaîne de partage~» -- dont la difficulté est plus
faible et qui regroupe les différentes contributions des participants.
Le développement de P2Pool sur la version principale de Bitcoin semble
avoir été mis en suspens en 2017. Toutefois, le procédé est mis en œuvre
sur Monero depuis octobre 2021 au moyen d'une coopérative du même
nom\footnote{«~le procédé est mis en œuvre sur Monero~»~:
  \url{https://p2pool.io/}.}.

Le seconde est le protocole Stratum V2\footnote{Braiins, \emph{Stratum
  V2 Protocol Overview}~:
  \url{https://braiins.com/stratum-v2\#job-selection}.}, qui permet
(entre autres) aux hacheurs de négocier le contenu des blocs. À défaut
de corriger complètement la délégation sur la sélection des
transactions, cette nouvelle version de Stratum a le mérite de rendre le
processus plus transparent. En novembre 2023, elle n'était déployée
qu'au sein de la coopérative Braiins Pool (anciennement Slush Pool), qui
est à l'origine de sa conception.

Cependant, ces propositions d'amélioration, bien que louables, ne
suppriment pas l'avantage économique découlant de la centralisation, qui
se retrouve par ailleurs dans toutes les industries (économie
d'échelle). La décentralisation a un coût, et celui-ci ne sera justifié
que lorsque le bénéfice apporté le surpassera, c'est-à-dire le jour où
le réseau sera réellement attaqué.

\section*{Un algorithme de consensus
novateur}\label{un-algorithme-de-consensus-novateur}
\addcontentsline{toc}{section}{Un algorithme de consensus novateur}

\markright{Un algorithme de consensus novateur}

Pour fonctionner en tant que système distribué de monnaie numérique,
Bitcoin se base sur un mécanisme de consensus novateur. Celui-ci met en
jeu une chaîne de blocs construite par les mineurs, qui sont rémunérés
pour leur travail. Chaque bloc est un ensemble horodaté de transactions,
qui contient une preuve de travail quantifiant l'énergie dépensée. Le
consensus est atteint par la sélection de la plus longue chaîne.

Cet algorithme de consensus a un fonctionnement objectif, ouvert et
robuste, ce qui explique le succès de Bitcoin par rapport à ses
prédécesseurs. Par son aspect essentiellement économique, il donne au
système une très grande résistance à la double dépense opportuniste,
notamment grâce à la gigantesque industrie minière qui le soutient.

Il existe cependant une menace plus importante, plus insidieuse~: celle
de la censure, dont nous parlerons dans le prochain chapitre.

\bookmarksetup{startatroot}

\chapter{La résistance à la censure}\label{ch:censure}

\phantomsection\label{enotezch:9}{}

{L}\textsc{\textquotesingle{}}un des problèmes croissants de notre
époque est la censure financière. Avec le développement de l'économie
mondialisée, reposant notamment sur Internet, le recours aux
intermédiaires financiers est devenu de plus en plus courant. Cette
évolution fait que l'entrave de transferts monétaires constitue
aujourd'hui une complication générale, expérimentée par une part
grandissante de la population.

Bitcoin forme une solution à ce problème. L'une de ses caractéristiques
primordiales est en effet sa résistance à la censure, c'est-à-dire le
fait qu'il est difficile pour une entité quelconque d'empêcher la
réalisation d'un paiement. En permettant «~aux paiements en ligne d'être
envoyés directement d'une partie à l'autre sans passer par une
institution financière~», Bitcoin contourne l'arsenal de contrôles
financiers qui gangrènent nos moyens de paiement et d'épargne modernes.

La résistance à la censure est, comme la confirmation des transactions,
un mécanisme économique. Elle se fonde de manière essentielle sur la
preuve de travail ainsi qu'elle est appliquée dans l'algorithme de
consensus de Nakamoto. De ce fait, les alternatives proposées comme les
algorithmes de preuve d'enjeu montrent une résistance à la censure bien
plus faible.

Dans ce chapitre, nous verrons d'abord comment la censure financière
intervient dans le monde bancaire aujourd'hui et pourquoi elle devrait
se généraliser à l'avenir avec le déploiement des monnaies numériques de
banque centrale. Puis, nous décrirons de quelle façon la censure peut
s'exercer dans Bitcoin et comment le système peut y résister. Nous
expliquerons enfin en quoi les propositions alternatives ne suffisent
pas.

\section*{Qu'entendons-nous par censure
financière~?}\label{quentendons-nous-par-censure-financiuxe8re}
\addcontentsline{toc}{section}{Qu'entendons-nous par censure
financière~?}

\markright{Qu'entendons-nous par censure financière~?}

La notion de censure peut paraître étrange de prime abord quand on parle
de monnaie. Au sens courant, la censure désigne la restriction de
l'expression, notamment par l'interdiction de la diffusion de certaines
idées. Néanmoins, il est possible de la comprendre dans un sens plus
large, qui mêle paiement et expression.

Le terme de censure vient du latin \emph{censeo} signifiant «~évaluer~»,
«~estimer~», «~déclarer~», «~juger~». Il trouve son origine dans une
institution importante de la République romaine, celle des censeurs,
deux magistrats qui avaient pour charge de procéder au dénombrement des
citoyens et de leurs biens (le \emph{census}), de collecter les impôts,
de superviser les travaux publics, de gérer la liste des personnes
admises au Sénat (l'\emph{album senatorium}) et de veiller au maintien
des «~bonnes mœurs~» de la population en administrant des blâmes ou des
peines temporaires. La première fonction des censeurs a donné sa
signification au mot recensement. La seconde aux concepts de cens et de
suffrage censitaire. Et la dernière a été à l'origine de ce que nous
appelons la censure.

Au Moyen Âge, le mot latin \emph{censura} a été repris par le
catholicisme pour prendre un sens religieux et se limiter ainsi au
discours, et en particulier aux textes. La censure s'apparentait alors à
un blâme (sens encore parfois employé, notamment en matière de critique
littéraire) ou à une interdiction. Elle se caractérisait par la
relecture et la correction des ouvrages rédigés pour s'assurer que tout
était conforme au dogme de l'Église catholique romaine.

Néanmoins, l'apparition de l'imprimerie au \textsc{xv}~siècle a
bouleversé les choses~: le nombre de livres a explosé, et ce faisant, a
retiré le contrôle que la hiérarchie catholique avait sur la publication
des écrits, contrôle qui a été transféré à l'État. La censure a par
conséquent acquis son sens politique actuel, en désignant l'examen que
le pouvoir étatique fait préalablement des livres, journaux, pièces de
théâtre,~etc., pour en permettre ou en prohiber la publication ou la
représentation. Par la suite, le terme a fini par nommer toute atteinte
à la liberté d'expression, quel que soit le support, que cela se fasse
avant (censure a priori) ou après la diffusion (censure a posteriori).

Avec le développement des médias de masse (journaux, radio, télévision)
et surtout des médias sociaux, le terme a acquis un sens élargi et on
s'est mis à parler de censure pour tout choix d'édition pris par une
entité privée vis-à-vis de ses clients ou de ses utilisateurs. Cette
censure privée n'est pas une atteinte à la liberté d'expression au sens
strict, mais elle pose problème lorsque le domaine est monopolisé par un
petit nombre d'acteurs bénéficiant souvent d'un avantage légal ou d'une
subvention étatique. De plus, cette censure peut être directement
l'émanation d'une intervention politique, la plateforme en question ne
faisant qu'appliquer les directives générales du pouvoir\footnote{Voir
  par exemple l'affaire des Twitter Files qui a révélé les manœuvres
  internes et l'intervention de l'État fédéral des États-Unis dans la
  politique de censure de la plateforme. -- Evan Perez, Donie
  O'Sullivan, Brian Fung, «~\emph{No directive: FBI agents, tech
  executives deny government ordered Twitter to suppress Hunter Biden
  story}~», \emph{CNN}, 23 décembre 2022~:
  \url{https://edition.cnn.com/2022/12/23/politics/twitter-files-elon-musk-fbi-hunter-biden-laptop/index.html}.}.

Cependant, cette censure de l'expression peut également être réalisée
par l'atteinte de l'activité économique de celui qui s'exprime. En
effet, en restreignant la capacité à gagner de l'argent d'une personne
et en lui faisant comprendre que son discours pose problème, on peut
l'amener à taire ce discours. C'est dans ce contexte qu'a émergé le
concept de censure financière, ou \emph{financial censorship} en
anglais, que l'organisation internationale \emph{Students for Liberty}
définit comme le fait de «~restreindre l'activité financière d'une
entité privée, de manière à inhiber ses opérations, avec l'intention
implicite de la réduire au silence\footnote{Students for Liberty,
  \emph{Financial Censorship}~:
  \url{https://studentsforliberty.org/blog/freedom-of-expression/financial-censorship/}.}~».
C'est aussi le sens que lui donne l'\emph{Electronic Frontier
Foundation}\footnote{«~C'est aussi le sens que lui donne
  l'\emph{Electronic Frontier Foundation}~»~: Electronic Frontier
  Foundation, \emph{Financial Censorship}~:
  \url{https://www.eff.org/issues/financial-censorship}.}.

Mais les répercussions du contrôle financier ne s'arrêtent pas à
l'expression et peuvent concerner l'action humaine en général. Ainsi, la
censure financière peut être saisie dans un sens plus large, une
signification par exemple adoptée par trois chercheurs de l'université
d'État de San José qui affirment que «~la censure financière se produit
lorsqu'une institution financière refuse ses services à une partie en
raison des opinions exprimées, des actions ou du secteur d'activité de
cette partie\footnote{Marco Pagani, George Whaley, David Czerwinski,
  «~\emph{Frameworks for Assessing Financial Censorship and Its
  Implications}~», in \emph{Journal of Accounting and Finance}, vol.~22,
  no.~1, 2022~:
  \url{https://articlegateway.com/index.php/JAF/article/download/4989/4759}.}~».

Enfin, on peut comprendre la censure financière comme la restriction
financière elle-même à condition qu'elle repose sur un critère subjectif
externe (respect de normes arbitraires) et non pas sur une donnée
économique objective, comme par exemple le paiement d'une commission. La
censure peut être appliquée de manière publique (interdiction légale
d'une transaction), privée (par une banque par exemple) ou les deux.
Cette définition conserve toujours en elle l'idée de modeler le
comportement extérieur de la personne par l'intervention sur ses
finances. C'est notamment cette signification qui est donnée à la
censure dans Bitcoin.

Au sens général, la censure financière consiste donc à restreindre
directement l'activité financière d'une entité de façon à inhiber son
expression ou son action. L'idée est d'influencer l'individu par le
contrôle sur la monnaie dont il se sert, un outil qui est essentiel à sa
survie économique. Aujourd'hui, la censure s'applique essentiellement au
crédit bancaire, dont les transferts sont hautement réglementés par le
pouvoir. Demain, elle pourra concerner la monnaie numérique gérée par la
banque centrale.

\section*{La banque et la censure}\label{la-banque-et-la-censure}
\addcontentsline{toc}{section}{La banque et la censure}

\markright{La banque et la censure}

La censure financière s'exerce par la maîtrise sur le transfert de
monnaie, de sorte que cette censure peut difficilement s'appliquer à
l'argent liquide physique. En effet, ce dernier (qu'il prenne la forme
de pièces de métal précieux ou de billets fiduciaires) permet l'échange
direct et confidentiel de personne à personne, ce qui empêche la mise en
place de toute restriction en dehors de quelques cas particuliers.

En revanche, dans le domaine bancaire, le client dispose d'un compte
courant sur lequel la banque inscrit les crédits et gère les transferts.
La restriction financière est de ce fait beaucoup plus simple~: la
banque peut sélectionner les transferts, geler le compte momentanément
et même refuser le retrait d'argent. C'est aussi le cas de tout les
services construits au-dessus du système bancaire traditionnel, comme
PayPal.

C'est donc tout naturellement que l'accroissement de la censure
financière a coïncidé avec la bancarisation de la société, qui a eu lieu
à partir des années 1960 en Occident\footnote{«~la bancarisation de la
  société, qui a eu lieu à partir des années 1960 en Occident~»~:
  \url{https://books.openedition.org/pur/121053?lang=fr}~;
  \url{https://www.the-american-interest.com/2019/02/25/bigger-fewer-riskier-the-evolution-of-u-s-banking-since-1950/}.},
et qui s'est caractérisée par la généralisation de l'usage du compte
courant et des moyens de paiement apparentés comme le chèque bancaire,
la carte de crédit et le virement. En quelques décennies, le paiement a
migré vers le domaine bancaire, favorisé par la loi et bien plus commode
à utiliser que les espèces, dont l'utilisation a elle-même été
restreinte légalement. D'où la meilleure efficacité de la censure~: si
le liquide ne permet plus de gérer ses affaires convenablement, alors la
possibilité de se retirer complètement du système bancaire n'est plus
une option viable.

Cette censure a été mise en place par l'intermédiaire de la surveillance
financière, qui est aujourd'hui particulièrement fréquente dans
l'industrie bancaire. Les banques ont en effet l'obligation légale de
surveiller leurs clients et d'intervenir dans le cas où elles constatent
un comportement «~suspect~» de leur part, en empêchant leurs virements
ou en gelant leurs comptes. Elles ne font pas cela de gaieté de cœur~:
elles ne procèdent pas à la surveillance de leurs clients pour les
«~protéger~», mais pour se protéger elles-mêmes contre les éventuelles
complications liées à la réglementation.

Cette réglementation s'est développée à mesure que l'activité bancaire
se popularisait. À partir des années 70, le prétexte de la lutte contre
le blanchiment d'argent (notamment dans le cadre de la guerre contre la
drogue) s'est imposé comme le principal prétexte derrière les
restrictions imposées aux banques. Aux États-Unis notamment, la
réglementation bancaire s'est particulièrement durcie suite à l'adoption
du \emph{Bank Secrecy Act} de 1970, qui se proposait de lutter contre le
blanchiment d'argent.

Puis, avec l'apparition du web dans les années 1990, l'utilisation des
banques internationales a demandé une réglementation accrue. Différents
organismes de surveillance ont ainsi été créés. Le Groupe d'action
financière (GAFI), un organisme intergouvernemental émettant
régulièrement des recommandations de normes réglementaires et de
sanctions économiques, a été créé en juillet 1989 dans le but de lutter
contre le blanchiment d'argent. Le Financial Crimes Enforcement Network
(FinCEN), le bureau du département du Trésor des États-Unis qui collecte
et analyse les informations sur les transactions financières, a été
formé dans ce sens le 25 avril 1990. L'équivalent français, la cellule
TRACFIN (Traitement du renseignement et action contre les circuits
financiers clandestins), est apparu en juillet 1990. Du côté européen,
la première directive de l'Union Européenne relative à la prévention de
l'utilisation du système financier aux fins du blanchiment de capitaux
est datée du 10 juin 1990\footnote{«~la première directive de l'Union
  Européenne relative à la prévention de l'utilisation du système
  financier aux fins du blanchiment de capitaux~»~:
  \url{https://eur-lex.europa.eu/legal-content/FR/TXT/PDF/?uri=CELEX:31991L0308&from=FR}.}.

Enfin, après les attentats islamistes du 11 septembre 2001, un autre
prétexte est apparu~: la lutte contre le financement du terrorisme.
Celle-ci s'est matérialisée aux États-Unis par l'adoption du
\emph{PATRIOT Act} en octobre 2001, dont le Titre 3 concerne les
restrictions financières. En France, la loi du 15 novembre 2001 relative
à la sécurité quotidienne a requalifié «~le fait de financer une
entreprise terroriste~» comme un acte de terrorisme en
lui-même\footnote{Code pénal, Article 421-2-2, 15 novembre 2001.}. La
surveillance financière s'est renforcée en conséquence.

Ces deux évolutions forment la base de ce qu'on appelle généralement la
lutte contre le blanchiment des capitaux et le financement du terrorisme
(LCB-FT) en France et les normes AML/CFT (pour \emph{Anti-Money
Laundering/Combating the Financing of Terrorism}) aux États-Unis. Ce
resserrement se caractérise notamment par la connaissance du client
(\emph{Know Your Customer} ou KYC), une pratique également appelée
vigilance à l'égard de la clientèle, qui consiste à vérifier l'identité,
la conformité et les risques liés à chaque client. Cette exigence
d'identification s'est insérée dans tous les services financiers
aujourd'hui.

En conséquence, le secret bancaire, c'est-à-dire l'obligation pour les
banques de ne pas livrer des informations sur leurs clients à des tiers,
a fini par disparaître\footnote{«~le secret bancaire {[}...{]} a fini
  par disparaître~»~: Anthony Amicelle, Jean Bérard, \emph{Vers la fin
  du secret bancaire ou de la vie privée~?}, 2019~:
  \url{https://journals.openedition.org/conflits/21291}.}, y compris en
Suisse\footnote{«~y compris en Suisse~»~: Mathilde Damgé, \emph{Comment
  la Suisse a renoncé au secret bancaire}, 11 février 2015~:
  \url{https://www.lemonde.fr/evasion-fiscale/article/2015/02/11/comment-la-suisse-a-renonce-au-secret-bancaire_4572485_4862750.html}.}.
L'usage d'un compte bancaire aujourd'hui présuppose la surveillance
générale des transactions et l'inspection minutieuse des opérations les
moins usuelles. Ainsi, il est aujourd'hui impossible de virer une
importante somme d'argent d'un compte à un autre sans devoir fournir une
justification.

Cette situation du domaine financier a été résumée en janvier 2009 par
Jonathan Thornbug sur la liste de diffusion en réponse à Satoshi
Nakamoto qui décrivait les utilisations qu'on pouvait faire de Bitcoin~:

«~Dans le monde moderne, aucun État important ne veut autoriser des
transactions financières internationales intraçables au-delà d'un
certain seuil de taille assez modeste. (Les phrases d'accroche
habituelles sont des choses comme ``blanchiment de l'argent de la
drogue'', ``évasion fiscale'', et/ou ``financement de groupes
terroristes''). À cette fin, les transactions financières électroniques
sont actuellement surveillées par divers États et leurs agences, et
toutes les transactions, sauf les plus petites, sont désormais assorties
de diverses exigences en matière d'identification des personnes à chaque
extrémité\footnote{Jonathan Thornbug, \emph{Re: Bitcoin v0.1 released},
  /01/2009 16:49:45 UTC~:
  \url{https://www.metzdowd.com/pipermail/cryptography/2009-January/015016.html}.}.~»

\section*{Les cas de censure
financière}\label{les-cas-de-censure-financiuxe8re}
\addcontentsline{toc}{section}{Les cas de censure financière}

\markright{Les cas de censure financière}

Au cours des dernières années, les cas célèbres de censure financière se
sont multipliés, à tel point qu'il est impossible d'en faire une liste
exhaustive. Nous nous contenterons d'en citer les exemples les plus
manifestes en Occident, tout en gardant en tête que cette censure n'est
généralement pas rendue publique par ceux qui la subissent.

L'exemple le plus connu est probablement le blocus financier contre
WikiLeaks mis en place par Mastercard, Visa, Western Union, Bank of
America et d'autres acteurs, en décembre 2010, dans le but de faire
taire l'organisation. En octobre 2011, un communiqué de WikiLeaks a
indiqué que le blocus financier avait fait disparaître 95~\% de ses
revenus\footnote{«~le blocus financier avait fait disparaître 95~\% de
  ses revenus~»~: WikiLeaks, \emph{Banking Blockade}, /10/2011 13:00
  UTC, \url{https://wikileaks.org/Banking-Blockade.html}.}. Cette
affaire a eu des répercussions directes dans l'histoire de Bitcoin,
comme nous l'avons raconté dans le chapitre~\hyperref[ch:mythe]{1}.

Un autre cas, qui visait cette fois la profession des personnes
censurées, est l'opération Choke Point mise en place entre 2013 et 2017
par le département de la Justice des États-Unis\footnote{«~opération
  Choke Point mise en place entre 2013 et 2017 par le département de la
  Justice des États-Unis~»~:
  \url{https://en.wikipedia.org/wiki/Operation_Choke_Point}~;
  \url{https://www.wsj.com/articles/SB10001424127887323838204578654411043000772},
  archive~: \url{https://archive.is/bF8KZ}.}. L'opération avait pour but
d'«~étouffer~» certains secteurs d'activité en restreignant leur accès
au crédit et à d'autres services bancaires. Ces activités jugées «~à
haut risque~» incluaient le prêt sur gages ou sur salaire, le jeu
d'argent, la pornographie, l'escorting, mais aussi la vente de tabac et
de produits pharmaceutiques, la vente de pièces de monnaie, les services
de rencontre ou encore l'organisation des clubs de voyage. La vente
d'armes et de munitions était aussi concernée~: Defense Distributed,
l'entreprise du crypto-anarchiste libertarien Cody Wilson, spécialisée
dans la diffusion de schémas de conception d'armes à feu fabriquées par
imprimante 3D, en a fait les frais en 2015 en subissant une fermeture de
ses comptes par Chase, PayPal et Stripe\footnote{«~en subissant une
  fermeture de ses comptes par Chase, PayPal et Strip~»~: Kelsey Bolar,
  \emph{Firearms Sellers Say They're Being Choked Off From Payment
  Processors}, 12 janvier 2015~:
  \url{https://www.dailysignal.com/2015/01/12/firearms-sellers-say-theyre-choked-off-payment-processors/}.}.

En 2018, c'est l'opinion politique qui a dû endurer la censure. De
nombreuses personnalités et organisations d'\emph{alt-right} américaine
ont ainsi été bannies des divers réseaux sociaux et ont perdu l'accès à
divers services financiers. L'exemple le plus emblématique était Alex
Jones, fondateur du site de réinformation InfoWars, qui, outre sa purge
des médias sociaux durant l'été 2018, a vu son compte PayPal être
clôturé\footnote{«~Alex Jones {[}...{]} a vu son compte PayPal être
  clôturé~»~: Brian Fung, \emph{PayPal bans Alex Jones, saying Infowars
  `promoted hate or discriminatory intolerance'}, 21 septembre 2018~:
  \url{https://www.washingtonpost.com/technology/2018/09/21/paypal-bans-alex-jones-saying-infowars-promoted-hate-or-discriminatory-intolerance/}}.
On peut aussi citer les cas du média social Gab (chassé de PayPal,
Stripe Cash App et Coinbase\footnote{«~chassé de PayPal, Stripe Cash App
  et Coinbase~»~:
  \url{https://www.bitcoininsider.org/article/44690/after-coinbase-paypal-bans-social-media-platform-gab-just-because}.}),
de Milo Yiannopoulos (banni de PayPal pour avoir fait un salut
nazi\footnote{«~banni de PayPal pour avoir fait un salut nazi~»~:
  \url{https://www.timesofisrael.com/paypal-suspends-milo-yiannopoulos-over-nazi-based-trolling-of-jewish-journalist/}.})
ou encore de Robert Spencer (chroniqueur du blog anti-islam Jihad Watch,
chassé de Patreon suite à la pression de Mastercard\footnote{«~chassé de
  Patreon suite à la pression de Mastercard~»~:
  \url{https://twitter.com/Patreon/status/1029551216886341634}.}). En
France, cette censure s'est manifestée à l'encontre d'Égalité et
Réconciliation, l'association de l'antisioniste Alain Soral, qui a été
exclue de PayPal en août 2018, dans le cadre d'une purge similaire à
celle des militants américains. L'association a également vu plusieurs
de ses comptes bancaires (Banque postale, BNP Paribas, Banque populaire)
être fermés au cours des années\footnote{«~L'association a également vu
  plusieurs de ses comptes bancaires (Banque postale, BNP Paribas,
  Banque populaire) être fermés~»~: Égalité et Réconciliation,
  \emph{Soutenez-nous~: la Banque populaire ferme le compte en banque
  d'Égalité \& Réconciliation}, 6 février 2022~:
  \url{https://www.egaliteetreconciliation.fr/Soutenez-nous-la-Banque-populaire-ferme-le-compte-en-banque-d-Egalite-Reconciliation-67155.html}.}.

Toujours dans le domaine politique, mais en Chine cette fois-ci, on peut
citer le cas du mouvement contre l'amendement de la loi d'extradition
par le gouvernement de Hong Kong, série de manifestations ayant eu lieu
entre mars 2019 et juillet 2020, qui a dû subir les interventions du
conglomérat bancaire international HSBC, probablement sous pression de
l'État central chinois. En novembre 2019, la filiale de Hong Kong a en
effet décidé de fermer un compte utilisé pour soutenir le mouvement de
protestation. Puis, elle a gelé le compte du démocrate Ted Hui en
décembre 2020\footnote{«~elle a gelé le compte du démocrate Ted Hui en
  décembre 2020~»~:
  \url{https://hongkongfp.com/2020/12/07/hsbc-re-freezes-accounts-belonging-to-family-of-exiled-democrat-ted-hui-amid-hong-kong-police-money-laundering-probe/}.}.
Par ailleurs, on a appris en 2023 qu'elle refusait aux Hongkongais ayant
fui au Royaume-Uni d'accéder légitimement à leurs fonds de pension, pour
un montant s'élevant à 2,2 milliards de livres sterling\footnote{«~elle
  refusait aux Hongkongais ayant fui au Royaume-Uni d'accéder
  légitimement à leurs fonds de pension, pour un montant s'élevant à 2,2
  milliards de livres sterling~»~:
  \url{https://www.lefigaro.fr/flash-eco/hsbc-complice-de-violation-des-droits-humains-a-hong-kong-selon-un-rapport-parlementaire-20230208},
  \url{https://www.telegraph.co.uk/business/2023/08/07/hsbc-exececutive-apologises-calling-uk-weak-on-china/}.}.

Plus récemment, la pandémie de Covid-19 a fourni d'autres occurrences de
censure financière. De nombreux activistes opposés aux mesures
coercitives comme le confinement, le port du masque et la vaccination
obligatoire, ont ainsi été largement censurés, généralement accusés de
propager la désinformation. Le groupe d'action néerlandais Viruswaarheid
-- s'opposant à la distanciation sociale, au confinement, au couvre-feu
et au programme de vaccination -- a ainsi vu son compte bancaire utilisé
pour recevoir des donations être fermé par ING Bank en février
2021\footnote{«~Viruswaarheid {[}...{]} a ainsi vu son compte bancaire
  utilisé pour recevoir des donations être fermé par ING Bank en février
  2021~»~: Andreas Kouwenhoven et Wilmer Heck, \emph{De complotdenker
  bankiert maar elders, zegt de bank}, 17 août 2021~:
  \url{https://www.nrc.nl/nieuws/2021/08/17/de-complotdenker-bankiert-maar-elders-zegt-de-bank-a4055125}~;
  archive~: \url{https://archive.is/8LI0k}.}.

Mais l'exemple qui ressort du lot est le mouvement canadien du «~Convoi
de la liberté~» de février 2022, initié par les camionneurs qui
s'opposaient à l'obligation vaccinale imposée pour entrer sur le
territoire par voie terrestre et qui ont manifesté leur mécontentement
en faisant route jusqu'à Ottawa pour occuper la ville. Ce mouvement a
fait face à une censure financière drastique. Il a dans un premier temps
été victime des plateformes de financement participatif, qui ont annulé
ses différentes campagnes qui avaient pour objectif de payer le
déplacement des camionneurs~: celle de GoFundMe, ayant réuni 10 millions
de dollars canadiens, a été retirée le 4 février\footnote{«~celle de
  GoFundMe, ayant réuni 10 millions de dollars canadiens, a été retirée
  le 4 février~»~: Radio-Canada, \emph{La campagne de sociofinancement
  du convoi des camionneurs retirée de GoFundMe}, 4 février 2022~:
  \url{https://ici.radio-canada.ca/nouvelle/1859918/retrait-campagne-gofundme-convoi-camionneurs-2022}.}~;
tandis que les fonds récupérés par les campagnes organisées sur la
plateforme chrétienne GiveSendGo (9 millions de dollars environ) ont été
gelés par le gouvernement ontarien, et n'ont pas pu être
distribués\footnote{«~les fonds récupérés par les campagnes organisées
  sur la plateforme chrétienne GiveSendGo {[}...{]} ont été gelés par le
  gouvernement ontarien~»~: Stephanie Taylor, \emph{Ontario court
  freezes access to donations for truckers' protest from GiveSendGo}, 10
  février 2022~:
  \url{https://www.ctvnews.ca/canada/ontario-court-freezes-access-to-donations-for-truckers-protest-from-givesendgo-1.5776674}.}.
La répression financière s'est considérablement amplifiée lorsque, suite
à l'entrée en vigueur de l'état d'urgence déclaré par Justin Trudeau le
14 février, le gouvernement canadien a décidé de geler des comptes
bancaires personnels ou professionnels en lien avec le mouvement~: 280
comptes contenant 8 millions de dollars au total ont été gelés de le
sorte\footnote{«~280 comptes contenant 8 millions de dollars au total
  ont été gelés de le sorte~»~: Bill Curry, Marsha McLeod, \emph{Deputy
  Minister of Finance describes race against time to prevent economic
  damage from border blockades}, 17 novembre 2022~:
  \url{https://www.theglobeandmail.com/politics/article-emergencies-act-inquiry-michael-sabia/}.}.
L'année suivante, le juge Paul Rouleau, en charge de la Commission sur
l'état d'urgence, a déclaré que le gel des comptes bancaires était un
«~outil puissant pour décourager la participation {[}aux
manifestations{]} et inciter les manifestants à abandonner\footnote{Rob
  Gillies, «~\emph{Judge: Canada right to invoke emergency act in truck
  protest}~», \emph{Associated Press News}, 17 février 2023~:
  \url{https://apnews.com/article/canada-government-justin-trudeau-ottawa-montana-9c1e37aa86d4315703e69f7794637e7f}.}~».

Un autre évènement important survenu durant le mois de février est le
durcissement des sanctions économiques mises en place par les États
occidentaux contre la Russie, suite à son invasion de
l'Ukraine\footnote{Les sanctions économiques internationales qui
  concernent le domaine financier rentrent dans la catégorie de la
  censure financière. Celles-ci ont en effet pour but premier d'empêcher
  le commerce avec la population gouvernée par un État ennemi. Le cas
  des Russes n'est pas un cas isolé, et de nombreuses autres populations
  n'ont pas accès aux services financiers disponibles pour les
  Occidentaux, comme les Palestiniens par exemple. Voir à ce sujet
  Electronic Frontier Foundation, \emph{Why Is PayPal Denying Service to
  Palestinians?}, 12 octobre 2021~:
  \url{https://www.eff.org/deeplinks/2021/10/why-paypal-denying-service-palestinians}.}.
Les sanctions financières incluaient l'exclusion de certaines banques
russes du système SWIFT, la prohibition du financement en Russie et de
l'achat de roubles, et l'interdiction de la fourniture de services de
portefeuille, de compte ou de conservation de crypto-actifs. De manière
générale, les virements vers la Russie ont été interdits, de sorte que
les citoyens russes exilés ne pouvaient plus envoyer d'argent à leur
famille. C'est aussi le cas des ressortissants ukrainiens dont les
proches sont restés sur le territoire occupé par l'armée russe, comme
cette Ukrainienne réfugiée en France qui ne pouvait pas envoyer un
virement bancaire de 100~euros à ses parents à Donetsk\footnote{Ben
  Canton, \emph{Un an de guerre en Ukraine~: la petite histoire de
  Valériia et de Binance}, 24 février 2023~:
  \url{https://journalducoin.com/analyses/un-an-guerre-ukraine-petite-histoire-valeriia-binance/}.}.

Du côté occidental, des mesures financières ont également été prises
dans le but de faire respecter la censure des médias financés par le
Kremlin. En janvier 2023, la chaîne d'information RT France, qui était
déjà interdite de diffusion en Europe, mais qui continuait d'être
accessible sur Internet, a ainsi subi le gel de ses avoirs, ce qui l'a
contrainte à fermer définitivement\footnote{«~RT France {[}...{]} a
  ainsi subi le gel de ses avoirs~»~: Le Parisien, \emph{RT France,
  branche française de la chaîne russe, annonce sa fermeture}, 23
  janvier 2022~:
  \url{https://www.leparisien.fr/international/rt-france-branche-francaise-de-la-chaine-russe-annonce-sa-fermeture-21-01-2023-YMOTSTASWZAF3KSGCYHAFFMG6U.php}.}.

Enfin, pour finir à propos des différentes occurrences de censure
financière, on ne peut pas ne pas évoquer les activités liées aux
cryptomonnaies, qui ont subi et continuent de subir des restrictions de
la part des organismes financiers. L'achat de cryptomonnaies est entravé
par les banques qui interdisent régulièrement à leurs clients (toujours
en prétendant les «~protéger~») d'envoyer des fonds vers les plateformes
de change\footnote{«~L'achat de cryptomonnaies est entravé par les
  banques qui interdisent régulièrement à leurs clients {[}...{]}
  d'envoyer des fonds vers les plateformes de change~»~: Jean-Luc
  (Bitcoin.fr), \emph{Les banques et Bitcoin --- Classement de janvier
  2023}, 9 janvier 2023~:
  \url{https://bitcoin.fr/bitcoin-et-les-banques-classement-de-janvier-2023/}.}.
De plus, les entreprises du secteur peinent régulièrement à ouvrir un
compte bancaire en raison de la méfiance des acteurs
traditionnels\footnote{Dans son livre \emph{Cryptomonnaie~: la nouvelle
  guerre}, François-Xavier Thoorens explique par exemple comment lui et
  sa famille ont été expulsés de leur banque familiale après avoir voulu
  ouvrir un compte professionnel pour recevoir des fonds récupérés lors
  de l'ICO de Ark (pp.~91 -- 97). Mais son cas est loin d'être une
  exception.}.

La censure financière est donc de plus en plus fréquente dans notre
société. Elle touche de nombreuses personnes de bords politiques
opposés, de nationalités diverses et de professions variées. Elle
s'exerce bien souvent sans décision juridique spécifique, ce qui donne
un caractère ésotérique, caché, arbitraire à l'application du pouvoir
réel. C'est ce qui en fait un problème subtil et difficile à expliciter.

L'intervention plus prononcée de cette censure a pour effet de pousser
les gens à s'intéresser à Bitcoin. En effet, l'expérience d'une telle
restriction provoque nécessairement le désir de trouver un moyen de la
contourner, quand bien même celle-ci serait légère. Lorsqu'une personne
prend pleinement conscience de la censure comme une réalité concrète et
non plus comme un risque abstrait, elle ressent le besoin de s'en
libérer et de se prémunir de ce danger, ce qui lui démontre (ou lui
confirme) la proposition de valeur de Bitcoin\footnote{Cet effet de
  l'expérience de la censure a été décrit par Nick Szabo au micro de
  Peter McCormack en 2019~: «~Certaines personnes doivent être frappées
  par la réalité. Si vous êtes censuré par une banque, comme c'est de
  plus en plus le cas -- et c'est d'ailleurs l'un des risques de la
  centralisation numérique -- c'est que les gens soient censurés et les
  activistes politiques de différents bords commencent à découvrir qu'on
  peut aller voir les banques et faire taire ses ennemis politiques et
  les gens qui font des choses qu'on ne veut pas qu'ils fassent, on les
  fait taire. On n'a pas nécessairement besoin de faire passer une loi,
  on peut convaincre certains régulateurs ou certains politiciens, et
  puis ils mettent la pression sur les banques, et boum~: c'est notre
  loi de facto juste là. Ça se produit de plus en plus souvent parce que
  la centralisation numérique rend les choses si vulnérables à ça. Il
  s'agit donc d'une tendance opposée et tout dépend de la vitesse à
  laquelle elle se développe, car à chaque fois que quelqu'un est
  censuré, boum~: c'est une réalité qui s'impose à lui et il devient fan
  de Bitcoin.~» -- What Bitcoin Did Podcast, \emph{Nick Szabo on
  Cypherpunks, Money and Bitcoin} (audio), 1 novembre 2019~:
  \url{https://www.whatbitcoindid.com/podcast/nick-szabo-on-cypherpunks-money-and-bitcoin}.}.
C'est le cas de l'auteur de cet ouvrage qui a vu son compte bancaire
être gelé sans préavis, sans que la banque ne mentionne la raison
derrière cette suspension, et qui n'a pu récupérer ses fonds que six
mois plus tard\footnote{«~C'est le cas de l'auteur de cet ouvrage~»~:
  Ludovic Lars sur Twitter, /02/2022 10:42 UTC~:
  \url{https://twitter.com/lugaxker/status/1493536121678147586}.}.

\section*{Censure et monnaie numérique de banque
centrale}\label{censure-et-monnaie-numuxe9rique-de-banque-centrale}
\addcontentsline{toc}{section}{Censure et monnaie numérique de banque
centrale}

\markright{Censure et monnaie numérique de banque centrale}

La tendance est donc claire~: avec l'utilisation intensive des comptes
bancaires en lieu et place des espèces, le pouvoir de censure financière
est devenu de plus en plus important. Ainsi, même si cette censure reste
aujourd'hui occasionnelle, nous pouvons nous attendre à ce qu'elle
constitue un problème grandissant à l'avenir. Plus précisément, elle
pourrait devenir une contrainte générale dans les décennies à venir avec
le déploiement progressif des monnaies numériques de banque centrale
(MNBC) et la disparition conjointe de l'argent liquide.

Tel que nous l'avons vu dans la section dédiée à la monnaie numérique de
banque centrale dans le chapitre~\hyperref[ch:adversaire]{4}, la
numérisation de la monnaie constitue la prochaine étape dans l'évolution
de la monnaie étatique. Depuis 2016, les banques centrales autour du
monde s'efforcent de concevoir des systèmes qui pourraient être utilisés
par le grand public et les communications à ce sujet se multiplient
depuis 2020.

Une telle monnaie numérique permettrait de récupérer un revenu de
seigneuriage supplémentaire en supprimant le coût de la production de
l'argent liquide remplacé et en reprenant une part de l'activité
monétaire qui a lieu aujourd'hui par l'intermédiaire du crédit émis par
les banques commerciales. Mais elle permettrait aussi (ce qui nous
intéresse ici) d'exercer un contrôle financier total sur les
transactions des citoyens en centralisant la gestion du système entre
les mains de la banque centrale et des organismes agréés.

Ce contrôle s'accompagnerait bien entendu d'une surveillance financière
accrue, qui serait justifiée par les mêmes prétextes utilisés
aujourd'hui, comme la lutte contre le blanchiment d'argent et le
financement du terrorisme. Ceci pourrait conduire à l'instauration d'un
système panoptique, où la surveillance se ferait à l'insu du
surveillé\footnote{«~un système panoptique, où la surveillance se ferait
  à l'insu du surveillé~»~: Le panoptique (en anglais,
  \emph{panopticon}) était un type d'architecture carcérale imaginée par
  le philosophe utilitariste Jeremy Bentham et son frère Samuel à la fin
  du \textsc{xviii}~siècle. L'objectif de la structure panoptique était
  de permettre à un gardien, logé dans une tour centrale, d'observer
  tous les prisonniers, enfermés dans des cellules individuelles autour
  de la tour, sans que ceux-ci puissent savoir s'ils étaient observés.}.
Les banques centrales nient vouloir aller dans cette direction, mais le
fait est qu'elles ne rendront jamais leurs systèmes strictement
confidentiels, réservant toujours un droit de regard aux autorités
compétentes.

Cette surveillance financière pourrait être affermie par la disparition
progressive de l'argent liquide, qui a déjà commencé à certains endroits
du monde. C'est le cas de la Suède, où la question de la fin des espèces
est déjà discutée et où l'État fait tout pour mettre à disposition des
moyens de paiement numérique innovants\footnote{«~C'est le cas de la
  Suède {[}...{]} où l'État fait tout pour mettre à disposition des
  moyens de paiement numérique innovants~»~: sweden.se, \emph{A cashless
  society}~: \url{https://sweden.se/life/society/a-cashless-society}.}.
C'est aussi le cas de la Chine, où l'essentiel des transferts se fait
par l'intermédiaire de systèmes de paiement mobile comme WeChat Pay et
Alipay. Ce n'est pas un hasard si ces deux pays ont été les premiers à
envisager sérieusement de développer une monnaie numérique.

La guerre contre l'argent liquide sévit déjà dans certains pays par le
biais de la démonétisation de certains billets en circulation, qui
peuvent être échangés contre d'autres billets ou être déposés sur un
compte bancaire, à condition d'attester de la provenance des fonds. En
Inde en novembre 2016, le gouvernement de Narendra Modi a ainsi
démonétisé les billets de 500 et 1000 roupies, équivalant à 7,5 et
15~\$, et représentant à eux seuls 86~\% de la monnaie en circulation,
dans le but affiché de lutter contre la contrefaçon de faux billets,
l'évasion fiscale et l'économie informelle\footnote{«~Narendra Modi a
  ainsi démonétisé les billets de 500 et 1000 roupies~»~: Ninon Renaud,
  Michel De Grandi, \emph{En Inde, la démonétisation des grosses
  coupures provoque l'émoi}, 13 novembre 2016~:
  \url{https://www.lesechos.fr/2016/11/en-inde-la-demonetisation-des-grosses-coupures-provoque-lemoi-216048}.}.
Au Nigéria, début 2023, le gouvernement a tenté (sans grand succès)
d'appliquer une mesure similaire, par la limitation des retraits et la
démonétisation des grosses coupures, dans le but de contrôler
l'inflation, de lutter contre la contrefaçon et de promouvoir le naïra
électronique (eNaira) lancé par la banque centrale en octobre
2021\footnote{«~la limitation des retraits et la démonétisation des
  grosses coupures~»~: Simi Jolaoso, \emph{Nigeria's naira shortage:
  Anger and chaos outside banks}, 14 février 2023~:
  \url{https://www.bbc.com/news/world-africa-64626127}.}. Cette pratique
de la démonétisation n'est cependant pas nouvelle puisqu'elle avait été
utilisée en Europe après la Seconde Guerre mondiale pour enrayer les
effets inflationnistes du faux-monnayage et pour détruire les profits du
marché noir, ce qui avait fait d'ailleurs dire au personnage du Dabe
dans \emph{Le cave se rebiffe} qu'«~en matière de monnaie, les États ont
tous les droits et les particuliers aucun~!~».

Une fois la monnaie numérique en place et l'argent liquide largement
limité, les gens respectueux de la loi n'auraient d'autre choix que
d'utiliser ce système surveillé. Le système pourrait limiter le montant
que les gens dépensent, ce pour quoi ils l'utilisent et avec qui ils
commercent. De plus, en tant que système informatique, il pourrait être
facilement programmé de façon à imposer des conditions de dépense pour
chaque montant de monnaie possédé par l'utilisateur. Une telle
programmabilité permettrait aux autorités en charge d'orienter le
comportement politique, économique et moral des individus dans le sens
désiré, ce qui donnerait à la censure financière une portée jamais vue
auparavant.

Au niveau économique d'abord, cela permettrait d'améliorer ce que les
banquiers centraux appellent la transmission de la politique monétaire,
c'est-à-dire le processus par lequel les décisions de politique
monétaire affectent l'économie en général et le niveau des prix en
particulier. Aujourd'hui cette transmission est essentiellement assurée
par la modification des taux d'intérêt directeurs. Demain, elle pourrait
se faire par la programmation de la monnaie. Cela permettrait notamment
de transformer le système d'aides sociales en un système de subvention
directe exigeant la dépense rapide dans un secteur économique précis,
dans le but de le stimuler.

Ensuite au niveau moral, cette programmabilité permettrait d'orienter
massivement les paroles et les actions des gens dans un sens déterminé,
dans la droite lignée des censeurs de la Rome antique. Dans notre
société moderne, cela pourrait être fait dans le cadre de la lutte
contre le changement climatique, en récompensant le comportement
«~écologique~», tel que la location d'un vélo pour se déplacer, et en
punissant l'attitude «~pollueuse~», telle que la consommation de viande.
Cette possibilité fait ainsi entrevoir l'instauration d'un système de
crédit social à la chinoise.

Enfin d'un point de vue politique, ce système permettrait de réduire
l'opposition au pouvoir en sanctionnant ceux qui pensent mal, ceux qui
s'expriment trop, ceux qui manifestent contre, etc. Le pouvoir politique
pourrait raffermir sa position en appliquant les interventions, non plus
de manière publique et légale (conformément à l'idée d'état de droit au
sens de \emph{Rechtsstaat}), mais de façon cachée et discrétionnaire.
Cela pourrait constituer les prémices d'un régime totalitaire où l'État
saurait tout, contrôlerait tout, et où il n'y aurait plus besoin de lois
formelles\footnote{«~où il n'y aurait plus besoin de lois formelles~»~:
  George Orwell, \emph{1984}, 1949~: «~Ce qu'il allait commencer,
  c'était son journal. Ce n'était pas illégal (rien n'était illégal,
  puisqu'il n'y avait plus de lois), mais s'il était découvert, il
  serait, sans aucun doute, puni de mort ou de vingt-cinq ans au moins
  de travaux forcés dans un camp.~»}. La MNBC serait un outil puissant
de surveillance financière de masse, pouvant œuvrer à la réalisation
d'un avenir orwellien dans lequel les individus n'auraient plus aucune
vie privée et dont le pouvoir de résistance à l'autorité serait réduit
au minimum.

Cette censure financière aurait lieu à une échelle jamais vue
auparavant. Par conséquent, il serait difficile de la mettre en place
par une gestion manuelle des êtres humains. C'est pour cette raison
qu'elle serait probablement déléguée à un algorithme doté d'une
intelligence artificielle, qui détecterait les mauvais paiements et les
bloquerait instantanément. Le système de MNBC pourrait ainsi nous mener
à une situation qui rappellerait celle décrite par saint Jean dans son
Apocalypse~:

«~Par ses manœuvres, tous, petits et grands, riches ou pauvres, libres
et esclaves, se feront marquer sur la main droite et sur le front, et
nul ne pourra rien acheter ni vendre s'il n'est pas marqué au nom de la
Bête ou au chiffre de son nom\footnote{Ap :16-17.}.~»

Dans ce monde dystopique dont nous pouvons à peine imaginer les
ramifications, l'espoir serait représenté par Bitcoin, dont la promesse
fondamentale est d'échapper à de telles interventions. Par sa résistance
à la censure, Bitcoin constituerait ainsi un oasis de liberté dans le
désert de la servitude généralisée. Il serait, en substance, le dernier
recours pour une population qui aurait sombré dans l'asservissement par
la technique.

\section*{La censure dans Bitcoin}\label{la-censure-dans-bitcoin}
\addcontentsline{toc}{section}{La censure dans Bitcoin}

\markright{La censure dans Bitcoin}

Pour bien comprendre comment Bitcoin s'oppose à la censure, il est
nécessaire de comprendre comment cette dernière peut s'exercer sur la
chaîne. En effet, si le modèle de Nakamoto est réputé \emph{résistant} à
la censure, ceci ne signifie pas pour autant qu'il est «~incensurable~».
La censure dans Bitcoin est non seulement possible, mais elle est aussi
probable au-delà d'un certain stade d'adoption.

Lorsqu'on parle de Bitcoin, le terme de censure possède un sens précis~:
il désigne l'action d'empêcher une transaction d'être réalisée sur une
base économiquement irrationnelle, en entravant son inscription pérenne
dans la chaîne de blocs. Cette définition rejoint l'idée de restreindre
l'activité financière d'une entité dans le but de modeler son
comportement. En un sens, cette censure ressemble également à de la
censure du discours, car il s'agit d'empêcher indirectement l'individu
d'écrire une transaction signée dans un registre.

La façon dont peut s'exercer la censure dans Bitcoin peut être
extrapolée à partir de ce qui existe déjà dans le monde bancaire et dans
le secteur des cryptomonnaies, à commencer par les prétextes utilisés
pour la défendre. D'une part, les justifications utilisées dans la
finance traditionnelle sont largement applicables à Bitcoin, comme la
lutte contre le blanchiment d'argent, le financement du terrorisme et la
protection des épargnants~: la cryptomonnaie permet en effet d'éviter
l'impôt, de financer tous les projets imaginables et de participer à des
escroqueries. D'autre part, de nouveaux prétextes émergent comme la
dévaluation de la monnaie locale (un instrument déflationniste
représente une concurrence déloyale) ou la lutte contre le changement
climatique (le minage émet du CO\textsubscript{2}).

De ces prétextes, les autorités tirent des réglementations générales qui
s'appliquent à l'échelle internationale, comme c'est déjà le cas dans le
système bancaire mondial. Les différentes juridictions se basent sur les
recommandations du GAFI, dont le rôle premier est la lutte contre le
blanchiment d'argent et le financement du terrorisme. Comme nous l'avons
expliqué en parlant de l'arbitrage juridictionnel (voir
chapitre~\hyperref[ch:adversaire]{4}), elles sont fortement poussées à
appliquer ces recommandations sous peine de subir les sanctions
économiques des États-membres. Le FMI peut également être mis à profit,
celui-ci ayant pour but d'assurer la stabilité du système monétaire
mondial (donc de protéger les monnaies des États-membres).

Cette coopération permet de constituer des listes noires d'adresses ne
rentrant pas en conformité avec les réglementations, listes qui sont
distribuées aux divers acteurs financiers réglementés. On peut citer par
exemple la liste dressée par l'\emph{Office of Foreign Assets Control}
(OFAC), l'organisme dépendant du Trésor étasunien en charge d'appliquer
les sanctions internationales des États-Unis dans le domaine
financier\footnote{«~la liste dressée par l'\emph{Office of Foreign
  Assets Control}~»~: U.S. Department of the Treasury, \emph{Treasury
  Sanctions IRGC-Affiliated Cyber Actors for Roles in Ransomware
  Activity}, 14 septembre 2022~:
  \url{https://home.treasury.gov/news/press-releases/jy0948}~;
  \url{https://home.treasury.gov/policy-issues/financial-sanctions/recent-actions/20220914}.},
qui fait autorité dans le domaine financier en raison de
l'extraterritorialité du droit étasunien.

La censure s'applique ainsi déjà dans une partie de l'économie basée sur
Bitcoin. Tous les acteurs qui se conforment aux réglementations bloquent
les bitcoins (et autres cryptomonnaies) provenant des adresses présentes
sur les listes noires et gèlent les comptes de l'utilisateur jusqu'à ce
qu'il se justifie. Toutefois, cette pratique conserve un caractère
partiel et implicite~: les transactions en elles-mêmes ne sont pas
encore explicitement interdites, mais les fonds ne doivent pas être
envoyés aux intermédiaires financiers réglementés, comme les plateformes
de change ou les processeurs de paiement\footnote{«~les processeurs de
  paiement~»~: Depuis le début de l'année 2021, BitPay demande par
  exemple à ses clients européens de s'inscrire et de vérifier leur
  identité avant de pouvoir effectuer un achat.}. Cette situation pousse
certaines plateformes à faire beaucoup de zèle dans le domaine en
refusant des bitcoins provenant de mélanges de pièces et geler les
comptes des personnes le faisant\footnote{«~en refusant des bitcoins
  provenant de mélanges de pièces et geler les comptes des personnes le
  faisant~»~: Jamie Redman, \emph{As FATF Regulations Galvanize, Crypto
  Mixing Applications Are Targeted}, 27 décembre 2019~:
  \url{https://news.bitcoin.com/as-fatf-regulations-galvanize-crypto-mixing-applications-are-targeted/}~;
  6102bitcoin, \emph{CoinJoin Flagging}~:
  \url{https://6102bitcoin.com/coinjoin-flagging/}.}, en l'absence d'une
réglementation explicite\footnote{Sur Ethereum, les adresses liées au
  contrat de mélange Tornado Cash ont été placées sur la liste de l'OFAC
  en août 2022. Mais sur BTC, aucune loi ni liste liée au mélange n'est
  connue~: il y a juste une suspicion généralisée.}.

La réglementation peut également s'étendre à l'industrie du minage.
L'activité minière tend naturellement à se centraliser, par l'agrégation
de la puissance de hachage en fermes de minage, par le rassemblement des
hacheurs en coopératives minières et par l'utilisation de relais de
communication par ces coopératives. Ces gros acteurs sont généralement
identifiables et se soumettent donc plus facilement aux réglementations
concernant les transactions à traiter. C'est ce qui pourrait amener une
censure sur le réseau.

Les mineurs peuvent dans un premier temps pratiquer une censure passive
en refusant systématiquement de confirmer des transactions, pour des
raisons économiquement irrationnelles, typiquement sous la pression du
régulateur. Ce type de censure a notamment été envisagé par la
coopérative du groupe Marathon, qui avait déclaré en 2021 vouloir
pratiquer le «~minage de blocs propres\footnote{Communiqué de Marathon
  et de DMG Blockchain Solutions, \emph{Marathon Patent Group and DMG
  Blockchain Solutions to Form the Digital Currency Miners of North
  America (DCMNA) and Launch North America's First Cooperative Mining
  Pool}, 5 janvier 2021~:
  \url{https://web.archive.org/web/20210128112455/https://www.marathonpg.com/news/press-releases/detail/1220/marathon-patent-group-and-dmg-blockchain-solutions-to-form}.}~»,
avant de se rétracter sous la pression populaire. Ce filtrage est mis en
place sur Ethereum avec les validateurs qui utilisent des relais
d'optimisation de MEV qui respectent les normes de l'OFAC et par
conséquent n'incluent pas les transactions considérées comme
sales\footnote{La valeur extractible maximale (\emph{maximal extractable
  value}), initialement appelée valeur extractible par les mineurs
  (\emph{miner extractable value}), est la valeur maximale que le
  validateur peut générer en modifiant l'ordre ou en excluant des
  transactions au sein de son bloc, profitant des différentes
  irrégularités des contrats autonomes, notamment en ce qui concerne les
  places de marché décentralisées. En octobre 2022, la quantité de
  validation passant par des relais appliquant ce type d'optimisation a
  dépassé les 50~\%, indiquant la potentialité d'une attaque. -- Voir
  MEV Watch~: \url{https://www.mevwatch.info/}.}. Les participants à ces
relais étaient principalement des plateformes de change en 2023.

Cette censure passive n'est pas très problématique car elle demande que
100~\% de la puissance de calcul s'y conforme pour être effective. Les
mineurs dissidents, c'est-à-dire ceux qui ignorent délibérément les
réglementations, débloquent la situation en validant les transactions
ignorées par les autres. Seuls les délais de confirmation sont affectés.

Cependant, cette situation peut devenir autrement plus grave si les
mineurs conformistes, à savoir les mineurs suivant méticuleusement les
réglementations, commencent à refuser les blocs contenant les
transactions «~sales~». C'est ce que nous appelons ici la censure
active, qui consiste à empêcher des transactions d'être confirmées en
rendant orphelins tous les blocs qui les contiennent. Pour être
maintenue dans le temps de manière certaine, elle nécessite de disposer
de la majorité de la puissance de calcul du réseau~: il s'agit donc
d'une attaque des 51~\%. Les branches faibles formées par l'attaque de
censure sont mises de côté en vertu du principe de la chaîne la plus
longue, comme illustré sur la
figure~\hyperref[fig:censorship-attack]{9.1}.

\begin{figure}

{\centering \includegraphics{chapters/img/mining-attack-censorship.png}

}

\caption{Attaque de censure active.}

\end{figure}%

Le coût d'une telle attaque peut être colossal suivant la puissance de
calcul déployée sur le réseau\footnote{On a vu dans le
  chapitre~\hyperref[ch:confirmation]{8} que le coût d'une telle attaque
  se chiffre en milliards de dollars sur le réseau Bitcoin principal.}.
Mais ce coût serait justifié par le développement des activités
illégales évitant l'impôt et le seigneuriage. En effet, comme montré
dans le chapitre~\hyperref[ch:adversaire]{4}, le profil-type de
l'attaquant est l'État dont le pouvoir de prélèvement repose grandement
sur son contrôle de la monnaie~: c'est pourquoi il se moque de réduire
(voire de détruire) l'utilité de Bitcoin ce faisant.

Cette attaque hypothétique serait précédée d'une déclaration de guerre
contre Bitcoin. Toute la tolérance vis-à-vis des utilisateurs
disparaîtrait, et ce qui n'était pas officiel le deviendrait~: toutes
les transactions qui ne sont pas explicitement autorisées seraient
déclarées interdites. L'utilisation libre serait criminalisée d'une
manière ou d'une autre, et le minage honnête aussi.

Ce durcissement permettrait de coopter plus largement les regroupements
miniers auxquels les directives étatiques seraient transmises. L'État
pourrait aussi réquisitionner ou acheter son propre matériel de hachage.
En somme, il disposerait à un moment donné d'une puissance de calcul
majoritaire. Une fois la puissance de calcul rassemblée, l'attaque
serait mise à exécution.

La censure active est insidieuse car il suffit que 51~\% l'applique pour
qu'elle continue. Son prolongement dans le temps peut finir par
constituer une nouvelle normalité. Par conséquent, les mineurs
économiquement rationnels ont tout intérêt à appliquer la censure, comme
l'a montré un article de Juraj Bednar sur le sujet\footnote{Juraj
  Bednar, \emph{Bitcoin censorship will most likely come, pt 2}, 18
  novembre 2020~:
  \url{https://juraj.bednar.io/en/blog-en/2020/11/18/bitcoin-censorship-will-most-likely-come-pt-2/}.}.
L'attaquant ne doit donc pas nécessairement disposer en permanence de la
majorité du taux de hachage.

La confidentialité n'empêche pas la censure d'avoir lieu, mais la rend
simplement plus coûteuse. Dans le cas où l'intégralité des utilisateurs
refuserait de se conformer aux normes de surveillance, les censeurs
devraient refuser l'ensemble des transactions et ne pas recevoir les
frais correspondants. L'attaque prendrait alors la forme d'une
destruction totale de l'utilité de la chaîne par le minage de blocs
vides, c'est-à-dire une attaque Goldfinger. Le nom de cette dernière
fait référence au principal antagoniste du film de James Bond éponyme
sorti en 1964, qui souhaitait irradier le stock d'or américain sécurisé
au dépôt de Fort Knox dans le but de le rendre durablement inutilisable
et d'augmenter la valeur du reste de l'or\footnote{Joshua A. Kroll, Ian
  C. Davey, Edward W. Felten, «~\emph{The Economics of Bitcoin Mining,
  or Bitcoin in the Presence of Adversaries}~», in \emph{Workshop on the
  Economics of Information Security}, 2013~:
  \url{https://asset-pdf.scinapse.io/prod/2188530018/2188530018.pdf}.}.

De ce fait, il est tout à fait possible d'exercer de la censure dans
Bitcoin. Toutefois, ce n'est ni facile, ni définitif, car il existe un
mécanisme au sein du protocole permettant de lutter contre ce type
d'attaque~: la résistance à la censure.

\section*{Le mécanisme de résistance à la
censure}\label{le-muxe9canisme-de-ruxe9sistance-uxe0-la-censure}
\addcontentsline{toc}{section}{Le mécanisme de résistance à la censure}

\markright{Le mécanisme de résistance à la censure}

La résistance à la censure désigne la difficulté à entraver
arbitrairement les transactions. Elle est couramment citée comme l'une
des deux grandes promesses de Bitcoin~: permettre à quiconque d'envoyer
des fonds à n'importe qui d'autre, quel que soit le moment, où que se
trouve le destinataire dans le monde, pourvu qu'il dispose d'un accès à
Internet.

La résistance à la censure constitue un élément essentiel de Bitcoin. Si
elle n'existait pas, le système ne pourrait tout simplement pas survivre
en tant que tel~: il deviendrait un système contrôlé centralement par
une autorité décidant des bonnes et des mauvaises transactions. Il
devrait s'adapter, tel GoldMoney ou PayPal, ou périr, à l'instar de
e-gold ou de Liberty Reserve. De plus, le pouvoir absolu sur la
sélection des transactions permettrait à cette autorité d'exercer
\emph{de facto} une influence irrésistible sur le protocole par le biais
de l'application de soft forks (comme nous le verrons dans les chapitres
\hyperref[ch:changement]{10} et \hyperref[ch:determination]{11}), ce qui
mènerait \emph{in fine} à la destruction de la politique monétaire
originelle. Sans résistance à la censure, la proposition de valeur de
Bitcoin s'effondrerait.

Cependant, cette résistance n'a jamais été décrite explicitement par
Satoshi Nakamoto. Dans ses interventions, le père de Bitcoin a expliqué
comment son système était sécurisé économiquement contre la double
dépense, ce qui était déjà une grande évolution par rapport aux modèles
décentralisés précédents. Mais il n'a en revanche pas indiqué comment le
système pouvait s'opposer à la censure, c'est-à-dire au blocage partiel
ou total de l'activité transactionnelle par une entité hostile. Il
semblait se reposer sur la bonne volonté des mineurs «~honnêtes~»,
pensant même qu'il y aurait «~probablement toujours des nœuds prêts à
traiter les transactions gratuitement\footnote{Satoshi Nakamoto,
  \emph{Bitcoin v0.1 released}, /01/2009 19:27:40 UTC~:
  \url{https://www.metzdowd.com/pipermail/cryptography/2009-January/014994.html}.}~»,
cette résistance allant de soi.

Le mécanisme de résistance à la censure de Bitcoin a été mis en lumière
en 2018, par le développeur et auteur Eric Voskuil, qui a montré qu'il
reposait de manière essentielle sur les frais de transaction\footnote{Le
  mécanisme de résistance à la censure a initialement été décrit par
  Eric Voskuil en janvier 2018~:
  \url{https://github.com/libbitcoin/libbitcoin-system/wiki/Other-Means-Principle/77d7556a14f89d1704f1bb97ca0aed04606363d0}.
  Voir aussi Eric Voskuil, «~Propriété de résistance à la censure~», in
  \emph{Cryptoéconomie~: Principes fondamentaux de Bitcoin}, Amazon KDP,
  2022, pp.~24--25.}. Comme dans le cas de la résistance à la double
dépense, la propriété de résistance à la censure n'est pas absolue mais
économique~: c'est une régulation financée par les frais des
transactions prohibées.

La sécurité minière, on le rappelle, repose sur un principe
majoritaire~: la quantité de puissance de calcul contrôlée par les
mineurs honnêtes doit être supérieure par rapport à celle des
attaquants. L'important n'est pas que le taux de hachage de Bitcoin soit
le plus haut possible~; c'est que les mineurs disposant d'une puissance
de calcul non négligeable soient prêts à miner systématiquement toutes
les transactions payant un montant correct de frais et à toujours
construire leurs blocs à partir de la plus longue chaîne.

Ainsi, cette sécurité ne dépend pas uniquement de la puissance de
calcul. Elle est aussi fonction de la distribution de cette puissance de
calcul et de la fraction de mineurs par rapport au reste de
l'humanité\footnote{Eric Voskuil, «~Modèle de sécurité qualitatif~», in
  \emph{Cryptoéconomie~: Principes fondamentaux de Bitcoin}, Amazon KDP,
  2022, pp.~59--62.}. En effet, un taux de hachage qui serait concentré
dans les mains d'un seul mineur créerait une sécurité équivalente à
celle d'un système centralisé, dépendante du mineur en question. Aussi,
un réseau équitablement distribué et déployant une grande quantité
puissance de calcul aura plus de risque d'être coopté s'il comporte un
petit nombre de mineurs que s'il en comporte un grand nombre.

La solution au problème de la censure provient des mineurs dissidents,
qui sont prêts à confirmer des transactions litigieuses ou décrétées
comme illégales par le pouvoir. Le risque pris par ces mineurs doit
alors être compensé économiquement.

Le mineur dissident a besoin de rester anonyme afin de pouvoir miner
dans la clandestinité. Cette possibilité est assurée par le fait que les
mineurs ne sont jamais contraints de s'identifier au sein du protocole.
Le signalement des blocs minés par les coopératives minières est en
effet une démarche purement optionnelle et volontaire, ayant pour but de
rassurer les utilisateurs (leurs clients) sur la distribution du
système.

La part du revenu du minage provenant de la création monétaire joue un
rôle accessoire dans la lutte contre la censure, que cette dernière soit
passive ou active. D'une part, cette partie de la récompense est la même
pour tous les mineurs, ce qui fait qu'elle n'influe pas sur leur choix
économique d'inclure une transaction ou non dans un bloc. D'autre part,
la potentielle chute de l'utilité (et donc du revenu de minage) du
système provoquée par une censure active (attaque), ne saurait empêcher
l'autorité à l'origine d'arriver à ses fins. Les motivations de cette
dernière sont en effet particulières~: elle ne cherche pas à réaliser un
profit direct mais à contrôler, voire détruire, le système en décrétant
quelles transactions sont autorisées et lesquelles ne le sont pas.

En revanche, les frais de transaction sont, eux, essentiels au mécanisme
de résistance à la censure. Par leur intégration dans le protocole, ces
commissions sont chacunes associées publiquement à une transaction.
Ainsi, les frais luttent d'une part contre la censure passive en
incitant les mineurs à confirmer les transactions, et découragent
d'autre part la censure active en donnant à la branche censurée une
importance économique plus grande.

Dans le cas d'une attaque de censure active, les censeurs acquièrent
plus de la moitié de la puissance de calcul du réseau et rejettent
ouvertement un groupe de transactions défini (par une liste noire par
exemple) en refusant les blocs qui contiendrait l'une d'entre elles. La
chaîne des censeurs est considérée par les nœuds honnêtes comme la
chaîne correcte car elle est plus longue.

C'est dans ce contexte que le mécanisme des frais intervient. Les
initiateurs des transactions censurées, voyant que leurs transactions ne
sont pas confirmées, augmentent leurs commissions. C'est un comportement
naturel que l'on observe déjà lors des périodes de congestion du réseau,
comme au sommet de la bulle de 2017, lorsque les frais médians par
transaction ont dépassé les 30~\$\footnote{«~lorsque les frais médians
  par transaction ont dépassé les 30~\$~»~:
  \url{https://bitinfocharts.com/comparison/bitcoin-median_transaction_fee.html\#alltime}.}.
En outre, il est logique de payer une grande quantité de frais pour
transférer de fortes sommes, celles-ci étant plus à risque que les
petits transferts\footnote{Dans Bitcoin, les frais sont aujourd'hui
  payés proportionnellement à la charge des données (taille ou poids de
  la transaction). Cependant, la menace de plus en plus claire de la
  censure pourrait pousser les utilisateurs à payer des frais
  proportionnels au montant transféré comme cela se fait dans le domaine
  financier en général.}.

Cette augmentation crée un supplément de frais, qui constitue la
différence entre les frais de toutes les transactions et ceux des
transactions non autorisées qui se retrouvent dans les mempools des
nœuds honnêtes. C'est ce supplément (et ce supplément uniquement) qui
incite les mineurs dissidents à déployer plus de puissance de calcul au
cours du temps~: plus l'économie supprimée est importante, plus le
différentiel de puissance de calcul résultant est grand.

Les mineurs dissidents se coordonnent en privé ou par la voie d'un
signalement pour planifier une riposte. Une fois que la puissance de
calcul est jugée suffisante, ils se mettent à confirmer les transactions
censurées. Puisque leur puissance de calcul est majoritaire, leur chaîne
devient la plus longue et la chaîne des censeurs est invalidée. De cette
manière, la censure est vaincue, du moins jusqu'à une nouvelle offensive
de l'ennemi.

Ainsi, le mécanisme de résistance à la censure est ancré profondément
dans le protocole. La preuve de travail, le caractère anonyme du minage,
le système de frais intégré~: ce sont autant d'éléments permettant de
coordonner un marché des frais afin de repousser les censeurs. Il est
impossible d'estimer quelle serait la part de l'économie censurée,
l'envergure de l'attaque étatique ou le montant de frais que les
utilisateurs seraient prêts à payer, de sorte qu'on ne peut pas garantir
l'incensurabilité de Bitcoin. Mais le mécanisme n'en est pas moins
fonctionnel.

Il est à noter que le rôle des frais de transaction, explicité en 2018
par Eric Voskuil, a été négligé par certains protocoles
cryptoéconomiques. C'est en particulier le cas d'Ethereum qui a fait le
choix de brûler une partie des frais du réseau dans le but de rendre
l'éther déflationniste au sens monétaire avec l'activation de l'EIP-1559
en août 2021\footnote{«~l'activation de l'EIP-1559 en août 2021~»~:
  Ludovic Lars, \emph{Ethereum face à une catastrophe annoncée~? Censure
  et volatilité~: l'EIP-1559, le cauchemar des mineurs}, 15 juillet
  2021~:
  \url{https://journalducoin.com/analyses/eip-1559-changement-nefaste-ethereum/}.}.
La communauté d'Ethereum a également choisi de passer en preuve d'enjeu
en septembre 2022, ce qui constitue un autre pas vers l'acceptation de
la censure comme nous l'expliquerons plus bas.

\section*{L'importance de la
confidentialité}\label{limportance-de-la-confidentialituxe9}
\addcontentsline{toc}{section}{L'importance de la confidentialité}

\markright{L'importance de la confidentialité}

La censure financière est étroitement apparentée à la surveillance des
transactions. Cette dernière permet en effet d'affiner la sélection des
transferts et d'exercer un pouvoir subtil sur l'économie, sans brusquer
les personnes les plus dociles. La chose vaut pour le monde bancaire
comme nous l'avons constaté, mais elle vaut aussi pour Bitcoin.

Il existe deux manières de protéger sa richesse et sa liberté~: par la
force physique et par la dissimulation. La première méthode consiste à
se prémunir contre le vol directement en défendant soi-même ses biens
(si besoin à l'aide d'une arme à feu), ou bien indirectement par le
recours aux services de police étatiques ou aux services de protection
privés (gardes du corps, quartiers sécurisés, agence de protection),
très prisés des personnes très fortunées. Cette méthode est importante
et utile contre les criminels communs, mais elle est relativement
inefficace contre la puissance dominante locale dont nous sommes à la
merci -- l'État.

C'est pourquoi les individus ont plus souvent recours à la seconde
méthode, qui consiste à dissimuler leur richesse pour ne pas qu'autrui
en ait connaissance et puisse s'en emparer directement. Cela permet de
dissuader le voleur usant la menace de violence d'aller plus loin~: il
pourrait nous interroger pour savoir où se trouve notre richesse, mais
cette action représenterait un coût supplémentaire (proportionnel à
notre refus de lui livrer cette information) qui freinerait sa
recherche.

Cette méthode est directement liée à la confidentialité (aussi appelée
\emph{privacy} ou protection de la vie privée) qui est le fait de
réserver des informations à un petit nombre de personnes déterminées. La
confidentialité est distincte du secret dans le sens où la personne peut
choisir de révéler sélectivement des informations (confidence). Dans le
contexte financier, il s'agit généralement de faire en sorte que les
détails d'une transaction ne soient connus que des participants.

La confidentialité forme la base de la liberté individuelle dans la
société et constitue une caractéristique essentielle pour tout le monde.
Elle sert en effet à \emph{créer une asymétrie} entre le faible et le
fort, entre l'individu et l'État, de sorte que ce dernier ne puisse pas
empiéter absolument sur les droits du premier. L'État veut vous
persuader du contraire, en vous disant que vous n'avez rien à craindre
si vous n'avez rien à cacher\footnote{«~Je dis que quiconque tremble en
  ce moment est coupable~; car jamais l'innocence ne redoute la
  surveillance publique.~» -- Maximilien de Robespierre, \emph{Discours
  du 11 germinal, an II}, 31 mars 1794.}, mais il n'y a rien
d'historiquement plus faux, comme l'ont montré les exemples des
totalitarisme du \textsc{xx}~siècle.

De ce fait, puisque la censure financière est issue de l'initiative
étatique, la résistance à la censure est en général intrinsèquement liée
à la confidentialité.

D'une part, la résistance à la censure de l'utilisateur individuel
repose sur la confidentialité du système. Si l'État connaît toutes les
transactions, il peut sanctionner l'utilisateur pour avoir effectué une
transaction non autorisée, quand bien même celle-ci serait confirmée par
le réseau. Certains promoteurs de BTC mettent en avant la transparence
du protocole, en la présentant comme un avantage par rapport aux
systèmes bancaires opaques, en insistant sur le pseudonymat et en
réservant la propriété d'anonymat aux «~cryptomonnaies confidentielles~»
comme Monero. Mais il s'agit d'une mécompréhension du rôle de cette
transparence~: les données dans Bitcoin sont publiques dans le but
unique d'assurer le consensus et l'audit, et Monero ne fait
qu'implémenter un compromis différent sur le degré de transparence des
transactions.

D'autre part, la confidentialité de l'utilisateur dépend de la
résistance à la censure du système. En effet, si l'État dispose d'un
contrôle total sur la sélection des transactions, alors il peut choisir
de ne confirmer que les transactions qui dévoilent l'identité de
l'expéditeur et celle du destinataire. Cette dépendance est souvent
remise en question par certains partisans de Monero qui estiment que la
confidentialité par défaut du système protège les utilisateurs de la
censure, considérant que l'État ne peut pas censurer une transaction
qu'il ne connaît pas. Néanmoins, cette vision est plutôt naïve car les
utilisateurs ont la possibilité technique de révéler les informations
relatives à leurs adresses aux organismes de surveillance\footnote{Dans
  Monero et dans les systèmes apparentés, la révélation des transactions
  liées à une adresse se fait par l'intermédiaire d'une clé privée
  d'inspection (\emph{private view key}).}~; la seule barrière à cela
est le coût supplémentaire qu'une telle surveillance représente.

La confidentialité et la résistance à la censure sont donc
interdépendantes dans Bitcoin. Sans confidentialité, il n'y a pas de
résistance à la censure individuelle~; et sans résistance à la censure,
il n'y a pas de confidentialité individuelle. C'est pour cette raison
que la surveillance généralisée, loin d'être anodine, constitue un
problème majeur.

La surveillance s'est étendue dans Bitcoin au cours de son développement
économique par la réglementation des intermédiaires financiers. Les
plateformes de change entre monnaies traditionnelles et cryptomonnaies
ont été progressivement contraintes d'appliquer des normes de
connaissance du client (KYC) et de lutte contre le blanchiment (AML)
similaires au système bancaire classique. Cette récupération
d'informations s'est accompagnée de l'émergence de sociétés d'analyses
de chaîne, telles que Chainalysis ou Ciphertrace, qui croisent les
données d'identification avec les évènements de la chaîne de blocs de
façon à en dégager une interprétation probable, et qui fournissent les
résultats à leurs clients qui sont les agences étatiques, les
institutions financières et les grandes entreprises du domaine. En
outre, l'étau est encore en train de se resserrer, avec l'apparition
d'une version modifiée de la «~règle du voyage~» (\emph{Travel Rule}),
recommandée par le GAFI et déjà imposée par la FINMA suisse, qui
consiste pour un intermédiaire à vérifier systématiquement l'adresse de
retrait du client\footnote{Dans le monde bancaire, la \emph{Travel Rule}
  a originellement été promulguée par le FinCEN étasunien en 1996 (voir
  31 CFR 103.33(g)). Elle exige que toutes les institutions financières
  transmettent des informations sur les expéditeurs à l'institution
  financière suivante lors de certains transferts de fonds. Dans le cas
  de Bitcoin et des cryptomonnaies, il s'agit d'émuler ce voyage en
  considérant que les utilisateurs sont des institutions financières
  lorsqu'ils réalisent des transactions souveraines. Le GAFI a ajouté le
  transfert d'«~actifs virtuels~» à ses recommandations en juin 2019,
  notamment en ce qui concerne la recommandation 16. Cette règle du
  voyage cryptomonétaire pourrait être appliquée par l'intégration dans
  les portefeuilles du protocole de preuve de propriété d'adresse (AOPP)
  proposé en janvier 2022.}\footnote{«~l'apparition d'une version
  modifiée de la "règle du voyage", recommandée par le GAFI et déjà
  imposée par la FINMA suisse~»~:
  \url{https://www.fincen.gov/sites/default/files/advisory/advissu7.pdf}
  -- Recommandation 16 du GAFI, mise à jour en juin 2019 pour inclure le
  transferts d'actifs virtuels~:
  \url{https://www.fatf-gafi.org/fr/publications/Recommandationsgafi/Recommandations-gafi.html}.
  -- \url{https://www.finma.ch/en/news/2019/08/20190826-mm-kryptogwg/}
  -- \url{https://aopp.group/}.}.

Cette évolution crée une réelle menace sur Bitcoin en général. C'est
pourquoi il se forme en face une résistance visant à déjouer la
surveillance, notamment par l'intermédiaire de techniques d'amélioration
de la confidentialité. C'est le cas par exemple du mélange de pièces, ou
CoinJoin, qui permet de brouiller les pistes. C'est aussi le cas des
méthodes intégrées dans Monero. Nous développerons cet aspect dans le
chapitre~\hyperref[ch:rouages]{12}.

Ainsi, la confidentialité est essentielle pour préserver sa richesse et
son autonomie. On ne peut pas être réellement libre sans protéger sa vie
privée. Comme l'écrivait le fabuliste Florian~: «~Pour vivre heureux
vivons cachés\footnote{Jean-Pierre Claris de Florian, «~Le Grillon~», in
  \emph{Fables de Florian}, 1793.}.~»

\section*{Les interventions humaines dans le
consensus}\label{les-interventions-humaines-dans-le-consensus}
\addcontentsline{toc}{section}{Les interventions humaines dans le
consensus}

\markright{Les interventions humaines dans le consensus}

La possibilité de censure dans Bitcoin provoque généralement une volonté
de trouver une solution, s'inscrivant dans la démarche d'ingénieur qui
caractérise les amateurs de cryptomonnaie. Beaucoup de personnes sont en
effet séduites par une alternative à la régulation par les frais~:
l'intervention humaine directe sur la chaîne. Celle-ci consiste à
recourir au «~consensus social\footnote{«~consensus social~»~: Arthur
  Breitman parlait de \emph{social consensus} dès août 2014 dans la
  première description formelle de Tezos. -- Arthur Breitman,
  \emph{Tezos: A Self-Amending Crypto-Ledger}, 3 août 2014~:
  \url{https://tezos.com/position-paper.pdf}.}~», c'est-à-dire au
mécanisme de détermination du protocole. Deux idées de ce type semblent
avoir un certain succès~: l'UASF anti-censure et l'UAHF de changement de
preuve de travail. Il s'agit cependant d'une tentation dangereuse comme
nous allons essayer de le montrer.

La première idée est de rejeter la censure en invalidant la branche des
censeurs partiellement ou totalement, c'est-à-dire en portant atteinte
au principe de la chaîne la plus longue. Le rejet peut se faire en
rendant les blocs de la chaîne des censeurs invalides ou en imposant la
validité de la chaîne concurrente par un point de contrôle temporaire.
Une telle mesure constitue un soft fork (à savoir une restriction des
règles de consensus) et doit être activée par les utilisateurs à un
horodatage ou à une hauteur de bloc donné, d'où le fait qu'on la désigne
comme un \emph{User Activated Soft Fork} (UASF). Elle provoque une
scission car elle n'est pas, dans le cas précis de la censure, soutenue
par la majorité de la puissance de calcul.

L'idée d'invalider la censure par consensus social était déjà évoquée
par Vitalik Buterin en 2016 dans le cas de la preuve d'enjeu~:

«~Sur des échelles de temps moyennes à longues, les humains sont assez
bons pour le consensus. Même si un adversaire avait accès à une
puissance de hachage illimitée, et qu'il parvenait à réaliser une
attaque des 51~\% contre une chaîne de blocs majeure en inversant ne
serait-ce que le dernier mois d'histoire, il serait beaucoup plus
difficile de convaincre la communauté que cette chaîne est légitime que
de simplement distancer la puissance de hachage de la chaîne principale.
{[}...{]} Ces considérations sociales sont ce qui protège finalement
toute chaîne de blocs à long terme, que la communauté de cette chaîne de
blocs l'admette ou non (notez que Bitcoin Core admet cette primauté de
la couche sociale)\footnote{Vitalik Buterin, \emph{A Proof of Stake
  Design Philosophy}, 30 décembre 2016~:
  \url{https://medium.com/@VitalikButerin/a-proof-of-stake-design-philosophy-506585978d51}.}.~»

Cette mesure peut être mise en place par une invalidation directe mais
celle-ci n'est facile à implémenter que si les censeurs marquent leurs
blocs d'une manière ou d'une autre\footnote{«~invalidation directe~»~:
  Cette méthode peut par exemple être mise en place dans l'esprit de
  l'\emph{User Resisted Soft Fork} proposé par Michael Folkson en avril
  2022 en réaction à la menace d'activation de la mise à niveau ``. --
  Michael Folkson, \emph{{[}bitcoin-dev{]} User Resisted Soft Fork for
  CTV}, /04/2022 16:45:20 UTC~:
  \url{https://lists.linuxfoundation.org/pipermail/bitcoin-dev/2022-April/020262.html}.}.
Cela a été réalisé par Bitcoin ABC le 1 décembre 2020 sur sa chaîne
nouvellement créée pour contrer la censure active d'un mineur mécontent
de la scission avec Bitcoin Cash\footnote{Un seul bloc (le bloc 662~687
  d'identifiant ``) de l'attaquant a été invalidé, faisant que 172 blocs
  ont été mis de côté, et que la chaîne non censurée est devenue la
  chaîne correcte. -- Nikita Zhavoronkov sur Twitter, /12/2020 21:59
  UTC~: \url{https://twitter.com/nikzh/status/1333893457920876550}.}.

Il est aussi possible d'inclure un point de contrôle (\emph{checkpoint})
dans le protocole. Un point de contrôle est un bloc considéré comme
valide par défaut. Ce mécanisme a été implémenté dans le logiciel de
Bitcoin dès juillet 2010 dans le but d'éviter une recoordination de
chaîne\footnote{«~implémenté dans le logiciel de Bitcoin dès juillet
  2010~»~: Satoshi Nakamoto, \emph{Bitcoin 0.3.2 released}, /07/2010
  21:35:51 UTC~:
  \url{https://bitcointalk.org/index.php?topic=437.msg3807\#msg3807}.}
et certains de ces points de contrôle sont encore présents dans Bitcoin
Core\footnote{Le point de contrôle le plus récent est celui du bloc
  295~000 miné le 9 avril 2014 (au même moment de l'arrivée de Wladimir
  van der Laan au poste de mainteneur principal) et ayant pour
  identifiant \texttt{.\ Voir\ le\ fichier} dans Bitcoin Core.}\footnote{«~certains
  de ces points de contrôle sont encore présents dans Bitcoin Core~»~:
  \url{https://github.com/bitcoin/bitcoin/blob/24.x/src/chainparams.cpp\#L148-L164}.}.
Dans cette logique, il suffit d'imposer un bloc comme obligatoire pour
invalider la chaîne des censeurs. Cela a été réalisé par Bitcoin SV en
août 2021, qui subissait alors une censure active\footnote{BSV
  Association sur Twitter, /09/2021 21:17 UTC~:
  \url{https://twitter.com/BitcoinAssn/status/1422668065024663554}.}.

Toutefois, même si ce type de recours peut effectivement fonctionner de
manière ponctuelle et temporaire, il ne constitue en rien un mécanisme
robuste de résistance à la censure. En effet, il crée beaucoup trop
d'instabilité en faisant en dernier lieu reposer le consensus sur
l'accord social. Il offre ainsi la possibilité pour une puissance
hostile de déstabiliser durablement le système en semant la zizanie dans
la communauté (notamment par la pression exercée sur les relais
d'opinion) et en créant par là des scissions multiples impossibles à
départager par un facteur objectif.

L'intervention directe de l'accord social dans la confirmation des
transactions est par conséquent une très mauvaise idée. Même dans les
cas où les participants sont d'accord pour dire qu'un tel évènement est
indésirable, ils sont souvent en total désaccord sur la manière de
traiter le problème, ainsi qu'on l'a observé lors de la scission entre
Ethereum et Ethereum Classic. Les être humains sont capables de se
mettre d'accord à long terme, comme le témoigne la convergence vers un
petit nombre de langues, de religions, de monnaies, etc. Néanmoins, à
court terme ce n'est très certainement pas le cas. D'où le recours au
mécanisme de consensus automatisé qu'est le minage.

Une autre mesure proposée, moins subjective mais plus perturbatrice, est
la modification de la fonction de preuve de travail. Celle-ci permet de
faire cesser l'attaque à court terme puisqu'elle rend le matériel
spécialisé des censeurs obsolète, leur faisant supporter une lourde
perte au passage. Il s'agit d'un hard fork (à savoir une modification
incompatible des règles de consensus) qui doit être activé par les
utilisateurs à un horodatage ou à une hauteur de bloc donné,
c'est-à-dire un \emph{User Activated Hard Fork} (UAHF). Cette option
extrême a été soutenue par les développeurs luke-jr et Gregory Maxwell
lors de la guerre des blocs en 2015 -- 2016\footnote{«~soutenue par les
  développeurs luke-jr et Gregory Maxwell~»~:
  \url{https://www.reddit.com/r/Bitcoin/comments/3fg0jw/could_a_cartel_of_pool_operators_collude_to/ctoat0d/}~;\url{https://www.reddit.com/r/bitcoinxt/comments/41pbmf/maxwell_considers_changing_the_pow_algorithm_in/}.}.
Elle a également été défendue par le développeur en chef de Bitcoin ABC
Amaury Séchet en novembre 2018 qui l'a qualifiée d'«~option nucléaire
{[}...{]} de dernier recours\footnote{Amaury Séchet (deadalnix) sur
  Twitter, /11/2018 11:42 UTC~:
  \url{https://twitter.com/deadalnix/status/1061947426096009216}.}~».

De même que dans le cas de l'invalidation de la censure par intervention
sociale, il s'agit d'une mesure plus néfaste à long terme que le statu
quo. Premièrement, la perte subie par les censeurs est aussi encaissée
par les mineurs honnêtes et dissidents. Deuxièmement, l'économie est
répartie entre deux chaînes distinctes, réduisant l'utilité monétaire
totale. Troisièmement, le coût d'une attaque est drastiquement réduit à
court terme. Quatrièmement, les mineurs perdent confiance dans le
protocole et doivent s'assurer contre le risque d'un nouveau changement,
réhaussant le coût de la sécurité minière par rapport au coût de
l'attaque. Et cinquièmement, la nouvelle distribution du minage n'est
pas forcément meilleure que l'ancienne, les gros mineurs pouvant
déployer du capital plus facilement.

De manière générale, l'intervention humaine à court terme est loin
d'être désirable. Si la chaîne subit une attaque minière, il est
probable qu'elle soit aussi attaquée au niveau social. Les interventions
ont ainsi toutes les chances de se multiplier, faisant sombrer la chaîne
dans une spirale de scissions et la menant à l'insignifiance économique.
Le cas de Bitcoin Cash est le plus éclairant~: en raison de hard forks
programmés tous les six mois, la chaîne a subi deux scissions majeures
après sa séparation avec Bitcoin-BTC (en 2018 avec BSV et en 2020 avec
XEC), ce qui a mené l'ensemble à être valorisé à moins de 1~\% de la
valeur agrégée du BTC. En outre, si ce caractère néfaste est valable
pour les cryptomonnaies en construction, qui peuvent se permettre ces
interventions en raison de la petitesse et de l'homogénéité de leur
économie, elle l'est d'autant plus pour une version mature de Bitcoin
qui soutiendrait une économie plus grande et plus diversifiée.

\section*{Les variantes des consensus par preuve de
travail}\label{les-variantes-des-consensus-par-preuve-de-travail}
\addcontentsline{toc}{section}{Les variantes des consensus par preuve de
travail}

\markright{Les variantes des consensus par preuve de travail}

Le risque de censure a également inspiré le développement d'algorithmes
de consensus alternatifs à celui de Bitcoin. L'alternative la plus
connue est la preuve d'enjeu, qui sera décrite dans la section suivante.
Les autres alternatives sont des variantes de l'algorithme de Nakamoto
par preuve de travail, dont les trois principales sont le minage
combiné, la preuve d'espace et la finalisation anticipée.

La première proposition est le minage combiné\footnote{«~le minage
  combiné~»~: Aljosha Judmayer, Alexei Zamyatin, Nicholas Stifter,
  Artemios G. Voyiatzis, Edgar Weippl, \emph{Merged Mining: Curse of
  Cure?}, 22 août 2017~: \url{https://eprint.iacr.org/2017/791}.}. Le
minage combiné, ou \emph{merge mining} en anglais, est l'action de miner
plusieurs chaînes en simultané par la réutilisation du travail fourni
sur une chaîne parente pour la validation des chaînes filles ou
auxiliaires.

Le procédé a été décrit par Satoshi Nakamoto en décembre 2010, dans un
message concernant BitDNS, le projet de système distribué de noms de
domaine à l'origine de Namecoin. Le créateur de Bitcoin écrivait ainsi
sur le forum~:

«~Je pense qu'il serait possible que BitDNS forme un réseau complètement
séparé et possède une chaîne de blocs distincte, tout en partageant la
puissance de calcul avec Bitcoin. Le seul chevauchement consisterait à
faire en sorte que les mineurs puissent rechercher des preuves de
travail pour les deux réseaux simultanément.

Les réseaux n'auraient besoin d'aucune coordination. Les mineurs
adhéreraient aux deux réseaux en parallèle. Ils scanneraient SHA de
telle sorte que s'ils obtenaient un résultat, ils pourraient résoudre
les deux en même temps. Une solution pourrait concerner un seul des
réseaux si l'un d'eux présente une difficulté moindre.

Je pense qu'un mineur externe pourrait appeler getwork sur les deux
programmes et combiner le travail. Peut-être appeler Bitcoin, en tirer
du travail, le remettre à getwork sur BitDNS pour le combiner en un
travail commun.

Au lieu d'une fragmentation, les réseaux partageraient et augmenteraient
la puissance de calcul totale de chacun. Cela résoudrait le problème des
réseaux multiples, qui constituent un danger les uns pour les autres si
la puissance de calcul disponible se concentre sur l'un d'entre eux. Au
lieu de cela, tous les réseaux du monde partageraient la puissance de
calcul combinée, augmentant ainsi la puissance totale. Il serait plus
facile pour les petits réseaux de se lancer en puisant dans une base
existante de mineurs\footnote{Satoshi Nakamoto, \emph{Re: BitDNS and
  Generalizing Bitcoin}, /12/2010 21:02:42 UTC~:
  \url{https://bitcointalk.org/index.php?topic=1790.msg28696\#msg28696}.}.~»

Le minage combiné consiste à réutiliser des preuves de travail
partielles d'une chaîne mère comme des preuves de travail valides sur
une chaîne fille. Ces preuves de travail, dite «~auxiliaires~» et
abrégées en AuxPOW, sont des sous-produits du minage de la chaîne mère,
et ne nécessitent pas de dépense d'énergie supplémentaire. La seule
charge imposée par le minage combiné est la gestion de la chaîne fille.

Les mineurs de la chaîne fille reçoivent des récompenses supplémentaires
qui sont constituées de la création monétaire locale (si la chaîne
utilise une nouvelle unité de compte) et des frais de transaction. Les
mineurs de la chaîne mère sont donc incités à tirer profit de cette
nouvelle manne. La chaîne fille peut de ce fait disposer d'un taux de
hachage conséquent assez rapidement.

Le minage combiné a été mis en avant comme une méthode permettant de
faciliter l'amorçage d'une nouvelle cryptomonnaie, en bénéficiant de
l'industrie minière établie. Ce type d'algorithme de consensus a ainsi
été mis en place sur Namecoin par rapport à Bitcoin et sur Dogecoin par
rapport à Litecoin. Il a aussi été suggéré comme mécanisme de
synchronisation des chaînes latérales. Il est ainsi implémenté de
manière hybride dans RSK. Il est plus largement envisagé par Paul Sztorc
dans sa proposition de Drivechain (voir
chapitre~\hyperref[ch:scalabilite]{14}).

Cependant, l'apport en sécurité du minage combiné par rapport au minage
classique est relativement faible. Le procédé permet d'augmenter le
nombre d'acteurs impliqués et de restreindre les attaquants possibles
(ceux-ci devant être des mineurs de la chaîne principale), mais il ne
modifie pas le coût de l'attaque, qui dépend du revenu minier de cette
chaîne et, dans le cas de la censure, des frais de transaction.

Une illustration éclatante de ce fait est l'exemple de Coiledcoin (CLC),
une cryptomonnaie alternative créée en janvier 2012 qui a subi une
attaque de censure fatale peu de temps après son lancement. L'attaque a
été réalisée par le développeur de Bitcoin luke-jr par le biais de sa
coopérative de minage, Eligius, sans qu'il n'en informe les hacheurs.
Dans son message d'explication, il précisait qu'aucun membre de la
coopérative n'avait subi de perte, le coût étant surtout le temps qu'il
avait passé à configurer le logiciel\footnote{luke-jr, \emph{Re:
  {[}DEAD{]} Coiledcoin - yet another cryptocurrency, but with
  OP\_EVAL!}, /01/2012 18:56:03 UTC~:
  \url{https://bitcointalk.org/index.php?topic=56675.msg678006\#msg678006}.}.

Le minage combiné a deux effets sur la sécurité minière de la chaîne
mère. D'une part, il augmente artificiellement la puissance de calcul
déployée pour miner des blocs, ce qui paraît bénéfique de prime abord.
Cependant, cette hausse artificielle n'agit en rien contre la censure
des transactions. D'autre part, le minage combiné entraîne une
centralisation de l'activité minière, en raison de la charge que
représente la gestion des chaînes auxiliaires~: si les chaînes
auxiliaires deviennent importantes économiquement, les mineurs de la
chaîne mère n'ont d'autre choix que de les miner pour rester rentables.

La deuxième alternative est la preuve d'espace\footnote{«~la preuve
  d'espace~»~: Stefan Dziembowski, Sebastian Faust, Vladimir Kolmogorov,
  Krzysztof Pietrzak, \emph{Proofs of Space}, 2013~:
  \url{https://eprint.iacr.org/2013/796}.} (de l'anglais \emph{proof of
space}), parfois aussi appelée preuve de capacité ou preuve de stockage,
qui se base, non pas sur le calcul informatique, mais sur la capacité à
garder des données en mémoire. La ressource n'est plus la puissance de
calcul, mais l'espace disque.

Cette idée a été partiellement incluse dans certains algorithmes
hybrides de preuve de travail, dans le but de décourager le
développement de matériel spécialisé (ASIC) et de favoriser le minage
par processeurs accessibles au grand public (CPU et GPU). C'est le cas
de la fonction scrypt (ou S-Crypt), une fonction de dérivation de clé
coûteuse en mémoire adaptée par le mineur ArtForz pour être intégrée au
sein de Tenebrix en septembre 2011\footnote{«~intégrée au sein de
  Tenebrix en septembre 2011~»~: Lolcust, \emph{{[}ANNOUNCE{]} Tenebrix,
  a CPU-friendly, GPU-hostile cryptocurrency}, /09/2011 00:09:44 UTC~:
  \url{https://bitcointalk.org/index.php?topic=45667.msg544675\#msg544675}.}.
Celle-ci a été héritée plus tard par Litecoin\footnote{«~héritée plus
  tard par Litecoin~»~: Charlie Lee, \emph{Re: {[}ANN{]} Litecoin - a
  lite version of Bitcoin. Be ready when is launches!}, /10/2011
  06:14:28 UTC~:
  \url{https://bitcointalk.org/index.php?topic=47417.msg564414\#msg564414}.}.
C'est également le cas de l'ancienne fonction de minage d'Ethereum
utilisé entre 2015 et 2022, ETHash, qui est une variante de l'algorithme
Dagger-Hashimoto et qui rend le calcul de la preuve plus coûteux en
mémoire par la nécessité de stocker un graphe acyclique orienté de
plusieurs gigaoctets\footnote{«~ETHash~»~:
  \url{https://ethereum.org/en/developers/docs/consensus-mechanisms/pow/mining-algorithms/ethash/}.}.
Ethereum utilisait de plus une version modifiée de l'algorithme de
Nakamoto, GHOST, qui avait pour intérêt de sélectionner la chaîne la
plus lourde en prenant en compte les blocs orphelins\footnote{«~GHOST~»~:
  \emph{Ethereum Whitepaper}, consulté le 11 mars 2023~:
  \url{https://ethereum.org/en/whitepaper/\#modified-ghost-implementation}.}.
Depuis novembre 2020, Ethereum Classic utilise une variante de ETHash
nommée ETCHash\footnote{«~ETCHash~»~:
  \url{https://github.com/eth-classic/etchash/blob/main/README.md}.}. Un
dernier exemple est l'algorithme RandomX, actif sur Monero depuis 2019,
qui est conçu spécialement pour favoriser le minage par CPU\footnote{«~RandomX~»~:
  \url{https://github.com/tevador/RandomX}.}.

Au-delà des fonctions de preuve de travail coûteuses en mémoire, il
existe des algorithmes de preuve d'espace pure. C'est en pratique le cas
du système Chia Network, projet de Bram Cohen, qui se base sur les
«~preuves d'espace et de temps~» pour déterminer la chaîne
correcte\footnote{«~Chia Network~»~: Ludovic Lars, \emph{Face à la
  preuve de travail de Bitcoin, la preuve d'espace, une fausse solution
  écologique}, 19 mai 2021~:
  \url{https://journalducoin.com/analyses/preuve-espace-fausse-solution-ecologique/}.}.

Ces algorithmes fondés à des degrés divers sur la mémoire informatique
sont censés être plus résistants à la censure en facilitant la
participation du grand public et en améliorant de ce fait la
distribution de la validation. Mais ils ne font que déplacer le
problème. Ce qu'il faut comprendre avec la preuve d'espace, c'est qu'il
s'agit de dépenser de l'énergie extérieure d'une autre manière. La
preuve d'espace est une preuve de travail déguisée~: elle revient en fin
de compte à effectuer une autre forme de travail, qui peut être
optimisée. Cette optimisation peut avoir lieu tant au niveau de la
conception du matériel (ASIC) qu'au niveau de l'organisation
industrielle (économie d'échelle), ce qui fait que les pressions
centralisatrices ne disparaissent pas complètement. Tout ce qu'on peut
espérer, c'est de rapprocher l'efficacité du matériel spécialisé de
celle d'un outil utilisé par tous, comme ce qui est fait par RandomX
avec le CPU.

La troisième alternative est la finalisation anticipée des blocs.
Celle-ci consiste à mettre en place des points de contrôle mobiles au
sein du protocole, de façon à considérer comme final tout bloc qui se
trouverait en-dessous d'une certaine profondeur. Vitalik Buterin parle
de «~subjectivité faible\footnote{Vitalik Buterin, \emph{Proof of Stake:
  How I Learned to Love Weak Subjectivity}, 25 novembre 2014~:
  \url{https://blog.ethereum.org/2014/11/25/proof-stake-learned-love-weak-subjectivity}.}~»
pour décrire ce type de mécanisme.

Un tel algorithme a été mis en place par Bitcoin ABC le 20 novembre 2018
au sein de Bitcoin Cash, face à la menace d'attaque de la part du camp
de Bitcoin SV, sous la forme d'une protection contre la recoordination
profonde, qui consistait à considérer un bloc comme final au bout de 11
confirmations\footnote{«~protection contre la recoordination
  profonde~»~: Bitcoin ABC, \emph{Bitcoin ABC 0.18.5 Released}, 20
  novembre 2018~:
  \url{https://www.bitcoinabc.org/2018-11-20-bitcoin-abc-0-18-5/}.}. Ce
procédé est encore présent dans certaines implémentations de Bitcoin
Cash et de XEC, et est appliqué par les grandes plateformes de change,
ce qui en fait \emph{de facto} une règle de consensus.

Dans Ethereum Classic, qui a subi de multiples attaques de double
dépense en 2019 et en 2020, une variante de cette finalisation a été
intégrée le 11 octobre 2020. L'algorithme en question est appelé
\emph{Modified Exponential Subjective Scoring} (MESS) et consiste à
attribuer différents scores aux branches concurrentes, privilégiant les
segments vus les premiers aux segments vus ultérieurement. Il
permettrait de diviser le coût d'une attaque par 31\footnote{«~MESS~»~:
  Dean Pappas, \emph{An Elegant MESS -- The Fast Solution to 51\%
  attacks and Low Hash Rate}, 18 septembre 2020~:
  \url{https://medium.com/ethereum-classic-labs/an-elegant-mess-the-fast-solution-to-51-attacks-and-low-hash-rate-4e8f8347bdfe}.}.

Ces algorithmes réduisent effectivement la probabilité d'une attaque
opportuniste, car ils empêchent les recoordinations. Cependant, ils ont
le résultat inverse sur les attaques de censure dont le but est de
détruire l'utilité fondamentale de la chaîne. Ces algorithmes sont en
effet sujets au problème de la subjectivité. Un nouveau nœud qui se
synchronise avec le réseau peut être trompé par un attaquant en suivant
la chaîne la plus longue et non la chaîne considérée comme valide par le
reste du réseau. De ce fait, un attaquant (réalisant une attaque
Goldfinger) peut facilement tirer profit de cette caractéristique en
créant des chaînes concurrentes plus longues pour causer la
confusion\footnote{Ce problème peut être atténué par une intervention
  sociale en décrétant un certain nombre de blocs comme valides par
  défaut. Mais on en revient alors à la situation discutée dans la
  section précédente.}.

Idéalement, le concept de Bitcoin n'intègre aucun point de contrôle à
l'exception du bloc de genèse défini préalablement, et la chaîne
correcte est déterminée uniquement par la quantité de travail accumulée.
Bien qu'il ait lui-même ajouté des points de contrôle manuels, Satoshi
Nakamoto expliquait~:

«~Il n'y a aucun moyen pour le logiciel de savoir automatiquement si une
chaîne est meilleure qu'une autre, sauf par la plus grande preuve de
travail. Dans la conception, il était nécessaire qu'il se tourne vers la
chaîne plus longue, quelle que soit la distance à
parcourir\footnote{À propos de l'acceptation de la plus longue chaîne
  par le logiciel, Satoshi ajoutait~: «~La seule exception à cela, ce
  sont les points de contrôle manuels que j'ai ajoutés. S'ils n'étaient
  pas là, il serait capable de se recoordonner en remontant jusqu'au
  premier bloc.~» -- Voir Satoshi Nakamoto, \emph{Re: checkpointing the
  block chain}, /08/2010 20:20:53 UTC~:
  \url{https://bitcointalk.org/index.php?topic=834.msg9816\#msg9816}.}.~»

\section*{La preuve d'enjeu}\label{la-preuve-denjeu}
\addcontentsline{toc}{section}{La preuve d'enjeu}

\markright{La preuve d'enjeu}

L'autre alternative à l'algorithme de Nakamoto par preuve de travail est
le recours à un autre mécanisme de résistance aux attaques Sybil~: la
preuve d'enjeu. La preuve d'enjeu, de l'anglais \emph{proof of stake},
est un procédé permettant à quelqu'un de démontrer son implication dans
un système par le biais d'un algorithme de signature, dans le cadre de
l'accès à un privilège. Dans le cas des systèmes cryptoéconomiques
gérant une unité de compte numérique, elle intervient dans le choix des
validateurs en charge de produire les blocs de transactions. Le
validateur d'un bloc donné est alors sélectionné par le réseau selon le
nombre d'unités qu'il met en jeu (ou selon un autre paramètre lié). La
preuve d'enjeu est parfois décrite comme du «~minage virtuel~» car les
jetons numériques jouent le même rôle que l'énergie électrique dans les
algorithmes basés sur la preuve de travail, la probablité de valider un
bloc étant la plupart du temps proportionnelle au nombre de jetons en
possession du validateur.

Les jetons du validateur sont mis en jeu dans le sens où ils sont
bloqués par le système et où ils sont détruits en cas de comportement
hostile au réseau. Cette dernière propriété permet d'éviter le problème
du «~rien à perdre~» (\emph{nothing-at-stake problem}) qui se poserait
dans le cas d'une mise en œuvre naïve du procédé, dans laquelle les
validateurs peuvent valider plusieurs chaînes concurrentes en même
temps, contrairement à la preuve de travail où l'énergie ne peut pas
être dupliquée. Par exemple, l'algorithme de consensus d'Ethereum,
Casper FFG, met en place une «~coupe des fonds~» (ou
\emph{slashing}\footnote{«~slashing~»~: Vitalik Buterin, \emph{Slasher:
  A Punitive Proof-of-Stake Algorithm}, 15 janvier 2014~:
  \url{https://blog.ethereum.org/2014/01/15/slasher-a-punitive-proof-of-stake-algorithm}.})
pour sanctionner progressivement les validateurs qui ne respectent pas
les règles de bonne conduite\footnote{Vitalik Buterin et al.,
  \emph{Combining GHOST and Casper}, 11 mai 2020~:
  \url{https://arxiv.org/pdf/2003.03052.pdf}.}. Cela permet au réseau de
se prémunir contre les attaques de courte portée. De plus, la preuve
d'enjeu étant subjective, elle nécessite des points de contrôles, qui
séparent différentes «~époques~», pour contrer les attaques de longue
portée.

L'idée de la preuve d'enjeu est une vieille idée puisqu'on la retrouve
dans la conception de b-money, le système imaginé par le cypherpunk Wei
Dai en 1998 et décrit dans le chapitre~\hyperref[ch:cybermonnaie]{6}.
Dans son modèle, chaque serveur devait déposer un certain montant de
b-money sur un compte spécial pour participer aux opérations du réseau.
Le montant servait de garantie pour pénaliser le serveur en cas de
mauvaise conduite.

Le terme «~\emph{proof of stake}~» a été inventé en juillet 2011 par un
membre du forum de Bitcoin utilisant le pseudonyme QuantumMechanic, qui
décrivait comment le concept pouvait être adapté aux systèmes
cryptomonétaires\footnote{QuantumMechanic, \emph{Proof of stake instead
  of proof of work}, /07/2011 04:12:45 UTC~:
  \url{https://bitcointalk.org/index.php?topic=27787.msg349645\#msg349645}.}.
Cette idée a été mise en œuvre un an plus tard, en août 2012, par Sunny
King et Scott Nadal, par le biais de leur protocole PPCoin\footnote{«~par
  le biais de leur protocole PPCoin~»~: Sunny King, Scott Nadal,
  \emph{PPCoin: Peer-to-Peer Crypto-Currency with Proof-of-Stake}, 19
  août 2012, archive~:
  \url{https://web.archive.org/web/20121021014644/http://www.ppcoin.org/static/ppcoin-paper.pdf}.}.
Ce dernier se basait sur un modèle hybride combinant énergie électrique
et âge des pièces (preuve de conservation) pour sa validation. Il est
aujourd'hui connu sous le nom de Peercoin.

De même que la preuve de travail peut être étendue en preuve de mémoire,
la preuve d'enjeu peut être dérivée en plusieurs variantes. La preuve
d'enjeu déléguée\footnote{«~preuve d'enjeu déléguée~»~: Dan Larimer,
  \emph{DPOS Consensus Algorithm - The Missing White Paper}, 29 mai
  2017~:
  \url{https://hive.blog/dpos/@dantheman/dpos-consensus-algorithm-this-missing-white-paper}.}
prend ainsi en compte les jetons possédés mais aussi les jetons délégués
aux validateurs. Il s'agit de la variante la plus répandue. Elle permet
de mettre en place une preuve d'enjeu liquide (à la Tezos\footnote{«~preuve
  d'enjeu liquide~»~: Jacob Arluck, \emph{Liquid Proof-of-Stake}, 30
  juillet 2018~:
  \url{https://medium.com/tezos/liquid-proof-of-stake-aec2f7ef1da7}.}),
mais a néanmoins pour inconvénient de centraliser la validation. Il
existe également d'autres variantes comme la preuve de conservation
(Peercoin), la preuve de vélocité (Reddcoin) ou la preuve d'importance
(NEM).

De manière générale, on peut regrouper les mécanismes de résistance aux
attaques Sybil des systèmes ouverts en deux catégories de preuve~: les
preuves externes, basées sur l'utilisation de l'énergie dans le monde
physique, et les preuves internes, basées sur l'état du registre
virtuel. Il y a ainsi une auto-référence dans le cas de la preuve
d'enjeu, ce qui peut poser problème.

Les défenseurs de la preuve d'enjeu prétendent que la preuve d'enjeu est
plus sécurisée, car le coût d'une attaque est un ordre de grandeur plus
élevé\footnote{«~le coût d'une attaque est un ordre de grandeur plus
  élevé~»~:
  \url{https://ethereum.org/en/developers/docs/consensus-mechanisms/pos/pos-vs-pow/\#security}.}.
Une attaque de censure pourrait en outre faire baisser le prix de
l'unité de compte, ce qui provoquerait une baisse de valeur du capital
de l'attaquant. Nous affirmons l'inverse~: la preuve d'enjeu offre une
résistance à la censure moins forte que la preuve de travail.

Tout d'abord, réunir les jetons nécessaires est loin d'être une tâche
impossible. Premièrement, tous les détenteurs ne sont pas impliqués dans
le consensus, ce qui veut dire que seule la portion des jetons mis en
jeu est concernée. Deuxièmement, contrairement à la preuve de travail
qui exige 51~\% de la puissance de calcul pour perturber le système, le
plupart des algorithmes par preuve d'enjeu sont des algorithmes
classiques dont l'attaque ne nécessite que 34~\% des fonds en jeu.
Troisièmement, une grande partie des jetons sont conservés par des
acteurs centralisés qui offrent généralement des services de
\emph{staking} (incitant l'accumulation), et qui sont réglementés et
donc particulièrement sensibles à la cooptation étatique.

Ensuite, un défaut de la preuve d'enjeu est qu'elle permet une meilleure
identification du validateur, associé à une clé publique liée aux fonds
sous séquestre, que dans le cas de la preuve de travail, où les mineurs
peuvent diriger leur puissance de calcul vers la chaîne libre plus
discrètement. La validation par preuve d'enjeu est donc moins
confidentielle que le minage qui est complètement anonyme par
conception.

Enfin, et surtout, la principale raison pour laquelle la preuve d'enjeu
produit une résistance à la censure plus faible est le caractère interne
de la preuve. Dans le cas de la preuve de travail, il est toujours
possible de combattre la censure~: il suffit de réunir une puissance de
calcul supérieure aux censeurs, en construisant des machines et en
apportant une énergie supplémentaire. Dans le cas de la preuve d'enjeu,
il n'est pas possible de créer de nouvelles unités sans modifier les
règles de consensus de sorte que les censeurs, qui contrôlent une
majorité des jetons existants et touchent par conséquent une majorité
des jetons créés, sont intouchables.

Pour répondre à ce problème, les partisans de la preuve d'enjeu sur
Ethereum prônent généralement le recours à l'accord social. Il ne s'agit
pas seulement de sélectionner la chaîne valide manuellement comme nous
l'avons expliqué précédemment, mais de rééquilibrer la distribution des
jetons de façon à retrouver un système de validation qui ne censure pas.
Puisque la création d'unités supplémentaires pose la question épineuse
de la destination desdites unités, ce rééquilibrage consiste plutôt à
détruire les fonds mis en jeu par les censeurs, une mesure appelée le
\emph{slashing} social\footnote{Eric Wall, \emph{The Case for Social
  Slashing}, 22 août 2022~:
  \url{https://ercwl.medium.com/the-case-for-social-slashing-59277ff4d9c7}.}.
Ce recours est notamment soutenu par Vitalik Buterin, qui écrivait la
chose suivante en 2020~:

«~Pour d'autres attaques plus difficiles à détecter (notamment une
coalition de 51~\% censurant tous les autres), la communauté peut se
coordonner pour réaliser un soft fork activé par les utilisateurs (UASF)
minoritaire dans lequel les fonds de l'attaquant sont {[}...{]}
largement détruits (dans Ethereum, cela se fait via le ``mécanisme de
fuite d'inactivité''). Aucun ``hard fork pour supprimer les pièces''
explicite n'est nécessaire~; à l'exception de la nécessité de coordonner
l'UASF pour sélectionner un bloc minoritaire, tout le reste est
automatisé et suit simplement l'exécution des règles du
protocole\footnote{Vitalik Buterin, \emph{Why Proof of Stake (Nov
  2020)}, 6 novembre 2020~:
  \url{https://vitalik.ca/general/2020/11/06/pos2020.html}.}.~»

À l'heure d'écriture de ces lignes, la mesure n'a jamais été appliquée
sur Ethereum. Le cas qui s'en rapproche le plus est le contentieux entre
la Fondation Tron de Justin Sun et la communauté historique de Steem qui
s'est conclu par le gel des fonds de la première par une intervention
externe de la communauté en mars 2020. Cette intervention a provoqué une
scission entre le protocole Steem contrôlé par la Fondation Tron et la
plateforme Hive\footnote{«~ une scission entre le protocole Steem
  contrôlé par la Fondation Tron et la plateforme Hive~»~: Tim Copeland,
  \emph{Steem vs Tron: The rebellion against a cryptocurrency empire},
  18 août 2020~:
  \url{https://decrypt.co/38050/steem-steemit-tron-justin-sun-cryptocurrency-war}.}.

Le recours à l'accord social paraît une nouvelle fois être une bonne
idée. Cependant, il s'agit clairement de jouer avec le feu~: le risque
de créer la confusion et de provoquer une scission est largement
sous-estimé. De manière générale, c'est ce qui différencie la
philosophie derrière la preuve d'enjeu de celle de la preuve de travail.
Les défenseurs de la preuve d'enjeu ne modélisent pas la menace de la
même manière, et c'est pourquoi le modèle de sécurité de Bitcoin est
bien plus exigeant que celui d'Ethereum.

\section*{Consommation d'énergie et résistance à la
censure}\label{consommation-duxe9nergie-et-ruxe9sistance-uxe0-la-censure}
\addcontentsline{toc}{section}{Consommation d'énergie et résistance à la
censure}

\markright{Consommation d'énergie et résistance à la censure}

Ainsi, la preuve de travail joue un rôle essentiel dans la résistance à
la censure de Bitcoin. Tout le génie de Nakamoto réside dans le fait
d'avoir découvert un mécanisme de consensus basé sur une grandeur
objective extérieure au système, qui permette la résolution de la
censure sans intervention humaine au niveau du protocole, même face à
une attaque étatique.

Il s'avère que la mise en œuvre de cette preuve de travail consomme une
importante quantité d'énergie électrique. Mais c'est cette consommation
qui ancre le protocole dans le réel et c'est donc le prix à payer pour
disposer d'un système réellement résistant à la censure. Elle peut être
réduite, mais elle ne peut pas être évitée.

La consommation d'énergie est l'un des arguments d'opposition à Bitcoin
les plus récurrents, en raison de son supposé impact
écologique\footnote{La première critique de la consommation d'énergie de
  Bitcoin a été faite par l'ancien cypherpunk John Gilmore en janvier
  2009~: «~La dernière chose dont nous avons besoin est de déployer un
  système conçu pour brûler tous les cycles disponibles, consommant de
  l'électricité et générant du dioxyde de carbone, partout sur internet,
  afin de produire de petites quantités de dollars binaires pour faire
  passer des courriels ou des spams.~» -- John Gilmore, \emph{Proof of
  Work -\textgreater{} atmospheric carbon}, /01/2009 22:40:45~:
  \url{https://www.metzdowd.com/pipermail/cryptography/2009-January/015042.html}.}.
Au vu de ce que nous avons dit dans ce chapitre, cette opposition de
façade, loin de lutter contre la consommation d'énergie de la
cryptomonnaie, contribue à renforcer le conflit qui existe entre le
contrôle financier et la résistance à la censure, et par conséquent à
augmenter l'énergie consommée des deux côtés. C'est pourquoi une bonne
façon de réduire la consommation d'énergie de Bitcoin serait de prôner
une plus grande concurrence monétaire et bancaire en vue de diminuer son
utilité réelle et potentielle.

La proposition de l'abandon de la preuve de travail, telle que celle
faite par Greenpeace en 2022\footnote{«~proposition de l'abandon de la
  preuve de travail, telle que celle faite par Greenpeace en 2022~»~:
  Tyler Kruse, \emph{Change The Code: Not The Climate --- Greenpeace
  USA, EWG, Others Launch Campaign to Push Bitcoin to Reduce Climate
  Pollution}, 29 mars 2022~:
  \url{https://www.greenpeace.org/usa/news/change-the-code-not-the-climate-greenpeace-usa-ewg-others-launch-campaign-to-push-bitcoin-to-reduce-climate-pollution/}.},
s'inscrit donc dans la deuxième catégorie d'attaques contre Bitcoin, à
savoir les attaques sociales. Heureusement, Bitcoin dispose également
d'un mécanisme de défense à ce niveau-là. Dans les chapitres suivants,
nous décrirons comment le protocole peut être modifié et quels principes
sous-jacents sont à l'œuvre dans sa détermination.

\bookmarksetup{startatroot}

\chapter{Le changement de la monnaie}\label{ch:changement}

\phantomsection\label{enotezch:10}{}

{U}\textsc{n}e monnaie est un accord concernant un moyen mutuellement
acceptable dans le commerce. Cet accord peut porter sur des propriétés
physiques, auquel cas le support monétaire est une marchandise, ou des
propriétés numériques, auquel cas le support monétaire est un protocole
informatique. Bitcoin appartient à cette seconde catégorie.

Par sa nature ouverte et libre, le code informatique de Bitcoin peut
être copié, modifié et réutilisé à volonté. Par conséquent, le protocole
(et la monnaie qu'il définit) peut lui aussi être changé, grâce à
l'application d'un code différent sur le réseau. Bitcoin n'est ainsi pas
un système figé qui serait géré par une autorité centrale, mais une
structure ouverte qui connaît une évolution organique au cours du temps.

\section*{Le protocole}\label{le-protocole}
\addcontentsline{toc}{section}{Le protocole}

\markright{Le protocole}

Bitcoin est par essence un protocole de communication informatique,
c'est-à-dire un ensemble de règles permettant à différentes parties d'un
réseau d'échanger des informations. Ce protocole permet aux nœuds du
réseau pair à pair de s'échanger des transactions et des blocs et de se
mettre d'accord sur le registre de propriété considéré comme correct. Le
résultat est un système monétaire.

Bitcoin se rapproche, de façon plus ou moins manifeste de protocoles
existants. C'est par exemple le cas d'autres protocoles construits sur
Internet, comme HTTP (\emph{HyperText Transfer Protocol}) qui est
utilisé pour l'affichage des pages web, SMTP (\emph{Simple Mail Transfer
Protocol}) qui est utilisé pour le courrier électronique, ou encore
BitTorrent, qui permet le partage de fichiers de pair à pair. C'est
également le cas des protocoles qui soutiennent Internet, appelés
protocoles de la suite TCP/IP en référence aux deux premiers qui la
composent~: IP (\emph{Internet Protocol}) qui assure la communication au
niveau de la couche réseau, et TCP (\emph{Transmission Control
Protocol}) qui assure la transmission au niveau de la couche transport,
en surcouche de la couche réseau.

Plus éloigné de Bitcoin, on peut citer la catégorie des langages de
programmation. Ces langages permettent d'écrire du code (texte
spécifique encodé en UTF-8), qui est transformé en fichier exécutable
par un compilateur (par exemple dans le cas du C, du C++ ou du Java) ou
qui est directement exécuté par un interpréteur (comme c'est le cas pour
Python ou Javascript). Dans le même ordre d'idées, les langues humaines
comme le français ou l'anglais sont aussi des protocoles de
communication, dont les règles sont moins formelles et moins bien
définies, mais qui permettent aux hommes d'échanger des informations.

Enfin, les monnaies peuvent être vues comme des sortes de protocole, en
constituant des moyens communs de communiquer de la valeur et de
formaliser l'échange économique. La monnaie se définit en particulier
par le support accepté dans le commerce~: pour une marchandise comme
l'or ou l'argent, ce support est un élément chimique~; pour la monnaie
fiat, il s'agit d'un certificat émis par une autorité.

Dans le cas de Bitcoin, le protocole est formé de l'ensemble des règles
qui permettent au réseau de communiquer et de se coordonner. Ce
protocole se divise en deux parties distinctes~: le protocole de
transmission, constitué des règles de réseau, et le protocole régissant
le contenu transmis, constitué des règles de consensus.

Les règles de réseau sont les règles qui permettent aux nœuds d'entrer
en communication sur Internet. Ces règles concernent le protocole de
transport sous-jacent (TCP, Tor, UDP pour FIBRE), le port réseau (8333
pour le réseau principal BTC), la procédure de découverte de pairs, la
syntaxe des messages de transmission de données\footnote{«~la syntaxe
  des messages de transmission de données~»~: Bitcoin Wiki,
  \emph{Protocol documentation: Common structures}~:
  \url{https://en.bitcoin.it/wiki/Protocol_documentation\#Common_structures}.},~etc.
Elles peuvent différer selon les nœuds sans briser formellement le
consensus~: il suffit qu'un nœud acceptant les deux ensembles de règles
fasse la liaison. De même, les nœuds sont libres de restreindre
(temporairement ou définitivement) leur connexion avec un autre nœud,
notamment dans le but d'éviter le spam.

Les règles de consensus sont les règles de construction et
d'organisation des blocs et des transactions. Elles régissent la
validité du registre sur lequel les membres du réseau arrivent à un
accord, d'où leur nom. Ces règles sont critiques~: un nœud qui
transmettrait une transaction ou un bloc invalide aux autres nœuds
verrait sa transaction ou son bloc être rejeté par le reste du réseau.

Les règles de consensus sont nombreuses. Certaines d'entre elles sont
largement connues et explicites. En voici quelques-unes ici~:

\begin{itemize}
\item
  Le montant en entrée d'une transaction doit être supérieur (ou égal)
  au montant en sortie, la différence représentant les frais collectés
  par le mineur~;
\item
  Chaque entrée doit contenir un script de déverrouillage (contenant la
  ou les signatures) qui correspond au script de verrouillage (l'adresse
  d'envoi) de la sortie dépensée~;
\item
  Une sortie transactionnelle ne peut être dépensée qu'une seule fois,
  en raison de l'interdiction de double dépense~;
\item
  Chaque bloc doit comporter une preuve de travail, produite par
  hachages répétés de l'entête par la fonction SHA-256, de degré
  supérieur à la difficulté du réseau~;
\item
  La subvention dans chaque bloc doit être inférieure à une limite, qui
  est divisée par deux tous les 210~000 blocs (4 ans environ)~;
\item
  La difficulté du minage est ajustée tous les 2016 blocs (2 semaines
  environ), de sorte à garantir un temps moyen de 10 minutes entre
  chaque bloc~;
\item
  Le poids des blocs est limité à 4 millions d'unités de poids (telles
  que définies par SegWit), ce qui restreint la capacité
  transactionnelle du système.
\end{itemize}

Les règles de consensus sont trop nombreuses pour être toutes
explicitées. Quand elles ne le sont pas, ces règles sont implicitement
définies dans l'implémentation logicielle de référence, qui est Bitcoin
Core dans le cas de BTC.

\section*{Les implémentations
logicielles}\label{les-impluxe9mentations-logicielles}
\addcontentsline{toc}{section}{Les implémentations logicielles}

\markright{Les implémentations logicielles}

Les implémentations logicielles sont les programmes informatiques qui
mettent en œuvre le protocole. Dans le cas des implémentations de nœud
complet, la totalité des règles de consensus sont appliquées. Les
implémentations peuvent également être partielles, auquel cas elles ne
mettent pas en œuvre l'intégralité des règles de consensus~: c'est par
exemple le cas des portefeuilles légers, qui procèdent à une
vérification simplifiée de leurs transactions.

Dans BTC, il existe plusieurs implémentations, dont Bitcoin Core,
Libbitcoin, btcd et Bitcoin Knots. La plus connue est Bitcoin Core, qui
est à la fois l'implémentation historique créée par Satoshi Nakamoto
(«~\emph{Satoshi client}~») et reprise par Gavin Andresen en 2010,
l'implémentation principale utilisée par plus de 99~\% des nœuds en
novembre 2023\footnote{«~l'implémentation principale utilisée par plus
  de 99~\% des nœuds en novembre 2023~»~:
  \url{https://coin.dance/nodes}.}, et l'implémentation de référence,
qui définit les règles de consensus implicites.

D'autres protocoles possèdent des implémentations différentes. Bitcoin
Cash présente une multiplicité d'implémentations dont les deux
principales sont Bitcoin Cash Node (l'implémentation de référence issue
de Bitcoin ABC, elle-même issue de Bitcoin Core) et Bitcoin Unlimited.
Ethereum repose également sur une diversité
d'implémentations\footnote{«~une diversité d'implémentations~»~:
  \url{https://clientdiversity.org/\#distribution}.}, qui gèrent la
transmission et la vérification des transactions (Geth, Nethermind,
etc.) ou celles des blocs (Prysm, Lighthouse, etc.)

Une implémentation est en règle générale un logiciel libre, c'est-à-dire
un logiciel dont le code est publié en accès libre sous une licence
permettant l'utilisation, la modification et la reproduction. Cette
caractéristique, technique et juridique, est \emph{essentielle} à
Bitcoin, car elle permet non seulement de vérifier le fonctionnement du
logiciel\footnote{«~Le code source ouvert signifie que n'importe qui
  peut examiner le code de manière indépendante. S'il s'agissait d'une
  source fermée, personne ne pourrait vérifier la sécurité. Je pense
  qu'il est essentiel pour un programme de cette nature d'avoir un code
  source ouvert.~» -- Satoshi Nakamoto, \emph{Re: Questions about
  Bitcoin}, /12/2009 20:49:02 UTC~:
  \url{https://bitcointalk.org/index.php?topic=13.msg46\#msg46}.}, mais
aussi de reprendre la main sur le code dans le cas où les développeurs
iraient dans une direction non désirée.

L'action de copier et de modifier un logiciel est appelé un \emph{fork}
ou embranchement. Il s'agit de créer un nouveau logiciel à partir du
code source d'un logiciel existant, dont l'existence découle d'une
vision différente du développement de ce logiciel. Les distributions
Linux sont ainsi formées de distributions antérieures\footnote{«~Les
  distributions Linux sont ainsi formées de distributions
  antérieures~»~: Andreas Lundqvist, Donjan Rodic, Mohammed A. Mustafa,
  Muhammad Herdiansyah, Fabio Loli, \emph{Linux Distribution Timeline},
  27 février 2021~:
  \url{https://commons.wikimedia.org/wiki/File:Linux_Distribution_Timeline_27_02_21.svg}.}.
On peut aussi citer OpenOffice.org qui a donné LibreOffice et Apache
OpenOffice.

Bitcoin Core descend directement de la première implémentation codée par
Satoshi Nakamoto et partagée publiquement par ce dernier le 8 janvier
2009. Initialement appelé simplement «~Bitcoin~», le logiciel a été
renommé en bitcoind~/~Bitcoin-Qt en 2011\footnote{«~le logiciel a été
  renommé en bitcoind~/~Bitcoin-Qt en 2011~»~: Gavin Andresen,
  \emph{Bitcoin-Qt/bitcoind version 0.5.0}, /11/2011 17:17:04 UTC~:
  \url{https://bitcointalk.org/index.php?topic=52480.msg626275\#msg626275}.},
puis en Bitcoin Core le 19 mars 2014\footnote{«~en Bitcoin Core le 19
  mars 2014~»~: Bitcoin Core, \emph{Bitcoin Core version 0.9.0
  released}, 19 mars 2014~:
  \url{https://bitcoin.org/en/release/v0.9.0\#rebranding-to-bitcoin-core}.}.

Bitcoin Core est un logiciel codé en C++. Initialement hébergé sur
SourceForge, le code est aujourd'hui présent sur GitHub\footnote{\emph{Bitcoin
  Core integration/staging tree}~:
  \url{https://github.com/bitcoin/bitcoin}.}. Il est publié sous licence
libre MIT, de sorte que quiconque peut le copier et le modifier à sa
guise. En particulier, la licence MIT est permissive~: elle n'empêche
pas la réutilisation du code comme partie ou comme base d'un logiciel
soumis à une licence privative. Cette licence a été choisie par Satoshi,
au détriment de la licence GPL, en raison de sa compatibilité avec les
autres licences\footnote{Satoshi Nakamoto, \emph{Re: Switch to GPL},
  /09/2010 19:24:53 UTC~:
  \url{https://bitcointalk.org/index.php?topic=989.msg12494\#msg12494}.}.

Le développement de Bitcoin Core se fait de manière ouverte et
méritocratique. Le dépôt GitHub est ouvert à tous et n'importe qui peut
contribuer au maintien et à l'amélioration du logiciel en faisant une
demande de modification du code (\emph{pull request}). Les contributeurs
fréquents sont appelés des «~\emph{core developers}~». Pour faciliter le
développement, les contributeurs communiquent par différents moyens,
mais les deux principaux sont le canal IRC bitcoin-core-dev où ont lieu
la plupart des discussions et la liste de diffusion
bitcoin-dev\footnote{«~la liste de diffusion bitcoin-dev~»~: The
  bitcoin-dev Archives~:
  \url{https://lists.linuxfoundation.org/pipermail/bitcoin-dev/}.}.

Toutefois, Bitcoin Core dispose d'une certaine hiérarchie. Le dépôt est
en effet géré par des mainteneurs qui sont responsables de fusionner les
demandes de modification créées par les contributeurs. L'inclusion dans
le code dépend ainsi de différents critères évalués par ces mainteneurs,
comme l'utilité démontrable du changement, le format correct suivant les
lignes directrices du projet, la revue par les pairs ou la réputation du
contributeur\footnote{«~Les mainteneurs prendront en considération un
  correctif s'il est en accord avec les principes généraux du projet~;
  s'il répond aux normes minimales d'inclusion~; et jugeront du
  consensus général des contributeurs.~» -- \emph{Contributing to
  Bitcoin Core}, 26 mai 2023~:
  \url{https://github.com/bitcoin/bitcoin/blob/25.x/CONTRIBUTING.md}.}.

La charge du logiciel était initialement allouée à un mainteneur
principal, qui avait pour rôle de nommer les mainteneurs normaux, de
décider du cycle de sortie du logiciel, de fusionner l'ensemble des
modifications et de modérer les débats. Cette mission était assurée au
début par Satoshi Nakamoto qui s'occupait d'intégrer les contributions
sur le dépôt SourceForge. Puis, le 23 février 2011, Satoshi a transmis
la responsabilité à Gavin Andresen, avant de disparaître définitivement.
Gavin s'est ensuite chargé du projet pendant plus de trois ans avant de
laisser sa place à Wladimir J. van der Laan le 7 avril 2014. Enfin, le 7
février 2023, ce dernier a démissionné après neuf ans de service. La
fonction de mainteneur principal a alors été supprimée et remplacée par
la responsabilité collective des mainteneurs\footnote{Wladimir J. van
  der Laan, \emph{The widening gyre}, 21 janvier 2021, archive~:
  \url{https://web.archive.org/web/20210121201607/https://laanwj.github.io/2021/01/21/decentralize.html}~;
  Wladimir J. van der Laan, \emph{Remove laanwj from trusted-keys (git
  commit)}, /02/2023 09:12 UTC~:
  \url{https://github.com/bitcoin/bitcoin/commit/aafa5e945cef7a4f65ddadcf548932dd4e27ada1}.}.

En novembre 2023, les mainteneurs de Bitcoin Core étaient au nombre de
cinq~: Michael Ford, Hennadii Stepanov, Andrew Chow, Gloria Zhao et Ryan
Ofsky\footnote{«~Ryan Ofsky~»~:
  \url{https://github.com/bitcoin/bitcoin/commit/59ebee3fb4181baf20fab263cf1b587ece1bd5e2}.}.
Ils suivent la voie de mainteneurs emblématiques (hors mainteneurs
principaux) comme Martti Malmi, Laszlo Hanyecz, Chris Moore, Pieter
Wuille, Jeff Garzik, Nils Schneider, Gregory Maxwell, Jonas Schnelli,
Samuel Dobson ou Marco Falke\footnote{«~la voie de mainteneurs
  emblématiques~»~: Andrew Chow, \emph{List of people who have had
  commit access to Bitcoin Core}, /07/2022 20:05:39 UTC~:
  \url{https://bitcointalk.org/index.php?topic=1774750.msg17700787\#msg17700787}.}.
Parmi les contributeurs actifs qui n'ont jamais été mainteneurs, on
retrouve Matt Corallo, practicalswift, luke-jr et John Newbery. Les
empreintes PGP des mainteneurs sont disponibles publiquement sur le
dépôt.

Ce fonctionnement ouvert donne au logiciel une sûreté plus grande que la
plupart des programmes informatiques. En effet, au vu des sommes en jeu,
la récompense pour l'exploitation réussie d'une faille majeure serait
énorme, si bien qu'on peut supposer qu'une telle faille n'a pas été
découverte. S'il y a effectivement des vulnérabilités dans le logiciel,
celles-ci sont très rares et très subtiles, de sorte qu'elles sont
généralement découvertes par des développeurs bienveillants, à l'instar
du développeur Awemany qui avait, en septembre 2018, divulgué de manière
responsable une faille inflationniste dans le code\footnote{Awemany,
  \emph{600 Microseconds}, 21 septembre 2018~:
  \url{https://medium.com/@awemany/600-microseconds-b70f87b0b2a6}.}.
Ainsi, le passage du temps renforce la confiance qu'on peut avoir dans
le logiciel (ainsi que dans le système) conformément à l'effet
Lindy\footnote{Comme le faisait remarquer Hal Finney en 2011~: «~Chaque
  jour qui passe sans que Bitcoin ne s'effondre en raison de problèmes
  juridiques ou techniques apporte de nouvelles informations au marché.
  Cela augmente les chances de succès de Bitcoin et justifie un prix
  plus élevé.~» -- Hal Finney, \emph{Re: Bitcoin and the Efficient
  Market Hypothesis}, /06/2011 23:36:04 UTC~:
  \url{https://bitcointalk.org/index.php?topic=11765.msg169026\#msg169026}.}.

\section*{Les propositions d'amélioration de
Bitcoin}\label{les-propositions-damuxe9lioration-de-bitcoin}
\addcontentsline{toc}{section}{Les propositions d'amélioration de
Bitcoin}

\markright{Les propositions d'amélioration de Bitcoin}

Les implémentations peuvent être mises à jour par leurs développeurs,
auquel cas elles ont chacune leur modèle de décision. Dans Bitcoin Core,
comme on l'a dit, tout le monde peut proposer une modification du code
mais le dernier mot est laissé aux développeurs. De même, les
changements internes liés aux portefeuilles sont gérés par leurs
développeurs propres.

Il existe néanmoins une façon de proposer des modifications pouvant
s'appliquer à toutes les implémentations~: les propositions
d'amélioration de Bitcoin (en anglais \emph{Bitcoin Improvement
Proposals} ou BIP), qui sont des documents décrivant des changements
possibles du protocole ou fournissant des informations générales à la
communauté. Ce système des BIP a été formalisé par Amir Taaki en 2011,
sur la base des \emph{Python Enhancement Proposals} (PEP) qui servent à
améliorer le langage de programmation Python. Initialement défini par le
BIP-1, le procédé est aujourd'hui décrit par le BIP-2, rédigé par
luke-jr. Il est hébergé sur un dépôt GitHub géré par Bitcoin
Core\footnote{«~hébergé sur un dépôt GitHub géré par Bitcoin Core~»~:
  \url{https://github.com/bitcoin/bips}}.

Les BIP peuvent être répartis en trois types~: le BIP de suivi de
standard (\emph{standards track BIP}), qui concerne les changements qui
affectent la plupart ou toutes les implémentations de Bitcoin~; le BIP
informationnel (\emph{informational BIP}), qui décrit un problème dans
la conception de Bitcoin ou donne des directives générales ou des
informations à la communauté de Bitcoin, mais ne propose pas de nouvelle
fonctionnalité~; le BIP de procédure (\emph{process BIP}), qui décrit
une procédure ou un changement de procédure à adopter. Les BIP de suivi
de standard sont les plus courants. Ils peuvent concerner différents
aspects~: les règles de consensus, le protocole de transmission
(\emph{Peer Services}), l'interface logicielle (\emph{API/RPC}) ou les
conventions utilisées dans les applications\footnote{Eric Lombrozo,
  \emph{BIP-123: BIP Classification}, 26 août 2015~:
  \url{https://github.com/bitcoin/bips/blob/master/bip-0123.mediawiki}.}.

Avant d'être adopté, un BIP doit passer par de nombreuses étapes.
D'abord, il est assigné à un ou plusieurs auteurs qui se chargent d'en
rédiger une première version respectant le format défini et prenant en
compte l'état de l'art correspondant. Puis, le BIP est partagé dans la
communauté des développeurs de Bitcoin, généralement par l'intermédiaire
de la liste de diffusion de développement (bitcoin-dev). Les discussions
ont lieu sur cette mailing list. Ensuite, le BIP est officiellement
proposé au système sous la forme d'une demande de modification du code
(\emph{pull request}) sur le dépôt GitHub, qui doit être approuvée par
l'éditeur désigné par Bitcoin Core (luke-jr depuis 2016). Enfin, un
numéro lui est assigné et il est intégré au dépôt sous la forme d'une
ébauche. Il peut par la suite changer de statut au cours du temps, selon
l'adoption de la communauté, l'objectif étant qu'il devienne définitif
ou actif.

\begin{figure}[H]

{\centering \includegraphics{chapters/img/bip-process-fr.png}

}

\caption{Schéma de la procédure d'adoption d'un BIP, inspiré du BIP-1.}

\end{figure}%

Notez que ces documents sont utilisés pour BTC mais également pour
d'autres protocoles. Par exemple, les BIP décrivant le fonctionnement
des portefeuilles (BIP-32, BIP-39, BIP-44) sont valides pour la grande
majorité des cryptomonnaies. Le SLIP-44 recense les cryptomonnaies
compatibles avec le BIP-44\footnote{«~SLIP-44~»~:
  \url{https://github.com/satoshilabs/slips/blob/master/slip-0044.md}}.
Les autres protocoles cryptoéconomiques disposent même parfois de leurs
propres systèmes de propositions. Ethereum utilise les EIP
(\emph{Ethereum Improvement Proposals}), Bitcoin Cash les
CHIP\footnote{«~les CHIP~»~: \url{https://bch.info/en/chips}.}
(\emph{Cash Improvement Proposals}), Litecoin les LIP\footnote{«~les
  LIP~»~: \url{https://github.com/litecoin-project/lips}.}, etc.

\section*{La vérification des règles de
consensus}\label{la-vuxe9rification-des-ruxe8gles-de-consensus}
\addcontentsline{toc}{section}{La vérification des règles de consensus}

\markright{La vérification des règles de consensus}

Bitcoin se base sur un réseau public d'ordinateurs accessible librement
sur Internet. Ce réseau suit un modèle pair à pair, c'est-à-dire un
modèle dans lequel tous les membres du réseau, appelés des nœuds,
possèdent les mêmes privilèges. Ce sont ces nœuds qui s'assurent que les
règles de consensus sont respectées. Si un bloc est invalide (en
contenant une transaction invalide par exemple), alors il est rejeté par
les nœuds appliquant les règles.

Dans Bitcoin, le rôle des nœuds est d'entretenir une copie du registre
des transactions (la fameuse chaîne de blocs) et, ce faisant, de
s'assurer de la validité des transactions et des blocs. Pour cela, ils
communiquent avec les autres nœuds du réseau et relaient les nouvelles
transactions et les nouveaux blocs, qui émanent respectivement des
utilisateurs et des mineurs.

La vérification des règles de consensus peut être complète. Dans ce cas,
on utilise parfois le pléonasme «~nœuds complets~» ou «~\emph{full
node}~» pour insister sur le fait qu'ils vérifient l'intégralité de la
chaîne. Ils téléchargent l'intégralité de la chaîne de blocs, vérifient
les règles de consensus et relaient les blocs et les transactions. C'est
une charge, que ce soit au niveau de la conservation des données (en
novembre 2023, la chaîne de Bitcoin pesait environ 530~Go de données et
l'ensemble des UTXO plus de 8,5~Go) que de la bande passante (la taille
moyenne des blocs minés toutes les 10 minutes gravitait autour de 1,7~Mo
en novembre 2023).

Les nœuds réduits (\emph{pruned nodes}), qui conservent l'état du réseau
mais pas l'entièreté de la chaîne, sont des nœuds à part entière
puisqu'ils ont vérifié la conformité des règles sur l'intégralité de la
chaîne. Ils ne sont juste pas en mesure d'accéder à l'historique de la
chaîne précédant une certaine date.

La vérification peut aussi être partielle, auquel cas on parle de client
léger (ou de «~nœud léger~» par abus de langage). Cela est utile pour
les personnes qui n'ont pas l'intérêt de faire tourner un nœud complet.
C'est par exemple le cas dans les logiciels de hachage (mettant en œuvre
Stratum) et dans les portefeuilles légers. Ils utilisent en particulier
une méthode conceptualisée dans le livre blanc de Bitcoin en 2008~: la
vérification de paiement simplifiée\footnote{Satoshi Nakamoto décrivait
  la vérification de paiement simplifiée comme suit~: «~Il est possible
  de vérifier les paiements sans faire fonctionner un nœud complet du
  réseau. Un utilisateur a seulement besoin de conserver une copie des
  entêtes des blocs de la plus longue chaîne de preuves de travail,
  qu'il peut obtenir en interrogeant les nœuds du réseau jusqu'à ce
  qu'il soit convaincu qu'il possède la plus longue chaîne, et obtenir
  la branche de Merkle liant la transaction au bloc dans lequel elle est
  horodatée. Il ne peut pas vérifier la transaction par lui-même, mais
  en la reliant à un endroit de la chaîne, il peut voir qu'un nœud du
  réseau l'a acceptée, et les blocs ajoutés après le confirment.~» --
  Satoshi Nakamoto, \emph{Bitcoin: A Peer-to-Peer Electronic Cash
  System}, 31 octobre 2008.}.

La vérification de paiement simplifiée (nommée en anglais
\emph{Simplified Payment Verification} et abrégée en SPV) est une
méthode astucieuse, qui permet aux utilisateurs néophytes et
occasionnels de pouvoir interagir facilement avec le protocole sans
devoir gérer un nœud complet, ni devoir faire aveuglément confiance à un
dépositaire. Elle permet de réduire considérablement la charge des
portefeuilles légers.

La vérification de paiement simplifiée se fonde sur la façon dont les
blocs de transactions sont chaînés et structurés comme nous avons pu le
voir dans le chapitre~\hyperref[ch:confirmation]{8}. Premièrement, la
chaîne de preuve de travail n'est pas à proprement parler une chaîne de
blocs, mais une chaîne d'entêtes. Cela fait que les clients légers n'ont
qu'à conserver cette chaîne des entêtes pour déterminer la chaîne
possédant le plus de travail accumulé. Puisque chaque entête pèse 80
octets, la taille des données à conserver reste modeste pour des
appareils modernes~: elle augmente d'environ 4~Mio par an, ce qui
représentait un peu plus de 62 Mio en novembre 2023.

Deuxièmement, les transactions sont agencées dans un arbre de Merkle, de
sorte que les clients légers peuvent se contenter de demander les
informations liées à la branche qui les intéresse pour s'assurer de la
confirmation d'une de leurs transactions. Le nombre d'empreintes à
obtenir et de hachages à effectuer dépend du logarithme binaire
(\(\log_{2}\)) du nombre de transactions présentes dans le bloc. Pour un
bloc de 3~000 transactions (moyenne haute sur BTC), la charge correspond
à demander 12 empreintes de 32 octets et à calculer 12 hachages pour
procéder à la vérification.

Cette vérification simplifiée permet d'alléger la charge des
portefeuilles, mais elle présente des défauts majeurs. D'abord, elle
manque de fiabilité~: les nœuds ne peuvent pas mentir en inventant une
transaction, mais peuvent omettre de transmettre des informations
nécessaires. Ce défaut peut être partiellement contrebalancé en
augmentant la diversité des connexions sur le réseau. Cependant, même
dans ce cas, la vérification est vulnérable si la chaîne est attaquée
par une entité disposant de la puissance de calcul
majoritaire\footnote{Ce cas a été décrit par Satoshi Nakamoto dans le
  livre blanc~: «~De ce fait, la vérification est fiable tant que les
  nœuds honnêtes contrôlent le réseau, mais est plus vulnérable si le
  réseau est maîtrisé par un attaquant. Alors que les nœuds du réseau
  peuvent vérifier les transactions par eux-mêmes, la méthode simplifiée
  peut être trompée par des transactions forgées par l'attaquant aussi
  longtemps que celui-ci maîtrise le réseau. Une stratégie pour se
  protéger serait d'accepter les alertes des nœuds du réseau lorsqu'ils
  détectent un bloc invalide, invitant le logiciel de l'utilisateur à
  télécharger le bloc complet et les transactions suspectes pour
  confirmer l'incohérence. Les entreprises qui reçoivent fréquemment des
  paiements voudront probablement toujours faire fonctionner leurs
  propres nœuds afin d'obtenir une sécurité plus indépendante et une
  vérification plus rapide.~» -- Satoshi Nakamoto, \emph{Bitcoin: A
  Peer-to-Peer Electronic Cash System}, 31 octobre 2008.}.

Ensuite, la vérification simplifiée possède aussi une insuffisance de
confidentialité, car le client doit dévoiler une partie de son activité
transactionnelle par les requêtes réalisées auprès des noeuds du réseau.
Une façon de corriger partiellement ce problème est d'accroître le
nombre d'informations demandées pour dissimuler les informations
essentielles, mais cette méthode est plus qu'imparfaite\footnote{Une
  première façon de remédier au problème de confidentialité était de
  mettre en place des filtres de Bloom, tels que décrits dans le BIP-37,
  mais cette méthode était peu efficace. Voir Arthur Gervais, Srdjan
  Capkun, Ghassan O. Karame, Damian Gruber, «~\emph{On the Privacy
  Provisions of Bloom Filters in Lightweight Bitcoin Clients}~», in
  \emph{Proceedings of the 30th Annual Computer Security Applications
  Conference}, décembre 2014, pp.~326---335~:
  \url{https://eprint.iacr.org/2014/763.pdf}. Il existe également
  Neutrino, décrit dans le BIP-157 et le BIP-158, qui fait usage du
  codage de Golomb-Rice et demande une plus grande bande passante.}.

Enfin, elle présente un défaut de vérification, en étant par définition
partielle. Toutes les règles de consensus ne sont pas vérifiées, ce qui
fait que les nœuds complets peuvent convenir d'un changement de règle
qui ne sera pas remarqué par le client léger. Par exemple, les clients
SPV ne vérifient pas les contraintes appliquées sur la taille des blocs,
et le réseau pourrait donc subir une modification de cette limite sans
qu'ils s'en rendent compte. C'est ce qui explique la stratégie des
promoteurs de SegWit2X en 2017, qui prévoyaient de doubler la taille
limite des blocs sans protection contre la rediffusion afin que les
portefeuilles à vérification de paiement simplifiée suivent simplement
la chaîne la plus longue\footnote{«~prévoyaient de doubler la taille
  limite des blocs sans protection contre la rediffusion afin que les
  portefeuilles à vérification de paiement simplifiée suivent simplement
  la chaîne la plus longue~»~: Mike Belshe, \emph{{[}Bitcoin-segwit2x{]}
  Strong 2-Way Replay Protection}, /10/2017 20:16:02 UTC~:
  \url{https://lists.linuxfoundation.org/pipermail/bitcoin-segwit2x/2017-October/000323.html}~
  «~Aujourd'hui, nous sommes en bonne voie pour déployer segwit2x avec
  une grande majorité de mineurs qui le signalent encore. En plus de
  cela, 99,94~\% des nœuds et des clients SPV suivront automatiquement
  la chaîne la plus longue (segwit2x).~»}.

Satoshi pensait que le système pourrait perdurer avec une vérification
centralisée entre les mains de quelques nœuds vérificateurs (dont les
mineurs) et que le reste des utilisateurs ferait usage des clients
légers\footnote{«~Satoshi pensait que le système pourrait perdurer avec
  une vérification centralisée entre les mains de quelques nœuds
  vérificateurs~»~: Satoshi Nakamoto a conservé cette vision jusqu'à son
  départ, comme en témoigne son courriel à Mike Hearn du 29 décembre
  2010~:}. Dans sa première réponse à James A. Donald en novembre 2008,
il indiquait ainsi~:

«~Bien avant que le réseau n'atteigne cette taille, les utilisateurs
pourront utiliser la vérification de paiement simplifiée (section 8)
pour contrôler les doubles dépenses, ce qui ne nécessite que la chaîne
des entêtes de bloc, soit environ 12~Ko par jour. Seules les personnes
essayant de créer de nouvelles pièces auront besoin de faire fonctionner
des nœuds de réseau\footnote{Satoshi Nakamoto, \emph{Re: Bitcoin P2P
  e-cash paper}, /11/2008, 01:37:43 UTC~:
  \url{https://www.metzdowd.com/pipermail/cryptography/2008-November/014815.html}.}.~»

En cela, il se trompait. La vérification des règles de consensus a
besoin d'être intégrale pour que celles-ci soient appliquées.

C'est donc au niveau du nœud complet que se joue cette vérification, une
réalité qui est parfois retranscrite par l'adage «~pas ton nœud, pas tes
règles\footnote{Understanding Bitcoin, \emph{Not Your Node Not Your
  Rules! w/ Ketominer, Udi Wertheimer, Francis Pouliot \& Mir Liponi}
  (vidéo), 5 avril 2019~:
  \url{https://www.youtube.com/watch?v=jwaKVIEm-rI}.}\footnote{«~pas ton
  nœud, pas tes règles~»~: L'adage «~\emph{not your node, not your
  rules}~» a été naturellement calqué sur l'adage «~\emph{not your keys,
  not your coins}~» (voir par exemple ce tweet de Udi Wertheimer~:
  \url{https://twitter.com/udiWertheimer/status/936215582487261184}). Il
  a été popularisé par le panel du même nom lors de la conférence
  Understanding Bitcoin, le 5 avril 2019. Le projet RaspiBlitz en a fait
  son slogan en 2020~:
  \url{https://github.com/rootzoll/raspiblitz/blob/bbeb5b21a982eeeb93306537e0aca2474bd23e03/README.md}.}~».
Ne faites pas confiance, vérifiez~! Un peu comme une langue résulte des
choix que font ses locuteurs, un protocole informatique résulte des
règles appliquées par les nœuds complets. Cette vérification joue donc
un rôle crucial dans la détermination du protocole.

\section*{Les hard forks}\label{les-hard-forks}
\addcontentsline{toc}{section}{Les hard forks}

\markright{Les hard forks}

Puisque Bitcoin est ouvert et libre, les règles de consensus peuvent
être modifiées à volonté par les nœuds du réseau au moyen d'un
changement d'acceptation des blocs et des transactions. Ces
modifications peuvent mener à des conflits sur le réseau, et
éventuellement à la séparation en deux réseaux distincts gérant chacun
sa propre chaîne et sa propre monnaie. D'où l'utilisation du mot
\emph{fork}, qui signifie «~embranchement~», «~bifurcation~» ou
«~fourche~» en français, pour parler de ce phénomène.

Les modifications des règles de consensus sont couramment rangées en
deux catégories~: celle des \emph{hard forks}, qui constituent des mises
à niveau brutes et incompatibles, et celle des \emph{soft forks}, qui
présentent une certaine rétrocompatibilité. Voyons comment ces
changements se manifestent, en commençant par les hard forks, avant de
décrire les soft forks.

Dans Bitcoin, il existe une polysémie au sujet du mot \emph{fork}, qui
possède quatre significations différentes~: le fork logiciel, le fork de
règles de consensus, le fork de chaîne commun et le fork de chaîne
persistant. Cette polysémie prête à confusion de sorte qu'on préfère
utiliser un terme différent pour chacun de ces sens.

Comme on l'a dit, le mot fork est d'abord utilisé dans le développement
logiciel, notamment dans le cadre du logiciel libre qui autorise et
encourage ce type de pratique. Il désigne la création d'un programme
dérivé du code source d'un programme existant et aussi, par abus de
langage, le programme dérivé en lui-même. En ce sens, l'implémentation
de référence peut subir un embranchement, créant un logiciel alternatif.
Ce logiciel peut respecter les règles de consensus (comme par exemple
Bitcoin Knots), mais il peut aussi les faire dévier, en créant un
nouveau protocole qui partage l'historique de la chaîne (Bitcoin ABC,
devenu Bitcoin Cash Node) ou non (Litecoin).

Le fork peut ensuite désigner l'embranchement commun de la chaîne de
blocs, par analogie avec le développement logiciel. La chaîne de blocs
n'est en effet pas une structure linéaire, mais une «~structure en forme
d'arbre\footnote{Satoshi Nakamoto, code source de la version 0.1 du
  logiciel Bitcoin~:
  \url{https://github.com/trottier/original-bitcoin/blob/4184ab26345d19e87045ce7d9291e60e7d36e096/src/main.h\#L1001-L1008}.}~»
qui peut posséder de multiples branches de blocs, pareillement
compatibles avec les règles de consensus acceptées par le réseau, la
sélection de la branche correcte se faisant par la plus longue
(possédant le plus de travail accumulé). Ce type d'embranchement se
produit régulièrement dans Bitcoin de manière tout à fait normale et
bénigne, lorsque deux mineurs trouvent simultanément un bloc différent
de leur côté, et est résolu lorsqu'un nouveau bloc est trouvé.

Le fork peut aussi se rapporter à une scission de la chaîne de blocs
causée par une incompatibilité des règles de consensus. On parle alors
de hard fork ou d'«~embranchement divergent~». Cette scission est
généralement permanente dans le sens où les deux branches ne peuvent pas
se réconcilier par le mécanisme de consensus de Nakamoto, sauf dans un
cas très précis~: si les règles de la branche majoritaire forment une
sous-partie restrictive des règles de la branche minoritaire. Les deux
chaînes résultantes sont, à terme, vouées à exister sur des réseaux
séparés.

Enfin, le terme fork peut, par métonymie, désigner une modification des
règles de consensus, qui est toujours susceptible de provoquer une
scission de chaîne et une séparation du réseau. Une restriction des
règles de consensus est appelée un soft fork, ou «~embranchement
convergent~», en vertu de sa capacité à résulter en une branche unique.
Toute autre modification des règles de consensus, qu'il s'agisse d'une
extension ou d'une modification strictement incompatible, est appelée un
hard fork, en référence à sa propension à créer une scission de chaîne.
C'est de ces deux modifications dont nous voulons parler ici\footnote{Notez
  que les concepts sont liés. Ainsi, un fork logiciel (copie et
  modification) peut implémenter un fork des règles de consensus (hard
  fork ou soft fork) qui finira par créer un fork persistant de la
  chaîne (scission).}.

Le hard fork est le concept le plus ancien si on le compare au soft
fork. Il était auparavant qualifié de «~changement
incompatible\footnote{David François (davout), \emph{Re: Small protocol
  changes for flexibility}, /12/2010 15:08:02 UTC~:
  \url{https://bitcointalk.org/index.php?topic=894.msg27757\#msg27757}.}~».
Le hard fork est une modification non restrictive des règles de
consensus. Il provoque un conflit sur le réseau entre les nœuds qui
appliquent les anciennes règles et les nœuds qui appliquent les
nouvelles.

Un hard fork peut être extensif, c'est-à-dire élargir les règles de
consensus sur les blocs et les transactions. Les anciens peuvent ainsi
produire des blocs valides sur la nouvelle chaîne, mais pas l'inverse.
L'exemple typique de ce genre de hard fork est l'augmentation de la
taille limite des blocs, qui consiste à accepter des blocs ayant une
taille ou un poids plus grand, comme 2~Mo au lieu de 1~Mo ou 8~MWU à la
place de 4~MWU. Ce hard fork extensif est illustré sur la
figure~\hyperref[fig:expanding-hard-fork]{10.1}.

\begin{figure}

{\centering \includegraphics{chapters/img/expanding-hard-fork-induced-forks.png}

}

\caption{Schéma d'un hard fork extensif~: si la chaîne suivant la
nouvelle règle est plus longue que celle suivant l'ancienne, les deux
chaînes persistent~; dans le cas contraire, seule la deuxième survit.}

\end{figure}%

Dans le cas où le hard fork extensif n'est pas soutenu par une majorité
de la puissance de calcul du réseau, celui-ci risque de ne pas créer une
branche persistante. Par exemple, les blocs de la branche imposant une
limite de taille plus petite sont entièrement compatibles avec les
nouvelles règles, de sorte que, si elle est plus longue, c'est elle qui
sera sélectionnée comme la branche correcte. C'est pour éviter cette
situation problématique que les hard forks sont généralement bilatéraux.

Le hard fork bitatéral est un hard fork qui crée une incompatibilité
totale entre les nouvelles règles et les anciennes. Il peut s'agir d'une
règle ajoutée comme l'exigence que le premier bloc de l'embranchement
inclue un changement incompatible. Dans notre cas de l'augmentation de
la taille limite des blocs, il s'agirait d'imposer au premier bloc
d'être strictement plus gros que la taille limite précédente, comme on
le voit sur la figure~\hyperref[fig:expanding-hard-fork-failure]{10.2}.
Cette règle supplémentaire est appelée protection contre la destruction
par recoordination (ou \emph{wipeout protection} en anglais).

\begin{figure}

{\centering \includegraphics{chapters/img/bilateral-hard-fork-induced-fork.png}

}

\caption{Schéma d'un hard fork bilatéral~: les nouvelles règles sont
strictement incompatibles avec les anciennes règles, de sorte que les
deux chaînes persistent.}

\end{figure}%

Un autre exemple est le changement de l'algorithme de signature des
transactions, qui rend l'intégralité des transactions signées et des
blocs non vides strictement incompatibles. Ce changement a pour effet de
permettre en plus une protection contre la rediffusion des transactions
(\emph{replay protection}), dans le cas où deux chaînes concurrentes
persisteraient.

Deux situations peuvent découler d'un hard fork~: soit la quasi-totalité
de l'économie procède au changement, auquel cas une seule chaîne
subsiste~; soit l'économie se fragmente, auquel cas les deux chaînes
persistent. La première situation est visée par le hard fork de mise à
niveau qui n'a pas vocation à créer deux chaînes distinctes. La seconde
est désirée par le hard fork contentieux, résultant d'une division de la
communauté au sujet du changement. Le hard fork accidentel, créé par une
modification non désirée des règles de consensus implicites, est écarté
ici\footnote{Le 11 mars 2013, le passage de la version 0.7 du logiciel à
  la version 0.8 implémentait la migration du système de base de données
  de Berkeley DB à LevelDB. Toutefois, il s'avérait que Berkeley DB
  faisait intervenir une limite par défaut (\emph{lock limit}) qui
  n'était pas présente dans LevelDB. Par conséquent, la migration
  constituait un hard fork accidentel et a provoqué un embranchement à
  partir du bloc 225~430 qui a duré environ 6 heures. La décision a
  finalement été prise de revenir à la version 0.7, invalidant la
  branche de 24 blocs minée du côté de la version 0.8, et de procéder à
  la migration quelques mois plus tard. -- Voir Vitalik Buterin,
  «~\emph{Bitcoin Network Shaken by Blockchain Fork}~», \emph{Bitcoin
  Magazine}, 13 mars 2013~:
  \url{https://bitcoinmagazine.com/technical/bitcoin-network-shaken-by-blockchain-fork-1363144448}~;
  et Gavin Andresen, \emph{BIP-50: March 2013 Chain Fork Post-Mortem},
  ~:
  \url{https://github.com/bitcoin/bips/blob/master/bip-0050.mediawiki}.}.

Le hard fork de mise à niveau est un hard fork qui nécessite une
synchronisation de la quasi-totalité de la communauté. Il résulte
généralement en une seule chaîne, de sorte qu'on peut considérer que le
protocole a été mis à niveau, alors qu'il s'agit essentiellement d'une
utilisation économique qui passe d'un protocole à un autre. Il peut pour
cela être extensif, même si la bilatéralité est préférée pour des
raisons de sécurité.

Le premier hard fork de mise à niveau connu est probablement l'ajout des
codes opération \texttt{OP\_NOP} à la version 0.3.6 de Bitcoin par
Satoshi Nakamoto en juillet 2010. L'augmentation de la taille des blocs
était également pensée comme un hard fork de mise à niveau, notamment
par Satoshi lui-même\footnote{En octobre 2010, à la suite de la
  proposition de Jeff Garzik d'augmenter la limite directement à
  7,168~Mo afin d'«~égaler le taux transactionnel moyen de PayPal~»,
  Satoshi -- bien conscient qu'il s'agissait d'un correctif
  «~incompatible avec le réseau~» -- écrivait~: «~{[}La mise à niveau{]}
  peut être introduite progressivement, par exemple~: if (blocknumber
  \textgreater{} 115000) maxblocksize = largerlimit. Elle peut commencer
  à être intégrée dans les versions bien avant, de sorte qu'au moment où
  elle atteint le numéro de bloc et entre en vigueur, les anciennes
  versions qui ne l'ont pas sont déjà obsolètes. Lorsque nous approchons
  du numéro de bloc limite, je peux envoyer une alerte aux anciennes
  versions pour qu'elles sachent qu'elles doivent effectuer une mise à
  jour.~» -- Satoshi Nakamoto, \emph{Re: {[}PATCH{]} increase block size
  limit}, /10/2010 19:48:40 UTC~:
  \url{https://bitcointalk.org/index.php?topic=1347.msg15366\#msg15366}.},
jusqu'au hard fork contentieux de Bitcoin Cash en 2017.

En dehors de BTC, les mises à niveaux par hard fork sont nombreuses,
notamment en raison d'une économie moins grande et~/~ou plus
centralisée. On peut citer les cas de Bitcoin Cash, de Monero,
d'Ethereum Classic et d'Ethereum, où des mises à niveau de ce type sont
réalisées régulièrement.

Le hard fork contentieux est un hard fork visant délibérément à créer
une nouvelle chaîne. Il est issu d'une dissension dans la communauté,
qui est si forte qu'elle pousse à la sécession. Il est généralement
bilatéral.

Le premier exemple de hard fork contentieux majeur est celui qui a eu
lieu sur Ethereum en juillet 2016, dans le contexte du piratage de
TheDAO\footnote{«~celui qui a eu lieu sur Ethereum en juillet 2016~»~:
  Simon Polrot, \emph{The DAO~: post mortem}, 24 janvier 2017~:
  \url{https://www.ethereum-france.com/the-dao-post-mortem/}~; Casey
  Detrio, \emph{EIP-779: Hardfork Meta: DAO Fork}, 26 novembre 2017~:
  \url{https://eips.ethereum.org/EIPS/eip-779}.}. Ce hard fork
consistait à reprendre les fonds du pirate par un «~changement d'état
irrégulier~». Celui-ci était rendu bilatéral par la règle imposant aux
10 premiers blocs d'inclure la chaîne de caractères ``. Puisque la
majorité économique se trouvait du côté de l'annulation, la chaîne
altérée a gardé le nom d'Ethereum et le sigle boursier ETH, tandis que
l'autre chaîne a pris le nom d'Ethereum Classic et le sigle boursier
ETC.

Le second exemple de hard fork contentieux est celui qui a mené à la
création de Bitcoin Cash en août 2017 suite au débat sur la scalabilité
et à la guerre des blocs\footnote{«~celui qui a mené à la création de
  Bitcoin Cash en août 2017~»~: Ludovic Lars, \emph{Bitcoin Cash~: la
  branche minoritaire issue du débat sur la scalabilité}, 30 janvier
  2022~:
  \url{https://journalducoin.com/analyses/bitcoin-cash-branche-minoritaire-debat-scalabilite/}~;
  \emph{BCH-UAHF: Bitcoin Cash User-Activated Hard Fork}, 24 juillet
  2017~: \url{https://reference.cash/protocol/forks/bch-uahf}.}. Ce hard
fork n'intégrait pas SegWit, augmentait la taille limite des blocs à
8~Mo et améliorait l'algorithme de signature. Il était rendu bilatéral
par une règle qui imposait au bloc suivant l'activation d'avoir une
taille strictement supérieure à 1~Mo. Il offrait aussi \emph{de facto}
une protection contre la rediffusion des transactions. Ce changement
ayant dû se faire sans l'accord de la majorité économique, la chaîne qui
ne modifiait pas les règles a pu conserver le nom de Bitcoin et le sigle
boursier BTC, tandis que la nouvelle chaîne a dû adopter un nouveau nom,
Bitcoin Cash, et un nouveau sigle, BCH.

\begin{figure}

{\centering \includegraphics{chapters/img/hard-forks-eth-etc-bch-btc.png}

}

\caption{Exemples de hard forks bilatéraux~: ETH~/~ETC et BCH~/~BTC.}

\end{figure}%

Notez qu'un tel hard fork peut être amené à modifier l'algorithme
d'ajustement de la difficulté. En effet, si la puissance de calcul est
trop faible pour le soutenir, il est possible que l'ajustement n'arrive
pas à terme. C'est pour cette raison que Bitcoin Cash a dû implémenter
un \emph{Emergency Difficulty Adjustment} (EDA) qui a permis de procéder
à l'adaptation sur une période plus courte. Ethereum Classic n'a
cependant pas dû le faire, car l'ajustement sur Ethereum avait déjà lieu
à tous les blocs.

\section*{Les soft forks}\label{les-soft-forks}
\addcontentsline{toc}{section}{Les soft forks}

\markright{Les soft forks}

Passons maintenant au soft fork, qui est un procédé de mise à niveau
souvent mal compris. Le soft fork est une restriction des règles de
consensus. Il consiste ainsi par essence à rendre l'ensemble des blocs
et des transactions valides plus petit, en ajoutant une règle ou en
modifiant une règle existante de façon plus restrictive. L'exemple
typique de ce genre de fork est la diminution de la taille limite des
blocs. L'ajout de la limite explicite des 1~Mo en octobre 2010 était de
ce fait un soft fork.

Le soft fork peut être appliqué en conservant une seule et même chaîne.
S'il est imposé par la majorité de la puissance de calcul du réseau, il
n'y a aucun risque de scission. En effet, l'ensemble des blocs créés par
les mineurs qui appliquent les nouvelles règles est entièrement
compatible avec les anciennes règles, de sorte que la branche appliquant
les nouvelles règles sera considérée comme la branche correcte par tous
les nœuds si elle est majoritaire. Si l'application du soft fork est en
revanche minoritaire, alors ce dernier résulte en deux chaînes
persistantes distinctes. Les deux cas de figure sont illustrés sur la
figure~\hyperref[fig:soft-fork]{10.4}.

\begin{figure}

{\centering \includegraphics{chapters/img/soft-fork-induced-forks.png}

}

\caption{Schéma d'un soft fork~: si la chaîne suivant la nouvelle règle
est plus longue que celle suivant l'ancienne, seule la première survit~;
dans le cas contraire, les deux chaînes persistent.}

\end{figure}%

Le concept de soft fork est postérieur à celui de hard fork. Il a été
formellement découvert par Gavin Andresen en octobre 2011 qui, suite à
son étude de la proposition d'ajout du code opération \texttt{OP\_EVAL}
par Nicolas van Saberhagen\footnote{Nicolas van Saberhagen (ByteCoin),
  \emph{OP\_EVAL proposal}, /10/2011 00:49:19 UTC~:
  \url{https://bitcointalk.org/index.php?topic=46538.msg553689\#msg553689}.},
s'est aperçu que la mise à niveau pouvait se faire grâce au code
opération \texttt{OP\_NOP1} sans nécessairement provoquer de
scission\footnote{«~Je lis probablement mal le code, mais je pense que
  OP\_EVAL ne provoquerait pas de scission de la chaîne de blocs~!~»
  s'est exprimé Gavin Andresen sur IRC. -- \#bitcoin-dev IRC logs, 2
  octobre 2010~:
  \url{https://web.archive.org/web/20131201200245/http://bitcoinstats.com/irc/bitcoin-dev/logs/2011/10/02}.}.

Les codes opération \texttt{OP\_NOP} sont des instructions du langage de
script de Bitcoin qui ont été ajoutés dans le code par Satoshi en
juillet 2010 avec pour seul commentaire «~expansion\footnote{Satoshi
  Nakamoto, \emph{reverted makefile.unix wx-config -- version 0.3.6 (git
  commit)}, /07/2010 18:27:12 UTC~:
  \url{https://sourceforge.net/p/bitcoin/code/119/}.}~». Le changement a
été rendu effectif avec la version 0.3.6 du logiciel qui corrigeait
également le 1 RETURN bug, publiée le 29 juillet\footnote{«~la version
  0.3.6 du logiciel {[}...{]} publiée le 29 juillet~»~: Satoshi
  Nakamoto, \emph{*** ALERT *** Upgrade to 0.3.6}, /07/2010 19:13:06
  UTC~:
  \url{https://bitcointalk.org/index.php?topic=626.msg6451\#msg6451}.}.
Leur rôle est initialement muet~: s'ils sont présents dans un script,
ils ne font rien mais ils n'invalident pas la transaction non plus. La
conséquence directe est qu'on peut modifier le comportement de ces codes
opération sans rendre les scripts incompatibles avec les anciennes
règles de consensus. L'ajout de cette caractéristique indique donc que
Satoshi avait saisi le mécanisme du soft fork.

Le soft fork possède un caractère «~rétrocompatible~» -- ou
postcompatible à proprement parler, car la compatibilité est ascendante
et non descendante -- dans le sens où les anciennes versions du logiciel
peuvent continuer d'interagir avec le système. En effet, les nœuds non
miniers suivant les anciennes règles continuent de voir les blocs
produits comme valides. Cette caractéristique est un avantage majeur par
rapport au hard fork.

Mais cette compatibilité ascendante ne veut pas dire qu'un soft fork est
«~doux~». Il possède un côté pernicieux dans le sens où il rend la
modification difficile à appréhender. Le soft fork présente ainsi
plusieurs inconvénients.

D'abord, il n'est pas optionnel. S'il est appliqué par la majorité de la
puissance de calcul, un soft fork s'apparente en effet à une attaque de
censure pour les utilisateurs qui suivent les anciennes règles. Le soft
fork possède donc un caractère coercitif que le hard fork n'a pas.

Puis, le soft fork est difficilement réversible. Les fonctionnalités
ajoutées ne peuvent pas être désactivées simplement~: une fois adopté,
il n'y a pas de retour en arrière facile. Les développeurs de Bitcoin SV
ont ainsi désactivé P2SH en février 2020 exposant les utilisateurs les
moins attentifs à des vols\footnote{«~Les développeurs de Bitcoin SV ont
  ainsi désactivé P2SH en février 2020~»~: Jon Southurst, \emph{Final
  Genesis specs released---bye P2SH}, 10 janvier 2020~:
  \url{https://coingeek.com/final-genesis-specs-released-bye-p2sh/}.}.

Ensuite, le soft fork n'est pas limité quant à ce qu'il peut faire. Il
peut augmenter la limite effective de taille des blocs (via un bloc
auxiliaire\footnote{«~bloc auxiliaire~»~:
  \url{https://bitcointalk.org/index.php?topic=283746.msg3036293\#msg3036293}.}
aussi appelé bloc d'extension\footnote{«~bloc d'extension~»~:
  \url{https://lists.linuxfoundation.org/pipermail/bitcoin-dev/2015-May/008356.html}.}
ou soft fork généralisé\footnote{«~soft fork généralisé~»~: ZoomT,
  \emph{Increasing the blocksize as a (generalized) softfork.}, /12/2015
  11:12:48 UTC~:
  \url{https://bitcointalk.org/index.php?topic=1296628.msg13305141\#msg13305141}.}).
Ce bloc d'extension peut également inclure des fonctionnalités
supplémentaires (comme MimbleWimble dans Litecoin). Il peut même
modifier la politique monétaire du protocole en redéfinissant l'unité de
base\footnote{La façon dont un soft fork peut introduire de l'inflation
  dans Bitcoin a été exposée par le développeur Peter Todd en 2016. --
  Peter Todd, \emph{Forced Soft Forks}, 18 janvier 2016~:
  \url{https://petertodd.org/2016/forced-soft-forks}.}.

Enfin, le soft fork, s'il est profond, crée une complexité
supplémentaire, liée aux contraintes de son application. En effet, il
ajoute de nouvelles exceptions aux règles de consensus, ce qui génère de
la dette technique pour les développeurs\footnote{«~dette technique~»~:
  Ward Cunningham, «~The WyCash Portfolio Management System~»,
  \emph{Addendum to the Proceedings of OOPSLA 1992}, octobre 1992~:
  \url{https://dl.acm.org/doi/pdf/10.1145/157710.157715}.}.

L'archétype du soft fork profond et complexe a été la mise à niveau
SegWit, ou \emph{Segregated Witness}, qui consistait à déplacer les
données de signature des transactions (appelées témoin ou
\emph{witness}) vers une structure de données séparée
(\emph{segregated}) afin de supprimer la malléabilité des transactions.
Cette mise à niveau, qui a eu lieu en le 24 août 2017, devait être
initialement un hard fork, avant que le développeur luke-jr ne décrive
en 2015 comment en faire un soft fork\footnote{«~avant que le
  développeur luke-jr ne décrive en 2015 comment en faire un soft
  fork~»~: Aaron van Wirdum, \emph{The Long Road to SegWit: How
  Bitcoin's Biggest Protocol Upgrade Became Reality}, 23 août 2017~:
  \url{https://bitcoinmagazine.com/technical/the-long-road-to-segwit-how-bitcoins-biggest-protocol-upgrade-became-reality}.}.
La rétrocompatibilité était assurée par la liaison du témoin au bloc via
un arbre de Merkle dont la racine était placée dans la transaction de
récompense et par l'utilisation de sorties transactionnelles dépensables
par n'importe qui (\emph{anyone-can-spend}). Outre la correction du
problème de malléabilité, elle a instauré un système de versionnage (qui
a permis l'intégration de Schnorr-Taproot par la suite) et a modérément
augmenté la capacité transactionnelle du réseau, de sorte que la taille
effective des blocs pouvait dépasser 1~Mo, jusqu'à 4~Mo en théorie. Elle
a également ajouté quatre nouveaux types d'adresse au protocole.

De plus, le soft fork requiert la majorité de la puissance de calcul du
réseau pour préserver son intérêt. S'il n'est pas suivi à moyen terme
par 51~\% de la puissance de calcul, alors son application provoque une
scission. C'est ce qui explique pourquoi l'activation par les mineurs
est généralement préférée à l'activation par les utilisateurs, même si
le pouvoir de décision revient à ces derniers comme on le verra dans le
chapitre~\hyperref[ch:determination]{11}.

D'une part, le soft fork activé par les utilisateurs (en anglais
\emph{user activated soft fork} ou UASF) consiste à implémenter le soft
fork dans le code source du logiciel de sorte à ce qu'il rentre en
application à une hauteur de bloc ou à un horodatage donné. Cette
méthode s'appuie sur la confiance que l'économie appliquant la mise à
niveau sera largement majoritaire et que l'activité minière suivra à
moyen terme en raison d'une récompense de bloc plus élevée.

D'autre part, le soft fork activé par les mineurs (en anglais
\emph{miner activated soft fork} ou MASF) consiste à faire dépendre
l'activation du signalement des mineurs au sein des blocs validés. Il
est activé lorsqu'un certain seuil de signalement (95~\% par exemple)
est dépassé. Cette méthode, dont la procédure a été notamment décrite
dans le BIP-9, permet de s'assurer autant que possible que les mineurs
appliquent la mise à niveau et qu'il ne subsiste qu'une seule chaîne.

La même distinction existe dans l'activation des hard forks, mais
celle-ci a peu de pertinence, la puissance de calcul ne pouvant pas
empêcher la scission. Ainsi, le hard fork activé par les mineurs ou
MAHF, longtemps soutenu par les partisans de l'augmentation de la taille
limite des blocs, n'a pas d'intérêt particulier. Comme les hard forks,
les soft forks peuvent être rangés en deux catégories plus ou moins
distinctes~: le soft fork de mise à niveau et le soft fork contentieux.
Le soft fork est idéal pour mettre à niveau le protocole. Cela permet
aux nœuds de ne pas se mettre à niveau tout de suite. Même s'il demande
une certaine synchronisation, celle-ci n'est pas aussi contraignante que
pour les hard forks.

Dans BTC, le soft fork est ainsi privilégié par les développeurs depuis
sa découverte. De nombreuses mises à niveau en étaient, comme \emph{Pay
to Script Hash} (BIP-16), ou l'obligation de spécifier la hauteur du
bloc dans la transaction de récompense (BIP-34), ou encore l'ajout d'un
standard d'encodage des signatures (BIP-66). Les ajouts des codes
opération \texttt{et} permettant l'usage de verrous temporels dans le
langage de script par l'utilisation respective des codes
\texttt{OP\_NOP2} et \texttt{OP\_NOP3} ont également été des soft forks.
Enfin, plus récemment, l'adoption de Schnorr-Taproot (ou Taproot pour
faire court) a été une mise à niveau par soft fork.

Litecoin fait aussi usage de ce type de transition. Le protocole a
notamment intégré SegWit en mai 2017, ainsi que Schnorr-Taproot et
MimbleWimble (MWEB) en mai 2022.

Le soft fork contentieux a pour objectif de contraindre la minorité de
la communauté à suivre la majorité. S'il réussit, il n'y a qu'une seule
chaîne, les opposants ayant le choix d'accepter les règles ou de
procéder eux-mêmes à un hard fork. S'il échoue, il en résulte deux
chaînes concurrentes.

SegWit est l'exemple typique d'un soft fork contentieux réussi. Il
n'était pas approuvé par l'ensemble des acteurs importants (les
partisans des gros blocs d'une part, les puristes du protocole comme
Mircea Popescu d'autre part\footnote{Mircea Popescu, \emph{There's a one
  Bitcoin reward for the death of Pieter Wuille. Details below.}, 10
  décembre 2015~:
  \url{http://trilema.com/2015/theres-a-one-bitcoin-reward-for-the-death-of-pieter-wuille-details-below/}.},
s'y opposaient), mais il a recueilli un soutien majoritaire de sorte
qu'il a pu perdurer et que les \emph{big blockers} mécontents ont dû
migrer vers Bitcoin Cash.

Un exemple de soft fork contentieux ayant échoué est la tentative de
l'équipe de Bitcoin ABC d'imposer une redirection de 8~\% de la
subvention de minage de Bitcoin Cash à son propre profit le 15 novembre
2020\footnote{Amaury Séchet, \emph{Bitcoin ABC's plan for the November
  2020 upgrade}, 6 août 2020~:
  \url{https://amaurysechet.medium.com/bitcoin-abcs-plan-for-the-november-2020-upgrade-65fb84c4348f}.}.
Cette tentative, qui était un soft fork en raison de son caractère
restrictif, a provoqué la scission entre une branche majoritaire sans
redirection (BCH) et une branche minoritaire avec, qui a été par la
suite renommée en «~eCash~» (XEC).

Ainsi, le soft fork, qu'il soit approuvé à l'unanimité ou bien seulement
par une majorité, est une méthode supérieure au hard fork. Bien qu'il
soit parfois plus complexe, il permet de ne pas requérir une
synchronisation de l'économie entière, cette dernière pouvant s'y
adapter progressivement, ce qui est un bienfait non négligeable dans le
cas d'un système ouvert utilisé par une grande diversité de personnes
comme Bitcoin. Le signalement supermajoritaire des mineurs permet de
minimiser le risque de scission et de conserver l'effet de réseau au
maximum.

Mais cet avantage majeur se fait au prix d'un sacrifice~: celui de la
clarté du consentement. Dans le cas du hard fork, le consentement est
clair~: les personnes qui souhaitent la modification se retrouvent sur
la chaîne qu'elles ont choisie. Dans le cas du soft fork, le
consentement est plus ambigu~: le fait d'opérer sur la chaîne n'indique
pas nécessairement une acceptation active du changement, mais une
résignation passive et un refus de réaliser un hard fork minoritaire.
Comme l'écrivait brillamment Vitalik Buterin en mars 2017~:

«~Les soft forks favorisent clairement la coercition par rapport à la
sécession d'un point de vue systémique, alors que les hard forks ont le
penchant inverse\footnote{Vitalik Buterin, \emph{Hard Forks, Soft Forks,
  Defaults and Coercion}, 14 mars 2017~:
  \url{https://vitalik.ca/general/2017/03/14/forks_and_markets.html}.}.~»

Ainsi, même s'ils sont supérieurs de manière générale, les soft forks ne
sont pas adaptés à toutes les situations.

\section*{L'évolution plurielle de
Bitcoin}\label{luxe9volution-plurielle-de-bitcoin}
\addcontentsline{toc}{section}{L'évolution plurielle de Bitcoin}

\markright{L'évolution plurielle de Bitcoin}

Le fonctionnement ouvert et libre de l'évolution de Bitcoin fait que le
protocole peut être modifié à volonté. Bitcoin évolue de manière
organique, lentement mais sûrement~: il n'est pas un système figé, dont
les règles seraient dictées par une autorité centrale. Et, par là, il
s'améliore avec le temps.

Cette ouverture implique aussi que la mise en œuvre de Bitcoin est
nécessairement plurielle. Bitcoin n'est pas un système unique, mais un
modèle ouvert qui est appliqué de façon plus ou moins fidèle par
plusieurs protocoles. L'ensemble des mises en œuvres de Bitcoin
constitue un arbre dont les branches proviennent d'un même tronc et des
mêmes racines.

Toutefois, toutes les branches ne sont pas équivalentes~: toutes les
mises en œuvre n'ont pas la même importance. L'une d'entre elle (BTC)
est aujourd'hui supermajoritaire, de sorte que nous l'appelons
naturellement Bitcoin, et sa modification est (heureusement) difficile.
Dans le prochain chapitre, nous examinerons le mécanisme sous-jacent qui
fait que Bitcoin est ce qu'il est aujourd'hui et comment l'évolution du
protocole est gouvernée.

\begin{figure}

{\centering \includegraphics{chapters/img/bitcoin-forks-tree.png}

}

\caption{Variations conceptuelles, modifications logicielles et forks de
consensus de Bitcoin.}

\end{figure}%

\bookmarksetup{startatroot}

\chapter{La détermination du protocole}\label{ch:determination}

\phantomsection\label{enotezch:11}{}

{D}\textsc{a}ns Bitcoin, le protocole est l'ensemble ouvert de règles
qui interviennent dans la formation et la transmission des blocs et des
transactions sur le réseau. Il est notamment constitué des règles de
consensus qui régissent la validité du registre sur lequel les nœuds du
réseau se mettent d'accord. Ces règles sont mises en œuvre par des
implémentations logicielles, qui peuvent être librement copiées,
modifiées et réutilisées à volonté.

Cette nature ouverte et libre fait qu'il n'y a pas d'autorité centrale
qui décrète quelles sont les règles comme cela se fait dans les modèles
centralisés, mais que cette prise de décision est répartie au sein de la
communauté. C'est pourquoi la détermination du protocole n'est pas un
mécanisme technique mais économique, conformément à la nature
essentiellement monétaire de Bitcoin.

Il s'agit d'un thème d'importance majeure, car ce mécanisme de
détermination garantit l'intégrité des règles de consensus et, par
conséquent, le bon fonctionnement du système. En particulier, c'est de
lui que provient la fameuse résistance à l'inflation, à savoir la
difficulté à créer plus de bitcoins. Il est donc fondamental d'avoir une
bonne conception de ce mécanisme si nous voulons nous convaincre de la
viabilité de Bitcoin.

\section*{La résistance à
l'inflation}\label{la-ruxe9sistance-uxe0-linflation}
\addcontentsline{toc}{section}{La résistance à l'inflation}

\markright{La résistance à l'inflation}

L'une des deux grandes promesses de Bitcoin est de résister à
l'inflation monétaire, c'est-à-dire de rendre difficile la création
supplémentaire d'unités par rapport à ce qui est accepté par le marché.
Cette promesse est énorme~: comme nous l'avons vu dans le
chapitre~\hyperref[ch:adversaire]{4}, l'État fait tout ce qu'il peut
pour profiter de la création monétaire, phénomène qu'on appelle le
seigneuriage. De prime abord, il paraît ainsi étonnant qu'un objet
numérique puisse posséder une telle propriété.

La politique monétaire classique du bitcoin a été établie par Satoshi
Nakamoto lors du lancement du prototype le 8 janvier 2009\footnote{«~établie
  par Satoshi Nakamoto lors du lancement du prototype le 8 janvier
  2009~»~: Satoshi Nakamoto, \emph{Bitcoin v0.1 released}, /01/2009
  19:27:40 UTC~:
  \url{https://www.metzdowd.com/pipermail/cryptography/2009-January/014994.html}.}.
Celle-ci possédait un caractère simple~: la création monétaire devait
être réduite de moitié tous les quatre ans, de façon à devenir
négligeable au fil du temps. Il devait se créer 10,5 millions d'unités
de manière linéaire les quatre premières années, 5,25 millions les
quatre suivantes, 2,625 millions les quatre d'après, et ainsi de suite,
ce qui limitait le nombre d'unités en circulation à 21 millions. Le
bitcoin devait finir par devenir une monnaie à quantité fixe, quelque
chose qui n'a jamais été vu dans l'histoire.

Cette politique monétaire a été l'un des arguments de vente de Bitcoin,
si bien que certaines personnes se sont imaginées qu'il s'agissait d'une
chose immuable, gravée dans le marbre, et que l'application mathématique
de ce décret de Satoshi Nakamoto était ce qui garantissait la résistance
à l'inflation du système. Par exemple, Tyler Winklevoss, ayant investi
dans le bitcoin avec son frère jumeau Cameron, se convainquait en 2013
qu'il avait acheté un actif dépourvu d'intervention humaine~:

«~Nous avons choisi de placer notre argent et notre confiance dans un
cadre mathématique exempt de politique et d'erreur humaine\footnote{Nathaniel
  Popper, Peter Lattman, «~\emph{As Big Investors Emerge, Bitcoin Gets
  Ready for its Close-Up}~», \emph{CNBC}, 11 avril 2013~:
  \url{https://www.cnbc.com/id/100635418}.}.~»

Toutefois, cette conception est au mieux une approximation maladroite.
Il ne suffit pas qu'une règle ait été décrétée par quelqu'un dans le
passé pour qu'elle se manifeste dans la réalité présente~; il faut aussi
que d'autres personnes l'acceptent et l'appliquent. Et cette acceptation
est précisément soumise à la politique et à l'erreur humaine.

Il existe au sujet de la politique monétaire fixe du bitcoin une
certaine confusion. Il faut dire que Satoshi Nakamoto n'a jamais précisé
comment elle pouvait être protégée. Plusieurs théories ont été
proposées, allant de l'intervention des mineurs à l'exigence d'unanimité
communautaire\footnote{«~l'exigence d'unanimité communautaire~»~: River
  Financial, \emph{Can Bitcoin's Hard Cap of 21 Million Be Changed?}~:
  \url{https://river.com/learn/can-bitcoins-hard-cap-of-21-million-be-changed}.},
en passant par le caractère juridique du décret de Satoshi\footnote{«~le
  caractère juridique du décret de Satoshi~»~:
  \url{https://craigwright.net/blog/law-regulation/forking-and-passing-off/}.}.
Dans tous les cas, il s'agissait de traiter la question de la
«~gouvernance\footnote{Pierre Rochard, \emph{Bitcoin Governance}, 9
  juillet 2018~:
  \url{https://pierre-rochard.medium.com/bitcoin-governance-37e86299470f}.}~»
ou du «~consensus social\footnote{Arthur Breitman parlait de
  \emph{social consensus} dès août 2014 dans la première description
  formelle de Tezos. -- Arthur Breitman, \emph{Tezos: A Self-Amending
  Crypto-Ledger}, 3 août 2014~:
  \url{https://tezos.com/position-paper.pdf}.}~», c'est-à-dire de la
façon dont les règles sont décidées dans Bitcoin. C'est cette
problématique que nous appelons ici la détermination du protocole.

\section*{Le pouvoir des commerçants sur le
protocole}\label{le-pouvoir-des-commeruxe7ants-sur-le-protocole}
\addcontentsline{toc}{section}{Le pouvoir des commerçants sur le
protocole}

\markright{Le pouvoir des commerçants sur le protocole}

Tel que nous l'avons laissé entendre, la détermination du protocole est
accomplie par l'économie. Puisque Bitcoin est un système économique, il
est naturel que les règles qui le composent résultent du marché, et non
d'un décret fixe passé ou d'une autorité centrale actuelle.

L'idée que l'économie permet de déterminer les règles n'est pas
nouvelle. Elle remonte au moins au printemps 2012 lorsque Meni Rosenfeld
écrivait sur Stack Overflow qu'un changement du protocole nécessitait
«~une majorité économique, c'est-à-dire l'adoption par les utilisateurs
et les entreprises qui donnent de la valeur à la monnaie\footnote{Meni
  Rosenfeld, \emph{Re: How could the bitcoin protocol be changed? Has
  this ever occurred?}, /06/2012 13:53:19 UTC~:
  \url{https://bitcoin.stackexchange.com/questions/3945/how-could-the-bitcoin-protocol-be-changed-has-this-ever-occurred\#comment4983_3948}.}~».
Gavin Andresen lui-même a mis en avant cette idée en mai 2015, alors que
la question d'augmenter la taille limite des blocs se posait~:

«~Si nous ne parvenons pas à un consensus ici, l'autorité ultime pour
déterminer le consensus est le code utilisé par la majorité des
commerçants, des plateformes de change et des mineurs\footnote{Gavin
  Andresen, \emph{{[}Bitcoin-development{]} Proposed alternatives to the
  20MB step function}, /05/2015 12:39:30 UTC,
  \url{https://lists.linuxfoundation.org/pipermail/bitcoin-dev/2015-May/008340.html}.}.~»

Mais la clarté de cette conception n'est arrivée qu'après les évènements
de la guerre des blocs, au cours de laquelle les mécanismes sous-jacents
ont pu s'exprimer. Ce n'étaient pas les développeurs qui décidaient des
règles, ce n'étaient pas les mineurs non plus, mais plutôt les
utilisateurs, et plus précisément les \emph{commerçants}. Eric Voskuil
écrivait ainsi en novembre 2018~:

«~Bitcoin ne repose pas sur un dépositaire, mais dans l'intérêt
d'établir un principe général, on peut considérer l'ensemble de tous les
commerçants comme le dépositaire collectif de Bitcoin\footnote{Eric
  Voskuil, «~Principe de risque de garde~», in \emph{Cryptoéconomie~:
  Principes fondamentaux de Bitcoin}, Amazon KDP, 2022, pp.~34--35.}.~»

Les commerçants, au sens large, sont les personnes qui fournissent des
biens, des services ou d'autres monnaies contre du bitcoin, à des prix
acceptables sur le marché. Cette prestation se manifeste par les
échanges effectifs avec les clients et s'estime par les recettes
perçues. En cela, les commerçants contribuent à l'utilité du bitcoin,
qui se mesure à la quantité de biens et de services qu'il permet
d'acquérir, et par conséquent à l'importance économique de la
chaîne\footnote{Cette réalité a été perçue en janvier 2010 par
  NewLibertyStandard, le premier commerçant de Bitcoin, lorsqu'il a
  déclaré~: «~Toutes les personnes qui achètent ou vendent des biens en
  utilisant des bitcoins, y compris les changeurs, font progresser
  l'économie de Bitcoin~!~» -- NewLibertyStandard, \emph{Re: New
  Exchange Service: "BTC 2 PSC"}, /01/2010 08:06:15 UTC~:
  \url{https://bitcointalk.org/index.php?topic=15.msg111\#msg111}.}. Par
l'utilisation d'un nœud permettant de vérifier les règles de consensus,
ils participent ainsi à la détermination du protocole en proportion de
leur activité économique potentielle.

Parler d'un protocole unique qui changerait est une inexactitude~: en
tant qu'ensemble de règles, les protocoles sont tous fixes, mais leur
utilisation (et leur utilité) varie. Modifier le protocole consiste donc
à constituer un nouveau protocole dont la chaîne résultante sera
économiquement plus importante que toute autre branche concurrente, y
compris celle liée au protocole originel\footnote{Jeff Garzik écrivait
  très justement en octobre 2010 que «~l'effort visant à augmenter la
  limite du taux de transaction {[}était{]} le même que celui visant à
  modifier la nature fondamentale des bitcoins~: convaincre la grande
  majorité de se mettre à niveau~». -- Jeff Garzik, \emph{Re:
  {[}PATCH{]} increase block size limit}, /10/2010 18:33:55 UTC~:
  \url{https://bitcointalk.org/index.php?topic=1347.msg15342\#msg15342}.}.
Par exemple, SegWit a été un soft fork contentieux, mais le protocole
résultant a été beaucoup plus valorisé que les protocoles concurrents
(BTC pré-SegWit et Bitcoin Cash), de sorte qu'on peut dire que le
protocole BTC a été mis à niveau par cette modification.

Bitcoin-le-concept englobe par nature une multiplicité de protocoles, en
raison de son caractère libre et ouvert. Il n'y a pas un seul protocole
Bitcoin, mais plusieurs, comme il y a plusieurs distributions Linux ou
plusieurs dollars. Et ces protocoles sont en concurrence pour acquérir
une utilité en étant adoptés par les commerçants.

Ce qui compte, c'est donc l'importance économique des chaînes créées par
ces protocoles. Chacun peut bien définir Bitcoin comme il le souhaite,
notamment en décrétant qu'il n'y a qu'un seul protocole et qu'il ne peut
pas être modifié sans unanimité, mais cette attitude ne change pas la
réalité économique des choses. Si la chaîne créée par une modification
rassemble 80~\% de l'activité économique, la chaîne suivant les règles
du protocole originel continuerait d'exister, mais serait lourdement
déclassée et perdrait en pertinence. Comme l'écrivait Arthur Breitman en
2014, «~l'option de s'en tenir au protocole originel n'est pas du tout
pertinente si la valeur de ses jetons est annihilée par un changement de
consensus\footnote{Arthur Breitman, \emph{Tezos: A Self-Amending
  Crypto-Ledger}, 3 août 2014~:
  \url{https://tezos.com/position-paper.pdf}.}~».

Tout ceci explique les usages qui se sont développés naturellement dans
l'écosystème. On appelle usuellement Bitcoin la mise en œuvre principale
et dominante économiquement du concept. En cas de scission, le nom et le
sigle boursier du protocole originel sont généralement conservés par la
branche majoritaire, que celle-ci garde les règles initiales
(Bitcoin-BTC) ou qu'elle les modifie (Ethereum-ETH)~; tandis que la
branche minoritaire doit modifier son propre nom, soit en le rallongeant
pour insister sur la continuité (Bitcoin Cash, Bitcoin SV, Ethereum
Classic), soit en le remplaçant par une nouvelle marque
(«~eCash~»~/~XEC).

Cette mécanique économique fait que la résistance au changement provient
des commerçants qui refusent d'intégrer les règles. Ainsi, une
modification qui amoindrirait les propriétés fondamentales de Bitcoin,
comme une introduction de censure ou d'inflation, ne serait effective
que si les commerçants l'acceptaient. Or ceux-ci sont récompensés par
ces propriétés en bénéficiant de la liberté liée à l'absence de censure
(permettant notamment l'évasion fiscale) et de l'augmentation en pouvoir
d'achat des fonds reçus, et sont par conséquent incités à ne pas
accepter un tel changement. En particulier, la «~déflation
naturelle\footnote{Satoshi Nakamoto, \emph{Re: A few suggestions},
  /12/2009 16:51:25 UTC~:
  \url{https://bitcointalk.org/index.php?topic=12.msg62\#msg62}.}~» du
bitcoin forme l'incitation économique qui maintient sa politique
monétaire singulière.

À l'instar de la sécurité minière, la sécurité commerciale d'une chaîne,
c'est-à-dire la difficulté à en modifier les propriétés fondamentales,
ne dépend pas uniquement de l'activité économique de la chaîne (les
recettes), mais aussi de la distribution de cette activité économique et
du nombre de commerçants par rapport au reste de la population
mondiale\footnote{Eric Voskuil, «~Modèle de sécurité qualitatif~», in
  \emph{Cryptoéconomie~: Principes fondamentaux de Bitcoin}, Amazon KDP,
  2022, pp.~59--62.}. Une activité économique concentrée dans les mains
d'un seul acteur rend très facile toute modification du protocole. De
même, si l'activité économique est élevée et équitablement répartie
entre un petit nombre de commerçants, alors le protocole a plus de
chances d'être modifié qu'en présence d'un grand nombre de commerçants.

Tout comme les mineurs qui délèguent leur pouvoir sur la sélection des
transactions («~hacheurs~»), les commerçants peuvent déléguer leur
pouvoir sur la vérification des règles de consensus. Les commerçants
abandonnent ce pouvoir aux services délégataires à qui ils versent une
commission dans le but de réduire la difficulté d'utilisation (le
déploiement d'un nœud) et le coût de transaction (lié aux remises des
frais). Ces services délégataires peuvent être des fournisseurs de
portefeuille (Electrum, Acinq, Edge, Ledger, Trezor), des processeurs de
paiement (BitPay, Coinbase Commerce) ou même des explorateurs de blocs
(Blockchair, Mempool.space).

La délégation de la vérification pose un problème évident de
centralisation. Même si l'économie peut s'adapter rapidement et
redevenir saine à moyen terme par le déploiement de nouveaux nœuds, la
sécurité commerciale instantanée de la chaîne est affectée par cette
délégation et une attaque de modification ou de suppression du protocole
peut causer des dégâts à court terme non négligeables.

Cet impact peut être d'autant plus fort si la délégation s'accompagne
d'une délégation de la propriété auprès d'un dépositaire, auquel cas le
réel commerçant devient le dépositaire en question, celui-ci ayant un
contrôle total sur les fonds. C'est notamment le cas des places de
marché en ligne qui achètent et vendent d'autres monnaies en bitcoins,
tout en mettant en place des carnets d'ordres internes pour résoudre
l'offre et la demande.

À l'heure d'écriture de ces lignes, la situation dans Bitcoin est
particulière, car l'activité économique est dominée par le change entre
le bitcoin et les monnaies officielles. Déjà à l'époque de Satoshi, les
changeurs constituaient les premiers commerçants de Bitcoin~: la
première chose achetée avec du bitcoin n'était pas une pizza comme on
aime le penser, mais de la monnaie, à savoir 5,02 dollars sur
PayPal\footnote{Le premier commerçant, NewLibertyStandard, a «~vendu~»
  5,02~\$ contre 5~050 BTC à Martti Malmi, le premier client, le 12
  octobre 2009. On peut aussi arguer que le mineur du bloc 2~817 qui a
  reçu 2 BTC en frais de transaction le 3 février 2009 a techniquement
  été le premier commerçant pour son service, mais la somme impliquée
  était négligeable.}. Aujourd'hui, les plateformes de change
centralisées telles que Kraken, Coinbase et Binance ont pris la relève,
ce qui fait que l'économie est aujourd'hui extrêmement centralisée et
sensible aux attaques.

Comme dans le cas de l'attaque de censure, l'attaque d'altération des
propriétés fondamentales de Bitcoin ne risque pas de provenir d'un
acteur économique rationnel, qui n'a aucun intérêt à le faire, mais
plutôt d'agents politiques agissant au nom de l'État. La nature d'une
telle attaque répondrait donc aux prérogatives étatiques comme la lutte
contre le blanchiment des capitaux et le financement du terrorisme
(LCB-FT) ou bien l'opposition à la spéculation contre la monnaie
nationale. Les plateformes de change, hautement réglementées seraient
les premières concernées par une telle attaque.

Ainsi, ce sont les commerçants qui déterminent le protocole en
choisissant les règles de consensus qui leur conviennent, qu'ils
vérifient systématiquement par l'intermédiaire de leurs nœuds. Le
pouvoir individuel du commerçant est pondéré par son offre économique
susceptible d'être acceptée, qui est estimée par son activité économique
réelle. Cependant, ce pouvoir n'est pas linéaire, dépendant en
particulier de l'effet de réseau.

\section*{L'effet de réseau}\label{leffet-de-ruxe9seau}
\addcontentsline{toc}{section}{L'effet de réseau}

\markright{L'effet de réseau}

Le pouvoir direct d'un commerçant n'est pas purement individuel. Bitcoin
étant une monnaie, il est soumis à des effets économiques, dont le
principal est l'effet de réseau. Ce dernier fait qu'il y a un nombre
moins élevé de mises en œuvre de Bitcoin que ce qu'on attendrait pour un
produit matériel classique.

L'effet de réseau est le phénomène par lequel l'utilité réelle d'une
technique ou d'un produit dépend de la quantité de ses utilisateurs. Il
s'agit d'un effet qui s'auto-alimente, qui fonctionne comme un cercle
vertueux~: plus un système compte d'utilisateurs, plus il a tendance à
en attirer de nouveaux.

Une monnaie est un réseau social et est donc soumise à l'effet de
réseau. L'utilité globale du réseau n'évolue pas de façon linéaire par
rapport à la taille de son économie, mais de façon superlinéaire. C'est
ce qu'exprime la loi de Metcalfe qui stipule que «~l'utilité d'un réseau
est proportionnelle au carré du nombre de ses utilisateurs\footnote{La
  loi de Metcalfe tient son nom de Robert Metcalfe, co-créateur du
  protocole Ethernet et fondateur de 3com, qui avait observé cet effet
  en 1980 au sujet de dispositifs communicants compatibles. La loi a été
  formellement énoncée par George Gilder en 1993 dans un article publié
  dans Forbes. Elle faisait varier l'utilité du réseau en \(n^2\) où
  \(n\) est le nombre d'utilisateurs, ce qui surestimait grossièrement
  l'effet de réseau réel. Une deuxième loi plus conservatrice, la loi
  d'Odlyzko, a été proposée en 2006 pour faire varier l'utilité du
  réseau en \(n~\textrm{log}(n)\). -- George Gilder, \emph{Metcalf's Law
  and Legacy}, 1 septembre 1993~:
  \url{https://www.discovery.org/a/41/}~; Bob Briscoe, Andrew Odlyzko,
  Benjamin Tilly, \emph{Metcalfe's Law is Wrong}, 1 juillet 2006~:
  \url{https://spectrum.ieee.org/metcalfes-law-is-wrong}.}~».

Lors de l'émergence d'Internet, la demande pour un protocole commun a
fait que TCP/IP a prévalu sur le modèle concurrent de l'époque, le
modèle OSI\footnote{«~TCP/IP a prévalu sur le modèle concurrent de
  l'époque, le modèle OSI~»~: Andrew L. Russell, «~\emph{"Rough
  Consensus and Running Code" and the Internet-OSI Standards War}~», in
  \emph{IEEE Annals of the History of Computing}, vol.~28, no. 3,
  juillet-septembre 2006, pp.~48--61~:
  \url{https://courses.cs.duke.edu/common/compsci092/papers/govern/consensus.pdf}.}.
De même, seul un nombre réduit de langues peut exister en raison des
contraintes induites par la communication. En ce qui concerne les
relations commerciales et diplomatiques internationales, il n'y a ainsi
généralement qu'une seule langue véhiculaire («~\emph{lingua franca}~»)
au sein d'une aire géographique donnée. C'était le cas de l'araméen et
le grec de la koinè au Proche-Orient durant l'Antiquité, de l'italien en
Europe au début de la Renaissance, du français comme langue diplomatique
aux \textsc{xviii} et \textsc{xix}~siècles, et c'est le cas de l'anglais
dans le monde aujourd'hui.

Pour la monnaie, cet effet provient de la préférence personnelle pour
une seule monnaie, qui s'explique d'une manière interne par le coût
(mental) du calcul économique qui découle de la gestion de plusieurs
monnaies, et d'une manière externe par le coût de change qui est payé
pour la conversion d'une monnaie en une autre. De ce fait, les individus
ont tendance à privilégier l'usage de la monnaie la plus populaire,
quand bien même celle-ci serait défectueuse. C'est également ce qui fait
qu'une monnaie utilisée par un petit nombre de personnes doit présenter
un avantage non négligeable par rapport aux autres si elle veut
perdurer. Avec le temps, les monnaies tendent à se consolider en une
seule, même s'il subsiste des barrières à ce résultat.

Dans Bitcoin, l'effet de réseau monétaire prédomine. Même s'il n'est pas
le seul effet de réseau, il est celui qui conduit les autres effets
(liés à la liquidité, au développement informatique, à la sécurité
économique et à la communication mercatique) à s'exprimer\footnote{Vitalik
  Buterin, \emph{On Bitcoin Maximalism, and Currency and Platform
  Network Effects}, 19 novembre 2014~:
  \url{https://blog.ethereum.org/2014/11/20/bitcoin-maximalism-currency-platform-network-effects}.}.

L'effet de réseau joue ainsi un rôle \emph{capital} dans la
détermination du protocole. L'existence d'un nombre limité de mises en
œuvre viables de Bitcoin et leur stabilité provient de cet effet. C'est
ce qui explique pourquoi l'existence d'une supermajorité économique est
souvent exigée avant de procéder à une modification du protocole. C'est
également ce qui encourage l'ossification du protocole qui se bâtit face
à la multiplication des propositions de changement. Il existe un point
de Schelling naturel qui s'oppose à l'altération des règles de
consensus\footnote{Le point de Schelling est, en théorie des jeux, une
  solution à laquelle les participants à un jeu de coordination pure ne
  pouvant communiquer auront tendance à se rallier, parce qu'elle leur
  semble présenter une caractéristique qui la fera choisir aussi par
  l'autre. L'exemple typique est l'endroit où peuvent se retrouver des
  gens en voyage dans un lieu, qui sera généralement un monument connu
  de tous, la Tour Eiffel à Paris par exemple. -- Thomas C. Schelling,
  \emph{The Strategy of Conflict}, Harvard University Press, 1960.}~: en
l'absence de volonté claire de modifier les règles ou dans le cas d'une
dispute, l'option de ne rien faire est privilégiée.

L'existence de l'effet de réseau explique la tendance au maximalisme qui
se manifeste au sein de communautés liées à des protocoles et unités de
compte particulières. Puisqu'il ne doit y avoir (logiquement) qu'un seul
Bitcoin, toute tentative de faire varier le concept s'apparente à une
démarche vaine et contreproductive, sinon à une escroquerie. Mais le
maximalisme ignore en cela l'effet de substitution.

\section*{L'effet de substitution}\label{leffet-de-substitution}
\addcontentsline{toc}{section}{L'effet de substitution}

\markright{L'effet de substitution}

Le second effet économique principal qui agit sur le pouvoir individuel
des commerçants est l'effet de substitution. Celui-ci s'oppose
diamétralement à l'effet de réseau, et a pour conséquence de créer un
nombre plus élevé de mises en œuvre de Bitcoin que ce à quoi on pourrait
s'attendre si le concept n'était pas naturellement limité.

Un produit de substitution est, en économie, un bien ou un service qui
peut être utilisé dans le même but qu'un autre, mais qui présente des
caractéristiques différentes de ce dernier. L'idée est que le
consommateur va demander le produit de substitution parce que celui-ci
est moins cher ou plus efficace dans la satisfaction apportée. Les
exemples sont nombreux~: le blé ou le riz pour l'apport en glucides, le
café et le thé pour la consommation de caféine, le train et l'avion pour
le transport en commun, etc. La substitution est généralement
imparfaite, dans le sens où le produit va posséder des différences ne
pouvant pas être quantifiées.

L'effet de substitution se manifeste lorsque les conditions de marché
changent de manière drastique. Le produit de base peut devenir plus
cher, ou moins abondant~; il peut devenir moins cher, ou plus abondant~;
ou bien le niveau de vie des gens peut augmenter ou baisser de telle
sorte qu'ils se mettent à préférer un produit à l'autre. Dans tous les
cas, il faut qu'un changement arrive pour que la substitution se
produise.

Cet effet de substitution se retrouve également dans les monnaies, et
peut s'exprimer par exemple lorsque la monnaie officielle s'effondre,
dans les pays en hyperinflation par exemple, ou qu'elle est interdite,
comme dans les prisons. On observe alors un phénomène de monétisation
des biens qui n'étaient initialement pas utilisés en tant que tels comme
les voitures ou les cigarettes.

Avec Bitcoin, cet effet de substitution s'exerce de manière
particulière. D'un côté, toute mise en œuvre de Bitcoin est limitée par
un plafond de capacité transactionnelle, qui est souvent explicité par
une taille ou un poids maximal des blocs\footnote{L'implémentation des
  solutions de surcouche telles que le réseau Lightning ne font
  qu'améliorer la capacité effective de transfert de valeur, comme nous
  le montrerons dans le chapitre~\hyperref[ch:scalabilite]{14}.}. De
l'autre, le nombre de bitcoins est aussi limité. De ce fait, lorsque la
demande d'activité monétaire augmente, il ne se crée pas plus de
bitcoins, mais le coût d'inclusion dans un bloc augmente.

Cette particularité a pour effet d'exclure économiquement les
transactions qui déplacent des sommes trop faibles pour que leur
inscription sur la chaîne soit jugée rentable. Toutefois, la demande
pour réaliser ces transferts ne disparaît pas. C'est pourquoi elle se
retrouve partiellement sur des chaînes alternatives à bas frais, comme
Litecoin ou Bitcoin Cash, dont la sécurité est moindre que celle de
BTC\footnote{Le déplacement effectif des transferts vers des systèmes
  apparentés moins chers a notamment été observé par Matt Ahlborg,
  consultant en étude de marché pour Bitrefill, une plateforme de vente
  de recharge téléphonique et de cartes-cadeaux. -- Matt Ahlborg sur
  Twitter, /04/2023 14:14 UTC~:
  \url{https://twitter.com/MattAhlborg/status/1647966711126147072}.}.

De même, ce qu'on caractérise souvent par un manque de fonctionnalités
dans Bitcoin-BTC correspond à une question de coût. Il est possible de
simuler toutes les fonctionnalités présentes sur les autres chaînes
d'une manière ingénieuse et détournée, mais il est bien moins coûteux et
plus facile d'utiliser des protocoles qui les intègrent directement.
C'est le cas de la confidentialité avec Monero, ou de la programmabilité
générale avec Ethereum-ETH et Ethereum Classic.

Ainsi, l'effet de substitution joue un rôle important dans Bitcoin et
dans les systèmes cryptoéconomiques en général, ce qui explique
l'existence de nombreuses «~cryptomonnaies alternatives~». En l'absence
de cet effet, l'activité économique aurait convergé naturellement vers
un seul protocole (BTC), mais on peut voir que ce n'est pas le cas,
notamment lors des congestions du réseau.

La présence de l'effet de substitution explique la tendance vers un
pluralisme cryptomonétaire extrême, dont les partisans prétendent que
n'importe quelle technique légèrement supérieure pourrait détrôner le
premier protocole du marché. Mais en cela, ils négligent lourdement
l'effet de réseau et commettent ainsi l'erreur opposée à celle des
maximalistes.

\section*{Pouvoir et influence}\label{pouvoir-et-influence}
\addcontentsline{toc}{section}{Pouvoir et influence}

\markright{Pouvoir et influence}

Pour comprendre plus finement comment le protocole en arrive à être ce
qu'il est, il faut différencier le pouvoir de l'influence qui s'exercent
au sein de Bitcoin. Dans le monde réel, le pouvoir est la capacité de
faire quelque chose sans un consentement tiers, ce qui se traduit en
dernier lieu par l'intervention de la force physique. L'influence est
quant à elle la capacité à influer sur le choix de ceux qui détiennent
le pouvoir, typiquement les forces religieuses.

Dans Bitcoin, le pouvoir se transcrit par le pouvoir économique des
commerçants sur le protocole. Par l'intermédiaire de leurs nœuds, ils
vérifient les règles de consensus liées à l'unité qu'ils acceptent dans
le commerce et apportent de ce fait une utilité économique à cette
unité. Toutefois, ce pouvoir économique direct est bien souvent
influencé par de nombreux acteurs.

Ces influences sont prises en compte dans le modèle de
gouvernance\footnote{«~modèle de gouvernance~»~: La gouvernance (mot
  venant du latin \emph{gubernare}, «~diriger un navire~») désigne la
  manière dont est dirigée une entité sociale, qu'elle se rapporte à un
  groupe humain spécifique (famille, tribu, entreprise, nation) ou à
  autre chose (projet, réseau, langue). Popularisée par son usage en
  entreprise, la gouvernance n'implique pas nécessairement le
  gouvernement et peut être issue de l'interaction volontaire entre les
  individus.} classique de Bitcoin, qui fait généralement intervenir le
triplet développeurs-mineurs-utilisateurs. Ces derniers acteurs forment
des forces intérieures du système, car ils y participent plus ou moins
directement.

En outre, l'influence sur le protocole provient également des entités
extérieures qui interviennent de manière diffuse sur l'ensemble du
système. Il est impossible d'établir une liste exhaustive de ces
influences extérieures, mais on peut en identifier les principales.
Elles s'exercent en substance de trois manières -- par le discours, par
l'argent et par la force -- et se rapportent à trois catégories
d'acteurs~: les relais d'opinion, les puissances financières et les
forces étatiques.

Ainsi, les commerçants subissent un ensemble d'influences de la part de
participants internes au système, comme les développeurs, les mineurs et
les autres utilisateurs, mais également de la part d'acteurs externes,
comme les relais idéologiques, les financiers et les régulateurs. Par
ailleurs, ces groupes interagissent mutuellement, de sorte que le tout
forme un ensemble sociologique complexe, qui influe sur le choix final
du protocole. Même si expliquer comment cet ensemble complexe se
comporte ne rentre pas dans nos compétences, essayons d'en esquisser un
modèle général en nous attardant sur les pressions intérieures avant
d'examiner les forces extérieures, plus diffuses.

\begin{figure}[H]

{\centering \includegraphics{chapters/img/bitcoin-governance-model.png}

}

\caption{Un modèle de gouvernance de Bitcoin~: interactions des
principaux groupes d'acteurs dans la détermination du protocole.}

\end{figure}%

\section*{L'influence des
développeurs}\label{linfluence-des-duxe9veloppeurs}
\addcontentsline{toc}{section}{L'influence des développeurs}

\markright{L'influence des développeurs}

La première catégorie d'acteurs internes est formée des développeurs.
Les développeurs sont les personnes qui travaillent directement au
maintien et aux mises à niveau des implémentations complètes ou
partielles du protocole. Ils œuvrent en particulier à la bonne santé de
la chaîne par le biais des implémentations utilisées par les commerçants
et par les mineurs. L'implémentation de référence, qui est la plus
utilisée et qui sert de modèle aux autres implémentations, est la plus
importante.

Ce rôle d'intermédiaire leur confère une influence non négligeable sur
les commerçants et les autres acteurs, qui ont rarement les capacités
d'observer et de comprendre le code directement. De plus, le maintien
d'un logiciel performant demande un travail coûteux qui ne peut pas être
réalisé par n'importe qui. Cette situation leur donne une position de
force dans la prise de décision sur le protocole.

Les développeurs sont nombreux et possèdent diverses opinions. Pour
remédier à ce problème, ils fondent souvent leur décision sur le concept
de consensus approximatif (\emph{rough consensus}) qui n'est pas un
consensus à proprement parler, mais l'estimation d'un sentiment de
groupe ou d'une volonté générale. Ce recours au consensus approximatif
permet en pratique d'obtenir une quasi unanimité sans qu'un élément
individuel puisse perturber le processus\footnote{Ce concept de
  \emph{rough consensus} provient de son utilisation en 1998, par
  l'\emph{Internet Engineering Task Force} (IETF), qui le décrivait
  comme suit dans ses procédures pour les groupes de travail~: «~Les
  groupes de travail prennent des décisions au moyen d'un processus de
  ``consensus approximatif''. Le consensus IETF ne requiert pas que
  chaque participant soit d'accord, bien que cela soit bien entendu
  préférable. De façon générale, l'opinion dominante du groupe de
  travail doit prévaloir (cependant, cette ``dominance'' ne doit pas
  être déterminée sur la base du volume ou de l'insistance, mais plutôt
  selon une impression plus générale d'accord). Le consensus peut être
  déterminé au moyen d'un vote à main levée, ou de n'importe quel autre
  moyen sur lequel le groupe de travail est d'accord. Il convient de
  noter que 51~\% des voix ne peut être considéré comme un ``consensus
  approximatif'', et qu'en sens inverse, 99~\% est mieux
  qu'approximatif. C'est au président de déterminer si un consensus
  approximatif est atteint.~» -- \emph{IETF Working Group Guidelines and
  Procedures}, septembre 1998~:
  \url{https://datatracker.ietf.org/doc/html/rfc2418}.}. Cette façon
d'exclure les éléments récalcitrants peut être critiquée (une personne
peut avoir raison contre le groupe), mais elle a l'intérêt de préserver
l'effet de réseau du protocole, en offrant une proposition unique aux
commerçants.

Pour BTC, l'implémentation de référence est Bitcoin Core, dirigée par
des mainteneurs. Ces mainteneurs, et plus généralement les développeurs,
sont vus comme les gardiens du protocole. L'utilisation d'une autre
implémentation (\emph{fork}) est toujours possible mais est à la fois
coûteuse et mal vue, de sorte qu'il existe une inertie jouant en faveur
de Bitcoin Core.

Cette dominance s'est manifestée au cours de l'histoire de Bitcoin par
le rejet d'un certain nombre de dissidences, qui ont parfois donné lieu
à la création d'une implémentation alternative. On peut citer~:

\begin{itemize}
\item
  Mike Hearn, qui, en 2014, voulait ajouter une requête de réseau
  \texttt{getutxos} à Bitcoin Core mais qui a été refusée pour cause de
  non-unanimité, ce qui a mené à la création de Bitcoin XT~;
\item
  Les partisans de l'augmentation de la limite de capacité
  transactionnelle du réseau durant la guerre des blocs, qui ont mis en
  place de multiples implémentations pour tenter, en vain, de faire
  adopter ce changement~: Bitcoin XT mi-2015, Bitcoin Classic début
  2016, Bitcoin Unlimited mi-2016 et btc1 mi-2017~;
\item
  Les opposants à la mise à niveau SegWit, soutenue largement par
  Bitcoin Core, qui n'ont eu d'autre choix que de développer Bitcoin
  ABC, qui augmentait dans le même temps la taille limite des blocs,
  menant à la création de Bitcoin Cash~;
\item
  Jeremy Rubin, qui a menacé de faire activer le BIP-119 (soft fork) par
  les mineurs en 2022, en raison du refus de Bitcoin Core d'intégrer sa
  modification au logiciel, mais qui a fini par se raviser, ayant
  probablement obtenu l'attention qu'il désirait\footnote{«~Jeremy
    Rubin, qui a menacé de faire activer le BIP-119 (soft fork) par les
    mineurs en 2022~»~: Jeremy Rubin, \emph{7 Theses on a next step for
    BIP-119}, 17 avril 2022~:
    \url{https://rubin.io/bitcoin/2022/04/17/next-steps-bip119/}~;
    archive~:
    \url{https://web.archive.org/web/20220419172825/https://rubin.io/bitcoin/2022/04/17/next-steps-bip119/}.
    -- On peut rapprocher son cas de celui de Paul Sztorc, qui travaille
    sur son concept de Drivechain depuis 2017, mais dont les
    propositions d'amélioration (BIP-300 et BIP-301) n'ont pas été
    intégrées par Bitcoin Core.}.
\end{itemize}

Les développeurs, et notamment ceux de Bitcoin Core, exercent ainsi une
influence importante sur le protocole. Cependant cette influence reste
limitée~: dans le cas où ils s'opposeraient à l'économie de façon trop
tranchée, ces derniers seraient remplacés par d'autres développeurs.

Le premier exemple d'une dissidence réussie se trouve dans l'histoire
des débuts de Monero\footnote{dEBRUYNE, \emph{Re: Monero inception - how
  did bitmonero become monero?}, /08/2016 16:21~:
  \url{https://monero.stackexchange.com/questions/1011/monero-inception-how-did-bitmonero-become-monero/1024\#1024}.}.
Monero a été créé sous le nom de Bitmonero en avril 2014 par un
développeur utilisant le pseudonyme thankful\_for\_today, dans le but de
relancer le projet Bytecoin qui avait fait l'objet d'un préminage
massif. Cependant, il s'est rapidement avéré que thankful\_for\_today,
«~dictateur bienveillant~» autoproclamé, procédait à des changements
sans consulter les autres personnes impliquées et il s'est donc vu être
évincé du projet après quelques jours. Une équipe de six développeurs a
alors décidé de forker le projet et de le renommer en Monero.

Le second exemple d'une dissidence réussie est l'opposition à Bitcoin
ABC en 2020 dans le cadre du protocole Bitcoin Cash. Bitcoin ABC,
l'implémentation de référence de Bitcoin Cash depuis 2017, avait pour
développeur en chef, Amaury Séchet. En 2020, ce dernier a approuvé la
suggestion des mineurs de procéder à un soft fork pour rediriger une
partie de la récompense de bloc vers les équipes de
développement\footnote{«~la suggestion des mineurs de procéder à un soft
  fork pour rediriger une partie de la récompense de bloc vers les
  équipes de développement~»~: Jiang Zhuoer, \emph{Infrastructure
  Funding Plan for Bitcoin Cash}, 22 janvier 2020~:
  \url{https://medium.com/@jiangzhuoer/infrastructure-funding-plan-for-bitcoin-cash-131fdcd2412e}~;
  archive~:
  \url{https://web.archive.org/web/20200123082358/https://medium.com/@jiangzhuoer/infrastructure-funding-plan-for-bitcoin-cash-131fdcd2412e}.}
et a fini en novembre par tenter d'imposer ce changement via une
intégration dans Bitcoin ABC\footnote{«~a fini en novembre par tenter
  d'imposer ce changement via une intégration dans Bitcoin ABC~»~:
  Amaury Séchet, \emph{Bitcoin ABC's plan for the November 2020
  upgrade}, 6 août 2020~:
  \url{https://amaurysechet.medium.com/bitcoin-abcs-plan-for-the-november-2020-upgrade-65fb84c4348f}.}.
Une implémentation alternative, Bitcoin Cash Node, a alors été créée
pour faire face à ce changement\footnote{«~Une implémentation
  alternative, Bitcoin Cash Node, a alors été créée pour faire face à ce
  changement~»~: Notamment grâce aux deux développeurs anonymes
  freetrader et imaginary\_username. -- freetrader, \emph{Bitcoin Cash
  Node}, 20 février 2020~:
  \url{https://read.cash/@freetrader/bitcoin-cash-node-662e4737}.}, et a
recueilli une large majorité économique, devenant ainsi l'implémentation
de référence de ce qu'on appelle toujours aujourd'hui Bitcoin Cash.
L'application de la redirection de la subvention du protocole a mené à
la création du protocole XEC.

Ainsi, l'influence des développeurs sur le protocole est réelle, mais
elle est profondément limitée par l'intervention de l'économie si elle a
lieu.

\section*{La pression des mineurs}\label{la-pression-des-mineurs}
\addcontentsline{toc}{section}{La pression des mineurs}

\markright{La pression des mineurs}

La deuxième catégorie d'acteurs impliqués dans l'influence sur le
protocole est constituée des mineurs. Les mineurs sont les personnes ou
les groupes de personnes qui s'occupent de la confirmation des
transactions grâce à la dépense énergétique liée à la preuve de travail.
Comme montré dans le chapitre~\hyperref[ch:censure]{9}, ils disposent
d'un pouvoir de sélection sur les transactions, leur conférant par là,
en cas de regroupement majoritaire, la possibilité de procéder à une
double dépense ou d'appliquer une censure active.

Contrairement à ce qu'on peut parfois s'imaginer, les mineurs n'ont de
pouvoir direct sur le protocole que dans le sens où ils forment une
catégorie particulière de commerçants. Ils interviennent dans l'économie
en acceptant de confirmer des transactions en échange de frais. Mais ce
pouvoir direct est extrêmement limité du fait de la petitesse de leur
activité économique par rapport à l'activité totale.

Il n'en reste pas moins que les mineurs possèdent une influence non
négligeable dans la prise de décision, qui procède de leur pouvoir
d'attaque sur le consensus. D'une part, les mineurs peuvent influer dans
le choix de l'économie lors d'une scission en attaquant la branche
concurrente dans le but de la discréditer. C'est ce qu'ont menacé de
faire les mineurs pro-BSV en novembre 2018 suite à la séparation avec
BCH\footnote{Une «~guerre du hachage~» s'est déroulée entre les mineurs
  de Bitcoin SV, soutenus par Craig Wright et Calvin Ayre, et ceux de
  Bitcoin ABC, soutenus par Roger Ver et Jihan Wu, notamment par la
  redirection de la puissance de calcul de leurs coopératives de minage
  respectives. -- Aaron van Wirdum, \emph{Week 2: How the Bitcoin Cash
  "Hash War" Came and Went and Not Much Happened}, 30 novembre 2018~:
  \url{https://bitcoinmagazine.com/technical/week-2-how-bitcoin-cash-hash-war-came-and-went-and-not-much-happened}.}.
C'est également ce qu'a fait le mineur pro-BCHN face à Bitcoin ABC en
novembre 2020 en censurant la chaîne de Bitcoin ABC\footnote{«~C'est
  également ce qu'a fait le mineur pro-BCHN face à Bitcoin ABC en
  novembre 2020 en censurant la chaîne~»~:
  \url{https://decrypt.co/49819/bitcoin-cash-rebels-launch-51-attack-to-destroy-bch-hard-fork}.}.

D'autre part, les mineurs peuvent influencer le choix de l'économie en
imposant un soft fork qui, dans son application, est indiscernable de la
censure. L'ensemble des règles de consensus initial reste le même, mais
ne peut plus s'exprimer pleinement, à tel point que cela peut induire
les commerçants à adopter le soft fork en arrêtant d'accepter les
transactions et les blocs qui ne s'y conforment pas. C'est ce que le
développeur Peter Todd a décrit comme un «~soft fork forcé\footnote{Peter
  Todd, \emph{Forced Soft Forks}, 18 janvier 2016~:
  \url{https://petertodd.org/2016/forced-soft-forks}.}~» ou que d'autres
appellent un «~fork maléfique\footnote{«~fork maléfique~»~:
  \url{https://www.reddit.com/r/Bitcoin/comments/3yrsxt/bitcoindev_an_implementation_of_bip102_as_a/cyg4m39/}.}~»
(\emph{evil fork}). La situation peut être résolue de deux manières~: ou
bien les commerçants continuent d'appliquer les anciennes règles et
créent par là un différentiel de frais encourageant les mineurs à
revenir à la normale~; ou bien ils conviennent d'adopter un hard fork
annulant ce soft fork, prenant alors le risque de la spirale de
scissions liée à l'intervention humaine rapide.

Toutefois, cette influence des mineurs s'arrête là. Les commerçants
continuent de déterminer les règles et les mineurs sont impuissants face
à cette réalité. Il est donc faux de prétendre que les mineurs sont en
charge du protocole (gouvernance par preuve de travail), comme le
faisaient une bonne partie des \emph{big blockers} durant la guerre des
blocs\footnote{C'était, par exemple, la conception du PDG de Coinbase,
  Brian Armstrong, qui écrivait le 3 janvier 2016~: «~Heureusement,
  Bitcoin dispose d'un mécanisme de mise à niveau intégré et élégant. Si
  la majorité des mineurs de Bitcoin ``votent'' pour une mise à niveau
  particulière, il s'agit par définition de la nouvelle version de
  Bitcoin. Le nombre de votes obtenus par chaque mineur est
  proportionnel à la quantité de puissance de calcul qu'il apporte au
  réseau (les votes ne peuvent donc pas être truqués).~» -- Brian
  Armstrong, \emph{Scaling Bitcoin: The Great Block Size Debate}, 3
  janvier 2016~:
  \url{https://www.coinbase.com/blog/scaling-bitcoin-the-great-block-size-debate}.}.
En effet, si c'était réellement le cas, alors le système économique de
Bitcoin serait voué à l'échec, les mineurs étant naturellement incités à
augmenter leurs revenus par l'inflation, à l'instar des banques
centrales.

\section*{L'importance des
utilisateurs}\label{limportance-des-utilisateurs}
\addcontentsline{toc}{section}{L'importance des utilisateurs}

\markright{L'importance des utilisateurs}

La troisième catégorie d'acteurs internes ayant une influence sur le
protocole est la catégorie des utilisateurs non commerçants. Les
utilisateurs sont souvent mis en avant comme les personnes ayant le
dernier mot sur le protocole\footnote{«~Le réseau Bitcoin n'appartient à
  personne, tout comme la technique derrière le courriel n'appartient à
  personne. Bitcoin est contrôlé par l'ensemble de ses utilisateurs
  autour du monde. Alors que les développeurs améliorent les logiciels,
  ils ne peuvent pas imposer de modification dans le protocole Bitcoin
  parce que chaque utilisateur est libre de choisir quel logiciel et
  quelle version il utilise. Afin de rester compatibles avec les autres,
  tous les utilisateurs doivent utiliser des logiciels se conformant aux
  mêmes règles. Bitcoin ne peut fonctionner correctement qu'avec un
  consensus total entre ses utilisateurs.~» -- Bitcoin.org FAQ~:
  \url{https://bitcoin.org/fr/faq\#qui-controle-le-reseau-bitcoin}.}.
Toutefois, le terme d'utilisateur est ambigu et peut prêter à confusion,
car l'utilisation du bitcoin englobe généralement trois actions
distinctes~: l'acceptation dans le commerce, la détention durant une
période donnée et la dépense auprès d'autres personnes. De là, on peut
dégager trois sous-catégories théoriques d'utilisateurs~: les
commerçants, les clients et les détenteurs. La première possède le
pouvoir effectif sur le protocole, tandis que les deux autres n'exercent
qu'une simple influence.

Parlons d'abord des clients, qui sont les personnes qui échangent leurs
bitcoins contre des biens et services dans le commerce, y compris
d'autres monnaies. Ils sont le pendant des commerçants, l'échange étant
par définition symétrique~: sans client (acheteur), il n'y a pas de
commerçant (vendeur), et vice versa. Il y a donc une interdépendance
entre les commerçants et les clients.

Dans la détermination du protocole, les clients exercent par conséquent
une très grande influence. Un commerçant, s'il veut continuer à
prospérer dans les affaires, devra choisir d'accepter (au moins) la
monnaie liée au protocole soutenu majoritairement par ses clients.
L'histoire du refus de SegWit2X en 2017 est l'exemple parfait de
l'influence des clients, où les utilisateurs ont réussi à influencer les
plus gros commerçants (les plateformes de change) et à les pousser à
renoncer au doublement de la taille limite des blocs en
novembre\footnote{Satoshi Nakamoto, \emph{Re: BitDNS and Generalizing
  Bitcoin}, /12/2010 17:29:28 UTC,
  \url{https://bitcointalk.org/index.php?topic=1790.msg28917\#msg28917}~:
  «~Les utilisateurs de Bitcoin pourraient devenir de plus en plus
  sectaires à propos de la limitation de la taille de la chaîne pour que
  son accès reste facile pour beaucoup d'utilisateurs et pour les petits
  appareils.~»}.

Toutefois, l'idée que ces clients partagent la maîtrise du protocole
avec les commerçants est erronée. Si la dissension est équilibrée parmi
les utilisateurs, alors c'est le commerçant qui tranche en optant pour
un protocole plutôt que l'autre pour offrir ses biens et services à des
prix acceptables. Au bout du compte, le client (qui se débarrasse de ses
bitcoins) n'apporte aucune utilité à la monnaie~; le commerçant, si.

Considérons ensuite les détenteurs, c'est-à-dire les personnes qui
conservent des bitcoins en réserve durant une période significative. Ces
détenteurs sont parfois appelés thésauriseurs\footnote{«~appelés
  thésauriseurs~»~: Daniel Krawisz, \emph{I'm Hoarding Bitcoins, and No
  You Can't Have Any}, 12 février 2014~:
  \url{https://nakamotoinstitute.org/mempool/im-hoarding-bitcoins-and-no-you-cant-have-any/}.}
ou HODLers (par déformation du verbe \emph{to hold}, «~garder~»,
«~conserver~»\footnote{GameKyuubi, \emph{I AM HODLING}, /12/2013
  10:03:03 UTC~:
  \url{https://bitcointalk.org/index.php?topic=375643.msg4022997\#msg4022997}.})
pour insister sur le fait qu'ils ne veulent pas se séparer de leurs
bitcoins de sitôt. Par cette action, ils restreignent l'offre de monnaie
à proprement parler ce qui, conjugué à une demande plus forte, a un
effet haussier sur le pouvoir d'achat de l'unité et sur son taux de
change avec le dollar, communément appelé «~le prix~».

Les détenteurs ont une influence sur les commerçants. Premièrement, par
leur épargne, ils augmentent la taille de la subvention du protocole, et
donc le budget du minage pour la protection contre la double dépense,
assurant une plus grande sécurité aux commerçants. Deuxièmement, la
détention offre au marché plus de liquidité potentielle, ce qui permet à
des utilisateurs plus importants d'entrer. Troisièmement, un prix plus
haut a un effet de communication non négligeable, notamment par
l'attention qu'un engouement spéculatif entraîne dans les médias. Ainsi,
si une scission a lieu, les détenteurs peuvent vendre la monnaie d'une
branche contre celle de l'autre et créer un différentiel favorable au
protocole privilégié (c'est ce qui s'est passé durant la scission entre
BTC et BCH).

La conception selon laquelle le pouvoir d'achat de la monnaie serait
primordial a poussé certains protocoles cryptoéconomiques comme Dash et
Tezos à innover en créant des systèmes de gouvernance internes
permettant de résoudre les disputes à propos de la modification du
protocole par un vote proportionnel à la possession de jetons
(gouvernance par preuve d'enjeu). Les détenteurs seraient assimilés aux
parties prenantes d'une société, possédant des parts dans cette
dernière, qui serait essentiellement une organisation autonome
décentralisée (DAO).

Toutefois, cette conception ne tient que dans la phase précoce de
Bitcoin, où la création monétaire forme l'essentiel du revenu minier, où
l'activité est encore hautement spéculative (l'achat et la vente de
monnaie fiat dans le but d'en tirer un profit) et où les principaux
commerçants sont les plateformes de change et leurs utilisateurs. À long
terme, la diminution de la subvention minière et la stabilisation réduit
cet effet et donne un rôle beaucoup plus important aux transactions non
spéculatives, car s'il existe une relation entre l'utilité et le prix de
l'unité, c'est la première qui prime sur le second.

Ainsi, l'influence générale des acteurs internes sur les commerçants --
développeurs, mineurs, clients et détenteurs -- est non négligeable et
joue un rôle dans la détermination du protocole. Mais ce n'est pas la
seule pression qui s'exerce, et il faut aussi compter les acteurs
externes au système, qui participent aussi à leur niveau dans le
mécanisme de gouvernance.

\section*{Le poids des relais
d'opinion}\label{le-poids-des-relais-dopinion}
\addcontentsline{toc}{section}{Le poids des relais d'opinion}

\markright{Le poids des relais d'opinion}

La première catégorie des influences extérieures est celle des relais
d'opinion, qui orientent l'avis des personnes actives dans Bitcoin. Ces
relais peuvent être individuels (influenceurs) ou collectifs (médias).
La raison de leur existence est qu'il est impossible de saisir par
soi-même toutes les subtilités de Bitcoin, de sorte que la plupart des
utilisateurs se contentent souvent d'une explication rudimentaire
proposée par autrui, et font reposer un partie de leur jugement sur la
confiance accordée à autrui.

Cette situation conduit à l'émergence d'acteurs plus influents que les
autres, par leur prestige individuel ou par les médias qu'ils dirigent.
On dit parfois que Bitcoin n'a pas de chef, de meneur, qu'il est
acéphale\footnote{Le terme de «~monnaie acéphale~» a été popularisé par
  Jacques Favier et Adli Takkal Bataille dans \emph{Bitcoin, la monnaie
  acéphale} en 2017.}. Cependant, force est de constater que ce n'est
pas le cas \emph{stricto sensu} et que certaines personnes ont un poids
plus important que d'autres dans la prise de décision, indépendamment de
leur activité économique.

D'abord, les experts techniques, qui sont censés mieux connaître les
méandres du protocole que les autres, rentrent dans cette catégorie. Ils
peuvent être développeurs eux-mêmes, avoir une activité proche, ou bien
être éducateurs ou rédacteurs. Nous pouvons par exemple citer Adam Back,
ancien cypherpunk et PDG de Blockstream, Andreas Antonopoulos, éducateur
de longue date, ou encore Aaron van Wirdum, rédacteur expérimenté pour
le Bitcoin Magazine et co-animateur du podcast Bitcoin Explained.

Ensuite, viennent les acteurs impliqués politiquement qui saisissent les
intérêts profonds de Bitcoin. On peut mentionner ici l'activiste Alex
Gladstein, directeur de la stratégie à la Human Rights Foundation. Puis
viennent les économistes, qui comprennent mieux que les autres les
mécanismes économiques à l'œuvre dans Bitcoin, comme l'économiste et
auteur Saifedean Ammous, l'experte en macroéconomie Lyn Alden, ou encore
le magistrat financier Yorick de Mombynes en France.

Enfin, nous avons les financiers, qui ont fait fortune avant de
découvrir Bitcoin ou bien grâce à lui. Ces personnes sont considérées
comme des modèles du fait de leur réussite financière, qui est
l'objectif principal de la majorité des gens qui s'intéressent à Bitcoin
en premier lieu. Roger Ver, les frères Winklevoss et Michael Saylor en
font partie. On peut aussi citer le milliardaire Elon Musk, qui est
l'archétype de ce type d'influence, et qui a notamment donné une seconde
vie à Dogecoin en le citant à de multiples reprises dans ses
interventions publiques.

Toutes ces personnalités sont souvent relayées par les médias, qui
jouent eux-même un rôle de relai d'opinion. Ces derniers exercent en
effet une certaine influence en choisissant quels contenus sont publiés
ou diffusés et lesquels ne le sont pas. Ils permettent au grand public
qui n'a pas le temps ou l'envie de lire sur le sujet de se forger un
avis.

Il peut s'agir des vidéastes individuels qui produisent du contenu sur
les cryptomonnaies, notamment sur la plateforme Youtube. Il y a aussi
les autres médias spécialisés comme les sites d'information
(Bitcoin.org, Bitcoin.fr), les médias d'actualité (Bitcoin Magazine,
Cointelegraph, Coindesk, Bitcoin.com à l'international~; Cryptoast et le
Journal du Coin en France), les chaînes vidéo (Grand Angle Crypto), les
podcasts, les lettres d'information payantes (The Big Whale). On peut
également mentionner les plateformes de discussion spécialisées, dont le
forum de discussion historique bitcointalk.org, les subreddits consacrés
à Bitcoin (r/bitcoin, r/btc), et aujourd'hui les groupes Telegram
dédiés.

Les médias généralistes exercent aussi une influence, bien qu'elle soit
encore plus diffuse. C'est par exemple le cas des chaînes d'informations
financières (CNBC, BFM Business) qui consacrent parfois des émissions au
sujet des crypto-actifs. On peut aussi citer tous les médias sociaux qui
peuvent modeler l'opinion publique à propos de Bitcoin, comme c'est le
cas de Twitter, lieu privilégié pour la communication sur
Bitcoin\footnote{«~Twitter, lieu privilégié pour la communication sur
  Bitcoin~»~: Cette dépendance à Twitter a poussé les bitcoineurs à
  développer leur propre protocole de média social décentralisé~: Nostr.}.

\section*{La puissance suggestive de la
finance}\label{la-puissance-suggestive-de-la-finance}
\addcontentsline{toc}{section}{La puissance suggestive de la finance}

\markright{La puissance suggestive de la finance}

Le discours n'est cependant pas la seule manière d'influencer les
acteurs du système~: il existe également le «~pouvoir~» de l'argent. Les
puissances financières jouent un rôle dans la détermination du protocole
en choissisant de financer l'écosystème et les influenceurs de la
variante de Bitcoin qui leur plaît. Elles peuvent par exemple fournir
des fonds pour la croissance commerciale (listage sur plateforme de
change), le développement logiciel, la création d'applications
innovantes, le marketing, le lobbying auprès des instances régulatrices,
etc.

Le financement de l'implémentation de référence est particulièrement
crucial. L'infrastructure logicielle n'est pas maintenue gratuitement,
mais elle n'apporte aucun revenu, du fait de l'absence nécessaire de
contrainte légale sur son utilisation. C'est pourquoi les développeurs
doivent trouver de l'argent quelque part\footnote{Divers modèles de
  financement ont été proposés~: celui de la Fondation Bitcoin entre
  2012 et 2014, celui du capital-risque avec le financement de
  Blockstream à partir de 2014, celui du financement participatif avec
  Lighthouse (BTC) en 2014 et Flipstarter (BCH) en 2020, et enfin celui
  de l'utilisation de la subvention de minage (Dash, Zcash, XEC) depuis
  2015.}. C'est ainsi que diverses organisations financent le
développement~: en 2023, le salaire versé aux personnes chargées de
l'écriture et de la révision du code dans Bitcoin Core provient
principalement (par ordre d'importance) de l'organisation caritative
Brink (elle-même financée par les principales plateformes de change), la
\emph{Digital Currency Initiative} du MIT Media Lab, le groupe de
développement et de recherche Chaincode Labs, l'entreprise Block de Jack
Dorsey et la plateforme de trading sur marge BitMEX\footnote{«~le
  salaire versé aux personnes chargées de l'écriture et de la révision
  du code dans Bitcoin Core provient principalement~»~:
  \url{https://blog.bitmex.com/wp-content/uploads/2022/10/Bitcoin-Grant-Presentation-1.pdf}}.

Cela donne aux puissances financières une influence particulière, chose
qui a été dénoncée au sujet de Blockstream depuis ses débuts,
l'entreprise ayant notamment reçu un investissement de la part d'AXA. On
peut aussi citer le cas de la \emph{Digital Currency Initiative} dont le
rôle est plus qu'ambigu. Cette entité a en effet été l'organisation en
charge du développement du prototype de monnaie numérique de banque
centrale des États-Unis tout en continuant de payer le mainteneur
principal de Bitcoin Core, Wladimir van der Laan.

\section*{La guerre de l'État contre le
protocole}\label{la-guerre-de-luxe9tat-contre-le-protocole}
\addcontentsline{toc}{section}{La guerre de l'État contre le protocole}

\markright{La guerre de l'État contre le protocole}

Pour finir, la troisième et dernière méthode utilisée pour influencer
les acteurs du système et donc le protocole, c'est la force, ou plus
précisément la menace d'utiliser la force, une spécialité largement
monopolisée par une grande institution appelée l'État.

L'existence de l'État est profondément liée au contrôle sur la monnaie,
qui facilite grandement la collecte de son revenu. En particulier, il
prélève un seigneuriage grâce à la domination qu'il exerce sur la
détermination du support monétaire. De ce fait, il existe un rapport
antagoniste entre l'État et Bitcoin, ce dernier redonnant aux individus
la maîtrise totale de leur monnaie.

Il est donc logique que l'État cherche à influencer l'évolution du
protocole, voire qu'il finisse par tenter de le décréter. Par la
définition du cadre légal, il peut en effet influer sur le choix des
commerçants. Son pouvoir politique n'est cependant pas illimité. S'il
s'y prend mal ou si le changement est trop brutal, ces commerçants
risquent de désobéir en masse et de rejoindre le marché noir, où aucune
autorisation n'est requise.

C'est le pouvoir économique qui détermine le protocole en dernier lieu.
Mais avec le temps l'État peut s'immiscer dans ses décisions pour
altérer doucement les propriétés de Bitcoin. Par des lois intelligentes,
il peut faire en sorte que son action reste largement acceptée et qu'une
bonne partie de l'économie continue d'avoir lieu sur le marché
réglementé. Il peut aussi influencer les différents acteurs qui jouent
un rôle dans le modèle de gouvernance de Bitcoin, comme les
développeurs, les mineurs ou les médias, sans que ceux-ci ne réagissent.

Les réglementations financières constituent des étapes préparatoires
dans le déploiement d'une telle influence. Il s'agit des contraintes
imposées aux plateformes de change (commerçants principaux) qui se
chargent d'effectuer la jonction entre le bitcoin et les monnaies
officielles. Initiées en 2013, ces réglementations imposent aujourd'hui
une procédure de connaissance du client (KYC) et de connaissance des
transactions (KYT) assez drastique, de sorte que l'anonymat dans ce type
d'échange devient de plus en plus difficile à préserver. Elles ont le
double avantage d'habituer les acteurs économiques à se conformer et de
restreindre leur nombre en requérant des contraintes de plus en plus
insurmontables pour les plus petites plateformes. Ces réglementations
peuvent également s'appliquer aux commerçants en général. De plus, la
loi les contraint déjà dans la plupart des juridictions à déclarer leurs
plus-values par rapport à la monnaie nationale, ce qui complique leur
activité.

Voyons maintenant comment le protocole pourrait être attaqué. Tout
d'abord, l'acceptation du bitcoin pourrait être rendue illégale, sans
alternative. Toute l'économie portée par les commerçants du marché
réglementé serait détruite d'un simple trait de plume. L'utilité du
système serait alors grandement réduite sur le moment, ainsi que la
valeur d'échange de l'unité de compte.

Ce type d'interdiction totale a déjà eu lieu dans certains pays, comme
le Maroc, l'Algérie, la Bolivie ou le Népal. D'autres pays ont choisi
d'interdire uniquement une section de l'économie, que ce soit le change
avec la monnaie nationale (Chine), la vente de biens et services sur le
territoire (Turquie, Équateur, Thaïlande) ou l'acquisition par des
acteurs financiers (Iran, Nigéria). Toutefois, sauf dans le cas de la
Chine, ces interdictions n'ont pas été réalisées par des puissances
majeures, de sorte que l'utilisation du bitcoin reste légale sur la
majorité de la planète. Une réelle interdiction, si elle avait lieu,
aurait besoin de se faire de manière internationale pour avoir une
influence réelle et amoindrir l'utilité de Bitcoin.

L'impact de cette interdiction pure et simple est difficile à mesurer.
En effet, on peut douter de la capacité d'application de ces lois. Une
interdiction sans soutien populaire aurait pour conséquence d'accroître
la taille du marché noir. De ce fait, il semble évident qu'une telle
attaque s'accompagnerait de la proposition d'une alternative contrôlée
de Bitcoin ayant pour but de contenter la partie la plus «~pragmatique~»
de l'économie.

Dans le but de s'opposer à Bitcoin, l'État pourrait ainsi déployer sa
propre version du protocole dans le but de ronger progressivement les
propriétés fondamentales de Bitcoin. Cette version altérée de Bitcoin
serait rendue légale et bénéficierait d'un régime accommodant, tandis
que la version originelle serait rendue illégale. Les acteurs
conformistes seraient récompensés à court terme par une augmentation du
prix, alors que les commerçants dissidents seraient punis par des
amendes et des peines de prison.

En premier lieu, des soft forks de censure pourraient être appliqués en
se basant sur les normes générales de LCB-FT. Ceux-ci pourraient
s'accompagner d'une attaque de censure active par les mineurs. Les
acteurs conformistes pourraient justifier leur choix en disant que ces
transactions n'ont rien à faire sur la chaîne de Bitcoin.

En deuxième lieu, un soft fork pourrait aller jusqu'à concerner toutes
les transactions, en requérant une autorisation étatique pour chacune
d'entre elles. À ce stade, les plus conformistes pourraient toujours
s'imaginer que la politique monétaire aurait été préservée.

En troisième lieu, un soft fork taxatoire pourrait être mis en place.
Celui-ci consisterait à prélever une taxe fixe sur toutes les
transactions, dans l'idée de la TVA, ou bien à extraire un demeurage, en
soustrayant un montant dépendant du temps de détention des fonds
dépensés. Ces impôts pourraient être réalisés dans l'objectif de réguler
la nature trop déflationniste du bitcoin et la répartition inique de la
richesse.

En quatrième lieu, un hard fork d'inflation pourrait être appliqué. À ce
stade, les acteurs restants n'auraient plus rien à voir avec les acteurs
originaux. Ce qui faisait la renommée de «~Bitcoin~» serait alors
totalement anéanti, et le système correspondant ressemblerait trait pour
trait à une monnaie numérique de banque centrale.

Ce scénario, bien qu'hypothétique, forme la conséquence logique de
l'influence de l'État sur la monnaie et est donc inévitable jusqu'à un
certain point. Toutefois, il suppose dans le même temps le développement
d'une économie parallèle dans laquelle l'acceptation du bitcoin se
ferait de manière clandestine. Face à la censure de plus en plus forte,
il se formerait ainsi une opposition, une résistance. Les altérations
progressives rendraient la chaîne officielle de moins en moins utile, en
faisant fuir les commerçants ne souhaitant pas se conformer.

À un certain moment, une scission aurait lieu. La version étatique
pourrait être minoritaire, auquel cas elle se séparerait naturellement
de l'autre chaîne (scénario optimiste). Mais elle pourrait aussi être
majoritaire, auquel cas la version clandestine serait contrainte de
procéder à un hard fork pour subsister (scénario pessimiste). Dans les
deux cas, la reconstruction de Bitcoin se produirait alors à partir de
la chaîne libre, au sujet de laquelle il n'y aurait aucune ambiguïté au
niveau du protocole. Cette chaîne pourrait cependant être attaquée du
point de vue minier, ce que nous avons décrit en détail dans le
chapitre~\hyperref[ch:censure]{9}.

\section*{Deux niveaux de
sécurité}\label{deux-niveaux-de-suxe9curituxe9}
\addcontentsline{toc}{section}{Deux niveaux de sécurité}

\markright{Deux niveaux de sécurité}

Bitcoin est un concept de monnaie numérique résistante à la censure et à
l'inflation. Ces deux propriétés fondamentales sont complémentaires,
mais elles nécessitent des sécurités différentes. La résistance à la
censure repose sur la sécurité minière~; la résistance à l'inflation sur
la sécurité commerciale.

La détermination du protocole -- ou plutôt des protocoles puisqu'il peut
y en avoir plusieurs -- est réalisée par les commerçants au sens large,
c'est-à-dire les personnes qui acceptent le bitcoin dans l'échange
contre des biens, des services ou d'autres monnaies. Les commerçants
vérifient les règles de consensus par l'intermédiaire de leurs nœuds. Ce
pouvoir sur le protocole est proportionnel à l'activité économique
potentielle du commerçant, qui peut être estimée par ses recettes
effectives. Il dépend aussi de l'effet de réseau qui fait que l'utilité
combinée apportée par les commerçants va évoluer de manière
superlinéaire.

Un certain nombre d'influences s'exercent sur les commerçants pour
qu'ils acceptent tel ou tel protocole. S'il est dur de comprendre
comment cet ensemble complexe interagit, il est possible d'en dessiner
les contours comme nous l'avons fait ici. En particulier, l'influence
majeure à ne pas négliger est celle de l'État, qui pourrait attaquer le
protocole en bonne et due forme par la coercition des commerçants et des
autres acteurs.

Pour que le protocole de Bitcoin soit réellement robuste, il faut donc
que l'activité économique soit décentralisée, à l'instar du minage~: que
les commerçants (ou petits groupes de commerçants) fassent fonctionner
leurs propres nœuds pour que, dans l'hypothèse d'un changement des
règles décrété par une autorité, les risques soient répartis dans
l'économie et que Bitcoin puisse continuer à survivre clandestinement.

Il est nécessaire que l'accord sur le protocole s'exerce à long terme.
La courte histoire de Bitcoin regorge de perturbations multiples qui
montrent que le changement des règles de consensus ne se passe pas
toujours dans les meilleures conditions. Seul le temps permet de faire
le tri entre les bonnes modifications et les mauvaises.

\bookmarksetup{startatroot}

\chapter{Les rouages de la machine}\label{ch:rouages}

\phantomsection\label{enotezch:12}{}

{B}\textsc{i}tcoin est une étrange machine. Né dans un rapport
antagoniste vis-à-vis de l'autorité, il possède des propriétés qui ne se
retrouvent pas dans les systèmes informatiques communs. En particulier,
il ne peut pas être modifié n'importe comment, ce qui explique sa
conception originelle et son évolution ultérieure.

D'une part, la représentation des unités de base, les satoshis, ne se
fait pas sous la forme de comptes où les soldes des utilisateurs
seraient mis à jour, mais par le biais de pièces de cryptomonnaies
pouvant être combinées et séparées dans les transactions. Ce
fonctionnement favorise la confidentialité et la scalabilité de la
chaîne, et s'adapte ainsi l'utilisation monétaire.

D'autre part, Bitcoin intègre un système de programmation interne
permettant d'intégrer des conditions de dépense dans les pièces, ce
qu'on appelle parfois des contrats autonomes ou \emph{smart contracts}.
Il a été amélioré au cours des années, parfois au prix d'une plus grande
complexité, notamment \emph{via} l'ajout de SegWit et de Taproot.

Dans ce chapitre, nous examinerons les rouages de cette machine
transactionnelle, avant de décrire comment elle peut être exploitée et
améliorée à des fins de confidentialité. Le prochain chapitre sera
consacré aux contrats en tant que tels.

\section*{Les transactions et les
pièces}\label{les-transactions-et-les-piuxe8ces}
\addcontentsline{toc}{section}{Les transactions et les pièces}

\markright{Les transactions et les pièces}

Dans Bitcoin, les transactions possèdent un rôle central. Le protocole
est fait pour échanger de la valeur conformément à son rôle monétaire,
donc de traiter les transferts de propriété. Tout le fonctionnement du
système a été pensé pour faciliter la construction, la signature et la
diffusion des transactions, leur conservation en mémoire dans la
\emph{mempool}, et leur ajout au registre par leur inclusion dans un
bloc.

Chaque transaction est constituée d'une ou plusieurs entrées et d'une ou
plusieurs sorties. Une sortie transactionnelle se compose simplement
d'une indication de destination et d'un montant en unités (satoshis).
Une entrée fait généralement référence à une sortie transactionnelle
précédente, sauf dans le cas de la transaction de récompense où elle
représente une «~base de pièce~» créant de nouvelles unités issues de
l'émission monétaire et des frais de transaction.

L'identifiant d'une transaction (\emph{transaction identifier} ou
\texttt{)\ est\ l\textquotesingle{}empreinte\ des\ données\ brutes\ qu\textquotesingle{}elle\ contient,\ obtenue\ *via*\ le\ hachage\ par\ double\ SHA-256.\ Chaque\ sortie\ transactionnelle\ est\ caractérisée\ par\ l\textquotesingle{}identifiant\ de\ la\ transaction\ dont\ elle\ est\ issue\ et\ par\ sa\ position\ dans\ cette\ transaction,\ qu\textquotesingle{}on\ appelle\ l\textquotesingle{}indice.\ Ce\ point\ de\ sortie\ (*outpoint*)\ sert\ d\textquotesingle{}indication\ de\ provenance.\ Un\ exemple\ de\ point\ de\ sortie\ est}.

Contrairement à ce que la description de la propriété dans le
chapitre~\hyperref[ch:propriete]{7} suggère, la destination et la
provenance des unités ne sont pas à proprement parler des adresses, mais
des scripts de verrouillage, c'est-à-dire des petits programmes qui
déterminent leurs conditions de dépense. Chaque sortie crée ainsi un
script qui bloque les fonds d'une façon spécifique. Le plus souvent, ce
script contient une clé publique ou une empreinte de clé publique, qui
peut être interprétée comme une adresse par le portefeuille.

Pour être valide, une entrée doit contenir un script de déverrouillage
dont l'exécution, combinée à celle du script de verrouillage, réussisse.
En général, ce script de déblocage des fonds contient une signature
numérique qui correspond à la clé publique liée au script de
verrouillage précédent~: la vérification de la signature permet de
s'assurer que la personne qui dépense les unités en est le propriétaire.

Ce fonctionnement fait que le modèle de représentation des unités est
contre-intuitif. Le protocole ne voit pas de comptes dont les soldes
seraient actualisés par les transactions, comme c'est le cas dans
Ethereum par exemple. Il voit simplement des sorties transactionnelles
détenues par des propriétaires, de manière similaire aux pièces de
monnaies dans le monde physique.

Ainsi, Bitcoin met en œuvre le concept de pièce de monnaie numérique qui
était discuté au sein de la communauté cypherpunk dans les années 1990.
Dans le \emph{Cyphernomicon} par exemple, Tim May estimait que la chose
était impossible, en raison du problème de la double dépense\footnote{«~Tim
  May estimait que la chose était impossible~»~: Timothy C. May,
  \emph{Cyphernomicon}, 12.3.8.}. Satoshi Nakamoto, en découvrant une
manière de résoudre ce problème, a pu rendre le concept viable et l'a
intégré dans Bitcoin. Dans le livre blanc, il décrivait la notion de
pièce numérique comme suit~:

«~Nous définissons une pièce de monnaie électronique comme une chaîne de
signatures numériques. Chaque propriétaire transfère la pièce au suivant
en signant numériquement l'empreinte de la transaction précédente et la
clé publique du propriétaire suivant, et en les ajoutant à la fin de la
pièce. Un bénéficiaire peut vérifier les signatures pour vérifier la
chaîne de propriété\footnote{Satoshi Nakamoto, \emph{Bitcoin: A
  Peer-to-Peer Electronic Cash System}, 31 octobre 2008.}.~»

Dans Bitcoin, les pièces existantes sont donc les sorties
transactionnelles non dépensées, nommées usuellement UTXO par
abréviation de l'anglais \emph{Unspent Transaction Outputs}, à savoir
les sorties transactionnelles qui n'ont pas été utilisées comme entrée
dans une autre transaction. L'ensemble de ces pièces, l'\emph{UTXO set},
constitue le registre de propriété. C'est l'état du système, qui peut
être récupéré à partir de son historique, la chaîne de blocs.

Chaque pièce est constituée d'un montant en unités (satoshis) et d'un
script de verrouillage. Il peut ainsi exister des pièces d'un milliard
de satoshis (10 bitcoins) tout comme on peut avoir des pièces de
546~satoshis (0,00000546 bitcoin).

Le script de verrouillage d'une pièce contient le plus souvent une clé
publique ou une empreinte déterminée, de sorte que la pièce peut être
vue comme étant détenue par l'adresse correspondante. De ce fait, deux
pièces partageant le même script de verrouillage sont détenues par la
même adresse. Un compte dans Bitcoin correspond à l'ensemble des
adresses contrôlées par un utilisateur. Le solde est récupéré en
balayant l'ensemble des UTXO de façon à retrouver les pièces détenues
par ces adresses.

\begin{figure}

{\centering \includegraphics{chapters/img/coins-utxos-account.png}

}

\caption{Exemples de pièces détenues par un même compte.}

\end{figure}%

Ce modèle de représentation par des pièces fait qu'on peut voir le
mécanisme de transaction comme une fonderie de pièces de monnaie. Chaque
transaction consiste à fondre ensemble une ou plusieurs pièces de
bitcoin en entrée et à frapper une ou plusieurs pièces en sortie. C'est
en ceci que le serveur d'horodatage distribué de Bitcoin vient remplacer
la monnaierie numérique centralisée permettant le remplacement
systématique des pièces, qui est présente dans eCash et RPOW par
exemple.

La construction d'une transaction implique de rassembler des pièces de
valeur suffisante en entrée pour les fondre et en frapper de nouvelles.
En général, deux pièces sont créées~: la première est créée sur
l'adresse fournie par le destinataire pour effectuer le paiement (sortie
principale) et la seconde est créée sur l'une des adresses de
l'expéditeur afin qu'il se «~rende la monnaie~» (sortie complémentaire).
La différence entre le montant en entrée et le montant en sortie est
prise en compte dans la récompense du mineur en tant que frais de
transaction.

Considérons quelques exemples en ignorant ces frais et supposons
qu'Alice veuille procéder à un paiement. Si Alice possède une pièce de
12 mBTC (0,012 BTC) et veut donner 7 mBTC à Bob, alors elle doit
construire et signer une transaction ayant pour entrée cette pièce de 12
mBTC et pour sorties une pièce de 7 mBTC vers l'adresse de Bob et une
pièce restante de 5 mBTC vers sa propre adresse. Cette transaction est
représentée par la figure~\hyperref[fig:transaction-1i-2o]{12.2}.

\begin{figure}

{\centering \includegraphics{chapters/img/transaction-1i-2o.png}

}

\caption{Schéma d'une transaction à 1 entrée et 2 sorties.}

\end{figure}%

Si Alice ne possède pas une pièce ayant une valeur faciale supérieure à
7 mBTC, alors elle doit regrouper des pièces pour réunir un montant
suffisant en entrée, par exemple une pièce de 6 mBTC et une pièce de 2
mBTC. Comme précédemment, elle doit créer une sortie complémentaire vers
elle-même dans le but de se rendre la monnaie. Dans ce cas, illustré sur
la figure~\hyperref[fig:transaction-2i-2o]{12.3}, on peut deviner en
observant la transaction que la pièce de 7 mBTC est le résultat du
paiement, car il serait économiquement irrationnel de fusionner
plusieurs pièces pour envoyer 1 mBTC.

\begin{figure}

{\centering \includegraphics{chapters/img/transaction-2i-2o.png}

}

\caption{Schéma d'une transaction à 2 entrées et 2 sorties.}

\end{figure}%

Si Alice désire transférer l'intégralité des fonds vers un autre compte,
alors elle rassemble l'ensemble de ses pièces (6 mBTC, 4 mBTC, 2 mBTC)
pour les envoyer vers une adresse unique, comme montré sur la
figure~\hyperref[fig:transaction-3i-1o]{12.4}. C'est ce qu'on appelle
une consolidation de portefeuille, qui peut être identifiée par un
observateur extérieur en raison de l'unicité de la sortie.

\begin{figure}

{\centering \includegraphics{chapters/img/transaction-3i-1o.png}

}

\caption{Schéma d'une transaction à 3 entrées et 1 sortie.}

\end{figure}%

Nous voyons ainsi que les transactions ne sont pas des transferts bruts
d'une adresse vers une autre, mais des combinaisons-séparations de
pièces de monnaies numériques. Ce fonctionnement est quelque peu
contre-intuitif, mais se révèle utile pour la scalabilité du système, en
permettant le traitement indépendant des pièces, et pour la
confidentialité des utilisateurs, en n'incitant pas au rassemblement sur
une même adresse et en facilitant l'implémentation de techniques
d'anonymisation comme le mélange des pièces. Ce modèle est donc
particulièrement adapté à l'utilisation monétaire\footnote{«~Ce modèle
  est donc particulièrement adapté à l'utilisation monétaire~»~: Ludovic
  Lars, \emph{Pièces et comptes~: deux modèles de représentation}, 20
  juillet 2019~:
  \url{https://viresinnumeris.fr/pieces-comptes-modeles-representation/}.}.

\section*{La machine virtuelle}\label{la-machine-virtuelle}
\addcontentsline{toc}{section}{La machine virtuelle}

\markright{La machine virtuelle}

Les scripts présents au sein des transactions font de Bitcoin un système
de monnaie programmable. Ces scripts permettent en effet la mise en
place d'une variété de conditions de dépense, aussi appelées clauses,
qui vont au-delà de l'exigence d'une signature simple, comme la
connaissance d'un secret, l'attente d'une période de temps ou la
production de signatures multiples.

La mise en œuvre de Bitcoin crée une machine abstraite dont le
fonctionnement est répliqué sur tous les nœuds du réseau grâce à
l'algorithme de consensus. Elle est simulée par l'intermédiaire de
l'implémentation logicielle, de sorte qu'on parle de machine virtuelle.
Plus précisément, il s'agit d'une machine à états, dont l'état courant
est l'ensemble des pièces existantes, c'est-à-dire l'ensemble des
sorties transactionnelles non dépensées (UTXO), et dont les transitions
sont les transactions, qui détruisent des pièces pour en créer de
nouvelles. Ces transactions sont assemblées dans des blocs qui sont
validés à intervalles réguliers par les mineurs. La diffusion d'un bloc
sur le réseau permet d'actualiser l'état de la machine virtuelle, qui
est (sauf dans le cas d'un embranchement) partagé par tous les nœuds.

Au sein d'une transaction, le déverrouillage des pièces se fait par
l'exécution de scripts. Les scripts sont des prédicats au sens
mathématique, c'est-à-dire des expressions incomplètes qui deviennent
des propositions pouvant être évaluées si elles sont complétées par un
ou plusieurs éléments. De ce fait, la dépense consiste à réunir le
script de verrouillage de la sortie précédente et le script de
déverrouillage, et à les exécuter l'un après l'autre~: le script de
déverrouillage d'abord, le script de verrouillage ensuite. L'utilisation
de la pièce comme entrée de transaction n'est approuvée que si
l'exécution réussit.

Les scripts sont écrits dans le langage de programmation interne de
Bitcoin, conçu par Satoshi Nakamoto dès 2008 et baptisé de façon peu
originale «~Script~». Ce langage de programmation fonctionne de manière
similaire à Forth, un langage utilisé dans les années 1970 et 1980. Il
se base en particulier sur deux piles de données, qui sont des
structures de données fondées sur le principe du «~dernier arrivé,
premier sorti~» (\emph{last in, first out}, ou LIFO). Le langage agit
essentiellement sur la pile primaire, de sorte que celle-ci est la plus
importante~; la pile secondaire permet seulement de mettre des données
de côté pendant l'exécution d'un script.

Satoshi Nakamoto a inclus ce système de scripts dans Bitcoin pour lui
permettre de gérer une grande variété de cas d'utilisation. En juin
2010, en réponse à Gavin Andresen, il écrivait la chose suivante sur le
forum~:

«~La nature de Bitcoin est telle que, dès la version 0.1 lancée, son
fonctionnement de base était gravé dans le marbre pour le reste de son
existence. C'est pour cette raison que je voulais concevoir Bitcoin pour
qu'il supporte tous les types de transaction auxquels je pouvais penser.
Le problème était que chaque élément requérait un code de prise en
charge et des champs de données spéciaux, qu'il soit utilisé ou non, et
ne pouvait couvrir qu'un cas particulier à la fois. Ç'aurait été une
explosion de cas particuliers. La solution était script, qui
généralisait le problème de façon à ce que les parties contractantes
puissent décrire leurs transactions comme des prédicats que les nœuds du
réseau évaluaient. Les nœuds ont seulement besoin de comprendre la
transaction dans la mesure où ils évaluent si les conditions de
l'expéditeur sont remplies ou non\footnote{Satoshi Nakamoto, \emph{Re:
  Transactions and Scripts: DUP HASH160 ... EQUALVERIFY CHECKSIG},
  /06/2010 18:46:08~:
  \url{https://bitcointalk.org/index.php?topic=195.msg1611\#msg1611}.}.~»

Le langage est constitué de plus d'une centaine d'opérateurs, aussi
appelés codes opération (\emph{opcodes}), qui agissent sur la pile
primaire d'une manière ou d'une autre\footnote{«~Le langage est
  constitué de plus d'une centaine d'opérateurs~»~: La liste des
  opérateurs et de leurs actions est disponible sur la page de Bitcoin
  Wiki consacrée à Script~: \url{https://en.bitcoin.it/wiki/Script}.}.
Les opérateurs sont des nombres codés sur 1 octet (allant de 0 à 255),
mais sont usuellement désignés par un nom décrivant leur fonction, dans
le but de rendre la lecture plus compréhensible par l'être humain. Ils
sont notés en majuscules et sont souvent précédés du préfixe
\texttt{OP\_} même s'il peut être omis en l'absence d'ambiguïté. Par
exemple, l'opérateur permettant de vérifier une signature
(\texttt{0xac}) est noté \texttt{OP\_CHECKSIG} ou \texttt{CHECKSIG}.

Les opérateurs allant de 1 à 75, parfois notés
\texttt{,\ ont\ pour\ action\ d\textquotesingle{}empiler\ des\ données\ ayant\ une\ taille\ allant\ de\ 1\ à\ 75\ octets.\ L\textquotesingle{}utilisation\ d\textquotesingle{}opérateurs\ supplémentaires\ spécifiques\ (notés})
permet cependant de placer une information plus grande sur la pile. Bien
qu'on puisse utiliser cette notation, il est généralement plus simple de
placer un élément entre chevrons pour indiquer qu'il est placé au sommet
de la pile. Par exemple, le fait d'écrire `` au sein d'un script
signifie que la signature est empilée.

La valeur retournée à la fin de l'exécution des scripts est un booléen,
de sorte que le script peut être valide, auquel cas la dépense de la
pièce est approuvée, ou bien invalide, auquel cas la transaction est
rejetée dans son ensemble. Le script est valide si et seulement si la
valeur \texttt{TRUE} («~vrai~») est présente en haut de la pile à la fin
de l'exécution. Il est invalide si ce n'est pas le cas ou si son
exécution s'est arrêtée avant la fin.

Le langage Script est cependant limité. Rien dans sa conception de base
ne permet de faire de boucles, ni d'accéder à des données extérieures à
celles de la transaction, contrairement au langage d'Ethereum qui est
quasi Turing-complet. Cette particularité fait qu'il est moins flexible,
mais qu'il a l'avantage d'être plus simple à appréhender et donc plus
sûr.

L'exemple typique de script, présenté par Andreas
Antonopoulos\footnote{Andreas M. Antonopoulos, «~\emph{Transactions}~»,
  in \emph{Mastering Bitcoin: Programming the Open Blockchain}, 2
  édition, 2017, pp.~117-148.}, est celui qui consiste à résoudre une
équation simple impliquant une addition. Si on considère l'équation
\(17 + x = 38\), alors le script de verrouillage qui correspond est~:

\begin{verbatim}
<17> ADD <38> EQUAL
\end{verbatim}

Toute personne disposant de la réponse peut dépenser la pièce, ce qui on
en convient n'est pas très sécurisé. La dépense requiert ici de fournir
le script de déverrouillage composé uniquement de la solution de
l'équation, à savoir 21~:

\begin{verbatim}
<21>
\end{verbatim}

L'exécution successive de ces deux scripts (voir
figure~\hyperref[fig:bitcoin-stack]{12.5}) a lieu comme suit~: 1)~la
valeur 21 est placée sur la pile~; 2)~la valeur 17 est placée
au-dessus~; 3)~l'opérateur \texttt{OP\_ADD} additionne les deux valeurs
en haut de la pile et les remplace par leur somme, ici 38~; 4)~la valeur
38 est placée au sommet de la pile~; 5)~l'opérateur \texttt{OP\_EQUAL}
compare les deux valeurs en haut de la pile et les remplace par le
booléen d'égalité, ici \texttt{TRUE}. L'exécution du script est donc un
succès.

\begin{figure}

{\centering \includegraphics{chapters/img/bitcoin-stack-example.png}

}

\caption{Exécution d'un script d'addition sur la pile de données.}

\end{figure}%

Si la valeur avait été différente, de 22 par exemple, alors la dernière
opération aurait retourné le booléen \texttt{FALSE} («~faux~») et la
transaction de dépense aurait été invalidée.

Beaucoup de conditions de dépense différentes peuvent être implémentées
par ce système. Certaines de ces conditions sont simples comme la
connaissance d'un secret spécifique ou la production d'une signature
valide correspondant à une clé publique particulière. La connaissance
d'un secret (dont l'empreinte est spécifiée dans l'UTXO) est vérifiée
par les scripts suivants qui placent le secret au sommet de la pile, le
hachent par SHA-256 et comparent le résultat à l'empreinte~:

\begin{verbatim}
<secret> || SHA256 <empreinte> EQUAL
\end{verbatim}

De même, la vérification de la validité d'une signature est réalisée par
les scripts suivants qui empilent d'abord la signature, puis la clé
publique avant de contrôler leur correspondance~:

\begin{verbatim}
<signature> || <clé publique> CHECKSIG
\end{verbatim}

En outre, il existe des conditions plus avancées comme les verrous
temporels. Ceux-ci permettent de bloquer les fonds de la pièce pour un
temps précis, que ce soit jusqu'à une date donnée, auquel cas on parle
de temps de verrouillage absolu, ou bien pendant une période donnée,
auquel cas on parle de temps de verrouillage relatif. Le premier est le
fait de l'opérateur
\texttt{dont\ les\ spécificités\ techniques\ sont\ décrites\ dans\ le\ BIP-65.\ Le\ second\ est\ appliqué\ par\ le\ code\ opération}
décrit dans le BIP-112.

\section*{Les schémas classiques}\label{les-schuxe9mas-classiques}
\addcontentsline{toc}{section}{Les schémas classiques}

\markright{Les schémas classiques}

Le langage Script permet de faire des choses diverses et variées.
Pendant les premiers temps de Bitcoin, le système était relativement
libre et autorisait les gens à écrire ce qu'ils voulaient dans les
scripts sans discrimination. Toutefois, cette situation était
considérablement risquée. La raison principale était que le
fonctionnement des codes opération n'était pas encore vérifié et testé,
comme l'avait montré la découverte en juillet 2010 d'une vulnérabilité
rendue possible par certains opérateurs binaires
(CVE-2010-5137\footnote{«~CVE-2010-5137~»~: NIST, \emph{CVE-2010-5137},
  8 juin 2012~: \url{https://nvd.nist.gov/vuln/detail/CVE-2010-5137}.}).
C'est pourquoi il a été décidé à la fin de l'année 2010, sous
l'impulsion de Gavin Andresen, de restreindre la facilité de
programmation du système\footnote{Gavin Andresen, \emph{svn r197:
  IsStandard check for transactions}, /12/2010 13:58:33 UTC~:
  \url{https://bitcointalk.org/index.php?topic=2129.msg27744\#msg27744}.}.

Cette restriction a été appliquée en imposant des schémas standards de
scripts, qui faisaient que les nœuds configurés par défaut ne relayaient
plus les transactions contenant des scripts qui ne respectaient pas ce
standard. Il ne s'agissait pas ainsi d'une restriction des règles
globales de consensus, mais des règles locales de mempool qui
s'appliquent à la transmission des transactions. Des schémas standards
rendant les choses plus simples et plus sûres ont ainsi été développés
au cours des années. Les schémas standards de sortie transactionnelle
étaient en 2023 au nombre de huit~: P2PK, P2PKH, P2MS, P2SH, NULLDATA,
P2WPKH, P2WSH et P2TR\footnote{«~P2PK, P2PKH, P2MS, P2SH, NULLDATA,
  P2WPKH, P2WSH et P2TR~»~:
  \url{https://github.com/bitcoin/bitcoin/blob/22.x/src/script/standard.h\#L59-L71}.}.

\subsection{P2PK~: Pay to Public Key}\label{p2pk-pay-to-public-key}

Le premier schéma s'appelle Pay to Public Key (P2PK), qu'on peut
traduire littéralement en français par «~payer à la clé publique~». Il
s'agit de créer une pièce liée à la clé publique d'un destinataire, que
lui seul peut dépenser en signant avec sa clé privée. Le script de
verrouillage permettant ce type d'envoi est~:

\begin{verbatim}
<clé publique> CHECKSIG
\end{verbatim}

La présence de la clé publique explique qu'on parle parfois de
«~scriptPubKey~» pour désigner le script de verrouillage en général,
indépendemment de ce qu'il contient.

Au moment de la dépense, le destinataire doit utiliser un script de
déverrouillage contenant simplement sa signature~:

\begin{verbatim}
<signature>
\end{verbatim}

La présence de la signature dans ce script explique qu'on parle parfois
de «~scriptSig~» pour désigner le script de déverrouillage en général,
indépendamment de ce qu'il contient. L'exécution successive de ces deux
scripts permet, comme on l'a vu, de vérifier que la signature fournie
par l'utilisateur correspond à sa clé publique, auquel cas elle est
valide.

Le schéma P2PK était utilisé dans les débuts de Bitcoin pour recevoir
les paiements par IP (P2IP) et pour récupérer la récompense de minage.
Il est aujourd'hui tombé en désuétude au profit d'un schéma rival~:
P2PKH.

\subsection{P2PKH~: Pay to Public Key
Hash}\label{p2pkh-pay-to-public-key-hash}

Le schéma Pay to Public Key Hash (P2PKH), qui est traduit littéralement
par «~payer à l'empreinte de la clé publique~», est le deuxième type de
format de réception apparu dans Bitcoin dès le début du fait de la
conception de Satoshi Nakamoto. Ce schéma permet non pas de réaliser un
paiement vers une clé publique, mais vers l'empreinte d'une clé
publique, tout en faisant en sorte que l'interpréteur vérifie quand même
la validité de la signature vis-à-vis de la clé publique lors de la
dépense des fonds. L'empreinte de la clé publique est alors considérée
comme la donnée essentielle («~charge utile~») de l'adresse, qui dans ce
cas commence toujours par un 1, comme par exemple ``. Le script de
verrouillage ici est~:

\begin{verbatim}
DUP HASH160 <empreinte de la clé publique> EQUALVERIFY CHECKSIG
\end{verbatim}

Et le script de déverrouillage est~:

\begin{verbatim}
<signature> <clé publique>
\end{verbatim}

L'exécution des deux scripts permet de~: 1) vérifier que le passage de
la clé publique par la fonction de hachage HASH-160 est égale à
l'empreinte qui est spécifiée dans le script ; 2) vérifier que la
signature correspond à la clé publique.

L'avantage de ce schéma est qu'il permet d'avoir des adresses plus
courtes (l'information à encoder n'est que de 20~octets au lieu de 33 ou
65 octets pour une clé publique), raison pour laquelle Satoshi Nakamoto
l'a implémenté. De plus, en ne révélant la clé publique qu'au moment de
la dépense, ce schéma accroît aussi la sécurité contre la menace (très
hypothétique) de l'ordinateur quantique.

\subsection{P2MS~: Pay To MultiSig}\label{p2ms-pay-to-multisig}

Le schéma Pay To MultiSig (P2SH), qui signifie littéralement «~payer à
la multisignature~», est un schéma de signature multipartite exigeant la
signature de M personnes parmi N participants prédéterminés
(«~M-parmi-N~», ou «~M-of-N~» en anglais). Il a été rendu standard sous
une forme limitée à 3 participants en mars 2012 avec la sortie de la
version 0.6.0 du logiciel\footnote{«~la sortie de la version 0.6.0 du
  logiciel~»~: Gavin Andresen, \emph{Version 0.6.0 released}, 30 mars
  2012~:
  \url{https://bitcointalk.org/index.php?topic=74737.msg827484\#msg827484}.}.
Le script de verrouillage est le suivant~:

\begin{verbatim}
M <clé publique 1> ... <clé publique N> N CHECKMULTISIG
\end{verbatim}

Le script de déverrouillage correspondant est~:

\begin{verbatim}
<leurre (0)> <signature 1> ... <signature M>
\end{verbatim}

La présence du leurre (généralement 0) est dû à un défaut dans
l'implémentation de l'exécution de l'opérateur `` par Satoshi, qui
requiert un élément de trop. Les développeurs n'ont pas jugé opportun de
corriger ce défaut, car cette correction constituait un hard fork.

C'est ce schéma, particulièrement exigeant au niveau de la mise en
place, qui a motivé la création du schéma P2SH.

\subsection{P2SH : Pay to Script Hash}\label{p2sh-pay-to-script-hash}

Le schéma Pay to Script Hash (P2SH), pouvant être traduit littéralement
par «~payer à l'empreinte du script~», reprend l'idée derrière P2PKH, à
la seule différence que la donnée hachée n'est pas une clé publique,
mais le script lui-même~! Le script en question est alors appelé script
de récupération (\emph{redeem script}) pour le différencier du script de
déverrouillage. Son empreinte est la donnée constituante de l'adresse,
cette dernière commençant toujours par un 3 à l'instar de ``.

Ce schéma donne à l'utilisateur la possibilité d'y inclure n'importe
quel script, sans discrimination sur son format, à condition qu'il
respecte bien sûr certaines limites. Il permet aussi de recevoir des
fonds depuis la quasi-totalité des portefeuilles existants, le fardeau
de la construction et du déverrouillage du script revenant uniquement au
destinataire, et n'est pas partagé à l'expéditeur comme dans le cas de
l'utilisation de scripts bruts.

Le script de verrouillage pour le schéma P2SH est~:

\begin{verbatim}
HASH160 <empreinte du script de récupération> EQUAL
\end{verbatim}

Et le script de déverrouillage est un script de la forme~:

\begin{verbatim}
[éléments de déverrouillage] <script de récupération>
\end{verbatim}

L'exécution de P2SH est plus complexe que pour les précédents schémas,
ce qui peut s'expliquer par le contexte dans lequel il a été développé.
L'idée d'implémenter un schéma de script qui utilise l'empreinte d'un
autre script comme l'empreinte de clé publique dans le schéma P2PKH est
née en 2011 par l'intermédiaire de plusieurs propositions\footnote{«~par
  l'intermédiaire de plusieurs propositions~»~: Mike Caldwell
  (casascius), \emph{Proposal to modify OP\_CHECKSIG}, /09/2011 02:21:17
  UTC~:
  \url{https://bitcointalk.org/index.php?topic=45211.msg538756\#msg538756}~;
  jimrandomh, \emph{Proposed extensions to the transaction protocol:
  Receiver scripts, OP\_TIME, more}, /10/2011 16:56:47 UTC~:
  \url{https://bitcointalk.org/index.php?topic=46429.msg553217\#msg553217}~;
  Gavin Andresen, \emph{Re: Proposal to modify OP\_CHECKSIG}, /10/2011
  00:26:42 UTC~:
  \url{https://bitcointalk.org/index.php?topic=45211.msg553668\#msg553668}.}.
Elle a été rendue plus concrète avec la proposition de l'opérateur
\texttt{OP\_EVAL} par Nicolas van Saberhagen le 2 octobre, un code
opération qui permettait l'exécution récursive d'un script à l'intérieur
d'un autre script\footnote{Nicolas van Saberhagen (ByteCoin),
  \emph{OP\_EVAL proposal}, /10/2011 00:49:19 UTC~:
  \url{https://bitcointalk.org/index.php?topic=46538.msg553689\#msg553689}.}.
Gavin Andresen a expliqué comment en faire un soft fork par le
remplacement de l'instruction sans effet \texttt{OP\_NOP1}\footnote{Gavin
  Andresen, \emph{Re: OP\_EVAL proposal}, /10/2011 20:42:32 UTC~:
  \url{https://bitcointalk.org/index.php?topic=46538.msg554620\#msg554620}.}.

L'opérateur \texttt{OP\_EVAL} devait permettre de former un nouveau
schéma standard. Le script de verrouillage aurait été~:

\begin{verbatim}
DUP HASH160 <empreinte du script de récupération> EQUALVERIFY EVAL
\end{verbatim}

tandis que le script de déverrouillage aurait été le même que pour P2SH.
L'exécution successive de ces deux scripts aurait permis dans un premier
temps de vérifier la conformité du hachage du script de récupération
avec l'empreinte~; puis dans un second temps d'exécuter le script de
récupération et de lui combiner les éléments de déverrouillage.
Néanmoins cette solution n'a pas été acceptée, celle-ci ayant été jugée
trop dangereuse à cause de son pouvoir de récursion. Il lui a été
préféré le modèle, plus restrictif, de P2SH.

L'exécution de P2SH fonctionne exactement comme le schéma lié à
\texttt{OP\_EVAL}, à l'exception qu'une partie du script n'est pas
explicitement indiquée. D'un côté, la vérification de la correspondance
entre l'empreinte indiquée et le script de récupération est bien
réalisée par le script de verrouillage. De l'autre côté, l'évaluation du
script de récupération est effectuée implicitement grâce à une exception
ajoutée au code source qui fait que les nœuds du réseau qui
reconnaissent le schéma l'interprètent différemment. Dans Bitcoin Core,
on peut observer cette condition au sein de la fonction `` de
l'interpréteur\footnote{«~fonction VerifyScript de l'interpréteur~»~:
  \url{https://github.com/bitcoin/bitcoin/blob/25.x/src/script/interpreter.cpp\#L2005-L2062}.}.

La proposition a été codifiée dans le BIP-16. Si cette solution est
pratique, elle crée de la complexité et n'est pas très élégante. Comme
le disait Gavin Andresen dans l'explication introductive de ce BIP~:

«~Reconnaître une forme ``spéciale'' de scriptPubKey et réaliser une
validation supplémentaire quand elle est détectée, c'est laid.
Cependant, l'avis général est que les alternatives sont soit encore plus
laides, soit plus complexes à implémenter, et/ou étendent le pouvoir du
langage d'expression de manière dangereuse\footnote{Gavin Andresen,
  \emph{BIP-16: Pay to Script Hash}, 3 janvier 2012~:
  \url{https://github.com/bitcoin/bips/blob/master/bip-0016.mediawiki\#rationale}.}.~»

Le schéma P2SH a fini par être activé le 1 avril 2012 sous la forme d'un
soft fork, en dépit de l'opposition notable de luke-jr qui proposait un
opérateur alternatif, ``, décrit dans le BIP-17\footnote{«~l'opposition
  notable de luke-jr~»~: Amir Taaki, \emph{The Truth behind BIP 16 and
  17}, 29 janvier 2012~:
  \url{http://bitcoinmedia.com/the-truth-behind-bip-16-and-17/}~;
  archive~:
  \url{https://web.archive.org/web/20120202032835/http://bitcoinmedia.com/the-truth-behind-bip-16-and-17/}.}.

\subsection{NULLDATA}\label{nulldata}

Le schéma NULLDATA, signifiant littéralement «~données insignifiantes~»,
est un schéma d'inscription de données arbitraires sur la chaîne. Il est
le quatrième schéma classique et a été rendu standard avec l'arrivée de
la version 0.9.0 de Bitcoin Core en mars 2014\footnote{«~l'arrivée de la
  version 0.9.0 de Bitcoin Core en mars 2014~»~: Bitcoin Core,
  \emph{Bitcoin Core version 0.9.0 released}, 19 mars 2014~:
  \url{https://bitcoin.org/en/release/v0.9.0\#opreturn-and-data-in-the-block-chain}.}.
Il se base sur l'instruction \texttt{OP\_RETURN} dont l'effet est de
mettre fin à l'exécution du script et de rendre indépensable la pièce
correspondante\footnote{L'instruction \texttt{OP\_RETURN} servait
  initialement à retourner la valeur au sommet de la pile, d'où son nom.
  Cependant, en juillet 2010, la découverte du «~\emph{1 RETURN bug}~»,
  qui permettait de dépenser toute sortie transactionnelle \emph{via} le
  script de déverrouillage
  `\texttt{,\ a\ poussé\ Satoshi\ Nakamoto\ à\ désactiver\ cette\ fonctionnalité\ en\ lui\ faisant\ renvoyer}FALSE`
  systématiquement. Voir Satoshi Nakamoto, \emph{reverted makefile.unix
  wx-config -- version 0.3.6 (git commit)}, /07/2010 18:27:12 UTC~:
  \url{https://sourceforge.net/p/bitcoin/code/119/}.}. Le script de
verrouillage du schéma commence toujours par \texttt{OP\_RETURN} et est
suivi des données empilées~:

\begin{verbatim}
RETURN [données arbitraires]
\end{verbatim}

La sortie contenant ce script est exempt de la limite standarde de
poussière, qui est actuellement de 546 satoshis pour les sorties P2PKH,
de sorte qu'elle peut être de 0 satoshi. La taille maximale des données
pouvant être inscrites est de 80 octets par transaction sur BTC. De
plus, en raison de leur caractère assurément indépensable, les sorties
peuvent être retranchées de l'ensemble des UTXO des nœuds. Tout ceci
fait de ce schéma le moyen normal d'inscrire des informations sur le
registre.

\section*{Les types de signatures}\label{les-types-de-signatures}
\addcontentsline{toc}{section}{Les types de signatures}

\markright{Les types de signatures}

La programmabilité de Bitcoin n'est pas seulement issue de son langage
de programmation mais aussi du système de signature qui permet de
sélectionner quelle partie de la transaction est signée. Ce facteur de
programmabilité est mis en œuvre par l'existence d'un indicateur, appelé
type de hachage de la signature ou \emph{signature hash type}, qui est
ajouté à la transaction non signée, puis à la signature elle-même.
Celui-ci indique quelle partie de la transaction doit être hachée avant
d'être soumise à l'algorithme de signature, d'où son nom.

Le type de signature est construit à partir de plusieurs signaux de
signature qui peuvent être combinés. Les quatre signaux de signature qui
existent sont~:

\begin{itemize}
\item
  `\texttt{(}0x01`) qui indique que toutes les sorties sont signées~;
\item
  `\texttt{(}0x03`) qui permet de ne signer qu'une seule sortie~;
\item
  `\texttt{(}0x02`) qui indique qu'aucune sortie n'est signée~;
\item
  `\texttt{(}0x80`) qui permet de ne signer qu'une seule entrée.
\end{itemize}

Les trois signaux concernant les sorties peuvent être associés à
\texttt{,\ ce\ qui\ permet\ de\ former\ finalement\ six\ types\ de\ signatures\ différents,\ représentés\ sur\ la\ figure~{[}12.6{]}(\#fig:signature-hash-types)\{reference-type="ref"\ reference="fig:signature-hash-types"\}.\ Le\ type\ de\ signature\ le\ plus\ fréquent\ est\ évidemment}
même si certains autres types peuvent parfois trouver une utilité. C'est
notamment le cas de `` qui permet de construire des transactions de type
\emph{anyone-can-pay}, dont les sorties sont déterminées, mais où chacun
peut signer sa propre entrée sans connaître les autres.

\begin{figure}

{\centering \includegraphics{chapters/img/signature-hash-types.png}

}

\caption{Les différents types de signatures dans Bitcoin.}

\end{figure}%

Ces signaux ont été implémentés dès le début par Satoshi Nakamoto au
sein du prototype. Il en manquait logiquement un, que Satoshi Nakamoto a
probablement jugé inutile~: celui qui ne signait aucune entrée.
Toutefois, avec le développement des canaux de paiements pour le réseau
Lightning, les développeurs se sont rendu compte qu'il pouvait avoir une
utilité. C'est dans cet esprit que le signal de signature `` a été
proposé en février 2016 par Joseph Poon\footnote{Joseph Poon,
  \emph{{[}bitcoin-dev{]} SIGHASH\_NOINPUT in Segregated Witness},
  /02/2016 01:07:46 UTC~:
  \url{https://lists.linuxfoundation.org/pipermail/bitcoin-dev/2016-February/012460.html}.}.

Ce type de signal pourrait être implémenté de manière partielle dans BTC
par l'intermédiaire du BIP-118, qui prévoit l'intégration de deux
nouveaux signaux au sein des scripts de Taproot~:
\texttt{et}\footnote{«~BIP-118~»~: Christian Decker, Anthony Towns,
  \emph{BIP-118: SIGHASH\_ANYPREVOUT for Taproot Scripts}, 28 février
  2017~:
  \url{https://github.com/bitcoin/bips/blob/master/bip-0118.mediawiki}.}.
Il permettrait d'améliorer le fonctionnement du réseau Lightning par la
mise en œuvre du protocole Eltoo reposant sur la construction de
transactions flottantes.

\section*{SegWit~: le témoin
séparé}\label{segwit-le-tuxe9moin-suxe9paruxe9}
\addcontentsline{toc}{section}{SegWit~: le témoin séparé}

\markright{SegWit~: le témoin séparé}

SegWit, abréviation de \emph{Segregated Witness}, qu'on peut traduire
littéralement par «~témoin séparé~», est une mise à niveau du protocole
ayant lieu sur Litecoin-LTC et sur Bitcoin-BTC en 2017. Elle a consisté
à faire en sorte que les données de déverrouillage des entrées
transactionnelles, telles que les signatures, se retrouvent dans une
structure de données séparée (\emph{segregated}) appelée le témoin
(\emph{witness}) afin de supprimer la malléabilité des transactions.
SegWit constituait ainsi une restructuration profonde des transactions.

Outre la correction de la malléabilité, SegWit a apporté une
augmentation de capacité transactionnelle et un versionnage des scripts
pour faciliter les mises à niveau ultérieures. Elle a également amélioré
l'algorithme de signature pour éviter les hachages redondants durant la
vérification et pour rendre plus sûre la signature
hors-ligne\footnote{«~Elle a également amélioré l'algorithme de
  signature pour éviter les hachages redondants durant la vérification
  et pour rendre plus sûre la signature hors-ligne~»~:
  \url{https://github.com/bitcoin/bips/blob/master/bip-0143.mediawiki}.}.

\subsection{La malléabilité}\label{la-malluxe9abilituxe9}

SegWit tire son origine du problème de la malléabilité des transactions,
un problème identifié depuis janvier 2012\footnote{«~problème identifié
  depuis janvier 2012~»~: Gavin Andresen,
  \emph{{[}Bitcoin-development{]} Extending IsStandard() to transaction
  scriptSigs}, /1/2012 16:29:29,
  \url{https://lists.linuxfoundation.org/pipermail/bitcoin-dev/2012-January/001066.html}.}.
Dans Bitcoin, les transactions sont malléables dans le sens où elles
peuvent être modifiées légèrement après leur diffusion sans devenir
invalides aux yeux du réseau. Cette propriété vient du fait qu'une
signature ne peut pas se prendre en compte elle-même et que, par
conséquent, le script de déverrouillage n'est pas signé avec le reste de
la transaction. La malléabilité peut ainsi prendre deux formes~: la
malléabilité intrinsèque à l'algorithme ECDSA, qui se base sur un nombre
aléatoire pour produire une signature (malléabilité par le signataire)~;
la malléabilité provenant de la forme des signatures et des scripts de
déverrouillage des entrées (malléabilité par un tiers).

La malléabilité n'est pas rédhibitoire pour la sécurité des fonds, mais
elle permet de modifier l'identifiant de la transaction après sa
publication, ce qui peut se révéler problématique dans certaines
situations. Ainsi, entre le 9 et le 11 février 2014, Mt. Gox et d'autres
plateformes de change ont subi des attaques exploitant cette
malléabilité des transactions. Les transactions de retrait ont été
modifiées par les attaquants, faisant croire aux plateformes, dont
l'infrastructure logicielle était mal configurée, que ces transactions
n'avaient pas été confirmées. Les pirates ont vu leurs comptes être
recrédités tout en conservant dans le même temps les bitcoins retirés.
Ces attaques ont mené à une perte totale de 64~564 bitcoins\footnote{Sur
  l'attaque de méalléabilité contre Mt. Gox, voir Ken Shirriff,
  \emph{The Bitcoin malleability attack graphed hour by hour}, 15
  février 2014~:
  \url{https://www.righto.com/2014/02/the-bitcoin-malleability-attack-hour-by.html}.
  Voir aussi Christian Decker, Roger Wattenhofer, \emph{Bitcoin
  Transaction Malleability and MtGox}, 26 mars 2014~:
  \url{https://arxiv.org/pdf/1403.6676.pdf}.}.

Des propositions ont tenté de corriger la malléabilité par un tiers en
contraignant au maximum la forme des transactions. C'est dans cet esprit
qu'a été créé le BIP-62 en mars 2014, dont l'une des exigences
(l'encodage standard des signatures décrit dans le BIP-66) a été incluse
dans les règles de consensus le 4 juillet 2015. Toutefois, ces
changements ne s'appliquaient pas à la malléabilité par le signataire,
ce qui créait une demande pour une correction généralisée.

Cette malléabilité signifiait que tout acteur participant à un contrat
de signature multipartite pouvait modifier la transaction et donc son
identifiant à tout moment. Cela altérait significativement la
possibilité d'implémentation du réseau Lightning, dont les canaux de
paiements, comme on le verra plus bas, se basent sur des transactions
non publiées auxquelles il faut faire référence et font intervenir des
signatures multiples.

La solution était de mettre de côté les scripts de déverrouillage dans
le processus de hachage de la transaction, pour qu'un changement de ces
scripts n'influence pas l'identifiant. Cette idée a été proposée
initialement par Gregory Maxwell en août 2013 sur IRC\footnote{«~proposée
  initialement par Gregory Maxwell en août 2013 sur IRC~»~: Gregory
  Maxwell, IRC, /08/2013 20:21 UTC~:
  \url{https://download.wpsoftware.net/bitcoin/wizards/2013/08/13-08-29.log}~:
  «~Je suggère de ne jamais hacher cette valeur dans le protocole. En
  gros, je dis que les scriptsigs pour une {[}transaction{]} seraient un
  arbre de hachage séparé. Il est toujours engagé dans la chaîne de
  blocs mais ce serait une branche séparée.~»}, avant d'être mise en
œuvre au sein de la version alpha du modèle de sidechain Elements,
annoncée le 8 juin 2015 par Blockstream\footnote{«~version alpha du
  modèle de sidechain Elements, annoncée le 8 juin 2015~»~: Adam Back,
  \emph{Announcing Sidechain Elements: Open-Source Code and Developer
  Sidechains for Advancing Bitcoin}, 8 juin 2015~:
  \url{https://blog.blockstream.com/en-714/}.}. Le même jour, Gregory
Maxwell présentait cette version d'Elements incluant \emph{Segregated
Witness} dans un séminaire de développeurs à San Francisco~: il
décrivait alors le témoin comme «~une valeur spécifique qui constitue
une preuve concrète d'affirmation existentielle\footnote{SF Bitcoin
  Developers, \emph{Sidechains: Bringing New Elements to Bitcoin}
  (vidéo), 8 juin 2015~:
  \url{https://www.youtube.com/watch?v=Twynh6xIKUc}.}~».

Cette solution a été adaptée pour Bitcoin au cours de l'automne 2015,
pour être appliquée comme un soft fork. La mise à niveau SegWit a été
officiellement introduite à la communauté par le développeur Pieter
Wuille le 7 décembre 2015, lors de la conférence Scaling Bitcoin
\textsc{II} à Hong Kong. En substance, elle consistait à déplacer les
scripts de déverrouillage dans le témoin de la transaction. Deux
identifiants étaient alors calculés~: l'identifiant classique
(\texttt{txid}), qui ne prend pas en compte ce témoin, et l'identifiant
complet (noté \texttt{wtxid} pour \emph{witness transaction
identifier}), qui recouvre l'intégralité de la transaction. Les
identifiants complets étaient regroupés dans un second arbre de Merkle,
l'arbre témoin, dont la racine était placée dans la transaction de
récompense du bloc, ce qui faisait que toutes les données étaient
engagées dans le calcul de la preuve de travail. De l'autre côté, les
transactions et les blocs restaient valides pour les nœuds n'ayant pas
été mis à niveau.

SegWit est active depuis le 24 août 2017. L'absence de script de
déverrouillage dans le calcul de l'identifiant classique permet de ne
plus avoir de malléabilité du tout, ni des signataires, ni d'un tiers
extérieur.

\subsection{L'augmentation de la capacité
transactionnelle}\label{laugmentation-de-la-capacituxe9-transactionnelle}

SegWit a aussi eu pour effet indirect de créer un bloc d'extension et
d'augmenter la capacité transactionnelle. En effet, les nœuds suivant
les anciennes règles ne voyaient pas le témoin, de sorte qu'ils ne le
comptabilisaient pas dans la taille du bloc. La question était alors de
savoir quelle limite mettre sur le témoin.

La réponse a été d'inventer une nouvelle métrique pour mesurer l'impact
des transactions et des blocs sur le réseau~: le poids (\emph{weight}),
qui est une moyenne pondérée de la taille de base et de la taille du
témoin. Exprimé en unités de poids (\emph{weight unit}), il est défini
comme la somme du quadruple de la taille de base (\(s_b\)) et de la
taille du témoin (\(s_w\))~:

\[w = 4 ~ s_b + s_w\]

Il en découle une taille virtuelle (\(s_v\)) qui est définie comme la
somme de la taille de base et du quart de la taille du témoin,
c'est-à-dire~: \(s_v = s_b + \frac{s_w}{4}\). La taille limite des blocs
est devenue un poids limite des blocs, qui était de 4 millions d'unités
au moment de la mise à niveau et qui était toujours le même en novembre
2023.

De ce fait, les frais qui étaient initialement calculés en satoshis par
octet (sat/o), sont, depuis SegWit, mesurés en satoshis par octet
virtuel (sat/ov). Les mineurs sélectionnent les transactions en fonction
de ce taux afin d'être les plus rentables possibles par rapport à cette
limite. Cet effet n'est valable que si la limite est atteinte.

Avec SegWit, il s'agit donc de pondérer l'impact des entrées par rapport
à celui des sorties sur le calcul des frais. Si l'activité rejoint le
plafond de capacité, alors les sorties sont quatre fois plus chères à
inscrire sur la chaîne que les scripts de déverrouillage contenus dans
les entrées. La mise à niveau, en plus d'installer une remise qui incite
à son usage, a créé une dissuasion à alourdir l'ensemble des UTXO. Le
facteur 4 se rapproche de la pondération matérielle\footnote{SegWit
  Resources, \emph{Why a discount factor of 4? Why not 2 or 8?}, 13
  janvier 2017~:
  \url{https://medium.com/segwit-co/why-a-discount-factor-of-4-why-not-2-or-8-bbcebe91721e}.}.

Cette limite de 4 millions d'unités de poids est indicative. La taille
réelle des blocs n'atteint généralement pas 4~Mo en raison de la forme
des transactions. Les données contenues dans une transaction normale ne
sont en effet pas regroupées dans le témoin, de sorte qu'elle ne
remplissent pas entièrement l'espace de bloc autorisé. Par exemple, si
nous prenons un bloc constitué uniquement de transactions à 2 entrées et
2 sorties utilisant SegWit, alors sa taille réelle sera de
1,784~Mo\footnote{Une transaction à 2 entrées et 2 sorties de type
  P2WPKH mesure 372 octets et pèse 834 unités de poids au maximum. De ce
  fait, il est possible d'inclure 4~796 transactions dans un bloc, ce
  qui nous permet de calculer sa taille réelle.}.

Les transactions dont les données de déverrouillage sont plus grandes
profitent mieux de cet espace de bloc supplémentaire. C'est le cas des
transactions qui utilisent la multisignature telles que les fermetures
de canaux de paiement. Il est ainsi possible d'approcher la taille des
4~Mo en maximisant la taille des données contenues dans le témoin. C'est
ce qui a été fait le 1 février 2023 avec la création d'un bloc de
3,955~Mo dont le témoin a servi à l'inscription d'une image\footnote{Voir
  le bloc 774~628, d'identifiant `` dont la taille était de
  3~955~272~octets et qui incluait une transaction qui mesurait à elle
  seule 3~938~383~octets.}.

\subsection{Le versionnage des
scripts}\label{le-versionnage-des-scripts}

Enfin, la mise à niveau SegWit a apporté un versionnage des scripts, qui
permettait le déploiement de futures mises à niveau. La version
permettait ainsi d'indiquer quelles règles étaient appliquées. La
première version de SegWit en 2017 utilisait la version 0, et le
déploiement de Taproot en 2021 a été fait au moyen de la version 1.

Trois types de sortie natifs liés à SegWit existent pour l'instant~: le
schéma P2WPKH, le schéma P2WSH et le schéma P2TR.

\subsection{P2WPKH~: Pay to Witness Public Key
Hash}\label{p2wpkh-pay-to-witness-public-key-hash}

Le schéma \emph{Pay to Witness Public Key Hash} (P2WPKH), qui signifie
littéralement «~payer à l'empreinte de la clé publique témoin~», est le
premier schéma mis en place par SegWit. L'empreinte de la clé publique
est obtenue par le hachage standard (SHA-256 suivi de RIPEMD-160). Le
script de verrouillage apparent est alors~:

\begin{verbatim}
<version (0)> <empreinte (hash160) de la clé publique>
\end{verbatim}

Ce script est semblable à un script \emph{anyone-can-spend}, que tout le
monde pourrait dépenser, mais l'interpréteur ajoute une condition
supplémentaire pour que ce ne soit pas le cas. Le type de la sortie est
détecté grâce à sa forme~: la version de SegWit (ici 0) et la taille de
l'empreinte (ici 20 octets). La version et l'empreinte forment
l'information essentielle de l'adresse, qui est encodée grâce au format
Bech32 et qui commence toujours par \texttt{bc1q}, à l'instar de ``.

Le script de déverrouillage est vide. Les données de déverrouillage sont
contenues dans le témoin de la transaction. La partie du témoin
correspondant à l'entrée est~:

\begin{verbatim}
<2> <signature> <clé publique>
\end{verbatim}

\subsection{P2WSH~: Pay to Witness Script
Hash}\label{p2wsh-pay-to-witness-script-hash}

Le schéma \emph{Pay to Witness Script Hash} (P2WSH), dont la traduction
littérale est «~payer à l'empreinte du script témoin~», est la
retranscription de P2SH pour SegWit. L'empreinte du script de
récupération est obtenue par SHA-256, par peur d'une collision de
RIPEMD-160 dans le cas d'une adresse générée par plusieurs
personnes\footnote{Gavin Andresen, \emph{{[}bitcoin-dev{]} Time to worry
  about 80-bit collision attacks or not?}, /01/2016 19:02:05 UTC~:
  \url{https://lists.linuxfoundation.org/pipermail/bitcoin-dev/2016-January/012198.html}.}.
Le script de verrouillage est le suivant~:

\begin{verbatim}
<version (0)> <empreinte (sha256) du script de récupération>
\end{verbatim}

Ce script est encore une fois \emph{anyone-can-spend} de manière
apparente. Le type de la sortie est détecté par l'interpréteur grâce à
sa forme~: la version de SegWit (ici 0) et la taille de l'empreinte (ici
32 octets). L'adresse est à nouveau constituée de ces deux informations
et encodée grâce au format Bech32.

Le script de déverrouillage est vide. Les données de déverrouillage sont
contenues dans le témoin de la transaction. La partie du témoin
correspondant à l'entrée est~:

\begin{verbatim}
<nombre d'élements + 1> [éléments de déverrouillage] <script de
récupération>
\end{verbatim}

Dans les deux cas, l'empreinte est aussi appelée «~programme du
témoin~».

\subsection{Les types imbriqués (P2SH-P2WPKH,
P2SH-P2WSH)}\label{les-types-imbriquuxe9s-p2sh-p2wpkh-p2sh-p2wsh}

SegWit a aussi modifié le format P2SH pour inclure de nouvelles
exceptions. Ces exceptions correspondent aux types imbriqués
(\emph{nested}) P2SH-P2WPKH et P2SH-P2WSH. Leur fonctionnement consiste
à inclure les scripts de verrouillages précédents (version + empreinte)
dans une sortie P2SH en tant que scripts de récupération. Le script de
récupération est alors exécuté différemment pour faire appel aux données
contenues dans le témoin.

Ces types imbriqués ont permis de faciliter la transition vers SegWit en
rendant les portefeuilles non mis à jour capables d'envoyer des fonds
vers ces adresses. L'utilisation d'adresses SegWit natives reste
néanmoins plus avantageuse.

\subsection{P2TR~: Pay to Taproot}\label{p2tr-pay-to-taproot}

Le dernier schéma à entrer en vigueur est le schéma \emph{Pay to
Taproot} (P2TR), dont le nom peut être traduit par «~payer à Taproot~».
Ce schéma permet de recevoir un paiement sur une clé publique externe
qui cache une clé privée servant à signer les transferts de fonds, ou
bien la racine pivot d'un arbre de Merkle contenant les clauses d'un
contrat autonome (MAST). Puisque la destination du paiement est une clé
publique, il s'agit en quelque sorte d'un retour au P2PK. Le script de
verrouillage présent dans la sortie transactionnelle est~:

\begin{verbatim}
<version (1)> <clé publique Taproot>
\end{verbatim}

La clé publique en question mesure 32 octets. La version et la clé
publique constituent les éléments constitutifs de l'adresse. Cette
dernière est encodée grâce au format Bech32m, qui est une variante de
l'encodage Bech32 ayant corrigé un petit bug dans le calcul de la somme
de contrôle. L'adresse résultante commence toujours par \texttt{bc1p}
comme par exemple ``. Le déverrouillage de la sortie se fait avec une
signature simple, ou bien avec l'exécution du MAST.

Toutes ces modifications majeures font de SegWit une mise à niveau
profonde du protocole, qui a amené beaucoup de choses dans Bitcoin.
L'exigence de passer par un soft fork explique la forme qu'elle a prise
et elle ne peut par conséquent être comprise que dans le contexte dans
lequel elle a été activée. Toutefois, cette mise à niveau a aussi
apporté des inconvénients majeurs, dont les deux principaux sont la
dette technique alourdissant le coût de maintien et d'amélioration du
code, et l'affaiblissement de la confidentialité générale due à
l'apparition de nouveaux types d'adresses partiellement adoptés. SegWit
était donc loin d'être une mise à niveau parfaite.

\section*{Le mélange de pièces}\label{le-muxe9lange-de-piuxe8ces}
\addcontentsline{toc}{section}{Le mélange de pièces}

\markright{Le mélange de pièces}

Le fait que les transactions soient publiées sur la chaîne entraîne une
surveillance. Comme nous l'avons fait remarquer précédemment, il est
possible de faire des suppositions pour deviner ce qui se passe
réellement sur la chaîne, en admettant que l'utilisateur cherche à
minimiser les frais payés au sein de ses transactions. Ces heuristiques
(telles que l'heuristique de co-dépense, l'heuristique de la sortie
complémentaire ou encore l'heuristique de l'empreinte du portefeuille)
forment la base d'une discipline appelée l'analyse de chaîne qui
consiste à recouper ces observations avec l'identification d'acteurs
réels afin de tirer des conclusions sur leur activité économique
effective. C'est pourquoi on parle parfois de «~transparence~» de la
chaîne.

Cependant, cette transparence est toute relative, car les données de la
chaîne ne révèlent pas l'identité des personnes~: le système est
pseudonyme, dans le sens où il recense les mouvements entre les
adresses, et pas entre les personnes. Le modèle de confidentialité de
Bitcoin, décrit par Satoshi Nakamoto dans le livre blanc en 2008,
consiste ainsi à garder secret le lien qui existe entre l'identité d'une
personne et ses adresses\footnote{«~Le modèle bancaire traditionnel
  atteint un certain niveau de confidentialité en limitant l'accès aux
  informations aux parties concernées et au tiers de confiance. La
  nécessité d'annoncer publiquement toutes les transactions exclut cette
  méthode, mais la confidentialité peut toujours être préservée en
  interrompant le flux d'informations à un autre endroit~: en gardant
  les clés publiques anonymes. Le public peut voir que quelqu'un envoie
  un montant à quelqu'un d'autre, mais ne dispose pas d'informations
  reliant la transaction à qui que ce soit.~» -- Satoshi Nakamoto,
  \emph{Bitcoin: A Peer-to-Peer Electronic Cash System}, 31 octobre
  2008.}.

\begin{figure}[H]

{\centering \includegraphics{chapters/img/white-paper-privacy-model-fr.png}

}

\caption{Modèle de confidentialité présenté dans le livre blanc de
Bitcoin.}

\end{figure}%

Ce modèle de confidentialité possède des faiblesses évidente~: les
fuites d'information accidentelles, qui ont toujours lieu en ce qui
concerne le numérique, et la divulgation volontaire de l'identité de
l'utilisateur par son interlocuteur dans l'échange. Par conséquent, nul
ne peut prétendre exercer une activité complètement secrète qui
échapperait absolument à la surveillance\footnote{«~nul ne peut
  prétendre exercer une activité complètement secrète qui échapperait
  absolument à la surveillance~»~: Les premiers utilisateurs de Bitcoin
  ont ainsi été bien imprudents, à l'instar de Hal Finney qui a révélé
  des informations entre 2013 et 2014 permettant de déduire qu'il
  possédait plus de 10~000~bitcoins en 2011. -- Hal Finney,
  \emph{Bitcoin and me}, /03/2013 20:40:02 UTC~:
  \url{https://bitcointalk.org/index.php?topic=155054.msg1643833\#msg1643833}~;
  Andy Greenberg, \emph{Nakamoto's Neighbor: My Hunt For Bitcoin's
  Creator Led To A Paralyzed Crypto Genius}, 25 mars 2014~:
  \url{https://www.forbes.com/sites/andygreenberg/2014/03/25/satoshi-nakamotos-neighbor-the-bitcoin-ghostwriter-who-wasnt/}.}.
C'est pour cette raison qu'il existe des méthodes permettant de limiter
l'effet de ces révélations afin de restaurer sa confidentialité en toute
sérénité.

La première mesure est l'usage unique des adresses. Elle consiste à
générer une nouvelle clé privée et une nouvelle adresse lors de chaque
paiement entrant ou sortant. L'apport de cette pratique est de réduire
l'impact de la divulgation du lien avec l'identité sur la
confidentialité générale~: tant que l'adresse n'est pas liée à d'autres
par l'observation d'une action sur la chaîne (co-dépense par exemple),
la fuite d'information se limite à cette seule adresse. Cette bonne
pratique, citée dans le livre blanc\footnote{«~Comme pare-feu
  supplémentaire, une nouvelle paire de clés devrait être utilisée pour
  chaque transaction afin de les empêcher d'être liées à un propriétaire
  commun. Certains liens sont toujours inévitables avec les transactions
  à entrées multiples, qui révèlent nécessairement que leurs entrées
  appartiennent au même propriétaire. Le risque est que si le
  propriétaire d'une clé est révélé, la liaison pourrait révéler
  d'autres transactions qui lui appartiennent.~» -- Satoshi Nakamoto,
  \emph{Bitcoin: A Peer-to-Peer Electronic Cash System}, 31 octobre
  2008.}, est aujourd'hui implémentée dans tous les bons portefeuilles.

Au-delà de la prévention, il existe également des méthodes pour corriger
ses erreurs. La plus connue d'entre elles est le mélange de pièces, qui
consiste à combiner ses UTXO avec les UTXO d'autres utilisateurs afin de
briser les liens déterministes qui existe entre les pièces et l'identité
de leurs propriétaires.

Le mélange de bitcoins était originellement pris en charge par des
services de mixage centralisés, appelés \emph{mixers} ou
\emph{tumblers}, qui recevaient les bitcoins des utilisateurs, les
fusionnaient et leur renvoyaient des bitcoins communs au bout d'un
certain temps, préférablement sous la forme de plusieurs transactions.
Le premier mélangeur de ce type était BitLaundry, une plateforme qui a
été lancée en septembre 2010 par Peter Vessenes\footnote{«~BitLaundry,
  une plateforme qui a été lancée en septembre 2010 par Peter
  Vessenes~»~: Peter Vessenes, \emph{Announcing: BitLaundry --
  decorrelated payment service}, /09/2010 05:52:25 UTC~:
  \url{https://bitcointalk.org/index.php?topic=963.msg11823\#msg11823}.}.
Ces services permettaient d'obscurcir la provenance des bitcoins pour un
observateur extérieur, mais pas pour leurs gestionnaires, qui pouvaient
aussi s'emparer des bitcoins au passage, ce qui constituait un risque
double.

Une technique pour procéder à ce type de mélange sans devoir passer par
un intermédiaire a été développée par la suite~: c'était CoinJoin, dont
la description formelle a été faite en août 2013 par Gregory
Maxwell\footnote{Gregory Maxwell, \emph{CoinJoin: Bitcoin privacy for
  the real world}, /08/2013 02:32:31 UTC~:
  \url{https://bitcointalk.org/index.php?topic=279249.msg2983902\#msg2983902}.}.
Cette méthode consiste à impliquer les pièces dans une transaction
jointe collaborative qui brise la correspondance entre les entrées et
une partie des sorties. La transaction classique que l'on se représente
est celle de plusieurs utilisateurs qui signent chacun une entrée, dont
le même nombre de sorties possèdent un montant égal, et dont le reste
des sorties forment les sorties complémentaires. Dans ce cas, les
sorties complémentaires sont toujours liées aux entrées, contrairement
aux sorties principales qui sont indiscernables les unes des autres.

\begin{figure}

{\centering \includegraphics{chapters/img/coinjoin-transaction-5i-10o.png}

}

\caption{Exemple d'une transaction CoinJoin à 5 utilisateurs.}

\end{figure}%

Ces mélanges reposent sur la notion d'«~ensemble d'anonymat~»
(\emph{anonymity set}) qui permet de mesurer la difficulté à faire le
lien entre l'entrée et la sortie à un moment donné. On peut ainsi
obtenir un score prospectif qui est le nombre de possibilités de pièces
en sorties auxquelles peuvent correspondre une pièce en entrée. Dans
notre exemple illustré par la
figure~\hyperref[fig:coinjoin-transaction]{12.7}, le score prospectif de
la sortie au moment de la transaction est de 5. Si la pièce avait subi
un nouveau mélange (comme c'est fait dans Whirlpool), alors elle aurait
eu un score prospectif de 9. De même, si l'une des autres pièces avait
été incluse dans un nouveau mélange, alors le score de la pièce observée
aurait augmenté d'autant. On peut aussi calculer un score rétrospectif
qui correspond au nombre de potentielles pièces en entrée auxquelles
peut être liée une sortie particulière, qu'on suppose être de 5 dans le
cas de notre transaction simple, mais qui peut être largement supérieur
si une ou plusieurs pièces ont déjà fait l'objet de mélanges
successifs\footnote{Loïc Morel, \emph{Comprendre et utiliser le CoinJoin
  sur Bitcoin}, 19 juillet 2022~:
  \url{https://www.pandul.fr/post/comprendre-et-utiliser-le-coinjoin-sur-bitcoin}.}.

Pour gérer le tout, le système utilise généralement un protocole qui
permet aux participants d'être mis en relation anonymement par le biais
d'un coordinateur sans risque de fuite d'information ou de vol des
fonds. Le plus connu est ZeroLink, développé par Adam Ficsor et TDevD en
août 2017, qui est un protocole qui utilise le procédé de signature
aveugle de David Chaum\footnote{Adam Ficsor (nopara73), TDevD,
  \emph{ZeroLink: The Bitcoin Fungibility Framework}, 14 août 2017~:
  \url{https://github.com/nopara73/ZeroLink/tree/32ad53927a343383534bea28fffb098af65fe62a}.}.
C'est en ce sens qu'on parle parfois de CoinJoin chaumien
(\emph{Chaumian CoinJoin}). Une implémentation classique de cette idée a
été réalisée par Whirlpool et par Wasabi 1.0. De plus, des variantes
(CoinShuffle, CoinShuffle++, CashShuffle, CashFusion) ont été
implémentées sur des variantes de Bitcoin comme Decred ou Bitcoin Cash.
Plus récemment le portefeuille Wasabi a intégré Wabisabi qui permet de
réaliser des mélanges avec des valeurs arbitraires en sortie, ce qui
complique l'estimation de la confidentialité apportée mais évite d'avoir
à gérer les sorties complémentaires d'une manière séparée.

Pour autant, les transactions collaboratives ne se limitent pas à
CoinJoin. Il existe par exemple une autre méthode, appelée PayJoin, qui
permet au commerçant de réaliser un mélange avec le client au moment du
paiement, en impliquant une pièce en entrée. Cette opération a pour
effet de fausser l'analyse de chaîne en faisant croire à l'observateur
extérieur qu'un seul utilisateur a réuni ses pièces en entrée et en
cachant le montant réel du paiement.

Reprenons notre exemple d'Alice qui paie 7 mBTC à Bob en réunissant deux
pièces de 6 et 2 mBTC afin d'atteindre un montant suffisant en entrée.
Dans ce cas, les deux entrées sont supposément liées entre elles
(heuristique de co-dépense) et liées à la sortie de 1 mBTC (heuristique
de la sortie complémentaire). Ici, comme représenté sur la
figure~\hyperref[fig:payjoin-transaction]{12.8}, l'application de
PayJoin consiste pour le commerçant à inclure une ou plusieurs pièces en
entrée et à augmenter d'autant le montant de la sortie qui lui est
destinée, de 7 mBTC par exemple.

\begin{figure}

{\centering \includegraphics{chapters/img/payjoin-transaction-3i-2o.png}

}

\caption{Exemple d'une transaction PayJoin.}

\end{figure}%

Cette technique a été conceptualisée en 2018 de plusieurs manières
indépendantes, notamment par le biais du protocole de paiement
Pay-to-EndPoint\footnote{«~protocole de paiement Pay-to-EndPoint~»~:
  Adam Ficsor, \emph{Pay To EndPoint}, 31 juillet 2018~:
  \url{https://nopara73.medium.com/pay-to-endpoint-56eb05d3cac6}.}
(P2EP) et par les transactions Stowaway de Samourai Wallet\footnote{«~transactions
  Stowaway de Samourai Wallet~»~: Samourai Wallet, \emph{Stowaway}~:
  \url{https://samouraiwallet.com/stowaway}. -- Il y a également
  protocole Bustapay qui a été proposé dans le BIP-79 en août 2018.}.
Leur implémentation s'est faite respectivement en 2019 pour les
transactions Stowaway\footnote{«~2019 pour les transactions Stowaway~»~:
  Samourai Wallet, \emph{Collaborative Transactions - "Cahoots"}, 11
  mars 2019~:
  \url{https://blog.samouraiwallet.com/post/183378923792/collaborative-transactions-cahoots}.}
et en 2020 pour P2EP\footnote{«~2020 pour P2EP~»~: Samson Mow, Daniel
  Williams, \emph{Bitcoin Privacy Improves With BTCPay Server's P2EP
  Implementation}, 16 avril 2020~:
  \url{https://blog.blockstream.com/en-bitcoin-privacy-improves-with-btcpay-servers-p2ep-implementation/}.}.

Enfin, une dernière méthode qui s'inscrit dans la logique du mélange de
pièces est Coinswap, qui est un procédé développé par Chris Belcher,
permettant à deux utilisateurs ou plus d'échanger leurs pièces sans
qu'ils aient besoin de se faire confiance et sans que cette opération
laisse une trace particulière sur la chaîne\footnote{Chris Belcher,
  \emph{Design for a CoinSwap Implementation for Massively Improving
  Bitcoin Privacy and Fungibility}, 25 mai 2020~:
  \url{https://gist.github.com/chris-belcher/9144bd57a91c194e332fb5ca371d0964}.}.
Cette technique comporte cependant un inconvénient supplémentaire dans
le sens où l'une des parties récupère l'historique entier de la pièce de
l'autre, et doit en assumer l'éventuelle responsabilité.

\section*{D'autres techniques de
confidentialité}\label{dautres-techniques-de-confidentialituxe9}
\addcontentsline{toc}{section}{D'autres techniques de confidentialité}

\markright{D'autres techniques de confidentialité}

Outre le mélange de pièces simple consistant à brouiller les pistes
qu'un observateur externe pourrait suivre, il existe un certain nombre
de techniques qui permettent d'améliorer la confidentialité de Bitcoin.
Celles-ci requièrent souvent la modification du protocole de base et
représentent des compromis, raison pour laquelle elles ne sont pas
forcément mises en œuvre.

Ces techniques ont été développées dans les années qui ont suivi
l'apparition de Bitcoin, notamment sur le forum Bitcointalk. N'étant
probablement pas un cryptographe universitaire, Satoshi Nakamoto s'est
surtout focalisé sur la robustesse du système lorsqu'il l'a conçu et n'a
pas cherché à y inclure des techniques avancées. Cependant, il était
ouvert à toutes les propositions qui permettraient de créer une «~mise
en œuvre de Bitcoin bien meilleure, plus facile et plus
pratique\footnote{Satoshi Nakamoto, \emph{Re: Not a suggestion},
  /08/2010 00:14:22 UTC~:
  \url{https://bitcointalk.org/index.php?topic=770.msg8637\#msg8637}.}~».

La première technique de cet ordre est le procédé de signature de cercle
(\emph{ring signature}), qui a été formalisé en 2001 par Ronald Rivest,
Adi Shamir et Yael Tauman\footnote{«~formalisé en 2001 par Ronald
  Rivest, Adi Shamir et Yael Tauman~»~: Ronald L. Rivest, Adi Shamir,
  Yael Tauman, «~\emph{How to Leak a Secret}~», \emph{Advances in
  Cryptology --- ASIACRYPT 2001}, 2001, pp.~552--565~:
  \url{https://people.csail.mit.edu/rivest/pubs/RST01.pdf}.}. Celui-ci
se base sur le procédé de signature de groupe\footnote{«~signature de
  groupe~»~: Satoshi faisait référence à ces signatures de groupe dans
  l'un de ses messages écrits sur le forum en 2010. -- Satoshi Nakamoto,
  \emph{Re: Not a suggestion}, /08/2010 19:28:47 UTC~:
  \url{https://bitcointalk.org/index.php?topic=770.msg9074\#msg9074}.},
introduit par David Chaum et Eugène van Heyst en 1991, qui permettait à
chaque membre d'un groupe de signer un message au nom du groupe sans que
ce membre puisse être identifié par un vérificateur externe, mais qui
reposait sur un administrateur central. La signature de cercle innovait
par le fait qu'elle ne requérait pas d'administrateur, pas de procédure
d'installation, pas de coordination, et qu'elle ne permettait pas à un
membre de révoquer son anonymat.

En ce qui concerne la cryptomonnaie, le principe est le suivant~: pour
chaque pièce en entrée de la transaction, le signataire rassemble
plusieurs autres pièces disponibles sur la chaîne (appelées sorties
leurres ou \emph{decoy outputs}), utilise leurs clés publiques et signe
avec sa clé privée. Il fournit également une image de clé (\emph{key
image}) correspondant à la pièce, qui est écrite sur la chaîne et qui
permet de garantir que la même pièce n'est pas dépensée deux fois. Plus
le cercle implique de sorties, plus l'ensemble d'anonymat est grand. Le
compromis est que l'utilisation des sorties transactionnelles en tant
que leurres oblige les nœuds à conserver l'ensemble de ces sorties,
puisqu'on ne peut pas savoir laquelle a été réellement dépensée.

La deuxième technique est le procédé des adresses furtives, qui a été
décrit en 2011 par Nicolas van Saberhagen et qui a été formalisé en 2014
par Peter Todd dans le cadre de Bitcoin\footnote{Nicolas van Saberhagen
  (ByteCoin), \emph{Untraceable transactions which can contain a secure
  message are inevitable}, /04/2011, 02:34:24 UTC~:
  \url{https://bitcointalk.org/index.php?topic=5965.msg87757\#msg87757}~;
  Peter Todd, \emph{{[}Bitcoin-development{]} Stealth Addresses},
  /01/2014 12:03:38 UTC~:
  \url{https://lists.linuxfoundation.org/pipermail/bitcoin-dev/2014-January/004020.html}.}.
Il utilise essentiellement le schéma d'échange de clés Diffie-Hellman
basé sur les courbes elliptiques, abrégé en ECDH, afin de permettre de
générer des adresses à usage unique.

Le fonctionnement de base est le suivant. Le destinataire génère une clé
privée et en déduit une clé publique qu'il transmet sous la forme d'une
méta-adresse. L'expéditeur génère une clé privée éphémère, appelée clé
privée de la transaction, et calcule la clé publique correspondante. Ils
peuvent calculer un secret partagé à partir de leur clé privée et de la
clé publique de l'autre (ECDH). L'expéditeur utilise ce secret et la clé
publique du destinataire pour construire une adresse à usage unique et y
envoie les fonds, que seul le destinataire peut dépenser sous condition
de connaître la clé publique de transaction (qui peut être stockée dans
une sortie NULLDATA). Au lieu d'utiliser une seule paire de clés, le
destinataire peut également en utiliser deux pour qu'elles aient des
rôles séparés~: les clés d'inspection (\emph{view keys}) et les clés de
dépense (\emph{spend key}). La clé privée d'inspection est le seul
élément non public qui intervient dans la construction de l'adresse côté
destinataire et sert donc à identifier les sorties correspondant à
l'adresse en question. La clé privée de dépense est celle qui sert,
comme son nom l'indique clairement, à dépenser les fonds\footnote{En
  termes mathématiques, si on note \(r\) et \(R\) les clés éphémères de
  transaction, \(v\) et \(V\) les clés d'inspection et \(k\) et \(K\)
  les clés de dépense, alors la méta-adresse est \(M = (K, V)\), le
  secret partagé est~:}.

Si elle est implémentée de manière externe au protocole, cette méthode a
l'inconvénient d'exiger de balayer l'entièreté de la chaîne de blocs
pour savoir si on a reçu un paiement. C'est dans l'idée d'éviter cette
charge que le BIP-47 a été proposé.

Le BIP-47 formalise ainsi une autre méthode apparentée aux adresses
furtives, plus complexe, qui est celle des codes de paiement
réutilisables (\emph{reusable payment codes}) et qui a été implémentée
sous la forme des PayNyms dans les portefeuilles Samourai et
Sparrow\footnote{«~BIP-47~»~: Justus Ranvier, \emph{Reusable Payment
  Codes for Hierarchical Deterministic Wallets}, 24 avril 2015~:
  \url{https://github.com/bitcoin/bips/blob/master/bip-0047.mediawiki}.}.
Une autre méthode apparentée et plus complexe est celle des codes de
paiement réutilisables (\emph{reusable payment codes}) formalisés par
Justus Ranvier dans le BIP-47, qui a été implémentée sous la forme des
PayNyms dans les portefeuilles Samourai et Sparrow. Dans ce procédé, les
codes de paiement de deux participants permettent de dériver les
adresses de réception grâce à la dérivation de clés. Cela implique qu'il
faut qu'ils connaissent leurs codes de paiement respectifs, et qu'au
moins l'un de ces deux codes reste secret. Le code de paiement du
destinataire est généralement public, de sorte que c'est celui de
l'expéditeur qui doit être caché. Ce dernier est transmis de manière
chiffrée sous la forme d'une transaction de notification envoyée à
l'adresse du destinataire. Ce schéma a donc pour gros défaut d'exiger la
réalisation d'une transaction (et le paiement des frais lié) pour
ajouter un destinataire possible.

Une dernière variante est le procédé des paiements silencieux
(\emph{silent payments}), proposé en 2022 par Ruben Somsen\footnote{Ruben
  Somsen, \emph{Silent Payments}, 13 mars 2022~:
  \url{https://gist.github.com/RubenSomsen/c43b79517e7cb701ebf77eec6dbb46b8}.},
qui évite la charge de la notification en utilisant la clé publique de
l'une des entrées de la transaction, et réduit la charge du balayage de
la chaîne, en se limitant à l'ensemble des UTXO ou à un sous-ensemble
comme les sorties P2TR par exemple.

La technique des signatures de cercle et le procédé des adresses
furtives ont été combinés en 2013 dans le concept de cryptomonnaie
CryptoNote par Nicolas van Saberhagen\footnote{Nicolas van Saberhagen,
  \emph{CryptoNote v2.0}, 17 octobre 2013~:
  \url{http://cryptonote.org/whitepaper.pdf}~; archive~:
  \url{https://web.archive.org/web/20140529235502/http://cryptonote.org/whitepaper.pdf}.}.
Dans celui-ci, les nœuds ont besoin de conserver l'ensemble des sorties
transactionnelles (car le procédé des signatures de cercle dissimule le
fait qu'une sortie a été dépensée) et chaque portefeuille a besoin de
balayer l'ensemble de ces sorties pour voir s'il a reçu un paiement.
L'intégration des \emph{stealth addresses} au protocole permet de
publier la clé publique éphémère directement dans la transaction (ce qui
en fait une clé de transaction) et d'éviter la nécessité de
notification. Le concept a été implémenté initialement dans le très
douteux Bytecoin en mars 2014, avant de se retrouver dans Monero en
avril de la même année, qui en est aujourd'hui le représentant
principal, mettant notamment en œuvre des signatures de cercle à 16
membres.

La troisième technique d'amélioration de la confidentialité est le
procédé des \emph{Confidential Transactions}, qui permet de dissimuler
les montants impliqués dans les échanges des utilisateurs, et qui en
toute logique devrait plutôt s'appeler \emph{Confidential Amounts}. Le
procédé a été décrit par Adam Back en 2013 et a été formalisé par
Gregory Maxwell en 2015\footnote{Adam Back, \emph{bitcoins with
  homomorphic value (validatable but encrypted)}, October 01, 2013,
  02:19:53 PM~:
  \url{https://bitcointalk.org/index.php?topic=305791.msg3277431\#msg3277431}~;
  Gregory Maxwell, \emph{Confidential Transactions}, 2015~:
  \url{https://web.archive.org/web/20150628230410/https://people.xiph.org/~greg/confidential_values.txt}.}.
Il impose à chaque sortie transactionnelle de contenir un engagement de
Perdersen (\emph{Pedersen commitment}) qui lie la pièce à la clé
publique du destinataire sans la dévoiler, et une preuve de portée
(\emph{range proof}) qui est une preuve à divulgation nulle de
connaissance (ZKP) démontrant la validité du montant sans le révéler.

Les Confidential Transactions ont été ajoutées en 2017 à Monero grâce au
travail de Shen Noether\footnote{«~grâce au travail de Shen Noether~»~:
  Shen Noether, \emph{Ring Confidential Transactions}, 2015~:
  \url{https://eprint.iacr.org/2015/1098.pdf}.}. RingCT, qui permet de
cacher les montants échangés, a ainsi été ajouté au protocole en janvier
2017 et a été rendu obligatoire en septembre de la même année. Il
alourdissait les transactions par rapport aux transactions classiques.
Néanmoins, depuis octobre 2018, ce compromis a été atténué grâce à
l'implémentation des bulletproofs, qui a allégé le fardeau des preuves
de portée et qui a permis de réduire de 80~\% la taille des
transactions\footnote{Benedikt Bünz, Jonathan Bootle, Dan Boneh, Andrew
  Poelstra, Pieter Wuille, Gregory Maxwell, \emph{Bulletproofs: Short
  Proofs for Confidential Transactions and More}, 2018~:
  \url{https://eprint.iacr.org/2017/1066.pdf}.}.

Un autre concept faisant usage des Confidential Transactions est
Mimblewimble\footnote{«~Mimblewimble~»~: Le nom du concept et le
  pseudonyme du créateur sont issus de l'univers d'Harry Potter~:
  Mimblewimble est la formule du sortilège à la langue de Plomb qui
  interdit à l'adversaire de parler en faisant des nœuds avec sa langue
  (Mimblewimble est censé «~empêcher la chaîne de blocs de parler des
  informations personnelles de ses utilisateurs~») et Tom Elvis Jedusor
  est le vrai nom de Voldemort dans la traduction française.}. Celui-ci
a été proposé le 1 août 2016 par un inconnu se faisant appeler Tom Elvis
Jedusor au sein du canal IRC ``\footnote{«~proposé le 1 août 2016 par un
  inconnu se faisant appeler Tom Elvis Jedusor au sein du canal IRC
  \#bitcoin-wizards~»~: Bitcoin-wizards logs, 1 août 2016~:
  \url{https://gnusha.org/bitcoin-wizards/2016-08-01.log}. Lien
  originel~: \url{http://5pdcbgndmprm4wud.onion/mimblewimble.txt}.} où
il partageait un lien vers un texte descriptif hébergé sur
Tor\footnote{Tom Elvis Jedusor, \emph{Mimblewimble}, 19 juillet 2016,
  archive~:
  \url{https://download.wpsoftware.net/bitcoin/wizardry/mimblewimble.txt}.}.
Mimblewimble a attiré l'attention de certains développeurs de Bitcoin,
dont le mathématicien Andrew Poelstra qui en a fait une description plus
avancée dans un papier daté du 6 octobre 2016\footnote{Andrew Poelstra,
  \emph{Mimblewimble}, 6 octobre 2016~:
  \url{https://download.wpsoftware.net/bitcoin/wizardry/mimblewimble.pdf}.}.

L'apport de Mimblewimble est de condenser l'historique des transactions
en chamboulant la structure des transactions. Il repose sur trois
primitives cryptographiques~: les Confidential Transactions, qui cachent
les montants, les signatures agrégées à sens unique (OWAS\footnote{«~OWAS~»~:
  Horas Yuan Mouton, \emph{Increasing Anonymity in Bitcoin}, 9 septembre
  2013~:
  \url{https://www.dropbox.com/s/nkh22cibel8stb4/horasyuanmouton.pdf}~;
  archive~:
  \url{https://download.wpsoftware.net/bitcoin/wizardry/horasyuanmouton-owas.pdf}.}),
qui permettent de combiner les transactions au sein d'un bloc, et le
sectionnage des transactions\footnote{«~sectionnage des transactions~»~:
  Gregory Maxwell, \emph{Transaction cut-through}, /08/2013 22:17:19
  UTC~:
  \url{https://bitcointalk.org/index.php?topic=281848.msg3014613\#msg3014613}.}
(\emph{transaction cut-through}), qui permet de supprimer les sorties
transactionnelles intermédiaires. Cette réduction, qui améliore la
confidentialité du système de manière relativement légère, se fait au
prix de la programmabilité, rendue directement impossible.

Mimblewimble a été mis en œuvre de manière native au sein du système
Grin développé par Ignotus Peverell à partir d'octobre 2016 et lancé le
15 janvier 2019. Une autre implémentation, également lancée en janvier
2019, était le réseau Beam. Mimblewimble a également été intégré à
Litecoin le 20 mai 2022 sous la forme d'un soft fork de bloc auxiliaire,
appelé MWEB pour \emph{MimbleWimble via Extension Blocks}.

Enfin, il existe d'autres techniques d'anonymisation basées sur des
preuves à divulgation nulle de connaissance. Les plus connues ont été
popularisées au moyen de deux protocoles rendus publics en 2013 et en
2014 par Matthew Green et ses étudiants~: Zerocoin et
Zerocash\footnote{Ian Miers, Christina Garman, Matthew Green, Aviel D.
  Rubin, «~\emph{Zerocoin: Anonymous Distributed E-Cash from Bitcoin}~»,
  in \emph{2013 IEEE Symposium on Security and Privacy}, 2013,
  pp.~397--411~: \url{https://ieeexplore.ieee.org/document/6547123}~;
  Eli Ben Sasson, Alessandro Chiesa, Christina Garman, Matthew Green,
  Ian Miers, Eran Tromer, Madars Virza, «~\emph{Zerocash: Decentralized
  Anonymous Payments from Bitcoin}~», \emph{2014 IEEE Symposium on
  Security and Privacy}, 2014, pp.~459--474~:
  \url{https://ieeexplore.ieee.org/document/6956581}.}. Le premier
protocole, Zerocoin, permet de cacher la provenance des fonds. Le second
protocole permet de cacher la provenance, la destination et les
montants, au moyen de zk-SNARK (\emph{Zero-Knowledge Succinct
Non-Interactive Arguments of Knowledge}\footnote{«~\emph{Zero-Knowledge
  Succinct Non-Interactive Arguments of Knowledge}~»~: On peut traduire
  ce terme par «~argument de connaissance succinct et non interactif à
  divulgation nulle de connaissance~» en français.}).

Zerocoin a été implémenté dans Zcoin en septembre 2016. À partir de
2019, Zcoin s'est progressivement éloigné du Zerocoin en adoptant les
protocoles Sigma et Lelantus, et est devenu Firo en 2020. Zerocash a lui
été implémenté au sein du système Zcash en octobre 2016. L'utilisation
de preuves à divulgation nulle de connaissance demandait une
configuration de confiance des paramètres publics. Tandis que les
développeurs de Zcoin qui ont fait le choix d'utiliser des paramètres
connus, ceux de Zcash ont décidé d'organiser un évènement, appelée
«~\emph{The Ceremony}~», dans le but de générer ces paramètres. Cette
cérémonie a eu lieu du 21 au 23 octobre 2016 et a réuni six
participants~: Andrew Miller, Peter Van Valkenburgh, Zooko
Wilcox-O'Hearn, Derek Hinch, Peter Todd et surtout Edward Snowden sous
le pseudonyme de John Dobbertin\footnote{Zooko Wilcox-O'Hearn, \emph{The
  Design of the Ceremony}, 26 octobre 2016~:
  \url{https://electriccoin.co/blog/the-design-of-the-ceremony/}.}.
Cette configuration de confiance a été rendu inutile en 2022 avec
l'intégration du protocole Halo.

De manière générale, tous ces procédés supposent des compromis au niveau
de la scalabilité (les preuves sont plus lourdes qu'une simple
signature), au niveau de l'auditabilité (ne pas voir les montants
implique de devoir faire entièrement confiance aux procédés et à leur
implémentation) et au niveau de la programmabilité (programmer des
pièces s'oppose au fait de les rendre indistinctes). C'est pourquoi ils
ont tous été mis en œuvre dans des versions alternatives de Bitcoin et
pas dans sa version principale (BTC), la communauté de cette dernière
étant plus conservatrice par nature.

\section*{Une machine complexe}\label{une-machine-complexe}
\addcontentsline{toc}{section}{Une machine complexe}

\markright{Une machine complexe}

Bitcoin forme ainsi une machine qui peut sembler, à première vue, assez
complexe. Cet enchevêtrement s'explique par ses objectifs et par les
évènements qui ont jalonné son histoire technique. Son but premier --
être une monnaie -- est à l'origine de la représentation des bitcoins en
circulation par des sorties transactionnelles non dépensées, une
représentation qui rend la parallélisation plus facile et favorise la
confidentialité des échanges (pouvant elle-même être accrue par le
mélange des pièces et les techniques cryptographiques dédiées).

De plus, la volonté de Satoshi d'automatiser divers mécanismes lui a
fait intégrer un véritable système de programmation au sein du
protocole. Celui-ci permet de mettre en place des contrats autonomes qui
exécutent des interactions financières complexes entre plusieurs
participants. Il facilite aussi, indirectement, l'inscription de données
arbitraires sur la chaîne. Ces deux utilisations (contractuelle et
notariale) forment les deux cas d'usage secondaires de Bitcoin, dont
nous parlerons dans le prochain chapitre.

\bookmarksetup{startatroot}

\chapter{Les contrats autonomes}\label{ch:contrats}

\phantomsection\label{enotezch:13}{}

{U}\textsc{n} contrat autonome, de l'anglais \emph{smart contract}, est
un programme informatique dont l'exécution ne nécessite pas
l'intervention d'un tiers de confiance. On parle aussi de contrat
auto-exécutable ou de contrat intelligent (traduction littérale). Chaque
contrat est constitué de clauses qui sont des conditions de dépense
spécifiques.

Bitcoin constitue la première implémentation concrète d'un système
hébergeant des contrats autonomes, par le biais de son système de
programmation interne qui met en œuvre des scripts au sein des
transactions. Il permet d'exécuter une variété de contrats allant du
compte multisignatures au canal de paiement, en passant par le dépôt
fiduciaire. L'ouverture apportée par cette possibilité facilite
l'inscription de données arbitraires sur la chaîne, un cas d'utilisation
strictement non monétaire du protocole.

\section*{Les contrats simples}\label{les-contrats-simples}
\addcontentsline{toc}{section}{Les contrats simples}

\markright{Les contrats simples}

La notion de contrat autonome\footnote{L'appellation «~contrat
  autonome~» visant à traduire \emph{smart contract} a été proposée par
  Jacques Favier, Adli Takkal-Bataille et Benoît Huguet dans
  \emph{Bitcoin~: Métamorphoses} (pp.~105--107) en 2018.} a germé au
sein du mouvement cypherpunk dans les années 1990. Elle a été exposée
par Nick Szabo en 1994, qui la définissait comme suit~:

«~Un contrat autonome est un protocole de transaction informatisé qui
exécute les termes d'un contrat. Les objectifs généraux de la conception
de contrats autonomes sont de satisfaire les conditions contractuelles
courantes (telles que les conditions de paiement, les privilèges, la
confidentialité et même l'exécution), de minimiser les exceptions, tant
malveillantes qu'accidentelles, et de minimiser le besoin
d'intermédiaires de confiance\footnote{Nick Szabo, \emph{Smart
  Contracts}, 1994, archive~:
  \url{https://web.archive.org/web/20011102030833/http://szabo.best.vwh.net:80/smart.contracts.html}.}.~»

Le transfert de valeur constitue le cas le plus simple de contrat
autonome, ne contenant qu'une seule clause~: la fourniture d'une
signature numérique correspondant à une clé publique donnée. Mais une
multitude d'autres contrats peuvent être implémentés sur Bitcoin, à tel
point qu'il est impossible d'en dresser une liste exhaustive. Nous nous
contenterons ici d'en décrire quelques exemples pour expliquer comment
ils peuvent être mis en place. Voyons d'abord les cas spécifiques du
compte multisignatures, du dépôt fiduciaire, du financement participatif
et de l'échange atomique.

\subsection{Le compte multisignatures}\label{le-compte-multisignatures}

Le compte multisignatures est un compte partagé entre plusieurs entités.
Il se base sur le schéma de signature multipartite décrit dans le
chapitre~\hyperref[ch:rouages]{12}, dans lequel la dépense des fonds
demande M signatures parmi N participants (ce qu'on appelle
«~M-parmi-N~» ou «~M-of-N~» en anglais). Par exemple, la dépense depuis
un compte 2-parmi-3 exige que 2 personnes parmi 3 participants
prédéterminés produisent une signature valide, peu importe l'identité
précise de ces personnes.

Ce type de contrat est utile pour avoir un compte joint entre époux (2
parmi 2), pour faciliter la détention par une entreprise (3 associés
parmi 7 par exemple) ou pour améliorer la conservation de bitcoins en
général. Les plateformes d'échange utilisent notamment ce type de
contrat pour conserver leurs avoirs. En novembre 2023, la deuxième
adresse la plus riche du monde en 2023 était ainsi l'adresse
multisignatures 3-parmi-5 de Bitfinex contenant plus de
178~000~BTC\footnote{L'adresse multisignatures 3-parmi-5 de Bitfinex est
  ``.}.

\subsection{Le dépôt fiduciaire}\label{le-duxe9puxf4t-fiduciaire}

Le dépôt fiduciaire, appelé \emph{escrow} en anglais, est une méthode
basée sur le recours à un tiers de confiance, comme un notaire, pour
sécuriser une transaction entre deux parties qui se méfient l'une de
l'autre. L'utilisation de la programmabilité de Bitcoin permet de
diminuer le pouvoir du tiers en incluant une limite dans la clause qui
le concerne. Ce type de contrat repose sur deux briques techniques de
base~: la signature multipartite et les verrous temporels.

Prenons l'exemple de deux personnes qui ne se connaissent pas, Alice et
Bob, et qui veulent réaliser une transaction en ligne\footnote{«~l'exemple
  de deux personnes qui ne se connaissent pas, Alice et Bob, et qui
  veulent réaliser une transaction en ligne~»~: Une description de ce
  contrat est faite dans le BIP-65~:
  \url{https://github.com/bitcoin/bips/blob/master/bip-0065.mediawiki}.}~:
Alice est l'acheteuse, Bob le vendeur. Les deux parties font appel à un
intermédiaire de confiance, Lenny, avec qui elles créent le contrat de
dépôt fiduciaire. Alice y envoie les fonds et attend de recevoir le
bien. Deux clauses peuvent alors être activées~:

\begin{itemize}
\item
  Le règlement à l'amiable~: le contrat est déverrouillé par les
  signatures des deux parties, qui peuvent choisir d'envoyer les fonds
  vers Bob (réussite de l'échange) ou bien de rembourser Alice (échec de
  l'échange)~;
\item
  Le litige~: après une période prédéterminée (par exemple 30 jours), le
  contrat est déverrouillé par la signature de Lenny et celle de l'une
  des deux parties~; dans ce cas, Lenny se charge de déterminer qui est
  la partie honnête et de lui envoyer les fonds.
\end{itemize}

\begin{figure}

{\centering \includegraphics{chapters/img/escrow-contract.png}

}

\caption{Contrat de dépôt fiduciaire.}

\end{figure}%

Ce fonctionnement, décrit sur la
figure~\hyperref[fig:escrow-contract]{13.1}, incite d'une part les deux
parties à coopérer pour ne pas perdre de temps, et empêche d'autre part
la collusion de la tierce partie (Lenny) avec l'une des deux autres
avant le délai prévu (30 jours ici). Le recours à la confiance est ainsi
minimisé autant que possible.

Ce type de contrat était soutenu par Satoshi Nakamoto dans le livre
blanc\footnote{«~Les acheteurs pourraient être facilement protégés par
  la mise en œuvre de mécanismes de dépôt fiduciaire routiniers.~» --
  Satoshi Nakamoto, \emph{Bitcoin: A Peer-to-Peer Electronic Cash
  System}, 31 octobre 2008.}. En effet, l'irréversibilité des transferts
dans Bitcoin offrait peu de garantie pour les commerçants, et le dépôt
fiduciaire permettait d'atténuer le problème. C'est typiquement ce genre
de mécanisme qui intervient aujourd'hui dans les plateformes de change
de pair à pair comme Bisq ou Hodl Hodl, même si l'implémentation diffère
de ce qui est présenté ici.

\subsection{Le financement
participatif}\label{le-financement-participatif}

Le financement participatif consiste à faire appel au grand public pour
contribuer au soutien d'un projet, par opposition au financement par
prêt bancaire ou par levée de fonds auprès des professionnels du
capital-risque. Il s'agit le plus souvent d'un accord informel entre le
promoteur du projet et le public ayant pour but de soutenir la création
d'un bien commun, qui profite à tous. Dans Bitcoin, il est possible
d'exécuter cet accord par le biais de promesses de paiement résiliables
qui ne sont pas soumises à l'arbitraire d'un tiers de confiance.

D'un point de vue technique, il s'agit de créer une transaction dite
\emph{anyone-can-pay} («~tout le monde peut payer~») où la signature de
chaque contributeur ne prend en compte que la sortie transactionnelle de
la levée de fonds et l'entrée du contributeur en question, donnant la
possibilité d'ajouter des entrées (voir
figure~\hyperref[fig:sighash-anyonecanpay]{13.2}). La transaction
résultante n'est valide que si le montant en entrée atteint le montant
indiqué en sortie, de sorte que les contributeurs conservent le contrôle
de leurs fonds jusqu'à la réalisation du paiement total et peuvent se
retirer à tout moment.

\begin{figure}

{\centering \includegraphics{chapters/img/sighash-anyonecanpay.png}

}

\caption{Transaction de financement participatif.}

\end{figure}%

Dans le monde du logiciel libre, ce type de financement participatif est
particulièrement important, car il n'y a pas de privilège lié à
l'écriture du code qui permette de gagner sa vie par la vente de
licences. C'est encore plus vrai dans le monde de la cryptomonnaie qui
dépend fortement du bon maintien des implémentations logicielles. C'est
pourquoi Mike Hearn, qui s'intéressait de près aux capacités de
programmation de Bitcoin, s'est vite approprié cette possibilité pour
déployer de tels «~contrats de garantie\footnote{«~contrats de
  garantie~»~: Mike Hearn, \emph{Bitcoin Wiki: Contracts}, 23 juin
  2011~:
  \url{https://en.bitcoin.it/wiki/Contract\#Example_3:_Assurance_contracts}~:
  «~Un contrat de garantie est une manière de financer la création d'un
  bien public, c'est-à-dire d'un bien qui, une fois créé, bénéficie à
  tous gratuitement. L'exemple typique est celui d'un phare~: bien que
  tout le monde puisse être d'accord sur le fait qu'il doit être
  construit, c'est bien trop cher pour justifier qu'un marin individuel
  en construise un, étant donné qu'il bénéficiera à tous ses
  concurrents. Une solution est que tout le monde promette de payer pour
  la création du bien public, de sorte à ce que les promesses soient
  appliquées seulement si la valeur totale des promesses dépasse le coût
  de création. Si le nombre de personnes qui contribuent n'est pas assez
  élevé, personne ne doit payer quoi que ce soit.~»}~» (\emph{assurance
contracts}) permettant de financer les biens publics. Il a mis le
concept en œuvre au sein de son application Lighthouse, dont une version
fonctionnelle est sortie en 2015, qui avait pour but de faciliter le
soutien communautaire des projets de l'écosystème. Avec le déclenchement
de la guerre des blocs, ce projet a été mis de côté par Hearn et a fini
par être abandonné. Le procédé a été néanmoins repris sur Bitcoin Cash
en 2020 par l'intermédiaire de Flipstarter, qui a permis de lever
d'importantes sommes pour le financement de l'infrastructure logicielle
du protocole\footnote{«~Flipstarter~»~: Ludovic Lars, \emph{Flipstarter,
  le financement participatif pour Bitcoin Cash}, 24 avril 2020~:
  \url{https://viresinnumeris.fr/flipstarter-financement-participatif-bitcoin-cash/}.}.

\subsection{L'échange atomique}\label{luxe9change-atomique}

L'échange atomique (\emph{atomic swap}) est une manière sûre d'échanger
deux cryptomonnaies fonctionnant sur des chaînes de blocs différentes,
sans passer par un intermédiaire de confiance. L'adjectif «~atomique~»
se rapporte à la nature insécable (en grec ancien
\foreignlanguage{greek}{ἄtomos}, átomos) de l'échange~: soit les deux
parties transfèrent leur dû, soit il ne se passe rien. Le concept a été
décrit par Sergio Lerner et Gregory Maxwell en juillet 2012 sur le forum
Bitcointalk\footnote{Sergio Demian Lerner, \emph{P2PTradeX: P2P Trading
  between cryptocurrencies}, /07/2012 23:49:48 UTC~:
  \url{https://bitcointalk.org/index.php?topic=91843.msg1011737\#msg1011737}~;
  Gregory Maxwell, \emph{Re: P2PTradeX: P2P Trading between
  cryptocurrencies}, /07/2012 02:17:02 UTC~:
  \url{https://bitcointalk.org/index.php?topic=91843.msg1011956\#msg1011956}.}.

L'échange atomique repose sur le concept de contrat verrouillé par une
empreinte et par un temps, appelé HTLC par abréviation du terme anglais
\emph{Hash Time Locked Contract}. Celui-ci est un contrat à deux
clauses, c'est-à-dire que les fonds peuvent être déverrouillés à deux
conditions\footnote{Pour assurer la bonne exécution du contrat (éviter
  le remplacement de la transaction durant l'attente de confirmation),
  des clés publiques sont assignées à chacune de ces conditions de sorte
  qu'une signature est systématiquement demandée au destinataire des
  fonds.}~:

\begin{itemize}
\item
  L'accord mutuel~: la révélation d'un secret qui est haché par une
  fonction de hachage et comparé à l'empreinte (\emph{hash}) inscrite
  dans le contrat~;
\item
  Le litige~: l'attente d'un certain temps (\emph{time}) de verrouillage
  déterminé dans le contrat.
\end{itemize}

Considérons l'exemple d'un échange atomique entre Alice, qui possède du
BTC, et Bob, qui possède du LTC. Alice (\emph{maker}) propose d'échanger
0,03 BTC pour 10~LTC, à un taux de change de 0,003~LTC par BTC, et Bob
(\emph{taker}) accepte cet échange. Cette négociation peut avoir lieu
par le biais d'un carnet d'ordres public ou privé. Alice choisit au
hasard un secret (noté \(s\)), qui est un nombre de 32 octets, dont elle
fournit l'empreinte cryptographique \(H(s)\) à Bob. Ils peuvent ainsi
construire un contrat chacun de leur côté pour effectuer l'échange
atomique. Son déroulé est décrit au sein de la
figure~\hyperref[fig:atomic-swap-contract]{13.3}.

La première phase est la phase d'engagement. D'abord, Alice construit,
signe et diffuse une transaction d'engagement envoyant 0,03 BTC vers le
contrat d'échange atomique sur la chaîne de Bitcoin. Elle fournit son
contenu et son adresse à Bob pour qu'il en vérifie la validité. Puis,
elle construit et signe une transaction de remboursement dépensant les
fonds de ce contrat qu'elle pourra diffuser après un délai prédéfini
(ici 16 heures). Ensuite, une fois que la transaction d'engagement
d'Alice a été confirmée, Bob fait de même de son côté~: il crée un
contrat équivalent sur la chaîne de Litecoin, où il envoie 10~LTC, et en
donne le contenu et l'adresse à Alice pour qu'elle s'assure que tout est
en ordre. Enfin, il construit et signe une transaction qui le
remboursera au bout d'un délai strictement inférieur à celui de la
transaction d'Alice~: ici 8 heures. Cette différence résulte du rapport
déséquilibré qui existe entre Alice (qui connaît le secret de
déverrouillage) et Bob (qui ne le connaît pas).

Lorsque les transactions d'engagement ont toutes deux été confirmées sur
leurs chaînes respectives, la seconde phase de l'échange atomique, la
phase de collecte, peut commencer. Alice construit, signe et diffuse une
transaction de collecte qui lui permet de récupérer les 10~LTC de Bob.
Pour cela, elle fournit le secret au sein de la transaction et, ce
faisant, le révèle nécessairement à Bob. Finalement, Bob peut lui-aussi
construire, signer et diffuser une transaction qui lui octroie les 0,03
BTC sur son compte. De cette manière, l'échange est clos~!

\begin{figure}

{\centering \includegraphics{chapters/img/atomic-swap-contract.png}

}

\caption{Contrats et transactions dans un échange atomique.}

\end{figure}%

Ce modèle garantit qu'aucun des deux participants ne peut se rembourser
avant la fin du temps de verrouillage de Bob (8 heures)~; qu'Alice ne
peut pas faire valoir sa transaction de remboursement au moment de la
diffusion de sa transaction de collecte~; et que Bob ne peut pas
s'approprier des fonds d'Alice tant qu'elle n'a pas diffusé sa
transaction de collecte. Ces garanties rendent le procédé logiquement
sécurisé, même si certains évènements perturbateurs peuvent survenir
comme une augmentation des temps de confirmation liée à la volatilité du
marché des frais.

Le premier \emph{atomic swap} réel a été réalisé entre Litecoin et
Decred le 19 septembre 2017 par Charlie Lee et Alex
Yocom-Piatt\footnote{Les adresses des contrats sur LTC et DCR étaient
  (respectivement) \texttt{et}. L'échange était de 1,337 LTC contre
  2,4066 DCR. -- \emph{Decred-compatible cross-chain atomic swapping},
  20 septembre 2017~:
  \url{https://github.com/decred/atomicswap/blob/master/README.md\#first-mainnet-dcr-ltc-atomic-swap}.}.
Aujourd'hui, les échanges atomiques sont rares, les carnets d'ordres de
plateformes spécialisées comme AtomicDEX étant très peu fournis.
Toutefois, avec le durcissement réglementaire sévissant dans
l'écosystème et rendant les plateformes centralisées moins fiables, il
n'est pas exclus qu'ils jouent un rôle majeur à l'avenir.

\section*{Les canaux de paiement}\label{les-canaux-de-paiement}
\addcontentsline{toc}{section}{Les canaux de paiement}

\markright{Les canaux de paiement}

Un cas particulier de l'application des contrats autonomes dans Bitcoin
est le déploiement de canaux de paiement. Un canal de paiement est une
manière pour deux utilisateurs d'effectuer des paiements répétés en
bitcoins de manière sûre et instantanée sans publier de transactions sur
la chaîne de blocs à partir de liquidités préalablement bloquées. Ces
canaux sont notamment à la base du réseau Lightning, construit en
surcouche de la chaîne.

\subsection{Les canaux de paiement de
Poon-Dryja}\label{les-canaux-de-paiement-de-poon-dryja}

Même si l'idée d'un canal de paiement était envisagée dès les
origines\footnote{«~l'idée d'un canal de paiement était envisagée dès
  les origines~»~: Satoshi Nakamoto, \emph{Re: Open sourced my Java SPV
  impl}, /03/2011 16:15 UTC~:
  \url{https://plan99.net/~mike/satoshi-emails/thread4.html}~; hashcoin,
  \emph{Instant TX for established business relationships (need
  replacements/nLockTime)}, /07/2011 02:16:23 UTC~:
  \url{https://bitcointalk.org/index.php?topic=25786.msg320931\#msg320931}~;
  Meni Rosenfeld, \emph{Trustless, instant, off-the-chain Bitcoin
  payments}, /07/2012 13:37:19 UTC~:
  \url{https://bitcointalk.org/index.php?topic=91732.msg1010405\#msg1010405}~;
  Jeremy Spliman, \emph{{[}Bitcoin-development{]} Anti DoS for tx
  replacement}, /04/2013 01:48:11 UTC~:
  \url{https://lists.linuxfoundation.org/pipermail/bitcoin-dev/2013-April/002433.html}~;
  Alex Akselrod, \emph{Bitcoin Wiki: Draft}, 12 mars 2013,
  \url{https://en.bitcoin.it/wiki/User:Aakselrod/Draft}~; Christian
  Decker, Roger Wattenhofer, \emph{A Fast and Scalable Payment Network
  with Bitcoin Duplex Micropayment Channels}, août 2015~:
  \url{https://www.researchgate.net/publication/277991245_A_Fast_and_Scalable_Payment_Network_with_Bitcoin_Duplex_Micropayment_Channels}.},
elle ne s'est concrétisée qu'avec le concept élaboré par Joseph Poon et
Thaddeus Dryja dans le cadre de leur projet du réseau
Lightning\footnote{Joseph Poon et Thaddeus Dryja, \emph{The Bitcoin
  Lightning Network DRAFT Version 0.5}, 28 février 2015~:
  \url{https://lightning.network/lightning-network-paper-DRAFT-0.5.pdf}.}.
Il s'agit d'un concept de canal bidirectionnel dont la sécurité repose
sur un mécanisme de punition. Les deux participants bloquent des fonds
dans un contrat et peuvent procéder à des paiements l'un vers l'autre
dans la limite des liquidités disponibles. La somme des deux soldes des
participants est appelée la capacité du canal.

Un canal traverse trois phases au cours de son existence~:

\begin{itemize}
\item
  La phase d'ouverture ou d'installation, lors de laquelle les fonds
  sont bloqués par les participants sur un contrat autonome de
  multisignature 2-parmi-2~;
\item
  La phase de négociation ou de mise à jour, durant laquelle la
  répartition des fonds au sein du canal est ajustée~;
\item
  La phase de fermeture ou de règlement, au cours de laquelle les fonds
  sont distribués aux participants sur la chaîne, généralement de
  manière coopérative selon le dernier état du canal.
\end{itemize}

La répartition initiale et la mise à jour du canal se font par
l'intermédiaire de transactions d'engagement qui sont échangées entre
les participants et \emph{qui ne sont pas diffusées} sur le réseau, sauf
dans le cas d'un litige, c'est-à-dire d'une fermeture non coopérative.
Ces transactions d'engagement sont asymétriques, dans le sens où les
participants en possèdent chacun leur propre version.

Supposons qu'Alice et Bob possèdent un canal, tel qu'illustré sur la
figure~\hyperref[fig:poon-dryja-contracts]{13.4}. Dans ce cas, la
dernière transaction d'engagement d'Alice, qui peut uniquement être
finalisée et diffusée par Bob, prend en compte l'état actualisé du canal
et répartit les fonds entre l'adresse d'Alice et un contrat de
réclamation. Ce contrat de réclamation contient deux clauses~:

\begin{itemize}
\item
  La récupération des fonds par Bob au terme d'un temps de verrouillage,
  ce qui répartit les fonds selon les soldes indiqués dans le canal~;
\item
  La récupération des fonds par Alice à l'aide d'une clé de révocation
  qui est révélée plus tard lorsque le canal est de nouveau mis à jour.
\end{itemize}

Si un paiement a lieu d'Alice vers Bob, la mise à jour du canal se fait
de la manière suivante. Alice construit et signe sa transaction
d'engagement en utilisant la clé publique de révocation de Bob que ce
dernier lui a transmise au préalable. Seul Bob peut finaliser la
signature de cette transaction et la diffuser sur le réseau. Bob lui
répond en lui envoyant sa clé privée de révocation, ce qui rend la
dernière transaction d'engagement d'Alice inopérante. La même chose se
produit ensuite de manière symétrique~: Bob construit et signe sa
transaction d'engagement qu'il transmet à Alice, et cette dernière lui
révèle en échange sa clé privée de révocation, ce qui rend la
transaction d'engagement de Bob impuissante\footnote{Andreas M.
  Antonopoulos, Olaoluwa Osuntokun, René Pickhardt, «~Payment
  Channels~», in \emph{Mastering the Lightning Network: A Second Layer
  Blockchain Protocol for Instant Bitcoin Payments}, O'Reilly Media,
  2022, pp.~149--184.}.

La révélation de la clé de révocation à chaque étape de mise à jour rend
possible l'activation d'un mécanisme de punition à tout moment. Si l'une
des deux parties diffuse une transaction d'engagement correspondant à un
état antérieur du canal, alors l'autre peut récupérer
\emph{l'intégralité} des fonds du canal. Par exemple, Alice pourrait
récupérer les fonds de Bob si ce dernier était amené à diffuser le
précédent état du canal dans le but d'«~annuler~» le dernier paiement
réalisé.

\begin{figure}

{\centering \includegraphics{chapters/img/lightning-poon-dryja-channel-contracts.png}

}

\caption{Contrats et transactions dans un canal de paiement de
Poon-Dryja~: cas d'un paiement de Bob de 2~mBTC à Alice.}

\end{figure}%

Le défaut principal de ce mécanisme de punition est qu'il faut
surveiller le réseau en permanence pour éviter un vol, ce qui se fait
avec un nœud complet ou bien avec un tiers de confiance bien choisi
(«~tour de garde~» ou «~\emph{watchtower}~»).

Ce fonctionnement des canaux de Poon-Dryja fait aussi que toute erreur
est très pénalisante~: la diffusion accidentelle d'une transaction
d'engagement antérieure mène à la récupération des fonds par l'autre
partie. Il a également d'autres défauts~: il impose de conserver
l'ensemble des états antérieurs du canal, il oblige les participants à
choisir les frais des transactions à l'avance et il alourdit
considérablement les innovations au sein du réseau Lightning. C'est par
volonté d'améliorer cette situation qu'ont été conceptualisés les canaux
dits «~de Decker-Russell-Osuntokun~».

\subsection{Les canaux de paiement de
Decker-Russell-Osuntokun}\label{les-canaux-de-paiement-de-decker-russell-osuntokun}

Les canaux de paiement de Decker-Russell-Osuntokun ont été décrits par
Christian Decker, Rusty Russell et Olaoluwa Osuntokun dans un livre
blanc publié en avril 2018\footnote{Christian Decker, Rusty Russell,
  Olaoluwa Osuntokun, \emph{eltoo: A Simple Layer2 Protocol for
  Bitcoin}, 30 avril 2018~: \url{https://blockstream.com/eltoo.pdf}.}.
Le protocole sous-jacent est appelé Eltoo, qui est une déformation de
l'anglais «~\emph{L2}~» (signifiant \emph{layer two}).

Le fonctionnement des canaux de Decker-Russel-Osuntokun se base sur une
chaîne de transactions, qui ne sont pas censées être diffusées sur la
chaîne, sauf celles d'ouverture et de fermeture
(cf.~figure~\hyperref[fig:eltoo]{13.5}). Le principe est le suivant~:

\begin{itemize}
\item
  Le canal est ouvert par une transaction d'ouverture (\(T_{u,0}\)),
  préalablement garantie par une transaction de règlement (\(T_{s,0}\))
  qui rembourse les participants en cas de litige~;
\item
  Le canal est mis à jour par des transactions de mise à jour
  (\(T_{u,i}\)) qui invalident les transactions de règlement précédentes
  (\(T_{s,i-1}\))~;
\item
  La fermeture du canal peut se faire après un certain délai
  d'expiration par la diffusion de la dernière transaction de règlement
  (\(T_{s,i}\)).
\end{itemize}

Ici il n'y a plus besoin de recourir à des clés de révocation pour
rendre les anciens états du canal inexploitables~: ce sont les
transactions elles-mêmes qui ont ce rôle. Eltoo fait intervenir ce qu'on
appelle des transactions flottantes, qui peuvent dépenser les fonds
issus de n'importe quelle transaction de mise à jour précédente. De
cette manière, chaque transaction de mise à jour est flottante, ainsi
que chaque transaction de règlement, ce qui permet d'omettre toutes les
mises à jour précédentes. De plus, un numéro d'état est inscrit dans
chaque transaction pour ordonner les transactions et ainsi éviter la
diffusion d'un état antérieur.

\begin{figure}

{\centering \includegraphics{chapters/img/eltoo-offchain-protocol.png}

}

\caption{Aperçu du protocole Eltoo.}

\end{figure}%

Une transaction supplémentaire est ajoutée à la chaîne de transactions
pour éviter que le délai d'expiration des transactions de règlement
\(T_{s,i}\) soit atteint et qu'elles soient diffusées sur la chaîne.
Cette transaction envoie simplement les fonds vers un compte
multisignatures classique, et est signée et diffusée après la signature
des premières transactions de mise à jour et de règlement (\(T_{u,0}\)
et \(T_{s,0}\)). Le délai d'expiration ne commence que lorsque la
transaction \(T_{u,0}\) est diffusée.

Ce fonctionnement permet d'obtenir un protocole simple de mise à jour du
canal, peu contraignant pour les nœuds, sans mécanisme de punition, et
permettant de ne pas à avoir à décider les frais à l'avance. Cette
facilité d'implémentation pourrait rendre plus aisée la création de
contrats plus complexes sur Lightning, comme les canaux de paiement à 3
participants ou plus. En outre, leur implémentation ne doit en aucun cas
remplacer celle des canaux de Poon-Dryja~: les deux modèles peuvent
coexister au sein d'un seul et même réseau de canaux de paiement.

Les transactions flottantes sont implémentées à l'aide de ``. La mise en
œuvre de Eltoo repose donc sur l'intégration du BIP-118 dans Bitcoin.

\section*{L'inscription de données
arbitraires}\label{linscription-de-donnuxe9es-arbitraires}
\addcontentsline{toc}{section}{L'inscription de données arbitraires}

\markright{L'inscription de données arbitraires}

Bitcoin permet d'inscrire des données non financières sur la chaîne,
c'est-à-dire des données qui ne sont pas nécessaires dans le blocage et
le déblocage des fonds et qui sont interprétées de manière extérieure au
protocole. Même en imposant toutes les restrictions possibles, on ne
peut pas empêcher l'inscription de ces données, même s'il est possible
de la rendre plus coûteuse.

La chaîne de blocs de la version principale de Bitcoin est largement
partagée autour du monde, et sera conservée par l'humanité, au moins
comme un reliquat historique, laissant supposer que ce qui y est stocké
sera conservé très longtemps. Cette caractéristique pousse les gens à y
inclure des choses qui leur tiennent à cœur. Il est dans la nature de
l'homme de chercher à laisser des traces de son passage sur Terre et
écrire sur un registre réputé immuable est une manière de le faire.

Il existe diverses méthodes d'inscription, qui ont chacune leurs
qualités et leurs défauts\footnote{«~diverses méthodes d'inscription~»~:
  Andrew Sward, Ivy Vecna, Forrest Stonedahl, \emph{Data Insertion in
  Bitcoin's Blockchain}, in \emph{Ledger}, vol.~3, avril 2018~:
  \url{https://doi.org/10.5195/ledger.2018.101}.}. Celles-ci ont évolué
au fur et à mesure des années, alors que cette utilisation se
libéralisait.

D'une part, l'écriture de données arbitraires peut être réalisée par les
mineurs au sein de l'entrée de transaction de récompense, et plus
précisément dans le script de déverrouillage. Ce champ est en effet
superflu conceptuellement, la base de pièce ne faisant référence à
aucune sortie existante, et peut donc être exploité de manière
discrétionnaire. C'est cette méthode dont Satoshi Nakamoto a fait usage
pour inscrire le désormais célèbre titre de une du Times du 3 janvier
2009 dans le bloc de genèse~:

\emph{The Times 03/Jan/2009 Chancellor on brink of second bailout for
banks}

D'autres blocs contiennent des messages emblématiques. Le bloc d'exode
de BCH (de hauteur 478~559) contenait un message de bienvenue pour Shuya
Yang, la fille du PDG de la coopérative ViaBTC\footnote{«~message de
  bienvenue pour Shuya Yang~»~: \emph{Welcome to the world, Shuya
  Yang!}, \url{https://blockchair.com/bitcoin-cash/block/478559}.}. Le
bloc précédant le troisième halving sur BTC en 2020 (de hauteur 629~999)
incluait le titre d'un article du New York Times du 9 avril annonçant
l'injection de liquidité record de la Réserve Fédérale (2~300 milliars
de dollars) en réaction à la crise du Covid-19~: «~\emph{NYTimes
09/Apr/2020 With \$2.3T Injection, Fed's Plan Far Exceeds 2008
Rescue}~».\footnote{«~Le bloc précédant le troisième halving sur BTC en
  2020 {[}...{]} incluait le titre d'un article du New York Times~»~:
  \url{https://blockchair.com/bitcoin/block/629999}.}

Le script de déverrouillage de la base de pièce peut être utilisé pour
écrire d'autres données. C'est le cas du nonce supplémentaire (le
critère qui a permis d'identifier les bitcoins de Satoshi\footnote{«~le
  critère qui a permis d'identifier les bitcoins de Satoshi~»~: Sergio
  Lerner, \emph{The Well Deserved Fortune of Satoshi Nakamoto, Bitcoin
  creator, Visionary and Genius}, 17 avril 2013~:
  \url{https://bitslog.com/2013/04/17/the-well-deserved-fortune-of-satoshi-nakamoto/}.}).
C'est aussi le cas du signalement des coopératives minières qui est
réalisé \emph{via} ce champ~: par exemple, la base de pièce du bloc
751~005 contient la chaîne de caractères ``, ce qui indique que sa
validation a probablement été réalisée par la coopérative chinoise
Poolin.

D'autre part, l'inscription des données arbitraires peut aussi être le
fait des utilisateurs, qui peuvent les inclure dans leurs transactions
et payer les frais correspondants. Plusieurs méthodes ont été exploitées
pour ce faire.

Avant 2014, on procédait la plupart du temps à ces inscriptions en
stockant les données dans les scripts de verrouillage, par exemple par
l'utilisation de l'instruction de dépilement
\texttt{OP\_DROP}\footnote{La transaction
  \texttt{,\ confirmée\ le\ 13\ décembre\ 2012,\ contient\ par\ exemple\ la\ chaîne\ de\ caractères}
  en référence au jeu vidéo Portal.}. Une autre pratique courante était
d'inscrire les données dans les sorties de type P2PKH, qui étaient
rendues indépensables au passage. Cette méthode était extrêmement
coûteuse en raison de la forme de la transaction (imposant l'inscription
dans les sorties transactionnelles) et le fait de devoir envoyer des
montants non nuls en sortie. Elle était également dommageable pour le
système dans son ensemble, car elle encombrait l'ensemble des UTXO.

Après 2014, une manière plus efficace de stocker des données a été
autorisée par le biais de la standardisation du schéma NULLDATA qui se
basait sur l'instruction \texttt{OP\_RETURN}. Ce changement permettait
de créer «~une sortie assurément élagable, pour éviter les schémas de
stockage d'informations {[}...{]} qui enregistraient des données
arbitraires, telles que des images, en tant que sorties
transactionnelles éternellement indépensables, gonflant ainsi la base de
données des UTXO de bitcoin\footnote{Bitcoin Core, \emph{Bitcoin Core
  version 0.9.0 released}, 19 mars 2014~:
  \url{https://bitcoin.org/en/release/v0.9.0\#opreturn-and-data-in-the-block-chain}.}~».
Il limitait aussi le gaspillage de fonds en autorisant la création d'une
sortie de 0 satoshi. Ce schéma s'est rapidement imposé comme la manière
la plus populaire pour publier des informations sur la chaîne.

En outre, il est aussi possible de stocker des données au sein des
entrées transactionnelles ou des témoins liés, lors de la dépense de
sorties P2SH, P2WSH ou P2TR. Cette écriture peut se faire dans les
scripts de récupération ou bien dans les éléments de déverrouillage.
Cette méthode a l'avantage de ne pas surcharger l'ensemble des UTXO.
Côté utilisateur, dans les entrées où SegWit s'applique, elle a pour
bénéfice de diviser le coût des données arbitraires inscrites dans la
transactions par quatre.

Ces différentes méthodes ont été utilisées pour inscrire toutes sortes
de choses sur la chaîne, dont notamment des empreintes cryptographiques,
du texte et des images\footnote{Ken Shirriff, \emph{Hidden surprises in
  the Bitcoin blockchain and how they are stored: Nelson Mandela,
  Wikileaks, photos, and Python software}, 16 février 2014~:
  \url{https://www.righto.com/2014/02/ascii-bernanke-wikileaks-photographs.html}.}.

D'abord, on peut inscrire une empreinte, l'inscription servant alors à
l'horodatage. Il s'agit d'inscrire l'empreinte d'un fichier sur la
chaîne en tant que preuve d'existence. Cette idée a été mise en avant en
février 2009 par Hal Finney dans un de ses courriels adressés à la liste
de diffusion dédiée à Bitcoin. Il suggérait alors que «~la pile de blocs
de bitcoin serait parfaite~» pour «~prouver qu'un certain document a
existé à un certain moment dans le passé\footnote{Hal Finney, \emph{Re:
  {[}bitcoin-list{]} Bitcoin v0.1.5 released}, /02/2009 20:00:12 UTC,
  archive~:
  \url{https://web.archive.org/web/20131016004925/http://sourceforge.net/p/bitcoin/mailman/bitcoin-list/?viewmonth=200902}.}~»,
un point de vue approuvé par Satoshi\footnote{«~un point de vue approuvé
  par Satoshi~»~: Satoshi Nakamoto, \emph{Re: {[}bitcoin-list{]} Bitcoin
  v0.1.5 released}, /03/2009 16:59:12 UTC, archive~:
  \url{https://web.archive.org/web/20131016004648/http://sourceforge.net/p/bitcoin/mailman/bitcoin-list/?viewmonth=200903}~:
  «~En effet, Bitcoin est un serveur d'horodatage sécurisé et distribué
  pour les transactions. Quelques lignes de code pourraient créer une
  transaction avec une empreinte supplémentaire de tout ce qui doit être
  horodaté. Je devrais ajouter une commande pour horodater un fichier de
  cette façon.~»}. En somme, cette pratique permet de démontrer la
connaissance d'une information avant sa publication, et donc
indirectement qu'on en est l'auteur probable. Ce type d'usage a
notamment été mis en œuvre par l'entreprise française Woleet.

Cette possibilité peut aussi être exploité par les systèmes
décentralisés d'hébergement de fichiers, comme le système IPFS
(InterPlanetary File System) qui utiliser les empreintes des fichiers
pour les identifier et permettre leur stockage par un réseau pair à pair
d'utilisateurs. Il est donc possible d'associer le texte écrit sur la
chaîne de blocs et des images ou des vidéos, hébergées de manière
décentralisée.

Ensuite, on peut inscrire un texte, qui est généralement encodé en
ASCII~/~UTF-8. Par exemple, la phrase «~La beauté sauvera le monde.~» a
été inscrite sur la chaîne de BTC le 10 août 2022 dans la transaction
d'identifiant ``. L'inscription de textes permet aussi de dessiner des
images en art ASCII. C'est le cas de l'hommage à Len Sassaman (voir
figure~\hyperref[fig:sassaman-tribute]{{[}fig:sassaman-tribute{]}}),
décédé en juillet 2011, qui a été inscrit sur la chaîne par les
développeurs Dan Kaminsky et Travis Goodspeed dans des sorties P2PKH, et
qui contient notamment une représentation de l'ancien président de la
Fed, Ben Bernanke\footnote{«~l'hommage à Len Sassaman {[}...{]} inscrit
  sur la chaîne par les développeurs Dan Kaminsky et Travis
  Goodspeed~»~: Cet hommage peut être retrouvé dans la transaction
  d'identifiant `` confirmée le 30 juillet 2011.}.

\begin{verbatim}
---BEGIN TRIBUTE---  =-=-=-=-=-=-=-=-=-=     ASCII BERNANKE
        #./BitLen            LEN "rabbi" SASSAMA  :'::.:::::.:::.::.:
        :::::::::::::::::::       1980-2011       : :.: ' ' ' ' : :':
        :::::::.::.::.:.:::  Len was our friend.  :.:     _.__    '.:
        :.: :.' ' ' ' ' : :  A brilliant mind,    :   _,^"   "^x,   :
        :.:'' ,,xiW,"4x, ''  a kind soul, and     '  x7'        `4,
        :  ,dWWWXXXXi,4WX,   a devious schemer;    XX7            4XX
        ' dWWWXXX7"     `X,  husband to Meredith   XX              XX
         lWWWXX7   __   _ X  brother to Calvin,    Xl ,xxx,   ,xxx,XX
        :WWWXX7 ,xXX7' "^^X  son to Jim and       ( ' _,+o, | ,o+,"
        lWWWX7, _.+,, _.+.,  Dana Hartshorn,       4   "-^' X "^-'" 7
        :WWW7,. `^"-" ,^-'   coauthor and          l,     ( ))     ,X
         WW",X:        X,    cofounder and         :Xx,_ ,xXXXxx,_,XX
         "7^^Xl.    _(_x7'   Shmoo and so much      4XXiX'-___-`XXXX'
         l ( :X:       __ _  more.  We dedicate      4XXi,_   _iXX7'
         `. " XX  ,xxWWWWX7  this silly hack to     , `4XXXXXXXXX^ _,
          )X- "" 4X" .___.   Len, who would have    Xx,  ""^^^XX7,xX
        ,W X     :Xi  _,,_   found it absolutely  W,"4WWx,_ _,XxWWX7'
        WW X      4XiyXWWXd  hilarious.           Xwi, "4WW7""4WW7',W
        "" ,,      4XWWWWXX  --Dan Kaminsky,      TXXWw, ^7 Xk 47 ,WH
        , R7X,       "^447^  Travis Goodspeed     :TXXXWw,_ "), ,wWT:
        R, "4RXk,      _, ,  P.S.  My apologies,  ::TTXXWWW lXl WWT:
        TWk  "4RXXi,   X',x  BitCoin people.  He  ----END TRIBUTE----
        lTWk,  "4RRR7' 4 XH  also would have
        :lWWWk,  ^"     `4   LOL'd at BitCoin's
        ::TTXWWi,_  Xll :..  new dependency upon
\end{verbatim}

Enfin, on peut inclure une image, qui peut être encodée dans de
multiples formats, dont notamment en JPEG ou en PNG. Un logo Bitcoin
inscrit le 13 mai 2011 peut par exemple être retrouvé\footnote{«~Un logo
  Bitcoin inscrit le 13 mai 2011 peut par exemple être retrouvé~»~:
  Transactions \texttt{et}}. Un hommage à Nelson Mandela accompagné
d'une photo a été publié le 7 décembre 2013, quelques jours après sa
mort\footnote{«~Un hommage à Nelson Mandela accompagné d'une photo a été
  publié le 7 décembre 2013, quelques jours après sa mort~»~: Voir
  transaction ``.}. En 2022, l'absence de restriction standarde sur la
taille des scripts de Taproot a permis de réaliser des inscriptions
volumineuses d'une manière bien plus transparente et directe. C'est ce
qui a notamment permis d'inscrire l'image des Taproot Wizards qui pesait
quasiment 4~Mo (voir figure~\hyperref[fig:taproot-wizards]{13.6}).

\begin{figure}

{\centering \includegraphics{chapters/img/taproot-wizards-small-0301e0480b374b32851a9462db29dc19fe830a7f7d7a88b81612b9d42099c0aei0.jpg}

}

\caption{Image (réduite) des Taproot Wizards.}

\end{figure}%

De manière générale, tout format de fichier peut être stocké sur la
chaîne au moyen de transactions multiples~: un document\footnote{«~un
  document~»~: Le PDF du livre blanc de Bitcoin a été inscrit sous forme
  de sorties P2MS au sein de la transaction ``, le 6 avril 2013.}, un
livre, une vidéo, un jeu\footnote{«~un jeu~»~: Nicholas Carlini,
  \emph{Yet Another Doom Clone}, 1 février 2023~:
  \url{https://ordinals.com/inscription/521f8eccffa4c41a3a7728dd012ea5a4a02feed81f41159231251ecf1e5c79dai0},
  \url{https://nicholas.carlini.com/writing/2019/javascript-doom-clone-game.html}.},
etc. Cependant, cette utilisation n'est pas forcément toujours
pertinente. L'inscription demande le paiement de frais, parfois élevés,
et la chaîne de blocs de BTC n'est pas franchement faite pour conserver
des données volumineuses. La publication de ces fichiers sur IPFS et sur
serveur local est généralement bien plus opportune.

Notons que la communauté de Bitcoin SV s'est focalisée sur le stockage
de données, considérant que son registre était une «~source universelle
de vérité\footnote{CoinGeek, \emph{Jerry Chan: Bitcoin's value is as a
  universal source of truth}, 17 juillet 2019~:
  \url{https://coingeek.com/jerry-chan-bitcoins-value-is-as-a-universal-source-of-truth-video/}.}~».
On peut ainsi retrouver un volume assez importants de données
météorologiques sur sa chaîne, qui y sont inscrites depuis
2019\footnote{«~données météorologiques~»~: Helen Partz, \emph{98\% of
  BSV Transactions Used for Writing Weather Data on Blockchain: Report},
  24 juin 2019~:
  \url{https://cointelegraph.com/news/98-of-bsv-transactions-used-for-writing-weather-data-on-blockchain-report}.}.
Cela fait que le réseau BSV est extrêmement centralisé tant du point de
vue minier que commercial, ce qui remet en cause l'utilité première de
l'inscription d'informations sur une chaîne de blocs~: l'immuabilité.

\section*{Les métaprotocoles}\label{les-muxe9taprotocoles}
\addcontentsline{toc}{section}{Les métaprotocoles}

\markright{Les métaprotocoles}

Les métaprotocoles sont des protocoles qui se servent du protocole de
base pour fonctionner. Ils font usage de l'inscription de données
arbitraires sur la chaîne pour inclure des instructions qui sont
interprétées par des implémentations logicielles spécifiques\footnote{«~Ils
  font usage de l'inscription de données arbitraires sur la chaîne pour
  inclure des instructions qui sont interprétées par des implémentations
  logicielles spécifiques~»~: C'est en ce sens que les métaprotocoles
  peuvent être appelés des surcouches, même si on préfère généralement
  utiliser ce terme pour parler des systèmes comme Lightning par
  exemple.}. Ils ont pour particularité d'être plus extensifs que le
protocole de base.

Il ne s'agit pas d'une idée nouvelle. Dès les premières années
d'existence de Bitcoin, certaines personnes ont souhaité l'exploiter
plus en profondeur, en se servant de lui d'une autre manière que comme
un instrument de transfert de valeur. Ce mouvement initial, visant à
ajouter des fonctionnalités à Bitcoin de cette façon, était appelé
«~Bitcoin 2.0~». Il a finalement mené à l'élaboration d'Ethereum à
partir de 2013.

Le premier type de métaprotocole qui a été élaboré est le procédé des
\emph{colored coins}, ou pièces colorées en français, qui consiste à
marquer des pièces (UTXO) par l'inscription annexe de données, comme
montré sur la figure~\hyperref[fig:colored-coin]{13.7}. Chaque type de
jeton créé est lié à un identifiant, que l'on peut assimiler à une
couleur, d'où le nom de ce procédé. L'idée a été présentée en 2012 par
Yoni Assia et Meni Rosenfeld\footnote{Yoni Assia, \emph{bitcoin 2.X (aka
  Colored Bitcoin) -- initial specs}, 27 mars 2012~:
  \url{https://yoniassia.com/coloredbitcoin/}~; Meni Rosenfeld,
  \emph{Overview of Colored Coins}, 4 décembre 2012~:
  \url{https://bitcoil.co.il/BitcoinX.pdf}.}.

\begin{figure}

{\centering \includegraphics{chapters/img/colored-coin.png}

}

\caption{Création et transfert d'un jeton émis sous la forme d'un
colored coin.}

\end{figure}%

L'implémentation de ce concept a été réalisée dès la fin de l'année 2012
par l'intermédiaire du ChromaWallet\footnote{«~ChromaWallet~»~: Alex
  Mizrahi, \emph{ChromaWallet (colored coins): issue and trade private
  currencies/stocks/bonds/..}, /09/2012 12:46:12 UTC~:
  \url{https://bitcointalk.org/index.php?topic=106373.msg1167516\#msg1167516}.}.
Cependant, elle n'a vraiment pris de l'ampleur qu'à partir de 2014, avec
l'apparition des Open Assets de Coinprism, des \emph{CoinSpark assets}
de Coin Sciences, et des Colored Coins de Colu. Ces usages sont depuis
tombés en désuétude, même si le procédé a pu servir de manière
sporadique au fil des années, comme dans le cas du jeton BSQ de Bisq
créé en 2018 comme base de sa DAO. Une tentative de restauration a
également été faite sur Bitcoin Cash avec les jetons SLP, sans grand
succès.

Au-delà des pièces colorées, il existait des protocoles plus évolués qui
avaient la particularité de gérer une unité de compte propre. Il
s'agissait essentiellement de Mastercoin, qui a été renommé en Omni en
mars 2015, et de Counterparty.

Le premier métaprotocole avancé a été Mastercoin\footnote{«~Mastercoin~»~:
  Le mot \emph{master} dans le nom de Mastercoin est l'acronyme de
  «~\emph{Metadata Archival by Standard Transaction Embedding
  Records}~», d'après les spécifications techniques~:
  \url{https://github.com/OmniLayer/spec/blob/master/OmniSpecification-v0.6.adoc}.},
dont le livre blanc, intitulé «~\emph{The Second Bitcoin Whitepaper}~»,
a été publié le 6 janvier 2012 par J.R. Willett\footnote{J.R. Willett,
  \emph{The Second Bitcoin Whitepaper}, 6 janvier 2012, archive~:
  \url{https://cryptochainuni.com/wp-content/uploads/Mastercoin-2nd-Bitcoin-Whitepaper.pdf}.}\footnote{«~«~le
  livre blanc, intitulé "\emph{The Second Bitcoin Whitepaper}", a été
  publié le 6 janvier 2012 par J.R. Willett~»~: J.R. Willett,
  \emph{{[}It's here{]} The Second Bitcoin Whitepaper}, /01/2012
  22:42:24 UTC~:
  \url{https://bitcointalk.org/index.php?topic=56901.msg678427\#msg678427}.}.
Il s'agissait d'un protocole permettant à ses utilisateurs de créer
leurs propres devises, appelées «~\emph{user currencies}~». Mastercoin
reposait sur une unité de compte notée le MSC, qui a fait l'objet d'une
prévente d'un mois en juillet-août 2013\footnote{Tous les bitcoins
  envoyés à l'adresse `` étaient transformés en MSC à raison de 100~MSC
  au début, taux dégressif au fil des semaines.}. C'était la première
\emph{Initial Coin Offering} de l'histoire, et elle a recueilli
5~120~BTC, soit plus de 500~000~\$ à ce moment-là.

Le plus grand succès de ce protocole a probablement été la création du
premier stablecoin, le Tether USD, qui a été émis sous le nom de
Realcoin en octobre 2014. Mastercoin~/~Omni a longtemps été l'unique
manière de posséder et de transférer de l'USDT avant que le jeton ne
soit émis massivement sur d'autres chaînes comme Ethereum et Tron.

Le second métaprotocole avancé a été Counterparty, lancé en janvier
2014. Cette plateforme reposait également sur un jeton natif, le XCP,
qui lui servait de carburant, et qui a été créé par brûlage de bitcoins
durant son premier mois d'existence\footnote{Tous les bitcoins envoyés à
  l'adresse `` entre le 2 janvier et le 3 février 2014 étaient convertis
  en XCP à un taux qui variait entre 1~000 et 1~500 XCP par BTC}. Ce
sont 2~140 bitcoins qui ont ainsi été rendus inutilisables pour donner
vie à plus de 2,6 millions de XCP, encore en circulation aujourd'hui.
Counterparty se voulait plus flexible que Mastercoin en rendant possible
l'implémentation de contrats autonomes, notamment dans le but de créer
des jetons et d'héberger des plateformes d'échange décentralisées,
appelées des «~distributeurs~».

En particulier, Counterparty a été la première plateforme à proposer la
gestion de jetons non fongibles (NFT). Il s'agissait là de mettre en
œuvre une vieille idée, qui avait notamment été mise en valeur par Hal
Finney en 1993 sur la liste de diffusion cypherpunk sous la forme de
«~cartes à collectionner cryptographiques\footnote{Hal Finney,
  \emph{Crypto trading cards.}, /01/1993 18:48:02 UTC~:
  \url{https://cypherpunks.venona.com/date/1993/01/msg00152.html}.}~».
Counterparty a ainsi hébergé une multitude de collections de tels
objets, comme les cartes à jouer de \emph{Spells of Genesis} et de
SaruTobi créées en 2015, ou les Rare Pepes émis entre 2016 et
2018\footnote{«~une multitude de collections de tels objets~»~: Vlad
  Costea, \emph{Bitcoin NFTs On Counterparty (And How To Get Or Create
  Your First One)}, 29 décembre 2021~:
  \url{https://bitcoin-takeover.com/bitcoin-nfts-on-counterparty-and-how-to-get-or-create-your-first-one/}.}.

En 2018, l'apparition de Bitcoin Cash a motivé la création d'un média
social dont les données seraient entièrement stockées sur la chaîne, les
développeurs de BCH étant plus libéraux à ce sujet. Le protocole
s'appelait Memo et consistait à publier de courts messages visibles
publiquement sous un profil défini et à pouvoir suivre les autres
utilisateurs, à aimer et répondre à leurs messages. L'idée était
d'obtenir une sorte de réseau social résistant à la censure, mais
souffrait néanmoins de la nécessité de payer des frais à chaque action.
\footnote{«~Le protocole s'appelait Memo et consistait à publier de
  courts messages visibles publiquement sous un profil défini et à
  pouvoir suivre les autres utilisateurs, à aimer et répondre à leurs
  messages~»~: \url{https://memo.cash/protocol}.}

Tous ces protocoles ont perdu leur attrait jusqu'à l'apparition du
protocole Ordinals, lancé en janvier 2023. Ce métaprotocole permettait
de créer et de gérer des «~artéfacts numériques~», c'est-à-dire des NFT
dont l'intégralité des données est stockée de manière immuable sur une
chaîne résistante à la censure. Le protocole Ordinals reposait sur une
«~théorie des ordinaux~» permettant de suivre et de transférer des
satoshis liés à une inscription, comme un texte, une image ou autre
chose\footnote{«~Le protocole Ordinals~»~:
  \url{https://docs.ordinals.com/}.}. En particulier, Ordinals a été
utilisé pour émuler la propriété et le transfert de jetons fongibles,
baptisés «~BRC-20~», dont le succès spéculatif a provoqué une congestion
du réseau menant à une hausse des frais de transaction historique. Le
succès d'Ordinals a également inspiré la création du protocole STAMPS,
qui se basait sur Counterparty pour le suivi des artéfacts et stockait
leurs données dans des sorties P2MS\footnote{«~STAMPS~»~: STAMP est
  l'acronyme de \emph{Secure, Tradeable Art Maintained Permanently}. --
  Mike In Space, \emph{STAMPS: A Protocol for Storing Images On-Chain in
  Transaction Outputs Immutably on Bitcoin}, 6 avril 2023~:
  \url{https://github.com/mikeinspace/stamps/tree/main}.}.

Toutes ces pratiques ont créé des débats. En effet, Bitcoin était
présenté comme un modèle de monnaie numérique et il semblait
contreproductif d'en faire un protocole de conservation de données qui
ne seraient pas relatives au transfert de bitcoins. Ainsi, dès décembre
2010, Jeff Garzik s'opposait au fait d'utiliser la chaîne pour le
stockage généralisé\footnote{Jeff Garzik, \emph{Resist the urge to use
  block chain for generalized storage}, /12/2010 22:04:54 UTC~:
  \url{https://bitcointalk.org/index.php?topic=2129.msg27884\#msg27884}.}.
Plus tard, en 2014, des disputes similaires ont éclaté au sujet de
Counterparty\footnote{BitMEX Research, \emph{The OP\_Return Wars of 2014
  -- Dapps Vs Bitcoin Transactions}, 12 juillet 2022~:
  \url{https://blog.bitmex.com/dapps-or-only-bitcoin-transactions-the-2014-debate/}.}.
En 2023, c'est également la même discorde qui a eu lieu suite au succès
d'Ordinals\footnote{pourteaux, \emph{Illegitimate bitcoin transactions},
  25 janvier 2023~:
  \url{https://read.pourteaux.xyz/p/illegitimate-bitcoin-transactions}.}.

Ces métaprotocoles présentent deux défauts majeurs. Le premier est que
la vérification de leurs règles dépend d'un petit sous-ensemble de nœuds
du réseau. En effet, la gestion d'un tel protocole construit en
surcouche demande des ressources supplémentaires, notamment en ce qui
concerne l'indexation pour les pièces colorées. De ce fait, peu de
personnes déploient une implémentation complète, ce qui centralise
considérablement le protocole et le rend sensible à l'altération par un
adversaire qui aurait pour but de le censurer.

Le second défaut concerne leurs frais d'utilisation parfois très élevés,
surtout si la limite de capacité transactionnelle du réseau est
atteinte. Les transactions qui mettent en place ces solutions sont
nécessairement plus volumineuses que les transactions normales et
entraînent par conséquent des frais plus élevés. Elles sont donc plus
facilement exclues par l'augmentation des frais issue de la congestion
du réseau.

C'est pour ces raisons que les personnes qui ont travaillé sur ces
solutions s'en sont vite détournées, préférant se réfugier vers des
plateformes alternatives comme NXT et surtout Ethereum. Vitalik Buterin
lui-même s'intéressait aux pièces colorées et à Mastercoin en 2013 avant
de commencer à bâtir ce qui allait devenir Ethereum\footnote{Yoni Assia,
  Vitalik Buterin, Meni Rosenfeld, Rotem Lev, \emph{Colored Coins
  whitepaper}, 2013~:
  \url{http://www.ma.senac.br/wp-content/uploads/2018/05/ColoredCoinswhitepaper-DigitalAssets.pdf}~;
  Vitalik Buterin, \emph{A Prehistory of the Ethereum Protocol}, 14
  septembre 2017~:
  \url{https://vitalik.ca/general/2017/09/14/prehistory.html}.}. C'est
aussi pour ces raisons que des solutions moins coûteuses (des surcouches
utilisant la chaîne comme un procédé de règlement et non pas comme un
lieu où inscrire toutes les opérations) sont aujourd'hui privilégiées
pour faire ce genre de choses comme RGB ou Taproot Assets.

\section*{Les contrats hors chaîne}\label{les-contrats-hors-chauxeene}
\addcontentsline{toc}{section}{Les contrats hors chaîne}

\markright{Les contrats hors chaîne}

La cryptographie permet de déployer des contrats sans que ceux-ci ne
doivent être inscrits sur la chaîne. Cette particularité a été facilitée
grâce à la mise à niveau Schnorr-Taproot, souvent simplement appelée
«~Taproot~», qui est survenue sur BTC le 14 novembre 2021 et qui
incluait deux éléments majeurs~: le schéma de signature de Schnorr et le
procédé de programmation de contrats Taproot. Ces fonctionnalités ont
été intégrées sous forme d'un soft fork au sein du schéma standard P2TR
correspondant à la version 1 de SegWit.

Le schéma de Schnorr implémenté est une dérivation du protocole
d'authentification du même nom décrit en 1989 par Claus-Peter Schnorr.
Il s'agit d'une alternative à ECDSA qui se base sur la même courbe
elliptique (secp256k1) et qui permet de signer des transactions grâce
aux mêmes paires de clés.

Comparé à ECDSA, le schéma de signature de Schnorr possède quelques
avantages\footnote{«~Comparé à ECDSA, le schéma de signature de Schnorr
  possède quelques avantages~»~: Pieter Wuille, Jonas Nick, Tim Ruffing,
  \emph{BIP-340: Schnorr Signatures for secp256k1}, 19 janvier 2020~:
  \url{https://github.com/bitcoin/bips/blob/master/bip-0340.mediawiki}.}.
Premièrement, il produit des signatures moins grandes. Deuxièmement, les
signatures produites ne sont pas malléables, le procédé ne faisant pas
intervenir de nombre aléatoire. Troisièmement, et c'est le plus
important, il présente une propriété de linéarité, ce qui permet
notamment de faire des choses comme la vérification par lots et
l'agrégation des clés.

Le schéma de Schnorr est supérieur à ECDSA et existait en 2008, mais
Satoshi Nakamoto n'a pas daigné s'en servir. Ce choix s'explique par le
fait que l'algorithme était breveté aux États-Unis jusqu'en février 2008
et que par conséquent il n'existait pas d'implémentation standardisée.
Le logiciel de Bitcoin utilisait en effet OpenSSL, qui n'intégrait pas
ce type d'algorithme.

Le schéma de Schnorr autorise le déploiement de \emph{Scriptless
Scripts}, de contrats «~sans script~» qui sont exécutés en dehors de la
chaîne et appliqués au sein des signatures. Le concept a été théorisé en
2017 par Andrew Poelstra\footnote{Andrew Poelstra, \emph{Using the Chain
  for what Chains are Good For} (vidéo), Scaling Bitcoin IV, 5 novembre
  2017~: \url{https://www.youtube.com/watch?v=3pd6xHjLbhs&t=5755s}~;
  Aaron van Wirdum, «~\emph{Scriptless Scripts: How Bitcoin Can Support
  Smart Contracts Without Smart Contracts}~», \emph{Bitcoin Magazine},
  27 novembre 2017~:
  \url{https://bitcoinmagazine.com/technical/scriptless-scripts-how-bitcoin-can-support-smart-contracts-without-smart-contracts}.}.
Il se retrouve dans des exemples comme le schéma de signature
multipartite MuSig2\footnote{«~MuSig2~»~: Jonas Nick, Tim Ruffing,
  Yannick Seurin, \emph{MuSig2: Simple Two-Round Schnorr
  Multi-Signatures}, 14 octobre 2020~:
  \url{https://eprint.iacr.org/2020/1261.pdf}.}, les \emph{Adaptor
Signatures} ou encore les \emph{Discreet Log Contracts}\footnote{«~\emph{Discreet
  Log Contracts}~»~: Thaddeus Dryja, \emph{Discreet Log Contracts},
  2017~: \url{https://adiabat.github.io/dlc.pdf}.}.

Outre cela, le schéma de Schnorr facilite grandement l'implémentation de
Taproot (BIP-341\footnote{«~BIP-341~»~: Pieter Wuille, Jonas Nick,
  Anthony Towns, \emph{BIP-341: Taproot: SegWit version 1 spending
  rules}, 19 janvier 2020~:
  \url{https://github.com/bitcoin/bips/blob/master/bip-0341.mediawiki}.}),
qui a été intégré au protocole au même moment. Taproot (dont le nom
signifie littéralement «~racine pivot~» en français) est un procédé de
programmation de contrats autonomes qui ancre les clauses d'un contrat
au sein d'un arbre de Merkle et qui cache cet arbre sous une clé
publique agrégée appartenant à ses participants. Il permet de ne publier
le contrat qu'en cas de litige, et même dans ce cas, de ne publier que
les conditions exécutées. Les scripts utilisés dans Taproot utilisent un
langage de programmation nommé Tapscript (BIP-342\footnote{«~BIP-342~»~:
  Pieter Wuille, Jonas Nick, Anthony Towns, \emph{BIP-342: Validation of
  Taproot Scripts}, 19 janvier 2020~:
  \url{https://github.com/bitcoin/bips/blob/master/bip-0342.mediawiki}.}),
basé sur le langage de script classique de Bitcoin.

Taproot repose sur un arbre de hachage, appelé un MAST\footnote{L'acronyme
  MAST signifie originellement \emph{Merklized Abstract Syntax Trees},
  ou «~arbre syntaxique abstrait merkélisé~» en français, et se réfère
  aux structures de données décrites par le BIP-114. Dans Taproot, ce ne
  sont pas vraiment des arbres syntaxiques abstraits qui interviennent,
  mais le terme reste utilisé. Les arbres de hachage de Taproot peuvent
  dans ce cas être appelés des \emph{Merklized Alternative Script
  Trees}, par rétroacronymie. Voir Anthony Towns,
  \emph{{[}bitcoin-dev{]} Safer sighashes and more granular
  SIGHASH\_NOINPUT}, /11/2018 05:03:30 UTC~:
  \url{https://lists.linuxfoundation.org/pipermail/bitcoin-dev/2018-November/016500.html}.},
dont les feuilles sont les clauses du contrat, c'est-à-dire les
conditions de dépense. Lors de l'exécution du MAST, les participants
concernés ont seulement besoin de révéler la clause appliquée et de
fournir les empreintes liées aux autres clauses (chemin de Merkle),
comme montré sur la figure~\hyperref[fig:taproot-mast]{13.8}. Les autres
conditions de dépense ne sont ainsi pas révélées.

\begin{figure}

{\centering \includegraphics{chapters/img/taproot-mast.png}

}

\caption{MAST impliquant les clauses d'un contrat.}

\end{figure}%

L'implémentation de tels MAST au sein de Bitcoin avait déjà été proposée
par le passé, que ce soit sous la forme d'une nouvelle version de SegWit
(BIP-114), ou bien d'un nouveau code opération appelé `` (BIP-116,
BIP-117). Mais Taproot a constitué une proposition supérieure en
permettant de ne pas révéler l'existence du MAST lui-même.

Taproot inclut en effet une condition de dépense coopérative intégrée.
La clé publique agrégée interne est modifiée légèrement (\emph{tweaked})
à l'aide de la racine du MAST, afin de prendre ce dernier en compte. La
clé obtenue est celle inscrite dans le script de verrouillage de la
pièce, de sorte qu'elle est indiscernable des autres sorties P2TR. De
même, la signature agrégée ne peut pas être distinguée d'une signature
classique. De ce fait, les participants peuvent dépenser les fonds à
l'amiable, tout en étant sûrs qu'un litige mènera au règlement sur la
chaîne.

Une alternative à Taproot est RGB, qui est un système de contrats
autonomes hors chaîne, construit à la fois en surcouche de Bitcoin et de
Lightning. Le nom est issu du standard RGB (\emph{Red Green Blue}) qui
permet de définir une couleur et constitue par conséquent une référence
directe aux \emph{colored coins}, RGB ayant été originellement conçu
comme «~une meilleure version des pièces colorées\footnote{RGB FAQ,
  \emph{What does `RGB' stand for?}, 14 décembre 2020~:
  \url{https://www.rgbfaq.com/faq/what-does-rgb-stand-for}.}~».
Cependant, même si RGB permet effectivement d'émettre et de gérer des
jetons, cette fonctionnalité est loin d'être la seule.

RGB se base sur deux primitives techniques conceptualisées en 2016 par
le développeur Peter Todd\footnote{«~deux primitives techniques
  conceptualisées en 2016 par le développeur Peter Todd~»~:
  \url{https://petertodd.org/2016/state-machine-consensus-building-blocks\#uniqueness-and-single-use-seals}~;
  \url{https://petertodd.org/2016/closed-seal-sets-and-truth-lists-for-privacy}.}~:
la validation côté client (\emph{client-side validation}) et les scellés
à usage unique (\emph{single-use seals}). Cela permet la gestion d'un
état indépendant, où la double dépense est empêchée par ces scellés.
Après avoir fait l'objet de recherches par Giacomo Zucco et le BHB
Network, RGB est actuellement développé par la LNP/BP Standards
Association.

L'implémentation de contrats en dehors de la chaîne est donc possible
sur Bitcoin, et ils apportent deux choses. D'une part, ils réduisent le
paiement de frais en ayant une empreinte minimale sur la chaîne. D'autre
part, ils améliorent la confidentialité de leurs participants. C'est
pourquoi ils ont le potentiel de jouer un grand rôle à long terme.

\section*{Une monnaie programmable}\label{une-monnaie-programmable}
\addcontentsline{toc}{section}{Une monnaie programmable}

\markright{Une monnaie programmable}

L'aspect programmable de Bitcoin est souvent négligé. Il n'est en effet
pas directement présenté dans le livre blanc, même si Satoshi Nakamoto
l'avait déjà élaboré à ce moment-là. Il est toutefois très utile et
constitue l'une des facettes essentielles de Bitcoin.

La programmabilité de la monnaie peut servir à contrôler, comme
l'illustrent les projets de MNBC qui fleurissent autour du monde. Mais
elle peut également rendre un fier service à la liberté individuelle. En
effet, cet aspect modulable donne l'occasion à des personnes qui ne se
connaissent pas d'échanger de la valeur de la façon la plus sûre
possible ou, comme l'exprimait Tim May dans son \emph{Manifeste crypto
anarchiste} de 1988, de «~faire des affaires et négocier des contrats
électroniques sans jamais connaître le Vrai Nom, ou l'identité légale,
de l'autre\footnote{Timothy C. May, \emph{The Crypto Anarchist
  Manifesto}, /11/1992 20:11:24 UTC~:
  \url{https://cypherpunks.venona.com/date/1992/11/msg00204.html}.}~».

Les contrats autonomes forment la pierre angulaire des relations
financières dans le cyberespace. Même la communauté de Monero, qui avait
particulièrement restreint cet aspect à des fins de confidentialité, est
revenu sur ses pas en intégrant au protocole la fonctionnalité de
signature multipartite, notamment dans le but de permettre les échanges
atomiques. Une monnaie réellement libre se doit de pouvoir être
programmée librement.

\bookmarksetup{startatroot}

\chapter{Le passage à l'échelle}\label{ch:scalabilite}

\phantomsection\label{enotezch:14}{}

{L}\textsc{a} scalabilité, calque de l'anglais \emph{scalability}, aussi
appelée extensibilité, désigne la capacité d'un système à passer à
l'échelle, c'est-à-dire à continuer de fonctionner de manière
équivalente à mesure que le nombre d'utilisateurs augmente. Dans un
système géré de manière centralisée, cette capacité est assurée par
l'apport de matériel informatique, soit en ajoutant de la puissance de
calcul à l'infrastructure existante (passage à l'échelle vertical), soit
en multipliant les instances de l'infrastructure pour partager le
traitement des requêtes (passage à l'échelle horizontal). La scalabilité
dépend par conséquent du niveau de prévision de l'entité qui s'occupe du
système.

Dans le cas d'un système distribué, qui a un comportement différent, la
scalabilité se rapporte à quelque chose de plus compliqué. L'ajout de
matériel ne suffit pas~: il faut également que les propriétés dudit
système se conservent avec la hausse de l'activité. Dans le cas de
Bitcoin, ce problème est particulièrement ardu, car toute montée en
charge affecte durablement les nœuds du réseau, en raison de la
nécessité de partage de la chaîne de blocs intégrale. En substance, le
système ne passe pas à l'échelle, ou très peu.

Ce problème de scalabilité de Bitcoin a été une préoccupation majeure de
la communauté, à tel point qu'il a provoqué un véritable conflit ouvert
entre 2015 et 2017~: la fameuse guerre des blocs que nous avons décrite
dans le chapitre~\hyperref[ch:clivages]{2}. Certains pensaient
qu'augmenter la taille des blocs suffiraient à gérer la demande sans
altérer l'offre, tandis que d'autres imaginaient que les solutions de
surcouche comme le réseau Lightning seraient assez efficaces pour
traiter tous les transferts. Ce chapitre a pour objectif de faire un
tour d'horizon de la situation et de proposer une troisième voie.

\section*{L'absence de scalabilité du
système}\label{labsence-de-scalabilituxe9-du-systuxe8me}
\addcontentsline{toc}{section}{L'absence de scalabilité du système}

\markright{L'absence de scalabilité du système}

La conception originelle de Bitcoin repose sur un principe simple~:
obtenir et vérifier l'intégralité des transactions dans le but de
s'assurer qu'il n'y a pas de double dépense. Comme l'écrivait Satoshi
Nakamoto dans le livre blanc, «~la seule façon de confirmer l'absence
d'une transaction est d'être au courant de toutes les
transactions\footnote{Satoshi Nakamoto, \emph{Bitcoin: A Peer-to-Peer
  Electronic Cash System}, 31 octobre 2008.}~». De ce fait, pour
disposer d'une sécurité maximale, chaque nœud doit, en principe,
entretenir une version complète de la chaîne de blocs.

Dès les origines, ce fonctionnement particulier a naturellement amené la
question de la montée en charge du système. Lorsque Satoshi Nakamoto a
présenté sa découverte sur la liste de diffusion dédiée à la
cryptographie de Metzdowd.com le 31 octobre 2008, la première réponse
qu'il a reçue concernait ainsi ce problème. Cette réponse était celle de
l'ancien cypherpunk James A. Donald le 2 novembre, qui écrivait~:

«~Nous avons vraiment, vraiment besoin d'un tel système, mais d'après ce
que je comprends de votre proposition, il ne semble pas pouvoir
s'adapter à la taille requise.

Pour que des jetons de preuve de travail transférables aient de la
valeur, ils doivent avoir une valeur monétaire. Pour avoir une valeur
monétaire, ils doivent être transférés au sein d'un très grand réseau
--- par exemple un réseau d'échange de fichiers semblable à bittorrent.

Pour détecter et rejeter un événement de double dépense dans un délai
convenable, il faut disposer de la plupart des transactions passées des
pièces impliquées dans la transaction, ce qui, mis en œuvre naïvement,
exige que chaque pair dispose de la plupart des transactions passées, ou
de la plupart des transactions passées qui ont eu lieu récemment. Si des
centaines de millions de personnes effectuent des transactions, cela
représente beaucoup de bande passante --- chacun doit avoir connaissance
de toutes les transactions ou d'une partie substantielle de
celles-ci\footnote{James A. Donald, \emph{Re: Bitcoin P2P e-cash paper},
  /11/2008, 23:46:23 UTC~:
  \url{https://www.metzdowd.com/pipermail/cryptography/2008-November/014814.html}.}.~»

James A. Donald mettait par là en valeur le manque de scalabilité de
Bitcoin. Pour tout système donné, une augmentation de volume
transactionnel accroît le nombre de transactions à obtenir et à traiter.
Cet accroissement rend plus difficile de faire fonctionner un nœud, ce
qui peut affecter la décentralisation du réseau et donc la sécurité. De
ce fait, il existe toujours un compromis entre l'utilité et la sécurité
du système, ou pour mieux le dire, entre la facilité de transaction et
la facilité de vérification\footnote{«~entre la facilité de transaction
  et la facilité de vérification~»~: Sosthène, \emph{Apologie des petits
  blocs}, 2 août 2018~:
  \url{https://www.sosthene.net/apologie-petits-blocs/}.}.

Ce compromis se manifeste généralement par une limite de capacité
transactionnelle\footnote{«~Ce compromis se manifeste généralement par
  une limite de capacité transactionnelle~»~: Dans le cas où elle n'est
  définie nulle part, cette limite est de toute manière inférieure à la
  limite de marché du mineur le plus efficace, car aucun mineur
  économiquement rationnel ne traiterait de transaction gratuitement.},
décrite par les règles de consensus (limite explicite) ou, plus
rarement, par les règles de réseau (limite implicite). La limite de
capacité transactionnelle était originellement décrite comme une taille
limite des blocs, qui interdisait aux mineurs de créer des blocs plus
grands qu'une certaine taille. Dans le prototype, cette taille était
définie implicitement par la taille maximale des messages du protocole
de transmission, c'est-à-dire 32~Mio. Puis, une limite explicite de 1
mégaoctet (1~Mo) a été ajoutée par Satoshi Nakamoto en septembre 2010
par le biais de la constante ``, sans annonce publique de sa part, dans
le but d'éviter les attaques par déni de service. Cette taille
correspondait, pour un réseau tournant à plein régime, à un volume
théorique de 4,5 transactions classiques par seconde, ce qui se ramenait
plutôt à 3 transactions par seconde en pratique.

Avec l'intégration de SegWit dans la version principale de Bitcoin en
août 2017, cette limitation est devenue un poids limite des blocs. Cette
nouvelle métrique donnait une importance plus grande de la taille de
base par rapport à la taille du témoin dans le calcul de la mesure du
bloc, modifiant également la façon dont comptaient les mineurs pour
ajouter les transactions au bloc. Cette modification était une
augmentation effective de la capacité transactionnelle du protocole,
portant le volume autorisé de transactions à 8 transactions par seconde
en théorie, et à 4,5 transactions par seconde en pratique.

L'existence d'une limite de capacité transactionnelle engendre
nécessairement une rareté de l'espace de bloc. Si elle est fixe, elle
rend l'offre par essence inélastique. Ainsi, une forte demande pour
l'espace de bloc fait, par effet d'enchère, augmenter le prix pour
l'inclusion, c'est-à-dire les frais de transaction. Le marché des frais
est stimulé par cette limite rigide au lieu de rester à son niveau
naturel, à savoir celui du coût d'inclusion par défaut des mineurs.

Par son effet sur le niveau des frais, la limite crée un plancher
d'utilité, c'est-à-dire un niveau de valeur en deçà duquel le transfert
et la détention ne sont pas considérés comme rentables par les
utilisateurs. En effet, les mineurs sont amenés à rejeter les
transactions qui ne paient pas un taux de frais suffisant par rapport à
leur taille. De ce fait, l'utilité d'une transaction peut être estimée
insuffisante par son auteur au regard du niveau de frais moyen de la
chaîne, auquel cas elle n'a pas lieu. Si une personne désire acheter un
café pour 2~\$ en BTC, mais que les frais usuels sont de 1~\$, cette
personne passera vite son chemin. De manière générale, les cas
d'utilisation requérant des frais «~faibles~» sont chassés de la chaîne,
à l'instar du service de jeu d'argent SatoshiDICE, qui a dû cesser ses
activités sur BTC en 2017 suite à l'augmentation des frais.

La limite de capacité transactionnelle a pour vertu de garantir que le
coût d'utilisation d'un nœud reste bas. Elle agit ainsi sur la
décentralisation \emph{potentielle} du réseau. En effet, contrairement
au matériel minier, le coût lié à la vérification n'est pas compensé par
un revenu proportionnel, de sorte qu'il atteint tout le monde de la même
manière. Les opérateurs de nœud les moins bien équipés ne peuvent pas
matériellement suivre le rythme, ce qui affecte la facilité du réseau à
se décentraliser \emph{effectivement}.

L'influence sur la décentralisation potentielle concerne à la fois le
minage et le commerce en empêchant les plus petits acteurs de pouvoir
réaliser ces activités à leur échelle. La centralisation du minage
augmente le risque de censure, tandis que la centralisation du commerce
augmente le risque d'altération du protocole, et donc le risque
d'inflation. C'est pourquoi la limite de capacité transactionnelle joue
un rôle majeur dans le modèle de sécurité~: moins cette limite est
élevée, plus la sécurité \emph{potentielle} du système est grande.

La limite de capacité transactionnelle est déterminée de manière
subjective par les commerçants, en fonction de leur \emph{perception de
la menace} et de leur \emph{utilisation personnelle} de la chaîne. Il
n'existe pas de limite de taille des blocs idéale~: il n'y a que des
êtres humains qui calculent un risque par rapport à une éventuelle
récompense. On peut tenter d'établir une moyenne pour estimer une limite
qui correspond à une utilisation donnée, mais cette estimation serait au
mieux imparfaite.

Par son effet sur la décentralisation, la limite crée un plafond
d'utilité, c'est-à-dire un niveau de valeur au-delà duquel le transfert
et la détention sont considérés trop risqués pour la sécurité effective
du système. En effet, aucune sécurité n'étant absolue, le transfert et
la détention d'une certaine valeur peut ne plus bénéficier suffisamment
de la protection apportée par le réseau. Par exemple, recevoir ou
conserver l'équivalent de plusieurs millions de dollars sur la chaîne de
Bitcoin SV est, c'est le moins qu'on puisse dire, imprudent.

Le plancher d'utilité (induit par l'action négative de la limite sur
l'espace de bloc) et le plafond d'utilité (induit par l'action positive
de la limite sur la sécurité) ont pour effet de borner une plage de
valeurs en dehors de laquelle le transfert et la détention ne sont plus
pertinents\footnote{Voir Eric Voskuil, «~Propriété du seuil d'utilité~»,
  in \emph{Cryptoéconomie~: Principes fondamentaux de Bitcoin}, Amazon
  KDP, 2022, pp.~317--318.}. C'est l'existence de cette plage de valeurs
qui entraîne l'apparition de substituts à un système donné.

L'arrivée de nouveaux utilisateurs et l'augmentation subséquente de la
demande pour l'espace de blocs rehaussent le plancher d'utilité. Toute
montée en charge du système en modifie les caractéristiques. Par
conséquent, tout système Bitcoin est en substance non scalable, au sens
premier du terme. Il existe cependant des méthodes pour contourner cette
absence de scalabilité.

\section*{L'amélioration de l'efficacité de
base}\label{lamuxe9lioration-de-lefficacituxe9-de-base}
\addcontentsline{toc}{section}{L'amélioration de l'efficacité de base}

\markright{L'amélioration de l'efficacité de base}

La première proposition vis-à-vis du passage à l'échelle a été
d'augmenter progressivement la limite de taille des blocs dans le but
d'accompagner l'accroissement de l'activité\footnote{Parmi les versions
  alternatives de Bitcoin, la voie de l'augmentation progressive de la
  taille limite des blocs a été choisie par Bitcoin Cash, qui prévoit
  d'intégrer un algorithme permettant de gérer cette augmentation
  automatiquement. Voir bitcoincashautist, \emph{CHIP-2023-04: Adaptive
  Blocksize Limit Algorithm for Bitcoin Cash}, 2 septembre 2023~:
  \url{https://gitlab.com/0353F40E/ebaa/-/blob/f4edacd134103a7e232740463a5f26379bf90f18/README.md}.}\footnote{«~augmenter
  progressivement la limite de taille des blocs dans le but
  d'accompagner l'accroissement de l'activité~»~: Dans Monero, la
  pénalité (\(P\)) liée à la taille d'un bloc (\(B\)) est calculée à
  partir la taille médiane des 100 derniers blocs (\(M\)) et la
  subvention de base (\(R\)) qui est de 0,6~XMR par bloc depuis 2022. Si
  \(B < M_0 = 300~\mathrm{ko}\), alors \(P = 0\). Sinon~:
  \[P = R  \left( \frac{B}{M} - 1 \right)^2~.\] La taille du bloc ne
  peut pas dépasser \(2 M\) (taille qui correspond à une pénalité
  maximale).}. C'était la solution soutenue par Satoshi
Nakamoto\footnote{«~C'était la solution soutenue par Satoshi
  Nakamoto~»~: Satoshi Nakamoto, \emph{Re: {[}PATCH{]} increase block
  size limit}, /10/2010 19:48:40 UTC~:
  \url{https://bitcointalk.org/index.php?topic=1347.msg15366\#msg15366}.},
comme en témoigne sa première réaction à la réponse de James A. Donald
le 3 novembre 2008~:

«~La bande passante n'est peut-être pas aussi prohibitive que vous le
pensez. Une transaction typique est d'environ 400 octets (la
cryptographie sur les courbes elliptiques est agréablement compacte).
Chaque transaction doit être diffusée deux fois, soit 1 Ko par
transaction. Visa a traité 37 milliards de transactions au cours de
l'année fiscale 2008, soit une moyenne de 100 millions de transactions
par jour. Un tel nombre de transactions nécessiterait 100 Go de bande
passante, soit la taille de 12 DVD ou de 2 films en qualité HD, ou
encore environ 18~\$ de bande passante au prix actuel. Si le réseau
devait atteindre cette taille, cela prendrait plusieurs années, et d'ici
là, l'envoi de 2 films en HD sur Internet ne semblera probablement pas
être un gros problème\footnote{Satoshi Nakamoto, \emph{Re: Bitcoin P2P
  e-cash paper}, /11/2008, 01:37:43 UTC~:
  \url{https://www.metzdowd.com/pipermail/cryptography/2008-November/014815.html}.}.~»

La vision de Satoshi était cependant bien trop optimiste. D'une part, il
ne voyait pas la centralisation du minage comme un problème existentiel,
prévoyant dès le début que la puissance de calcul du réseau reposerait
sur des «~fermes de serveurs composées de matériel spécialisé~». D'autre
part, il pensait que la vérification de paiement simplifiée suffirait,
ne tenant pas compte de ses défauts de fiabilité et de confidentialité,
ni de son incapacité à exercer un pouvoir sur la détermination des
règles de consensus. Le plan de Satoshi était donc faillible sans pour
autant être entièrement mauvais.

Le fonctionnement d'un nœud dépend d'un certain nombre de charges. Les
principales sont le stockage sur disque dur (HDD) pour l'historique
(chaîne de blocs), le stockage en mémoire flash (SSD) pour l'état
(ensemble des UTXO), le stockage en mémoire vive (barrette de RAM) pour
la réserve des transactions non confirmées (mempool) et la réserve des
blocs orphelins, le maintien d'une bande passante (ou débit binaire,
usuellement exprimé en Mbit/s) permettant de recevoir et d'envoyer les
blocs et les transactions, et le calcul du processeur (CPU) pour la
vérification des données et notamment des signatures. Faire diminuer le
coût d'un nœud consiste ainsi à réduire l'une de ces charges.

Même si l'augmentation naïve de la taille limite des blocs ne constitue
pas en soi une méthode de scalabilité, il s'avère qu'elle peut être
compensée par le progrès technique provenant de l'optimisation
logicielle, matérielle ou algorithmique. Premièrement, les performances
du logiciel (pour un ensemble donné de règles de consensus) peuvent être
améliorées, et il s'agit même de l'une des tâches de base de l'équipe de
Bitcoin Core\footnote{Voir par exemple l'article de Jameson Lopp sur
  l'évolution de la performance de Bitcoin Core dans lequel décrit
  comment la première synchronisation sur sa machine s'est améliorée au
  cours des années. -- Jameson Lopp, \emph{Bitcoin Core Performance
  Evolution}, 5 mars 2022~:
  \url{https://blog.lopp.net/bitcoin-core-performance-evolution/}.}.
Deuxièmement, le matériel informatique peut être rendu plus efficace,
certains composants devenant progressivement moins coûteux (loi de
Moore\footnote{La loi de Moore est une conjecture énoncée par Gordon E.
  Moore en 1965 ayant postulé que la complexité des semi-conducteurs
  doublait chaque année. Cette loi était citée par Satoshi Nakamoto dans
  le livre blanc, qui écrivait~: «~La loi de Moore prédisant une
  croissance actuelle de 1,2 Go par an, le stockage ne devrait pas poser
  de problème même si les entêtes de blocs doivent être conservés en
  mémoire.~» -- Satoshi Nakamoto, \emph{Bitcoin: A Peer-to-Peer
  Electronic Cash System}, 31 octobre 2008.}). Troisièmement, le
protocole peut lui-même être perfectionné au niveau algorithmique, par
la découverte et l'adoption de nouvelles techniques plus efficaces~:
c'est par exemple le cas de l'algorithme de signature de Schnorr qui
produit des signatures plus compactes qu'ECDSA (40~octets au lieu de
72), ou bien des bulletproofs qui rendent les preuves de portée des
Confidential Transactions beaucoup moins volumineuses.

Au-delà de ces optimisations, il n'existe pas de manière d'augmenter le
volume transactionnel de la chaîne sans faire de compromis au niveau du
modèle de Bitcoin. La solution consiste ainsi à modifier le comportement
du système, de telle manière qu'il n'affecte pas trop le modèle de
sécurité. Plusieurs facteurs peuvent ainsi être optimisés, dont la
taille de la chaîne à conserver, l'\emph{Initial Block Download} (IDB)
et la taille de l'ensemble des UTXO.

D'abord, on peut choisir de supprimer les blocs les plus anciens une
fois qu'on les a vérifiés. On conserve simplement la chaîne des entêtes,
l'état du réseau, ainsi que les blocs les plus récents afin de pouvoir
rejoindre le consensus dans le cas d'une recoordination profonde. Cette
méthode est appelée l'élagage ou \emph{pruning}.

Mais cette méthode n'enlève pas la charge de l'IBD, c'est-à-dire le
processus de téléchargement et de vérification de la chaîne de blocs
jusqu'à sa hauteur actuelle. Pour ce faire, on peut procéder à diverses
techniques plus ou moins risquées. La première est la supposition de
validité des signatures, basée sur le paramètre \texttt{assumevalid},
qui a été introduite dans Bitcoin Core en 2017\footnote{Bitcoin Core,
  \emph{Bitcoin Core 0.14.0}, 8 mars 2017~:
  \url{https://bitcoincore.org/en/2017/03/08/release-0.14.0/\#assumed-valid-blocks}.},
et qui consiste à sauter la vérification des signatures jusqu'à un bloc
d'empreinte donnée, faisant gagner beaucoup de temps dans la
synchronisation initiale. Cette méthode n'est pas un point de contrôle
(elle n'impose pas au bloc d'exister) et le risque qu'elle comporte est
minime. La deuxième technique est AssumeUTXO, qui a été proposée en 2019
par James O'Beirne et est toujours en développement\footnote{James
  O'Beirne, \emph{AssumeUTXO Proposal}, 24 avril 2019~:
  \url{https://github.com/jamesob/assumeutxo-docs/tree/2019-04-proposal/proposal}.},
et qui implique de supposer valide un ensemble des UTXO donné (identifié
par son empreinte) à une hauteur de bloc déterminée~: l'opérateur de
nœud télécharge la sauvegarde de l'ensemble des UTXO auprès d'un tiers
et débute la synchronisation initiale à partir de celui-ci, remettant à
plus tard (ou ignorant complètement) le téléchargement et la
vérification des blocs précédents. Cette méthode présente un défaut de
vérification (au moins temporaire) de sorte que l'opérateur est exposé à
la tromperie, mais le risque est considéré comme acceptable\footnote{«~Cette
  méthode présente un défaut de vérification (au moins temporaire) de
  sorte que l'opérateur est exposé à la tromperie, mais le risque est
  considéré comme acceptable~»~: Les risques liées à AssumeValid et à
  AssumeUTXO sont discutés dans le chapitre~5 de l'ouvrage
  \emph{Bitcoin: A Work in Progress} de Sjors Provoost publié en 2022.}.
Il existe également une troisième technique plus radicale, l'engagement
des UTXO (ou \emph{UTXO commitments}), qui est un soft fork obligeant
les mineurs à ajouter l'empreinte de l'ensemble des UTXO dans le
bloc\footnote{Mark Friedenbach, \emph{{[}soft fork{]} Block v3: miner
  commitments with compact proofs}, 28 mars 2014~:
  \url{https://github.com/bitcoin/bitcoin/pull/3977}~; Pieter Wuille,
  \emph{{[}bitcoin-dev{]} Rolling UTXO set hashes}, /05/2017 20:01:14
  UTC~:
  \url{https://lists.linuxfoundation.org/pipermail/bitcoin-dev/2017-May/014337.html}.}~:
cet engagement permettrait de disposer d'une source bien plus fiable
pour télécharger la sauvegarde à partir de laquelle commencer la
synchronisation.

Ensuite, au-delà de l'IBD, reste le problème de la taille de l'ensemble
des UTXO, qui est l'un des facteurs limitants les plus importants. La
première idée pour réduire cette taille est une proposition de Cory
Fields appelée UHS (pour \emph{UTXO Hash Set}) qui consiste uniquement à
stocker les empreintes (\emph{hashes}) des UTXO
individuelles\footnote{Cory Fields, \emph{UHS: Full-node security
  without maintaining a full UTXO set}, /05/2018 16:36:35 UTC~:
  \url{https://lists.linuxfoundation.org/pipermail/bitcoin-dev/2018-May/015967.html}.}.
La deuxième idée est de se servir d'accumulateurs cryptographiques,
comme l'a fait Thaddeus Dryja avec sa proposition nommée Utreexo, qui
implique de regrouper les UTXO dans des arbres de Merkle afin de
condenser l'ensemble à conserver en mémoire, au prix d'un compromis sur
la bande passante\footnote{Thaddeus Dryja, \emph{Utreexo: A dynamic
  hash-based accumulator optimized for the Bitcoin UTXO set}, 6 juin
  2019~: \url{https://eprint.iacr.org/2019/611.pdf}.}.

Enfin, on peut également choisir d'éclater le minage et la vérification
des transactions en séparant l'historique et l'état du système en
plusieurs fragments, qui sont chacun soutenus par une partie (variable)
du réseau. C'est ce qu'on appelle le partitionnement ou \emph{sharding}.
C'était l'idée derrière l'utilisation d'un arbre préfixe de
Merkle-Patricia (aussi appelé arbre Merklix) vaguement envisagée par les
développeurs de Bitcoin Cash\footnote{«~vaguement envisagée par les
  développeurs de Bitcoin Cash~»~: Amaury Séchet, \emph{Using Merklix
  tree to shard block validation}, 6 novembre 2016~:
  \url{http://www.deadalnix.me/2016/11/06/using-merklix-tree-to-shard-block-validation},
  archive~:
  \url{https://web.archive.org/web/20170716220359/https://www.deadalnix.me/2016/11/06/using-merklix-tree-to-shard-block-validation/}~;
  Joannes Vermorel, Amaury Séchet, Shammah Chancellor, Jason Cox,
  \emph{Merklix tree for Bitcoin}, juillet 2018~:
  \url{https://blog.vermorel.com/pdf/merklix-tree-for-bitcoin-2018-07.pdf}.},
ou le \emph{danksharding} qui pourrait être implémenté dans
Ethereum\footnote{«~le danksharding qui pourrait être implémenté dans
  Ethereum~»~: \url{https://ethereum.org/en/roadmap/danksharding/}.}.
Cependant, il s'agit là d'une modification importante du protocole qui
pourrait ne jamais être implémentée dans une version de Bitcoin.

Ces propositions sont des compromis se faisant au niveau de la chaîne
qui affectent souvent le système dans son entièreté. Toutefois, il est
également possible de faire un compromis différent, au niveau des pièces
individuelles, par l'utilisation de banques et, surtout, de surcouches.

\section*{Les banques et les
surcouches}\label{les-banques-et-les-surcouches}
\addcontentsline{toc}{section}{Les banques et les surcouches}

\markright{Les banques et les surcouches}

Les autres propositions généralement citées comme alternatives à
l'augmentation de la taille des blocs sont des solutions consistant à ne
pas réaliser tous les transferts sur la chaîne, mais à en déporter les
plus petits ailleurs, ceux-ci étant «~regroupés~» dans des transactions
plus grosses. La chaîne est alors utilisée pour régler les dettes,
contractées de manière analogique (contrat juridique) ou numérique
(contrat autonome). Cela consiste à considérer le protocole comme un
protocole de règlement.

La première manière de faire est de réintroduire de la confiance dans le
système en contractant des obligations de manière traditionnelle, auprès
de ce que nous appellerons ici des banques. Les banques en question
peuvent émettre une monnaie représentative en gardant l'intégralité des
fonds, ou bien offrir du crédit en ne conservant que des réserves
fractionnaires. L'utilisation de la chaîne sert au règlement entre les
banques, qui assure le transfert de fonds entre leurs clients. C'est en
somme le modèle de la banque libre promu par George Selgin et Larry
White dans les années 1990.

Cette première conception a été défendue par Hal Finney, qui avait
connaissance des travaux de Selgin et de White, comme nous l'avons vu
dans le chapitre~\hyperref[ch:cybermonnaie]{6}. Le 30 décembre 2010, il
faisait ainsi l'apologie d'un modèle de banque libre basé sur le
bitcoin~:

«~En fait, il existe une très bonne raison pour que les banques basées
sur Bitcoin existent et émettent leur propre monnaie numérique,
convertible en bitcoin. Bitcoin lui-même ne peut pas passer à l'échelle
pour que chaque transaction financière dans le monde soit diffusée
publiquement et incluse dans la chaîne de blocs. Il doit y avoir un
niveau secondaire de systèmes de paiement, plus léger et plus efficace.
{[}...{]} Les banques basées sur Bitcoin résoudront ces problèmes. Elles
pourront fonctionner comme les banques le faisaient avant la
nationalisation de la monnaie. Les différentes banques pourront avoir
des politiques différentes, certaines plus agressives, d'autres plus
conservatrices. {[}...{]} Je pense que tel sera le destin ultime du
bitcoin, à savoir être la ``monnaie de base'' qui sert de monnaie de
réserve aux banques qui émettent leur propre argent liquide. La plupart
des transactions en bitcoin se feront entre banques, pour régler les
transferts nets. Les transactions en bitcoin effectuées par des
particuliers seront aussi rares que... eh bien, que les achats en
bitcoin le sont aujourd'hui\footnote{Hal Finney, \emph{Re: Bitcoin
  Bank}, /12/2010 01:38:40
  UTC,\url{https://bitcointalk.org/index.php?topic=2500.msg34211\#msg34211}.}.~»

Cette vision a été reprise en 2018 par Saifedean Ammous dans son livre,
\emph{L'Étalon-Bitcoin}, dans lequel il soutenait que le rôle principal
du bitcoin était d'être une monnaie de réserve\footnote{«~Bitcoin peut
  être vu comme un système nouveau et émergent de monnaie de réserve
  pour les transactions en ligne, dans lequel les banques en ligne
  émettront des jetons adossés au bitcoin pour leurs utilisateurs, tout
  en gardant leurs réserves en bitcoins dans un stockage hors-ligne.
  Chaque individu pourra auditer en temps réel les possessions de
  l'intermédiaire, et des systèmes de vérification et de réputation
  permettront de s'assurer qu'aucune inflation n'a lieu.~» -- Saifedean
  Ammous, \emph{The Bitcoin Standard}, 2018.}. Cette thèse a été par la
suite développée par d'autres personnes comme Nik Bhatia\footnote{«~Cette
  thèse a été par la suite développée par d'autres personnes comme Nik
  Bhatia~»~: Nik Bhatia, \emph{Layered Money: From Gold and Dollars to
  Bitcoin and Central Bank Digital Currencies}, 2021.}.

Dans la réalité, cette manière de détourner l'activité de la chaîne
s'est effectivement matérialisée avec les places de marché, qui
permettaient de traiter les nombreux ordres d'achat et de vente liés à
la spéculation. Elle s'est aussi manifestée par le biais des plateformes
de casino qui regroupaient les opérations liées au jeu d'argent. Enfin,
elle a été mise en œuvre par les services fiduciaires comme Grayscale
qui offraient aux institutions financières la possibilité d'intégrer du
bitcoin à leur bilan.

Toutefois, il ne s'agit nullement d'un passage à l'échelle de Bitcoin.
Le traitement bancaire n'est pas résistant à la censure, ni résistant à
l'inflation, et sa généralisation conduirait \emph{in fine} à la
destruction totale de la proposition de valeur de Bitcoin. Ainsi, on
peut vraisemblablement supposer qu'une telle solution ne peut
fonctionner qu'à petite échelle, pour des montants modestes, dans la
mesure où l'État ne va pas intervenir, comme dans le cas de Bitcoin
Beach au Salvador.

La deuxième variante de cette solution est de passer, non plus par des
contrats juridiques reposant sur la confiance, mais par des contrats
autonomes, dans le but de gérer les transferts en dehors de la chaîne.
L'idée est ainsi de minimiser la confiance pour rendre le procédé
viable. C'était par exemple la démarche derrière les \emph{fidelity
bonds}, proposés par Peter Todd en 2013, dont le but était de réduire
l'influence du tiers en préservant de la confidentialité financière des
clients et en permettant d'auditer efficacement les banques\footnote{Peter
  Todd, \emph{Fidelity-bonded banks: decentralized, auditable, private,
  off-chain payments}, /02/2023 17:49:34 UTC~:
  \url{https://bitcointalk.org/index.php?topic=146307.msg1553349\#msg1553349}.}.

Cette démarche s'est popularisée au moyen de ce qu'on appelle
généralement le passage en surcouche (\emph{layering}) qui consiste à
déporter l'activité financière vers des protocoles ouverts et
décentralisés, préservant partiellement les propriétés de la chaîne.
L'idée est de condenser une multitude de transferts en un petit nombre
de transactions effectuées sur la couche de base, c'est-à-dire la chaîne
de blocs. Cette terminologie est issue de la décomposition en couches de
la suite des protocoles Internet, qui est organisée en couches multiples
dépendant l'une de l'autre, comme TCP qui dépend de IP.

Dans le passage en surcouche, le compromis de sécurité est partiel
(seuls certains bitcoins sont concernés) et limité dans le temps (ces
bitcoins peuvent être récupérés sur la chaîne), par opposition au
compromis de sécurité total et persistant imposé par l'augmentation de
la taille limite des blocs. Il s'agit d'une méthode conforme au modèle à
deux couches que Nick Szabo imaginait pour bit gold, avec une couche de
base dont le rôle était de garantir la rareté infalsifiable de la
monnaie, et une couche supérieure qui permettait de réaliser les
paiements effectifs.

Il existe ainsi une diversité de propositions permettant d'effectuer ce
passage en surcouche en réalisant un compromis plus ou moins important.
Les principales sont les chaînes latérales, le réseau Lightning et
Fedimint, dont nous parlerons en détail par la suite. Il existe
également d'autres propositions comme l'échange d'objets physiques
(OpenDime\footnote{«~OpenDime~»~: Les clés Opendime de Coinkite sont les
  produits les plus réputés pour l'échange physique en dehors de la
  chaîne. L'utilisateur peut vérifier que le scellé d'une clé n'a pas
  été brisé et que le contenu de celle-ci correspond au montant indiqué,
  de sorte qu'il peut l'accepter en tant que moyen de paiement. L'un des
  inconvénients majeurs est que la perte et le vol sont beaucoup plus
  faciles que dans le cas d'une portefeuille numérique bien géré. --
  Voir \url{https://opendime.com/}.}), le protocole Rumple\footnote{«~le
  protocole Rumple~»~: Fiatjaf, \emph{idea: Rumple} /10/2020 21:42 UTC~:
  \url{https://fiatjaf.com/rumple.html}.}, les statechains\footnote{«~les
  statechains~»~: Ruben Somsen, \emph{Statechains: Non-custodial
  Off-chain Bitcoin Transfer}, 4 juin 2019~:
  \url{https://medium.com/@RubenSomsen/statechains-non-custodial-off-chain-bitcoin-transfer-1ae4845a4a39}.},
les ZK-rollups\footnote{«~les ZK-rollups~»~: Rollkit, \emph{Sovereign
  rollups on Bitcoin with Rollkit}, 5 mars 2023~:
  \url{https://rollkit.dev/blog/sovereign-rollups-on-bitcoin-with-rollkit}~;
  archive~:
  \url{http://web.archive.org/web/20230511021256/https://rollkit.dev/blog/sovereign-rollups-on-bitcoin/}.}
ou encore le protocole Ark\footnote{«~le protocole Ark~»~: Kudzai
  Kutukwa, «~\emph{Introducing Ark: An Alternative Bitcoin Scaling
  Solution Focused on Preserving Privacy}~», \emph{Bitcoin Magazine}, 11
  juin 2023~:
  \url{https://bitcoinmagazine.com/technical/how-ark-plans-to-scale-private-bitcoin-payments}.}.

\section*{Les chaînes latérales}\label{les-chauxeenes-latuxe9rales}
\addcontentsline{toc}{section}{Les chaînes latérales}

\markright{Les chaînes latérales}

Les chaînes latérales, ou \emph{sidechains} en anglais, sont des chaînes
de blocs secondaires fonctionnant parallèlement à une autre chaîne de
blocs dite «~principale~». Elles ont été formalisées en octobre 2014 par
les développeurs de Blockstream\footnote{Adam Back, Matt Corallo, Luke
  Dashjr, Mark Friedenbach, Gregory Maxwell, Andrew Miller, Andrew
  Poelstra, Jorge Timón, Pieter Wuille, \emph{Enabling Blockchain
  Innovations with Pegged Sidechains}, 22 octobre 2014~:
  \url{https://blockstream.com/sidechains.pdf}.}. Cette solution
technique apporte une capacité de traitement supplémentaire et une
extensibilité supérieure, au prix d'une sécurité locale sensiblement
amoindrie. En 2014, Blockstream envisageait de construire ainsi tout un
écosystème de chaînes latérales permettant d'accomplir des tâches
impossibles sur la chaîne principale comme l'émission d'actifs natifs,
le déploiement de contrats autonomes avancés ou la gestion de noms de
domaine.

Une chaîne latérale est une chaîne de blocs parallèle à une autre qui
permet de transférer des fonds d'une chaîne à l'autre sans mettre en jeu
l'intégrité des fonds déplacés. Il s'agit généralement de procéder à un
ancrage bilatéral (\emph{two-way peg}) permettant aux bitcoins d'être
transférés d'une chaîne à l'autre sans perte de valeur, comme représenté
sur la figure~\hyperref[fig:sidechain-two-way-peg]{14.1}. Dans un sens,
les bitcoins sont verrouillés sur la chaîne principale et leur
équivalent est créé sur la chaîne latérale~; dans l'autre, les bitcoins
sont détruits sur la chaîne latérale et leur équivalent est déverrouillé
sur la chaîne principale.

Deux aspects différencient le modèle de sécurité d'une chaîne latérale
de celui de la chaîne principale~: le maintien de l'ancrage bilatéral et
le mécanisme de consensus. Le premier consiste à décider qui peut
déverrouiller les fonds lors d'un transfert de la chaîne latérale vers
la chaîne principale. En effet, puisque la chaîne latérale est voulue
comme un complément (et pas comme une extension), les nœuds de la chaîne
principale n'ont pas connaissance de cette chaîne latérale. De ce fait,
le retrait est soumis à une certaine confiance, placée usuellement dans
une fédération de participants qui se méfient les uns des
autres\footnote{«~une fédération de participants qui se méfient les uns
  des autres~»~: Johnny Dilley, Andrew Poelstra, Jonathan Wilkins, Marta
  Piekarska, Ben Gorlick, Mark Friedenbach, \emph{Strong Federations: An
  Interoperable Blockchain Solution to Centralized Third-Party Risks},
  2016~: \url{https://blockstream.com/strong-federations.pdf}.}.

\begin{figure}

{\centering \includegraphics{chapters/img/sidechain-two-way-peg.png}

}

\caption{Chaîne latérale~: dépôt, transfert et retrait.}

\end{figure}%

Le second aspect concerne la confirmation des transactions sur la chaîne
latérale, et ici les options sont plus variées. Le consensus peut
reposer sur le minage combiné, auquel cas c'est le travail de la chaîne
principale qui est utilisé. Il peut se fonder sur la preuve d'enjeu,
auquel cas c'est l'unité de la chaîne principale qui est impliquée. Ou
il peut recourir à une fédération se mettant d'accord grâce à un
algorithme BFT classique, auquel cas c'est l'appartenance à cette
fédération qui importe (preuve d'autorité).

Cette vision s'est concrétisée avec le lancement sur BTC de deux chaînes
latérales distinctes en 2018. La première était RSK (aussi appelée
Rootstock), qui a été lancée par Sergio Lerner en janvier de cette
année-là et qui était focalisée sur l'exécution d'une machine virtuelle
Turing-complète s'approchant de celle d'Ethereum. La seconde était
Liquid, qui était la mise en œuvre du modèle Elements développé par
Blockstream et dont le but primaire était de faciliter les transactions
entre les différents acteurs financiers du secteur, dont notamment les
plateformes d'échange. Dans Liquid, la sécurité repose sur une
fédération de fonctionnaires qui assurent les deux rôles~: ils
maintiennent l'ancrage du L-BTC en tant que gardiens (\emph{watchmen})
et participent au consensus de la chaîne en tant que signataires de
blocs (\emph{blocksigners}). RSK allie le minage combiné et une
fédération de «~notaires~» pour assurer à la fois l'ancrage du RBTC et
le traitement des transactions.

Les deux chaînes latérales n'ont cependant pas réussi à attirer une
activité significative au fil des années, en raison des risques liés. En
effet, utiliser ces chaînes requiert toujours une forme de confiance
qui, bien que réduite au possible, reste bien présente. Un exemple
malheureux d'une sidechain qui a mal tourné est celui de la chaîne
latérale SmartBCH de Bitcoin Cash, où la société qui gérait le plus gros
pont entre les deux chaînes, CoinFLEX, a fait faillite et n'a pas été en
mesure de rembourser les utilisateurs\footnote{«~la chaîne latérale
  SmartBCH de Bitcoin Cash~»~: CheapLightning, \emph{The Resolution of
  the smartBCH Experiment}, 2 août 2022~:
  \url{https://read.cash/@CheapLightning/the-resolution-of-the-smartbch-experiment-b06eb075}.}.

Pour répondre à ces inconvénients et réduire la confiance impliquée, un
protocole plus avancé a été élaboré par le chercheur Paul Sztorc depuis
novembre 2015~: Drivechain\footnote{Paul Sztorc, \emph{Drivechain - The
  Simple Two Way Peg}, 24 novembre 2015~:
  \url{https://www.truthcoin.info/blog/drivechain/}.}. Comme son nom
l'indique (\emph{drive chain} signifie chaîne de transmission), il
s'agit d'une véritable machine de création et de gestion de chaînes
latérales.

La caractéristique principale de Drivechain est que l'ancrage bilatéral
est confié aux mineurs, grâce au dépôt fiduciaire par taux de hachage
(\emph{hashrate escrow}) défini dans le BIP-300\footnote{«~BIP-300~»~:
  Paul Sztorc, CryptAxe, \emph{BIP-300: Hashrate Escrows}, 23 mai 2017~:
  \url{https://github.com/bitcoin/bips/blob/master/bip-0300.mediawiki}.}.
Durant chaque période de six mois (26~300~blocs), les mineurs votent
pour la transaction de retrait de la chaîne latérale distribuant les
fonds aux utilisateurs en ayant fait la requête. La transparence et la
lenteur de ces transactions permettent à l'ensemble des commerçants de
la chaîne principale de les auditer. Les transferts courants, plus
rapides, se font par des échanges atomiques ou par des services
centralisés.

La validation des transactions sur la chaîne latérale utilisant
Drivechain peut être assurée par n'importe quel algorithme de consensus.
Mais le plus naturel est d'utiliser le minage combiné. C'est pourquoi le
projet Drivechain contient également la proposition du minage combiné
«~aveugle~» (BIP-301\footnote{«~BIP-301~»~: Paul Sztorc, CryptAxe,
  \emph{BIP-301: Blind Merged Mining}, 23 juillet 2019~:
  \url{https://github.com/bitcoin/bips/blob/master/bip-0301.mediawiki}.}),
une technique permettant aux mineurs de la chaîne principale de déléguer
automatiquement la validation d'une chaîne latérale à autrui contre une
rémunération. Le validateur gagne la différence entre le revenu de la
chaîne latérale et l'achat du «~droit de trouver un bloc~». Ceci a pour
effet de ne pas obliger les mineurs à gérer les chaînes latérales tout
en touchant une partie de leur revenu.

Drivechain est un concept astucieux qui aurait pour avantage de
pleinement réaliser la vision de Blockstream de 2014. Cependant, il
présente un inconvénient majeur~: le modèle de sécurité de son ancrage
bilatéral. Celui-ci repose en effet sur le recours éventuel à un soft
fork réalisé par les commerçants dans le but de corriger une transaction
de retrait frauduleuse, qui serait par exemple le fait de mineurs
malintentionnés cherchant à voler l'argent du dépôt fiduciaire. Il se
fonde donc sur la propension des commerçants à suivre l'activité de la
chaîne latérale en question d'une part, et à procéder à une modification
du protocole pour geler la transaction incriminée d'autre part. C'est
pourquoi cette proposition est, encore aujourd'hui en 2023, largement
disputée.

\section*{Le réseau Lightning}\label{le-ruxe9seau-lightning}
\addcontentsline{toc}{section}{Le réseau Lightning}

\markright{Le réseau Lightning}

Le réseau Lightning, ou \emph{Lightning Network} en anglais, est un
concept de réseau de canaux de paiements bidirectionnels. Celui-ci a été
présenté pour la première fois le 23 février 2015 par Joseph Poon et
Thaddeus Dryja lors d'un séminaire de développeurs Bitcoin à San
Francisco\footnote{Taariq Lewis, \emph{SF Bitcoin Devs Seminar: Scaling
  Bitcoin to Billions of Transactions Per Day}, 5 mars 2015~:
  \url{https://www.youtube.com/watch?v=8zVzw912wPo}.}. À l'époque, des
propositions concurrentes basées sur des idées similaires existaient
comme Amiko Pay (conceptualisé par Corné Plooy), Impulse (développé par
Jeff Garzik pour Bitpay) et Ström (imaginé par la start-up Strawpay),
mais Lightning est rapidement devenu dominant. En 2023, il s'agissait de
la solution favorisée par les utilisateurs de BTC pour effectuer
davantage de transferts, si bien que le sigle LNP/BP a émergé pour
désigner l'ensemble des protocoles intervenant dans le passage en
surcouche (à l'instar de TCP/IP vis-à-vis d'Internet)\footnote{«~le
  sigle LNP/BP a émergé pour désigner l'ensemble des protocoles
  intervenant dans le passage en surcouche~»~: Giacomo Zucco,
  \emph{LNP/BP: A gentle introduction}, 21 juillet 2020, archive~:
  \url{https://web.archive.org/web/20200820123506/https://alzashop.com/lnp-bp-lightning-netowrk-and-bitcoin-protocols}.}.

L'infrastructure du réseau Lighning repose sur des canaux de paiement
qui sont ouverts et fermés entre des participants. Un canal de paiement
est, comme décrit dans le chapitre~\hyperref[ch:contrats]{13}, un
ensemble de contrats autonomes qui permet à deux personnes d'effectuer
des paiements répétés de manière sûre et instantanée à partir de
liquidités préalablement bloquées. L'utilisation d'un canal est par
conséquent limitée par sa capacité, c'est-à-dire la somme des deux
soldes des acteurs concernés.

Le principe du réseau Lightning est de router les paiements au travers
de ces canaux par l'intermédiaire de HTLC, qui sont des contrats
d'engagement plus complexes permettant de mettre à jour les canaux
concernés\footnote{En pratique, ces HTLC sont souvent aussi utilisés
  pour mettre à jour les canaux directement, afin de simplifier la mise
  en œuvre et d'améliorer la confidentialité. -- Voir Andreas M.
  Antonopoulos, Olaoluwa Osuntokun, René Pickhardt, «~Routing on a
  Network of Payment Channels~», in \emph{Mastering the Lightning
  Network: A Second Layer Blockchain Protocol for Instant Bitcoin
  Payments}, O'Reilly Media, 2022, pp.~185--207.}. Un paiement transite
sur le réseau moyennant des frais minimes qui vont aux nœuds le
relayant. Le réseau Lightning est donc semblable à un boulier, dont les
tiges sont des canaux et dont les boules sont les satoshis qui
transitent d'un côté ou de l'autre des canaux, comme on peut le voir sur
la figure~\hyperref[fig:lightning-network-abacus]{14.2}.

\begin{figure}

{\centering \includegraphics{chapters/img/lightning-network-abacus.png}

}

\caption{Paiement de 3~mBTC sur le réseau Lightning.}

\end{figure}%

Ce fonctionnement offre la possibilité de réaliser des paiements quasi
instantanés et peu chers. Il permet de faire plus de transferts en
bitcoins sans réaliser davantage de transactions sur la chaîne et sans
déléguer explicitement la gestion des fonds à un tiers. De plus, le
modèle conserve toute la programmabilité de Bitcoin et ouvre le champ
des possibles quant à l'utilisation monétaire sur Internet.

Toutefois, l'apport de Lightning est à nuancer car il n'est pas exempt
de défauts. D'abord, il hérite des inconvénients liés au modèle des
canaux de paiement où, dans le cas du protocole de Poon-Dryja, une
erreur peut mener à la perte des fonds. Puis, les contraintes liées à la
capacité et au routage créent nécessairement une tendance à la
centralisation, notamment par l'émergence de ce qu'on appelle les
\emph{Lightning Service Providers}, ce qui pourrait mener à
l'installation d'une certaine censure. Ensuite, contrairement aux idées
reçues, la confidentialité sur Lightning est faible, les paiements se
faisant entre des clés publiques identifiées et transitant par des
intermédiaires. Enfin, le réseau est soumis au niveau des frais sur la
chaîne principale, ces derniers étant nécessaires au règlement des
contrats, ce qui limite la capacité transactionnelle supplémentaire
apportée.

Le réseau Lightning est donc adapté pour traiter les paiements du
quotidien et les micropaiements, ne nécessitant pas forcément la
confidentialité et la résistance à la censure offertes par la chaîne de
blocs, à partir de canaux bien pourvus et régulièrement
réapprovisionnés. Il a été mis en œuvre à partir de janvier 2018,
principalement en surcouche de BTC, et s'est développé considérablement
depuis, tant d'un point de vue technique qu'économique. Trois
implémentations logicielles ont été maintenues par trois entités
différentes (lnd par Lightning Labs, c-lightning par Blockstream, eclair
par ACINQ) et un système de standards techniques (appelés \emph{Bases of
Lightning Technology} ou BOLT) a fini par émerger. Du côté économique,
le réseau a rencontré un certain succès en attirant les capitaux et, en
novembre 2023, une capacité totale de 5~400~BTC, équivalant à environ
200 millions de dollars, était réservée pour fournir de la liquidité
pour les paiements\footnote{«~en novembre 2023, une capacité totale de
  5~400~BTC, équivalant à environ 200 millions de dollars~»~:
  \url{https://bitcoinvisuals.com/ln-capacity}.}.

\section*{Les banques chaumiennes de
Fedimint}\label{les-banques-chaumiennes-de-fedimint}
\addcontentsline{toc}{section}{Les banques chaumiennes de Fedimint}

\markright{Les banques chaumiennes de Fedimint}

Une autre proposition est Fedimint\footnote{Le fonctionnement de
  Fedimint est décrit dans la documentation présente sur le site web~:
  \url{https://fedimint.org/docs/intro}.}, qui est un protocole de garde
et d'échange confidentiel de bitcoins dans un contexte communautaire.
D'un point de vue technique, il s'agit de confier la garde des bitcoins
à une fédération et d'échanger des billets chaumiens (eCash) émis par
ladite fédération. Ce fonctionnement explique le nom du protocole, qui
est une abréviation approximative de \emph{Federated Chaumian Mint}
(«~monnaierie chaumienne fédérée~» en français).

Fedimint a été imaginé par le cypherpunk Eric Sirion au cours de l'année
2021 et implémenté sous forme minimale sous le nom de
MiniMint\footnote{«~MiniMint~»~: Kiara Bickers, \emph{Blockstream
  Sponsors Federated E-Cash as a Bitcoin Scaling Technology}, 26 octobre
  2021~:
  \url{https://medium.com/blockstream/blockstream-sponsors-federated-e-cash-as-a-bitcoin-scaling-technology-637ba05de7b3}.}.
Sirion a été doublement inspiré par les tentatives d'appliquer eCash en
surcouche de Bitcoin comme SCRIT\footnote{«~SCRIT~»~:
  \url{https://github.com/scritcash/scrit-whitepaper/blob/master/scrit-whitepaper.pdf}.}
et par les approches communautaires telles que Bitcoin Beach au
Salvador. La première transaction d'une fédération Fedimint a eu lieu le
28 septembre 2022 durant le Hackers Congress de Paralelni Polis.

La première composante de Fedimint est la banque chaumienne qui est
gérée par la fédération. Celle-ci utilise le procédé de signature
aveugle de David Chaum pour émettre des certificats adossés à un certain
montant de satoshis, qui peut être récupéré à tout moment sur la chaîne
ou sur le réseau Lightning. Cette composante assure la confidentialité
financière partielle des participants~: la banque n'a pas connaissance
des échanges réalisés par les clients, mais son rôle de prévention de la
double dépense exige qu'elle voie les revenus des
commerçants\footnote{Le fonctionnement technique des systèmes chaumiens
  a été décrit dans la section «~eCash~: l'argent liquide chaumien~» du
  chapitre~\hyperref[ch:cybermonnaie]{6}.}.

L'idée d'utiliser eCash en surcouche de Bitcoin n'est pas une idée
nouvelle. Elle a été proposée et implémentée pour la première fois le 17
août 2010, par un individu intervenant sous le pseudonyme
fellowtraveller sur le forum de Bitcoin sous la forme de son projet Open
Transactions\footnote{«~Open Transactions~»~: fellowtraveler, \emph{Open
  Transactions: untraceable digital cash}, /08/2010 20:58:05 UTC~:
  \url{https://bitcointalk.org/index.php?topic=847.msg9976\#msg9976}.}.
Le projet ne s'est jamais imposé, car le besoin ne se faisait pas
ressentir et le système était probablement trop complexe. Cependant,
l'idée est revenue timidement sur les devants de la scène au cours du
débat sur la scalabilité avec la proposition des «~certificats aveugles
au porteur\footnote{«~certificats aveugles au porteur~»~: theymos,
  \emph{Blinded bearer certificates}, /12/2016 21:44:24~:
  \url{https://www.reddit.com/r/Bitcoin/comments/5ksu3o/blinded_bearer_certificates/}.}~»
de Theymos (administrateur du subreddit r/Bitcoin et du forum
Bitcointalk) en décembre 2016. Elle a également été reprise en 2019 par
Frank Braun et Jonathan Logan (co-animateurs du podcast Cypherpunk
Bitstream) au moyen de SCRIT, un projet de système chaumien fédéré dont
le nom est l'acronyme de «~\emph{Secure, Confidential, Reliable, Instant
Transactions}~». Le dernier projet en date d'une mise en œuvre d'un
système chaumien centralisé est Cashu, un protocole développé en 2022
par le développeur callebtc, qui permet la création et l'échange de
certificats-bitcoin en surcouche de Lightning et de nouveaux
jetons\footnote{«~Cashu~»~: callebtc sur Twitter, /09/2022 09:47 UTC~:
  \url{https://twitter.com/callebtc/status/1569986110272540674}.}.

L'intérêt de Fedimint, tout comme son prédécesseur SCRIT, est de
décentraliser la garde de bitcoins. Pour ce faire, il combine le système
chaumien avec une approche dite «~communautaire~», consistant à déployer
une banque gérée par les membres de confiance d'une communauté locale.

Cette approche a été illustrée par l'expérience de Bitcoin Beach, un
projet de développement économique durable autour de la plage d'El Zonte
au Salvador. Une banque communautaire a ainsi vu le jour en 2020 et
permet depuis lors aux locaux d'échanger des bitcoins de façon sûre et
fiable, par le biais du Bitcoin Beach Wallet (devenu Blink) développé
par Galoy. C'est cette expérience qui a inspiré l'adoption du cours
légal à l'échelle nationale en septembre 2021.

La deuxième composante de Fedimint est donc une fédération, semblable
aux fédérations des chaînes latérales comme Liquid ou RSK, mais composée
de personnes de confiance qui possèdent les capacités techniques
nécessaires à la gestion d'un nœud. Les membres de cette fédération,
appelés gardiens, sont responsables de la mise en place de
l'infrastructure et se chargent de conserver les fonds des utilisateurs
et d'assurer le bon fonctionnement de la banque chaumienne. Ils se
coordonnent en utilisant un algorithme de consensus classique (appelé
HBBFT\footnote{«~HBBFT~»~: Il s'agit du sigle de \emph{Honey Badger
  Byzantine Fault Tolerant}.}) qui, comme tous les algorithmes de ce
type, demande un minimum de 66~\% d'acteurs honnêtes pour fonctionner.

L'emploi de cette fédération représente un compromis technique évident
entre la propriété entière des fonds et sa délégation auprès d'un acteur
unique. Ce compromis apporte des avantages majeurs au niveau des frais
de traitement des transactions et de la facilité d'utilisation, mais
engendre également des risques importants. Ceux-ci sont le risque de
garde (la fédération peut voler ou perdre des fonds), le risque
d'émission frauduleuse (elle peut émettre plus de certificats qu'elle
n'a de bitcoins), le risque de censure (elle peut refuser de valider une
transaction) et le risque réglementaire (la fédération peut être saisie
et fermée sur décision étatique).

Tout ceci fait que Fedimint ne peut pas être conçue comme une solution
de scalabilité, mais comme une proposition de remplacement des
applications dépositaires. L'objectif de Fedimint est d'améliorer la
garde de bitcoins en la décentralisant et en accroissant la
confidentialité des échanges internes. Son caractère local doit
permettre d'échapper aux réglementations financières, et ainsi de ne pas
subir le sort réservé aux banques classiques.

\section*{Le passage à l'échelle par
substitution}\label{le-passage-uxe0-luxe9chelle-par-substitution}
\addcontentsline{toc}{section}{Le passage à l'échelle par substitution}

\markright{Le passage à l'échelle par substitution}

Le passage en surcouche est une manière correcte d'accroître le volume
économique lié à une chaîne donnée sans trop affecter ses
caractéristiques premières. Néanmoins, cette approche présente aussi des
limites~: non seulement les différentes surcouches ont leurs défauts
propres, mais surtout elles reposent en dernier lieu sur le règlement
réalisé sur la chaîne de blocs, dont la capacité est limitée. Par
conséquent, le plancher d'utilité n'est pas supprimé par le passage en
surcouche et on ne peut ainsi pas voir ce dernier comme un moyen
miraculeux de traiter une infinité de transactions.

La plage de valeurs desservie par un système cryptomonétaire donné a
pour effet de créer une demande pour des systèmes de substitution plus à
même d'assurer le transfert hors de cette plage. Un système dont le
niveau de frais est élevé laisse la voie libre à l'utilisation d'un
système moins sûr mais moins cher, permettant le traitement des plus
petites transactions. À l'inverse, un système dont le niveau de sécurité
est faible favorise l'émergence d'un système plus cher mais aussi plus
sûr, autorisant les plus gros transferts. Il existe de ce fait une
certaine complémentarité entre les différentes mises en œuvre de Bitcoin
qui permettent de gérer l'intégralité de l'activité transactionnelle
émanant des utilisateurs\footnote{Eric Voskuil, «~Principe de
  substitution~», in \emph{Cryptoéconomie~: Principes fondamentaux de
  Bitcoin}, Amazon KDP, 2022, pp.~315--316.}.

Au cours de l'histoire, une telle complémentarité s'est manifestée par
l'utilisation de plusieurs métaux précieux comme base monétaire. L'or ne
pouvait pas permettre de couvrir toutes les plages de valeurs~: celui-ci
était adapté au transfert de grosses sommes, chose pour laquelle il a
été sélectionné comme monnaie de réserve mondiale, mais pas à l'échange
de petite monnaie. C'est pour remplir ce dernier rôle complémentaire que
l'argent, et d'autres métaux moins précieux comme le cuivre, ont été
utilisés. L'argent, mot qu'on utilise encore aujourd'hui en français
comme synonyme de monnaie, était la monnaie du quotidien tandis que l'or
servait essentiellement aux règlements plus onéreux.

Cet aspect bimétallique (voire trimétallique) de la monnaie a perduré
pendant des siècles, de la Haute Antiquité jusqu'au \textsc{xix} siècle.
Il était reconnu par les pouvoirs publics qui définissaient leur monnaie
comme un poids en or ou en argent, et frappaient des pièces d'or et
d'argent en décrétant un taux de change selon le ratio or-argent du
marché. On constate d'ailleurs que ce ratio or-argent a été relativement
stable au cours de l'histoire en variant entre 10 et 18\footnote{«~ce
  ratio or-argent a été relativement stable au cours de l'histoire en
  variant entre 10 et 18~»~: William Jacob, \emph{An Historical Inquiry
  Into the Production and Consumption of the Precious Metals}, 1831.},
ce qui confirme le rôle monétaire de l'argent aux côtés de l'or.

Toutefois, avec l'émergence de l'étalon-or et la disparition du
bimétallisme à la fin du \textsc{xix}~siècle, l'argent a peu à peu perdu
son rôle monétaire pour être remplacé par la monnaie papier, dans un
premier temps adossée à l'or, bien plus commode pour effectuer des
échanges. Le ratio a augmenté en conséquence et est passé de 15,5 en
1870 à 80 aujourd'hui, ce qui correspond à une perte de valeur de
l'argent de plus de 80~\% par rapport à l'or.

L'analogie avec les métaux précieux est éclairante. Puisque la version
principale de Bitcoin (BTC) n'est pas adaptée pour traiter les
transferts de plus petite valeur, il s'ensuit que ces transferts
potentiels sont réalisés au moyen d'une monnaie de substitution (de la
cryptomonnaie, de la monnaie fiat liquide, du crédit déplacé par des
services bancaires permissifs, etc.) voire ne sont pas traités du tout.
Litecoin, dont la principale narration est qu'il s'agirait d'un argent
numérique au même titre que Bitcoin serait un or numérique, répond tout
à fait à cette demande. Il était ainsi présenté dès son lancement comme
une «~version allégée de Bitcoin~» ayant pour but d'être «~à l'argent ce
que Bitcoin est à l'or\footnote{Charlie Lee, \emph{Re: {[}ANN{]}
  Litecoin - a lite version of Bitcoin. Be ready when is launches!},
  /10/2011 06:14:28 UTC~:
  \url{https://bitcointalk.org/index.php?topic=47417.msg564414\#msg564414}.}~».
Cette désignation ne provient pas tant du fait qu'il y a quatre fois
plus de litecoins que de bitcoins, ce qui n'a aucune incidence sur le
système, mais plutôt du fait que la capacité transactionnelle maximale
de LTC est quatre fois plus grande, ce qui amoindrit la sécurité
potentielle du système. Cette analyse vaut également pour Bitcoin Cash à
une échelle encore plus grande.

Dans cette vision, les mises en œuvre alternatives de Bitcoin
serviraient à traiter toutes les transactions, au prix de nécessaires
transferts entre les chaînes. Ces derniers seraient assurés par des
services de change centralisés ou par des systèmes d'échanges atomiques
basés sur des carnets d'ordres publics. Cette solution, bien
qu'imparfaite, serait tout à fait naturelle et est d'ailleurs déjà
pratiquée aujourd'hui.

L'extensivité est également concernée par cet effet. Le coût technique
d'une utilisation complexe de Bitcoin peut être compensé par des
systèmes de substitution de moindre qualité. La confidentialité à bas
coût peut être assurée par Monero et la programmabilité simplifiée par
Ethereum Classic par exemple. Comme le remarquait très justement Satoshi
Nakamoto en décembre 2010, à propos de la pertinence de BitDNS (le futur
Namecoin)~:

«~Empiler tous les systèmes de quorum par preuve de travail dans une
seule base de données ne passe pas à l'échelle. Bitcoin et BitDNS
peuvent être utilisés séparément. {[}...{]} Les réseaux ont besoin
d'avoir des destins différents. Les utilisateurs de BitDNS pourraient
être complètement tolérants vis-à-vis de l'ajout de fonctionnalités
permettant de traiter des données volumineuses puisque peu de
registraires de noms de domaine seraient nécessaires, tandis que les
utilisateurs de Bitcoin pourraient devenir de plus en plus sectaires à
propos de la limitation de la taille de la chaîne pour que son accès
reste facile pour beaucoup d'utilisateurs et pour les petits
appareils\footnote{Satoshi Nakamoto, \emph{Re: BitDNS and Generalizing
  Bitcoin}, /12/2010, 17:29:28~:
  \url{https://bitcointalk.org/index.php?topic=1790.msg28917\#msg28917}.}.~»

\section*{Trois types de compromis}\label{trois-types-de-compromis}
\addcontentsline{toc}{section}{Trois types de compromis}

\markright{Trois types de compromis}

La scalabilité de Bitcoin est un sujet complexe. Contrairement à ce qui
est parfois affirmé, un système donné est très peu scalable. Sa capacité
à passer à l'échelle ne peut être améliorée qu'au moyen d'optimisations
logicielles, matérielles ou algorithmiques. Le gain en performance sur
la chaîne se fait la plupart du temps au prix d'un compromis direct,
avec l'augmentation de la limite de capacité transactionnelle, ou
indirect, avec l'altération du modèle de sécurité.

C'est la raison de l'existence du passage en surcouche, qui consiste à
déporter une partie des transferts économiques vers des protocoles
ouverts et décentralisés, préservant partiellement les propriétés de
Bitcoin et reposant sur le règlement des litiges sur la chaîne. Dans
cette démarche, le compromis de sécurité est partiel et limité dans le
temps, contrairement au cas de l'augmentation de capacité
transactionnelle où il est total et persistant. Le passage de surcouche
s'est développé sur BTC au cours du temps par le biais des chaînes
latérales, proposées en 2014 et mises en œuvre en 2018, du réseau
Lightning, proposé en 2015 et déployé depuis 2018, et de Fedimint,
proposé en 2021.

L'autre alternative est le passage à l'échelle par substitution, qui
consiste, en substance, à déplacer les transactions les moins à risque
vers des substituts de moins bonne qualité, c'est-à-dire des mises en
œuvre moins sécurisées du concept Bitcoin. Cet effet s'est réellement
manifesté pour la première fois en 2017 avec les premières congestions
du réseau BTC et la hausse de la demande pour des contrats autonomes
statiques (Ethereum), qui se sont notamment accompagnées d'une baisse de
la dominance économique de la version principale de Bitcoin. Les
maximalistes ont tendance à prétendre que le passage en surcouche permet
de traiter l'ensemble des utilisations pertinentes de Bitcoin, mais,
jusqu'à preuve du contraire, ce n'est pas le cas.

\bookmarksetup{startatroot}

\chapter{L'avenir de Bitcoin}\label{ch:avenir}

\phantomsection\label{enotezch:15}{}

{L}\textsc{a} découverte de Bitcoin par Satoshi Nakamoto constitue une
révolution conceptuelle profonde dans le domaine monétaire. C'est ce qui
explique pourquoi, depuis 2008, il a suscité les plus grandes passions
tant au sein de ses partisans que parmi ses détracteurs. Certains ont
voulu y voir la solution à tous les problèmes de ce monde, une monnaie
universelle qui devait remplacer l'or et toutes les monnaies fiat, sans
résistance de la part de l'adversaire. D'autres ont tenté de le
présenter sous les traits d'un système nuisible et pollueur
d'escroquerie organisée, dans un rejet épidermique propre aux
institutions pour lesquelles ils travaillaient.

Dans cet ouvrage, nous avons tenté de faire la part des choses, en
décrivant précisément d'où vient Bitcoin, à quels enjeux il fait face et
quels sont les principes qui le soutiennent. Par sa conception, il
constitue un outil d'une rare élégance dont les mécanismes méritent
d'être détaillés, ce qui a été fait ici. En guise de conclusion,
résumons ce que nous avons développé avant de nous concentrer sur
l'avenir de Bitcoin en tant que tel.

\section*{L'élégance de Bitcoin}\label{luxe9luxe9gance-de-bitcoin}
\addcontentsline{toc}{section}{L'élégance de Bitcoin}

\markright{L'élégance de Bitcoin}

D'abord, rappelons que Bitcoin n'est pas sorti de nulle part. Il est un
produit de l'évolution technique qui a eu lieu durant la seconde moitié
du \textsc{xx}~siècle, en reposant largement sur l'ordinateur personnel,
sur la cryptographie asymétrique et sur le réseau Internet. Du côté
idéologique, il provient de mouvements divers, comme l'agorisme, le
librisme ou l'extropianisme, dont la particularité commune était
d'appeler à la pratique, de recommander d'agir dans le réel au lieu de
se contenter de le théoriser. En particulier, il est issu du mouvement
des cypherpunks qui, dès le début des années 90, préconisaient
d'utiliser la cryptographie de manière proactive en vue de protéger la
confidentialité et les droits des personnes dans le cyberespace
naissant. La valeur principale derrière Bitcoin est donc la liberté.

En outre, il est le résultat d'une longue quête pour la cybermonnaie,
qui avait notamment été entreprise par les cypherpunks. Bitcoin doit son
existence au système chaumien d'eCash, qui a eu son heure de gloire au
milieu des années 90 avant de disparaître. Il s'inspire des tentatives
de monnaies numériques privées comme le Liberty Dollar, e-gold et
Liberty Reserve, qui ont toutes été arrêtées par l'État à l'aube du
\textsc{xxi}~siècle. Il s'inscrit dans la lignée des concepts de monnaie
décentralisée qu'étaient b-money, bit gold, RPOW et, dans une certaine
mesure, Ripple.

Bitcoin a été découvert par Satoshi Nakamoto en 2007, qui en a publié le
livre blanc descriptif le 31 octobre 2008, avant de finaliser le
prototype et de lancer le réseau en janvier 2009. Après des débuts
difficiles, la cryptomonnaie a timidement émergé du néant en attirant à
elle les personnes intéressées par son potentiel. Ces personnes ont
contribué à construire Bitcoin en participant à son développement
informatique, au minage et au commerce. Une fois le projet
définitivement lancé en 2010, Satoshi a disparu progressivement et a
laissé la main à ses collaborateurs de confiance. Son anonymat demeure
complet à ce jour.

Après le départ du fondateur, la communauté a dû s'organiser. C'était
l'époque des premières conférences, des premières discussions autour de
l'avenir du protocole et du développement des premiers portefeuilles
légers. Cependant, la décentralisation du développement de Bitcoin a
fait qu'il n'y avait plus un seul point de vue dominant à son sujet, ce
qui a créé de multiples conflits, à commencer par la querelle de P2SH en
2011--2012. Quatre clivages majeurs ont émergé~: le premier concernait
la financiarisation, c'est-à-dire la réintroduction partielle de tiers
de confiance~; le deuxième se concentrait sur le passage à l'échelle, et
le choix de savoir s'il fallait augmenter la capacité transactionnelle
de la chaîne ou utiliser des solutions de surcouche~; le troisième
gravitait autour du développement des cryptomonnaies alternatives,
vivement décrié d'un côté et applaudi de l'autre~; le quatrième se
basait sur l'intégration institutionnelle, c'est-à-dire la question de
la coopération ou du rejet vis-à-vis de l'autorité. Ces conflits ont
fait de Bitcoin ce qu'il est aujourd'hui.

Bitcoin constitue une nouvelle forme de monnaie. Il s'agit d'un
intermédiaire d'échange dont la gestion est distribuée, c'est-à-dire
qu'elle ne repose pas sur une autorité centrale. Même si sa résistance
au changement le rapproche des biens tangibles, le bitcoin n'est pas une
monnaie-marchandise, car ses propriétés ne proviennent pas de
caractéristiques intrinsèques du monde physique. Même s'il reprend le
caractère numérique du système bancaire, ce n'est pas une monnaie
scripturale, car les entrées sur son registre ne correspondent pas à des
créances. Même s'il n'a pas d'utilisation non monétaire significative,
ce n'est pas une monnaie fiduciaire centralisée, car il ne repose pas
sur la confiance placée dans un acteur unique. En définitive, le bitcoin
appartient à une nouvelle catégorie et peut être décrit comme une
monnaie réticulaire (en référence à son réseau) ou une monnaie
fiduciaire distribuée, dans le sens où il répartit la confiance sur le
réseau de nœuds utilisés par les commerçants plutôt que de la concentrer
entre les mains d'une entité unique.

Bitcoin est un «~système d'argent liquide électronique pair à pair~» qui
permet «~aux paiements en ligne d'être envoyés directement d'une partie
à l'autre sans passer par une institution financière~». Il constitue un
concept de monnaie numérique résistante à la censure et à l'inflation,
qui rend difficile l'entrave des transactions et la création d'unités
supplémentaires. Bitcoin est un outil dont le domaine d'application
naturel se situe à la marge, à la limite de la légalité, voire dans
l'illégalité. Il est une monnaie de désobéissance utilisée par les
activistes politiques, par les lanceurs d'alerte et par les
organisations qui s'opposent à l'autorité. Il est une monnaie de la
liberté utilisée par les personnes censurées comme celles dont les
professions sont jugées déviantes, celles qui font l'erreur d'exprimer
une opinion discordante ou celles qui ont eu la malchance de naître dans
le mauvais pays. Il est une monnaie du marché noir utilisée par
l'économie souterraine, notamment dans le cadre de la résistance
fiscale.

Ce statut de monnaie de la liberté fait qu'il s'inscrit dans un rapport
antagoniste avec l'État, dont la nature est de chercher à s'étendre
toujours plus, notamment par l'affermissement de son contrôle sur la
monnaie. Par sa supervision de la banque, l'État a altéré le support de
la monnaie en la faisant reposer sur des pièces et des billets
fiduciaires plutôt que sur des métaux précieux, et il pourrait
recommencer demain en transformant la monnaie physique en une monnaie
numérique de banque centrale accessible à tous, sujette à la
surveillance et à la censure généralisées. Ce comportement prédateur de
l'État est la raison derrière le fonctionnement distribué de Bitcoin,
qui partage les risques entre les différents acteurs du système et
confère à ce dernier une robustesse sans précédent.

Bitcoin utilise un certain nombre de briques techniques pour fonctionner
correctement. La première est la signature numérique qui permet
d'assurer la propriété au sein du système. L'utilisateur peut posséder
pleinement ses bitcoins par le contrôle exclusif qu'il exerce sur ses
clés privées. Ce mécanisme offre la liberté unique de pouvoir gérer des
fonds numériques de manière souveraine, mais demande aussi une certaine
responsabilité vis-à-vis de la perte et du vol, qui n'existe pas dans le
cadre d'une relation avec un tiers de confiance.

Pour lutter contre la double dépense, Bitcoin repose sur un algorithme
de consensus novateur, qui met à jour une chaîne de blocs horodatés de
transactions, au moyen d'un procédé de preuve de travail. Son
fonctionnement ouvert et robuste le distingue des algorithmes de
consensus classiques qui avaient jusqu'alors été mis en œuvre au sein
des systèmes distribués. Le génie de Nakamoto est d'avoir sacrifié une
partie de la sécurité de l'algorithme (en la rendant probabiliste plutôt
qu'absolue) pour garantir la tolérance aux pannes byzantines. Ce modèle
se fonde sur les incitations économiques des mineurs, qui estiment que
miner la chaîne dans les règles est plus rentable que de l'attaquer.

Toutefois, le génie de la conception de Bitcoin ne s'arrête pas là.
Celle-ci ne décourage pas seulement la double dépense, mais aussi la
censure financière, qui constitue l'un des fléaux du transfert numérique
aujourd'hui. La censure de Bitcoin consiste à miner une chaîne plus
longue ne contenant pas les transactions indésirables. Grâce au paiement
intégré de frais de transactions et au caractère externe de la preuve de
travail, une telle suppression peut être combattue efficacement,
conformément à la propriété de résistance à la censure du modèle.

Bitcoin est un concept de monnaie ouvert et libre, de sorte qu'il est
par nature changeant et multiple. Il existe ainsi une diversité de mises
en œuvre de Bitcoin, qui est affectée par deux effets contraires~:
l'effet de réseau et l'effet de substitution. Ainsi, la nature monétaire
de Bitcoin fait qu'il ne peut que subsister qu'un petit nombre de ces
mises en œuvres, tandis que son absence de scalabilité invite à penser
qu'il en subsistera plusieurs.

La détermination du protocole, ou des protocoles, se fait de manière
économique, par le biais de l'acceptation de la monnaie par les
commerçants. Ces derniers ont le rôle le plus important en ayant le
dernier mot sur les règles de consensus grâce à leur activité économique
vérifiée par leurs nœuds. Plus généralement, le modèle de gouvernance
est en réalité bien plus complexe sociologiquement, les commerçants
étant influencés par d'autres personnes participant au système, comme
leurs clients, les détenteurs, les développeurs ou les mineurs, et d'une
manière plus diffuse, par des acteurs externes, tels que les relais
d'opinion, les puissances financières ou encore l'État.

La résistance à l'inflation, ou la difficulté à créer plus de bitcoins,
émerge ainsi de la dynamique économique opposée à l'altération de la
politique monétaire. Elle ne provient pas de l'absence d'unanimité de la
communauté ou de l'établissement originel de la politique monétaire par
Satoshi Nakamoto. La limite des 21~millions, en dépit de son caractère
emblématique, n'est ainsi pas absolue et dépend à chaque instant de la
décision des commerçants.

Le fonctionnement technique de Bitcoin est optimisé pour la monnaie,
comme en témoigne son modèle de représentation des unités qui se base
sur des pièces, et non sur des comptes comme Ethereum. Bien qu'aucune
technique avancée n'ait été intégrée dans le prototype, Bitcoin est
également conçu pour être confidentiel, la préservation de la vie privée
étant nécessaire pour la fongibilité de la monnaie et sa résistance à la
censure.

De plus, Bitcoin est programmable, de sorte qu'il est possible d'imposer
des conditions de dépense à différentes pièces. Cet aspect modulable des
transactions donne la possibilité à des inconnus d'échanger de la valeur
de manière la plus confidentielle et sûre possible. Il est aussi à la
base des protocoles de surcouche, comme le réseau Lightning, qui
augmentent la capacité de traitement des échanges sans compromettre la
sécurité du système de base.

Toutes ces propriétés font que Bitcoin forme un ensemble cohérent d'une
rare élégance. Bitcoin constitue la pièce manquante du puzzle de liberté
sur Internet. Bitcoin représente l'espoir d'une génération face à
l'autorité étatique grandissante. Bitcoin incarne le projet d'un système
monétaire alternatif robuste et durable. Et c'est ce qui explique le
formidable élan qui l'a accompagné dans les premières années.

\section*{Les quatre menaces qui planent sur
Bitcoin}\label{les-quatre-menaces-qui-planent-sur-bitcoin}
\addcontentsline{toc}{section}{Les quatre menaces qui planent sur
Bitcoin}

\markright{Les quatre menaces qui planent sur Bitcoin}

Comme nous l'avons évoqué tout au long de cet ouvrage, Bitcoin n'est pas
entièrement à l'abri des assauts de l'adversaire. Dans cette section,
nous évoquerons les principales menaces qui planent sur Bitcoin
aujourd'hui. Nous ne parlerons pas des risques techniques, que des
personnes mieux informées ont déjà traités\footnote{«~Nous ne parlerons
  pas des risques techniques, que des personnes mieux informées ont déjà
  traités~»~: Voir par exemple Sjors Provoost, \emph{Bitcoin: A Work in
  Progress}, 2022.}~; nous décrirons uniquement les dangers liés au
comportement humain, qui émanent de l'action des acteurs économiques du
système. Ces derniers sont en effet pour nous bien plus importants.

Les menaces humaines sont subtiles, car les attaques qu'elles facilitent
surviennent généralement de manière soudaine. L'accroissement de ces
menaces est similaire à une sorte de jeu de chaises musicales, où les
participants tournent naïvement autour des chaises sans les surveiller.
Tant que la musique retentit dans la pièce, tout va bien~: l'adversaire
enlève les chaises une par une, calmement, mais la ronde continue. C'est
au moment où la musique s'arrête que les problèmes se manifestent.

Nous distinguons quatre menaces de ce type susceptibles de nuire à
Bitcoin~: la centralisation de l'activité économique, la centralisation
de l'activité minière, la généralisation de la garde de fonds et
l'effacement de la confidentialité. Celles-ci ne sont pas entièrement
indépendantes, mais elles correspondent chacune à un comportement
différent des acteurs.

La première menace est la centralisation de l'activité économique, qui
émerge par l'intermédiaire du commerce important réalisé auprès des
plateformes de change réglementées et par le recours quasi systématique
à des processeurs de paiement externes et à des fournisseurs de
portefeuille tiers. Celle-ci peut mener, comme nous l'avons décrit dans
le chapitre~\hyperref[ch:determination]{11}, à une attaque d'altération
du protocole, sous la forme d'un hard fork d'inflation, d'un soft fork
taxatoire ou d'un soft fork de censure. Il est probable que cette
attaque crée une scission d'une façon ou d'une autre. Elle est
spécialement dommageable dans le cas où la chaîne altérée est
majoritaire en raison de l'effet de réseau. L'attaque n'est néanmoins
pas fatale pour le système car l'économie peut se reconstruire
progressivement à partir de la chaîne libre.

La deuxième menace est la centralisation de l'activité minière, qui se
manifeste notamment par le rapprochement géographique du matériel de
minage, par le regroupement des hacheurs en coopératives et par
l'utilisation collective de relais centralisés par les mineurs. Ce
risque peut mener, comme vu dans le chapitre~\hyperref[ch:censure]{9}, à
une attaque de censure des transactions par la majorité de la puissance
de calcul du réseau. Cette attaque a logiquement des chances de se
produire après la tentative d'altération du protocole, sur la chaîne
libre ayant refusé les modifications. Elle a pour effet de paralyser une
partie de l'activité en empêchant sa confirmation sur la chaîne. Elle
bénéficie de l'analyse de chaîne qui lui permet d'isoler les
transactions problématiques plutôt que de supprimer l'intégralité de
l'activité. Elle n'est cependant pas mortelle pour le système, car du
matériel de minage supplémentaire peut être déployé, suite à
l'accroissement des frais des transactions censurées, afin de restaurer
la situation initiale.

La troisième menace, apparentée à la centralisation de l'activité
économique, est la généralisation de la garde de fonds par des
dépositaires qui suivent les réglementations légales. Non seulement
cette pratique n'est pas pertinente du point de vue individuel (un
dépositaire peut censurer les transactions, saisir les fonds et gonfler
la quantité de bitcoins-papiers qu'il distribue), mais sa propagation
dans l'écosystème crée aussi un risque systémique. Cette menace se
manifeste aujourd'hui par le développement de dépositaires
institutionnels comme Coinbase Custody qui détiennent un pourcentage non
négligeable des bitcoins en circulation\footnote{«~dépositaires
  institutionnels comme Coinbase Custody qui détiennent un pourcentage
  non négligeable des bitcoins en circulation~»~:
  \url{https://platform.arkhamintelligence.com/explorer/entity/coinbase},
  \url{https://twitter.com/brian_armstrong/status/1595126425371414528}.}
et par l'accroissement des services adressés aux petits porteurs. Elle
est plus dangereuse que la centralisation de l'économie, car l'économie
«~hébergée~» ne peut pas se reformer s'il y a une attaque contre le
protocole~: ce sont les dépositaires réglementés qui sont les réels
propriétaires des bitcoins, pas leurs clients. Il s'agit donc d'une
dégénérescence persistante du système, qui se résorbe plus difficilement
qu'une simple centralisation minière ou commerciale.

La quatrième menace, plus insidieuse, est l'effacement de la
confidentialité, qui se matérialise par la surveillance généralisée
(connaissance du client, preuve de propriété d'adresse) et,
accessoirement, par l'analyse de chaîne qui l'accompagne. À l'instar de
la garde de fonds par une entité réglementée, la complète transparence
vis-à-vis de l'État constitue non seulement un errement individuel (la
personne n'est protégée ni de la censure, ni de la saisie), mais aussi
un risque systémique dans le cas où elle se généralise. En effet, une
surveillance plus grande crée une économie davantage contrôlable, et
rend par conséquent le protocole plus vulnérable. En outre,
l'identification des acteurs a pour conséquence de réduire l'ensemble
d'anonymat qui profite à tout le monde, et de diminuer la possibilité
d'exercer une activité secrète. L'effacement de la confidentialité forme
ainsi une dégénérescence subtile du système, qui ne peut être guérie que
par la lutte contre les liens d'identification \emph{via} l'application
de bonnes pratiques, comme le mélange des pièces.

Ces menaces dépendent des actions des acteurs économiques de Bitcoin, et
notamment de ses utilisateurs. Pour combattre ces menaces, il convient
donc de pousser les utilisateurs à retirer leurs bitcoins sur un
portefeuille, à arrêter de se soumettre à la connaisance du client, à
rendre leurs bitcoins intraçables et à utiliser leurs propres nœuds,
individuels ou communautaires. Cela concerne en particulier les nouveaux
utilisateurs, ce qui nous amène au thème de l'adoption.

\section*{Les deux adoptions de la
cryptomonnaie}\label{les-deux-adoptions-de-la-cryptomonnaie}
\addcontentsline{toc}{section}{Les deux adoptions de la cryptomonnaie}

\markright{Les deux adoptions de la cryptomonnaie}

Bitcoin est un système fondé sur des incitations économiques, dans
lequel les personnes qui le font vivre sont récompensées. D'une part,
les mineurs sont incités à confirmer les transactions pour toucher les
frais de transactions. D'autre part, les commerçants sont incités à
vérifier les règles de consensus pour bénéficier en toute quiétude de la
proposition de valeur de Bitcoin. En outre, les détenteurs sont incités
à promouvoir Bitcoin pour agrandir l'économie et profiter de la hausse
résultante du pouvoir d'achat (ou du prix en dollars) de l'unité de
compte. Cet agrandissement de l'économie, aussi appelé l'adoption,
constitue donc logiquement l'un des objectifs naturels de ceux qui
possèdent du bitcoin.

L'adoption de Bitcoin peut avoir lieu de multiples manières, mais deux
modèles principaux se distinguent. Le premier est l'adoption par les
individus et par les petites entreprises, qui correspond à un apport
financier modeste à la valeur agrégée du bitcoin. Le second est
l'adoption par les grandes entreprises, par les sociétés de courtage et
par les institutions financières, qui représente un plus gros gain pour
les détenteurs. Dans les premiers temps, il était impossible de
convaincre cette dernière catégorie du bienfondé du bitcoin, mais avec
le développement économique et grâce à une certaine conformité de la
communication, il est devenu aujourd'hui bien plus aisé de la persuader
d'y participer. Puisque cette adoption était beaucoup plus rentable pour
les détenteurs, beaucoup d'entre eux ont choisi la voie de la facilité
en remplissant leur discours d'éléments de langage destinés aux acteurs
réglementés.

Mais cette seconde adoption du bitcoin, bien qu'elle soit certainement
rentable sur le moment et qu'elle possède des mérites propres, a pour
particularité de devenir stérile à long terme. En effet, elle crée une
économie centralisée, surveillée voire entièrement dépositaire,
c'est-à-dire une économie fragile à la merci des décisions étatiques.
C'est pourquoi on peut la qualifier de «~mauvaise adoption~».

Ainsi, la seule adoption à laquelle il vaut la peine de s'intéresser est
celle de l'économie libre et indépendante, pour laquelle Bitcoin est
adapté en premier lieu. Cette économie possède en effet les
caractéristiques qui permettent à Bitcoin de perdurer. Elle est
décentralisée et répartit les risques entre tous ses membres, pour
bénéficier au maximum de la proposition de valeur de Bitcoin. Elle est
désobéissante, dans le sens où elle refuse toute modification du
protocole qui altérerait les propriétés fondamentales de Bitcoin. Elle
protège sa propre confidentialité, car elle sait qu'elle a quelque chose
à craindre de ceux qui la surveillent, quand bien même elle ne ferait
rien d'illégal sur le moment. Elle est circulaire, au sens où elle évite
le plus possible le recours à la monnaie étatique, surtout sous sa forme
numérique, car elle sent que cette dernière est de plus en plus
contrôlée. Enfin, elle est exigeante, en demandant de l'individu un
certain discernement et une certaine responsabilité, des qualités trop
souvent négligées à notre époque moderne.

Toutefois, les contraintes de cette «~bonne adoption~» font qu'elle
n'est pas accessible à tous. Non seulement l'utilisation souveraine de
Bitcoin demande d'être un minimum responsable, mais elle présente aussi
des inconvénients majeurs, qui sont (à l'heure actuelle) la volatilité
du pouvoir d'achat, le coût de transaction, le manque de scalabilité et
la réglementation dissuasive. De ce fait, il est difficile d'envisager
que tout le monde fera du bitcoin sa monnaie de prédilection à court ou
moyen terme. Pour le dire autrement~: l'adoption de masse n'aura pas
lieu de sitôt, et l'utilisation de Bitcoin restera dans un premier temps
confinée à la portion de la population qui cherche à s'extraire du
système étatico-bancaire et à résister aux puissances de ce monde.

Il est donc illusoire de s'attendre à une
«~hyperbitcoinisation\footnote{«~hyperbitcoinisation~»~: Daniel Krawisz,
  \emph{Hyperbitcoinization}, 29 mars 2014~:
  \url{https://nakamotoinstitute.org/mempool/hyperbitcoinization/}~;
  Pierre Rochard, \emph{Speculative Attack}, 4 juillet 2014~:
  \url{https://nakamotoinstitute.org/mempool/speculative-attack/}.}~»,
c'est-à-dire à un remplacement rapide des monnaies fiat par le bitcoin.
Tant qu'il y existe une masse de gens qui continuera d'obéir aveuglément
au pouvoir, la monnaie étatique subsistera. Seule la nécessité pourra
pousser cette masse à faire un usage opportuniste et temporaire de
Bitcoin.

\section*{Une culture en gestation}\label{une-culture-en-gestation}
\addcontentsline{toc}{section}{Une culture en gestation}

\markright{Une culture en gestation}

La culture est l'ensemble des aspects matériels, intellectuels,
affectifs et spirituels, qui caractérisent des sociétés ou des groupes
sociaux\footnote{«~La culture est l'ensemble des aspects matériels,
  intellectuels, affectifs et spirituels, qui caractérisent une société
  ou un groupe social~»~: UNESCO, «~\emph{Déclaration de Mexico sur les
  politiques culturelles}~», \emph{Conférence mondiale sur les
  politiques culturelles}, 26 juillet -- 6 août 1982~:
  \url{https://www.culture.gouv.fr/Media/Thematiques/Egalite-et-diversite/College-de-la-Diversite/Declaration-de-Mexico}~:
  «~Dans son sens le plus large, la culture peut aujourd'hui être
  considérée comme l'ensemble des traits distinctifs, spirituels et
  matériels, intellectuels et affectifs, qui caractérisent une société
  ou un groupe social. Elle englobe, outre les arts et les lettres, les
  modes de vie, les droits fondamentaux de l'être humain, les systèmes
  de valeurs, les traditions et les croyances.~»}. Chaque association
humaine durable a tendance à développer une culture propre. La
communauté de Bitcoin, bien que vaguement délimitée, n'échappe pas à ce
phénomène. Des éléments culturels ont émergé dans Bitcoin dès ses débuts
et se sont multipliés à mesure que le réseau grandissait, pour finir par
donner naissance à une véritable \emph{subculture}.

Cette culture est logiquement pétrie de politique, entre l'animosité à
l'égard des représentants de l'autorité et les références multiples aux
cypherpunks et aux économistes autrichiens. Elle est aussi constituée de
pratiques monétaires ritualisées, de recommandations hygiéniques
(notamment à l'encontre des crypto-actifs douteux), d'œuvres d'art
futuristes, de livres et de podcasts en tous genres, de regroupements
(rencontres mensuelles, conférences) et de commémorations régulières
d'évènements qui ont marqué l'histoire de la cryptomonnaie. La culture
de Bitcoin, la monnaie d'Internet, repose également beaucoup sur des
formules courtes répétées à foison et sur des mèmes humoristiques,
particulièrement adaptés pour la propagation sur les médias sociaux.

La culture, et plus précisément la part de la culture que l'on pourrait
qualifier de religieuse\footnote{«~la part de la culture que l'on
  pourrait qualifier de religieuse~»~: Émile Durkheim, \emph{Les formes
  élémentaires de la vie religieuse}, 1912~: «~Une religion est un
  système solidaire de croyances et de pratiques relatives à des choses
  sacrées, c'est-à-dire séparées, interdites, croyances et pratiques qui
  unissent en une même communauté morale, appelée Église, tous ceux qui
  y adhèrent.~»}, a pour conséquence d'orienter les actions des
individus. Puisque Bitcoin est un outil dont la sécurité dépend de
l'utilisation qui en est faite, cet aspect culturel est fondamental. Par
exemple la phrase «~\emph{not your keys, not your bitcoins}~» inventée
par Andreas Antonopoulos est bien plus convaincante pour pousser les
gens à placer leurs fonds dans des portefeuilles que n'importe quel
exposé historique des faillites et des gels de compte associés aux
plateformes dépositaires. Mais la culture peut aussi, par une mauvaise
orientation, induire de mauvais comportements et finalement nuire au
système.

En tant qu'objet spéculatif dont le prix a été multiplié par 30~millions
en l'espace de 14 ans, le bitcoin a attiré les personnes avides de gains
financiers. Celui-ci bénéficiait d'une rareté absolue par conception, ce
qui ne s'était jamais vu dans l'histoire, et il était normal qu'il en
soit ainsi. C'était là l'un des choix essentiels de Satoshi Nakamoto,
car cet attrait spéculatif a permis en partie d'amorcer le processus de
monétisation et de faire découvrir Bitcoin à des personnes qui s'en
seraient sinon détournées.

Cependant, la culture de Bitcoin en a été profondément influencée dans
le même temps. Il s'est ainsi créé une réelle tendance à l'avarice au
sein de la communauté, qui s'est réflétée par des mèmes et des formules
en tous genres. En particulier, il existe cette présupposition que le
nombre, c'est-à-dire le prix en dollars, doit monter (\emph{number go
up}), qu'il doit être propulsé «~jusqu'à la lune~» (\emph{to the moon})
en vertu du fait que la richesse du monde est infinie et qu'il n'y a que
21 millions de bitcoins (\(\infty / 21\)\footnote{«~infini sur 21~»~:
  Knut Svanholm, \emph{Bitcoin: Everything divided by 21 million}, 2022.}).
En conséquence, l'individu doit accumuler des satoshis (\emph{stack
sats}) et les thésauriser («~HODL\footnote{«~HODL~»~: GameKyuubi,
  \emph{I AM HODLING}, /12/2013 10:03:03 UTC~:
  \url{https://bitcointalk.org/index.php?topic=375643.msg4022997\#msg4022997}~;
  Coindesk, \emph{Maybe Don't HODL Bitcoin... -- Hodl Guy}, 11 janvier
  2019~: \url{https://www.youtube.com/watch?v=6lAPU2yP6rw}.}~») dans le
but de profiter d'une vie meilleure. Cet aspect se retrouve dans les
représentations de Bitcoin et des bitcoineurs, comme le taureau du
marché haussier\footnote{«~le taureau du marché haussier~»~: Vijay
  Boyapati, \emph{The Bullish Case for Bitcoin}, 2021.} ou bien les yeux
laser (\#LaserRayUntil100K).

Cette volonté d'obtenir un niveau de prix toujours plus haut repose sur
la délusion de l'adoption de masse que nous avons évoqué ci-dessus. Pour
que le prix atteigne les sommets, il faut en effet que tout le monde
finisse par posséder du bitcoin d'une manière ou d'une autre. Comme la
plupart des gens ne sont pas prêts à utiliser Bitcoin de manière
souveraine, cette adoption a été réalisée au moyen de dépositaires. De
ce fait, la culture basée sur le gain financier a conduit
l'affaiblissement subtil de Bitcoin par l'acception de l'installation
généralisée d'intermédiaires financiers, par le consentement à
l'identification de masse et par la promotion auprès des institutions et
des États.

L'adoption de masse n'est pas un objectif réaliste ni à court, ni à
moyen terme. Lorsque nous vantons les avantages de Bitcoin, nous ne nous
adressons pas à la masse proprement dite~; nous nous adressons au reste,
aux quelques-uns qui comprennent les tenants et aboutissants des
problèmes qu'il permet de résoudre et qui sont susceptibles d'être
intéressés\footnote{«~nous nous adressons au reste~»~: Albert Jay Nock,
  \emph{Isaiah's Job}, 1936~:
  \url{https://www.theatlantic.com/magazine/archive/1936/06/isaiahs-job/652293/}.}.
C'est pourquoi il est essentiel de ne pas aseptiser le discours~: pour
ne pas perdre ces personnes, il faut dire la vérité~; et si cette vérité
peut être voilée, elle ne doit jamais être déformée.

Bitcoin vit de la tension qui existe entre l'économie officielle, qui
approuve le pouvoir sur la monnaie, et la contre-économie, qui s'y
oppose. Du fait de cette tension, la culture cryptomonétaire est
également constamment attaquée, notamment par les médias de masse, par
les banquiers centraux et par les représentants de l'État. Il existe
ainsi un nombre stupéfiant de détracteurs qui, travaillant pour
l'adversaire\footnote{«~travaillant pour l'adversaire~»~: Upton
  Sinclair, \emph{I, Candidate for Governor, and How I Got Licked},
  1934~: «~Il est difficile de faire comprendre quelque chose à un homme
  lorsque son salaire dépend précisément du fait qu'il ne la comprenne
  pas.~»}, répètent à l'envi leur argumentaire de mauvaise foi. S'il est
utile de se confronter à eux pour rétablir la vérité devant un public
qui doute, il est vain de croire qu'ils disparaîtront ou perdront en
visibilité. C'est pourquoi Bitcoin a besoin d'une tradition, d'une
transmission culturelle d'individu à individu, qui permettrait
d'expliquer ses principes de manière saine et organique au nouveau venu.

En particulier, le message de Bitcoin devrait toujours être un appel à
la pratique, conformément aux mouvements idéologiques qui l'ont précédé,
à commencer par les cypherpunks. Chacun devrait se sentir poussé à
écrire (et à lire) du code, à déployer des fermes de minage dans la
mesure du possible, à participer à l'économie circulaire, à conserver du
bitcoin et à éduquer les autres sur le sujet, quand bien même cela
n'apporterait pas un gain financier direct. Car c'est aussi de cette
manière que Bitcoin prospère.

Quoi qu'il en soit, Bitcoin ne peut pas être oublié. La découverte de
Satoshi Nakamoto est là pour rester. Elle a déjà joué un rôle dans le
combat pour la liberté humaine et devra probablement jouer un rôle
encore plus grand à l'avenir. Son succès dépendra de l'action des
personnes qui le soutiennent. La révolution ne sera pas centralisée.

\bookmarksetup{startatroot}

\chapter{Ordinals}\label{ordinals}

Cet ouvrage est lié à 21 jetons non fongibles (NFT) émis grâce au
protocole Ordinals, dont les inscriptions ont pour identifiants~:

\begin{enumerate}
\def\labelenumi{\arabic{enumi})}
\item
  ``
\item
  ``
\item
  ``
\item
  ``
\item
  ``
\item
  ``
\item
  ``
\item
  ``
\item
  ``
\item
  ``
\item
  ``
\item
  ``
\item
  ``
\item
  ``
\item
  ``
\item
  ``
\item
  ``
\item
  ``
\item
  ``
\item
  ``
\item
  ``

  Ross Ubricht a réitéré cette manœuvre sur le \emph{Bitcoin Forum}, où
  il a écrit le 29 janvier~: «~Quelqu'un a-t-il déjà visité Silk Road~?
  C'est un peu comme un amazon.com anonyme. Je ne pense pas qu'il y ait
  de l'héroïne sur ce site, mais ils vendent d'autres choses. Ils
  utilisent essentiellement bitcoin et tor pour négocier des
  transactions anonymes.~» -- Ross Ulbricht, \emph{Re: A Heroin Store},
  /01/2011 19:44:51 UTC,
  \url{https://bitcointalk.org/index.php?topic=175.msg42670\#msg42670}.

  «~C'est pourquoi toutes les choses faisant objet de transaction
  doivent être d'une façon quelconque commensurables entre elles. C'est
  à cette fin que la monnaie a été introduite, devenant une sorte de
  moyen terme, car elle mesure toutes choses et par suite l'excès et le
  défaut.~» (trad. de J. Tricot, 1133a) Il explicite ensuite son rôle
  d'intermédiaire d'échange, qui est pour lui issu d'une convention
  légale~:

  «~Cet étalon n'est autre, en réalité, que le besoin, qui est le lien
  universel (car si les hommes n'avaient besoin de rien, ou si leurs
  besoins n'étaient pas pareils, il n'y aurait plus d'échange du tout,
  ou les échanges seraient différents)~; mais la monnaie est devenue une
  sorte de substitut du besoin et cela par convention, et c'est
  d'ailleurs pour cette raison que la monnaie reçoit le nom de
  \foreignlanguage{greek}{nomisma}, parce qu'elle existe non pas par
  nature, mais en vertu de la loi (\foreignlanguage{greek}{nomos}), et
  qu'il est en notre pouvoir de la changer et de la rendre
  inutilisable.~» (trad. de J. Tricot, 1133a) Il attribue enfin à la
  monnaie une fonction de réserve de valeur dans le temps~:

  «~Pour les échanges éventuels, dans l'hypothèse où nous n'avons besoin
  de rien pour le moment, la monnaie est pour nous une sorte de gage,
  donnant l'assurance que l'échange sera possible si jamais le besoin
  s'en fait sentir, car on doit pouvoir, en remettant l'argent, obtenir
  ce dont on manque.~» (trad. de J. Tricot, 1133b)

  «~Pendant des années, j'ai été frustré et démoralisé par ce qui
  semblait être des barrières insurmontables entre le monde actuel et le
  monde que je voulais. J'ai longtemps cherché la vérité sur ce qui est
  bien, mal et bon pour l'humanité. J'ai discuté, appris et lu les
  œuvres de personnes brillantes à la recherche de la vérité. C'est une
  chose sacrément difficile à faire avec toute la désinformation et les
  distractions présentes dans l'océan d'opinions où nous vivons. Mais
  j'ai fini par trouver quelque chose avec quoi je pouvais être
  entièrement d'accord. Quelque chose qui avait du sens, qui était
  simple, élégant et cohérent dans tous les cas. Je parle de la théorie
  économique autrichienne, du volontarisme, de l'anarcho-capitalisme, de
  l'agorisme, etc. embrassés par des gens comme Mises et Rothbard avant
  leur mort, et Salerno et Rockwell aujourd'hui.

  Grâce à leurs travaux, j'ai compris les mécanismes de la liberté et
  les répercutions de la tyrannie. Mais une telle vision était une
  malédiction. Partout où je posais les yeux, je voyais l'État et
  l'horrible effet d'étiolement qu'il avait sur l'esprit humain. C'était
  horriblement déprimant. C'était comme se réveiller d'un rêve agité
  pour se retrouver dans une cage sans échappatoire. Mais j'ai aussi vu
  des esprits libres essayant de se libérer de leurs chaînes, faisant
  tout ce qu'ils pouvaient pour servir leur prochain et subvenir à leurs
  besoins et à ceux de leurs proches. J'ai vu l'effet magique et
  puissant de création de richesse du marché, la façon dont il
  encourageait la coopération, la courtoisie et la tolérance. Comment il
  transformait les étrangers, ou même les ennemis, en partenaires
  commerciaux. Comment il coordonnait les actions de chaque personne sur
  la planète d'une manière trop complexe pour qu'un seul esprit puisse
  l'imaginer, afin de produire une abondance débordante de richesses, où
  rien n'est gaspillé et où le pouvoir et la responsabilité sont donnés
  aux les personnes les plus méritantes et les plus capables. J'ai vu
  une meilleure voie, mais je ne connaissais aucun moyen d'y parvenir.

  J'ai lu tout ce que je pouvais pour approfondir ma compréhension de
  l'économie et de la liberté, mais tout était cérébral et il n'y avait
  pas d'appel à l'action, si ce n'est dire aux gens autour de moi ce que
  j'avais appris et espérer leur faire voir la lumière. C'était jusqu'à
  ce que je lise "Alongside night" et les travaux de Samuel Edward
  Konkin \textsc{iii}. La pièce manquante du puzzle était enfin là~!
  Tout d'un coup, tout était clair : chaque action qu'on entreprenait en
  dehors du champ de contrôle du gouvernement renforçait le marché et
  affaiblissait l'État. J'ai vu comment l'État vivait de façon
  parasitaire aux dépens des personnes productives du monde, et à quelle
  vitesse il s'effondrerait s'il n'obtenait pas ses recettes fiscales.
  Pas de soldats si vous ne pouvez pas les payer. Pas de guerre contre
  la drogue sans les milliards de dollars détournés des personnes que
  vous opprimez.~» Dread Pirate Roberts, \emph{chat}, 20 mars 2012~:
  \url{https://antilop.cc/sr/users/dpr/threads/20120320-1103-chat.html}.

  Sera puni d'une amende de 3~750~€ et d'un emprisonnement de six mois
  quiconque aura incité le public à refuser ou à retarder le paiement de
  l'impôt.~»

  Lors de la rédaction de cette loi en août 1936, le franc avait perdu
  80~\% de sa valeur en or à la suite de la Grande Guerre et allait en
  perdre encore 30~\% suite à la dévaluation du 25 septembre suivant.

  «~Beaucoup de gens l'ignorent, mais la mission de PayPal était de
  créer une monnaie mondiale qui était indépendante de l'ingérence des
  cartels bancaires corrompus et des États qui dévaluaient leurs
  monnaies. Nous avons réussi à construire quelque chose de très
  puissant économiquement, qui a rendu possible de nombreuses petites
  entreprises, nous en sommes super fiers, mais nous n'avons jamais
  accompli cette mission. Je ne pense pas que {[}le problème de la
  monnaie numérique{]} soit résolu par PayPal, précisément en raison du
  fait que {[}...{]} PayPal est simplement trop centralisé et trop
  attaché aux grandes institutions financières comme Visa, MasterCard,
  le réseau ACH, le réseau SWIFT.~» Reserve, \emph{Luke Nosek speaks to
  Nevin Freeman about Reserve and the original vision of PayPal - Davos
  2019} (vidéo), 22 mai 2019~:
  \url{https://www.youtube.com/watch?v=hOeOzhOxeMU&t=40s}.

  «~Le chercheur en sécurité Nick Szabo a inventé le terme bit gold pour
  désigner un concept similaire de jetons qui représentent
  intrinsèquement un certain niveau d'effort. Le concept de Nick est
  plus complexe que le système simple des RPOW, mais son idée
  s'applique~: à certains égards, un jeton de RPOW peut être considéré
  comme ayant les propriétés d'une substance rare comme l'or. Miner et
  frapper des pièces d'or demande un effort et une dépense, ce qui les
  rend intrinsèquement rares. Les pièces d'or peuvent alors être
  transmises d'une personne à une autre, et chaque bénéficiaire peut
  vérifier l'authenticité de la frappe monétaire.

  De la même manière, la création de jetons de RPOW demande un certain
  degré d'effort et de dépense. Ils débutent tous avec une collision
  hashcash qui, au plus haut degré, prendra des heures voire des jours
  de calcul pour être créée. Les jetons de RPOW peuvent être validés et
  vérifiés à la réception en étant échangés contre un nouveau jeton de
  RPOW sur un serveur RPOW. Cela leur permet d'être transmis d'une
  personne à une autre tout comme des pièces.

  Plus important encore, le système RPOW est conçu dans un but
  primordial~: empêcher quiconque, y compris le propriétaire du serveur
  RPOW et le développeur du logiciel RPOW, de violer les règles du
  système et de falsifier des jetons de RPOW. Sans cette garantie contre
  la falsification, les jetons de RPOW ne représenteraient pas de
  manière crédible le travail effectué pour les créer. Des jetons
  falsifiables ressembleraient davantage à du papier-monnaie qu'à du bit
  gold. Mon objectif avec ce projet était de donner vie à une
  concrétisation simple qui démontre la puissance du concept de bit
  gold. Pour ce faire, une résistance à la falsification est nécessaire,
  et c'est cet objectif qui a dominé tous les aspects de la
  conception.~» Hal Finney, \emph{RPOW Theory}, 15 août 2004~:
  \url{http://rpow.net/theory.html}~; archive~:
  \url{https://web.archive.org/web/20040815154951/http://rpow.net/theory.html}.

  Nick Szabo~: «~Comme je l'ai déclaré à plusieurs reprises auparavant,
  toute cette spéculation est flatteuse, mais incorrecte -- je ne suis
  pas Satoshi.~» -- Nathaniel Popper, \emph{Decoding the Enigma of
  Satoshi Nakamoto and the Birth of Bitcoin}, 15 mai 2015.

  Hal Finney~: «~Aujourd'hui, la véritable identité de Satoshi est
  devenue un mystère. Mais à l'époque, je pensais avoir affaire à un
  jeune homme d'origine japonaise, très intelligent et sincère.~» -- Hal
  Finney, \emph{Bitcoin and me}, /03/2013 20:40:02 UTC~:
  \url{https://bitcointalk.org/index.php?topic=155054.msg1643833\#msg1643833}.

  G =
  \&\textsubscript{(}\mathtt{0x79be667ef9dcbbac55a06295ce870b07029bfcdb2dce28d959f2815b16f81798},
  \textbackslash\&\textsubscript{\mathtt{0x483ada7726a3c4655da4fbfc0e1108a8fd17b448a68554199c47d08ffb10d4b8}})\textasciitilde.
  \textbackslash end\{aligned\}\[ Il a pour ordre le nombre premier \]n
  =
  \mathtt{0xfffffffffffffffffffffffffffffffebaaedce6af48a03bbfd25e8cd0364141}\textasciitilde,\$\$
  de sorte que \(n~G = 0\).

  \begin{itemize}
  \item
    Choisir aléatoirement une clé éphémère \(l\) inférieure à \(n-1\)~;
  \item
    Calculer les coordonnées \((i,j)\) du point \(l~G\)~;
  \item
    Calculer \(r = i \mod n\)~; si \(r = 0\), choisir un autre \(l\)~;
  \item
    Calculer \(s = l^{-1} ( H(m) + k r ) \mod n\)~; si \(s = 0\),
    choisir un autre \(l\)~;
  \item
    La signature est \(( r, s )\).
  \end{itemize}

  L'algorithme de vérification est le suivant~:

  \begin{itemize}
  \item
    Vérifier que \(K \ne 0\) et que \(K\) appartient à la courbe~;
  \item
    Vérifier que \(n~K = 0\)~;
  \item
    Vérifier que \(1 \leq r \leq n - 1\) et \(1 \leq s \leq n - 1\)~;
  \item
    Calculer
    \((i, j) = ( H(m) s^{-1} \mod n )~G + ( r s^{-1} \mod n  )~K\)~;
  \item
    Vérifier que \(r = i \mod n\).
  \end{itemize}

  Cette préoccupation a conduit Hal Finney à écrire son troisième et
  dernier tweet sur Bitcoin où il affirmait réfléchir «~à la manière de
  réduire les émissions de CO\textsubscript{2} que produiraient une mise
  en œuvre généralisée de Bitcoin~». -- Hal Finney sur Twitter, /01/2009
  20:14 UTC~: \url{https://twitter.com/halfin/status/1153096538}.

  Les alertes décrites par Satoshi sont aujourd'hui appelées preuves de
  fraude mais sont toujours en phase de développement.

  «~Un jour, lorsque nous aurons des implémentations fonctionnant
  uniquement en mode client, la taille de la chaîne de blocs n'aura plus
  beaucoup d'importance. D'ici là, tant que tous les utilisateurs ont
  toujours à télécharger la chaîne de blocs entière pour commencer, il
  est bon de pouvoir la maintenir à une taille raisonnable.~» Satoshi
  Nakamoto, \emph{Re: More BitCoin questions}, /12/2010 21:42 UTC~:
  \url{https://plan99.net/~mike/satoshi-emails/thread3.html}.

  Une double dépense conséquente a été réalisée~: macbook-air, \emph{A
  successful DOUBLE SPEND US\$10000 against OKPAY this morning.},
  /03/2013, 18:22:02 UTC~:
  \url{https://bitcointalk.org/index.php?topic=152348.msg1616747\#msg1616747}.

  \[S = r~V = r~v~G = v~r~G = v~R\]

  et la clé publique de réception est~:

  \[P = K + H(S)~G~.\]

  Le destinataire peut aussi n'utiliser qu'une seule paire de clés
  \((k, K)\). Dans ce cas, \(M = K\).

  Elle a également été favorisée de manière plus conservatrice par
  Monero, qui possède une taille de bloc dynamique basée sur un
  mécanisme de pénalité pour compenser les \emph{excès} par rapport à la
  normale. Voir SerHack, \emph{Mastering Monero: The Future of Private
  Transactions}, Amazon KDP, 2018, pp.~136--139.
\end{enumerate}


\backmatter


\end{document}
