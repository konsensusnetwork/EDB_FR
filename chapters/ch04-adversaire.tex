% Copyright (c) 2022 Ludovic Lars
% This work is licensed under the CC BY-NC-SA 4.0 International License

\chapter{La nécessité de décentralisation}
\label{ch:adversaire}

Bitcoin donne aux individus la propriété entière sur leur argent, en leur permettant de le conserver de manière souveraine et de l'envoyer à n'importe quelle personne, n'importe où dans le monde, à n'importe quel moment, quel que soit le motif. Il leur permet aussi de préserver leur pouvoir d'achat en évitant la création de monnaie arbitraire. Par cette double proposition de valeur, il s'oppose au contrôle sur la monnaie par l'État.

L'État est l'incarnation du transfert de richesse non consenti. Il se rémunère par l'impôt prélevé directement sur le contribuable, et par le seigneuriage prélevé indirectement sur l'épargnant. En offrant une liberté monétaire, Bitcoin remet ce contrôle en question et constitue en cela une menace pour le prélèvement. Il s'inscrit donc dans un rapport antagoniste avec l'État.

C'est pourquoi l'aspect décentralisé de Bitcoin est une nécessité, bien qu'il constitue une perte d'efficacité. L'État ne tolère pas de concurrence en matière de monnaie. Plutôt qu'attaquer frontalement, il faut contourner le problème.

Dans ce chapitre, nous étudierons le transfert de richesse massif réalisé par l'État, nous verrons comment le contrôle sur la monnaie incite naturellement à la centralisation du pouvoir et nous expliquerons pourquoi les systèmes alternatifs centralisés n'ont pas pu tenir.

\section*{La production, la spoliation et l'État}
\addcontentsline{toc}{section}{La production, la spoliation et l'État}

% Deux moyens d'obtenir de la richesse
Pour l'être humain, il n'existe que deux moyens de se procurer les choses auxquelles il accorde de la valeur~: la production et la spoliation, ou pour le dire autrement, le travail et le vol. Le premier moyen implique un effort personnel~; le second à s'approprier l'effort d'autrui. Le premier moyen est volontaire~; le second involontaire. Le premier moyen est créateur~; le second destructeur. Ils sont ainsi diamétralement opposés, et l'ont toujours été depuis que l'homme est homme.

% Moyen économique
D'une part, une personne peut acquérir de la richesse par son travail. Elle peut fabriquer le bien désiré, par l'application de ses facultés personnelles à une ressource (terre), généralement grâce à l'utilisation d'un outil (capital). Elle peut aussi l'obtenir de manière indirecte, par l'échange, auquel cas elle doit fournir à son interlocuteur la chose qu'il désire en retour. L'organisation de cette production prend la forme de l'économie de marché basée sur la division du travail et le commerce. C'est pourquoi cette première méthode d'acquisition de richesse s'appelle le moyen économique.

% Moyen politique
D'autre part, une personne peut s'emparer de la richesse d'autrui. Elle peut le faire de manière directe par la menace de violence, par l'expropriation simple ou par l'escroquerie. Elle peut aussi le faire de manière indirecte par l'application coercitive d'un privilège afin de bénéficier d'un avantage concurrentiel sur le marché. L'organisation de cette spoliation prend la forme de l'exploitation économique d'un groupe de personnes par autre groupe. La richesse extraite peut être un paiement en nature ou en monnaie (tribut) ou un travail forcé (esclavage). Une telle pratique se retrouve dans le crime organisé privé, mais aussi et surtout à la base de ce que nous appelons communément l'État. C'est pourquoi cette seconde méthode d'acquisition de richesse est nommée le moyen politique\sendnote{La distinction entre le moyen économique et le moyen politique a été brillamment exposée par le sociologue allemand Franz Oppenheimer dans son ouvrage \emph{L'État, ses origines, son évolution et son avenir} publié en 1907. Cependant, cette distinction n'était pas nouvelle, voir par exemple~: Frédéric Bastiat, \emph{Physiologie de la Spoliation}, 1848~: \url{http://bastiat.org/fr/physiologie_de_la_spoliation.html}.}.

% Définition de l'État
Du point de vue sociologique, l'État se définit classiquement comme une autorité souveraine qui s'exerce sur un territoire déterminé et sur un peuple qu'elle représente officiellement. Il en ressort que trois éléments le caractérisent~: l'autorité, la territorialité et l'acceptation sociale.

% Autorité
Premièrement, l'État repose sur l'agression pour fonctionner. Son \emph{modus operandi} est la contrainte, imposée par la violence ou la menace de violence, sans laquelle il ne pourrait aucunement faire ce qu'il fait. Il restreint ainsi la liberté naturelle de ses sujets et, en particulier, il lève un impôt qui n'est pas soumis au consentement individuel.

% L'État existant est né dans l'agression et la conquête. Le contrat social est un mythe. Le sociologue allemand Franz Oppenheimer, qui a étudié la formation historique de l'État, écrivait~:
%
% \begin{quote}
% «~L'État est, entièrement quant à son origine, et presque entièrement quant à sa nature pendant les premiers stages de son existence, une organisation sociale imposée par un groupe vainqueur à un groupe vaincu, organisation dont l'unique but est de réglementer la domination du premier sur le second en défendant son autorité contre les révoltes intérieures et les attaques extérieures. Et cette domination n'a jamais eu d'autre but que l'exploitation économique du vaincu par le vainqueur.\sendnote{Franz Oppenheimer, \emph{L'État, ses origines, son évolution et son avenir}, 1907.}~»
% \end{quote}

% Territorialité
Deuxièmement, l'autorité de l'État s'exerce sur un territoire donné. Cette caractéristique lui permet de consolider son prélèvement au sein de frontières définies. Les êtres humains étant nécessairement liés à la terre (et parfois à la mer) pour exercer leurs facultés, le contrôle du territoire facilite énormément leur soumission. La domination sur la terre explique l'organisation féodale (du latin médiéval \emph{feodum}, «~fief~») de l'État dans les sociétés agraires. Celle sur la mer explique la formation de thalassocraties (du grec ancien \foreignlanguage{greek}{tálassa}, thálassa, «~mer~», et de \foreignlanguage{greek}{krátos}, krátos, «~pouvoir~») qui profitent de l'activité commerciale maritime.

% Acceptation générale de la population
Troisièmement, l'État a la particularité de revendiquer représenter les intérêts de ses sujets qui vivent sur son territoire. Il se différencie de l'activité des criminels communs et des autres groupes organisés\sendnote{Certains groupements du crime organisé privé comme les mafias italiennes, les cartels centro-américains et sud-américains, les triades chinoises, les yakuzas japonais, etc. peuvent parfois former \emph{de facto} une version émergente d'État, et se substituer à l'État officiel sur un territoire donné.} par le fait que son action bénéficie d'une \emph{large acceptation} par la population, acceptation qui peut aller de l'approbation active à la résignation passive. De là vient l'idée de «~contrat social~», qui n'a rien d'un réel contrat juridique, mais qui constitue une constatation de la situation. L'État tire ainsi son nom du fait qu'il incarne l'état actuel du rapport de force au sein de la société.

% Pérénnité du prélèvement
L'acceptation de l'État, qui n'est jamais totale, permet la pérennité du prélèvement fiscal, en faisant en sorte que celui-ci n'ait pas pas besoin d'être maintenu par la force pure. Elle minimise le coût du prélèvement, par la propagande basée sur la tromperie\sendnote{«~L'État, c'est le plus froid de tous les monstres froids~: Il ment froidement et voici le mensonge qui rampe de sa bouche~: "Moi, l'État, je suis le Peuple." [...] L'État ment dans toutes ses langues du bien et du mal~; et, dans tout ce qu'il dit, il ment -- et tout ce qu'il a, il l'a volé.~» -- Nietzsche, \eng{De la nouvelle idole}.}, de façon à limiter le nombre d'individus récalcitrants à réprimer. Dans l'histoire, cette acceptation a été liée à une religion transcendante (monarchie de droit divin) ou à une idéologie séculière (démocratie représentative, socialisme scientifique), déjà largement adoptée par la masse. % Staat heißt das kälteste aller kalten Ungeheuer. Kalt lügt es auch; und diese Lüge kriecht aus seinem Munde: »Ich, der Staat, bin das Volk.« (...) Aber der Staat lügt in allen Zungen der Guten und Bösen; und was er auch redet, er lügt – und was er auch hat, gestohlen hat er's.

% Maintien de l'ordre et défense contre les menaces
L'État revendique un monopole sur la violence défensive\sendnote{«~L'État est l'institution qui possède, dans une collectivité donnée, le monopole de la violence légitime.~» -- Max Weber, \emph{Le savant et le politique}, 1919.}, au travers du maintien de l'ordre intérieur (par l'intermédiaire de la police) et et de la défense contre les ennemis extérieurs (par le biais de l'armée). Cet élément est souvent avancé comme l'origine et la justification de l'État, mais il ne s'agit que d'une manière de rendre viable le prélèvement fiscal en protégeant les forces productives taxées des perturbations internes et externes. De plus, ce monopole constitue en lui-même une agression du fait de la contrainte de paiement et de l'entrave de la concurrence~: il s'agit d'un chantage à la protection tel que pratiqué par les mafias italiennes.

% Agression institutionnalisée
L'État, au sens où il est entendu habituellement, est donc l'incarnation de l'agression institutionnalisée. Il a pour but l'exploitation de l'homme par l'homme, le transfert de richesse non consenti, ou comme on l'a dit, l'organisation du moyen politique.

% Limites du prélèvement
Toutefois, sa capacité de prélèvement n'est pas infinie. Tout d'abord, il existe une pondération subtile entre le niveau effectif du prélèvement de richesse et la destruction économique induite par ce prélèvement, qui varie selon la préférence temporelle des bénéficiaires\sendnote{Voir Hans-Hermann Hoppe, \emph{Démocratie, le dieu qui a échoué}, 2001~; et plus précisément le chapitre 1~: «~La préférence temporelle, l'État et le processus de décivilisation~».}. Ensuite, le niveau de prélèvement dépend du niveau d'acceptation de la population et il existe nécessairement un point au-delà l'accroissement du taux de prélèvement se traduit par un amoindrissement du prélèvement total, phénomène illustré par la fameuse courbe de Laffer\sendnote{Arthur B. Laffer, \eng{The Laffer Curve: Past, Present, and Future}, 1\ier{} juin 2004~: \url{https://www.heritage.org/taxes/report/the-laffer-curve-past-present-and-future}.}. Enfin, la capacité de prélèvement dépend des outils à la disposition de la population pour résister fiscalement, ce qui inclut les armes et, bien entendu, Bitcoin.

%  État comme concept
Nous comprenons ici l'État comme un concept, qui ne désigne pas spécifiquement un groupe d'individus, mais un groupe d'actions réalisées par des individus dans un contexte spécifique. Une personne peut œuvrer en faveur du prélèvement de richesse et agir ainsi en concordance avec l'intérêt de l'État~; mais elle peut également agir à l'encontre de cet intérêt en entravant le prélèvement, par attachement à la liberté individuelle par exemple. Les êtres humains ont des intérêts politiques, mais ils ont également des intérêts économiques, et l'existence de l'État dépend de leurs actions. % La raison d'État est le principe politique en vertu duquel l'intérêt de l'État, conçu comme une préoccupation supérieure émanant de l'intérêt général, peut nécessiter de déroger à certaines règles juridiques ou morales, notamment dans des circonstances exceptionnelles.

% État comme instance d'un concept général
Le concept d'État se manifeste dans des instances plus ou moins indépendantes, appelées États ou états. Chaque instance de ce type agit sur une population spécifique dans des conditions particulières, de sorte qu'il existe une multitude d'États aujourd'hui. Toutefois, ces instances s'influencent mutuellement, par la guerre notamment\sendnote{«~La guerre n'est que la simple continuation de la politique par d'autres moyens.~» -- Carl von Clausewitz, \emph{De la guerre}, 1832.}, ce qui crée une tendance générale à la centralisation du pouvoir, qui pourrait un jour mener à la formation d'un État mondial.

% --- Évolution de l'État : État féodal, État-marchand, État religieux, État-providence ---

% État féodal (seigneurs)
Un État suit toujours la même évolution. À l'origine, l'impôt est échu à une caste dominante de guerriers. Ce sont les seigneurs féodaux qui perçoivent les revenus de leur territoire, invoquant la défense de ce territoire comme leur devoir (c'est le rôle de la noblesse). Ce modèle est décentralisé, à cause de la nature du pouvoir essentiellement basé sur la violence qui est toujours visible.

% État religieux (clergé)
Ensuite, à mesure que le groupe dominant s'intègre à la population dominée, il faut pouvoir justifier la domination. Une partie du revenu étatique revient donc aux membres de la classe qui se charge d'asseoir la légitimité du pouvoir. C'était le cas de la dîme au Moyen Âge et durant l'Ancien Régime, une taxe agricole qui était destinée à l'entretien du clergé. C'est le cas aujourd'hui des nombreux impôts qui subventionnent les écoles, les universités, les médias et la culture en général.

% État-marchand (industries)
Puis, à mesure que la division du travail se développe, les divers acteurs économiques demandent leur part du gâteau, en bénéficiant de privilèges (payés indirectement par les consommateurs) ou de subventions (directement payées par le contribuable). On peut citer les grands industriels au \textsc{xix}\ieme{}~siècle et les banquiers au \textsc{xx}\ieme{}~siècle.

% Cet effet de ce qui se voit et ce qui ne se voit pas a été mis en lumière par l'économiste français Frédéric Bastiat au milieu du \textsc{xix}\ieme{}~siècle, puis par l'économiste américain Henry Hazlitt au sortir de la Deuxième Guerre mondiale\sendnote{Frédéric Bastiat, \emph{Sophismes économiques}, 1845-1848~; Henry Hazlitt, \emph{L'économie en une leçon}, 1946.}.

% État-providence (peuple)
Enfin, l'État finit par devenir le lieu de la participation de tous, où chacun demande sa part. C'est le modèle de l'État-providence où l'État assure un certain nombre de services publics tels que la sécurité sociale, le régime de retraites, les transports, l'enseignement, l'énergie ou les communications. C'est ce qui a notamment fait dire à l'économiste français Frédéric Bastiat en 1848, alors député de la Deuxième République, que l'État était «~la grande fiction à travers laquelle tout le monde s'efforce de vivre aux dépens de tout le monde.\sendnote{Frédéric Bastiat, \emph{L'État}, Journal des Débats, 25 septembre 1848.}~». % fin observateur des dynamiques politiques de son époque,

% Lutte des classes
Ce recours à l'État par tout le monde fait que chacun essaie d'utiliser l'État pour optimiser sa situation, notamment en cherchant à devenir un bénéficiaire du dispositif et ne plus en être un contributeur. Les personnes dans la même situation ont tendance à demander la même chose, formant des groupes plus ou moins bien définis, de sorte que l'évolution de l'État a pour effet de créer une véritable lutte des classes\sendnote{«~L'histoire de toute société jusqu'à nos jours n'a été que l'histoire des luttes de classes. Homme libre et esclave, patricien et plébéien, baron et serf, maître de jurande et compagnon, en un mot oppresseurs et opprimés, en opposition constante, ont mené une guerre ininterrompue, tantôt ouverte, tantôt dissimulée, une guerre qui finissait toujours soit par une transformation révolutionnaire de la société tout entière, soit par la destruction des deux classes en lutte.~» -- Karl Marx, \eng{Manifeste du parti communiste}, février 1848.

«~L'intérêt de chaque classe met en mouvement une quantité absolue de forces coordonnées, lesquelles tendent avec une vitesse déterminée vers un but déterminé. Ce but est le même pour toutes les classes~: le produit total du travail consacré par tous les citoyens à la production de biens. Chaque classe aspire à une part aussi grande que possible du produit national, et comme toutes ont les mêmes désirs, la lutte de classe est l'essence même de toute histoire de l'État.~» -- Franz Oppenheimer, \emph{L'État, ses origines, son évolution et son avenir}, 1907.}, qui se manifeste notamment par l'existence du clivage gauche-droite entre progressistes et conservateurs\sendnote{Philippe Fabry, Léo Portal, \emph{Islamogauchisme, populisme et nouveau clivage gauche-droite}, 2021.}.

\section*{Le prélèvement direct~: l'impôt}
\addcontentsline{toc}{section}{Le prélèvement direct~: l'impôt}

% Financement de l'État
Comme on l'a évoqué, l'État utilise deux moyens principaux de prélèvement~: l'impôt, qui se caractérise par la ponction directe du contribuable, et le seigneuriage, qui se caractérise par la spoliation indirecte de l'épargnant via l'émission privilégiée de monnaie. L'endettement, souvent cité comme un troisième moyen, n'est en réalité qu'un impôt différé ou un seigneuriage déguisé.

% L'impôt est un vol
L'impôt est l'acte de ponctionner directement de la richesse -- en monnaie ou en nature -- généralement par le biais de la menace de violence. Le terme, qui vient du latin \emph{impōnere}, porte en lui la notion d'imposer, de forcer, de prescrire, de sorte qu'il n'a rien de volontaire. L'impôt constitue une appropriation non consentie de la richesse d'autrui, c'est-à-dire un vol par définition\sendnote{«~L'impôt est un vol, purement et simplement, même si ce vol est commis à un niveau colossal auquel les criminels ordinaires n'oseraient prétendre.~» -- Murray Rothbard, \emph{L'Éthique de la liberté}, 1982.}. Même si l'on considère que ce transfert de richesse se justifie par l'«~intérêt général~» ou qu'il constitue un «~mal nécessaire~», la nature première de l'impôt reste la même.

% Évolution de l'impôt
L'évolution de l'impôt suit celle de l'État qu'il sert à financer. Sous sa forme primitive, l'impôt prend souvent la forme d'un tribut imposé par un pouvoir central à un peuple vaincu comme signe de dépendance. Mais, avec l'élargissement des bénéficiaires, il devient de plus en plus complexe et de moins en moins arbitraire. Aujourd'hui les impôts servent à payer toutes sortes de choses conformément à l'idéal de redistribution de l'État-providence. % taille seigneuriale ou la taille royale au Moyen Âge et durant la Rennaissance

% Impôts direct et indirect
L'impôt peut être «~direct~» et s'appliquer nominalement à la personne (physique ou morale) qui le paie. Mais il peut aussi être «~indirect~» et être versé par une autre personne que celle qui paie réellement l'impôt. Ce dernier fonctionnement est très prisé par l'État car il permet de diminuer la quantité de personnes à contrôler (les intermédiaires) et de rendre l'impôt «~indolore~» pour les populations concernées en obscurcissant le prélèvement. La taxe sur la valeur ajoutée (TVA), créée en France en 1954, est l'incarnation de cette taxation à demi cachée~: il s'agit du prélèvement d'un taux fixe du prix de vente des biens et des services (généralement 20~\%), payée par l'acheteur et versée par le commerçant. % «~contributions~» directes de la Révolution, dites les quatre vieilles). Mais il peut être aussi indirect

% Impôt en France, enfer fiscal
En France, les administrations parlent de prélèvements obligatoires, et les appellent impôts, taxes ou cotisations selon leur affectation. Ces prélèvements sont redirigés principalement vers l'État (au sens administratif du terme), la Sécurité sociale et les collectivités territoriales\sendnote{En France, les principaux prélèvements obligatoires sont, par ordre d'importance~: la taxe sur la valeur ajoutée (TVA), créée en 1954, qui est un impôt indirect et contribue au budget de l'État~; la contribution sociale généralisée (CSG), créée en 1990, qui est retenue à la source par l'employeur, pour la Sécurité sociale~; l'impôt sur le revenu (IR), créé en 1914 et entré en vigueur en 1916~; l'impôt sur les sociétés (IS), créé en 1948~; la taxe foncière, qui est issue des quatre «~contributions directes~» de la Révolution~; la taxe intérieure de consommation sur les produits énergétiques (TICPE), qui est un impôt indirect créé en 1928 sous la forme d'une taxe intérieure pétrolière.}. En 2022, le poids des prélèvements obligatoires s'élevait à 45,3~\% du produit intérieur brut du pays\sendnote{\url{https://www.insee.fr/fr/statistiques/2381412}}. C'est l'un des taux les plus élevés du monde, ce qui fait du pays ce qu'on peut appeler un enfer fiscal.

% --- Contrôles ---

L'impôt est aujourd'hui levé grâce à une multitude de contrôles réalisés par les États. En France, la charge est attribuée à la direction générale des Finances publiques (DGFiP), à la Direction générale des Douanes et Droits indirects (DGDDI) et aux administrations de sécurité sociale (ASSO). Aux États-Unis, le prélèvement de l'impôt sur le revenu (représentant la moitié du revenu fiscal fédéral) est géré par l'\eng{Internal Revenue Service} (IRS).

% Surveillance
Ces contrôles passent d'abord par le surveillance financière, qui s'applique notamment dans le domaine bancaire. Les banques et les autres organismes financiers sont responsables devant l'administration fiscale, à qui ils doivent transmettre les informations douteuses. En France, cette surveillance est assurée par l'Autorité de contrôle prudentiel et de résolution (ACPR), qui inspecte l'activité des banques et des assurances, par l'Autorité des marchés financiers (AMF), qui observe les marchés boursiers et entreprises d'investissement, et par Tracfin, le service de renseignement français chargé de la lutte contre le blanchiment d'argent, le financement du terrorisme et l'évasion fiscale.

% Contrôle fiscal intérieur
La collecte de l'impôt à l'intérieur du territoire repose sur le contrôle fiscal, c'est-à-dire l'ensemble des méthodes d'intervention permettant d'examiner les déclarations, de les confronter à la réalité des faits et de réhausser, le cas échéant, les bases d'imposition. Elle est aussi soutenue par un certain nombre de lois qui permettent de faciliter la surveillance, comme par exemple les restrictions sur l'argent liquide. % Lutte contre la fraude

% Contrôle fiscal frontalier
La préservation du revenu fiscal de l'État repose également sur l'entrave des flux de richesse sortant du territoire. Celle-ci prend la forme des contrôles douaniers et des contrôles de capitaux. Le recours à des mesures plus restrictives a émergé avec la mondialisation généralisée de l'économie depuis le \textsc{xix}\ieme{}~siècle. % Contrôle des changes est une forme de contrôle des capitaux

% Contrôle des capitaux~: Le contrôle des capitaux désigne l'ensemble des mesures légales ou réglementaires qui permettent à un État de réguler ou contrôler les entrées et les sorties de capitaux d'un périmètre donné, généralement d'un pays entier. Ces mesures sont généralement mises en place afin de limiter l'instabilité financière et le risque de crise économique.

% Contrôle des changes~: Le contrôle des changes est un instrument conçu pour lutter contre la fuite des capitaux et la spéculation, consistant plus particulièrement en des mesures prises par un État pour réglementer l'achat et la vente de monnaies concurrentes, et plus spécifiquement des devises étrangères, par ses ressortissants. Le contrôle des changes peut se faire par rapport à

Ces contrôles sont étroitement liés à la question de la censure financière qui est traitée dans le chapitre~\ref{ch:censure} du présent ouvrage.

% --- Acceptation de l'impôt ---

Enfin, l'impôt n'est pas une simple contrainte parmi d'autres~: il s'agit de la pierre angulaire de la construction étatique. C'est pourquoi la résistance fiscale a été sévèrement réprimée au cours de l'histoire. C'est pourquoi son acceptation est constamment mise en avant, notamment par le concept (très théorique) de «~consentement à l'impôt~». C'est pourquoi son évitement est systématiquement diabolisé, y compris lorsqu'il est légal.

% Liberté d'expression et impôt
Cette fabrique de l'acceptation passe en particulier par la limitation de l'expression. En France, s'il est aujourd'hui autorisé de dire toutes sortes de choses, il est en revanche interdit d'appeler à arrêter de payer l'impôt, d'après l'article 1747 du \emph{Code général des impôts}\sendnote{Article 1747 du \emph{Code général des impôts}~: \begin{quote}
\footnotesize «~Quiconque, par voies de fait, menaces ou manoeuvres concertées, aura organisé ou tenté d'organiser le refus collectif de l'impôt, sera puni des peines prévues à l'article 1er de la loi du 18 août 1936 réprimant les atteintes au crédit de la nation [c'est-à-dire de deux ans de prison et d'une amende de 9000 euros].

Sera puni d'une amende de 3 750~\euro{} et d'un emprisonnement de six mois quiconque aura incité le public à refuser ou à retarder le paiement de l'impôt.~»
\end{quote}}. Cette contrainte montre le caractère sacré et essentiel de l'impôt. % https://www.legifrance.gouv.fr/codes/id/LEGIARTI000006313764/2011-05-19

\section*{Le prélèvement indirect~: le seigneuriage}
\addcontentsline{toc}{section}{Le prélèvement indirect~: le seigneuriage}

% Introduction
Si l'impôt constitue la manière la plus simple de prélever autrui, il n'en est pas pour autant la seule. Il existe un autre moyen principal, qui consister à tirer profit d'une industrie particulière~: la production de monnaie. Ce moyen est appelé le seigneuriage.

% Définition du seigneuriage
Le seigneuriage est l'avantage financier direct qui découle de l'émission de monnaie pour l'émetteur. Le nom est issu de l'ancien français \emph{seignorage}, qui désignait le privilège de battre monnaie au Moyen Âge, un privilège généralement réservé aux seigneurs féodaux.

% --- Mesures mises en place ---

Le seigneuriage est le résultat de quatre mesures légales principales dont peut tirer profit le producteur~: la contrefaçon légalisée, le monopole sur la production, l'imposition du cours légal et la suspension des paiements\sendnote{Jörg Guido Hülsmann, \emph{The Ethics of Money Production}, chapitres 8 -- 11, 2008.}. Conformément à notre idée de l'État, ces actions sont largement acceptées (voire approuvées) dans le cas où elles émanent de la puissance publique, tandis qu'elles sont passibles de lourdes peines lorsqu'elles sont réalisées par des criminels communs.

% Contrefaçon légalisée
La contrefaçon légalisée consiste à faire circuler de la monnaie dont le certificat ne correspond pas à ce qui attendu par la population générale. Typiquement, il s'agit de faire circuler des pièces possédant une teneur moindre en métal que les pièces similaires existantes ou bien de billets représentatifs dont la monnaie de base de garantie est en réalité conservée de manière fractionnaire.

% Monopole sur la production
Le monopole sur la production de monnaie est le privilège exclusif d'émission monétaire accordé à une entité, la dédouanant de toute concurrence et lui permettant de vendre sa monnaie à un prix supérieur qu'il ne l'aurait été sur le marché libre. Ce privilège est habituellement délégué à une entité contrôlée par l'État comme un hôtel de la Monnaie ou une banque centrale. % Il peut également être accordé à un ensemble d'acteurs, auquel cas on parle d'oligopole.

% Imposition du cours légal
Le cours légal est l'obligation imposée aux acteurs économiques d'accepter une monnaie à la valeur nominale dictée par l'État. L'imposition du cours légal peut être restreinte et ne concerner que les paiements différés (c'est le sens de la \eng{legal tender} anglo-saxonne) ou bien être plus large et se rapporter à tous les paiements (comme c'est souvent le cas en Europe continentale\sendnote{Par exemple, le cours légal de l'argent liquide est imposé en France par l'article R642-3 du code pénal~: «~Le fait de refuser de recevoir des pièces de monnaie ou des billets de banque ayant cours légal en France selon la valeur pour laquelle ils ont cours est puni de l'amende prévue pour les contraventions de la 2e classe.~»}). Il s'agit d'avantager la monnaie dont l'État dispose d'un monopole d'émission en la surestimant par rapport aux monnaies concurrentes.

Ce cours légal a pris plusieurs formes au cours de l'histoire. Il se retrouvait dans le bimétallisme (double étalon) où le ratio entre l'or et l'argent était fixé à une valeur arbitraire, avantageant l'un ou l'autre des deux métaux. Il se manifestait pendant la période de l'étalon-or classique lorsque les certificats représentatifs devaient être échangés au même cours que le numéraire. Il était également institué par l'étalon de change-or qui imposait que les monnaies nationales des pays secondaires ait cours à un taux déterminé par rapport à la livre ou le dollar, le taux du marché étant maintenu artificiellement haut par le contrôle des changes. Aujourd'hui, le cours légal se définit par rapport au cours sur le marché des changes et il se manifeste par l'interdiction de proposer systématiquement un prix différent selon l'intermédiaire d'échange utilisé. % Un contrôle particulier est le contrôle des changes consistant à réglementer l'achat et la vente de monnaies concurrentes, et plus spécifiquement des devises étrangères, par ses ressortissants.

% Suspension des paiements pour la monnaie représentative
La suspension des paiements consiste, pour une banque centrale, à interrompre momentanément le remboursement de ses clients, auquel cas on parle de cours forcé. Dans le cas des billets représentatifs, la possibilité de recourir à cette mesure légale permettait de ne pas conserver l'intégralité de l'or en réserve, en empêchant les retraits dans le cas d'une chute de confiance.

% Cette suspension peut être réalisée par les banques à réserves intégrales, qui s'engagent à conserver l'intégralité des fonds de leurs clients, ou bien les banques à réserves fractionnaires, qui émettent essentiellement du crédit (la frontière étant parfois floue entre les deux). La monnaie alors émise par les banques (sous forme de billets et de dépôts) devient alors essentiellement une monnaie fiat ayant cours légal aux côtés de la monnaie traditionnelle.

% --- Seigneuriage avec les métaux précieux ----

Comme on l'a dit, depuis l'Antiquité jusqu'au \textsc{xix}\ieme{}~siècle, la monnaie était constituée essentiellement de pièces de métaux précieux, essentiellement de l'or, de l'argent et parfois du cuivre. Il était donc impossible pour le souverain de créer de nouvelles unités à partir de rien. Cependant, il pouvait dévaluer les pièces existantes en réduisant leur teneur en métal.

À l'époque romaine, le \emph{denarius} d'argent (qui a donné son nom au denier) a été dévalué à de nombreuses reprises, lentement d'abord, avant de voir sa teneur en métal être réduite à l'excès au cours du \textsc{iii}\ieme{}~siècle. Le seigneuriage retiré a permis à l'Empire romain de continuer à financer sa domination, sans pour autant continuer son expansion territoriale. De même, tous les souverains européens ont procédé à ce type de manipulation au cours du Moyen Âge. Ces pratiques ont notamment été observées par le philosophe chrétien Nicolas Oresme au \textsc{xiv}\ieme{}~siècle\sendnote{Benoît Malbranque, \emph{Oresme et les dangers de la dévaluation monétaire}, 14 juillet 2017~: \url{https://www.institutcoppet.org/oresme-et-les-dangers-de-la-devaluation-monetaire/}.}. % Nicolas Oresme, Tractatus de origine, natura, iure et mutationibus monetarum ; Traité sur l'origine, la nature, l'altération des monnaies ; 1355

% Loi de Gresham
Cette manipulation des pièces de monnaie a un effet autrement malencontreux~: celle de chasser de la circulation la monnaie sous-estimée, qui se retrouve thésaurisée ou exportée à l'étranger. C'est ce qu'on appelle la loi de Gresham, qui tire son nom de Sir Thomas Gresham, un grand marchand et financier anglais du \textsc{xvi}\ieme{}~siècle, qui avait établi le lien causal entre la disparition des meilleures pièces d'argent de la circulation et les mesures légales du pouvoir de l'époque\sendnote{La loi de Gresham a été formalisée par l'économiste écossais Henry Dunning Macleod dans ses \eng{Elements of Political Economy} publiés en 1858.}. Cette loi, couramment résumée par l'expression proverbiale «~la mauvaise monnaie chasse la bonne~», stipule qu'en l'existence d'un taux de change légal fixe entre deux monnaies, la mauvaise monnaie (c'est-à-dire celle qui est surestimée) a tendance à remplacer la bonne monnaie (c'est-à-dire celle qui est sous-estimée) en tant que moyen de paiement dans le commerce. Cette loi s'applique également, dans une moindre mesure, pour la monnaie représentative et pour la monnaie fiat.

% --- Seigneuriage avec les billets représentatifs ---

Le développement des banques à partir de la Renaissance a provoqué l'apparition des billets de banque convertibles à vue, bien plus pratique pour déplacer de la valeur dans l'espace. Le pouvoir a repris cette invention à son compte en monopolisant l'émission des billets et en faisant des billets supposés représentatifs. Dans ce cas, le seigneuriage consiste à créer plus de billets qu'il n'y a de métal précieux en réserve, c'est-à-dire réaliser une fraude financière.

Cependant, une contrainte subsiste~: une grande partie du métal doit être conservée, sous peine de voir les créanciers vider les coffres. L'État peut choisir de suspendre les paiements (ce qu'il a fait dans l'histoire), mais une telle mesure s'accompagne alors d'une baisse drastique de confiance dans les billets par rapport au métal qu'ils représentent. C'est pourquoi le régime de l'étalon-or, bien qu'il soit resté relativement stable au niveau monétaire, a pavé la voie à un régime autrement plus inflationniste~: celui du papier-monnaie.

% --- Seigneuriage avec le papier-monnaie ---

Le seigneuriage a acquis un rôle majeur avec l'apparition du papier-monnaie. Le papier-monnaie est une monnaie fiduciaire basée sur un support physique. Le seigneuriage consiste alors juste à créer plus de billets dont l'usage est imposé sur le territoire, ce qui est considérablement plus efficace que la dévaluation des pièces en métal précieux et la fraude sur les billets représentatifs.

% Coût de production des billets vs leur valeur nominale, quel profit~?

% Financement de la guerre
Dès l'origine, le papier-monnaie a permis de financer les projets pharaoniques des États, et en particulier la guerre. Il est ainsi indissociable de la guerre moderne comme l'illustrent les greenbacks américains émis durant la guerre de Sécession aux États-Unis (appelés comme ça à cause de l'encre verte utilisée pour imprimer le verso). Ainsi, la Première guerre mondiale a été majoritairement financée par la création monétaire et par la réduction de la dette liée à l'inflation\sendnote{Vincent Duchaussoy et Éric Monnet, \emph{La Banque de France et le financement direct et indirect du ministère des Finances pendant la Première Guerre mondiale~: un modèle français~?}, \url{https://books.openedition.org/igpde/4132}.}.

%la livre sterling durant les guerres napoléoniennes (qui avait cours forcé entre entre 1797 et 1821\sendnote{La Banque d'Angleterre a suspendu la convertibilité de ses billets entre 1797 et 1821 pour faire face à la fuite des capitaux résultant de la Guerre de la Première Coalition contre la France révolutionnaire et de l'éclatement de la bulle spéculative sur la terre aux États-Unis.}), les  ou encore le franc entre 1870 et 1875

Toutefois, la capacité de profiter du papier-monnaie n'est pas illimitée~: la production de pièces et de billets fiduciaires et la lutte contre la contrefaçon privée ont un coût incompressible faisant que le seigneuriage est réduit. C'est pourquoi il est aujourd'hui question de remplacer cet argent liquide par une monnaie numérique.

% --- Liberté d'expression et seigneuriage ---

Tout comme l'impôt, le seigneuriage repose sur l'acceptation de la population, qui est soutenue en particulier par une limitation de l'expression. Il est ainsi interdit de faire douter quelqu'un de la solidité de la monnaie, conformément à la loi du 18 août 1836 réprimant les atteintes au crédit de la nation\sendnote{L'article 1\ier{} de la loi du 18 août 1836 réprimant les atteintes au crédit de la nation stipule~:

\begin{quote}
«~Sera puni de deux ans de prison et d'une amende de 9 000 euros quiconque, par des voies ou des moyens quelconques, aura sciemment répandu dans le public des faits faux ou des allégations mensongères de nature à ébranler directement ou indirectement sa confiance dans la solidité de la monnaie, la valeur des fonds d'État de toute nature, des fonds des départements et des communes, des établissements publics et, d'une manière générale, de tous les organismes où les collectivités précédentes ont une participation directe ou indirecte.~» (\url{https://www.legifrance.gouv.fr/loda/article_lc/LEGIARTI000006529798})
\end{quote}

Lors de la rédaction de cette loi en août 1936, le franc avait perdu 80~\% de sa valeur en or avec la Grande Guerre et allait en perdre encore 30~\% suite à la dévaluation du 25 septembre suivant.}. % abrogeait et remplaçait la loi du 12 févier 1924, est toujours en vigueur aujourd'hui.

% --- Seigneuriage avec le crédit ---

Le crédit est aussi, dans une certaine mesure, un moyen de réaliser un seigneuriage. Les banques sont formées en un cartel étroitement lié à l'État, et bénéficient d'un privilège légal à émettre du crédit. Elles sont protégées par la banque centrale qui constitue un prêteur en dernier ressort et par le Trésor qui peut procéder à un renflouement externe. Leurs clients sont encouragés à garder leurs fonds en banque en étant partiellement couverts contre le risque de faillite par un système de garantie des dépôts, géré par exemple par le Fonds de Garantie des Dépôts et de Résolution (FGDR) en France et par la \eng{Federal Deposit Insurance Corporation} (FDIC) aux États-Unis.

% Taux de refinancement
Grâce au taux de refinancement, la banque centrale peut prélever un revenu sur ce seigneuriage. Ceci crée un système à deux couches où l'épargnant subit un seigneuriage double~: celui de l'État lié à la monnaie de base, et celui des banques commerciales lié au crédit.

% Cycles du crédit économiques et financiers
De plus, l'expansion du crédit, encouragée par la protection de déposants, provoque des cycles économiques et financiers haussiers (malinvestissement) et baissiers, terriblement néfastes pour l'économie.

% Ce sur quoi il faut se concentrer est le seigneriage, c'est-à-dire la distorsion de la production de monnaie sur le marché libre, par laquelle est alimentée une organisation antagoniste, d'une manière quasi ésotérique.

\section*{L'inflation des prix}
\addcontentsline{toc}{section}{L'inflation des prix}

% Modèle pour une instance étatique unique

% Définition de l'inflation des prix
L'inflation des prix (du latin \emph{inflatio} qui signifie gonflement, enflure, dilatation) est la perte du pouvoir d'achat de la monnaie qui se traduit par une augmentation générale et durable des prix. Contrairement à ce qu'on croit toute hausse des prix n'est pas une inflation. À cause de son caractère durable, l'inflation est par nature structurelle et non conjoncturelle. Les mesures temporaires imposées par un État peuvent faire augmenter les prix, mais cet effet ne constitue pas en soi de l'inflation.

% Origine de l'inflation
L'inflation peut provenir d'une augmentation générale de la demande ou d'une diminution de l'offre de biens et de services. Elle peut être théoriquement être le fait de plusieurs facteurs comme l'inflation monétaire, la raréfaction de l'énergie, la destruction de richesse par la guerre ou la sortie des capitaux. En pratique, c'est-à-dire dans le cas d'une économie croissante, pacifiée et indépendante, l'inflation des prix à long terme est, en règle générale, une conséquence de l'inflation monétaire. % augmentation de l'impôt (la TVA par exemple) : comme le bouclier tarifaire, ne modifie pas le prix réel pratiqué

% --- L'inflation monétaire ---

L'inflation monétaire est l'excédent de production de monnaie par rapport à la production naturelle sur le marché libre\sendnote{Cette définition me vient de Guido Hülsmann pour qui l'inflation est «~l'augmentation de la quantité nominale d'un moyen d'échange au-delà de la quantité qui aurait été produite sur le marché libre~» (Jörg Guido Hülsmann, \emph{The Ethics of Money Production}, p. 85).}.

Une monnaie-marchandise de marché comme l'or possède un pouvoir d'achat stable. Toute hausse de pouvoir d'achat de la monnaie augmente la rentabilité de son extraction et de la fonte de bijoux, et sa quantité en circulation~; à l'inverse, toute baisse du pouvoir d'achat réduit la rentabilité de cette production, et pousse les producteurs à rediriger leurs capitaux vers une activité plus lucrative. Comme toute autre production, les producteurs répondent à la demande de monnaie et il s'ensuit que sa valeur d'échange reste stable. Cette stabilité a pu être vérifiée dans l'histoire, si l'on met les innovations techniques de côté.

La monnaie étatique n'est cependant pas soumise à ce type de contrainte. Le seigneuriage comme on l'a vu consiste essentiellement à maintenir un monopole sur la production et à imposer l'usage de la monnaie, pour faire en sorte que sa valeur d'échange objective soit supérieure à son coût de fabrication. Ce caractère fixe empêche l'État de correctement jauger la production~: s'il produit trop peu de monnaie, le pouvoir d'achat de la monnaie s'apprécie~; s'il en produit trop, son pouvoir d'achat se déprécie. De plus, l'État n'hésite pas à sacrifier le pouvoir d'achat à long terme de la monnaie pour en tirer un revenu à court terme, par exemple dans la contexte d'une crise (militaire, politique, sanitaire).

% L'inflation des prix
L'inflation des prix est un phénomène qui a pu être observé dans de nombreuses économies. Elle a ainsi été observé dans l'Empire romain et a culminé sous le règne de l'empereur Dioclétien (qui coïncide avec le début de l'effondrement de l'Empire). Elle a également pu être observée dans nos économies modernes, suite aux deux guerres mondiales, dans les années 1970 et plus récemment dans les années 2020.

% --- L'effet Cantillon ---

Le phénomène de l'inflation des prix est souvent mal appréhendé car il n'est pas le phénomène uniforme et brusque que l'on a tendance à imaginer. Une injection de monnaie dans l'économie exerce un effet progressif et différencié sur les prix au fur et à mesure que la monnaie se propage par les échanges. C'est ce qu'on appelle l'effet Cantillon, observé en 1730 par l'économiste physiocrate Richard Cantillon dans son \emph{Essai sur la Nature du Commerce en Général} où il déclarait qu'«~une augmentation d'argent effectif [causait] dans un État une augmentation proportionnée de consommation, qui [produisait] par degrés l'augmentation des prix\sendnote{Richard Cantillon, \emph{Essai sur la Nature du Commerce en Général}, 1755.}~». % L'inflation des prix n'est ainsi pas un phénomène uniforme, ni instantané.

Cet effet Cantillon s'applique à l'espace et au temps. La monnaie produite peut se retrouver en certaines régions spécifiques (les régions urbanisées par exemple). Elle peut se concentrer dans certaines régions du monde. Elle peut se retrouver dans certaines secteurs économiques particuliers, comme la finance. Elle peut être ralentie par certaines pratiques, comme le paiement de salaires mensuels. Cependant, l'effet de la hausse de la quantité finit par se répercuter progressivement sur l'ensemble de l'économie.

Entretemps, les personnes proches de l'émission monétaire s'enrichissent. Le producteur de monnaie la dépense en apportant une demande supplémentaire et propose un prix supérieur pour obtenir le bien. Le commerçant qui la reçoit, devenu momentanément plus riche, réitère cette dépense plus généreuse auprès d'un autre commerçant. Et ce phénomène se poursuit jusqu'à atteindre les confins de la société économique, de telle sorte que les personnes les plus éloignées de l'émission monétaire s'en retrouvent les plus lésées.

% --- La mesure de l'inflation ---

Un autre aspect de l'inflation est le fait qu'elle est difficile à mesurer. L'inflation des prix est, comme on l'a dit, un gonflement qui met du temps à s'établir, et qui n'est pas uniforme. Il est donc ardu d'estimer dans quelle mesure les prix «~enflent~».

La mesure la plus utilisée pour mesurer la hausse des prix est l'indice des prix à la consommation (IPC), calculé par l'INSEE en France, qui correspond au niveau moyen des prix des biens et services consommés par les ménages, pondérés par leur part dans la consommation moyenne de ces ménages. Mais cet indice souffre de nombreux défauts évidents.

D'abord, il ne tient pas compte des innovations et des nouveaux produits. Deuxièmement, il prétend se limiter aux biens (et aux services) de consommation mais on ne peut pas déterminer si un bien est utilisé comme bien de consommation ou comme bien de production (il inclut les loyers immobiliers, mais pas les biens eux-mêmes par exemple). Troisièmement, l'indice doit aussi évoluer aussi au cours du temps, pour tenir compte des  évolue aussi au cours du temps, ce qui efface progressivement la pertinence d'une comparaison avec le passé. Tout ceci fait que l'indice minimise l'inflation réelle.

% --- Fuite des capitaux et hyperinflation ---

L'inflation des prix peut être banale et être maintenue sous un certain niveau, de sorte que le pouvoir d'achat de la monnaie. Mais il arrive que le phénomène s'emballe et conduise, \emph{in fine}, à la destruction de la dénomination. Dans ce cas, l'inflation n'est plus liée à la production de monnaie (qui peine à suivre le rythme), mais à la fuite de la valeur vers d'autres monnaies jugées plus fortes ou vers des biens liquides. L'inflation devient galopante~: c'est ce qu'on appelle l'hyperinflation\sendnote{La Commission européenne définit une économie hyperinflationniste par les caractéristiques suivantes~: 1) la population en général préfère conserver sa richesse en actifs non monétaires ou en une monnaie étrangère relativement stable~; 2) la population en général apprécie les montants monétaires, non pas dans la monnaie locale, mais dans une monnaie étrangère relativement stable, les prix pouvant être exprimés dans cette monnaie~; 3) les ventes et les achats à crédit sont conclus à des prix qui tiennent compte de la perte de pouvoir d'achat attendue durant la durée du crédit, même si cette période est courte~; 4) les taux d'intérêt, les salaires et les prix sont liés à un indice de prix~; et 5) le taux cumulé de l'inflation sur trois ans approche ou dépasse 100~\%. Elle est ainsi liée à la perte des fonctions de réserve de valeur et d'unité de compte de la monnaie. -- Voir IAS 29~: «~Information financière dans les économies hyperinflationnistes~», 29 octobre 2018~: \url{http://www.focusifrs.com/menu_gauche/normes_et_interpretations/textes_des_normes_et_interpretations/ias_29_information_financiere_dans_les_economies_hyperinflationnistes}.}.

% Fuite des capitaux
Cette hyperinflation peut provenir en particulier d'une fuite des capitaux en dehors de l'économie. C'est la deuxième raison derrière le contrôle des capitaux, qui ne sert pas uniquement à soutenir l'impôt. C'est la raison d'être du contrôle des changes qui sert à maintenir la valeur de la monnaie artificiellement haut, souvent sous prétexte de «~lutte contre la spéculation~». % Un contrôle particulier est le contrôle des changes consistant à réglementer l'achat et la vente de monnaies concurrentes, et plus spécifiquement des devises étrangères, par ses ressortissants.

% --- Tentatives de papier-monnaie et exemples d'hyperinflation ---

Les exemples d'hyperinflation dans l'histoire sont nombreux. Ils coïncident la plupart du temps avec les premières expériences de papier-monnaie durant une période troublée par la guerre ou par la révolution. D'une part, l'État préfère détruire sa monnaie et son économie plutôt que de perdre la guerre. D'autre part, l'État concurrent fait tout pour que cela se passe et que les habitants utilisent sa monnaie.

% Continental (1775, hyp. 1781)
Aux États-Unis, le continental (\eng{continental currency dollar}) émis par le Congrès entre 1775 et 1779 pour financer la guerre d'Indépendance, qui finit par ne plus être accepté par personne et termine en hyperinflation en 1781. % Le greenback, appelé à cause de l'encre verte utilisée pour imprimer le verso, émis pour financer la guerre de Sécession entre 1862 et 1871\sendnote{Il y a eu en réalité deux greenbacks~: la \eng{Demand Note} émise entre août 1861 et avril 1862, et la \eng{United States Note} émise entre février 1862 et 1871, qui avait cours légal et qui a fini par remplacer la première.}.

% Assignats (1791, hyp. 1793 - 1795)
L'assignat pendant la Révolution, était à l'origine un titre d'emprunt émis par le Trésor en 1789, dont la valeur était gagée sur les biens nationaux par assignation et qui est devenu une monnaie d'échange à cours forcé en 1791. Sa valeur s'est effondrée entre 1793 et 1795. Son cours légal, qui n'était plus respecté, a été supprimé en 1797.

% Papiermark (1914, hyp. 1922 - 1924)
L'Allemagne qui avait instauré \eng{papiermark} en 1914, s'est retrouvée à payer un lourd tribut à ses adversaires après la Grande Guerre. Par conséquent, le papiermark a fini par connaître l'hyperinflation entre 1922 et 1924.

% Rouble (1914, hyp. 1917 - 1924)
La Russie tsariste a également suspendu la convertibilité de sa monnaie et a imposé son cours forcé. Après les révolutions de Février et d'Octobre, le Tsar a abdiqué et la Russie s'est retirée de la guerre, mais elle est cependant rentrée dans une guerre civile. C'est durant cette période que le rouble a connu une hyperinflation qui a duré jusqu'au retour d'un rouble-or en 1924.

% Yuan (1935, hyp. 1946 - 1949)
La Chine a également connu une période d'hyperinflation. En 1935, le Parti nationaliste au pouvoir fait du yuan une monnaie-papier et interdit l'utilisation du yuan-argent par le public. À cause de la guerre civile contre les communistes et de la guerre contre le Japon, la création monétaire est exploitée. Le yuan entre en hyperinflation après la guerre en 1946. Elle se conclut avec le rétablissement d'un yuan-or entre 1948 et 1949, puis avec la victoire des communistes et l'exil des nationalistes vers Taïwan en 1949.

% Exemples plus récents : rouble soviétique (1922, hyp. 1991 - 1993), dollar zimbabwéen (1980, hyp. 2000 - 2009), bolivar fort vénézuélien (2008, hyp. 2016 - ?)
D'autres exemples plus récents existent. On peut citer le cas de l'hyperinflation du rouble soviétique suite à la dislocation de l'URSS en 1991, qui s'est terminée par la création d'un nouveau rouble en Russie. Une autre cas est celui du dollar zimbabwéen, qui a perdu la totalité de sa valeur entre 2000 et 2009, pour être finalement remplacé par le dollar étasunien en 2009. Un dernier cas est celui du bolivar vénézuélien, qui est en hyperinflation depuis 2016.

\section*{Les banques et la prise de contrôle sur la monnaie}
\addcontentsline{toc}{section}{Les banques et la prise de contrôle sur la monnaie}

% Définition de la banque
L'activité bancaire (ou la banque) est l'activité consistant à faire commerce de la monnaie et du crédit, en recevant des capitaux sous la forme de dépôts, en émettant des prêts et en offrant des services de paiements. Au sens actuel, la banque est par essence un organisme de crédit~: sauf indication contraire, le déposant prête sa monnaie à la banque ce qui permet à cette dernière de constituer une réserve et d'émettre du crédit à partir de là. Le déposant est pour cela récompensé par un intérêt (qui peut par exemple se manifester par la gratuité de sa tenue de compte), au risque de tout perdre dans le cas de la faillite de la banque.

% Wiktionary: Commerce de l'argent et du crédit qui consiste à recevoir des capitaux en compte courant avec ou sans intérêt ; à échanger des effets ou à les escompter avec des espèces, à des taux et moyennant des commissions variables ; à exécuter pour le compte de tiers toutes opérations de ce genre et à se charger de tous services financiers ; à créer et à émettre des lettres de change ; d'une façon générale, commercer de l'argent, ainsi que des titres et valeurs.

% --- Banques privées ---

% Premières banques
Les banques existaient déjà lors de l'Antiquité, mais elles n'avaient pas du tout la forme aboutie qu'elles ont prises depuis. L'activité bancaire moderne est née au cours de la Renaissance en Italie du Nord («~Lombardie~»). Elle a émergé au sein des cités-États florissantes que sont Pise, Venise, Gênes, Milan et Florence, ayant alors acquis leur indépendance vis-à-vis de de l'Empire byzantin et du Saint-Empire germanique.  % milieu du \textsc{xv}\ieme{}~siècle

% Change de monnaies
La banque est née du change des monnaies. Les banquiers étaient originellement des changeurs qui procédaient à la conversion d'une monnaie en une autre et qui en tiraient profit. Le terme de banque provient de l'italien \emph{banca} qui désignait la table en bois utilisée par ces changeurs. Ils pratiquaient également le prêt sur gage, ce qui explique pourquoi on parle parfois de crédit lombard pour désigner cette activité.

% Deux innovations
Deux innovations majeures ont soutenu le développement de l'activité bancaire moderne~: le dépôt à vue et la lettre de change.

% Dépôt à vue
Le dépôt à vue est un dépôt, rémunéré ou non, dont les fonds peuvent être retirés partiellement ou totalement à tout instant, dans la limite de la capacité de la banque au regard de sa liquidité et de sa solvabilité. Il se base sur la comptabilité en partie double, qui consiste à enregistrer deux fois un transfert (en consignant à la fois l'origine et la destination des fonds) pour en vérifier la validité\sendnote{La comptabilité a été facilitée par l'importation du système de numération indo-arabe, notamment recensée au sein de l'ouvrage \emph{Liber abaci} écrit par Leonardo Fibonacci en 1202. La comptabilité en partie double a été codifiée par Luca Pacioli au sein de son ouvrage \emph{Summa de arithmetica geometria proportioni et propotionalita}, publié en 1494.}. Le dépôt à vue permettait de conserver de la valeur en toute sécurité et à moindre coût, d'où son succès dans la population fortunée.

% Lettre de change et billet à ordre
La lettre de change est un écrit par lequel une un créancier donne à un débiteur l'ordre de payer à l'échéance fixée, une certaine somme, à un bénéficiaire. La lettre de change était un moyen de paiement international, permettant d'éviter de déplacer des pièces d'or et d'argent sur de longues distances. Elle s'est très vite transformée en billet à ordre, payé à vue au porteur en espèces sans identification requise, qui était beaucoup plus pratique à transmettre.

% Substituts monétaires
Ces deux évolutions ont fait que c'était de moins en moins la monnaie elle-même qui était échangée, mais les substituts monétaires basés sur cette monnaie. Le dépôt à vue a donné le compte courant (de l'italien \emph{conto corrente}) permettant d'écrire des chèques et d'initier des virements~; la lettre de change a donné le billet de banque.

% Monétisation du crédit
Dans le cas des banques privées, ces substituts monétaires constituaient généralement du crédit bancaire. Même si la banque pouvait conserver l'intégralité des fonds pour garantir ses substituts, elle choisissait le plus souvent d'avoir seulement une fraction de la monnaie en réserve. Cette pratique permettait d'éviter de demander aux clients de payer des frais de conservation, voire de leur verser un intérêt.\sendnote{Voir George Selgin, \eng{Those Dishonest Goldsmiths}, 2011~: \url{https://papers.ssrn.com/sol3/papers.cfm?abstract_id=1589709}.}

% Cycles économiques et financiers
Cette utilisation du crédit comme instrument monétaire a eu pour conséquence de créer des cycles, où l'expansion du crédit (croissance) était suivie d'un resserrement (dépression, récession). L'expansion du crédit était néanmoins limitée par le fait qu'une banque faisait faillite si elle n'avait plus la liquidité nécessaire. Une panique bancaire (\eng{bank run}) pouvait faire tomber une banque, ce qui imposait à la banque d'être prudente.

% La monnaie représentative / la réserve intégrale a été le grand mensonge par lequel l'État a pu imposer progressivement le papier-monnaie. Full-reserve banking, 100% reserve banking, narrow banking.

% --- Banques publiques ---

% Banques publiques municipales
L'activité bancaire a été peu à peu récupérée par le pouvoir et centralisée dans les mains d'un banque publique, c'est-à-dire d'une banque bénéficiant d'un statut légal particulier octroyé par le pouvoir. Ces banques publiques municipales justifiaient leur existence par le fait qu'elles pratiquaient la réserve intégrale et ne présentaient (théoriquement) pas de risque de crédit. La première banque publique de ce type est la \emph{Taula de canvi} de Barcelone fondée en 1401. À Gênes, la \emph{Casa delle Compere di San Giorgio} a été fondée en 1407, et après avoir fermé en 1444, celle-ci a repris ses activités en 1586. À Venise, la Banque du Rialto (\emph{Banco della Piazza di Rialto}) a été créée en 1587. La Banque du Giro (\emph{Banco del Giro}) a été créée en 1619 et a absorbé la Banque du Rialto en 1637. La Banque d'Amsterdam (\eng{Amsterdamsche Wisselbank}) a été créée en 1609. La Banque de Stockholm (\eng{Stockholms Banco}) créée en 1656 par Johan Palmstruch, qui a été condamné à mort.

% Monnaie représentative et réserve intégrale
Ces banques publiques garantissaient a priori les billets (monnaie représentative) et les dépôts (réserve intégrale). Toutefois, à cause de la pression du pouvoir, la garantie ne tenait qu'un temps. C'est le cas de la Banque d'Amsterdam, qui a largement crédité le compte de la Compagnie néerlandaise des Indes orientales au cours de son existence.

% Banques nationales
Ces banques publiques ont ensuite été étendues au niveau national. En Suède, la Banque des États du royaume (\eng{Riksens Ständers Bank}), plus tard renommée en banque royale de Suède (\eng{Sveriges Riksbank}), a été fondée en 1668 sur les ruines de la Banque de Stockholm. La Banque d'Angleterre, fondée sur le modèle de la Banque d'Amsterdam, a vu le jour en 1694.

% Même si ces banques sont réputées privées (Banque de France, Banque d'Angleterre), nous considérons qu'elles sont publiques en raison des privilèges accordés.

% Banque centrale~/~nationale qui assure la convertibilité des billets en métal précieux (typiquement l'or et l'argent). La banque centrale monopolise cette convertibilité.

% Banque nationale en France
En France, la première banque nationale a probablement été la Banque générale, créée en 1716 par l'écossais John Law (alors appelé Jean Lass en France\sendnote{Voltaire, \eng{Précis du siècle de Louis XV}, 1768~: \url{https://fr.wikisource.org/wiki/Précis_du_siècle_de_Louis_XV/Chapitre_2}.}) sur le modèle de la Banque d'Angleterre et devenue Banque royale en 1719. Le système de Law, étroitement lié à la Compagnie du Mississippi, avait pour but de de prendre en charge la dette à court terme de l'État accumulée par le défunt Louis \textsc{xiv} et de développer le potentiel commercial de la Louisiane française en émettant des actions de la Compagnie. Les billets émis, initialement convertibles en or et en argent, ont servi à financer ces actions, ce qui a créé l'une des premières bulles financières mondiales de l'histoire. Le système a bien entendu fini par s'effondrer en 1720.\sendnote{Antoin E. Murphy, «~John Law et la bulle de la Compagnie du Mississippi~», \emph{L'Économie politique}, 2010/4 (n° 48), p. 7-22~: \url{https://www.cairn.info/revue-l-economie-politique-2010-4-page-7.htm}.}

Par la suite, le pouvoir royal français s'est contenté de gérer une Caisse d'escompte jusqu'à la Révolution. Une réelle banque nationale n'est revenue qu'avec la création de la Banque de France en 1800 par Napoléon Bonaparte.

% Banque nationale en Prusse
Outre-Rhin, la Banque royale de Prusse a été fondée par Frédéric \textsc{ii} en 1765. Elle a par la suite servi de modèle à la Reichsbank allemande, créée en 1876 après l'unification de l'Allemagne par Bismarck.

% Banque nationale aux États-Unis
Aux États-Unis, plusieurs tentatives de banques fédérales ont eu lieu après l'Indépendance~: la Banque de l'Amérique du Nord entre 1782 et 1785, élaborée par le financier Robert Morris~; la \eng{First Bank of the United States} entre 1791 et 1811, qui est l'œuvre du révolutionnaire fédéraliste Alexander Hamilton~; et la \eng{Second Bank of the United States} entre 1816 et 1836, en continuité avec la première. Toutefois, la construction d'une banque nationale a été retardée par la tendance individualiste et anti-fédéraliste des citoyens, incarnée par la président Andrew Jackson~: pendant plusieurs décennies, chaque état individuel a possédé sa propre banque indépendante dont le rôle était limité, ce qui a correspondu à une période relativement libre pour l'industrie bancaire privée. Ce n'est qu'après la guerre de Sécession qu'on a pu voir un système de banques nationales émerger où les banques publiques de New York possédaient un statut privilégié par rapport aux autres banques publiques (\eng{central reserve city banks}).\sendnote{Voir Murray Rothbard, \eng{A History of Money and Banking in the United States: The Colonial Era to World War \textsc{ii}}, 2002.} % La réelle banque nationale des États-Unis n'est venue qu'avec sa banque centrale~: la Réserve Fédérale créée en 1913.

% ---Banques centrales ---

% Définition d'une banque centrale
Cependant, ces banques nationales n'étaient pas encore des banques centrales. Une banque publique ne devient centrale qu'au moment où elle acquiert le monopole d'émission des billets. La Banque d'Angleterre a acquis ce privilège grâce au \eng{Bank Charter Act} de 1844. En Prusse, le décret du 11 avril 1846 a permis à la Banque royale de bénéficier d'un monopole d'émission sur le même modèle que le Royaume-Uni. La Banque de France a vu son privilège d'émission (à l'origine limité à Paris) être étendu à l'ensemble du territoire en 1848. Aux États-Unis, la banque centrale n'a été créée que tardivement au travers de la Réserve Fédérale créée en 1913.

La justification principale derrière la création d'une banque centrale était son rôle de prêteur en dernier ressort, théorisé au cours du \textsc{xix}\ieme{}~siècle\sendnote{Henry Thornton, \eng{An Enquiry into the Nature and Effects of the Paper Credit of Great Britain}, 1802~; Walter Bagehot, \eng{Lombard Street: A Description of the Money Market}, 1873.}. Ce rôle consiste à prêter de la monnaie créée pour l'occasion afin de fournir de la liquidité aux banques en difficulté lors du resserrement du crédit.

La banque centrale est pleinement intégrée à l'État. Il n'y a pas d'indépendance~: la banque centrale repose sur la force de l'État pour assurer son monopole et l'application du cours légal~; l'État quant à lui dépend de la banque centrale pour prélever un seigneuriage.

% Papier-monnaie
Cette implémentation des banques centrales a mené à une installation durable du papier-monnaie. Les tentatives de papier-monnaie dans l'histoire de l'Occident ont souvent échoué car les gens finissaient par revenir aux métaux précieux qui concurrençaient les billets de banque dans la circulation. Il a donc fallu que les billets deviennent monnaie courante avant de voir la monnaie fiduciaire être largement acceptée.\sendnote{Divers épisodes de cours forcé ont eu lieu au cours de l'histoire, mais ils étaient généralement temporaires. La Banque d'Angleterre a suspendu la convertibilité de ses billets entre 1797 et 1821 pour faire face à la fuite des capitaux résultant de la Guerre de la Première Coalition contre la France révolutionnaire et de l'éclatement de la bulle spéculative sur la terre aux États-Unis. En France, les billets de la Banque de France ont eu cours forcé entre 1848 et 1850, puis entre 1870 et 1875, toujours dans le cadre de crises nationales. Ce n'est qu'en 1914, avec l'entrée en guerre des États européens, que le non convertibilité en or a pu atteindre un statut permanent. Aux États-Unis il a fallu attendre la Nouvelle donne de Roosevelt en 1933 pour que la convertibilité directe soit interrompue.}

% --- Fonctionnement des banques centrales ---

% Rôle
Les banques centrales ont acquis aujourd'hui un rôle prépondérant. Leur rôle est cadré dans le but de limiter les épisodes d'hyperinflation. Leur mission est bien souvent de limiter l'indice des prix à la consommation à 2~\% par an, même si elles peuvent avoir d'autres objectifs comme la baisse du chômage. Leur politique monétaire a pour but d'intervenir dans l'économie. Trois missions lui sont généralement attribuées~: la production de la monnaie physique, le rachat de titres sur les marchés financiers et l'influence sur l'émission du crédit par le biais de taux directeurs.

% Production de la monnaie
Tout d'abord, la banque centrale peut avoir pour tâche de fabriquer le papier-monnaie. Mais cette tâche peut également être déléguée. La Fed délègue cette tâche au Bureau de la gravure et de l'impression. La BCE aux banques nationales des États-membres.

% Rachat de titres sur les marchés financiers
Ensuite, la banque centrale peut se rendre sur les marchés financiers afin d'y intervenir. Elle réalise traditionnellement des opérations d'open market c'est-à-dire des achats et des ventes de titres, en particulier d'obligations publiques (bons du Trésor), sur le marché interbancaire. Les politiques monétaires non conventionnelles lui permettent également de mener des opérations d'assouplissement quantitatif, plus longues et plus agressives, ce qui permet d'apporter de la liquidité pour soutenir l'économie en cas de crise. Mais ces achats permettent surtout de financer la dette de l'État~: puisque la taille du bilan est strictement croissante\sendnote{Fed~: \url{https://www.federalreserve.gov/monetarypolicy/bst_recenttrends.htm}~; BCE~: \url{https://www.ecb.europa.eu/pub/annual/balance/html/index.fr.html}.}, on peut considérer qu'une partie de ces achats représente de la pure création monétaire.

% Taux directeurs
Enfin, la banque centrale influence l'émission du crédit bancaire, à l'aide de ses taux directeurs. Les taux directeurs, appelés différemment selon les pays, sont généralement au nombre de trois~: le taux de refinancement, le taux de prêt marginal et le taux de rémunération des dépôts. Le taux de refinancement est le taux pour lequel les banques commerciales peuvent obtenir de la monnaie centrale et sert à limiter la création de crédit bancaire. Le taux de prêt marginal est le taux de prêt à court terme en cas d'urgence et maintient le système bancaire en place en cas de crise grave (prêteur en dernier ressort). Le taux de rémunération des dépôts est comme son nom l'indique, le taux d'intérêt payé par la banque centrale pour conserver des liquidités en réserve et permet de décourager ou d'encourager le prêt commercial. Le taux de prêt marginal est nécessairement plus élevé que le taux de refinancement et ce dernier est nécessaire plus élevé que le taux de rémunération des dépôts. Ces taux ne sont pas des taux d'intérêt issus du marché et peuvent donc être négatifs.

C'est ce système banco-monétaire auquel faisait référence Satoshi Nakamoto lorsqu'en février 2009, soucieux d'amener les gens à s'intéresser à Bitcoin, il écrivait~:

\begin{quote}
«~Le problème fondamental de la monnaie conventionnelle est toute la confiance nécessaire pour la faire fonctionner. Il faut faire confiance à la banque centrale pour qu'elle ne déprécie pas la monnaie, mais l'histoire des monnaies fiat est pleine de violations de cette confiance. Il faut faire confiance aux banques pour détenir notre argent et le transférer par voie électronique, mais elles le prêtent par vagues de bulles de crédit avec à peine une fraction en réserve.\sendnote{Satoshi Nakamoto, \eng{Bitcoin open source implementation of P2P currency}, 11 février 2009~: \url{https://p2pfoundation.ning.com/forum/topics/bitcoin-open-source}.}~»
\end{quote}

% Ces taux ne sont pas des taux d'intérêt. On dit que les banques commerciales empruntent de l'argent à la banque centrale, mais il ne s'agit pas d'un prêt de monnaie existante~: c'est de la création monétaire.

% --- Prise de contrôle ---

Ainsi, la prise de contrôle totale sur les billets de banque a été finalisée avec la monnaie fiat. Les expériences de papier-monnaie dans l'histoire n'ont pas abouti à une adoption parce qu'il subsistait une utilisation forte des métaux précieux dans la population. L'étalon-or et la démonétisation de l'argent ont préparé le terrain.

La prise de contrôle sur les dépôts à vue est en cours. Avec la monétisation générale du crédit bancaire caractérisée par la garantie étatique des dépôts, tous les ingrédients sont présents pour que la mutation finale ait lieu. C'est l'objet du développement des monnaies numériques de banque centrale.

\section*{Les monnaies numériques de banque centrale}
\addcontentsline{toc}{section}{Les monnaies numériques de banque centrale}

Une monnaie numérique de banque centrale (MNBC), de l'anglais \eng{central bank digital currencies} (CBDC), est une monnaie fiduciaire numérique émise par une banque centrale. Il s'agit d'une sorte de monnaie entièrement numérique qui ne représente pas une créance. Les monnaies numériques de banque centrale sont \textcolor{darkgray}{actuellement} développées autour du monde. Leur déploiement pourrait constituer une évolution majeure dans l'histoire de la monnaie par l'accaparement des dépôts bancaires par l'État.

% --- Origine ---

Disposer d'une monnaie qui serait gérée intégralement par une banque centrale et qui concurrencerait la monnaie scripturale des banques commerciales n'est pas une idée nouvelle. Cette idée remonte en particulier à une période antérieure à la démocratisation d'Internet. On la retrouve sous la plume de l'économiste keynésien James Tobin, lauréat du prix Nobel, qui faisait une suggestion similaire en 1987 en écrivant~:

\begin{quote}
«~Je pense que l'État devrait mettre à la disposition du public un intermédiaire de paiement offrant la commodité des dépôts et la sécurité des espèces, qui serait essentiellement de la monnaie sous forme de dépôt, transférable pour tout montant par chèque ou autre ordre.\sendnote{James Tobin, \eng{The Case for Preserving Regulatory Distinctions}, 1987~: \url{https://www.kansascityfed.org/documents/3828/1987-S87TOBIN.pdf}.}~»
\end{quote} % "I think the government should make available to the public a medium with the convenience of deposits and the safety of currency, essentially currency on deposit, transferable in any amount by check or other order."

% Fedcoin~: article satirique et réflexions
Avec l'émergence de Bitcoin dans les années 2010, l'idée d'une monnaie numérique gérée par une banque centrale et mise à disposition des particuliers a été remise au goût du jour. Et elle est tout d'abord venue de l'intérieur de la communauté de Bitcoin~: c'est un utilisateur qui l'a évoquée le 26 mars 2013 sous la forme de «~Fedcoin~», un concept satirique d'une «~une alternative centralisée aux monnaies pair-à-pair~» qui serait contrôlée par la Réserve fédérale des États-Unis\sendnote{peculium, \eng{Fedcoin: A centrally-issued alternative to peer-to-peer currencies}, 26 mars 2013~: \url{http://peculium.net/2013/03/26/fedcoin-a-centrally-issued-alternative-to-peer-to-peer-currencies/}~; archive~: \url{https://web.archive.org/web/20130404231341/http://peculium.net/2013/03/26/fedcoin-a-centrally-issued-alternative-to-peer-to-peer-currencies/}}. Du côté de l'Europe, l'idée d'un Eurocoin a été évoquée par Bitcoin.fr le 1\ier{} avril 2014\sendnote{Jean-Luc de Bitcoin.fr, \emph{Naissance de l'Eurocoin}, 1\ier{} avril 2014~: \url{https://bitcoin.fr/naissance-de-l-eurocoin/}.}. Bien qu'ironique, cette idée a mené à diverses réflexions sur la pertinence d'un tel système et sur les conséquences de sa potentielle implémentation\sendnote{John Paul Koning, \eng{Fedcoin}, 19 octobre 2014~: \url{https://jpkoning.blogspot.com/2014/10/fedcoin.html}.}.

% Fedcoin~: article de David Andolfatto
Le sujet est devenu plus sérieux au début de l'année 2015 lorsque David Andolfatto, alors vice-président de la Federal Reserve Bank de Saint-Louis, en a fait la promotion dans une présentation donnée durant la conférence, puis dans un article publié sur son blog\sendnote{David Andolfatto, \eng{Fedcoin: On the Desirability of a Government Cryptocurrency}, 3 février 2015~: \url{https://andolfatto.blogspot.com/2015/02/fedcoin-on-desirability-of-government.html}.}. Sa proposition était de faire en sorte que, contrairement à Bitcoin, le système d'émission monétaire soit contrôlé par la Réserve fédérale, qui se chargerait d'assurer la convertibilité de l'unité numérique en dollars. Elle restait néanmoins mesurée~: pour Andolfatto, Fedcoin devrait être un système ouvert et anonyme.

% Central bank digital currency~: discours de Ben Broadbent
Le concept de monnaie numérique de banque centrale a pleinement émergé avec le discours du 2 mars 2016 de Ben Broadbent, gouverneur adjoint pour la politique monétaire à la Banque d'Angleterre, prononcé à la London School of Economics, qui créait le terme de «~central bank digital currency~»\sendnote{Ben Broadbent, \eng{Central banks and digital currencies}, 2 mars 2016~: \url{https://www.bankofengland.co.uk/speech/2016/central-banks-and-digital-currencies}.}. Dans ce discours, le banquier expliquait comment un registre distribué pouvait permettre de remplacer l'actuel modèle de compensation et de règlement interbancaire, d'en élargir l'accès aux acteurs financiers et aux particuliers en leur permettant de posséder un compte auprès de la banque centrale, et de faire ainsi concurrence à l'argent liquide et aux dépôts dans les banques commerciales.

% Mise en œuvre
Depuis, le concept a été mis en œuvre. La Banque populaire de Chine, qui a monté programme de recherche (appelé \eng{Digital Currency Electronic Payment} ou DCEP\sendnote{Xinyu Liu, Fan Lu, Wanlu Shan, Jiayuan Zhang, \eng{The Progress of Digital Currency Electronic Payment}, 2021~: \url{https://www.atlantis-press.com/article/125965904.pdf}.}) dès 2014, a commencé à déployer progressivement son yuan numérique (\eng{digital renminbi}) depuis 2020. La Riksbank suédoise a envisagé de mettre en place une couronne électronique (ou e-Krona) en novembre 2016\sendnote{Cecilia Skingsley, \eng{Skingsley: Borde Riksbanken ge ut e-kronor?}, 16 novembre 2016~: \url{https://www.riksbank.se/sv/press-och-publicerat/Tal/2016/Skingsley-Borde-Riksbanken-ge-ut-e-kronor/}~; archive~: \url{https://web.archive.org/web/20161117155655/https://www.riksbank.se/sv/press-och-publicerat/Tal/2016/Skingsley-Borde-Riksbanken-ge-ut-e-kronor/}.}, qui est toujours en phase de tests.

Aux États-Unis, l'effort est pris en charge par la \eng{Digital Currency Initiative} du MIT Media Lab, une initiative créée en 2015 dans le but «~de réunir les esprits les plus brillants [...] pour mener les recherches nécessaires au développement des monnaies numériques et de la technologie blockchain\sendnote{\url{https://dci.mit.edu/about}}~» et qui a notamment financé certains développeurs de Bitcoin Core comme Gavin Andresen, Wladimir van der Laan, Cory Fields. Cette initiative a abouti au projet Hamilton en février 2022, un prototype de monnaie numérique développé conjointement avec la \eng{Federal Reserve Bank} de Boston\sendnote{\url{https://www.bostonfed.org/publications/one-time-pubs/project-hamilton-phase-1-executive-summary.aspx}}.

Du côté de la Grande-Bretagne, la Banque d'Angleterre a annoncé former un groupe de travail en avril 2021 en collaboration avec le trésor de Sa Majesté\sendnote{Bank of England, \eng{Bank of England statement on Central Bank Digital Currency}, 19 avril 2021~: \url{https://www.bankofengland.co.uk/news/2021/april/bank-of-england-statement-on-central-bank-digital-currency}.}. En Europe continentale, la BCE a a annoncé en juillet 2021 vouloir développer un euro numérique\sendnote{\url{https://www.ecb.europa.eu/press/pr/date/2021/html/ecb.pr210714~d99198ea23.en.html}}.

% --- Concept ---

% Monnaie centrale
La monnaie numérique de banque centrale se fonde sur un concept déjà existant~: celui de la monnaie centrale, qui est constituée des avoirs monétaires détenus par les banques commerciales auprès de la banque centrale. Cette monnaie interbancaire est destinée à fluidifier les règlements entre les banques, plutôt que de passer par des espèces. Elle constitue, avec les pièces et les billets en circulation, ce qu'on appelle la base monétaire ou monnaie de base. Celle-ci est fiduciaire par nature, dans le sens où elle tire sa valeur essentiellement de la confiance que lui portent ses utilisateurs et pas d'une propriété intrinsèque.

% Monnaie numérique
L'idée derrière la monnaie numérique de banque centrale est d'étendre l'accès de cette monnaie centrale aux autres entreprises et aux particuliers. Les banques centrales parlent parfois de «~MNBC de détail~» (\eng{retail CBDC}) pour différencier ce projet de celui d'une modernisation de la monnaie centrale existante, qui constituerait une «~MNBC de gros~» (\eng{wholesale CBDC})\sendnote{Fabio Panetta, \eng{Demystifying wholesale central bank digital currency
}, 26 septembre 2022~: \url{https://www.ecb.europa.eu/press/key/date/2022/html/ecb.sp220926~5f9b85685a.en.html}.}. Nous parlons ici de la MNBC de détail.

% --- Technique ---

% Registre distribué
Une monnaie numérique serait basée sur un registre distribué, utilisant un mécanisme de consensus de type classique, très bien adapté pour traiter un grand nombre de transactions. Cela permettrait de répliquer les données financières sur un ensemble de serveurs et éviter ainsi toute perte en cas de panne ou de cyberattaque.

% Identification
Le système serait accessible via une identification de l'utilisation, probablement basée sur une identité numérique, dans le but de satisfaire les exigences de lutte contre le blanchiment et le financement du terrorisme. Les transactions des utilisateurs seraient cachées au public, mais pourraient être observées par une autorité compétente.

% Programmabilité
Comme tout système informatique, une tel système serait programmable, et des conditions de dépenses pourraient être ajoutés aux fonds. De plus, ce système pourrait être amélioré au cours du temps pour inclure de nouvelles fonctionnalités.

% Avantages
Les apports directs de la monnaie numérique pour l'utilisateur seraient multiples. D'abord, elle éliminerait le risque de contrepartie lié au crédit~: l'utilisateur pourrait jouir théoriquement de tous les avantages apportés par un compte bancaire sans subir le risque de faillite de la banque. Ensuite, elle fournirait une plus grande accessibilité et favoriserait l'inclusion financière en permettant de «~bancariser les non-bancarisés~» à moindre frais. Enfin, elle automatiserait les opérations financières de façon à améliorer considérablement la qualité des services en ligne.

% Inconvénients
Grâce à ces avantages, la monnaie numérique de banque centrale paraît représenter un progrès, une modernisation de la monnaie physique dépassée par la numérisation de la société. Toutefois, ce serait ignorer son potentiel majeur pour le pouvoir et les inconvénients majeurs pour l'utilisateur individuel.

% --- Pontentiel de contrôle ---

Le potentiel des monnaies numériques de banque centrale est double. Premièrement, la monnaie numérique de banque centrale a le potentiel d'apporter un contrôle financier total.

D'une part, la généralisation de la monnaie de banque centrale formerait une base légale à partir de laquelle supprimer l'argent liquide. En effet, contrairement au crédit bancaire, la MNBC constitue une monnaie de base dont il est aisé de définir le cours légal sur le territoire. On pourrait donc assister à une disparition progressive des supports physiques de la monnaie.

D'autre part, elle permettrait d'améliorer la surveillance financière et offrirait une possibilité d'intervention supérieure, notamment grâce aux outils automatisés et à l'intelligence artificielle. En particulier, une MNBC faciliterait la collecte de l'impôt, le prélèvement pouvant se faire directement sur le compte des contribuables.

Cet aspect est traité dans la section du chapitre~\ref{ch:censure} sur la résistance à la censure.

% --- Potentiel inflationniste ---

Deuxièmement, la monnaie numérique de banque centrale possède un potentiel inflationniste non négligeable. D'une part, le remplacement de l'argent liquide permettrait d'éliminer les coûts de production, de distribution et de destruction des supports monétaires (pièces et billets). Cela améliorerait le seigneuriage sur la monnaie de base, en diminuant largement le coût de production. C'est déjà le cas avec la monnaie centrale.

D'autre part, le remplacement progressif du crédit bancaire permettrait de récupérer le seigneuriage réalisé sur le crédit par les banques commerciales, comme cela se fait déjà partiellement grâce au taux de refinancement. Cette capture se ferait aux dépend des banques, qui verrait leur capacité à prêter être réduite voire annihilée. C'est pourquoi elles devraient gagner quelque chose au change, par exemple en obtenir un rôle d'intermédiaire dans le système\sendnote{C'est le sens de l'idée de MNBC «~synthétique~» évoquée par Tobias Adrian (économiste du FMI) en 2019 (\eng{Stablecoins, Central Bank Digital Currencies, and Cross-Border Payments: A New Look at the International Monetary System. Remarks by Tobias Adrian at the IMF-Swiss National Bank Conference}, 13 mai 2019~: \url{https://www.imf.org/en/News/Articles/2019/05/13/sp051419-stablecoins-central-bank-digital-currencies-and-cross-border-payments}). Cette idée a été intégrée dans le prototype Aurum de la Banque des règlements internationaux (BRI) présenté en octobre 2022 (\eng{Project Aurum: a prototype for two-tier central bank digital currency (CBDC)}, 21 octobre 2021~: \url{https://www.bis.org/publ/othp57.htm}).}.

% Une seule banque, manifeste du parti communiste
Les banques commerciales pourraient être pleinement absorbées par la banque centrale, dont elles deviendraient les succursales. Ainsi, le vieux rêve marxiste de centraliser le crédit entre les mains d'une seule banque serait réalisé\sendnote{Le point 5 du programme dressé dans le Manifeste du parti communiste prône la «~centralisation du crédit entre les mains de l'État, au moyen d'une banque nationale, dont le capital appartiendra à l'État et qui jouira d'un monopole exclusif.~» -- Karl Marx, \eng{Manifeste du parti communiste}, février 1848.}. À l'instar de la Gosbank, la banque centrale de l'Union soviétique et seule banque autorisée 1932 et 1987, cette banque unique suivrait les directives du pouvoir central en accordant des prêts financés par création monétaire, non aux emprunteurs solvables, mais aux entités favorisées par la planification économique.

% --- Acceptation ---

Quand on voit les dangers que créent la généralisation d'un tel système, on peut penser que la population ne pourrait pas accepter la transition. Personne ne  personne ne veut naturellement adopter un tel système, \textcolor{darkgray}{comme on le voit au Nigéria}. En Occident, une réaction de rejet existe, notamment à droite, et des personnalités publiques attachées aux libertés ont déjà affirmé leur opposition, comme le le lanceur d'alerte Edward Snowden qui qualifiait cette monnaie numérique de «~monnaie cryptofasciste\sendnote{Edward Snowden, \eng{Your Money AND Your Life}, 9 octobre 2021~: \url{https://edwardsnowden.substack.com/p/cbdcs}.}~» en octobre 2021.

%
L'acceptation promet donc d'être complexe. Mais elle est loin d'être impossible~: le carburant de l'État étant la tentation de s'emparer du bien du voisin, l'acceptation pourrait être mue par une telle envie, qui se manifesterait par un système de récompense et de punition.

% Récompense (carotte)
La récompense devrait être présentée dans un premier temps. Le récompense serait constituée de diverses subventions pour encourager l'usage, versées aux commerçants et aux consommateurs. C'est ce qui est déjà pratiqué en Chine dans le cadre du yuan numérique\sendnote{China Daily, \eng{E-CNY boosts holiday consumption}, 1\ier{} février 2023~: \url{https://www.chinadaily.com.cn/a/202302/01/WS63d9bb3fa31057c47ebac36e.html}}. Cela rappelle les \eng{airdrops} (largages) pratiqués dans le milieu des crypto-actifs dans le but d'attirer l'attention sur un projet.

% Punition (bâton)
Dans un second temps, une fois la base d'utilisateurs suffisamment large, la punition pourrait commencer à être appliquée. Le cours légal de la monnaie numérique pourrait être imposé pour contraindre les commerçants à l'accepter. Certains services publics ne pourraient ne plus être accessibles sans compte à la banque centrale. La mouvance anti-MNBC, jugée complotiste, pourrait être censurée.

%
Quoi qu'il en soit, la monnaie numérique de banque centrale repose, comme pour toute mesure étatique sur l'acceptation de la population générale. L'opinion publique est donc le champ de bataille ici, mais on peut être en droit d'imaginer que l'État l'emportera \emph{in fine}, comme il l'a fait avec le papier-monnaie. Mais heureusement, Bitcoin existe.

% le succès de la monnaie numérique repose sur l'abandon progressif de l'argent liquide et sa diabolisation dans l'opinion publique (n'est-il pas déjà associé au crime, au blanchiment, à la pollution, à la transmission de maladies ?)

% Le grand mensonge de l'État dans le cadre de sa prise de contrôle sur la monnaie est de persuader les gens que le crédit est de la monnaie et que la monnaie est du crédit. Celui-ci lui a permis de prendre le contrôle sur les billets de banque, et lui permet peu à peu de prendre le contrôle sur les dépôts.

\section*{L'impérialisme monétaire} % Centralisation du pouvoir
\addcontentsline{toc}{section}{L'impérialisme monétaire}

% Introduction sur l'impérialisme monétaire
Une monnaie ne sert pas uniquement à prélever la richesses des citoyens à l'intérieur du territoire entièrement dominé par un État, mais peut aussi être utilisée pour prélever celle des personnes se trouvant dans le reste du monde. Cette pratique est appelé l'impérialisme monétaire\sendnote{Voir Hans-Hermann Hoppe, \eng{Banking, Nation States, and International Politics: A Sociological Reconstruction of the Present Economic Order}, 1990~: \url{https://mises.org/library/banking-nation-states-and-international-politics-sociological-reconstruction-present}.}.

% Définition de l'impérialisme
L'impérialisme désigne la volonté de conquête d'un État, visant à mettre d'autres États sous sa dépendance politique, économique ou culturelle. Au sens militaire, il s'agit d'étendre le territoire contrôlé aux dépens d'autres États pour accroître le prélèvement total. Le résultat est une empire, qui se distingue d'une nation par le fait qu'il intègre des différences ethniques fortes. L'histoire regorge d'occurrences de tels empires comme l'Empire akkadien, l'Empire romain, l'Empire mongol ou encore l'Empire colonial britannique. L'impérialisme peut se manifester par le contrôle territorial direct, par le prélèvement d'un impôt sur l'économie \emph{via} la domination des voies commerciales, ou d'une ingérence dans les affaires politiques internes du pays, pouvant amener à la mise en place d'un gouvernement fantoche obéissant aux directives du pouvoir central.

% Définition de l'impéralisme monétaire
L'impérialisme se transcrit dans le domaine économique par l'impérialisme monétaire, qui consiste à favoriser, par la violence ou la menace de violence, l'usage d'une monnaie sur un territoire étranger et à en retirer un avantage. L'avantage visé ordinairement est le revenu de seigneuriage supplémentaire rendu possible grâce à une plus grande utilisation de la monnaie. Ce phénomène est parfois schématisé par l'idée que l'État dominant «~exporte son inflation~».

% Xénomonétisation
Le cas le plus clair est la situation dans laquelle un État dominant peut imposer l'usage de sa monnaie au sein de la population générale d'un État dominé. On parle alors de xénomonétisation\sendnote{\url{https://www.cairn.info/revue-mondes-en-developpement-2005-2-page-15.htm}}, en référence au fait que l'État dominé abandonne sa monnaie nationale au profit d'une monnaie étrangère, ou bien de dollarisation, le cas le plus fréquent aujourd'hui concernant le dollar étasunien.

% Monnaie de réserve, indexation (taux de change fixe)
Mais l'État dominant peut aussi exiger que l'État dominé conserve sa monnaie en tant que réserve de change, auquel cas on parle de monnaie de réserve. L'État dominant peut également exiger que la monnaie nationale soit indexée à sa monnaie pour formaliser cette domination. Un tel système peut prendre la forme d'un système bimétallique où l'État dominant émet une monnaie d'argent ayant cours légal sur le territoire de l'État dominé et dont le taux de change par rapport à l'or est délibérément surestimé. Il peut aussi prendre la forme d'un étalon de change-or, où l'État dominé émet des billets représentatifs adossés théoriquement à la monnaie de l'État dominant et impose leur utilisation sur le territoire. Dans le cas de la monnaie fiat, il n'y a plus d'indexation strict, mais la domination peut perdurer comme en témoigne le statut de monnaie de réserve mondiale du dollar depuis 1971.

% Compromis
Dans tous les cas, il s'agit d'un compromis, où les deux États trouvent leur intérêt dans la situation de domination qui est la leur. L'État dominant prélève un revenu de seigneuriage sur l'État dominé, qui lui-même retire un revenu de seigneuriage de sa population. L'État dominé peut bénéficier en retour du soutien géostratégique de l'État dominant.

% Nécessité de domination militaire
L'impérialisme requiert nécessairement une domination militaire. L'État dominant doit être en mesure de battre l'État dominé afin d'imposer l'utilisation de sa monnaie et d'en garantir la force. Si l'État impérialiste de disposait plus d'une supériorité militaire, un État dominé concerné pourrait émettre une monnaie moins inflationniste, privilégier la croissance de son économie pour accroître ses revenus, et devenir progressivement l'État dominant. C'est pourquoi la monnaie dominante dans une région du monde a toujours été la transcription d'une domination militaire, sur terre ou sur mer\sendnote{Dans l'histoire, les principales monnaies dominantes ont été la drachme grecque, l'aureus romain, le solidus byzantin, le dinar arabe, le florin florentin, le ducat vénitien, le réal portugais, le réal espagnol, le florin néerlandais (\eng{Gulden}), la livre tournois française, la livre sterling britannique et enfin le dollar étasunien.}.

% En effet, dans l'ordre international, tant que les différents États restent plus ou moins indépendants, c'est la loi de Thiers qui prédomine~: la bonne monnaie remplace la mauvaise. Ainsi, si l'État impérialiste ne dispose plus de la supériorité militaire, un État dominé peut émettre une monnaie moins inflationniste ou privilégier la croissance de son économie, de façon à ce que sa monnaie devienne à son tour la monnaie la plus forte.

% La phase expansionniste d'un État coïncide avec une forte liberté interne. Albert Jay Nock, \eng{Our Enemy the State}, 1935. Hoppe~: "\eng{Paradoxical as it may first seem, the more liberal a state is internally, the more likely it will engage in outward aggression. Internal liberalism makes a society richer; a richer society to extract from makes the state richer; and a richer state makes for more and more successful expansionist wars. And this tendency of richer states toward foreign intervention is still further strengthened, if they succeed in creating a "liberationist" nationalism among the public, i.e., the ideology that above all it is in the name and for the sake of the general public's own internal liberties and its own relatively higher standards of living that war must be waged or foreign expeditions undertaken.}"

% Sanctions économiques
Cette domination s'exprime notamment aujourd'hui par l'application de sanctions économiques de la part de l'État dominant et de ses États vassaux. Ces sanctions, qui ont un caractère politique et non économique, consistent à restreindre la circulations des biens, des personnes et les transferts financiers depuis et vers un État sanctionné. Il s'agit d'une manière détournée de faire la guerre en empêchant le commerce entre les deux populations~: les sanctions économiques appauvrissent les deux côtés, mais peuvent servir à faire capituler l'autre en appauvrissant sa population et en réduisant par là son revenu fiscal. Une sanction économique peut se manifester par un embargo, qui est une interdiction faite aux navires qui sont dans un port ou sur une rade d'en sortir sans autorisation, afin de contrôler ce qui en sort.

% --- Exemple d'impérialisme monétaire dans les colonies ---

De tels mécanismes étaient utilisés entre les nations européennes et leurs colonies.

% Livre sterling dans l'Empire colonial britannique
L'Empire colonial britannique profitait ainsi du rôle prépondérant de la livre sterling sur son territoire, notamment grâce au cours légal imposé dans la plupart des colonies\sendnote{H. A. Shannon, \eng{Evolution of the Colonial Sterling Exchange Standard}, 1\ier{} janvier 1951~: \url{https://www.elibrary.imf.org/view/journals/024/1951/001/article-A002-en.xml}.}. Après l'abandon de l'étalon-or par le Royaume-Uni en 1931 et l'amorçage de la décolonisation, cet état de fait est devenu la zone sterling, l'ensemble des pays où la livre sterling servait de monnaie nationale ou de monnaie de réserve à partir de laquelle était indexée la monnaie nationale. Ce système à perduré dans la plupart de l'espace colonial jusqu'en 1972 et l'abandon du taux de change fixe avec le dollar\sendnote{\url{https://eh.net/encyclopedia/the-sterling-area/}}.

% Franc dans l'ancien espace colonial français
La France imposait de même sa propre monnaie dans son second empire colonial\sendnote{On peut notamment citer la Banque de l'Algérie qui émettait des billets de francs algériens, payables à vue au porteur, à parité fixe avec le franc français.}. De même, cet empire s'est muté en une zone franc, instaurée à la fin de la Seconde Guerre mondiale, constituée du franc Pacifique (ayant cours dans les collectivités d'outre-mer de l'océan Pacifique), du franc CFA de l'UEMOA, du franc CFA de la CEMAC et du franc comorien. Cette zone franc a perduré après la décolonisation, probablement grâce à l'action du général De Gaulle, et fait aujourd'hui les affaires de l'Union Européennes, les différents francs étant indexés sur l'euro.

La zone CFA en particulier a été maintenue par l'ingérence française dans les affaires des États africains\sendnote{Un exemple spécifique de l'ingérence de la France dans la politique monétaire des États africains est l'opération Persil, menée en Guinée en 1960 après sa sortie de la zone CFA, qui a consisté à inonder le pays de faux francs guinéens dans le but de déstabiliser l'économie. Cette opération a été un échec, mais donne une idée de ce qui pouvait attendre les autres États s'ils cherchaient à s'extraire de l'influence française.}, dans le cadre plus général de la «~Françafrique~». Cette relation de dominance est décrite par Alex Gladstein comme un «~colonialisme monétaire\sendnote{Alex Gladstein, \eng{Fighting Monetary Colonialism with Open-Source Code}, 23 juin 2021~: \url{https://bitcoinmagazine.com/culture/bitcoin-a-currency-of-decolonization}.}~». % Assassinat de Sylvanus Olympio au Togo en 1963

% --- Étalon de change-or britannique ---

Les espaces coloniaux ne sont pas les seuls concernés par l'impérialisme monétaire, et celui-ci peut également s'exercer à l'échelle supérieure, entre puissances majeures. Ainsi, dès 1925, la domination britannique sur l'Europe s'est brièvement matérialisée par la mise en place  d'un étalon de change-or basé sur la livre sterling. Toutefois, sa domination était déjà déclinante et cet étalon-sterling s'est terminé en 1931. De fait, l'Empire britannique était déjà dépassé par une puissance montante, qui s'est démarquée au sortir de la Seconde Guerre mondiale~: les États-Unis d'Amérique.

% --- Empire américain ---

Les États-Unis se sont construits comme un empire dès leur origine. La conquête des territoires indigènes, les guerres contre les puissances européennes en Amérique du Nord, les occupations dans les Caraïbes et dans le Pacifique, les interventions en Amérique centrale et en Amérique du Sud, les interpositions dans les deux guerres mondiales, l'installation de bases militaires tout autour du monde, la guerre froide, l'ingérence au Moyen Orient~: toutes ces actions témoignent d'un empire réel, dont l'ambition est aujourd'hui mondiale.

% Justification de l'empire
Néanmoins, cet empire possède un caractère défensif. Il porte en lui la double idée de la liberté et de la démocratie~: c'est pourquoi il doit justifier ses interventions dans le but de passer pour le libérateur des individus et des peuples. C'est ce qui explique son recours avancé à l'influence douce («~\eng{soft power}~»), dont la monnaie fait partie.  % bataille de Fort Sumter en 1861, explosion de l'U.S.S. Maine en 1898, télégramme Zimmermann en 1917, Pearl Harbor en 1941

% Le rôle central du dollar
Le dollar constitue un rouage central de l'Empire américain. Le dollar étasunien a été formellement créé en 1792 par une loi du Congrès, sur la base du dollar espagnol, qui était une monnaie d'argent issue du \eng{thaler} originaire d'Allemagne. Après l'expansion continentale durant le \textsc{xix}\ieme{}~siècle, les États-Unis se sont étendus au-delà de leurs frontières. À la suite de la guerre contre l'Espagne en 1898, les États-Unis ont pris le contrôle des Philippines, de Guam et de Porto Rico et exercé leur influence sur Cuba. Cette prise de contrôle s'est accompagnée par un impérialisme monétaire classique, illustré par l'instauration d'un peso-argent en 1903 aux Philippines ayant un taux de change fixe avec un peso-or fictif basé sur le dollar\sendnote{E. W. Kemmerer, \eng{The Establishment of the Gold Exchange Standard in the Philippines}, août 1905~: \url{https://www.jstor.org/stable/pdf/1885290.pdf}.}. Par la suite, l'essentiel des Amériques a subi plus ou moins l'influence des États-Unis conformément à l'interprétation impérialiste de la doctrine Monroe par le président Theodore Roosevelt (doctrine du Big Stick), ce qui se traduit aujourd'hui par l'utilisation massive du dollar dans une bonne partie de l'Amérique centrale, des Caraïbes et de l'Amérique Latine.

Durant la présidence de W. H. Taft, la doctrine de la «~diplomatie du dollar~» prévalait. Aujourd'hui, le dollar permet de justifier une partie des interventions relatives à l'extraterritorialité du droit américain. Sauf que cette requête est en réalité partiellement circulaire.

% Étalon de change-or
Le tournant mondial du dollar est l'instauration de l'étalon de change-or par les accords de Bretton Woods en 1944. Au sortir de la Seconde Guerre mondiale, les nations européennes considérablement affaiblies ne faisaient plus le poids face à la puissance impériale étasunienne, qui a pu s'étendre. Les accords de Bretton Woods ont aussi permis le développement d'institutions financières internationales, comme le Fonds monétaire international (FMI), la Banque mondiale (BM) ou l'Organisation mondiale du commerce (OMC), affermissant la domination des États-Unis. En 1964, Valéry Giscard d'Estaing, alors ministre des Finances sous De Gaulle, dénonçait le «~privilège exorbitant du dollar~».

% Monnaie de réserve mondiale après 1971 et 1991
Le dollar s'est assurée la place de monnaie de réserve mondiale \textcolor{darkgray}{même si cette situation pourrait être perturbée dans les prochaines années...} Son utilisation par les pays exportateurs de pétrole (pays du Golfe) en font un pétrodollar. Cette domination est due à ses interventions dans la région.

% --- Vers une monnaie mondiale ---

% Monnaie mondiale
L'impérialisme monétaire est la raison pour laquelle il existe une tendance à la diminution du nombre de monnaies indépendantes. Cette observation nous pousse à nous demander si l'on ne va pas vers une monnaie mondiale, qu'il s'agisse du dollar, d'une autre monnaie (yuan) ou d'une monnaie synthétique\sendnote{L'ancien gouverneur de la Banque d'Angleterre Mark Carney a utilisé l'expression «~monnaie hégémonique synthétique~». -- Mark Carney, \eng{The Growing Challenges for Monetary Policy in the current International Monetary and Financial System}, 23 août 2019~: \url{https://www.bis.org/review/r190827b.pdf}.} construite sur le même modèle que l'euro. % L'euro n'est-il pas un cas d'impérialisme monétaire de l'Allemagne ?

% Monnaie numérique et monnaie mondiale
Une monnaie numérique faciliterait grandement ces évolutions. D'une part, une monnaie numérique permet de faire de l'ingérence à bas prix au sein des autres États (guerre monétaire). D'autre part, le caractère modifiable des monnaies numériques rend la scission ou la fusion très facile en ne requérant plus de remplacer les pièces et les billets (monnaie synthétique).

Une monnaie mondiale permettrait la création d'un État mondial, dont la Société des Nations et l'ONU n'auraient constitué que des prémices. Une fois la mondialisation de l'État atteinte, il n'existerait plus de concurrences entre les États, et le seigneuriage ne serait plus limité.

% Réaction à la centralisation, arbitrage jurdictionnel
Face à cette centralisation de plus en plus évidente, la réaction logique est la sécession. Certains font reposer leur espoir dans l'arbitrage juridictionnel qui consiste pour une personne à tirer parti des divergences qui existent entre des juridictions concurrentes\sendnote{Cette thèse est notamment exposée dans l'ouvrage The Sovereign Individual publié en 1997 qui prédisait à tort la fin des États-Nations~:  «~À l'Ère de l'Information, cependant, la personne rationnelle ne réagira pas à la perspective d'une augmentation des impôts pour financer les déficits en augmentant son taux d'épargne~; elle déplacera son domicile ou effectuera ses transactions ailleurs. [...] Il faut donc s'attendre à ce que les Individus Souverains et les autres personnes rationnelles fuient les juridictions ayant d'importants engagements non financés.~» -- William Rees-Mogg, James Dale Davidson, \eng{The Sovereign Individual}, 1997.}. Une meilleure monnaie pourrait ainsi être émise par un État pour faire concurrence à la monnaie impériale inflationniste, comme par exemple une monnaie basée sur l'or\sendnote{\url{https://www.aucoffre.com/academie/retour-etalon-or-possible/}} ou sur le bitcoin\sendnote{Parket Lewis, \eng{Bitcoin Cannot be Banned}, 11 août 2019~: \url{https://unchained.com/blog/bitcoin-cannot-be-banned/}~; archive~:\url{https://web.archive.org/web/20210916080203/https://unchained.com/blog/bitcoin-cannot-be-banned/}.}, et bénéficier des avantages économiques résultants. Toutefois, une telle démarche implique un renoncement au seigneuriage et une confrontation avec la puissance impériale qui pourrait appliquer des mesures comme des sanctions économiques. Les petits États faibles militairement, qui auraient beaucoup à en tirer, ne pourraient pas faire face~; et les gros États ont de toute manière peu d'avantages à gagner, ayant déjà une dominance économique forte. Il est donc peu probable que l'arbitrage juridictionnel s'applique plus qu'il ne le fait déjà, c'est-à-dire à la marge et dans la mesure où il n'affaiblit pas le pouvoir central. % The Sovereign Individual: "In the Information Age, however, the rational person will not respond to the prospect of higher taxes to fund deficits by increasing his savings rate; he will transfer his domicile, or lodge his transactions elsewhere. [...] The result to be expected is that Sovereign individuals and other rational people will flee jurisdictions with large unfunded liabilities."

% Eric Hughes, 1993: "The Really Big Question is, how large can the flow of money on the nets get before the government requires reporting of every small transaction? Because if the flows can get large enough, past some threshold, then there might be enough aggregate money to provide an economic incentive for a transnational service to issue money, and it wouldn't matter what one government does." https://kk.org/mt-files/outofcontrol/ch12-f.html

% Troisième voie
Il existe cependant une troisième voie~: l'affaiblissement interne de la puissance centrale par la sécession individuelle. Et c'est ce que permet de faire Bitcoin~: il permet de mener une guerre par d'autres moyens.

% \begin{quote}
% «~À tout bien considérer, il semble que l'Utopie soit beaucoup plus proche de nous que quiconque ne l'eût pu imaginer, il y a seulement quinze ans. À cette époque, je l'avais lancée à six cents ans dans l'avenir. Aujourd'hui, il semble pratiquement possible que cette horreur puisse s'être abattue sur nous dans le délai d'un siècle. Du moins, si nous nous abstenons, d'ici là, de nous faire sauter en miettes. En vérité, à moins que nous ne nous décidions à décentraliser et à utiliser la science appliquée, non pas comme une fin en vue de laquelle les êtres humains doivent être réduits à l'état de moyens, mais bien comme le moyen de produire une race d'individus libres, nous n'avons le choix qu'entre deux solutions: ou bien un certain nombre de totalitarismes nationaux, militarisés, ayant comme racine la terreur de la bombe atomique, et comme conséquence la destruction de la civilisation (ou, si la guerre est limitée, la perpétuation du militarisme); ou bien un seul totalitarisme supranational, suscité par le chaos social résultant du progrès technique rapide en général et de la révolution atomique en particulier, et se développant, sous le besoin du rendement et de la stabilité, pour prendre la forme de la tyrannie-providence de l'Utopie. C'est vous qui voyez.\sendnote{Aldous Huxley, \eng{Préface de 1946 au Meilleur des mondes}, 1946.}~» % All things considered it looks as though Utopia were far closer to us than anyone, only fifteen years ago, could have imagined. Then, I projected it six hundred years into the future. Today it seems quite possible that the horror may be upon us within a single century. That is, if we refrain from blowing ourselves to smithereens in the interval. Indeed, unless we choose to decentralize and to use applied science, not as the end to which human beings are to be made the means, but as the means to producing a race of free individuals, we have only two alternatives to choose from: either a number of national, militarized totalitarianisms, having as their root the terror of the atomic bomb and as their consequence the destruction of civilization (or, if the warfare is limited, the perpetuation of militarism); or else one supranational totalitarianism, called into existence by the social chaos resulting from rapid technological progress in general and the atomic revolution in particular, and developing, under the need for efficiency and stability, into the welfare-tyranny of Utopia. You pays your money and you takes your choice.
% \end{quote}

\section*{Choisir de ne plus participer} % systèmes alternatifs centralisés
\addcontentsline{toc}{section}{Choisir de ne plus participer}

Face à cet ordre monétaire imposé par la force de façon plus ou moins directe, certaines personnes ont tenté de s'en extraire, au moins partiellement, par la création de systèmes alternatifs. Toutefois, cela s'est toujours soldé par un échec à cause du caractère centralisé de ces systèmes.

% Le contrôle sur la monnaie est essentiel dans le prélèvement du citoyen. C'est pourquoi la gestion de cette fonction a toujours été une prérogative de l'autorité dominante. L'État maintient ainsi en place plusieurs contraintes légales consistant à limiter l'évasion fiscale et à maintenir la valeur d'échange de la monnaie artificiellement haute. % Ces contraintes se manifestent notamment par la surveillance financière, appliquée notamment aux organismes bancaires et financiers, le contrôle des capitaux et le contrôle des changes.

% Il y a une raison pour laquelle toutes les monnaies privées et toutes les monnaies numériques précédentes ont échoué~: c'est que l'État, l'autorité qui s'exerce sur un territoire déterminé et sur un peuple qu'elle prétend représenter, n'aime pas la concurrence. L'État impose en effet un monopole monétaire, monopole qui lui permet de tirer son revenu. Par le contrôle des capitaux et la surveillance financière offerts par sa monnaie, il s'assure de prélever l'impôt de manière efficace. Par le seigneuriage et l'endettement, il s'assure profiter de la création monétaire. L'idée d'une monnaie qui échappe à son contrôle est donc intolérable pour un État, dont la tendance naturelle est de grossir le plus possible.

% --- Monnaies complémentaires communautaires ---

Il existe des monnaies complémentaires communautaires, qui sont divisées entre monnaies locales et monnaies sectorielles. Les monnaies locales sont des monnaies échangées dans une zone géographique prédéterminée, généralement à l'échelle d'une ville ou d'une région. Comme leur nom l'indique, les monnaies complémentaires ont pour objectif d'être \emph{complémentaires} et ne veulent pas pas faire concurrence aux monnaies officielles. Elles sont en général liées de près ou de loin à la monnaie nationale et se conforment aux réglementations juridiques\sendnote{Coline Renault, \emph{Bayonne : un accord trouvé sur la monnaie basque}, \url{https://www.lefigaro.fr/actualite-france/2018/06/09/01016-20180609ARTFIG00056-bayonne-un-accord-trouve-sur-la-monnaie-basque.php}.}.

Elles sont là pour réaliser des expérimentations dans le cadre d'associations, comme l'expérience de Wörgl d'une monnaie fondante\sendnote{Claude Bourdet, \emph{Wörgl ou l'«~argent fondant~»}, 1933~: \url{https://fr.wikisource.org/wiki/Wörgl_ou_l’«_argent_fondant_»}.} ou celle de la banque WIR visant à former une banque stable et résistante aux crises systémiques. Les Ithaca Hours, échangées à Ithaca dans l'État de New York entre 1991 et les années 2000, se voulaient être une mesure du temps de travail, basées sur les expériences socialistes utopiques de Robert Owen et de Josiah Warren\sendnote{Paul Glover, \url{https://criterical.net/wp-content/uploads/2016/06/ccmag_10_06.pdf\#page=36}.}.

% Ğ1
L'émergence de Bitcoin a inspiré la création de protocoles permettant de mettre en place ce type d'expérience. C'est le cas de la Ğ1 (la «~june~») qui repose sur une «~toile de confiance~» (accès fermé) et qui propose un «~dividende universel~» à tous ses utilisateurs. % Proof-of-authority

De manière générale, ces monnaies n'ont aucune viabilité économique à grande échelle. On l'a par exemple vu avec le Crédito argentin créé en 1995 qui a subi un hyperinflation à cause de son fonctionnement interne et de la contrefaçon\sendnote{Pepita Ould-Ahmed, \emph{Les «~clubs de troc~» argentins~: un microcosme monétaire Crédito dépendant du macrocosme Peso}, 2010~: \url{https://journals.openedition.org/regulation/7799}.}.

% --- Monnaies privées ---

C'est une autre chose des monnaies privées en général, qui sont ouvertes à tous et librement échangeables sur le marché.

% Monnaies privées en France
En France comme dans le reste de l'Europe, la monnaie a toujours été de manière générale la prérogative du seigneur, d'où le nom qu'on donne au seigneuriage. La charge a donc été déléguée à des ateliers monétaires. Ainsi, en dehors des quelques expériences de monnaies de nécessité (notamment après l'abandon de l'étalon-or au lendemain de la Première Guerre mondiale), tolérées par le pouvoir, la frappe n'a jamais été libre en France. De même, la courte expérience de liberté bancaire après la Révolution entre 1796 et 1803 a été écourtée par l'arrivée au pouvoir de Napoléon Bonaparte\sendnote{Kevin Dowd, \eng{The Experience of Free Banking}, 1992.}.

% Franc libre
Cette interdiction des monnaies privées a été de nouveau affirmée en 2022 lorsque l'ancien gendarme Alexandre Juving-Brunet, figure de l'extrême-droite française, a voulu lancer un système de franc libre, gérant à la fois les transactions électroniques et les supports physiques (billets et pièces). Ce dernier a été mis en examen et placé en détention pendant 111 jours\sendnote{Les chefs d'accusation retenus contre Alexandre Juving-Brunet étaient l'«~exercice illégal de l'activité d'émetteur de monnaie électronique~», la «~fourniture de services bancaires de paiement à titre habituel par une personne autre qu'un établissement de crédit~», la «~mise en circulation de monnaie non autorisée en vue de remplacer la monnaie ayant cours légal~» et l'«~escroquerie en bande organisée~». -- Ouest-France, \emph{Var. Un ex-gendarme et candidat aux législatives mis en examen pour diffusion d'une monnaie illégale}, 30 novembre 2022~: \url{https://www.ouest-france.fr/societe/faits-divers/un-ex-gendarme-et-candidat-aux-legislatives-mis-en-examen-pour-l-emission-d-une-monnaie-illegale-178bf7b8-70de-11ed-b658-d40122929dc2}.}.

% Monnaies privées aux États-Unis
Les États-Unis en revanche possèdent une grande culture des monnaies privées, conformément à l'esprit de liberté individuelle qui les a caractérisés. Pendant la période coloniale et durant la première moitié du \textsc{xix}\ieme{}~siècle, la frappe privée de pièces était tout à fait autorisée et pratiquée\sendnote{Brian Summers, \eng{Private Coinage in America}, 1\ier{} juillet 1976~: \url{https://fee.org/articles/private-coinage-in-america/}.}. De même, l'activité bancaire a été relativement libre à partir de 1837, année de fin du mandat de la \eng{Second Bank}, non renouvelé par le président Andrew Jackson.

% Interruption de la liberté monétaire et bancaire aux États-Unis (1864)
Cette liberté monétaire et bancaire a été cependant interrompue par les mesures prises à la suite de la guerre de Sécession. D'une part, une loi du Congrès du 8 juin 1864 a interdit la frappe privée des pièces\sendnote{Cette loi du 8 juin 1864 est devenue la section 486 du titre 18 du Code des États-Unis~:
\begin{quote}
\footnotesize «~Quiconque, sauf dans le cas où cela est autorisé par la loi, fabrique, met en circulation ou fait passer, ou tente de mettre en circulation ou de faire passer, des pièces d'or ou d'argent ou d'autres métaux, ou des alliages de métaux, destinées à être utilisées comme monnaie courante, qu'elles ressemblent à des pièces des États-Unis ou de pays étrangers, ou qu'elles soient de conception originale, sera condamné à une amende en vertu du présent titre ou à une peine d'emprisonnement de cinq ans au maximum, ou aux deux.~»
\end{quote} % "Whoever, except as authorized by law, makes or utters or passes, or attempts to utter or pass, any coins of gold or silver or other metal, or alloys of metals, intended for use as current money, whether in the resemblance of coins of the United States or of foreign countries, or of original design, shall be fined under this title or imprisoned not more than five years, or both."
(\eng{18 U.S. Code § 486 - Uttering coins of gold, silver or other metal}~: \url{https://www.law.cornell.edu/uscode/text/18/486})}. D'autre part, les \eng{National Banking Acts} de 1863 et 1864 ont définitivement mis fin à l'horizontalité et l'indépendance des banques.

% Secret Service
C'est à cette occasion qu'a été créé le \eng{Secret Service}, une agence étatique ayant pour mission de lutter contre le faux-monnayage et la fraude financière en général. Créé le 14 avril 1865 (le jour de l'assassinat d'Abraham Lincoln), son rôle a depuis été étendu à la protection des hautes figures du gouvernement. À l'époque, il servait, d'une façon détournée, à affermir le monopole sur la production de monnaie.

% Monopole monétaire finalisé
Cette transition a été finalisée avec la création de la Réserve Fédérale en 1913 et la prohibition de la détention d'or promulguée par l'ordre exécutif 6102 signé par F.D. Roosevelt le 5 avril 1933.

% Monnaies privées
Après l'abandon de toute référence à l'or dans le système monétaire mondial (et l'abrogation consécutive de l'ordre exécutif en 1975) et le développement du réseau informationnel Internet, l'idée de déployer une monnaie privée est réapparue. Puisque l'État fédéral pouvait gérer arbitrairement sa monnaie, pourquoi ne pouvait-il pas en être autant des individus~? C'est ainsi que des individus ont entrepris, dans une démarche purement hayekienne, de déployer leur propre monnaie sur le marché. Parmi ces projets de monnaie privée, nous pouvons en citer quatre~: ALH\&Co, le Liberty Dollar, e-gold et Liberty Reserve.

% --- ALH&Co (1976-2004) ---

Le premier était ALH\&Co, une banque libre offrant la possibilité à ses clients d'avoir des comptes bancaires libellés en or ou en dollars\sendnote{Wendy McElroy, \eng{Anthony L. Hargis And The Trusted Third Party Trap}, 14 mai 2020~: \url{https://www.agoristnexus.com/anthony-l-hargis-and-the-trusted-third-party-trap/}.}. Cette banque a été créée par Anthony L. Hargis, un libertarien proche de Samuel Edward Konkin et de son idée agoriste. Bien que la banque elle-même faisait toutes les démarches pour être légales, elle n'empêchait pas l'évasion fiscale. Konkin lui-même a décrit le fonctionnement de ALH\&Co dans son ouvrage \eng{Counter-Economics} publié à titre posthume\sendnote{Samuel Edward Konkin \textsc{iii}, \eng{Counter-Economics: From the Back Alleys... To the Stars}, 2018.}.

Les clients pouvaient rédiger des «~ordres de transfert~» qui fonctionnaient comme des chèques bancaires entre les différentes entreprises qui les reconnaissaient, ou bien les soumettre à AHL\&Co et recevoir en retour une chèque bancaire classique ou demander à ALH\&Co de payer leurs factures régulières. ALH\&Co a existé pendant près de 30 ans, entre 1976 et 2004, du fait de son caractère confidentiel. À un moment donné, la banque avait 253 clients et utilisait 9 comptes bancaires classiques sur lesquels étaient déposés 7,2 millions de dollars.

En mai 1993, les locaux d'ALH\&Co ont subi une descente des agents fédéraux, suite à un signalement de suspicion de blanchiment d'argent lié au trafic de drogue. Les agents se sont emparés des dossiers des clients. Cependant, les opérations d'ALH\&Co on pu continuer pendant une décennie.

Hargis a finalement été inculpé en mars 2004, et ALH\&Co a définitivement été fermée. L'IRS a estimé que l'évasion fiscale des clients s'élevait à 24 millions de dollars\sendnote{\url{https://www.latimes.com/archives/la-xpm-2004-mar-10-fi-taxscam10-story.html}}.

% --- Liberty Dollar (1998 - juin 2009) ---

Le deuxième exemple contemporain de monnaie privée aux États-Unis est le Liberty Dollar, une monnaie basée sur l'or et l'argent qu'on pouvait retrouver sous forme de pièces d'argent, de billets représentatifs et d'unités électroniques. Le Liberty dollar a été créé en 1998 par Bernard von NotHaus au travers de son organisation à but non lucratif NORFED\sendnote{NORFED est l'acronyme de \eng{National Organization for the Repeal of the Federal Reserve and Internal Revenue Code}, en français~: l'Organisation nationale pour l'abrogation de la Réserve fédérale et de l'\eng{Internal Revenue Code}.}.

Ce système a connu un certain succès, notamment après l'introduction du système de monnaie numérique en 2003. Outre les pièces de monnaies en circulation, les coffres de NORFED contenaient environ 8 millions de dollars en métaux précieux pour assurer la convertibilité de la devise, dont 6 pour garantir l'unité numérique\sendnote{P. Carl Mullan, \eng{A History of Digital Currency in the United States}, 2016.}.

Toutefois, en septembre 2006, la Monnaie des États-Unis, l'institution en charge de frapper et mettre en circulation les pièces de monnaie américaines, a émis un communiqué de presse, écrit conjointement avec le département de la Justice, dans lequel elle concluait que les «~médaillons~» de NORFED violaient la section 486 du titre 18 du Code des États-Unis et constituaient «~un crime\sendnote{\url{https://www.usmint.gov/news/press-releases/20060914-liberty-dollars-not-legal-tender-united-states-mint-warns-consumers}}~». Le communiqué rappelait également que les pièces frappées ressemblaient au dollar ce qui s'apparentait de la contrefaçon\sendnote{\eng{18 U.S. Code § 485 - Coins or bars}~: \url{https://www.law.cornell.edu/uscode/text/18/485}.}.

Après un descente du FBI dans les locaux de NORFED en 2007\sendnote{Bernard von NotHaus\eng{FBI Raids Liberty Dollar}, 14 novembre 2007~: \url{http://www.libertydollar.org/ld/legal/raidday1.htm}.}, les violations ont été retenues contre NotHaus et ses associés, qui ont arrêtés en 2009 et condamnés en mars 2011. En conséquence de ce jugement, les pièces pouvaient être considérés comme de la contrebande et être saisies comme telles\sendnote{\url{https://www.coinworld.com/news/precious-metals/liberty-dollars-may-be-subject-to-seizure.html}}. Les ventes de ces pièces ont également été interdites sur eBay en décembre 2012, sous la pression du Secret Service\sendnote{Jon Matonis, \eng{U.S. Secret Service Bans Certain Gold and Silver Coins On eBay}, 15 décembre 2012~: \url{https://www.forbes.com/sites/jonmatonis/2012/12/15/u-s-secret-service-bans-certain-gold-and-silver-coins-on-ebay/}.}.

% Lien du Liberty Dollar avec Bitcoin
Le Liberty Dollar n'était pas inconnu des premiers utilisateurs de Bitcoin. Ainsi, Dustin Trammell, l'un des premiers opérateurs de nœud sur le réseau, s'intéressait à ce système avant de découvrir l'invention de Satoshi Nakamoto comme en témoigne son article sur le sujet en décembre 2008\sendnote{Dustin Trammell, \eng{The Problem With the Liberty Dollar}, 7 décembre 2008~: \url{https://blog.dustintrammell.com/the-problem-with-the-liberty-dollar/}.}.

% --- e-gold (1996 -- 27 avril 2007 / 21 juillet 2008 / novembre 2009) ---

Le troisième cas de monnaie privée est l'e-gold\sendnote{Ludovic Lars, \emph{L'e-gold de Douglas Jackson~: la cryptomonnaie "or"}, 8 mars 2020~: \url{https://journalducoin.com/analyses/gold-douglas-jackson-cryptomonnaie-or-1996/}.}, une devise en or numérique (\eng{digital gold currency} en anglais), transférée électroniquement et garantie à 100~\% par une quantité équivalente en or conservée en lieu sûr. Le système e-gold a été co-fondé par Douglas Jackson et Barry Downey en 1996, deux ans avant PayPal. Douglas Jackson était un oncologue américain vivant en Floride. Adepte de Hayek, il souhaitait créer une meilleure monnaie avec e-gold.

% Le système gérait une unité de compte du même nom.

L'e-gold était par essence une monnaie représentative, chaque montant d'e-gold pouvant être converti en or réel. La détention et la conversion d'or était administrée par une société créée pour l'occasion et basée aux États-Unis, \eng{Gold \& Silver Reserve Inc.} (G\&SR). La société garantissait également de l'e-silver, de l'e-platinum et de l'e-palladium sur le même modèle.

Le système informatique était géré par une deuxième entreprise, \eng{e-gold Ltd.} enregistrée à Saint-Christophe-et-Niévès dans les Caraïbes. Pour l'époque, il était très performant, mettant à profit un système à règlement brut en temps réel inspiré du virement interbancaire. Le système tirait profit des navigateurs web et en particulier de Netscape, de sorte que chaque client pouvait avoir accès à son compte depuis le site web.

Le système e-gold a rencontré ainsi un grand succès, à tel point qu'il représentait à un moment donné le deuxième système de paiement en ligne mondial derrière PayPal. À son apogée en 2006, il garantissait 3,6 tonnes d'or, soit plus de 80 millions de dollars, traitait 75~000 transactions par jour, pour un volume annualisé de 3 milliards de dollars, et gérait plus de 2,7 millions de comptes.

Toutefois, ce succès fulgurant a été de courte durée. Au terme d'une enquête menée par le Secret Service\sendnote{\url{https://www.secretservice.gov/press/releases/2008/07/us-secret-service-led-investigation-digital-currency-business-e-gold-pleads}}, Douglas Jackson, ses deux sociétés et ses associés ont été inculpés le 27 avril 2007 par le département de la Justice pour facilitation de blanchiment d'argent et activité de transfert d'argent sans licence\sendnote{\eng{18 U.S. Code § 1960 - Prohibition of unlicensed money transmitting businesses}~: \url{https://www.law.cornell.edu/uscode/text/18/1960}.}.

% Le 27 avril 2007, Douglas Jackson, ses deux sociétés (G&SR et e-gold Ltd) et ses associés sont inculpés par la justice américaine pour facilitation de blanchiment d'argent et activité de transfert d'argent sans licence.

Jackson a été condamné à 3 ans de liberté surveillée, incluant 6 mois d'assignation à résidence sous surveillance électronique, et à 300 heures de travail communautaire. Ses deux entreprises ont dû payer une amende de 300 000~\$. Après une tentative infructueuse d'obtenir une licence, e-gold a dû fermer ses portes définitivement en novembre 2009\sendnote{\url{https://web.archive.org/web/20100103135107/http://blog.e-gold.com/2009/11/egold-update-value-access.html}}.

% Autres devises en or numériques
Un indicateur du succès d'e-gold est l'émergence de systèmes similaires de devise en or numérique~: nous pouvons citer GoldMoney, fondé par James Turk et son fils en février 2001, qui s'est aujourd'hui adapté aux réglementations financières~; e-Bullion, fondé par James Fayed en juillet 2001 et fermé en 2008~; et Pecunix fondé par Simon Davis en 2002, entreprise enregistrée au Panama, qui a fermé ses portes en 2015, dans le cadre d'une escroquerie de sortie. Le Liberty Dollar électronique (eLD) lancé en 2003 ne faisait ainsi que suivre la vague.

% Lien de e-gold avec Bitcoin
Ces devises en or numérique étaient encore utilisées du temps de Bitcoin, de sorte que ses premiers utilisateurs en avaient connaissance. Satoshi Nakamoto lui-même savait bien comment ces systèmes fonctionnaient, comme le montre l'un de ses courriels adressé à la \eng{Cryptography Mailing List}\sendnote{«~Il est intéressant de noter que l'un des systèmes d'e-gold a déjà une forme de spam appelé "dusting". Les spammeurs envoient une minuscule quantité de poussière d'or afin de placer un message de spam dans le champ de commentaire de la transaction.~» -- Satoshi Nakamoto, \eng{Re: Bitcoin v0.1 released}, \wtime{25/01/2009 15:47:10 UTC}~: \url{https://www.metzdowd.com/pipermail/cryptography/2009-January/015041.html}.}. De même, Ross Ulbricht avait envisagé d'utiliser Pecunix pour Silk Road avant de trouver Bitcoin\sendnote{Correspondance par courriel entre Ross Ulbricht et Arto Bendiken (GX-270), septembre 2009~: \url{https://antilop.cc/sr/exhibits/253456462-Silk-Road-exhibits-GX-270.pdf}}.

% --- Liberty Reserve (2006-mai 2013) ---

Le quatrième et dernier exemple de projet de monnaie privée était le système Liberty Reserve, qui permettait de détenir et de transférer des devises indexées sur le dollar étasunien, sur l'euro ou sur l'or\sendnote{Ludovic Lars, \emph{La Liberty Reserve d'Arthur Budovsky~: plongée dans l'obscure préhistoire de Bitcoin}, 21 mars 2020~: \url{https://journalducoin.com/analyses/liberty-reserve-bitcoin/}.}. Le système était la création d'Arthur Budovsky, un Américain d'origine ukrainienne, aux côtés de Vladmir Kats. En 2006, Budovsky s'est expatrié au Costa Rica, qui était alors considéré comme une paradis fiscal facilitant le blanchiment d'argent, où il a enregistré sa société, Liberty Reserve S.A.  % Arthur Budovsky et Vladimir Kats. Liberty Reserve S.A. a été enregistrée au Costa Rica. Budovsky s'est marié à une Costaricaine en juin 2008 dans le but d'obtenir la nationalité du pays (mariage de complaisance).

L'inculpation d'e-gold en avril 2007 a profité grandement à Liberty Reserve qui a pu prendre la relève. Le système a ainsi rencontré un grand succès. En mai  2013, l'acte d'accusation du département de la Justice estimait que Liberty Reserve possédait plus d'un million d’utilisateurs dans le monde, dont plus de 200~000 aux États-Unis, et traitait 12 millions de transactions financières annuellement, pour un volume combiné de plus de 1,4 milliards de dollars\sendnote{\url{https://www.justice.gov/sites/default/files/usao-sdny/legacy/2015/03/25/Liberty\%20Reserve\%2C\%20et\%20al.\%20Indictment\%20-\%20Redacted_0.pdf}}.

Toutefois, ce succès s'est accompagné de complications sérieuses. En 2009, la \emph{Superintendencia General de Entidades Financieras} (SUGEF) costaricaine s'est intéressée au cas de Liberty Reserve, lui demandant d'obtenir une licence (chose qu'elle n'est pas parvenue à faire). En mars 2011, une enquête a été ouverte. En novembre 2011, c'est au tour du FinCEN étasunien qui délivre un avis selon lequel LR était «~utilisée par les criminels pour effectuer des transactions anonymes\sendnote{\url{https://www.justice.gov/sites/default/files/usao-sdny/legacy/2015/03/25/Liberty\%20Reserve\%2C\%20et\%20al.\%20Indictment\%20-\%20Redacted_0.pdf}}~».

La fin de Liberty Reserve a été retentissante, au terme d'une opération coordonnée internationalement. Le 24 mai 2013, Arthur Budovsky et les principaux gestionnaires de Liberty Reserve ont été inculpés et arrêtés, dans des juridictions différentes~: en Espagne, aux États-Unis et au Costa Rica. Après environ un an et demi de détention, en octobre 2014, Arthur Budovsky a été extradé de l'Espagne vers New York aux États-Unis, où s'est déroulé son procès. En 2016, Arthur Budovsky a finalement plaidé coupable pour blanchiment d'argent, et, a été condamné à 20 ans de prison ferme.

% Lien avec Bitcoin
Liberty Reserve a probablement la dernière monnaie libre centralisée de grande envergure sur Internet. Le système était encore massivement utilisé lorsque Bitcoin faisait ses premiers pas. Liberty Reserve a ainsi été utilisé pour acheter du bitcoin sur toutes premières plateformes d'échange, y compris sur le fameux Mt. Gox~!

% --- Conclusion ---

De ce fait, en France comme aux États-Unis, il est \emph{de facto} interdit de fournir des services bancaires sans licence (ALH\&Co), de frapper ses propres pièces de monnaie et d'imprimer ses propres billets (Liberty Dollar), ou de gérer des comptes électroniques en or (e-gold) ou dans la devise nationale (Liberty Reserve), dans la mesure où cela fait concurrence à l'État. Bien qu'il y ait des raisons multiples aux fermetures de ces systèmes, on ne peut que constater que toutes les alternatives sérieuses au système monétaire étatique ont été éliminés\sendnote{Lawrence White, \eng{The Troubling Suppression of Competition from Alternative Monies: The Cases of the Liberty Dollar and E-Gold}, 2014~: \url{https://ciaotest.cc.columbia.edu/journals/cato/v34i2/f_0031473_25521.pdf}.}.

Le monopole monétaire est souvent imposé subtilement par des lois liées à la contrefaçon ou au blanchiment d'argent. C'est pour cela qu'il n'y aujourd'hui aucune alternative légale. C'est pour cela que les innovations dans le domaine financier, comme PayPal\sendnote{Luke Nosek, co-fondateur de PayPal, affirmait (peut-être en exagérant) durant le Forum économique mondial de Davos le 31 janvier 2019~:

\begin{quote}
\footnotesize «~Beaucoup de gens l'ignorent, mais la mission de PayPal était de créer une monnaie mondiale qui était indépendante de l'ingérence des cartels bancaires corrompus et des États qui dévaluaient leurs monnaies. Nous avons réussi à construire quelque chose de très puissant économiquement, qui a rendu possible de nombreuses petites entreprises, nous en sommes super fiers, mais nous n'avons jamais accompli cette mission. Je ne pense pas que [le problème de la monnaie numérique] soit résolu par PayPal, précisément en raison du fait que [...] PayPal est simplement trop centralisé et trop attaché aux grandes institutions financières comme Visa, MasterCard, le réseau ACH, le réseau SWIFT.~»
% "Well, many people don't know this, but the mission of PayPal was to create a global currency that was independent of interference by these, you know, corrupt cartels of banks and governments that were debasing their currencies. We succeed at building something economically very powerful, enabled many small businesses, we're super proud of it, but, we never achieved the mission. I don't think [the problem of digital money] is solved by PayPal precisely for the reason that you brought in the Venezuela case where PayPal is simply too centralized and too attached to the big financial institutions: Visa, MasterCard, the ACH network, the SWIFT network."
\end{quote}
(Reserve, \eng{Luke Nosek speaks to Nevin Freeman about Reserve and the original vision of PayPal - Davos 2019}, 22 mai 2019~: \url{https://www.youtube.com/watch?v=hOeOzhOxeMU\&t=40s})} ou GoldMoney, se sont conformées aux réglementations existantes pour pouvoir survivre.

L'intervention étatique est là pour s'immiscer dans le système monétaire et le contrôler, au besoin en détruisant les alternatives. C'est pour s'opposer à cette fatalité que Bitcoin a été conçu tel qu'il existe aujourd'hui.

% Aujourd'hui, les États détiennent, directement ou indirectement, un monopole sur la monnaie et il paraît illusoire qu'ils renoncent à ce contrôle sans réagir. En tant que protocole neutre existant en dehors des lois gouvernementales, Bitcoin constitue une potentielle menace à leur pouvoir. En cela, s'il en venait à être vraiment populaire, il faut s'attendre à ce que Bitcoin soit interdit par la loi. C'est en tout cas la thèse d'Eric Voskuil.

\section*{Bitcoin contre l'État} % la nécessité de décentralisation
\addcontentsline{toc}{section}{Bitcoin contre l'État}

L'État est la manifestation organisée, territorialisée et institutionnalisée du transfert de richesse non consenti. Son prélèvement se fait de deux façons principales~: soit il est réalisé directement, par l'impôt~; soit indirectement, par le seigneuriage. Dans les deux cas, il repose sur le contrôle sur la monnaie.

Au fur et à mesure du temps, le contrôle sur la monnaie s'est fait de plus en plus pernicieux. Avec l'émergence des banques durant le Renaissance, il s'est accéléré. L'usage des billets de banque a été récupéré par l'État, qui s'est arrogé le monopole exclusif sur leur production, jusqu'à les transformer en papier-monnaie. L'usage des dépôts bancaires est aujourd'hui surveillé de près et pourrait, dans un futur proche, être également repris par l'État à son avantage par l'intermédiaire de la monnaie numérique de banque centrale.

% Ce prélèvement a pour conséquence de créer des situations catastrophiques, de détruire la coopération volontaire.

Il est illusoire de croire que l'État renonce à son prélèvement~: il faudrait pour cela que ses bénéficiaires demandent eux-mêmes cette transition. On pourrait croire qu'un petit État aurait la possibilité et l'intérêt géostratégique de le faire, mais ce serait ignorer les vélléités impérialistes des puissances dominantes. Il ne suffit donc pas de faire tourner un serveur dans une juridiction accomodante pour gérer une monnaie numérique comme l'a montré le cas de Liberty Reserve.

C'est pourquoi Bitcoin est comme il est. Il est spécifiquement conçu pour résister à l'intervention de l'État et constitue une tentative de construire une alternative robuste au système monétaire actuel. Bitcoin résout cette problématique en distribuant le fonctionnement du système au sein un réseau pair-à-pair de nœuds. Cette distribution à égalité permet de partager les risques entre les personnes qui s'en occupent, et de faire en sorte que la sécurité du système repose sur leur actions économiques combinées plutôt que sur celle d'un seul individu ou d'une seule entreprise\sendnote{Eric Voskuil, \emph{Cryptoéconomie}, «~Principe de partage des risques~» (p. 63).}.

Ce fonctionnement n'est pas sorti de nulle part et c'est ce que nous allons voir dans le prochain chapitre.

\printendnotes