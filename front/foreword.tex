% Copyright (c) 2023 Jacques Favier
% This work is licensed under the CC BY-NC-SA 4.0 International License

\chapter*{Préface de Jacques Favier}
\addcontentsline{toc}{chapter}{Préface de Jacques Favier}

Ludovic Lars est assez unanimement considéré comme l'un des grands «~érudits~» francophones en matière de Bitcoin. Ce mot, qui est revenu dans plusieurs conversations au sujet de son projet éditorial, m'a inspiré le fil de trame de cette préface.

En me faisant l'honneur de me demander celle-ci, il m'avait prévenu que le ton «~libéral~» de son \emph{Élégance du Bitcoin} pourrait trancher avec ma propre sensibilité. J'y ai perçu une forme d'\emph{élégance} morale. En vérité, l'auteur a lu les économistes dits «~autrichiens~» mais aussi Proudhon et il n'est pas davantage que moi maximaliste buté ou toxique. Étant surtout ennemi des extrémismes idéologiques et des raisonnements à une seule dimension, je ne saurais m'offusquer de ce qu'au sein d'une communauté supportant l'essor de solutions décentralisées règnent des opinions différentes, avec ce que cela implique comme visions ou comme biais.

Tout en assumant ce que l'on appelait jadis un vrai «~travail de bénédictin~» l'auteur a d'ailleurs demandé et obtenu la confiance de multiples spécialistes qui ont assuré à son travail une prise en compte d'un très large spectre de connaissances et une relecture soigneuse. Il y a eu, au-delà de ce concours d'experts, un véritable engagement communautaire, financier et moral pour que soit publié le présent livre.

On trouvera donc ici un travail qui, tant par un ton rarement polémique que par une profonde érudition et une inscription dans un mouvement collectif, participe de la tradition de l'\emph{Encyclopédie} française. On sait que les promoteurs de la \emph{Britannica} accusèrent celle de Diderot et d'Alembert de «~propager l'anarchie~» et l'on ne peut nier que, rédigée alors qu'éclosaient les Lumières, cette somme des connaissances de toute nature – tant théoriques que pratiques – n'ait pris courageusement parti dans les combats politiques et philosophiques de son temps, avec l'intention explicite d'ouvrir une réflexion critique et de «~changer la façon commune de penser~». Sans doute pourrait-on en dire autant ici~: Bitcoin et ce livre ne vous invitent pas tant, ou pas seulement, à changer de monnaie qu'à changer de pensée.

Ayant senti cela, je suis allé fureter dans l'\emph{Encyclopédie}. L'article «~érudition~», rédigé par d'Alembert lui-même, expose que celle-ci «~renferme trois branches principales, la connaissance de l'Histoire, celle des Langues, \& celle des Livres~». En changeant peut-être \emph{langues} par \emph{protocoles}, il aurait pu goûter lui-aussi et préfacer mieux que moi le livre que vous venez d'ouvrir.

Mon propre esprit, formé aux études historiques, s'est délecté des premiers chapitres, qui constituent de véritables Annales de Bitcoin. Les historiens d'aujourd'hui et de demain ne pourront qu'apprécier l'ampleur des informations fiables et des références compilées. Mathématicien de formation, l'auteur a produit d'abord un très important travail archivistique, dont atteste près d'un millier de notes savantes.

Comme me l'a écrit le créateur du site Bitcoin.fr «~il déniche et déchiffre des débats abscons qui ont pourtant eu une importance capitale dans l'évolution du protocole, et les rend compréhensible à tous~». Ainsi, si ce qu'il restitue de l'histoire de la monnaie peut être critiqué ou remis dans la perspective de ses convictions personnelles, ce qu'il construit de l'histoire de Bitcoin est un apport dont d'autres feront utilement leur miel.

Au-delà de l'Histoire, il y a donc les Langues et les Livres~: des références, du code, de la théorie des jeux et des mathématiques. Il y a beaucoup à glaner~dans ces pages, dont beaucoup de choses austères mais aussi de petits faits plaisants. Si le livre narre en détail l'inévitable geste de la fameuse pizza, il rappelle aussi qu'un robinet à bitcoin a fonctionné 2 ans en envoyant 5 bitcoins à chaque demande, ou que celui qui a découvert la première faille a gentiment prévenu Satoshi au lieu de profiter de sa découverte pour tricher. Il souligne ainsi qu'avant sa phase de «~croissance conflictuelle~» les premières années de l'aventure ont vu «~une croissance organique et prudente, à l'abri de l'opportunisme et de la propagande de notre monde~» et que la communauté a fait montre, depuis l'origine, d'une extraordinaire résilience, chose qui doit être méditée.

L'abondance des citations rend justice aux cypherpunks, parfois traités comme de sinistres sires fomentant une révolte fiscale autour d'un intempestif barbecue. Elle restitue la profondeur historique et intellectuelle de ce qui fut un mouvement de fond collectif et non une réaction épidermique sectaire. Accessoirement, les trajets individuels finement retracés montrent que l'influence autrichienne, non négligeable, ne fut ni universelle ni complète. Bien des cryptographes, cypherpunks ou non, n'y ont pas adhéré comme à un dogme révélé ou à une vérité scientifiquement établie.

Ludovic Lars rappelle en outre ce point crucial~: les cypherpunks ne furent pas les seuls à essayer de construire des systèmes distribués qui puissent servir à l'échange monétaire. Parce qu'il y avait un vrai problème et un vrai besoin. Dans le bouillonnement intellectuel, les échanges étaient nombreux~: il est amusant de rappeler que Ripple s'inspira aussi du localisme des SEL~! En fait la différence avec toutes les autres tentatives c'est que Bitcoin (le premier à ne pas reposer sur une confiance au sens classique) a réussi comme monnaie parce qu'il a réussi à construire une communauté élective, philosophique, politique. Bitcoin est la plus large monnaie communautaire de tous les temps.

S'intéresser à sa (longue) geste avant autant qu'après 2009 n'est donc pas une marotte d'historien. Outre une compréhension indispensable de ses racines, des intentions et des ambitions qui animaient précurseurs et témoins de sa naissance, on trouve de quoi démonter bien des escroqueries intellectuelles hélas persistantes. Non, les monnaies numériques de banques centrales ou les \emph{stablecoins} algorithmiques ne représentent pas des perfectionnements de Bitcoin ni d'ailleurs des promesses d'amélioration de notre existence à venir.

Les \emph{altcoins} plus ou moins communautaires, souvent entrepreneuriaux voire bancaires, sont largement cités, essentiellement pour illustrer le propos, l'enrichir d'exemples, souligner des impasses ou des objections, jamais, il faut le répéter, et même si l'auteur les connaît fort bien, pour «~dépasser~» ou «~perfectionner~» Bitcoin, dont le développement organique et le perfectionnement est l'affaire des bitcoineurs.

L'auteur est un expert technique mais il sait aussi écrire. Tout ce qui est technique (et ignoré par beaucoup de gens, même de ceux qui se présentent comme des «~experts~»), tout ce que d'Alembert nommerait «~les Langues~» est disséqué dans cet ouvrage avec un scalpel extrêmement méticuleux et restitué dans la langue où ce qui est bien conçu «~s'énonce clairement~». Ceci mériterait d'être donné à lire au prochain politique, financier, économiste ou publiciste qui dira que «~ça ne repose sur rien~»~!

Le titre du livre est aussi celui d'une conclusion agréablement équilibrée, entre ceux qui voient en Bitcoin la solution à tout et ceux qui n'y voient que du mal. Elle pourra surprendre certains adeptes fervents et naïfs mais elle reste dans l'esprit de d'Alembert~: «~Il y a dans la critique deux excès à fuir également, trop d'indulgence, \& trop de sévérité~».

Curieusement, l'auteur s'appesantit peu sur le mot d'\emph{élégance} lui-même, que sa formation mathématique lui fait sans doute percevoir comme embrassant les sens de vérité, de beauté et de rigueur. Comme Aristote, il a pu penser ici à l'ordre, à la précision, à la capacité de faire jouer ensemble plusieurs concepts, de les ajuster ensemble efficacement, performance que Satoshi Nakamoto a réalisée au plus haut point.

Pour m'adresser au lecteur au seuil de ce livre utile, dense et à tous égards distingué, je donne une dernière fois la parole à d'Alembert qui opinait que «~les secours que nous avons aujourd'hui pour l'érudition la facilitent tellement, que notre paresse seroit inexcusable, si nous n'en profitions pas~».

\begin{flushright}Jacques Favier, 21 novembre 2023\end{flushright}

