% Copyright (c) 2023 Ludovic Lars
% This work is licensed under the CC BY-NC-SA 4.0 International License

\chapter*{Avant-propos}
\addcontentsline{toc}{chapter}{Avant-propos} % Pourquoi j'ai écrit ce livre

% Succès de Bitcoin
Depuis sa conception en 2008 par Satoshi Nakamoto, Bitcoin a fait couler beaucoup d'encre. Au fil des années, il a suscité les plus grandes passions et il a été l'objet récurrent de débats enflammés. À son sujet, des milliers d'articles ont été écrits, des centaines de vidéos ont été tournées, et des dizaines de livres ont été publiés. La hausse de son prix lui a donné une visibilité extraordinaire dans les médias, à tel point qu'il s'est fait une place dans l'imaginaire collectif mondial.

% Incompréhension de Bitcoin
Cependant, Bitcoin reste largement incompris. D'un côté, beaucoup de gens en parlent en n'ayant qu'une connaissance artificielle du sujet et ne parviennent pas à distinguer son utilité. Certains pensent qu'il ne sert qu'à spéculer, d'autres imaginent qu'il ne devrait être utilisé que par les criminels, d'autres encore vont jusqu'à dire qu'il ne s'agit que d'une pyramide de Ponzi. De l'autre côté, un certain nombre de personnes nourrissent des attentes démesurées, pensant qu'il pourrait devenir la monnaie de réserve mondiale, voire remplacer tous les échanges monétaires dans l'économie en quelques années seulement. Dans cette délusion, elles s'attachent à l'espoir que son prix atteigne des niveaux stratosphériques, dans la continuité des hausses spéculatives précédentes. Toutefois, peu de gens tentent d'adopter un point de vue réaliste et sobre, qui ferait la part des choses entre la vision des vendeurs de rêve pour qui Bitcoin serait la solution à tous les problèmes du monde, et les détracteurs de mauvaise foi pour qui Bitcoin représenterait un fléau à arrêter à tout prix.

% Découverte personnelle de Bitcoin
J'ai personnellement entendu parler de Bitcoin pour la première fois en avril 2013, suite à la crise financière chypriote. Initialement assez sceptique, je me suis quand même intéressé à ce système, car celui-ci était mise en avant par les libéraux français et les libertariens américains que je suivais. Le 9 juillet 2015, j'ai essayé la chose~: je me suis procuré 50~\euro{} de bitcoins (0,2~BTC) auprès de la plateforme d'achat-vente suisse Fastcoin (nommée aujourd'hui Bity) que j'ai reçus sur mon portefeuille Electrum nouvellement créé. J'ai réalisé ma première transaction sur la chaîne de Bitcoin dans la journée. Ces quelques fractions de bitcoin m'ont servi à faire des dons~: d'abord au blogueur H16, puis à l'activiste Adam Kokesh, ensuite au projet DarkWallet de Amir Taaki et Cody Wilson, et enfin à la plateforme Sci-Hub gérée par Alexandra Elbakyan. % « libertariens américains » : Adam Kokesh, Julia Tourianski ; h16 : 1BuyJKZLeEG5YkpbGn4QhtNTxhUqtpEGKf ; Adam Kokesh : 19rX5FLPWXhRnpLdDdUaewusfbNHvFWni8 ; DarkWallet : 31oSGBBNrpCiENH3XMZpiP6GTC4tad4bMy ; Sci-Hub : 1K4t2vSBSS2xFjZ6PofYnbgZewjeqbG1TM

% Implication dans Bitcoin
Mon implication dans Bitcoin n'a débuté réellement qu'au printemps 2017, lorsque le prix a recommencé à monter après des années de stagnation. Jusque-là, je m'étais contenté de suivre la cryptomonnaie de loin et cette hausse m'a intrigué. C'est à ce moment-là que je me suis pleinement plongé dans cet univers. J'ai lu à ce propos, notamment en me procurant des ouvrages comme \emph{Bitcoin, la monnaie acéphale} d'Adli Takkal-Bataille et Jacques Favier, \emph{Mastering Bitcoin} d'Andreas Antonopoulos ou encore \emph{Digital Gold} de Nathaniel Popper. Je me suis également mis à spéculer à mon échelle en achetant du bitcoin, puis toutes sortes de cryptomonnaies alternatives.

% Écriture
En parallèle, j'ai commencé à écrire sur le sujet, si bien que je suis devenu rédacteur pour des sites spécialisés comme Cryptoast et le Journal du Coin. Au fil des années, j'ai rédigé plus de 150 articles de fond, sur divers sujets liés à la cryptomonnaie, que ce soit sous un angle technique, économique ou politique. Ma vision de Bitcoin a mûri en conséquence, de telle sorte que je pouvais prétendre «~comprendre Bitcoin~», même si ma conception restait évidemment parcellaire et influencée par ma propre perspective.

% Pourquoi j'ai écrit ce livre
Cependant, ce n'était pas le cas autour de moi, où les gens en avaient une idée superficielle, n'ayant probablement pas le temps de creuser davantage. C'est ce qui m'a poussé à écrire ce livre. En particulier, puisque le protocole monétaire dépendait des actions économiques de ses utilisateurs, il me paraissait important de partager la connaissance réelle qui avait émergé de mes recherches et de mon expérience. De plus, avec la progression de la censure bancaire, le développement des monnaies numériques de banque centrale, la guerre contre l'argent liquide et le retour de l'inflation, je pense qu'il est plus que jamais essentiel de bien appréhender cet outil afin de pouvoir l'utiliser correctement à l'avenir.

% Présentation de l'ouvrage
Cet ouvrage a pour but de présenter Bitcoin de manière claire et complète, en adoptant de multiples points de vue. Celui-ci narre le long cheminement qui a mené à sa création, ainsi que sa courte mais dense histoire des origines à aujourd'hui. Il décrit son fonctionnement essentiellement économique découlant de sa nature monétaire. Il aborde les enjeux politiques qui lui sont liés, et en particulier le problème de la censure. Et il examine enfin ses rouages techniques de manière détaillée. Comme moi, vous verrez peut-être en Bitcoin un ensemble harmonieux, dont le modèle de base est d'une rare élégance.

% Mot de la fin
J'espère en tout cas que vous saurez apprécier ma modeste contribution à quelque chose qui me dépasse~: le projet d'une monnaie alternative, libre et résiliente, offrant aux simples individus la possibilité de résister aux puissances de ce monde.

\begin{flushright}Ludovic Lars, 1\ier{} décembre 2023\end{flushright}

% Depuis sa création en 2008 par Satoshi Nakamoto, Bitcoin a fait couler beaucoup d'encre. Au cours des années, il a suscité les plus grandes passions et est devenu l'objet récurrent de débats enflammés. Des milliers d'articles ont été écrits, des centaines de vidéos ont été tournées, des dizaines de livres ont été publiés, tout cela à son sujet. La hausse de son prix lui a donné une visibilité extraordinaire dans les médias si bien qu'une bonne part de l'humanité en a aujourd'hui déjà entendu parler.
%
% Cependant, Bitcoin reste largement incompris. D'un côté, beaucoup en parlent en n'ayant qu'une connaissance artificielle du sujet et passent à côté de son utilité : certains pensent qu'il ne sert qu'à spéculer, d'autres imaginent qu'il ne devrait être utilisé que par les criminels, d'autres encore vont jusqu'à dire qu'il ne s'agit que d'une pyramide de Ponzi. De l'autre côté, un certain nombre de gens nourrissent des attentes démesurées : il devrait devenir la monnaie de réserve mondiale voire remplacer tous les échanges monétaires dans l'économie en quelques années seulement et, de ce fait, son prix devrait atteindre des niveaux stratosphériques. Mais peu tentent d'adopter un point de vue réaliste et raisonnable, qui ferait la part des choses entre les vendeurs de rêve pour qui Bitcoin serait la solution à tous les problèmes du monde, et les détracteurs de mauvaise foi pour qui Bitcoin représenterait un fléau sans précédent.
%
% Bitcoin est une révolution conceptuelle, Bitcoin est l'incarnation de la monnaie libre et résiliente, Bitcoin est un moyen élégant et puissant de résister à l'autorité. Mais Bitcoin n'est qu'un outil qui, comme tous les outils, possède des limites et des défauts. Si on veut en faire une bonne utilisation, il est par conséquent nécessaire de bien appréhender cet outil.
%
% L'objectif de cet ouvrage est d'expliquer en profondeur ce qu'est Bitcoin, dans quel contexte il s'inscrit, comment il fonctionne, ce qu'il permet et comment s'en servir correctement. Avec le retour de l'inflation, l'accroissement de la censure bancaire et le développement des monnaies numériques de banque centrale, il est en effet aujourd'hui devenu fondamental d'améliorer sa compréhension de Bitcoin pour protéger sa liberté et sa richesse.
%
% Ce texte constituait la section introductive de la présentation du projet en mars 2022.