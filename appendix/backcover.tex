% Copyright (c) 2023 Ludovic Lars
% This work is licensed under the CC BY-NC-SA 4.0 International License

\documentclass[a5paper, 11pt]{article}

% Basic packages
\usepackage[french]{babel} % language
\usepackage[utf8]{inputenc}                 % accents
\usepackage[T1]{fontenc}                    % french characters
\usepackage[margin=2cm]{geometry}           % margin
\usepackage{setspace}                       % spacing
\usepackage{hyperref}                       % cross-referencing

% Custom packages
\usepackage{newtxtext}                      % Times New Roman font

\usepackage[type={CC}, modifier={by-nc-sa}, version={4.0}]{doclicense} % license

% Custom settings
\setlength{\parskip}{1ex}                   % paragraph spacing

\title{L'Élégance de Bitcoin (quatrième de couverture)}     % title
\author{Ludovic Lars}                       % author
\date{\today}                               % date

\hypersetup{
    pdftitle={\csname @title\endcsname},
    pdfauthor={\csname @author\endcsname}
}

\begin{document}

\section*{L'Élégance de Bitcoin (quatrième de couverture)}
\thispagestyle{empty}

Bitcoin est un modèle novateur de monnaie numérique décentralisée, dont l'existence vient bousculer l'ordre établi en proposant une alternative audacieuse au système bancaire classique.

Depuis son énigmatique conception par Satoshi Nakamoto en 2008, Bitcoin a connu une croissance fulgurante qui a marqué les esprits. Il a déchaîné toutes les passions, de l'enthousiasme démesuré de ses promoteurs au rejet épidermique de ses détracteurs.

Plongez dans cet ouvrage captivant où l'auteur propose un point de vue réaliste et pragmatique sur Bitcoin. Vous découvrirez sa fantastique histoire, ainsi que les principes économiques, idéologiques et techniques qui ont assuré sa survie dans un environnement hostile.

Vous en ressortirez avec une compréhension de Bitcoin susceptible de transformer votre vision du monde.

\textbf{À propos de l'auteur.} Ludovic est rédacteur et formateur dans le monde de la cryptomonnaie. Sa découverte de Bitcoin en 2013 l'a progressivement conduit à reconnaître le formidable potentiel de libération apporté par cet outil. Il se consacre depuis 2018 à en décrire le fonctionnement de la façon la plus fidèle possible.

\end{document}